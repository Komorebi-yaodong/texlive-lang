% \iffalse meta-comment
% ArabLuaTeX -- Processing ArabTeX notation under LuaLaTeX
% Copyright (C) 2016--2020  Robert Alessi
%
% Please send error reports and suggestions for improvements to Robert
% Alessi <alessi@robertalessi.net>
%
% This program is free software: you can redistribute it and/or modify
% it under the terms of the GNU General Public License as published by
% the Free Software Foundation, either version 3 of the License, or
% (at your option) any later version.
%
% This program is distributed in the hope that it will be useful, but
% WITHOUT ANY WARRANTY; without even the implied warranty of
% MERCHANTABILITY or FITNESS FOR A PARTICULAR PURPOSE.  See the GNU
% General Public License for more details.
%
% You should have received a copy of the GNU General Public License
% along with this program.  If not, see
% <http://www.gnu.org/licenses/>.
% \fi
%
% \iffalse
%<*driver>
\ProvidesFile{arabluatex.dtx}
%</driver>
%<package>\NeedsTeXFormat{LaTeX2e}[1999/12/01]
%<package>\ProvidesPackage{arabluatex}
%<*package>
    [2020/03/23 v1.20 ArabTeX for LuaLaTeX]
%</package>
%
%<*driver>
\documentclass{ltxdoc}
\usepackage{filecontents}
\begin{filecontents}{\jobname.bib}
% This file is part of the `arabluatex' package
%
% ArabLuaTeX -- Processing ArabTeX notation under LuaLaTeX
% Copyright (C) 2016--2020  Robert Alessi
%
% Please send error reports and suggestions for improvements to Robert
% Alessi <alessi@robertalessi.net>
%
% This program is free software: you can redistribute it and/or modify
% it under the terms of the GNU General Public License as published by
% the Free Software Foundation, either version 3 of the License, or
% (at your option) any later version.
%
% This program is distributed in the hope that it will be useful, but
% WITHOUT ANY WARRANTY; without even the implied warranty of
% MERCHANTABILITY or FITNESS FOR A PARTICULAR PURPOSE.  See the GNU
% General Public License for more details.
%
% You should have received a copy of the GNU General Public License
% along with this program.  If not, see
% <http://www.gnu.org/licenses/>.

@software{pkg:arabtex,
  author =	 {Lagally, Klaus},
  maintainer =	 {Lagally, Klaus},
  title =	 {Arab\TeX},
  indextitle =   {Arab\TeX},
  date =	 {2004-11-03},
  version =	 {4.00},
  url =
    {http://mirrors.ctan.org/language/arabic/arabtex/doc/html/arabtex.htm},
  subtitle =	 {Typesetting Arabic and Hebrew},
  titleaddon =	 {User Manual Version 4.00}
}

@software{pkg:amiri,
  author =	 {Hosny, Khaled},
  maintainer =	 {Hosny, Khaled},
  title =	 {Amiri},
  indextitle =	 {Amiri},
  date =	 {2017-12-13},
  url =		 {http://www.amirifont.org/}
}

@Book{Habash,
  author =	 {Habash, Nizar Y.},
  title =	 {Introduction to Arabic Natural Language Processing},
  year =	 2010,
  series =	 {Synthesis Lectures on Human Language Technologies},
  number =	 10,
  publisher =	 {Morgan \& Claypool Publishers},
  location =	 {Toronto}
}

@software{pkg:lua-ul,
  title =	{The Lua-ul package},
  subtitle =	{Underlining for LuaLaTeX},
  author =	{Krüger, Marcel},
  maintainer =	{Krüger, Marcel},
  url =		{http://www.ctan.org/pkg/lua-ul},
  date =	{2020-03-12},
  version =	{0.0.1}
}

@MVBook{Wright,
  author =	 {Wright, W. LL.D},
  title =	 {A Grammar of the Arabic Language},
  indextitle =	 {Grammar of the Arabic Language, A},
  year =	 1896,
  editor =	 {Robertson Smith, W. and de Goeje, M. J.},
  editortype =	 {reviser},
  foreword =	 {Cachia, Pierre},
  edition =	 3,
  volumes =	 2,
  pagination =	 {none},
  publisher =	 {Librairie du Liban},
  location =	 {Beirut},
  annote =	 {New impression, 1996}
}

@Manual{din31635,
  label =	 {{DIN 31~635}},
  title =	 {Information and Documentation - Romanization of the
                  Arabic Alphabet for Arabic, Ottoman-Turkish,
                  Persian, Kurdish, Urdu and Pushto},
  date =	 {2011-07},
  url =		 {http://www.din.de}
}

@InProceedings{dmg,
  author =	 {Brockelmann, Carl and Fischer, August and Heffening,
                  W. and Taeschner, Franz},
  shorttitle =	 {Die Transliteration der arabischen Schrift},
  title =	 {Die Transliteration der arabischen Schrift in ihrer
                  Anwendung auf die Hauptliteratursprachen der
                  islamischen Welt},
  indextitle =	 {Transliteration der arabischen Schrift, Die},
  year =	 1935,
  booktitle =	 {Denkschrift dem 19. internationalen
                  Orientalistenkongreß in Rom vorgelegt von der
                  Transkriptionkommission der Deutschen
                  Morgenländischen Gesellschaft},
  editor =	 {van Ronkel, Ph. S. and Spies, Otto},
  editortype =	 {collaborator},
  publisher =	 {Deutsche Morgenländische Gesellschaft, in Kommission
                  bei F. A. Brockaus},
  url =
                  {http://www.naher-osten.uni-muenchen.de/studium_lehre/werkzeugkasten/dmgtransliteration.pdf},
  location =	 {Leipzig}
}

@MVBook{Lane,
  author =	 {Lane, Edward William},
  title =	 {An Arabic-English lexicon},
  date =	 {1863/1893},
  indextitle = 	 {Arabic-English Lexicon, An},
  volumes =	 8,
  shorthand = 	 {Lane, \emph{Lexicon}},
  pagination = 	 {none},
  publisher =	 {Williams and Norgate},
  location =	 {London -- Edinburgh}
}
\end{filecontents}
\usepackage{fontspec}
\usepackage[english]{babel}
\usepackage{dox}
\doxitem{Option}{option}{options}
\usepackage{microtype}
\babelfont{rm}{Old Standard}
\babelfont{sf}{NewComputerModern Sans}
\babelfont{tt}{NewComputerModern Mono}
\usepackage{metalogox}
\usepackage{arabluatex}[2020/03/23]
\SetArbUp{\textsuperscript{\thinspace#1}} % Old Standard needs this
\usepackage[nopar]{quran}
\usepackage[noindex]{nameauth}
\usepackage{varioref}
\usepackage{hyperxmp}
\PassOptionsToPackage{pdfa}{hyperref}
\usepackage{hypdoc}
\usepackage{uri}
\usepackage{bookmark}
\usepackage{authblk}
\usepackage{latexcolors}
\hypersetup{unicode=true, colorlinks, allcolors=cinnamon,
  linktocpage=true, pdfauthor={Robert Alessi}, pdftitle={The
    arabluatex package}, pdfcontactemail={alessi@robertalessi.net},
  pdfcontacturl={http://www.robertalessi.net/arabluatex},
  pdfcopyright={Copyright (C) 2016--2020 Robert Alessi
    <alessi@robertalessi.net>. This document is licensed under the
    Creative Commons Attribution-ShareAlike 4.0 International
    License. To view a copy of this license, visit
    http://creativecommons.org/licenses/by-sa/4.0/ or send a letter to
    Creative Commons, PO Box 1866, Mountain View, CA 94042, USA.},
  pdflicenseurl={https://creativecommons.org/licenses/by-sa/4.0/legalcode},
  pdfmetalang={en-US}, pdftype={Text}, pdfkeywords={Arabic language,
    arabtex, luatex}}
\usepackage[scale=1.5]{ccicons}
\usepackage[lot]{multitoc}
\usepackage{enumitem}
\setlist{nosep}
\setlist[itemize]{label=\textendash}
\setlist[enumerate,1]{label=(\alph*)}
\setlist[enumerate,2]{label=\roman*.}
\newlist{enumabjad}{enumerate}{10}
\setlist[enumabjad]{label={\abjad{\arabic*}}}
\usepackage{multicol}
\usepackage{cleveref}
\crefname{footnote}{note}{notes}
\usepackage{quoting}
\quotingsetup{noorphans, rightmargin=0pt}
\renewcommand*{\quotingfont}{\footnotesize}
\usepackage[position=below]{caption}
\usepackage{lineno}
\usepackage{longtable}
\usepackage{booktabs}
\usepackage[defaultlines=3,all]{nowidow}
\usepackage{needspace}
\usepackage{addlines}
\usepackage{relsize}
\usepackage{tikz}
\usepackage[breakable, skins, xparse, minted]{tcolorbox}
\tcbset{colback=white, boxrule=.15mm, colframe=cinnamon,
  breakable}
\newtcbox{\arabluabox}{boxrule=.3mm, left=0mm, right=0mm, top=0mm,
  bottom=0mm}
\newtcblisting{example}{minted options={linenos, numbersep=0mm,
    fontsize=\smaller}}
\newtcblisting{code}{minted options={linenos, numbersep=0mm,
    fontsize=\smaller}, listing only}
\newcommand{\package}[1]{\textsf{#1}\index{#1=#1 (package)}}
\usepackage[contents]{colordoc}
\usepackage{csquotes}
\DeclareQuoteStyle{arabic}
{\rmfamily\textquotedblright}{\rmfamily\textquotedblleft}
{\rmfamily\textquoteright}{\rmfamily\textquoteleft}
\usepackage[style=authoryear, indexing=cite]{biblatex}
\DeclareIndexFieldFormat{indextitle}{\index{#1=\emph{#1}}}
\addbibresource{arabluatex.bib}
\usepackage{etoc}
\etocsettocdepth{paragraph}
\newcommand{\altableofcontents}{%
  \begingroup
  \etocsetstyle{section}{}{}
  {\etocsavedsectiontocline{%
      \numberline{\etocnumber}\etocname}{\etocpage}}{}
  \etocsetstyle{subsection}{}{}
  {\etocsavedsubsectiontocline{%
      \numberline{\etocnumber}\etocname}{\etocpage}}{}%
  \etocsetstyle{subsubsection}{}{}
  {\etocsavedsubsubsectiontocline{%
      \numberline{\etocnumber}\etocname}{\etocpage}}{}%
  \etocsetstyle{paragraph}{}{\leftskip2cm\rightskip 2.2em \parfillskip
    0pt plus 1fil\relax \nobreak}
  {\noindent\etocname{} \etocpage{} }{\par}%
  \etocmulticolstyle[2]{\section*{Contents}}
  \pdfbookmark[1]{Contents}{toc}
  \tableofcontents
  \endgroup}
\EnableCrossrefs
\RecordChanges
\CodelineIndex
\begin{document}
  \DocInput{arabluatex.dtx}
  \printbibliography[heading=bibintoc]
  \phantomsection
  \addcontentsline{toc}{section}{Change History}
  \PrintChanges
  \phantomsection
  \addcontentsline{toc}{section}{Index}
  \PrintIndex
\end{document}
%</driver>
% \fi
%
% \CheckSum{1059}
%
% \CharacterTable
%  {Upper-case    \A\B\C\D\E\F\G\H\I\J\K\L\M\N\O\P\Q\R\S\T\U\V\W\X\Y\Z
%   Lower-case    \a\b\c\d\e\f\g\h\i\j\k\l\m\n\o\p\q\r\s\t\u\v\w\x\y\z
%   Digits        \0\1\2\3\4\5\6\7\8\9
%   Exclamation   \!     Double quote  \"     Hash (number) \#
%   Dollar        \$     Percent       \%     Ampersand     \&
%   Acute accent  \'     Left paren    \(     Right paren   \)
%   Asterisk      \*     Plus          \+     Comma         \,
%   Minus         \-     Point         \.     Solidus       \/
%   Colon         \:     Semicolon     \;     Less than     \<
%   Equals        \=     Greater than  \>     Question mark \?
%   Commercial at \@     Left bracket  \[     Backslash     \\
%   Right bracket \]     Circumflex    \^     Underscore    \_
%   Grave accent  \`     Left brace    \{     Vertical bar  \|
%   Right brace   \}     Tilde         \~}
%
%   \makeatletter
%   \let\org@changes@\changes@
%   \def\my@changes v#1.#2.#3\@nil{%
%   \org@changes@{v#1.\six@digits{#2}.#3=v#1.#2.#3}%
% }%
%   \newcommand*{\six@digits}[1]{%
%   \ifnum#1<100000 0\fi
%   \ifnum#1<10000 0\fi
%   \ifnum#1<1000 0\fi
%   \ifnum#1<100 0\fi
%   \two@digits{#1}%
% }%
%   \renewcommand*{\changes@}[1]{%
%   \my@changes#1.\@nil
% }%
%   \makeatother
%
% \changes{v1.0}{2016/03/29}{Initial release}
% \changes{v1.0.1}{2016/03/31}{Minor update of the documentation}
%
% \DoNotIndex{\newcommand,\newenvironment,\renewcommand}
% \DoNotIndex{\~,\AtBeginDocument,\bgroup,\csname}
% \DoNotIndex{\DeclareDocumentCommand,\def,\edef,\egroup}
% \DoNotIndex{\else,\endcsname,\endinput,\expandafter,\fi}
% \DoNotIndex{\ifdef,\ifdefined,\ifx,\MessageBreak,\NeedsTeXFormat}
% \DoNotIndex{\NewDocumentCommand,\newif,\PackageError,\PackageWarning}
% \DoNotIndex{\relax,\RenewDocumentCommand,\string,\verb,\let}
% \DoNotIndex{\enskip}
% 
% \providecommand*{\url}{\texttt}
% \GetFileInfo{arabluatex.dtx}
% 
% \newcommand*{\NEWfeature}[1]{%
%     \hskip 1sp \marginpar{\small\sffamily\raggedright
%     New feature\\#1}}
% 
% \title{\tcbox[colframe=black, enhanced, tikznode, drop lifted
%   shadow, colback=white, boxrule=.25mm]%
% {The \textsf{arabluatex} package\\
% \fileversion\ -- \filedate}}
% 
% \author{Robert Alessi \\
% \href{mailto:alessi@robertalessi.net?Subject=arabluatex package}%
% {\texttt{alessi@robertalessi.net}}}
% \date{}
% 
% \maketitle
% \footnotesize
% \altableofcontents
% \listoftables
% \normalsize
% 
% \begin{abstract}
%   This package provides for {\LuaLaTeX} an Arab{\TeX}-like interface
%   to generate Arabic writing from an \textsc{ascii}
%   transliteration. It is particularly well-suited for complex
%   documents such as technical documents or critical editions where a
%   lot of left-to-right commands intertwine with Arabic
%   writing. \package{arabluatex} is able to process any Arab\TeX\
%   input notation. Its output can be set in the same modes of
%   vocalization as Arab\TeX, or in different roman
%   transliterations. It further allows many typographical
%   refinements. It will eventually interact with some other packages
%   yet to come to produce from \verb|.tex| source files, in addition
%   to printed books, \texttt{TEI xml} compliant critical editions
%   and/or lexicons that can be searched, analyzed and correlated in
%   various ways.
% \end{abstract}
%
% \section*{License and disclamer}
% \addcontentsline{toc}{section}{License and disclamer}
% \subsection*{License applicable to the software}
% \label{sec:license-software}
%
% \package{arabluatex} --- Processing Arab\TeX\ notation under Lua\LaTeX.\\
% Copyright \textcopyright\ 2016--2020  Robert Alessi
%
% Please send error reports and suggestions for improvements to Robert
% Alessi:
% \begin{itemize}
% \item email: \mailto[arabluatex package]{alessi@roberalessi.net}
% \item website: \url{http://www.robertalessi.net/arabluatex}
% \item development: \url{http://git.robertalessi.net/arabluatex}
% \item comments, feature requests, bug reports:
% \url{https://gitlab.com/ralessi/arabluatex/issues}
% \end{itemize}
%
% \marginpar{\texttt{gpl3+}}
% This program is free software: you can redistribute it and/or modify
% it under the terms of the GNU General Public License as published by
% the Free Software Foundation, either version 3 of the License, or
% (at your option) any later version.
%
% This program is distributed in the hope that it will be useful, but
% WITHOUT ANY WARRANTY; without even the implied warranty of
% MERCHANTABILITY or FITNESS FOR A PARTICULAR PURPOSE.  See the GNU
% General Public License for more details.
%
% You should have received a copy of the GNU General Public License
% along with this program.  If not, see
% <http://www.gnu.org/licenses/>.
%
% This release of \package{arabluatex} consists of the following
% source files:
% \begin{itemize}
% \item |arabluatex.ins|
% \item |arabluatex.dtx|
% \item |arabluatex.lua|
% \item |arabluatex_voc.lua|
% \item |arabluatex_fullvoc.lua|
% \item |arabluatex_novoc.lua|
% \item |arabluatex_trans.lua|
% \item |arabluatex.el|
% \end{itemize}
% 
% \subsection*{License applicable to this document}
% \label{sec:documentation-license}
% Copyright \textcopyright\ 2016--2020  Robert Alessi
%
% \ccbysa\marginpar{\texttt{CC BY-SA 4.0}}
% This document is licensed under the Creative Commons
% Attribution-ShareAlike 4.0 International License. To view a copy of
% this license, visit
% \url{http://creativecommons.org/licenses/by-sa/4.0/} or send a
% letter to Creative Commons, PO Box 1866, Mountain View, CA 94042,
% USA.
%
% \section{Introduction}
% In comparison to Prof. Lagally's outstanding Arab\TeX,\footnote{See
% \url{http://ctan.org/pkg/arabtex}} Arab{\LuaTeX} is at present
% nothing more than a modest piece of software. Hopefully---if I may
% say so---it will eventually provide all of its valuable qualities to
% the {\LuaLaTeX} users.
%
% \package{arabtex} dates back to 1992. As far as I know, it was then
% the first and only way to typeset Arabic texts with \TeX\ and
% \LaTeX. To achieve that, \package{arabtex} provided---and still
% does---an Arabic font in \arb[trans]{\uc{nasxI}} style and a macro
% package that defined its own input notation which was, as the author
% stated, \enquote{both machine, and human, readable, and suited for
% electronic transmission and e-mail
% communication}.\footnote{\textcite[2]{pkg:arabtex}.}  Even if the
% same can be said about Unicode, Arab\TeX\ \textsc{ASCII} input
% notation still surpasses Unicode input, in my opinion, when it comes
% to typesetting complex documents, such as scientific documents or
% critical editions where footnotes and other kind of annotations can
% be particulary abundant. It must also be said that most text editors
% have trouble in displaying Arabic script connected with preceding or
% following \LaTeX\ commands: it often happens that commands seem
% misplaced, not to mention punctuation marks, or opening or closing
% braces, brackets or parentheses that are unexpectedly displayed in
% the wrong direction. Of course, some text editors provide ways to
% get around such difficulties by inserting invisible Unicode
% characters, such as LEFT-TO-RIGHT or RIGHT-TO-LEFT MARKS
% (\texttt{U+200E}, \texttt{U+200F}), RTL/LTR \enquote{embed}
% characters (\texttt{U+202B}, \texttt{U+202A}) and RLO/LRO
% \enquote{bidi-override} characters (\texttt{U+202E},
% \texttt{U+202D}).\footnote{Gáspár Sinai's Yudit probably has the
% best Unicode support. See \url{http://www.yudit.org}.} Nonetheless,
% it remains that inserting all the time these invisible characters in
% complex documents rapidly becomes confusing and cumbersome.
%
% The great advantage of Arab\TeX\ notation is that it is immune from
% all these difficulties, let alone its being clear and
% straightforward. One also must remember that computers are designed
% to process code. Arab\TeX\ notation is a way of encoding Arabic
% language, just as \TeX\ \enquote{mathematics mode} is a way of
% processing code to display mathematics. As such, not only does it
% allow greater control over typographical features, but it also can
% be processed in several different ways: so without going into
% details, depending on one's wishes, Arab\TeX\ input can be full
% vocalized Arabic (\emph{scriptio plena}), vocalized Arabic or
% non-vocalized Arabic (\emph{scriptio defectiva}); it further can be
% transliterated into whichever romanization standard the user may
% choose.
%
% \label{ref:tei-to-come}
% But there may be more to be said on that point, as encoding Arabic
% also naturally encourages the coder to vocalize the texts---without
% compelling him to do so, of course. Accurate coding may even have
% other virtuous effects. For instance, hyphens may be used for tying
% particles or prefixes to words, or to mark inflectional endings, and
% so forth. In other words, accurate coding produces accurate texts
% that can stand to close grammatical scrutiny and to complex textual
% searches as well.
%
% Having that in mind, I started \package{arabluatex}. With the help
% of Lua, it will eventually interact with some other packages yet to
% come to produce from \verb|.tex| source files, in addition to
% printed books, \texttt{TEI xml} compliant critical editions and/or
% lexicons that can be searched, analyzed and correlated in various
% ways.
%
%\subsection{\package{arabluatex} is for {\LuaLaTeX}}
% It goes without saying that \package{arabluatex} requires
% {\LuaLaTeX}. \TeX\ and \LaTeX\ have \package{arabtex}, and
% {\XeLaTeX} has \package{arabxetex}. Both of them are much more
% advanced than \package{arabluatex}, as they can process a number of
% different languages,\footnote{\label{fn:arabtex-languages}To date,
% both packages support Arabic, Maghribi, Urdu, Pashto, Sindhi,
% Kashmiri, Uighuric and Old Malay; in addition to these,
% \package{arabtex} also has a Hebrew mode, including Judeo-Arabic and
% Yiddish.} whereas \package{arabluatex} can process only Arabic for
% the time being. More languages will be included in future releases
% of \package{arabluatex}.
%
% In comparison to \package{arabxetex}, \package{arabluatex} works in
% a very different way. The former relies on the
% \href{http://scripts.sil.org/TECkitIntro}{\texttt{TECkit}} engine
% which converts Arab\TeX\ input on the fly into Unicode Arabic
% script, whereas the latter passes Arab\TeX\ input on to a set of Lua
% functions. At first, \LaTeX\ commands are taken care of in different
% ways: some, as \cs{emph}, \cs{textbf} and the like are expected to
% have Arabic text as arguments, while others, as \cs{LR}, for
% \enquote{left-to-right text}, are not. Then, once what is Arabic is
% carefully separated form what is not, it is processed by other Lua
% functions which rely on different sets of correpondence tables to do
% the actual conversion in accordance with one's wishes. Finally, Lua
% returns to \TeX\ the converted strings---which may in turn contain
% some other Arab\TeX\ input yet to be processed---for further
% processing.
%
% \section{The basics of \package{arabluatex}}
% \subsection{Activating \package{arabluatex}}
% \package{arabluatex} is loaded the usual way:\\
% \tcboxverb{\usepackage{arabluatex}}\\
% The only requirement of \package{arabluatex} is {\LuaLaTeX};
% it will complain if the document is compiled with another
% engine. That aside, \package{arabluatex} does not load packages such
% as \package{polyglossia}. Although it can work with
% \package{polyglossia}, it does not require it.
%
% \paragraph{Font setup}
% Any Arabic font can be defined to be used with
% \package{arabluatex}. For example, assuming that \package{fontspec}
% is loaded, this line may be inserted in the preamble, just above the
% line that loads \package{arabluatex}:
% \arabluabox{\cs{newfontfamily}\cs{arabicfont}\marg{fontname}[Script=Arabic]}
% \noindent where \meta{fontname} is the standard name of the Arabic
% font to be used.
%
% By default, if no Arabic font is selected, \package{arabluatex} will
% issue a warning message and attempt to load the Amiri
% font\footnote{\textcite{pkg:amiri}.} like so:---\\
% \tcboxverb{\newfontfamily\arabicfont{Amiri}[Script=Arabic]}
% \begin{quoting}
%   \textsc{Rem.} By default Amiri places the \arb[trans]{kasraT} in
%   combination with the \arb[trans]{ta^sdId} below the consonant,
%   like so: \arb{BBi}\,. That is correct, as at least in the oldest
%   manuscripts {\renewfontfamily\arabicfont{Amiri}[Script=Arabic,
%   RawFeature={+ss05}]\arb{BBi}} may stand for \arb{BBa} as
%   well as \arb{BBi}\,. See \textcite[i. 14 C--D]{Wright}. The placement
%   of the \arb[trans]{kasraT} above the consonant may be obtained by
%   selecting the |ss05| feature of the Amiri font, like
%   so:---\footnote{See the documentation of \package{amiri},
%   \textcite[6]{pkg:amiri}.}\\
%   \tcboxverb{\newfontfamily\arabicfont{Amiri}[Script=Arabic,RawFeature={+ss05}]}
%
%   Other Arabic fonts may behave differently.
% \end{quoting}
%
% \subsection{Options}
% \label{sec:options}
% \package{arabluatex} may be loaded with five global options, the
% first four of which are mutually exclusive and may be overriden at
% any point of the document (see below \vref{sec:local-options}):
% 
% \DescribeOption{voc}\hfill\tcboxverb{Default}\\ In this mode,
% which is the one selected by default, every short vowel written
% generates its corresponding diacritical mark: \arb[trans]{.dammaT}
% (\arb{Bu}), \arb[trans]{fat.haT} (\arb{Ba}) and \arb[trans]{kasraT}
% (\arb{Bi}). If a vowel is followed by |N|, viz. \meta{uN, aN, iN},
% then the corresponding \arb[trans]{tanwIn} (\arb{BuN}, \arb{B|aN}\,,
% \arb{TaN}, \arb{BaN_A} or \arb{BiN}) is generated. Finally, \meta{u,
% a, i} at the commencement of a word indicate a \enquote{connective
% \arb[trans]{'alif}\,} (\arb[trans]{'alifu 'l-wa.sli}), but |voc|
% mode does not show the \arb[trans]{wa.slaT} above the
% \arb[trans]{'alif}; instead, the accompanying vowel may be expressed
% at the beginning of a sentence (\arb{u} \arb{a} \arb{i}).
%
% \DescribeOption{fullvoc}\\ \label{fullvoc-mode}In addition to what
% the |voc| mode does, |fullvoc| expresses the \arb[trans]{sukUn} and
% the \arb[trans]{wa.slaT}.
%
% \DescribeOption{novoc}\\ None of the diacritics is showed in |novoc|
% mode, unless otherwise specified (see \enquote{quoting} technique
% below \vref{sec:quoting}).
%
% \DescribeOption{trans} \\ \label{ref:describe-trans}This mode
% transliterates the Arab\TeX\ input into one of the accepted
% standards. At present, three standards are supported (see below
% \vref{sec:transliteration} for more details):
% \begin{description}
% \item[dmg] \emph{Deutsche Morgenländische Gesellschaft}, which is
%   selected by default;
% \item[loc] \emph{Library of Congress};
% \item[arabica] \emph{Arabica}.
% \end{description}
% More standards will be included in future releases of
% \package{arabluatex}.
%
% \label{ref:export-global-opt}
% \DescribeOption{export} |export|$=$|true|\verb+|+|false|
% \hfill\tcboxverb{Default: false}\\ \label{export-mode}
% \NEWfeature{v.1.13} This option acts as a named argument and does
% not need a value as it defaults to |true| if it is used. It enables
% \package{arabluatex} to produce a duplicate of the original |.tex|
% source file in which all \textsc{ascii} strings are replaced with
% Unicode equivalents. See below \vref{sec:arabtex2utf} for more
% information.
% 
% \subsubsection{Classic contrasted with modern typesetting of Arabic}
% \label{sec:classic-modern-typesetting}
% \NEWfeature{v.1.2} By default, \package{arabluatex} typesets Arabic
% in a classic, traditional style the most prominent features of which
% are the following:
% \begin{itemize}
% \item \enquote*{Classic} \arb[trans]{maddaT}: when
% \arb[trans]{'alif} and \arb[trans]{hamzaT} accompanied by a simple
% vowel or \arb[trans]{tanwIn} is preceded by an \arb[trans]{'alif} of
% prolongation (\arb[voc]{BA}), then a mere \arb[trans]{hamzaT} is
% written on the line, and a \arb[trans]{maddaT} is placed over the
% \arb[trans]{'alif}, like so:---
% \begin{quote}
%   |samA'uN| \arb[voc]{samA'uN} \arb[trans]{samA'uN}, |jA'a|
%   \arb[voc]{jA'a} \arb[trans]{jA'a}, |yatasA'alUna|
%   \arb[voc]{yatasA'alUna} \arb[trans]{yatasA'alUna}\footnote{Note
%   that in old mss. such forms as \arb[voc]{samA"'a"'a},
%   \arb[voc]{jA"'a"'a} are also found; see \textcite[i. 24
%   D]{Wright}.} (see \vpageref{ref:madda} for further details).
% \end{quote}
% \item The euphonic \arb[trans]{ta^sdId} is generated (see
%   \vpageref{ref:euphonic-tashdid}).
% \item In |fullvoc| mode, the \arb[trans]{sukUn} is expressed.
% \item In such words as \arb{^say'aN}, \arb{.zim'aN} and the like,
%   the \arb[trans]{hamzaT} alone is not written over the letter
%   \arb[trans]{yA'} with no diacritical points below as in
%   \arb{^sayy"'aN}, \arb{.zimy"'aN}, but over a horizontal stroke
%   placed in the continuation of the preceding letter. %
% \iffalse
%<*example>
% \fi
\begin{tcblisting}{text only}
  Please note that only few Arabic fonts provide such contrivances. In
  case this feature is not supported by some Arabic font, it is
  advisable to use \cs{SetArbEasy}.
\end{tcblisting}
% \iffalse
%</example>
% \fi
% \end{itemize}
% 
% \DescribeMacro{\SetArbEasy} \NEWfeature{v1.4.4} Such refinements as
% \enquote*{classic} \arb[trans]{maddaT} may be discarded by the
% \cs{SetArbEasy} command, either globally in the preamble or locally
% at any point of the document. The difference between \cs{SetArbEasy}
% and its \enquote*{starred} version
% \DescribeMacro{\SetArbEasy*}\cs{SetArbEasy*} is that the former
% keeps the \arb[trans]{sukUn} that is generated by the |fullvoc|
% mode, while the latter further takes it away. Default
% \enquote*{classic} rules may be set back at any point of the
% document with the \DescribeMacro{\SetArbDflt}\cs{SetArbDflt}
% command. \NEWfeature{v1.6} \DescribeMacro{\SetArbDflt*}Assimilation
% rules laid on \vref{ref:assimilation} may also be applied by the
% \enquote*{starred} version of this command \cs{SetArbDflt*} either
% in the preamble or at any point of the document.\footnote{For an
% example, see \vref{sec:poetry-example}.} Examples follow:---
% \begin{enumerate}
% \item \cs{SetArbDflt}:
%   \begin{enumerate}
%   \item |voc| \arb[voc]{wa-mAta istisqA'aN qabla 'an yutimma
%     kitAba-hu fI nujUm-i 'l-samA'-i}
%   \item |fullvoc| \arb[fullvoc]{wa-mAta istisqA'aN qabla 'an yutimma
%     kitAba-hu fI nujUm-i 'l-samA'-i}
%   \item |trans| \arb[trans]{wa-mAta istisqA'aN qabla 'an yutimma
%     kitAba-hu fI nujUm-i 'l-samA'-i}
%   \end{enumerate}
% \item \cs{SetArbDflt*}:
%   \begin{enumerate}\SetArbDflt*
%   \item |voc| \arb[voc]{wa-mAta istisqA'aN qabla 'an yutimma
%     kitAba-hu fI nujUm-i 'l-samA'-i}
%   \item |fullvoc| \arb[fullvoc]{wa-mAta istisqA'aN qabla 'an yutimma
%     kitAba-hu fI nujUm-i 'l-samA'-i}
%   \item |trans| \arb[trans]{wa-mAta istisqA'aN qabla 'an yutimma
%     kitAba-hu fI nujUm-i 'l-samA'-i}\SetArbDflt
%   \end{enumerate}
% \item \cs{SetArbEasy}:
%   \begin{enumerate}\SetArbEasy
%   \item |voc| \arb[voc]{wa-mAta istisqA'aN qabla 'an yutimma
%     kitAba-hu fI nujUm-i 'l-samA'-i}
%   \item |fullvoc| \arb[fullvoc]{wa-mAta istisqA'aN qabla 'an yutimma
%     kitAba-hu fI nujUm-i 'l-samA'-i}
%   \item |trans| \arb[trans]{wa-mAta istisqA'aN qabla 'an yutimma
%     kitAba-hu fI nujUm-i 'l-samA'-i}\SetArbDflt
%   \end{enumerate}
% \item \cs{SetArbEasy*}:
%   \begin{enumerate}\SetArbEasy*
%   \item |voc| \arb[voc]{wa-mAta istisqA'aN qabla 'an yutimma
%     kitAba-hu fI nujUm-i 'l-samA'-i}
%   \item |fullvoc| \arb[fullvoc]{wa-mAta istisqA'aN qabla 'an yutimma
%     kitAba-hu fI nujUm-i 'l-samA'-i}
%   \item |trans| \arb[trans]{wa-mAta istisqA'aN qabla 'an yutimma
%     kitAba-hu fI nujUm-i 'l-samA'-i}\SetArbDflt
%   \end{enumerate}
% \end{enumerate}
%
% \iffalse
%<*example>
% \fi
\begin{tcblisting}{text only}
  Please note that this document is typeset with \cs{SetArbDflt}
  throughout.
\end{tcblisting}
% \iffalse
%</example>
% \fi
%
%\subsection{Typing Arabic}
% \DescribeMacro{\arb} Once \package{arabluatex} is loaded, a
% \cs{arb}\marg{Arabic text} command is available for inserting Arabic
% text in paragraphs, like so:---%
% \iffalse
%<*example>
% \fi
\begin{example}
  From \textcite[i. 1 A]{Wright}:--- Arabic, like Hebrew and
  Syriac, is written and read from right to left. The letters
  of the alphabet (\arb{.hurUf-u 'l-hijA'-i}, \arb{.hurUf-u
    'l-tahajjI}, \arb{al-.hurUf-u 'l-hijA'iyyaT-u}, or
  \arb{.hurUf-u 'l-mu`jam-i}) are twenty-eight in number and
  are all consonants, though three of them are also used as
  vowels (see §~3).
\end{example}
% \iffalse
%</example>
% \fi
%
% The following example comes from \textcite[i. 213
% C]{Wright}:--- %
% \iffalse
%<*example>
% \fi
\begin{example}
  \begin{enumerate}[label=\Roman*., start=16]
  \item \arb{fawA`ilu}*.
    \begin{enumerate}[label=\arabic*.]
    \item \arb{fA`aluN}; as \arb{_hAtamuN} \emph{a
        signet-ring}, ...
    \end{enumerate}
  \end{enumerate}
\end{example}
% \iffalse
%</example>
% \fi
%
% \DescribeEnv{arab} Running paragraphs of Arabic text should rather
% be placed inside an \emph{Arabic environment}
% 
% \iffalse
%<*example>
% \fi
\begin{code}
  \begin{arab}
  [...]
  \end{arab}
\end{code}
% \iffalse
%</example>
% \fi
% like so:---\label{ref:juha-code}
% \setquotestyle{arabic}
% \iffalse
%<*example>
% \fi
\begin{example}
 \begin{arab}
   'at_A .sadIquN 'il_A ju.hA ya.tlubu min-hu .himAra-hu
   li-yarkaba-hu fI safraTiN qa.sIraTiN fa-qAla la-hu:
   \enquote{sawfa 'u`Idu-hu 'ilay-ka fI 'l-masA'-i
   wa-'adfa`u la-ka 'ujraTaN.} fa-qAla ju.hA:
   \enquote{'anA 'AsifuN jiddaN 'annI lA 'asta.tI`u 'an
   'u.haqqiqa la-ka ra.gbata-ka fa-'l-.himAr-u laysa hunA
   'l-yawm-a.}  wa-qabla 'an yutimma ju.hA kalAma-hu bada'a
   'l-.himAr-u yanhaqu fI 'i.s.tabli-hi. fa-qAla la-hu
   .sadIqu-hu: \enquote{'innI 'asma`u .himAra-ka yA ju.hA
   yanhaqu.} fa-qAla la-hu ju.hA: \enquote{.garIbuN
   'amru-ka yA .sadIqI 'a-tu.saddiqu 'l-.himAr-a
   wa-tuka_d_diba-nI?}
  \end{arab}
\end{example}
% \iffalse
%</example>
% \fi
% \setquotestyle{english}
%
% \subsubsection{Local options}
% \label{sec:local-options}
% As seen above in \vref{sec:options}, \package{arabluatex} may be
% loaded with four mutually exclusive global options: |voc| (which is
% the default option), |fullvoc|, |novoc| and |trans|. Whatever choice
% has been made globally, it may be overriden at any point of the
% document, as the \cs{arb} command may take any of the |voc|,
% |fullvoc|, |novoc| or |trans| modes as optional argument, like
% so:---
% \begin{itemize}
% \item \DescribeOption{voc} \cs{arb}|[voc]|\marg{Arabic text};
% \item \DescribeOption{fullvoc} \cs{arb}|[fullvoc]|\marg{Arabic text};
% \item \DescribeOption{novoc} \cs{arb}|[novoc]|\marg{Arabic text};
% \item \DescribeOption{trans} \cs{arb}|[trans]|\marg{Arabic text}.
% \end{itemize}
%
% The same optional arguments may be passed to the environment |arab|:
% one may have \cs{begin}|{arab}|\oarg{mode}|...|\cs{end}|{arab}|,
% where \meta{mode} may be any of |voc|, |fullvoc|, |novoc| or
% |trans|.
%
%\section{Standard Arab\TeX\ input}
%\subsection{Consonants}
% \Cref{tab:arabtex-consonants} gives the Arab\TeX\ equivalents for
% all of the Arabic consonants.
%
% \addlines[2]
% \newcommand{\dmg}[1]{%
%   \SetTranslitConvention{dmg}%
%   \arb[trans]{#1}}
% \newcommand{\loc}[1]{%
%   \SetTranslitConvention{loc}%
%   \arb[trans]{#1}\SetTranslitConvention{dmg}}
% \newcommand{\brill}[1]{%
%   \SetTranslitConvention{arabica}%
%   \arb[trans]{#1}\SetTranslitConvention{dmg}}
% \begin{longtable}{lllll}
% \bottomrule
% \caption*{\Cref*{tab:arabtex-consonants}: Standard Arab\TeX\
% (consonants)}
% \endfoot
% \captionlistentry{Arab\TeX\ consonants}\\[-1em]
% \toprule
% Letter & \multicolumn{3}{l}{Transliteration\footnotemark}
% & Arab\TeX\ notation \\
%        & \texttt{dmg} & \texttt{loc} & \texttt{arabica} & \\ \midrule
% \endfirsthead
% \toprule
% Letter & \multicolumn{3}{l}{Transliteration}
% & Arab\TeX\ notation \\
%        & \texttt{dmg} & \texttt{loc} & \texttt{arabica} & \\ \midrule
% \endhead \footnotetext{See below \vref{sec:transliteration}.}
% \label{tab:arabtex-consonants}
% \arb[voc]{'i 'a 'u}\footnote{See below, \emph{Rem{.} a.} For
% \arb[trans]{'alif} as a consonant, see \textcite[i. 16
% D]{Wright}. The \arb[trans]{hamzaT} itself is encoded \texttt{<'>}
% and may be followed by either \meta{u, a} or \meta{i}. See below
% \vref{ref:hamza}.} & \dmg{'u 'a 'i} &
% \loc{|'u, |'a, |'i} & \brill{|'u, |'a, |'i} & |'u| or |'a| or |'i| \\
% \arb[novoc]{b} & \dmg{b} & \loc{b} & \brill{b} & |b| \\
% \arb[novoc]{t} & \dmg{t} & \loc{t} & \brill{t} & |t| \\
% \arb[novoc]{_t} & \dmg{_t} & \loc{_t} & \brill{_t} & |_t| \\
% \arb[novoc]{j} & \dmg{j} & \loc{j} & \brill{j} & |^g| or |j| \\
% \arb[novoc]{.h} & \dmg{.h} & \loc{.h} & \brill{.h} & |.h| \\
% \arb[novoc]{x} & \dmg{x} & \loc{x} & \brill{x} & |_h| or |x|\\
% \arb[novoc]{d} & \dmg{d} & \loc{d} & \brill{d} & |d| \\
% \arb[novoc]{_d} & \dmg{_d} & \loc{_d} & \brill{_d} & |_d| \\
% \arb[novoc]{r} & \dmg{r} & \loc{r} & \brill{r} & |r| \\
% \arb[novoc]{z} & \dmg{z} & \loc{z} & \brill{z} & |z| \\
% \arb[novoc]{s} & \dmg{s} & \loc{s} & \brill{s} & |s| \\
% \arb[novoc]{^s} & \dmg{^s} & \loc{^s} & \brill{^s} & |^s| \\
% \arb[novoc]{.s} & \dmg{.s} & \loc{.s} & \brill{.s} & |.s| \\
% \arb[novoc]{.d} & \dmg{.d} & \loc{.d} & \brill{.d} & |.d| \\
% \arb[novoc]{.t} & \dmg{.t} & \loc{.t} & \brill{.t} & |.t| \\
% \arb[novoc]{.z} & \dmg{.z} & \loc{.z} & \brill{.z} & |.z| \\
% \arb[novoc]{`} & \dmg{`} & \loc{`} & \brill{`} & |`| \\
% \arb[novoc]{.g} & \dmg{.g} & \loc{.g} & \brill{.g} & |.g| \\
% \arb[novoc]{f} & \dmg{f} & \loc{f} & \brill{f} & |f| \\
% \arb[novoc]{q} & \dmg{q} & \loc{q} & \brill{q} & |q| \\
% \arb[novoc]{k} & \dmg{k} & \loc{k} & \brill{k} & |k| \\
% \arb[novoc]{l} & \dmg{l} & \loc{l} & \brill{l} & |l| \\
% \arb[novoc]{m} & \dmg{m} & \loc{m} & \brill{m} & |m| \\
% \arb[novoc]{n} & \dmg{n} & \loc{n} & \brill{n} & |n| \\
% \arb[novoc]{h} & \dmg{h} & \loc{h} & \brill{h} & |h| \\
% \arb[novoc]{w} & \dmg{w} & \loc{w} & \brill{w} & |w| \\
% \arb[novoc]{y} & \dmg{y} & \loc{y} & \brill{y} & |y|\footnote{For
% the letter \arb[novoc]{.y} with no diacritical points below, see
% \emph{Rem{.} b.} below.} \\
% \arb[novoc]{T} & \dmg{aT} & \loc{aT} & \brill{aT} & |T| \\
% \end{longtable}
% \begin{quoting}
%   \textsc{Rem.}~\emph{a.} Please note that in all cases of elision,
%   the \arb[trans]{'alifu 'l-wa.sli} is expressed only by the vowel
%   that accompanies the omitted \arb[trans]{hamzaT}: \meta{u, a, i}
%   as in |wa-inhazama| \arb[fullvoc]{wa-inhazama}
%   \arb[trans]{wa-inhazama}. For more details on the definite article
%   and the \arb[trans]{'alifu 'l-wa.sli} see
%   \vref{ref:definite-article}.
%
%   That said, \arb{.A} as a consonant is actually the \emph{spiritus
%   lenis} of the Greeks and is distinguished by the
%   \arb[trans]{hamzaT} \arb[novoc]{(|"')} as it is shown in the above
%   table. However, the bare \arb[trans]{'alif} may also be encoded as
%   |.A| whether it be followed by a vowel or not, like so: |wa-.An|
%   \arb{wa-.An} \arb[trans]{wa-.An} (where the dot symbolizes the
%   absence of vowel), |wa-.Aan| \arb{wa-.Aan} \arb[trans]{wa-.Aan},
%   |wa-.Ain| \arb{wa-.Ain} \arb[trans]{wa-.Ain}.
%   
%   \textsc{Rem.}~\emph{b.} The letter \arb[novoc]{y} with two points
%   below, \arb{al-yA'u 'l-mu_tannATu min ta.hti-hA}, may also be
%   written without diacritical points as \arb[novoc]{Y}. When it is
%   used as a consonant, it is encoded |aY|, where |a| recalls the
%   \arb[trans]{fat.haT} placed above the preceding letter in
%   vocalized Arabic, like so: |qaY'uN| \arb{qaY'uN}
%   \arb[trans]{qaY'uN}, |^saY'uN| \arb{^saY'uN} \arb[trans]{^saY'uN},
%   |^saY'aN| \arb[trans]{^saY'aN} \arb{^saY'aN}.
%
%   The same result may be achieved by encoding this letter as |.y|,
%   like so: |qa.y'uN| \arb{qa.y'uN} \arb[trans]{qa.y'uN}, |^sa.y'uN|
%   \arb{^sa.y'uN} \arb[trans]{^sa.y'uN}, |^sa.y'aN|
%   \arb[trans]{^sa.y'aN} \arb{^sa.y'aN}.
% \end{quoting}
%
% \subsection{Additional characters}
% \changes{v1.8.5}{2017/06/20}{Six additional Persian characters are
% now available}
% \NEWfeature{v1.8.5}\Cref{tab:arabtex-additional-characters} gives
% the Arab\TeX\ equivalents for some additional Persian characters.
% 
% \begin{longtable}{lllll}
% \bottomrule
% \caption*{\Cref*{tab:arabtex-additional-characters}: Standard
% Arab\TeX\ (additional characters)}
% \endfoot
% \captionlistentry{Arab\TeX\ additional characters}\\[-1em]
% \toprule
% Letter & \multicolumn{3}{l}{Transliteration\footnotemark}
% & Arab\TeX\ notation \\
%        & \texttt{dmg} & \texttt{loc} & \texttt{arabica}\footnotemark
%        & \\ \midrule
% \endfirsthead
% \toprule
% Letter & \multicolumn{3}{l}{Transliteration}
% & Arab\TeX\ notation \\
%        & \texttt{dmg} & \texttt{loc} & \texttt{arabica} & \\ \midrule
% \endhead
% \addtocounter{footnote}{-1}
% \footnotetext{See below \vref{sec:transliteration}.}
% \stepcounter{footnote}
% \footnotetext{The characters that are listed in this table are not
% included in this standard. However, as \texttt{arabica} is based on
% \texttt{dmg}, the \texttt{dmg} equivalents have been used here.}
% \label{tab:arabtex-additional-characters}
% \hskip-1em\arb[novoc]{p} & \dmg{p} & \loc{p} & \brill{p} & \verb|p| \\
% \arb[novoc]{^c} & \dmg{^c} & \loc{^c} & \brill{^c} & \verb|^c| \\
% \arb[novoc]{^z} & \dmg{^z} & \loc{^z} & \brill{^z} & \verb|^z| \\
% \arb[novoc]{v}\footnote{\label{fn:not-in-dmg}This character is not found in
% \textcite[2]{dmg}. It is taken from the \textcite{din31635} standard.} &
% \dmg{v} & \loc{v} & \brill{v} & \verb|v| \\
% \arb[novoc]{g} & \dmg{g} & \loc{g} & \brill{g} & \verb|g| \\
% \arb[novoc]{^n}\footnote{See \cref{fn:not-in-dmg}.} & \dmg{^n} &
% \loc{^n} & \brill{^n} & \verb|^n| \\
% \end{longtable}
% \begin{quoting}
%   \textsc{Rem.} The alveolar consonants \arb[novoc]{^c} and
%   \arb[novoc]{^z} are processed as solar letters by
%   \package{arabluatex}.
% \end{quoting}
% 
% \subsection{Vowels}
% \subsubsection{Long vowels}
% \Cref{tab:arabtex-long-vowels} gives the Arab\TeX\ equivalents for
% the Arabic long vowels.
% \enlargethispage{1\baselineskip}
% \begin{longtable}{lllll}
% \bottomrule
% \caption*{\Cref*{tab:arabtex-long-vowels}: Standard Arab\TeX\ (long
% vowels)}
% \endfoot
% \captionlistentry{Arab\TeX\ long vowels}\\[-1em]
% \toprule
% Letter & \multicolumn{3}{l}{Transliteration\footnotemark}
% & Arab\TeX\ notation \\
%        & \texttt{dmg} & \texttt{loc} & \texttt{arabica} & \\ \midrule
% \endfirsthead
% \toprule
% Letter & \multicolumn{3}{l}{Transliteration}
% & Arab\TeX\ notation \\
%        & \texttt{dmg} & \texttt{loc} & \texttt{arabica} & \\ \midrule
% \endhead \footnotetext{See below \vref{sec:transliteration}.}
% \label{tab:arabtex-long-vowels}
% \arb[novoc]{A} & \dmg{A} & \loc{A} & \brill{A} & \verb|A| \\
% \arb[novoc]{U} & \dmg{U} & \loc{U} & \brill{U} & \verb|U| \\
% \arb[novoc]{I} & \dmg{I} & \loc{I} & \brill{I} &
% \verb|I|\footnote{For the letter \arb[novoc]{I} with no diacritical
% points, see \emph{Rem{.} c.} below.} \\
% \arb[novoc]{_A}\footnote{$=$ \arb[trans]{al-'alif-u 'l-maq.sUraT-u}.}
%            & \dmg{_A} & \loc{_A} & \brill{_A} & \verb|_A| or \verb|Y| \\
% \arb[novoc]{B_a} & \dmg{B_a} & \loc{B_a} & \brill{B_a} & \verb|_a| \\
% \arb[novoc]{B_u} & \dmg{B_u} & \loc{B_u} & \brill{B_u} & \verb|_u| \\
% \arb[novoc]{B_i} & \dmg{B_i} & \loc{B_i} & \brill{B_i} & \verb|_i| \\
% \end{longtable}
%
% \begin{quoting}
%   \textsc{Rem.}~\emph{a.} The long vowels \arb[trans]{A, U, I},
%   otherwise called \arb[trans]{.hurUf-u 'l-madd-i}, \emph{the
%   letters of prolongation}, involve the placing of the short vowels
%   \arb[trans]{Ba, Bu, Bi} before the letters \arb[novoc]{A},
%   \arb[novoc]{U}, \arb[novoc]{I} respectively. \package{arabluatex}
%   does that automatically in case any from |voc|, |fullvoc| or
%   |trans| modes is selected e.g. \arb[voc]{qAla} \arb[trans]{qAla},
%   \arb[voc]{qIla} \arb[trans]{qIla}, \arb[voc]{yaqUlu}
%   \arb[trans]{yaqUlu}.
%
%   \textsc{Rem.}~\emph{b.} Defective writings, such as
%   \arb[novoc]{B_a}, \arb[trans]{al-'alif-u 'l-ma.h_dUfaT-u}, or
%   defective writings of \arb[trans]{B_u} and \arb[trans]{B_i} are
%   encoded |_a| |_u| and |_i| respectively, e.g. |_d_alika|
%   \arb[voc]{_d_alika}, |al-mal_a'ikaT-u| |'l-ra.hm_an-u|
%   \arb[voc]{al-mal_a'ikaT-u 'l-ra.hm_an-u}, |.hu_dayfaT-u| |bn-u|
%   |'l-yamAn_i| \arb[fullvoc]{.hu_dayfaT-u bn-u 'l-yamAn_i} for
%   \arb[trans]{\uc{.hu_dayfaT-u} bn-u 'l-\uc{yamAn_i}}, etc.
%
%   \textsc{Rem.}~\emph{c.} The letter \arb[novoc]{y} with two points
%   below, \arb{al-yA'u 'l-mu_tannATu min ta.hti-hA}, may also be
%   written without diacritical points as \arb[novoc]{Y}. When it is
%   used as a long vowel, it is encoded |iY|, where |i| recalls the
%   \arb[trans]{kasraT} placed below the preceding letter in vocalized
%   Arabic, like so: |liY| \arb{liY} \arb[trans]{liY}, |yam^siY|
%   \arb{yam^siY} \arb[trans]{yam^siY}.
% \end{quoting}
% 
% \subsubsection{Short vowels}
% \Cref{tab:arabtex-short-vowels} gives the Arab\TeX\ equivalents for
% the Arabic short vowels.
% 
% \begin{longtable}{lllll}
% \bottomrule
% \caption*{\Cref*{tab:arabtex-short-vowels}: Standard Arab\TeX\
% (short vowels)}
% \endfoot
% \captionlistentry{Arab\TeX\ short vowels}\\[-1em]
% \toprule
% Letter & \multicolumn{3}{l}{Transliteration\footnotemark}
% & Arab\TeX\ notation \\
%        & \texttt{dmg} & \texttt{loc} & \texttt{arabica} & \\ \midrule
% \endfirsthead
% \toprule
% Letter & \multicolumn{3}{l}{Transliteration}
% & Arab\TeX\ notation \\
%        & \texttt{dmg} & \texttt{loc} & \texttt{arabica} & \\ \midrule
% \endhead \footnotetext{See below \vref{sec:transliteration}.}
% \label{tab:arabtex-short-vowels}
% \arb[voc]{Ba} & \dmg{Ba} & \loc{Ba} & \brill{Ba} & \verb|a| \\
% \arb[voc]{Bu} & \dmg{Bu} & \loc{Bu} & \brill{Bu} & \verb|u| \\
% \arb[voc]{Bi} & \dmg{Bi} & \loc{Bi} & \brill{Bi} & \verb|i| \\
% \arb[voc]{BaN} & \dmg{BaN} & \loc{BaN} & \brill{BaN} & \verb|aN| \\
% \arb[voc]{BuN} & \dmg{BuN} & \loc{BuN} & \brill{BuN} & \verb|uN| \\
% \arb[voc]{BiN} & \dmg{BiN} & \loc{BiN} & \brill{BiN} & \verb|iN| \\
% \end{longtable}
%
% Whether Arabic texts be vocalized or not is essentially a matter of
% personal choice. So one may use |voc| mode and decide not to write
% vowels except at some particular places for disambiguation purposes,
% or use |novoc| mode, not write vowels---as |novoc| normally does not
% show them---except, again,  where disambiguation is needed.\footnote{See
% below \vref{sec:quoting}.}
%
% \iffalse
%<*example>
% \fi
\begin{tcblisting}{text only}
  However, it may be wise to always write the vowels, leaving to the
  various modes provided by \package{arabluatex} to take care of
  showing or not showing the vowels.
\end{tcblisting}
% \iffalse
%</example>
% \fi
%
% That said, there is no need to write the short vowels
% \arb[trans]{fat.haT}, \arb[trans]{.dammaT} or \arb[trans]{kasraT}
% except in the following cases:---
% \begin{itemize}
% \item at the commencement of a word, to indicate that a connective
% \arb[trans]{'alif} is needed, with the exception of the article (see
% below \vref{sec:quoting});
% \item when \package{arabluatex} needs to perform a contextual
%   analysis to determine the carrier of the \arb[trans]{hamzaT};
% \item in the various transliteration modes, as vowels are always
% expressed  in romanized Arabic.
% \end{itemize}
%
%\section{\package{arabluatex} in action}
%\subsection{The vowels and diphthongs}
% \paragraph{Short vowels} As said above, they are written \meta{a, u,
% i}:
% \begin{quote}
%   |_halaqa| (or |xalaqa|) \arb[voc]{xalaqa} \arb[trans]{xalaqa},
%   |^samsuN| \arb[voc]{^samsuN} \arb[trans]{^samsuN}, |karImuN|
%   \arb[voc]{karImuN} \arb[trans]{\uc{karImuN}}.
%
%   |bi-hi| \arb[voc]{bi-hi} \arb[trans]{bi-hi}, |'aqi.tuN|
%   \arb[voc]{'aqi.tuN} \arb[trans]{'aqi.tuN}.
%
%   |la-hu| \arb[voc]{la-hu} \arb[trans]{la-hu}, |.hujjaTuN|
%   \arb[voc]{.hujjaTuN} \arb[trans]{.hujjaTuN}.
% \end{quote}
%
%\paragraph{Long vowels} They are written \meta{U, A, I}:
% \begin{quote}
%   |qAla| \arb[voc]{qAla} \arb[trans]{qAla}, |bI`a| \arb[voc]{bI`a}
%   \arb[trans]{bI`a}, |.tUruN| \arb[voc]{.tUruN} \arb[trans]{.tUruN},
%   |.tInuN| \arb[voc]{.tInuN} \arb[trans]{.tInuN}, |murU'aTuN|
%   \arb[voc]{murU'aTuN} \arb[trans]{murU'aTuN}.
% \end{quote}
%
% \paragraph{\texorpdfstring{\arb[trans]{'alif maq.sUraT}}{ʾalif
% maqṣūrah}} It is written \meta{\_A} or \meta{Y}:
% \begin{quote}
%   |al-fat_A| \arb[voc]{al-fat_A} \arb[trans]{al-fat_A}, |al-maqh_A|
%   \arb[voc]{al-maqh_A} \arb[trans]{al-maqh_A}, |'il_A|
%   \arb[voc]{'il_A} \arb[trans]{'il_A}.
% \end{quote}
% 
% \paragraph{\texorpdfstring{\arb[trans]{'alif} \emph{otiosum}}{ʾalif
% otiosum}} Said \arb[trans]{'alif-u 'l-wiqAyaT-i}, \enquote{the
% guarding \arb[trans]{'alif}}\,, after \arb[novoc]{U} at the end of a
% word, both when preceded by \arb[trans]{.dammaT} and by
% \arb[trans]{fat.haT} is written \meta{UA} or \meta{aW, aWA}:
% \begin{quote}
%   |na.sarUA| \arb[voc]{na.sarUA} \arb[trans]{na.sarUA}, |katabUA|
%   \arb[voc]{katabUA} \arb[trans]{katabUA}, |ya.gzUA|
%   \arb[voc]{ya.gzUA} \arb[trans]{ya.gzUA}, |ramaW|
%   \arb[fullvoc]{ramaW} \arb[trans]{ramaW}, |banaWA|
%   \arb[fullvoc]{banaWA}, \arb[trans]{banaWA}.
% \end{quote}
% 
% \paragraph{\texorpdfstring{\arb[trans]{'alif ma.h_dUfaT} and
% defective \arb[trans]{U, I}}{ʾalif maḥḏūfah and defective ū, ī}}
% They are written \meta{\_a, \_i \_u}:
% \begin{quote}
%   |al-l_ah-u| \arb[voc]{al-l_ah-u} \arb[trans]{al-l_ah-u},
%   |'il_ahuN| \arb[voc]{'il_ahuN} \arb[trans]{'il_ahuN}.
%
%   |al-ra.hm_an-u| \arb[voc]{al-ra.hm_an-u}
%   \arb[trans]{al-ra.hm_an-u}, |l_akin| \arb[voc]{l_akin}
%   \arb[trans]{l_akin}, |h_ahunA| \arb[voc]{h_ahunA}
%   \arb[trans]{h_ahunA}, |.hunayn-u| |bn-u| |'is.h_aq-a|
%   \arb[voc]{.hunayn-u bn-u 'is.h_aq-a} \arb[trans]{\uc{.hunayn}-u
%   bn-u \uc{'is.h_aq-a}}, |rabb_i| \arb[voc]{rabb_i}
%   \arb[trans]{rabb_i}, |al-`A.s_i| \arb[voc]{al-`A.s_i}
%   \arb[trans]{al-\uc{`A.s_i}}.
% \end{quote}
% 
% \paragraph{\texorpdfstring{Silent
% \arb[novoc]{U}/\arb[novoc]{I}}{Silent ي/و}}
% Some words ending with \arb[voc]{BAT} are usually written
% \arb[voc]{BawT} or \arb[voc]{B_aUT} instead of \arb[voc]{BAT}: see
% \textcite[i. 12 A]{Wright}. \package{arabluatex} preserves that
% particular writing; the same applies to words ending in
% \arb[voc]{BayT} for \arb[voc]{BAT}\,. Long vowels \meta{U, I} shall
% receive no \arb[trans]{sukUn} after a \arb[trans]{'alif ma.h_dUfaT}
% and are discarded in |trans| mode:
% \begin{quote}
%   |.hay_aUTuN| \arb[voc]{.hay_aUTuN} \arb[trans]{.hay_aUTuN},
%   |.sal_aUTuN| \arb[voc]{.sal_aUTuN} \arb[trans]{.sal_aUTuN},
%   |mi^sk_aUTuN| \arb[voc]{mi^sk_aUTuN} \arb[trans]{mi^s\-k_aUTuN},
%   |tawr_aITuN| \arb[voc]{tawr_aITuN} \arb[trans]{tawr_aITuN}.
%
%   And so also: |al-rib_aIT-u| \arb[voc]{al-rib_aIT-u}
%   \arb[trans]{al-rib_aIT-u}.
% \end{quote}
%
% \paragraph{\texorpdfstring{\arb[trans]{\uc{`amruNU}}, and the silent
% \arb[novoc]{U}}{ʿAmrun, and the silent و}} To that name a silent
% \arb[novoc]{U} is added to distinguish it from
% \arb[trans]{\uc{`umar-u}}: see \textcite[i. 12 C]{Wright}. In no way
% this affects the sound of the \arb[trans]{tanwIn}, so it has to be
% discarded in |trans| mode:
% \begin{quote}
%   |`amruNU| \arb[voc]{`amruNU} \arb[trans]{`amruNU}, |`amraNU|
%   \arb[voc]{`amraNU} \arb[trans]{`amraNU}, |`amriNU|
%   \arb[voc]{`amriNU} \arb[trans]{`amriNU}.
%
%   When the \arb[trans]{tanwIn} falls away \parencite[i.  249
%   B]{Wright}: |`amr-uU| |bn-u| |mu.hammadiN| \arb[fullvoc]{`amr-uU
%   bnu mu.hammadiN} \arb[trans]{\uc{`amr-uU} bn-u
%   \uc{mu.hammadiN}}, |mu.hammad-u| |bn-u| |`amr-iU| |bn-i|
%   |_hAlidiN| \arb[fullvoc]{mu.hammad-u bn-u `amr-iU bn-i _hAlidiN}
%   \arb[trans]{\uc{mu.hammad-u} bn-u \uc{`amr-iU} bn-i
%   \uc{_hAlidiN}}.
%
%   And so also: |al-rib_aUA| \arb[voc]{al-rib_aUA}
%   \arb[trans]{al-rib_aUA}, |ribaNU| \arb[voc]{ribaNU}
%   \arb[trans]{ribaNU}.
% \end{quote}
%
% \paragraph{\texorpdfstring{\arb[trans]{tanwIn}}{tanwīn}}
% The marks of doubled short vowels, \arb{BuN}, \arb{BaN}, \arb{BiN},
% are written \meta{uN, aN, iN} respectively. \package{arabluatex}
% deals with special cases, such as \arb{BaN} taking an \arb[novoc]{A}
% after all consonants except \arb[novoc]{T}, and \arb[trans]{tanwIn}
% preceding \arb[novoc]{Y} as in \arb[voc]{hudaN_A}, which is written
% \meta{aN\_A} or \meta{aNY}:
% \begin{quote}
%   |mAluN| \arb[voc]{mAluN} \arb[trans]{mAluN}, |bAbaN|
%   \arb[voc]{bAbaN} \arb[trans]{bAbaN}, |madInaTaN|
%   \arb[voc]{madInaTaN} \arb[trans]{madInaTaN}, |bintiN|
%   \arb[voc]{bintiN} \arb[trans]{bintiN} |maqhaN_A|
%   \arb[voc]{maqhaN_A} \arb[trans]{maqhaN_A}, |fataNY|
%   \arb[voc]{fataNY} \arb[trans]{fataNY}.
%
%   \package{arabluatex} is aware of special orthographies: |^say'uN|
%   \arb[voc]{^say'uN} \arb[trans]{^say'uN}, |^say'aN|
%   \arb[voc]{^say'aN} \arb[trans]{^say'aN}, |^say'iN|
%   \arb[voc]{^say'iN} \arb[trans]{^say'iN}.
% \end{quote}
%
% In some cases, it may be useful to mark the root form of defective
% words so as to produce a more accurate transliteration of ending
% \arb[trans]{tanwIn}. As seen above, \arb[trans]{tanwIn} preceding
% \arb[novoc]{_A} is written \meta{aN\_A} or \meta{aNY}. Such forms as
% \arb[voc]{qA.diNI} may likewise be written \meta{iNI}:---
% \begin{quote}
%   |al-qA.dI| \arb[voc]{al-qA.dI} \arb[trans]{al-qA.dI}, |qA.diyaN|
%   \arb[voc]{qA.diyaN} \arb[trans]{qA.diyaN}, |qA.diNI|
%   \arb[voc]{qA.diNI} \arb[trans]{qA.diNI}.
% \end{quote}
% 
% \subsection{Other orthographic signs}
% \paragraph{\texorpdfstring{\arb[trans]{tA' marbU.taT}}{tāʾ marbūṭah}}
% It is written \meta{T}:
% \begin{quote}
%   |madInaTuN| \arb[voc]{madInaTuN} \arb[trans]{madInaTuN},
%   |madInaTaN| \arb[voc]{madInaTaN} \arb[trans]{madInaTaN},
%   |madInaTiN| \arb[voc]{madInaTiN} \arb[trans]{madInaTiN}.
% \end{quote}
%
% \paragraph{\texorpdfstring{\arb[trans]{hamzaT}}{hamzah}}
% \label{ref:hamza}It is written \meta{\texttt{'}}, its carrier being
% determined by contextual analysis. In case one wishes to bypass this
% mechanism, he can use the \enquote{quoting} feature that is
% described below in \vref{sec:quoting}.
% \begin{quote}
%   \textbf{Initial \arb[trans]{hamzaT}}: |'asaduN| \arb[voc]{'asaduN}
%   \arb[trans]{'asaduN}, |'u_htuN| \arb[voc]{'u_htuN}
%   \arb[trans]{'u_htuN}, |'iqlIduN| \arb[voc]{'iqlIduN}
%   \arb[trans]{'iqlIduN}, |'anna| \arb[voc]{'anna}
%   \arb[trans]{'anna}, |'inna| \arb[voc]{'inna} \arb[trans]{'inna}.
%
%   \label{ref:initial-hamza}
%   \arb[trans]{hamzaT} followed by the long vowel \arb[novoc]{U} is
%   encoded |'_U|: |'_Ul_A| \arb[voc]{'_Ul_A} \arb[trans]{'_Ul_A}, |'_UlU|
%   \arb[voc]{'_UlU} \arb[trans]{'_UlU}, |'_UlA'ika|
%   \arb[voc]{'_UlA'ika} \arb[trans]{'_UlA'ika}.
%
%   \arb[trans]{hamzaT} followed by the long vowel \arb[novoc]{I} is
%   encoded |'_I|: |'_ImAnuN| \arb[voc]{'_ImAnuN}
%   \arb[trans]{'_ImA\-nuN}\footnote{For another way of encoding the
%   initial \arb[trans]{hamzaT} followed by a long vowel, see the
%   \arb[trans]{ta_hfIf-u 'l-hamzaT-i}\vpageref{ref:taxfif-hamzah}.}.
%
%   \textbf{Middle \arb[trans]{hamzaT}}: |xA.ti'-Ina|
%   \arb[voc]{xA.ti'-Ina} \arb[trans]{xA.ti'-Ina}, |ru'UsuN|
%   \arb[voc]{ru'UsuN}, \arb[trans]{ru'UsuN}, |xa.tI'aTuN|
%   \arb[voc]{xa.tI'aTuN} \arb[trans]{xa.tI'aTuN}, |su'ila|
%   \arb[voc]{su'ila} \arb[trans]{su'ila}, |'as'ilaTuN|
%   \arb[voc]{'as'i\-laTuN} \arb[trans]{'as'ilaTuN}, |mas'alaTuN|
%   \arb[voc]{mas'alaTuN} \arb[trans]{mas'alaTuN}, |'as'alu|
%   \arb[voc]{'as'alu} \arb[trans]{'as'alu}, |yatasA'alUna|
%   \arb[voc]{yatasA'alUna} \arb[trans]{yatasA'alUna}, |murU'aTuN|
%   \arb[voc]{murU'aTuN} \arb[trans]{murU'aTuN}, |taw'amuN|
%   \arb[fullvoc]{taw'amuN} \arb[trans]{taw'amuN}, |ta'xIruN|
%   \arb[fullvoc]{ta'xIruN} \arb[trans]{ta'xIruN},
%   |ta'ax|\allowbreak|xara| \arb[voc]{ta'axxara}
%   \arb[trans]{ta'axxara}, |ji'tu-ka| \arb[voc]{ji'tu-ka}
%   \arb[trans]{ji'tu-ka}, |qA'iluN| \arb[voc]{qA'iluN}
%   \arb[trans]{qA'iluN}, |.hIna'i_diN| \arb[trans]{.hIna'i_diN}
%   \arb[voc]{.hIna'i_diN}, |hay'aTuN| \arb[voc]{hay'aTuN}
%   \arb[trans]{hay\-'aTuN}, |hay'AtuN| \arb[voc]{hay'AtuN}
%   \arb[trans]{hay'AtuN}.
%
%   From \textcite[i. 14 B]{Wright}:--- All consonants, whatsoever,
%   not even \arb[trans]{'alif} \emph{hèmzatum} excepted, admit of
%   being doubled and take \arb[trans]{ta^sdId}. Hence we speak and
%   write |ra''AsuN| \arb[voc]{ra''AsuN} \arb[trans]{ra''AsuN},
%   |sa''AluN| \arb[voc]{sa''AluN} \arb[trans]{sa''AluN}, |na''AjuN|
%   \arb[voc]{na''AjuN} \arb[trans]{na''AjuN}.
%
%   \textbf{Final \arb[trans]{hamzaT}}: |xa.ta'uN| \arb[voc]{xa.ta'uN}
%   \arb[trans]{xa.ta'uN}, |xa.ta'aN| \arb[voc]{xa.ta'aN}
%   \arb[trans]{xa.ta'aN}, |xa.ta'iN| \arb[voc]{xa.ta'iN}
%   \arb[trans]{xa.ta'iN}, |'aqra'u| \arb[voc]{'aqra'u}
%   \arb[trans]{'aqra'u}, |taqra'Ina| \arb[voc]{taqra'Ina}
%   \arb[trans]{taqra'Ina}, |taqra'Una| \arb[voc]{taqra'Una}
%   \arb[trans]{taqra'Una}, |yaqra'na| \arb[fullvoc]{yaqra'na}
%   \arb[trans]{yaqra'na}, |yaxba'Ani| \arb[voc]{yaxba'Ani}
%   \arb[trans]{yaxba'Ani}, |xaba'A| \arb[voc]{xaba'A}
%   \arb[trans]{xaba'A}, |xubi'a| \arb[voc]{xubi'a}
%   \arb[trans]{xubi'a}, |xubi'UA| \arb[voc]{xubi'UA}
%   \arb[trans]{xubi'UA}, |jA'a| \arb[voc]{jA'a} \arb[trans]{jA'a},
%   |ridA'uN| \arb[voc]{ridA'uN} \arb[trans]{ridA'uN}, |ridA'aN|
%   \arb[voc]{ridA'aN} \arb[trans]{ridA'aN}, |jI'a| \arb[voc]{jI'a}
%   \arb[trans]{jI'a}, |radI'iN| \arb[voc]{radI'iN}
%   \arb[trans]{radI'iN}, |sU'uN| \arb[voc]{sU'uN} \arb[trans]{sU'uN},
%   |.daw'uN| \arb[voc]{.daw'uN} \arb[trans]{.daw'uN}, |qay'iN|
%   \arb[voc]{qay'iN} \arb[trans]{qay'iN}, |^sifA'I|
%   \arb[voc]{^sifA'I} \arb[trans]{^sifA'I}, |man^sa'I|
%   \arb[trans]{man^sa'I} \arb[voc]{man^sa'I}, |nisA'uN|
%   \arb[voc]{nisA'uN} \arb[trans]{ni\-sA'uN}, |nisA'u-hu|
%   \arb[voc]{nisA'u-hu} \arb[trans]{nisA'u-hu}, |nisA'i-hi|
%   \arb[voc]{nisA'i-hi} \arb[trans]{nisA'i-hi}, |nisA'I|
%   \arb[voc]{nisA'I} \arb[trans]{nisA'I}.
%
%   |^say'uN| \arb[voc]{^say'uN} \arb[trans]{^say'uN}, |^say'aN|
%   \arb[voc]{^say'aN} \arb[trans]{^say'aN}, |^say'iN|
%   \arb[voc]{^say'iN} \arb[trans]{^say'iN}, |al-^say'-u|
%   \arb[voc]{al-^say'-u} \arb[trans]{al-^say'-u}, |'a^syA'-u|
%   \arb[voc]{'a^syA'-u} \arb[trans]{'a^syA'-u}, |'a^syA'-a|
%   \arb[voc]{'a^syA'-a} \arb[trans]{'a^syA'-a}, |.zim'aN|
%   \arb[voc]{.zim'aN} \arb[trans]{.zim'aN}, |radI'aN|
%   \arb[voc]{radI'aN} \arb[trans]{radI'aN}.
%
%   \label{ref:taxfif-hamzah}
%   \textbf{\arb[trans]{ta_hfIf-u 'l-hamzaT-i}}: if the
%   \arb[trans]{hamzaT} has \arb[trans]{jazmaT} and is preceded by
%   \emph{\arb[trans]{'alif} hamzatum}, it must be changed into the
%   letter of prolongation that is homogeneous with the preceding
%   vowel; hence: |'a'mana| \arb[voc]{'a'mana} \arb[trans]{'a'mana},
%   |'u'minu| \arb[voc]{'u'minu} \arb[trans]{'u'minu}, |'i'mAnuN|
%   \arb[voc]{'i'mAnuN} \arb[trans]{'i'mAnuN}. For other possible ways
%   of encoding such sequences, see \vpageref{ref:initial-hamza}
%   (\arb[trans]{hamzaT} followed by \arb[novoc]{U} and \arb[novoc]{I})
%   and the \arb[trans]{maddaT} \vpageref{ref:madda}.
%
%   Imperatives of verbs that have the \arb[trans]{hamzaT} as the
%   first radical are other cases of \arb[trans]{ta_hfIf-u
%   'l-hamzaT-i}: |i'sir| \arb[fullvoc]{i'sir} \arb[trans]{i'sir},
%   |i'_dan| \arb[fullvoc]{i'_dan} \arb[trans]{i'_dan}, |u'mul|
%   \arb[fullvoc]{u'mul} \arb[trans]{u'mul}.  \package{arabluatex}
%   also provides ways of encoding those words when the initial
%   \arb[trans]{'alif} comes into \arb[trans]{wa.sl}, so as to make
%   the \arb[trans]{'alif wa.sl} fall away when preceded by
%   \arb[voc]{wa} or \arb[voc]{fa}: |wa-'sir| \arb[fullvoc]{wa-'sir}
%   \arb[trans]{wa-'sir}, |fa-'_dan| \arb[fullvoc]{fa-'_dan}
%   \arb[trans]{fa-'_dan}, |fa-'ti| \arb[fullvoc]{fa-'ti}
%   \arb[trans]{fa-'ti}, |wa-'tamirUA| \arb[fullvoc]{wa-'tamirUA}
%   \arb[trans]{wa-'tamirUA}; or be retained outside the imperative,
%   as in |fa-i'tazarat| \arb[fullvoc]{fa-i'tazarat}
%   \arb[trans]{fa-i'tazarat}, |ba`da| |i'tilAfiN| \arb[fullvoc]{ba`da
%   i'tilAfiN} \arb[trans]{ba`da i'tilAfiN}.
%
%   \textbf{The strange spelling of \arb[trans]{mi'aTuN}}: |mi'aTuN|
%   \arb[voc]{mi'aTuN} \arb[trans]{mi'aTuN}, \linebreak |mi'atAni|
%   \arb[voc]{mi'atAni} \arb[trans]{mi'atAni}, |mi'atayni|
%   \arb[voc]{mi'atayni} \arb[trans]{mi'atayni}, |mi'Una|
%   \arb[voc]{mi'Una} \arb[trans]{mi'Una}, |mi'AtuN|
%   \arb[voc]{mi'AtuN} \arb[trans]{mi'AtuN}, |mi'aN_A|
%   \arb[voc]{mi'aN_A} \arb[trans]{mi'aN_A}. Of course, the
%   \enquote*{pipe} character can be used to prevent this rule from
%   being applied (see \vref{sec:pipe}): \verb+mi'a|TuN+
%   \arb[voc]{mi'a|TuN} \arb[trans]{mi'a|TuN}.
% \end{quote}
%
% \paragraph{\texorpdfstring{\arb[trans]{maddaT}}{maddah}}
% \label{ref:madda}At the beginning of a syllabe, \arb[trans]{'alif}
% with \arb[trans]{hamzaT} and \arb[trans]{fat.haT} (\arb[voc]{'a})
% followed by \arb[trans]{'alifu 'l-maddi} (\arb[trans]{'alif} of
% prolongation) or \arb[trans]{'alif} with \arb[trans]{hamzaT} and
% \arb[trans]{jazmaT} (\arb[voc]{a"'"}) are both represented in
% writing \arb[trans]{'alif} with \arb[trans]{maddaT}: \arb[voc]{A"'}
% \parencite[see][i. 25 A--B]{Wright}.
%
% \iffalse
%<*example>
% \fi
\begin{tcblisting}{text only}
  Hence one should keep to this distinction and encode |'a'kulu|
  \arb[voc]{'a'kulu} \arb[trans]{'a'kulu} and |'AkiluN|
  \arb[voc]{'AkiluN} \arb[trans]{'AkiluN} respectively.
\end{tcblisting}
% \iffalse
%</example>
% \fi
%
% \package{arabluatex} otherwise determines \arb[trans]{al-'alif-u
% 'l-mamdUdaT-u} by context analysis.
% 
% \begin{quote}
%   |'is'AduN| \arb[voc]{'is'AduN} \arb[trans]{'is'AduN}, |'AkilUna|
%   \arb[voc]{'AkilUna} \arb[trans]{'AkilUna}, |'a'mannA|
%   \arb[voc]{'a'mannA} \arb[trans]{'a'mannA}, |al-qur'An-u|
%   \arb[voc]{al-qur'An-u} \arb[trans]{al-qur'An-u}.
%
%   |jA'a| \arb[voc]{jA'a} \arb[trans]{jA'a}, |yatasA'alUna|
%   \arb[voc]{yatasA'alUna} \arb[trans]{yatasA'alUna}, |ridA'uN|
%   \arb[voc]{ridA'uN} \arb[trans]{ridA'uN}, |xaba'A|
%   \arb[voc]{xaba'A} \arb[trans]{xaba'A}, |yaxba'Ani|
%   \arb[voc]{yaxba'Ani} \arb[trans]{yaxba'Ani}.
% \end{quote}
%
% \paragraph{\texorpdfstring{\arb[trans]{^saddaT}}{šaddah}}
% \arb[trans]{ta^sdId} is either \emph{necessary} or \emph{euphonic}.
%
% \subparagraph{The necessary \arb[trans]{ta^sdId}}
% \label{ref:necessary-tashdid}always follows a vowel, whether short
% or long \parencite[see][i. 15 A--B]{Wright}. It is encoded in
% writing the consonant that carries it twice:
% \begin{quote}
%   |`allaqa| \arb[voc]{`allaqa} \arb[trans]{`allaqa}, |mAdduN|
%   \arb[voc]{mAdduN} \arb[trans]{mAdduN}, |'ammara|
%   \arb[voc]{'ammara} \arb[trans]{ammara}, |murruN| \arb[voc]{murruN}
%   \arb[trans]{murruN}.
% \end{quote}
% 
% \subparagraph{The euphonic \arb[trans]{ta^sdId}}
% \label{ref:euphonic-tashdid} always follows a vowelless consonant
% which is passed over in pronunciation and assimilated to a following
% consonant. It may be found \parencite[i. 15 B--16 C]{Wright}:---
% \begin{enumerate}
% \item With the \emph{solar} letters \arb[novoc]{t}, \arb[novoc]{_t},
%   \arb[novoc]{d}, \arb[novoc]{_d}, \arb[novoc]{r}, \arb[novoc]{z},
%   \arb[novoc]{s}, \arb[novoc]{^s}, \arb[novoc]{.s}, \arb[novoc]{.d},
%   \arb[novoc]{.t}, \arb[novoc]{.z}, \arb[novoc]{l}, \arb[novoc]{n},
%   after the article \arb[fullvoc]{al-}:---
% \iffalse
%<*example>
% \fi
\begin{tcblisting}{text only}
  Unlike \package{arabtex} and \package{arabxetex},
  \package{arabluatex} \emph{never requires the solar letter to be
    written twice}, as it automatically generates the euphonic
  \arb[trans]{ta^sdId} above the letter that carries it, whether the
  article be written in the assimilated form or not, e.g. |al-^sams-u|
  \arb[voc]{al-^sams-u} \arb[trans]{al-^sams-u}, or |a^s-^sams-u|
  \arb[voc]{a^s-^sams-u} \arb[trans]{a^s-^sams-u}.
\end{tcblisting}
% \iffalse
%</example>
% \fi
% \begin{quote}
%   |al-tamr-u| \arb[voc]{al-tamr-u} \arb[trans]{al-tamr-u},
%   |al-ra.hm_an-u| \arb[voc]{al-ra.hm_an-u}
%   \arb[trans]{al-ra.hm_an-u}, |al-.zulm-u| \arb[voc]{al-.zulm-u}
%   \arb[trans]{al-.zulm-u}, |al-lu.gaT-u| \arb[voc]{al-lu.gaT-u}
%   \arb[trans]{al-lu.gaT-u}.
% \end{quote}
% \item \label{ref:assimilation} With the letters \arb[novoc]{r},
%   \arb[novoc]{l}, \arb[novoc]{m}, \arb[novoc]{w}, \arb[novoc]{y}
%   after \arb[voc]{n} with \arb[trans]{jazmaT}, and also after the
%   \arb[trans]{tanwIn}:---
% \begin{quote}\SetArbDflt*
%   Note the absence of \arb[trans]{sukUn} above the passed over
%   \arb[novoc]{n} in the following examples, each of which is
%   accompanied by a consistent transliteration: |min rabbi-hi|
%   \arb[fullvoc]{min rabbi-hi}, \arb[trans]{min rabbi-hi},
%   |min| |layliN| \arb[fullvoc]{min layliN} \arb[trans]{min layliN},
%   |'an| |yaqtula| \arb[fullvoc]{'an yaqtula} \arb[trans]{'an yaqtula}.
%   
%   With \arb[trans]{tanwIn}: |kitAbuN| |mubInuN| \arb[voc]{kitAbuN
%   mubInuN} \arb[trans]{kitAbuN mubInuN}.\SetArbDflt
% \end{quote}
% \iffalse
%<*example>
% \fi
\begin{tcblisting}{text only}
  \textsc{Rem.} This particular feature must be put into operation by
  the \cs{SetArbDflt*} command explicitly. See above
  \vref{sec:classic-modern-typesetting} for further details. Other
  kinds of assimilations, including the various cases of
  \arb[trans]{'id.gAm}, will be included in \package{arabluatex}
  gradually.
\end{tcblisting}
% \iffalse
%</example>
% \fi
% \item With the letter \arb[voc]{t} after the dentals
%   \arb[novoc]{_t}, \arb[novoc]{d}, \arb[novoc]{_d}, \arb[novoc]{.d},
%   \arb[novoc]{.t}, \arb[novoc]{.z} in certain parts of the verb:
%   this kind of assimilation, e.g. \arb[voc]{labi_tttu} for
%   \arb[voc]{labi_ttu} \arb[trans]{labi_ttu}, will be discarded here,
%   as it is largely condemned by the
%   grammarians \parencite[see][i. 16 B--C]{Wright}.
% \end{enumerate}
%
% \paragraph{\texorpdfstring{The definite article and the
% \arb[trans]{'alif-u 'l-wa.sl-i}}{The definite article and the ʾalifu
% 'l-waṣli}}
% \label{ref:definite-article}
% At the beginning of a sentence, \arb[fullvoc]{"a} is never written,
% as \arb[fullvoc]{'l-.hamd-u li-ll_ah-i}; instead, to indicate that
% the \arb[trans]{'alif} is a connective \arb[trans]{'alif}
% (\arb[trans]{'alif-u 'l-wa.sl-i}), the \arb[trans]{hamzaT} is
% omitted and only its accompanying vowel is expressed:
% \begin{quote}
%   |al-.hamd-u| |li-l-l_ah-i| \arb[fullvoc]{al-.hamd-u li-l-l_ah-i}
%   \arb[trans]{al-.hamd-u li-l-l_ah-i}.
% \end{quote}
% As said above on \cpageref{fullvoc-mode}, |fullvoc| is the mode
% in which \package{arabluatex} expresses the \arb[trans]{sukUn} and
% the \arb[trans]{wa.slaT}. \package{arabluatex} will take care of
% doing that automatically provided that the vowel which is to be
% absorbed by the final vowel of the preceding word be properly
% encoded, like so:---
% \begin{enumerate}
% \item Definite article at the beginning of a sentence is encoded\\
%   \tcboxverb{al-}, or \tcboxverb{a<solar letter>-}\\ if one
%   wishes to mark the assimilation---which is in no way required, as
%   \package{arabulatex} will detect all cases of assimilation.
% \item Definite article inside sentences is encoded\\ \tcboxverb{'l-}
%   or \tcboxverb{'<solar letter>-}.
% \item In all remaining cases of elision, the \arb[trans]{'alifu
% 'l-wa.sli} is expressed by the vowel that accompanies the omitted
% \arb[trans]{hamzaT}: \meta{u, a, i}.
% \end{enumerate}
% \begin{quote}
%   \textbf{Article}: |bAb-u| |'l-madrasaT-i| \arb[fullvoc]{bAb-u
%   'l-madrasaT-i} \arb[trans]{bAb-u 'l-madrasaT-i},
%   |al-maqA|\allowbreak|laT-u| |'l-'_Ul_A| \arb[fullvoc]{al-maqAlaT-u
%   'l-'_Ul_A} \arb[trans]{al-maqAlaT-u 'l-'_Ul_A}, |al-lu.gaT-u|
%   |'l-`ara|\allowbreak|biyyaT-u| \arb[fullvoc]{al-lu.gaT-u
%   'l-`arabiyyaT-u} \arb[trans]{al-lu.gaT-u 'l-`arabiyyaT-u}, |fI|
%   |.sinA`aT-i| |'l-.tibb-i| \arb[fullvoc]{fI .sinA`aT-i 'l-.tibb-i}
%   \arb[trans]{fI .sinA`aT-i 'l-.tibb-i}, |'il_A| |'l-intiqA.d-i|
%   \arb[fullvoc]{'il_A 'l-intiqA.d-i} \arb[trans]{'il_A
%   'l-intiqA.d-i}, |fI| |'l-ibtidA'-i| \arb[fullvoc]{fI 'l-ibtidA'-i}
%   \arb[trans]{fI 'l-ibtidA'-i}, |'abU| |'l-wazIr-i|
%   \arb[fullvoc]{'abU 'l-wazIr-i} \arb[trans]{'abU 'l-wazIr-i},
%   |fa-lammA| |ra'aW| |'l-najm-a| \arb[fullvoc]{fa-lammA ra'aW
%   'l-najm-a} \arb[trans]{fa-lammA ra'aW 'l-najm-a}.
%
%   \textbf{Particles}:---
%   \begin{enumerate}
%   \item \arb[trans]{li-}: \arb[trans]{'alif-u 'l-wa.sl-i} is omitted
%     in the article \arb[fullvoc]{al} when it is preceded by the
%     preposition \arb[fullvoc]{li}: |li-l-rajul-i|
%     \arb[fullvoc]{li-l-rajul-i}
%     \arb[trans]{li-l-rajul-i}.\\
%     If the first letter of the noun be \arb[novoc]{l}, then the
%     \arb[novoc]{l} of the article also falls away, but
%     \package{arabluatex} is aware of that: |li-l-laylaT-i|
%     \arb[fullvoc]{li-l-laylaT-i} \arb[trans]{li-l-laylaT-i}.
%   \item \arb[trans]{la-}: the same applies to the affirmative
%   particle \arb[fullvoc]{la}: |la-l-.haqq-u|
%   \arb[fullvoc]{la-l-.haqq-u} \arb[trans]{la-l-.haqq-u}.
% \item With the other particles, \arb[trans]{'alif-u 'l-wa.sl-i} is
%   expressed: |fI| |'l-madIna|\allowbreak|T-i| \arb[fullvoc]{fI
%   'l-madInaT-i} \arb[trans]{fI 'l-madInaT-i}, |wa-'l-rajul-u|
%   \arb[fullvoc]{wa-'l-rajul-u} \arb[trans]{wa-'l-rajul-u},
%   |bi-'l-|\allowbreak|qalam-i| \arb[fullvoc]{bi-'l-qalam-i}
%   \arb[trans]{bi-'l-qalam-i}, |bi-'l-ru`b-i|
%   \arb[fullvoc]{bi-'l-ru`b-i} \arb[trans]{bi-'l-ru`b-i}.
%   \end{enumerate}
%   
%   \textbf{Perfect active, imperative, nomen actionis}: |qAla|
%   |isma`| \arb[fullvoc]{qAla isma`} \arb[trans]{qAla isma`}, |qAla|
%   |uqtul| \arb[fullvoc]{qAla uqtul} \arb[trans]{qAla uqtul}, |huwa|
%   |inhazama| \arb[fullvoc]{huwa inhazama} \arb[trans]{huwa
%   inhazama}, |wa-ustu`mila| \arb[fullvoc]{wa-ustu`mila}
%   \arb[trans]{wa-ustu`mila}, |qadi| |in.sarafa| \arb[fullvoc]{qadi
%   in.sarafa} \arb[trans]{qadi in.sarafa}, |al-iqtidAr-u|
%   \arb[fullvoc]{al-iqtidAr-u} \arb[trans]{al-iqtidAr-u}, |'il_A|
%   |'l-inti|\allowbreak|qA.d-i| \arb[fullvoc]{'il_A 'l-intiqA.d-i}
%   \arb[trans]{'il_A 'l-intiqA.d-i}, |law| |istaqbala|
%   \arb[fullvoc]{law istaqbala} \arb[trans]{law istaqbala}.
%
%   \textbf{Other cases}: |'awi| |ismu-hu| \arb[fullvoc]{'awi ismu-hu}
%   \arb[trans]{'awi ismu-hu}, |zayduN| |ibn-u| |`amriNU|
%   \arb[fullvoc]{\uc{z}ayduN ibn-u \uc{`amriNU}}
%   \arb[trans]{\uc{z}ayduN ibn-u
%   \uc{`amriNU}},\footnote{\label{fn:zayd-is-son}%
%   \enquote{\arb[trans]{\uc{z}ayd} is the son of
%   \arb[trans]{\uc{`a}mr}}: the second noun is not in apposition to
%   the first, but forms part of the predicate. Hence \arb[voc]{zayduN
%   ibn-u `amriNU} and not \arb[voc]{zayd-u bn-u `amriNU},
%   \enquote{Zayd, son of ʿAmr}.}  |`umar-u| |ibn-u| |'l-_ha.t.tAb-i|
%   \arb[fullvoc]{\uc{`umar}-u ibn-u \uc{'l-_ha.t.tAb-i}}
%   \arb[trans]{\uc{`umar}-u ibn-u
%   \uc{'l-_ha.t.tAb-i}},\footnote{\enquote{\arb[trans]{\uc{`umar}}
%   is the son of \arb[trans]{\uc{al-_ha.t.tAb}}} (see
%   \vref{fn:zayd-is-son}).}  |imru'-u| |'l-qays-i|
%   \arb[fullvoc]{imru'-u 'l-qays-i} \arb[trans]{\uc{i}mru'-u
%   \uc{'l-qays-i}}, |la-aymun-u| |'l-l_ah-i|
%   \arb[fullvoc]{la-aymun-u 'l-l_ah-i} \arb[trans]{la-aymun-u
%   'l-l_ah-i}.
% \end{quote}
%
% \subparagraph{\arb[trans]{'alif-u 'l-wa.sl-i} preceded by a long
% vowel} The long vowel preceding the connective \arb[trans]{'alif} is
% shortened in pronunciation \parencite[i. 21 B--D]{Wright}. This does
% not appear in the Arabic script, but \package{arabluatex} takes it
% into account in some transliteration standards:---
% \begin{quote}
%   |fI| |'l-nAs-i| \arb[fullvoc]{fI 'l-nAs-i} \arb[trans]{fI
%   'l-nAs-i}, |'abU| |'l-wazIr-i| \arb[fullvoc]{'abU 'l-wazIr-i}
%   \arb[trans]{'abU 'l-wazIr-i}, |fI| |'l-ibtidA'-i| \arb[fullvoc]{fI
%   'l-ibtidA'-i} \arb[trans]{fI 'l-ibtidA'-i}, |_dU 'l-i`lAl-i|
%   \arb[fullvoc]{_dU 'l-i`lAl-i} \arb[trans]{_dU 'l-i`lAl-i},
%   |maqh_A| |'l-'amIr-i| \arb[voc]{maqh_A 'l-'amIr-i}
%   \arb[trans]{maqh_A 'l-'amIr-i}.
% \end{quote}
%
% \subparagraph{\arb[trans]{'alif-u 'l-wa.sl-i} preceded by a diphthong}
% \label{sec:diphthong-alif}
% The diphthong is resolved into two simple vowels \parencite[i. 21
% D--22 A]{Wright} viz. \emph{ay}~→ \emph{\u{a}\u{i}} and \emph{aw}~→
% \emph{\u{a}\u{u}}. \package{arabluatex} detects the cases in which
% this rule applies:---
% \begin{quote}
%   |fI| |`aynay| |'l-malik-i| \arb[fullvoc]{fI `aynay 'l-malik-i}
%   \arb[trans]{fI `aynay 'l-malik-i}, |ix^say|
%   |'l-qaw|\allowbreak|m-a| \arb[fullvoc]{ix^say 'l-qawm-a}
%   \arb[trans]{ix^say 'l-qawm-a}, |mu.s.tafaw| |'l-l_ah-i|
%   \arb[fullvoc]{mu.s.tafaw 'l-l_ah-i} \arb[trans]{mu.s.ta\-faw
%   'l-l_ah-i}.
%
%   |ramaW| |'l-.hijAraT-a| \arb[fullvoc]{ramaW 'l-.hijAraT-a}
%   \arb[trans]{ramaW 'l-.hijAraT-a}, |fa-lammA| |ra'aW |\allowbreak{}
%   |'l-najm-a| \arb[fullvoc]{fa-lammA ra'aW 'l-najm-a}
%   \arb[trans]{fa-lammA ra'aW 'l-najm-a}.
% \end{quote}
%
% \subparagraph{\arb[trans]{'alif-u 'l-wa.sl-i} preceded by a consonant
% with \arb[trans]{sukUn}} The vowel which the consonant takes is
% either its original vowel, or that which belongs to the connective
% \arb[trans]{'alif} or the \arb[trans]{kasraT}; in most of the
% cases \parencite[i. 22 A--C]{Wright}, it is encoded explicitly, like
% so:---
% \begin{quote}
%   |'antumu| |'l-kA_dib-Una| \arb[fullvoc]{'antumu 'l-kA_dib-Una}
%   \arb[trans]{'antumu 'l-kA_dib-Una}, |ra'aytumu| |'l-rajul-a|
%   \arb[fullvoc]{ra'aytumu 'l-rajul-a} \arb[trans]{ra'aytumu
%   'l-rajul-a}, |mani| |'l-ka_d_dAb-u| \arb[fullvoc]{mani
%   'l-ka_d_dAb-u} \arb[trans]{mani 'l-ka_d_dAb-u}, |qatalati|
%   |'l-rUm-u| \arb[fullvoc]{qatalati 'l-rUm-u} \arb[trans]{qatalati
%   \uc{'l-rUm-u}}.
% \end{quote}
% \label{ref:muhammaduni}
% However, the Arabic script does not show the \arb[trans]{kasraT} or
% the \arb[trans]{.dammaT} which may be taken by the nouns having
% \arb[trans]{tanwIn} although it is explicit in pronunciation and
% must appear in some transliteration standards. \package{arabluatex}
% takes care of that automatically:---
% \begin{quote}
%   |mu.hammaduN| |'l-nabI| \arb[fullvoc]{mu.hammaduN 'l-nabI}
%   \arb[trans]{\uc{m}u.hammaduN 'l-nabI}, |salAmuN| |ud_hulUA|
%   \arb[fullvoc]{salAmuN ud_hulUA} \arb[trans]{salAmuN ud_hulUA},
%   |qa.sIdata-hu| |fI| |qatl-i| |\uc{'a}bI|
%   |\uc{m}|\allowbreak|uslimiN| |'llatI| |yaqUlu| |fI-hA|
%   \arb[fullvoc]{qa.sIdata-hu fI qatl-i \uc{'a}bI \uc{m}uslimiN
%   'llatI yaqUlu fI-hA} \arb[trans]{qa.sIdata-hu fI qatl-i \uc{'a}bI
%   \uc{m}uslimiN 'llatI yaqUlu fI-hA}.
% \end{quote}
%
% \subsection{Special orthographies}
% \paragraph{The name of God}
% The name of God, \arb[voc]{al-l_ahu}, is compounded of the article
% \arb[fullvoc]{al-}, and \arb[fullvoc]{'ilAh-u} (noted
% \arb[fullvoc]{'il_ah-u} with the defective \arb[trans]{'alif}) so
% that it becomes \arb[fullvoc]{al-'ilAh-u}; then the
% \arb[trans]{hamzaT} is suppressed, its vowel being transferred to
% the \arb[novoc]{l} before it, so that there remains
% \arb[voc]{alil_ah-u} \parencite[I refer to][I. 83
% col. 1]{Lane}. Finally, the first \arb[novoc]{l} is made quiescent
% and incorporated into the other, hence the \arb[trans]{ta^sdId}
% above it. As \package{arabluatex} never requires a solar letter to
% be written twice (see above, \vpageref{ref:euphonic-tashdid}), the
% name of God is therefore encoded |al-l_ah-u| or |'l-l_ah-u|:---
% \begin{quote}
%   |al-l_ah-u| \arb[fullvoc]{al-l_ah-u} \arb[trans]{al-l_ah-u},
%   \verb+yA|+\footnote{\label{fn:pipe-allah-01}Note the
%   \enquote{pipe} character \enquote*{\textbar} here after |yA| and
%   below after |fa| before footnote mark \ref{fn:pipe-allah-02}: it
%   is needed by the |dmg| transliteration mode as in this mode any
%   vowel at the commencement of a word preceded by a word that ends
%   with a vowel, either short or long, is absorbed by this vowel
%   viz. \arb[trans]{`al_A 'l-.tarIq-i}. See \vref{sec:pipe} on the
%   \enquote{pipe} and \vref{sec:transliteration} on |dmg| mode.}
%   |al-l_ah-u| \arb[fullvoc]{yA| al-l_ah-u} \arb[trans]{yA|
%   al-l_ah-u}, \verb+'a-fa|+\footnote{\label{fn:pipe-allah-02}See
%   \cref{fn:pipe-allah-01}.}|-al-|\allowbreak|l_ah-i|
%   |la-ta.g`alanna| \arb[fullvoc]{'a-fa|-al-l_ah-i la-ta.g`alanna}
%   \arb[trans]{'a-fa|-al-l_ah-i la-ta.g`alanna},
%   |bi-'l-|\allowbreak|l_ah-i| \arb[fullvoc]{bi-'l-l_ah-i}
%   \arb[trans]{bi-'l-l_ah-i}, |wa-'l-l_ah-i|
%   \arb[fullvoc]{wa-'l-l_ah-i} \arb[trans]{wa-'l-l_ah-i}, |bi-sm-i|
%   |'l-l_ah-i| \arb[fullvoc]{bi-sm-i 'l-l_ah-i} \arb[trans]{bi-sm-i
%   'l-l_ah-i}, |al-.hamd-u| |li-l-l_ah-i| \arb[fullvoc]{al-.hamd-u
%   li-l-l_ah-i} \arb[trans]{al-.hamd-u li-l-l_ah-i}, |li-l-l_ah-i|
%   |'l-qA'il-u| \arb[fullvoc]{li-l-l_ah-i 'l-qA'il-u}
%   \arb[trans]{li-l-l_ah-i 'l-qA'il-u}.
% \end{quote}
%
% \paragraph{\texorpdfstring{The conjunctive \arb[voc]{alla_dI}}{The
% conjunctive اَلَّذِي}}
% Although it is compounded of the article \arb[fullvoc]{al}, the
% demonstrative letter \arb[novoc]{l} and the demonstrative pronoun
% \arb[voc]{_dA}, both masculine and feminine forms that are written
% defectively are encoded |alla_dI| and |allatI| respectively. Forms
% starting with the connective \arb[trans]{'alif} are encoded
% |'lla_dI| and |'llatI|:---
% \begin{quote}
%   |'a_hAfu| |mina| |'l-malik-i| |'lla_dI| |ya.zlimu| |'l-nAs-a|
%   \arb[fullvoc]{'a_hAfu mina 'l-malik-i 'lla_dI ya.zlimu 'l-nAs-a}
%   \arb[trans]{'a_hAfu mina 'l-malik-i 'lla_dI ya.zlimu 'l-nAs-a},
%   |`udtu| |'l-^say_h-a| |'lla_dI| |huwa| |marI.duN|
%   \arb[fullvoc]{`udtu 'l-^say_h-a 'lla_dI huwa marI.duN}
%   \arb[trans]{`udtu 'l-^say_h-a 'lla_dI huwa marI.duN}, |mA| |'anA|
%   |bi-'lla_dI| |qA'iluN| |la-ka| |^say'aN| \arb[fullvoc]{mA 'anA
%   bi-'lla_dI qA'iluN la-ka ^say'aN} \arb[trans]{mA 'anA bi-'lla_dI
%   qA'iluN la-ka ^say'aN}.
%
%   |'ari-nA| |'lla_dayni| |'a.dallA-nA| |mina| |'l-jinn-i|
%   |wa-'l-'ins-i| \arb[fullvoc]{'ari-nA 'lla_dayni 'a.dallA-nA mina
%   'l-jinn-i wa-'l-'ins-i} \arb[trans]{'ari-nA 'lla_dayni 'a.dallA-nA
%   mina 'l-jinn-i wa-'l-'ins-i}.
% \end{quote}
% The other forms are encoded regularly as |al-l| or |'l-l|:---
% \begin{quote}
%   |fa-'innA| |na_dkuru| |'l-.sawt-ayni| |'l-la_dayni| |rawaynA-humA|
%   |`an| |ja.h.zaT-a| \arb[fullvoc]{fa-'innA na_dkuru 'l-.sawt-ayni
%   'l-la_dayni rawaynA-humA `an \uc{ja.h.zaT-a}}
%   \arb[trans]{fa-'innA na_dkuru 'l-.sawt-ayni 'l-la_dayni
%   rawaynA-humA `an \uc{ja.h.zaT-a}}.
%
%   And also: |al-la_dAni| \arb[fullvoc]{al-la_dAni}
%   \arb[trans]{al-la_dAni}, |al-la_dayni| \arb[fullvoc]{al-la_dayni}
%   \arb[trans]{al-la_dayni}, |al-latAni| \arb[fullvoc]{al-latAni}
%   \arb[trans]{al-latAni}, |al-latayni| \arb[fullvoc]{al-latayni}
%   \arb[trans]{al-latayni}, |al-lAtI| \arb[fullvoc]{al-lAtI}
%   \arb[trans]{al-lAtI},
%   \verb+al-lA'|Ati+\footnote{\label{fn:pipe-madda}Note here the
%   \enquote{pipe} character \enquote*{\textbar}: as already stated
%   \vpageref{ref:madda}, the sequence |'A| usually encodes
%   \arb[trans]{'alif} with \arb[trans]{hamzaT} followed by
%   \arb[trans]{'alif} of prolongation, which is represented in writing
%   \arb[trans]{'alif} with \arb[trans]{maddaT}: \arb[voc]{A"'}. The
%   \enquote{pipe} character prevents this rule from being
%   applied. See \vref{sec:pipe}.}  \arb[fullvoc]{al-lA'|Ati}
%   \arb[trans]{al-lA'|Ati}, |al-lA'I| \arb[fullvoc]{al-lA'I}
%   \arb[trans]{al-lA'I}, and so forth.
% \end{quote}
%
% \subsection{Quoting}
% \label{sec:quoting}
% It is here referred to \enquote{quoting} after the \package{arabtex}
% package.\footnote{See \textcite[22]{pkg:arabtex}} The
% \enquote{quoting} mechanism of \package{arabluatex} is designed to
% be very similar in effect to the one of \package{arabtex}.
%
% To start with an example, suppose one types the following in |novoc|
% mode: \arb[novoc]{`ullima `ilm-a 'l-hay'aT-i}; is it
% \arb[fullvoc]{`ullima}, \emph{he was taught the science of
% astronomy}, or \arb[fullvoc]{`allama}, \emph{he taught the science
% of astronomy}? In order to disambiguate this clause, it may be
% sensible to put a \arb[trans]{.dammaT} above the first \arb[voc]{`}:
% \arb[novoc]{`"ullima `ilm-a 'l-hay'aT-i}, which is achieved by
% \enquote{quoting} the vowel |u|, like so: |`"ullima|, or, with no
% other vowel than the required |u|: |`"ullm|.
%
% This is how the \enquote{quoting} mechanism works: metaphorically
% speaking, it acts as a \emph{toggle switch}. If something, in a
% given mode, is supposed to be visible, \enquote{quoting} hides it;
% conversely, if it is supposed not to, it makes it visible.
%
% As shown above, \enquote{quoting} means inserting one straight
% double quote (|"|) \emph{before} the letter that is to be acted
% upon. Its effects depend on the mode which is currently selected,
% either |novoc|, |voc| or |fullvoc|:---
%
% \paragraph{\texttt{novoc}} In this mode, \enquote{quoting}
% essentially means make visible something that ought not to be so.
% \begin{enumerate}
% \item Quoting a vowel, either short or long, makes the
%   \arb[trans]{.dammaT}, \arb[trans]{fat.haT} or \arb[trans]{kasraT}
%   appear above the appropriate consonant:---
% \begin{quote}
%   |`"ullima| |`ilm-a| |'l-hay'aT-i| \arb[novoc]{`"ullima `ilm-a
%   'l-hay'aT-i} \arb[trans]{`"ullima `ilm-a 'l-hay'aT-i}, |ya.gz"UA|
%   \arb[novoc]{ya.gz"UA} \arb[trans]{ya.gz"UA}.
% \end{quote}
% \item The same applies when \enquote{quoting} the
%   \arb[trans]{tanwIn}:---
% \begin{quote}
%   |wa-'innA| |sawfa| |tudriku-nA| |'l-manAyA| |muqadd"araT"aN|
%   \arb[novoc]{wa-'innA sawfa tudriku-nA 'l-manAyA muqadd"araT"aN},
%   \arb[trans]{wa-'innA sawfa tudriku-nA 'l-manAyA muqadd"araT"aN}.
% \end{quote}
% \item \label{ref:quoted-sukun-b}If no vowel follows the straight
%   double quote, then a \arb[trans]{sukUn} is put above the preceding
%   consonant:---
% \begin{quote}
%   |qAla isma`"| \arb[novoc]{qAla isma`"} \arb[trans]{qAla isma`"},
%   |jA'at"| |hinduN| \arb[voc]{jA'at" \uc{hinduN}}
%   \arb[trans]{jA'at" \uc{hinduN}}, |^sabIhuN| |bi-man| |q"u.ti`at"|
%   |qadamA-hu| \arb[novoc]{^sabIhuN bi-man q"u.ti`at" qadamA-hu}
%   \arb[trans]{^sabIhuN bi-man q"u.ti`at" qadamA-hu}.
% \end{quote}
% \item At the commencement of a word, the straight double quote is
%   interpreted as \arb[trans]{'alif-u 'l-wa.sl-i}:---
% \begin{quote}
%   |wa-"ust"u`mila| \arb[novoc]{wa-"ust"u`mila}
%   \arb[trans]{wa-"ust"u`mila}, |huwa| |"inhazama| \arb[novoc]{huwa
%   "inhazama} \arb[trans]{huwa "inhazama}, |al-"intiqA.d-u|
%   \arb[novoc]{al-"intiqA.d-u} \arb[trans]{al-"intiqA.d-u}.
% \end{quote}
% \end{enumerate}
%
% \paragraph{\texttt{voc}}
% In accordance with the general rule, in this mode, \enquote{quoting}
% makes the vowels and the \arb[trans]{tanwIn} disappear, should this
% feature be required for some reason:---
% \begin{enumerate}
% \item Short and long vowels:---
%   \begin{quote}
%     |q"Ala q"A'iluN| \arb[voc]{q"Ala q"A'iluN} \arb[trans]{q"Ala
%     q"A'iluN}, |ibn-u 'abI 'u.saybi`aT-"a| \arb[voc]{ibn-u 'abI
%     'u.saybi`aT-"a} \arb[trans]{\uc{ibn-u} \uc{'abI}
%     \uc{'u.saybi`aT-"a}}.
%   \end{quote}
% \item \arb[trans]{tanwIn}:---
%   \begin{quote}
%     |madInaT"aN| \arb[voc]{madInaT"aN} \arb[trans]{madInaT"aN},
%     |bAb"aN| \arb[voc]{bAb"aN} \arb[trans]{bAb"aN}, |hud"aN_A|
%     \arb[voc]{hud"aN_A} \arb[trans]{hud"aN_A}, |^say'"iN|
%     \arb[voc]{^say'"iN} \arb[trans]{^say'"iN}.
%   \end{quote}
% \end{enumerate}
% One may more usefully \enquote{quote} the initial vowels to write
% the \arb[trans]{wa.slaT} above the \arb[trans]{'alif} or insert a
% straight double quote after a consonant not followed by a vowel to
% make the \arb[trans]{sukUn} appear:---
% \begin{enumerate}
% \item \arb[trans]{'alif-u 'l-wa.sl-i}:---
%   \begin{quote}
%     |fI "istiq.sA'-iN| \arb[voc]{fI "istiq.sA'-iN} \arb[trans]{fI
%     "istiq.sA'-iN}, |wa-"istiq.sA'-uN| \arb[voc]{wa-"istiq.sA'-uN}
%     \allowbreak\arb[trans]{wa-"istiq.sA'-uN}, |qAla| |"uhrub|
%     |fa-lan| |tuqtala| \arb[voc]{qAla "uhrub fa-lan tuqtala}
%     \arb[trans]{qAla "uhrub fa-lan tuqtala}.
%   \end{quote}
% \item \arb[trans]{sukUn}:---
%   \begin{quote}
%     |qAla| |"uqtul"| |fa-lan| |tuqtala| \arb[voc]{qAla "uqtul"
%     fa-lan tuqtala} \arb[trans]{qAla "uqtul" fa-lan tuqtala}, |mA|
%     |jA'at"| |mini| |imra'aTiN| \arb[voc]{mA jA'at" mini imra'aTiN}
%     \arb[trans]{mA jA'at" mini imra'aTiN}, |kam"| |qad"| |ma.dat"|
%     |min"| |laylaTiN| \arb[voc]{kam" qad" ma.dat" min" laylaTiN}
%     \arb[trans]{kam" qad" ma.dat" min"
%     laylaTiN}.\label{ref:quoted-sukun-e}
%   \end{quote}
% \end{enumerate}
%
% \paragraph{\texttt{fullvoc}}
% In this mode, \enquote{quoting} can be used to take away any short
% vowel (or \arb[trans]{tanwIn}, as seen above) or any
% \arb[trans]{sukUn}:---
% \begin{quote}\label{ref:qrannun-full}
%   |al-jamr-u| |'l-.sayfiyy-u| |'lla_dI| |kAna|
%   \verb+bi-q"rAn"|nUn-a+ \arb[fullvoc]{al-jamr-u 'l-.sayfiyy-u
%   'lla_dI kAna \uc{bi-q"rAn"|nUn-a}} \arb[trans]{al-jamr-u
%   'l-.sayfiyy-u 'lla_dI kAna \uc{bi-q"rAn"|nUn-a}}.
% \end{quote}
%
% \subsubsection{\texorpdfstring{Quoting the
% \arb[trans]{hamzaT}}{Quoting the hamzah}}
% \label{sec:quoting-hamza}
% As said above in \vref{ref:hamza}, the \arb[trans]{hamzaT} is always
% written \meta{\texttt{'}}, its carrier being determined by contextual
% analysis. \enquote{Quoting} that straight single quote character
% like so: \meta{\texttt{"'}} allows to determine the carrier of the
% \arb[trans]{hamzaT} freely, without any consideration for the
% context. \Cref{tab:quoted-hamza} gives the equivalents for all the
% possible carriers the \arb[trans]{hamzaT} may take.
% 
% \begin{longtable}{lllll}
% \bottomrule
% \caption*{\Cref*{tab:quoted-hamza}: \enquote{Quoted}
% \arb[trans]{hamzaT}}
% \endfoot
% \captionlistentry{\enquote{Quoted} \arb[trans]{hamzaT}}\\[-1em]
% \toprule Letter & \multicolumn{3}{l}{Transliteration\footnotemark}
% & Arab\TeX\ notation \\
% & \texttt{dmg} & \texttt{loc} & \texttt{arabica} & \\ \midrule
% \endfirsthead
% \toprule Letter & \multicolumn{3}{l}{Transliteration}
% & Arab\TeX\ notation \\
% & \texttt{dmg} & \texttt{loc} & \texttt{arabica} & \\ \midrule
% \endhead
% \footnotetext{See below \vref{sec:transliteration}.}
% \label{tab:quoted-hamza}
% \arb[novoc]{|"'} & \dmg{|"'} & \loc{|"'} & \brill{|"'} & \verb+|"'+ \\
% \pagebreak[1]
% \arb[novoc]{A"'} & \dmg{A"'} & \loc{A"'} & \brill{A"'} & \verb|A"'| \\
% \arb[novoc]{a"'} & \dmg{a"'} & \loc{a"'} & \brill{a"'} & \verb|a"'| \\
% \arb[novoc]{u"'} & \dmg{u"'} & \loc{u"'} & \brill{u"'} & \verb|u"'| \\
% \arb[novoc]{w"'} & \dmg{w"'} & \loc{w"'} & \brill{w"'} & \verb|w"'| \\
% \arb[novoc]{i"'} & \dmg{i"'} & \loc{i"'} & \brill{i"'} & \verb|i"'| \\
% \arb[novoc]{y"'} & \dmg{y"'} & \loc{y"'} & \brill{y"'} & \verb|y"'| \\
% \end{longtable}
%
% As one can see from \vref{tab:quoted-hamza}, the carrier of the
% \arb[trans]{hamzaT} is inferred from the letter that precedes the
% straight double quote \meta{\texttt{"}}. Of course, any
% \enquote{quoted} \arb[trans]{hamzaT} may take a short vowel, which
% is to be written \emph{after} the Arab\TeX\ equivalent for the
% \arb[trans]{hamzaT} itself, namely \meta{\texttt{'}}. For example,
% \arb[voc]{w"'a} is encoded \meta{\texttt{w"'a}}, while
% \arb[voc]{w"'"} is encoded \meta{\texttt{w"'"}}. In the latter
% example, the second straight double quote encodes the
% \arb[trans]{sukUn} in |voc| mode in accordance with the rule laid
% above \vpagerefrange{ref:quoted-sukun-b}{ref:quoted-sukun-e}.
% \begin{quote}
%   |'a`dA'ukum| \arb[fullvoc]{'a`dA'ukum} \arb[trans]{'a`dA'ukum},
%   \verb+'a`dA|"'ukum+ \arb[fullvoc]{'a`daA"'|"'ukum}
%   \arb[trans]{'a`dA|"'ukum}, |'a`dA'ikum| \arb[fullvoc]{'a`dA'ikum}
%   \arb[trans]{'a`dA'ikum}, \verb+'a`dA|"'ikum+
%   \arb[fullvoc]{'a`daA"'|"'ikum} \arb[trans]{'a`dA|"'ikum}.
% \end{quote}
%
% \subsection{\texorpdfstring{The \enquote*{pipe} character
% (\textbar)}{The ‘pipe’ character (\textbar)}}
% \label{sec:pipe}
% In the terminology of Arab\TeX, the \enquote{pipe} character
% \enquote*{\textbar} is referred to as the \enquote{invisible
% consonant}. Hence, as already seen above in
% \vref{sec:quoting-hamza}, its usage to encode the
% \arb[trans]{hamzaT} alone, with no carrier: \verb+|"'+
% \arb[novoc]{|"'}.
%
% Aside from that usage, the \enquote{pipe} character is used to
% prevent almost any of the contextual analysis rules that are
% described above from being applied. Two examples have already been
% given to demonstrate how that particular mechanism works in
% \vref{fn:pipe-allah-01} and in \vref{fn:pipe-madda}. One more example
% follows:---
% \begin{quote}
%   \verb+bi-qrAn|nUn-a+ \arb[voc]{\uc{bi-qrAn|nUn-a}}
%   \arb[trans]{\uc{bi-qrAn|nUn-a}}, \enquote{in Crannon} (Thessaly,
%   Greece).\footnote{See more context \vpageref{ref:qrannun-full}.}
% \end{quote}
% As one can see, the \enquote{pipe} character between the two
% \meta{n} prevents the necessary \arb[trans]{ta^sdId} rule
% (\cpageref{ref:necessary-tashdid}) from being applied.
%
% \subsection{Putting back on broken contextual analysis rules}
% \label{sec:arbnull}
% \NEWfeature{v1.7} In complex documents such as critical editions
% where footnotes and other kind of annotations can be particularly
% abundant, the contextual analysis rules that are described above may
% be broken by \LaTeX\ commands. To take an example, consider the
% following:---%
% \iffalse
%<*example>
% \fi
\begin{example}
  This is wrong:
  \begin{arab}[fullvoc]
    fa-lammA ra'aW\LRfootnote{A footnote which interferes with
      the contextual analysis.} 'l-na^gma...
  \end{arab}
\end{example}
% \iffalse
%</example>
% \fi%
% According to the rule stated \vpageref{sec:diphthong-alif}, the
% diphthong in \arb[trans]{ra'aW} must be resolved into two simple
% vowels before the \arb[trans]{'alif-u 'l-wa.sl-i}, as
% \arb[fullvoc]{ra'aW 'l-na^gma}.
%
% \DescribeMacro{\arbnull} The \cs{arbnull} command is provided so as
% to put back on contextual analysis rules in such situations. It
% takes as argument the word that must be brought back for any given
% rule to be applied as it ought to. Depending on the contexts that
% have to be restored, \cs{arbnull} may be found just after or before
% Arabic words.%
% \iffalse
%<*example>
% \fi
\begin{tcblisting}{text only}
  In any case, \emph{no space must be left} after or before the Arabic
  word that \cs{arbnull} is applied to.
\end{tcblisting}
% \iffalse
%</example>
% \fi%
% The following shows how the Arabic should have been written in the
% preceding example and gives further illustrations of the same
% technique:---%
% \iffalse
%<*example>
% \fi
\begin{example}
  \begin{arab}[fullvoc]
    fa-lammA ra'aW\arbnull{'l-na^gma}\LRfootnote{A footnote
      which interferes with the contextual analysis.}
    'l-na^gma...

    qAla\LRfootnote{A footnote which interferes with the
      contextual analysis.} \arbnull{qAla}uhrub fa-lan tuqtala.

    \uc{z}ayduN\arbnull{ibnu}\LRfootnote{A footnote which
      interferes with the contextual analysis.}
    \arbnull{zayduN}ibn-u \uc{`a}mriNU.\LRfootnote{See
      \vref{fn:zayd-is-son}.}
  \end{arab}
  
  \begin{arab}[trans]
    \uc{z}ayduN\arbnull{ibnu}\LRfootnote{A footnote which
      interferes with the contextual analysis.}
    \arbnull{zayduN}ibn-u \uc{`a}mriNU.\LRfootnote{See
      \vref{fn:zayd-is-son}.}
  \end{arab}
\end{example}
% \iffalse
%</example>
% \fi%
%
% \subsection{\texorpdfstring{Stretching characters: the
% \arb[trans]{ta.twIl}}{Stretching characters: the taṭwīl}}
% \label{sec:tatwil}
% A double hyphen \meta{-\,-} stretches the ligature in which one
% letter is bound to another. Although it is always better to rely on
% automatic stretching, this technique can be used to a modest extent,
% especially to increase legibility of letters and diacritics which
% stand one above the other:--
% \begin{quote}
%   |.hunayn-u| |bn-u| |'is.h--_aq-a| \arb[voc]{.hunayn-u bn-u
%   'is.h--_aq-a} \arb[trans]{\uc{.hunayn-u} bn-u \uc{'is.h--_aq-a}}
% \end{quote} 
%
% \subsection{Digits}
% \label{sec:digits}
% \subsubsection{Numerical figures}
% \label{sec:numerical-figures}
% The \emph{Indian numbers}, \arb[trans]{al-raqam-u 'l-hindiyy-u}, are
% ten in number, and they are compounded in exactly the same way as
% our numerals:---
% \begin{quote}
%   |1874| \arb[voc]{1874}, |123-456,789| \arb[voc]{123-456,789}, |fI|
%   |sanaT-i| |1024| \arb[voc]{fI sanaT-i 1024}
% \end{quote}
%
% \subsubsection{The \emph{abjad}}
% \label{sec:abjad}
% The numbers may also be expressed with letters from right to left
% arranged in accordance with the order of the Hebrew and Aramaic
% alphabets \parencite[see][i. 28 B--C]{Wright}. The
% \arb[trans]{'abjad} numbers are usually distinguished from the
% surrounding words by a stroke placed over them.
%
% \DescribeMacro{\abjad} \NEWfeature{v.1.1} \arb[trans]{'abjad}
% numbers are inserted with the \cs{abjad}\marg{number} command in any
% of the |voc|, |fullvoc| and |novoc| modes, where \meta{number} may
% be any number between 1 and 1999, like so:---
% \begin{quote}
%   |\abjad{45}| |kitAbu-hu| |fI| |'l-`AdAt-i| \arb[voc]{\abjad{45}
%   kitAbu-hu fI 'l-`AdAt-i} \arb[trans]{\abjad{45} kitAbu-hu fI
%   'l-`AdAt-i}.
% \end{quote}
% \begin{quoting}
%   \textsc{Rem.}~\emph{a.} As can be seen in the above given example,
%   \package{arabluatex} expresses the \arb[trans]{'abjad} numbers in
%   Roman numerals if it finds the \cs{abjad} command in any of the
%   transliteration modes.
%
%   \textsc{Rem.}~\emph{b.} \cs{abjad} may also be found outside
%   Arabic environments. In that case, \package{arabluatex} does not
%   print the stroke as a distinctive mark over the number for it is
%   not surrounded by other Arabic words. In case one nonetheless
%   wishes to print the stroke, he can either use the \cs{aoline*}
%   command that is described below in \vref{sec:underlining} or
%   insert the \arb[trans]{'ab^gad} number in |\arb[novoc]{}|:---
%   \begin{quote}
%     |The| |\arb[trans]{'abjad}| |number| |for| |1874| |is|
%     |\abjad{1874}| The \arb[trans]{'abjad} number for 1874 is
%     \abjad{1874}.
%
%     |The| |\arb[trans]{'abjad}| |number| |for| |1874| |is|
%     |\aoline*{\abjad{1874}}| The \arb[trans]{'abjad} number for 1874
%     is \aoline*{\abjad{1874}}.
%     
%     |The| |\arb[trans]{'abjad}| |number| |for| |1874| |is|
%     |\arb[novoc]{\abjad{1874}}| The \arb[trans]{'abjad} number for
%     1874 is \arb[novoc]{\abjad{1874}}.
%   \end{quote}
% \end{quoting}
%
% \NEWfeature{v1.12}\cs{abjad} may also be used to convert values of
% counters into \arb[trans]{'ab^gad} numbers, like so:--- %
% \changes{v1.12}{2018/06/24}{\cs{abjad} can now process \LaTeX\
% counters}%
% \iffalse
%<*example>
% \fi
\begin{example}
  The \arb[trans]{'ab^gad} number for the current page (\thepage) is
  \abjad{\thepage}.
\end{example}
% \iffalse
%</example>
% \fi%
%
% This technique can be used to produce abjad-numbered lists as will
% be demonstrated \vpageref{ref:abjad-list}.
%
% \subsection{Additional characters}
% \label{sec:additional-characters}
% In the manuscripts, the unpointed letters, \arb[trans]{al-.hurUf-u
% 'l-muhmalaT-u}, are sometimes further distinguished from the pointed
% by various contrivances, as explained in \textcite[i. 4
% B--C]{Wright}. One may find these letters written in a smaller size
% below the line, or with a dot or another mark below. As representing
% all the possible contrivances leads to much complexity and also
% needs to be agreed among scholars, new ways of encoding them will be
% proposed and gradually included as \package{arabluatex} will mature.
%
% For the time being, the following is included:---
% \begin{longtable}{lllll}
% \bottomrule
% \caption*{\Cref*{tab:additional-arabic-codings}: Additional Arabic
% codings}
% \endfoot
% \captionlistentry{Additional Arabic codings}\\[-1em]
% \toprule
% Letter & \multicolumn{3}{l}{Transliteration\footnotemark}
% & Arab\TeX\ notation \\
%        & \texttt{dmg} & \texttt{loc} & \texttt{arabica} & \\ \midrule
% \endfirsthead
% \toprule
% Letter & \multicolumn{3}{l}{Transliteration}
% & Arab\TeX\ notation \\
%        & \texttt{dmg} & \texttt{loc} & \texttt{arabica} \\ \midrule
% \endhead \footnotetext{See below \vref{sec:transliteration}.}
% \label{tab:additional-arabic-codings}
% \arb[novoc]{.b} & \dmg{.b} & \loc{.b} & \brill{.b} & |.b| \\
% \arb[novoc]{^d} & \dmg{^d} & \loc{^d} & \brill{^d} & |^d| \\
% \arb[novoc]{.f} & \dmg{.f} & \loc{.f} & \brill{.f} & |.f| \\
% \arb[novoc]{.q} & \dmg{.q} & \loc{.q} & \brill{.q} & |.q| \\
% \arb[novoc]{.k} & \dmg{.k} & \loc{.k} & \brill{.k} & |.k| \\
% \pagebreak[1]
% \arb[novoc]{.n} & \dmg{.n} & \loc{.n} & \brill{.n} & |.n| \\
% \arb[novoc]{((} & \dmg{((} & \loc{((} & \brill{((} & |((| \\
% \arb[novoc]{))} & \dmg{))} & \loc{))} & \brill{))} & |))| \\
% \end{longtable}
% 
% \begin{quote}
%   |'afAman.tUs| Gal.(M) |.fmn.n.ts| (sic) Gal.(E1),
%   \arb[novoc]{'afAman.tUs} Gal.(M) \arb[novoc]{.fmn.n.ts} (sic)
%   Gal.(E1), \arb[trans]{'afAman.tUs} Gal.(M) \arb[trans]{.fmn.n.ts}
%   (sic) Gal.(E1).
% \end{quote}
%
% \subsection{Arabic emphasis}
% \label{sec:emphasis}
% As already seen in \vref{sec:abjad}, the \arb[trans]{'abjad} numbers
% are distinguished from the surrounding words by a stroke placed
% over them. This technique is used to distinguish further words that
% are proper names or book titles.
%
% \DescribeMacro{\aemph} One may use the \cs{aemph}\marg{Arabic
% text} command to use the same technique to emphasize words, like so:---
% \begin{quote}
%   |\abjad{45}:| |kitAbu-hu| |\aemph{fI| |'l-`AdAt-i}|
%   \arb[voc]{\abjad{45}: kitAbu-hu \aemph{fI 'l-`AdAt-i}}
%   \arb[trans]{\abjad{45}: kitAbu-hu \aemph{fI 'l-\uc{`AdAt-i}}}.
% \end{quote}
% 
% \begin{quoting}
%   \textsc{Rem.}~\emph{a.} As the above example shows,
%   \package{arabluatex} places the horizontal stroke \emph{under} the
%   emphasized words in any of the transliteration modes.
%
%   \textsc{Rem.}~\emph{b.} \NEWfeature{v1.9.2}\DescribeMacro{\aemph*}
%   \cs{aemph*} is also provided should one wish to always have the
%   horizontal stroke printed over the emphasized words, like so:
%   |\abjad{45}:| |kitAbu-hu| |\aemph*{fI| |'l-`AdAt-i}|
%   \arb[voc]{\abjad{45}: kitAbu-hu \aemph*{fI 'l-`AdAt-i}}
%   \arb[trans]{\abjad{45}: kitAbu-hu \aemph*{fI 'l-\uc{`AdAt-i}}}.
% \end{quoting}
%
% \subsubsection{Underlining words or numbers}
% \label{sec:underlining}
% \DescribeMacro{\aoline}%
% \DescribeMacro{\aoline*}%
% \DescribeMacro{\auline}%
% \NEWfeature{v1.19}Three additional, non context-sensitive commands
% are provided to distinguish words or numbers:---
% \begin{enumerate}
% \item \cs{aoline}, which is equivalent to \cs{aemph*} described
%   above.
% \item \cs{aoline*}, which is the same as \cs{aoline}, but better
%   suited for \arb[trans]{'ab^gad} numbers.\footnote{See the example
%   provided above \vref{sec:abjad}.}
% \item \cs{auline}, which can be used to underline Arabic words.
% \end{enumerate}
%
% \section{Arabic poetry}
% \label{sec:poetry}
% \NEWfeature{v1.6} \package{arabluatex} provides a special
% environment for typesetting Arabic poetry. Every line in this
% environment must end with |\\|.
%
% \DescribeEnv{arabverse} The |arabverse| environment may take up to
% eight optional \enquote*{named arguments} each of which is set using
% the syntax \meta{key}$=$\meta{value}, like so:---%
% \iffalse
%<*example>
% \fi
\begin{code}
  \begin{arabverse}[key1=value1, key2=value2, ...]
    <verses>
  \end{arabverse}
\end{code}
% \iffalse
%</example>
% \fi
%
% The description of the optional arguments follows:---
%
% \DescribeOption{mode} |mode|$=$\meta{mode}, either |voc|, |fullvoc|,
% |novoc| or |trans|. The default mode is the one that is set at load
% time as already seen \vref{sec:options}.
%
% \DescribeOption{width} |width|$=$\meta{length}
% \hfill\tcboxverb{Default: 0.3\linewidth}\\ The default width of
% each hemistich that the verse consists of. It may be expressed in
% any accepted unit of measurement, such as |4cm| or |2in|. However,
% one must keep in mind that the total length of the two hemistichs
% added to the one of the gutter that separates them must not exceed
% the length of the base line, unless one wishes to have the
% hemistichs distributed on subsequent lines.
%
% \DescribeOption{gutter} |gutter|$=$\meta{width}
% \hfill\tcboxverb{Default: 0.15 x (hemistich width)}\\ The gutter
% consists of the blank space that is between the two hemistichs. By
% default, it is commensurate with the width of the hemistich, but it
% may be expressed in any accepted unit of measurement as well.
%
% \DescribeOption{metre} |metre|$=$\meta{name}
% \hfill\tcboxverb{Default: none}\\ If the name of the metre is
% expressed, it is printed after the lines and set flush left in
% |voc|, |fullvoc| and |novoc| modes or flush right in |trans| mode.
%
% \DescribeOption{delim} |delim|$=$|true|\verb+|+|false|
% \hfill\tcboxverb{Default: false}\\ This named argument does not
% need a value as it defaults to |true| if it is used. If so, a
% delimiter is printed between each of the hemistichs. By default, it
% is set to the \enquote*{star} character \enquote*{*}. The
% \DescribeMacro{\SetHemistichDelim}\cs{SetHemistichDelim}\marg{delimiter}
% command may be used at any point of the document to change this
% default setting.
%
% \DescribeOption{utf} |utf|$=$|true|\verb+|+|false|
% \hfill\tcboxverb{Default: false}\\ As the preceding one, this
% named argument does not need a value as it defaults to |true| if it
% is used. If so, Unicode Arabic input is expected in the |arabverse|
% environment instead of \textsc{ascii} Arab\TeX\ or Buckwalter input
% schemes. See \vref{sec:unicode-input} for more details.
%
% \DescribeOption{color} |color|$=$\meta{color name}
% \hfill\tcboxverb{Default: not set}\\
% \NEWfeature{v1.13} The color in which lines of poetry are to be
% rendered.
%
% \label{ref:poetry-export}
% \DescribeOption{export} |export|$=$|true|\verb+|+|false|
% \hfill\tcboxverb{Default: false}\\
% \NEWfeature{v.1.13} This named argument does not need a value as it
% defaults to |true| if it is used. If |export| is set as a global
% option as well (see above \vpageref{export-mode}), all the lines
% will be converted to Unicode and exported to the external selected
% file. See below \vref{sec:arabtex2utf} for more details.
% 
% \DescribeMacro{\bayt} Inside the |arabverse| environment, each line
% is typeset by the \cs{bayt} command which takes two mandatory
% arguments and may accept one optional
% argument.\footnote{\label{ref:bayt-star}A \enquote*{starred} version
% \cs{bayt*} is also defined. \package{arabluatex} uses it internally
% when \texttt{export} is set to \texttt{true} to instruct some Lua
% functions that lines of poetry have already been processed. That
% aside, \cs{bayt} and \cs{bayt*} do the same, and only \cs{bayt}
% should be used.}  Additionally, every \cs{bayt} command \emph{must}
% be followed with |\\| like so:---%
% \iffalse
%<*example>
% \fi
\begin{tcblisting}{text only}
  \cs{bayt}\marg{\arb[trans]{.sadr}}\oarg{\arb[trans]{tadwIr}}%
  \marg{\arb[trans]{`ajuz}}|\\|
\end{tcblisting}
% \iffalse
%</example>
% \fi
%
% That two subsequent hemistichs should be connected with one another
% is technically named \arb[trans]{tadwIr}. Should that happen, either
% the \arb[trans]{.sadr} or the \arb[trans]{`ajuz} or both of them,
% may be connected to one another by letters that are naturally bound
% to the following or the preceding ones over the
% \arb[trans]{tadwIr}. The optional argument of the \cs{bayt} command
% is designed to deal with the various situations that may arise:---
% \begin{enumerate}
% \item If the two hemistichs be connected with one another by a
% prominent horizontal flexible stroke, the \arb[trans]{ta.twIl} should
% be used, like so: |[--]| (see \vref{sec:tatwil}). Of course, the
% ending word of the \arb[trans]{.sadr} and the word at the
% commencement of the \arb[trans]{`a^guz} must have the
% \arb[trans]{ta.twIl} too so that the proper shapes of the letters be
% selected. Consider for example the following:---
% \iffalse
%<*example>
% \fi
\begin{example}
  \begin{arabverse}[mode=fullvoc, width=.3\linewidth]
    \bayt{lA 'ar_A man `ahidtu fI-hA fa-'abkI 'l---}[--]{---yawma
      dalhaN wa-mA yaruddu 'l-bukA'u}\\
  \end{arabverse}
\end{example}
% \iffalse
%</example>
% \fi%
% As one can see, \emph{triple hyphens} have been used. In the
% \arb[trans]{.sadr}, the first hyphen triggers the rules that are
% related to the definite article and the \arb[trans]{'alif-u
% 'l-wa.sl-i},\footnote{See \vref{ref:definite-article}.} while the
% following two select the figure of the letter \arb[trans]{lAm}
% connected with a following letter. In the \arb[trans]{`a^guz}, the
% last two hyphens select the letter \arb[trans]{yA'} connected with a
% preceding letter, while the first one is simply discarded in this
% mode, but still may appear as it should, if the |trans| mode be
% selected:---%
% \iffalse
%<*example>
% \fi
\begin{example}
  \begin{arabverse}[mode=trans, width=.4\linewidth]
    \bayt{lA 'ar_A man `ahidtu fI-hA fa-'abkI 'l---}[--]{---yawma
      dalhaN wa-mA yaruddu 'l-bukA'u}\\
  \end{arabverse}
\end{example}
% \iffalse
%</example>
% \fi%
% \item In some other cases, it may seem difficult, if not fairly
%   impossible, to split a given word into two parts. This happens
%   mostly because of the \arb[trans]{^saddaT}. Consider for example
%   the following:---
% \iffalse
%<*example>
% \fi
\begin{example}
  \begin{arabverse}[mode=fullvoc, width=.25\linewidth,
    gutter=1cm]
    \bayt{.gayra 'annI qad 'asta`Inu `al_A 'l-ha--}[--mmi ]{'i_dA
      _haffa bi-'l-_tawiyyi 'l-na^gA'u}\\
    \bayt{bi-zaf--UfiN ka-'anna-hA hiq|--laTuN}[ 'ummu ]{ri'AliN
      dawwiyyaTuN saqfA'u}\\
  \end{arabverse}
\end{example}
% \iffalse
%</example>
% \fi%
% In the first line, the word \arb{al-hammi} should be split into
% \arb{al-ham"-- --mi} as the first part of it belongs to the
% \arb[trans]{.sadr} and the second to the \arb[trans]{`a^guz}. One
% solution to avoid splitting this word in such a way is to write
% inside the \arb[trans]{tadwIr} the part of it that belongs to either
% hemistich, without omitting to add a space after it. In the second
% line, the word \arb{'ummu} should be split into \arb{'um"-- --mu},
% so that the only way to avoid splitting it into two parts is to
% write it all inside the \arb[trans]{tadwIr}. In that case, as the
% word is to be placed in the middle, it has been surrounded by
% spaces.
% \end{enumerate}
%
% \paragraph{Scaling and distortion of characters}
% The |arabverse| environment and the \cs{bayt} command are designed
% to typeset the verses in a two-column, fixed width layout. This may
% result in a somewhat distorted text. Should that happen, one may
% adapt the layout by modifying the values of the above described
% |width| and |gutter| named arguments until the visual aspect of the
% layout be satisfactory. It has to be noted that distortion and
% warping may be even more perceptible in Roman than in Arabic
% characters.
%
% \DescribeMacro{\StretchBayt} \cs{StretchBayt}\verb+[true|false]+
% \hfill\tcboxverb{Default: true}\\
% \NEWfeature{v1.20} \cs{StretchBayt} takes one optional argument,
% either |true| or |false| and can be used to remove the stretching
% form lines of Arabic poetry. As a side effect, there will be more
% space between words, but this can be compensated by inserting double
% hyphens between letters (on this technique, see
% \vref{sec:tatwil}). Should it be desired to extend further the
% strokes, four hyphens may be inserted (|----|), viz. a multiple of
% two. \cs{StretchBayt} may be used at any point of the document, even
% between two subsequent lines of poetry. Note that
% \cs{StretchBayt}|[false]| may require to carefuly adjust the width
% of the hemistichs to avoid overlapping.
%
% \paragraph{Footnotes}
% Footnotes are not set by default inside the \cs{bayt} command, but
% there are two easy ways to have them printed.
%
% If they are little in number, each footnote may be split into pairs
% of \cs{footnote\allowbreak{}mark}|{}| (please mind the braces or
% \enquote{declare} |footnotemark| using \cs{MkArbBreak} to take it
% out of the Arabic environment\footnote{See
% \vref{sec:declare-new-commands}.}) in the argument of the \cs{bayt}
% command and \cs{footnotetext} outside the \cs{bayt} command.
%
% If the footnotes are abundant in number, it is advised to load the
% \package{footnotehyper} package which \package{arabluatex} will then
% use to typeset any kind of footnote that is called from the
% arguments of the \cs{bayt} command.\footnote{The \package{footnote}
% package can also be used for the same effect. However, it must be
% loaded \emph{after} \package{arabluatex}.}
%
% \paragraph{Line numbering}
% Inside the |arabverse| environment, the |linenumbers| environment of
% the \package{lineno} package can be used to have the lines of
% succeeding verses numbered. Please refer to the documentation of
% this package for more information or to the example below for a
% basic implementation of this technique.
%
% \subsection{Example}
% \label{sec:poetry-example}
% Here follow the first lines of \prname{imru'u 'l-qaysi}'s
% \arb[trans]{\uc{m}u`allaqaT}. In this example, \cs{SetArbDflt*} has
% been selected so as to mark the \arb[trans]{'id.gAm} that is fit to
% this declamatory poetry:---\footnote{Please note that for the time
% being only the assimilation rules that are laid on
% \vref{ref:assimilation} are applied. See
% \vref{sec:classic-modern-typesetting} for more information. None of
% the editions of the \arb[trans]{\uc{M}u`allaqAt} that I know of
% feature the \arb[trans]{'id.gAm} in the Arabic text, although it is
% often strongly marked in declamation.}%
% \iffalse
%<*example>
% \fi
\begin{code}
  \begin{arab}[fullvoc]
    qAla imru'u 'l-\uc{q}aysi fI mu`allaqati-hi:
  \end{arab}

  \begin{arabverse}[mode=fullvoc, metre={(al-.darbu 'l-_tAnI mina
      'l-`arU.di 'l-'_Ul_A mina 'l-.tawIli)}]
    \SetArbDflt*
    \begin{linenumbers*}
      \bayt{qifA nabki min _dikr_A .habIbiN wa-manzili}{bi-saq.ti
        'l-liw_A bayna \uc{'l-d}a_hUli fa-\uc{.h}awmali}\\
      \bayt{fa-\uc{t}U.di.ha fa-'l-\uc{m}iqrATi lam ya`fu
        rasmu-hA}{limA nasa^gat-hA min ^ganUbiN wa-^sam'ali}\\
      \bayt{tar_A ba`ara 'l-'ar'Ami fI `ara.sAti-hA}{wa-qI`Ani-hA
        ka-'anna-hu .habbu fulfuli}\\
      \bayt{ka-'annI .gadATa 'l-bayni yawma ta.hammalUA}{lad_A
        samurAti 'l-.hayyi nAqifu .han.zali}\\
      \bayt{wuqUfaN bi-hA .sa.hbI `alayya ma.tiyya-hum}{yaqUlUna
        lA tahlik 'asaN_A wa-ta^gammali}\\
      \bayt{wa-'inna ^sifA'I `abraTuN muharAqaTuN}{fa-hal `inda
        rasmiN dArisiN min mu`awwali}\\
    \end{linenumbers*}
  \end{arabverse}
\end{code}
% \iffalse
%</example>
% \fi%
%
% \medskip
% \noindent\textbf{\cs{StretchBayt}|[true]| (Default)}:---
% \begin{arab}[fullvoc]
%   qAla imru'u 'l-\uc{q}aysi fI mu`allaqati-hi:
% \end{arab}
% 
% \begin{arabverse}[mode=fullvoc, metre={(al-.darbu 'l-_tAnI mina
%  'l-`arU.di 'l-'_Ul_A mina 'l-.tawIli)}, width=.25\linewidth]
%  \SetArbDflt*
%   \begin{linenumbers*}
%     \bayt{qifA nabki min _dikr_A .habIbiN wa-manzili}{bi-saq.ti
%     'l-liw_A bayna \uc{'l-d}a_hUli fa-\uc{.h}awmali}\\
%     \bayt{fa-\uc{t}U.di.ha fa-'l-\uc{m}iqrATi lam ya`fu
%     rasmu-hA}{limA nasa^gat-hA min ^ganUbiN wa-^sam'ali}\\
%     \bayt{tar_A ba`ara 'l-'ar'Ami fI `ara.sAti-hA}{wa-qI`Ani-hA
%     ka-'anna-hu .habbu fulfuli}\\
%     \bayt{ka-'annI .gadATa 'l-bayni yawma ta.hammalUA}{lad_A
%     samurAti 'l-.hayyi nAqifu .han.zali}\\
%     \bayt{wuqUfaN bi-hA .sa.hbI `alayya ma.tiyya-hum}{yaqUlUna lA
%     tahlik 'asaN_A wa-ta^gammali}\\
%     \bayt{wa-'inna ^sifA'I `abraTuN muharAqaTuN}{fa-hal `inda rasmiN
%     dArisiN min mu`awwali}\\
%   \end{linenumbers*}
% \end{arabverse}
%
% \medskip
%
% \begin{arab}[trans]
%   qAla imru'u 'l-\uc{q}aysi fI mu`allaqati-hi:
% \end{arab}
% 
% \begin{arabverse}[mode=trans, metre={(al-.darbu 'l-_tAnI mina
%  'l-`arU.di 'l-'_Ul_A mina 'l-.tawIli)}, width=.4\linewidth]
%  \SetArbDflt*
%   \begin{linenumbers*}
%     \bayt{qifA nabki min _dikr_A .habIbiN wa-manzili}{bi-saq.ti
%     'l-liw_A bayna \uc{'l-d}a_hUli fa-\uc{.h}awmali}\\
%     \bayt{fa-\uc{t}U.di.ha fa-'l-\uc{m}iqrATi lam ya`fu
%     rasmu-hA}{limA nasa^gat-hA min ^ganUbiN wa-^sam'ali}\\
%     \bayt{tar_A ba`ara 'l-'ar'Ami fI `ara.sAti-hA}{wa-qI`Ani-hA
%     ka-'anna-hu .habbu fulfuli}\\
%     \bayt{ka-'annI .gadATa 'l-bayni yawma ta.hammalUA}{lad_A
%     samurAti 'l-.hayyi nAqifu .han.zali}\\
%     \bayt{wuqUfaN bi-hA .sa.hbI `alayya ma.tiyya-hum}{yaqUlUna lA
%     tahlik 'asaN_A wa-ta^gammali}\\
%     \bayt{wa-'inna ^sifA'I `abraTuN muharAqaTuN}{fa-hal `inda rasmiN
%     dArisiN min mu`awwali}\\
%   \end{linenumbers*}
% \end{arabverse}
% 
% \medskip
% \noindent\textbf{\cs{StretchBayt}|[false]|}:---
% \StretchBayt[false]\\
% In what follows, |width| has been set to |0.3\linewidth| and double
% hyphens have been inserted between some letters to prolong their
% horizontal strokes.
% \begin{arab}[fullvoc]
%   qAla imru'u 'l-\uc{q}aysi fI mu`allaqati-hi:
% \end{arab}
% 
% \begin{arabverse}[mode=fullvoc, metre={(al-.darbu 'l-_tAnI mina
%  'l-`arU.di 'l-'_Ul_A mina 'l-.tawIli)}, width=.3\linewidth]
%  \SetArbDflt*
%   \begin{linenumbers*}
%     \bayt{qifA nabki min _dikr_A .habIbiN wa-manzili}{bi-saq.ti
%     'l-liw_A bayna \uc{'l-d}a_hUli fa-\uc{.h}awmali}\\
%     \bayt{fa-\uc{t}U.di.ha fa-'l-\uc{m}iqrATi lam ya`--fu
%     rasmu---hA}{limA nasa^gat-hA mi--n ^gan----UbiN wa-^sam'ali}\\
%     \bayt{ta--r_A ba--`ara 'l-'ar'Ami fI `ara.sAti-hA}{wa-qI--`Ani-hA
%     ka-'anna---hu .ha----bbu fu--{l}--fu----li}\\
%     \bayt{ka-'annI .ga--dATa 'l-bay--ni ya--wma ta.hammalUA}{lad_A
%     samurAti 'l-.hayyi nAqifu .han.zali}\\
%     \bayt{wuq--UfaN bi-hA .sa.hbI `a--layya ma--.tiyya---hu--m}
%     {ya--q--Ul--Una lA tahli--k 'asaN_A wa-ta^gamma----li}\\
%     \bayt{wa-'inna ^si----f----A'I `a--{b}----raTuN muharAqa----TuN}
%     {fa---ha--l `inda rasmiN dArisiN min mu`awwali}\\
%   \end{linenumbers*}
% \end{arabverse}
%
% \medskip
%
% In what follows, |width| has been set to |0.375\linewidth| and
% |\scriptsize| has been used so as to avoid overlapping.
% \begin{arab}[trans]
%   qAla imru'u 'l-\uc{q}aysi fI mu`allaqati-hi:
% \end{arab}
% 
% \begin{arabverse}[mode=trans, metre={(al-.darbu 'l-_tAnI mina
%  'l-`arU.di 'l-'_Ul_A mina 'l-.tawIli)}, width=.375\linewidth]
%  \SetArbDflt*
%  \scriptsize
%   \begin{linenumbers*}
%     \bayt{qifA nabki min _dikr_A .habIbiN wa-manzili}{bi-saq.ti
%     'l-liw_A bayna \uc{'l-d}a_hUli fa-\uc{.h}awmali}\\
%     \bayt{fa-\uc{t}U.di.ha fa-'l-\uc{m}iqrATi lam ya`fu
%     rasmu-hA}{limA nasa^gat-hA min ^ganUbiN wa-^sam'ali}\\
%     \bayt{tar_A ba`ara 'l-'ar'Ami fI `ara.sAti-hA}{wa-qI`Ani-hA
%     ka-'anna-hu .habbu fulfuli}\\
%     \bayt{ka-'annI .gadATa 'l-bayni yawma ta.hammalUA}{lad_A
%     samurAti 'l-.hayyi nAqifu .han.zali}\\
%     \bayt{wuqUfaN bi-hA .sa.hbI `alayya ma.tiyya-hum}{yaqUlUna lA
%     tahlik 'asaN_A wa-ta^gammali}\\
%     \bayt{wa-'inna ^sifA'I `abraTuN muharAqaTuN}{fa-hal `inda rasmiN
%     dArisiN min mu`awwali}\\
%   \end{linenumbers*}
% \end{arabverse}
% \StretchBayt[true]
% 
% \section{Special applications}
% \label{sec:special-applications}
% \phantomsection
% \paragraph{Linguistics}
% The same horizontal stroke as the \arb[trans]{ta.twIl} (see
% \vref{sec:tatwil}) may be encoded \meta{B}; \meta{BB} will receive
% the \arb[trans]{ta^sdId}. This is useful to make linguistic
% annotations and comments on vowels:---
% \begin{quote}
%   |Bu| |Ba| |Bi| |BuN| |BaN| |BiN| \arb[voc]{Bu Ba Bi BuN BaN BiN}
%   \arb[trans]{Bu Ba Bi BuN BaN BiN}, |BBu| |BBa| |BBi| \arb[voc]{BBu
%   BBa BBi} \arb[trans]{BBu BBa BBi}, |B--aN| \arb[voc]{B--aN}
%   \arb[trans]{B--aN}, |B"| \arb[voc]{B"}\,.
% \end{quote}
%
% \paragraph{Brackets}
% \phantomsection
% \NEWfeature{v1.4.3} The various bracket symbols are useful in
% technical documents such as critical editions for indicating that
% some words or some letters must be added or
% removed. \package{arabluatex} will automatically fit those symbols
% to the direction of the text. For the time being, the following
% symbols are supported:
% \begin{itemize}
% \item parentheses: |()|
% \item square brackets: |[]|
% \item angle brackets: |<>|
% \item braces: |{}|
% \end{itemize}
%
% \DescribeMacro{\abraces} Parentheses, square and angle brackets may
% be input directly at the keyboard; however, words or letters that
% are to be read between braces must be passed as arguments to the
% \cs{abraces} command:---%
% \iffalse
%<*example>
% \fi
\begin{example}
  \begin{arab}
    \abraces{wa-qAla} 'inna 'abI kAna mina 'l-muqAtilaTi
    wa-kAna--<--t> 'ummI min `u.zamA'i buyUti 'l-zamAzimaTi.
  \end{arab}
\end{example}
% \iffalse
%</example>
% \fi
%
% \paragraph{Additional Arabic marks}
% \label{sec:arabic-marks}
% In addition to common letters, many symbols and ligatures are
% encoded in Arabic Unicode standard, such as honorifics consisting of
% complex ligatures, and annotation signs used in the
% \arb[trans]{\uc{qur'An}} or in classical poetry.
%
% \DescribeMacro{\arbmark}
% \NEWfeature{v1.11}\cs{arbmark}\oarg{rl\textbar{}lr}\marg{shorthand}
% can be used to insert such characters either in Unicode or in
% romanized Arabic environments. It takes as argument a shorthand
% defined beforehand in a default list which consists of the following
% at the time of writing:---\\
% \begin{longtable}{llp{.3\linewidth}p{.3\linewidth}}
% \bottomrule
% \caption*{\Cref*{tab:arabtex-additional-marks}: Additional Arabic
% marks}
% \endfoot
% \captionlistentry{Additional Arabic marks}\\[-1em]
% \toprule
% Codepoint & Shorthand & Glyph & Transliteration \\ \midrule
% \endfirsthead
% \toprule
% Codepoint & Shorthand & Glyph & Transliteration \\ \midrule
% \endhead\label{tab:arabtex-additional-marks}%
% |FDFD| & |bismillah| & \arb{\arbmark{bismillah}} &
% \arbmark{bismillah} \\
% |FDF5| & |salam| & \arb{\arbmark{salam}} & \arbmark{salam} \\
% |FDFA| & |slm| & \arb{\arbmark{slm}} & \arbmark{slm} \\
% |FDFB| & |jalla| & \arb{\arbmark{jalla}} & \arbmark{jalla} \\
% \end{longtable}
%
% \NEWfeature{v1.13}
% The mark to be inserted is determined by contextual analysis, or by
% an optional argument, either |rl| to have the Arabic glyph printed,
% or |lr| to print the tranliterated equivalent.
% 
% \DescribeMacro{\newarbmark} \NEWfeature{v1.11} \cs{newarbmark} is
% also provided should one wish to define new marks in addition to the
% marks defined above. This command takes three arguments, like so:---
% \arabluabox{\cs{newarbmark}\marg{shorthand}\marg{RTL
% codepoint}\marg{LTR rendition}}
%
% As regards the right-to-left codepoint, it may be either typed in
% Unicode or selected as Unicode codepoint. To that end, the \LaTeX\
% command \tcboxverb{\symbol{"XYZT}} or its plain \TeX\ variant
% \tcboxverb{\char"XYZT\relax} may be used, where |XYZT| are
% uppercase hex digits (|0| to |9| or |A| to |F|).
%
% It is also possible to use the so-called `|^^^^| notation' like so:
% \tcboxverb{^^^^xyzt}, where |xyzt| are lowercase hex digits (|0|
% to |9| or |a| to |f|).
%
% As regards the third argument (left-to-right rendition), it may be
% either left empty or typed by means of
% \cs{arb}|[trans]|\marg{arabtex code} so as to have it printed in
% romanized Arabic.%
% \iffalse
%<*example>
% \fi
\begin{tcblisting}{text only}
  It must be noted that \cs{newarbmark} expects Arab\TeX\ input scheme
  inside \cs{arb}|[trans]{}| to the exclusion of \textsf{buckwalter}
  input scheme.
\end{tcblisting}
% \iffalse
%</example>
% \fi%
% 
% The example below provides an implementation of this technique. It
% may be observed that \cs{arbcolor} is used so as to have the marks
% printed in red:---%
% \iffalse
%<*example>
% \fi
\begin{example}
  \SetArbDflt*
  \newarbmark{sly}{\arbcolor[red]{^^^^06d6}}{}
  \newarbmark{jim}{\arbcolor[red]{^^^^06da}}{}
  \begin{arab}
    sUraTu 'l-nisA'i, 19:
  \end{arab}
  \begin{center}
    \begin{arab}
      \arbmark{bismillah}
    \end{arab}
  \end{center}
  \begin{arab}[fullvoc]
    y_a'ayyuhA 'lla_dIna 'a'manUA lA ya.hillu la-kum 'an tari_tUA
    'l-nisA'a karhaN\arbmark{sly} wa-lA ta`.dulU-hunna li-ta_dhabUA
    bi-ba`.di mA 'a'taytumU-hunna 'illA 'an ya'tIna bi-fA.hi^saTiN
    mubayyinaTiN\arbmark{jim} wa-`A^sirU-hunna
    bi-'l-ma`rUfi\arbmark{jim} fa-'in karihtumU-hunna fa-`as_A_a
    'an takrahUA ^say'aN wa-ya^g`ala 'l-l_ahu fI-hi _hayraN
    ka_tIraN ((19))
  \end{arab}
\end{example}
% \iffalse
%</example>
% \fi
%
% \paragraph{\texorpdfstring{The \enquote*{Zero width joiner}
% character (\texttt{U+200D})}%
% {The ‘Zero width joiner’ character (\texttt{U+200D})}}
% \phantomsection%
% \NEWfeature{v1.18}%
% The \enquote*{Zero width joiner} character (|U+200D|) belongs to the
% \enquote*{General Punctuation} block (range |2000|--|206F|) of the
% Unicode standard.  It is a non-printing character which, when it is
% placed between two characters that would for some reason not be
% connected, causes them to be printed in their connected forms.
%
% It is encoded |&| in Arab\TeX\ scheme.
%
% In elegantly printed books where many of the letters are interwoven
% with one another so as to form ligatures, it may be convenient to
% bring the letters into line in some instances.  In the following
% example, the \enquote*{zero width joiner} is used to prevent two
% adjacent letters, viz.\ \arb[novoc]{s} and \arb[novoc]{.h}, from
% standing one above the other in the name of \prname{'is.h_aq}
% (\arb[fullvoc]{'is.h_aq"}):\footnote{\cs{underLine} and
% \cs{highLight} are taken from the \package{lua-ul} package which is
% loaded by \package{arabluatex}. See \textcite{pkg:lua-ul}.}---
% \iffalse
%<*example>
% \fi
\begin{example}
  \begin{arab}[fullvoc]
    huwa 'abU zaydiN .hunaynu bnu 'is&\underLine{&.h_a}qa
    'l-`a\underLine{bA}diyyu bi-fat.hi 'l-`ayni wa-ta_hfIfi 'l-bA'i.
    
    huwa 'abU zaydiN .hunaynu bnu 'is&\highLight{&.h_a}qa
    'l-`a\highLight{bA}diyyu bi-fat.hi 'l-`ayni wa-ta_hfIfi 'l-bA'i.
  \end{arab}
\end{example}
% \iffalse
%</example>
% \fi
%
% \subsection{\texorpdfstring{The \prname{qur'An}}{The Qurʾān}}
% \label{sec:the-quran}
% This sub-part is destined to become a part of its own, as fine
% typesetting of \prname{qur'An}ic text is planned in the versions of
% \package{arabluatex} to come in the medium-term. New functions and
% new Arabic modes will be available as \package{arabluatex} will
% mature.
%
% \DescribeMacro{\ayah}\NEWfeature{v1.15}For the time being,
% \cs{ayah}\marg{3-digit number} is provided so as to typeset the
% number of the \arb[trans]{'AyaT} that it is referred to inside the
% dedicated mark---Unicode |U+06DD|: \txarb{^^^^06dd}---in Arabic
% script or inside parentheses in romanized Arabic:---
% \begin{quote}
%   |\ayah{123}| \arb{\ayah{123}} \arb[trans]{\ayah{123}}.
% \end{quote}
%
% An example follows:---
% \iffalse
%<*example>
% \fi
\begin{example}
  \SetArbDflt*
  \newarbmark{alifsp}{^^^^0627}{\arb[trans]{'alif} }
  \newarbmark{lamsp}{^^^^0644^^^^0653}{\arb[trans]{lAm} }
  \newarbmark{mim}{^^^^0645^^^^0653}{\arb[trans]{mIm}}
  \begin{arab}[fullvoc]
    min ((sUraTi \uc{'l-b}aqaraTi)):
  \end{arab}
  \begin{arab}[fullvoc]
    \arbmark{alifsp}\arbmark{lamsp}\arbmark{mim}~\ayah{1}
    _d_alika 'l-kit_abu lA rayba fI-hi hudaN_A
    li-l-muttaqIna~\ayah{2} 'lla_dIna yu'minUna bi-'l-.gaybi
    wa-yuqImUna 'l-.sal_aUTa wa-mimmA razaqn_a-hum
    yunfiqUna~\ayah{3}
  \end{arab}
\end{example}
\begin{tcblisting}{text only}
  \SetArbDflt*
  \begin{arab}[trans]
    min ((sUraTi \uc{'l-b}aqaraTi)):
  \end{arab}
  \begin{arab}[trans]
    \arbmark{alifsp}\arbmark{lamsp}\arbmark{mim}~\ayah{1}
    _d_alika 'l-kit_abu lA rayba fI-hi hudaN_A
    li-l-muttaqIna~\ayah{2} 'lla_dIna yu'minUna bi-'l-.gaybi
    wa-yuqImUna 'l-.sal_aUTa wa-mimmA razaqn_a-hum
    yunfiqUna~\ayah{3}
  \end{arab}
\end{tcblisting}
% \iffalse
%</example>
% \fi
%
% \paragraph{Caveat}
% For some reason, most of the Arabic fonts do not show the number
% properly: some are only able to display at most two digits, while
% others display the digits outside the \enquote*{end of
% \arb[trans]{'AyaT}} sign, let alone those that print the digits
% stacked. To the knowledge of the writer, this should be reported to
% the developers of those fonts.
% 
% \section{Color}
% \label{sec:color}
% \NEWfeature{v1.12}\package{arabluatex} is able to render in color
% either words, parts of words or diacritics. As the techniques
% implemented in this section may lead to some complexity, the reader
% should first become well acquainted with the following
% points:\footnote{Regarding the colors themselves and the way new
% colors can be defined in addition to those that are already
% available, please refer to the \package{xcolor} package.}---
% \begin{enumerate}
% \item The \enquote{pipe} character (\textbar, \vref{sec:pipe});
% \item \enquote*{Quoting} technique (\vref{sec:quoting}), and more
%   specifically \enquote*{quoting the \arb[trans]{hamzaT}}
%   (\vpageref{sec:quoting-hamza});
% \item Putting back on broken contextual analysis rules
%   (\vref{sec:arbnull});
% \item Arabic marks (\vref{sec:arabic-marks}).
% \end{enumerate}
%
% \DescribeMacro{\arbcolor} \cs{arbcolor} takes the text to be colored
% into \meta{color} as an argument:---
% \arabluabox{\cs{arbcolor}\oarg{color}\marg{Arabic text}}
% 
% \iffalse
%<*example>
% \fi
\begin{example}
  \begin{arab}
    \arbcolor[red]{al-bAbu 'l-_hAmisu} fI .tabaqAti 'l-'a.tibbA'i
    'lla_dIna kAnUA mun_du zamAni \uc{^gAlInUsa} wa-qarIbaN
    min-hu. \arbcolor[red]{\uc{^gAlInUsu}}: wa-l-na.da` 'awwalaN
    kalAmaN kulliyyaN fI 'a_hbAri \uc{^gAlInUsa} wa-mA kAna
    `alay-hi...
  \end{arab}
  \begin{arab}[trans]
    \arbcolor[red]{al-bAbu 'l-_hAmisu} fI .tabaqAti 'l-'a.tibbA'i
    'lla_dIna kAnUA mun_du zamAni \uc{^gAlInUsa} wa-qarIbaN
    min-hu. \arbcolor[red]{\uc{^gAlInUsu}}: wa-l-na.da` 'awwalaN
    kalAmaN kulliyyaN fI 'a_hbAri \uc{^gAlInUsa} wa-mA kAna
    `alay-hi...
  \end{arab}
\end{example}
% \iffalse
%</example>
% \fi
%
% As this example shows, \cs{arbcolor} has been used to render
% headings in red with the same encoding both in vocalized and in
% romanized Arabic. The same technique also applies to syllabes inside
% words. \package{arabluatex} takes care of selecting the appropriate
% shape of the letters while coloring them:---
% \begin{quote}\textbf{\enquote*{voc} mode}:\\
%   |i^stara\arbcolor[brown]{y}tu-hu| |bi-_tama\arbcolor[red]{niN}|
%   |'a`\arbcolor[blue]{^ga}ba-ka|
%   \arb{i^stara\arbcolor[brown]{y}tu-hu bi-_tama\arbcolor[red]{niN}
%   'a`\arbcolor[blue]{^ga}ba-ka}
%   \arb[trans]{i^stara\arbcolor[brown]{y}tu-hu
%   bi-_tama\-\arbcolor[red]{niN} 'a`\arbcolor[blue]{^ga}ba-ka}.
% \end{quote}
% \begin{quote}\textbf{\enquote*{fullvoc} mode}:\\
%   |i^stara\arbcolor[brown]{y}tu-hu| |bi-_tama\arbcolor[red]{niN}|
%   |'a`\arbcolor[blue]{^ga}ba-ka|
%   \arb[fullvoc]{i^stara\arbcolor[brown]{y}tu-hu
%   bi-_tama\arbcolor[red]{niN} 'a`\arbcolor[blue]{^ga}ba-ka}
%   \arb[trans]{i^stara\arbcolor[brown]{y}tu-hu
%   bi-_tama\-\arbcolor[red]{niN} 'a`\arbcolor[blue]{^ga}ba-ka}.
% \end{quote}
%
% \subsection{Tricks of the trade}
% \label{sec:color-tricks}
% \paragraph{Diacritics}
% Depending on the mode selected, either |voc|, |novoc| or |fullvoc|,
% coloring the diacritics requires more attention for the insertion of
% \cs{arbcolor} may prevent contextual analysis from being applied.
%
% Furthermore, depending on the surrounding letters, the standard
% encoding of short vowels \meta{u, a, i} may result either in
% diacritics or in a connective \arb[trans]{'alif} with the
% \arb[trans]{wa.slaT} or its accompanying vowel. As for the
% \arb[trans]{sukUn}, it is generated by contextual analysis. Thus
% applying colors to bare diacritics requires them to have specific
% encodings.
%
% \Cref{tab:arbcolor-diacritics} gives the Arab\TeX\ equivalents for
% the diacritics to be printed inside or just after \cs{arbcolor}.
%
% \needspace{7\baselineskip}
% \begin{longtable}{lllll}
% \bottomrule
% \caption*{\Cref*{tab:arbcolor-diacritics}: Arab\TeX\ diacritics
% for \cs{arbcolor}}
% \endfoot
% \captionlistentry{Arab\TeX\ diacritics for \cs{arbcolor}}\\[-1em]
% \toprule
% Diacritic & \multicolumn{3}{l}{Transliteration\footnotemark}
% & Arab\TeX\ notation \\
%        & \texttt{dmg} & \texttt{loc} & \texttt{arabica} & \\ \midrule
% \endfirsthead
% \toprule
% Diacritic & \multicolumn{3}{l}{Transliteration}
% & Arab\TeX\ notation \\
%        & \texttt{dmg} & \texttt{loc} & \texttt{arabica} & \\ \midrule
% \endhead \footnotetext{See below \vref{sec:transliteration}.}
% \label{tab:arbcolor-diacritics}
% \arb{B.a} & \dmg{.a} & \loc{.a} & \brill{.a} & \verb|.a| \\
% \pagebreak[1]
% \arb{B.u} & \dmg{.u} & \loc{.u} & \brill{.u} & \verb|.u| \\
% \pagebreak[1]
% \arb{B.i} & \dmg{.i} & \loc{.i} & \brill{.i} & \verb|.i| \\ \midrule
% \arb{Bo} & \dmg{o} & \loc{o} & \brill{o} & \verb|o| \\
% \end{longtable}
%
% The following examples show how the letters, or the diacritics above
% or under them or both the letters and the diacritics can be rendered
% in different colors:---
% \begin{quote}\textbf{\enquote*{voc} mode}:\\
%   |i^staraytu-hu| |bi-_taman\arbcolor[red]{iN}|
%   |'a`^g\arbcolor[red]|\allowbreak|{.a}ba-ka|
%   \arb{i^staraytu-hu bi-_taman\arbcolor[red]{iN}
%   'a`^g\arbcolor[red]{.a}ba-ka}
%   \arb[trans]{i^staraytu-hu bi-_taman\arbcolor[red]{iN}
%   'a`^g\arbcolor[red]{.a}ba-ka}.
%   
%   |i^staraytu-hu| |bi-_tama\arbcolor[red]{n}iN|
%   |'a`\arbcolor[red]|\allowbreak|{^g}.aba-ka|
%   \arb{i^staraytu-hu bi-_tama\arbcolor[red]{n}iN
%   'a`\arbcolor[red]{^g}.aba-ka}
%   \arb[trans]{i^staraytu-hu bi-_tama\arbcolor[red]{n}iN
%   'a`\arbcolor[red]{^g}.aba-ka}.
%   
%   |i^staraytu-hu| |bi-_tama\arbcolor[red]{n}\arbcolor[blue]{iN}|
%   \allowbreak
%   |'a`\arbcolor[red]|\allowbreak|{^g}\arbcolor[blue]{.a}ba-ka|
%   \arb{i^staraytu-hu bi-_tama\arbcolor[red]{n}\arbcolor[blue]{iN}
%   'a`\arbcolor[red]{^g}\arbcolor[blue]{.a}ba-ka} \linebreak
%   \arb[trans]{i^staraytu-hu
%   bi-_tama\arbcolor[red]{n}\arbcolor[blue]{iN}
%   'a`\arbcolor[red]{^g}\arbcolor[blue]{.a}ba-ka}.
% \end{quote}
% 
% \begin{quote}\textbf{\enquote*{fullvoc} mode}:\\
%   |i^staray"\arbcolor[red]{o}tu-hu| |bi-_taman"\arbcolor[red]{iN}|
%   |'a`^g"\arbcolor[red]{.a}ba-ka|
%   \arb[fullvoc]{i^staray"\arbcolor[red]{o}tu-hu
%   bi-_taman"\arbcolor[red]{iN} 'a`^g"\arbcolor[red]{.a}ba-ka}
%   \arb[trans]{i^staray"\arbcolor[red]{o}tu-hu
%   bi-_taman"\arbcolor[red]{iN}
%   \linebreak 'a`^g"\arbcolor[red]{.a}ba-ka}.
%   
%   |i^stara\arbcolor[red]{y"}otu-hu| |bi-_tama\arbcolor[red]{n"}iN|
%   |'a`\arbcolor[red]|\allowbreak|{^g"}.aba-ka|
%   \arb[fullvoc]{i^stara\arbcolor[red]{y"}otu-hu
%   bi-_tama\arbcolor[red]{n"}iN 'a`\arbcolor[red]{^g"}.aba-ka}
%   \arb[trans]{i^stara\arbcolor[red]{y"}otu-hu
%   bi-_tama\arbcolor[red]{n"}iN 'a`\arbcolor[red]{^g"}.aba-ka}.
%   
%   |i^stara\arbcolor[red]{y"}\arbcolor[blue]{o}tu-hu|
%   |bi-_tama\arb|\allowbreak|color[red]{n"}\arbcolor[blue]{iN}|
%   |'a`\arbcolor[red]|\allowbreak|{^g"}\arb|\allowbreak%
%   |color[blue]{.a}ba-ka|
%   \arb[fullvoc]{i^stara\arbcolor[red]{y"}\arbcolor[blue]{o}tu-hu
%   bi-_tama\arbcolor[red]{n"}\arbcolor[blue]{iN}
%   'a`\arbcolor[red]{^g"}\arbcolor[blue]{.a}ba-ka}
%   \arb[trans]{i^stara\arbcolor[red]{y"}\arbcolor[blue]{o}tu-hu
%   bi-_tama\arbcolor[red]{n"}\arbcolor[blue]{iN}
%   'a`\arbcolor[red]{^g"}\arbcolor[blue]{.a}ba-ka}.
% \end{quote}
%
% As can be seen, |fullvoc| required the letters |y|, |n| and |^g|
% before \cs{arbcolor} to be \enquote*{quoted}. Otherwise, unwanted
% \arb[trans]{sukUn}\txtrans{s} would have been generated because of
% the absence of a vowel after those consonants.
%
% \paragraph{\texorpdfstring{\arb[trans]{tanwIn}}{tanwīn}}
% \cs{arbnull} must be used with \arb[trans]{fat.haTAn} (\arb{BaN}) so
% as to put back on contextual analysis rules:---
% \begin{quote}
%   |mu`allim\arbcolor[red]{\arbnull{m}aN}|
%   \arb{mu`allim\arbcolor[red]{\arbnull{m}aN}}
%   \arb[trans]{mu`allim\arbcolor[red]{\arbnull{m}aN}},\\
%   |istisqA'\arbcolor[red]{\arbnull{A'}aN}|
%   \arb{istisqA'\arbcolor[red]{\arbnull{A'}aN}}
%   \arb[trans]{istisqA'\arbcolor[red]{\arbnull{A'}aN}},\\
%   |^say'\arbcolor[red]{\arbnull{ay'}aN}|
%   \arb{^say'\arbcolor[red]{\arbnull{ay'}aN}}
%   \arb[trans]{^say'\arbcolor[red]{\arbnull{ay'}aN}},\\
%   \verb+^gAmi`aT|\arbcolor[red]{\arbnull{T}aN}+
%   \arb{^gAmi`aT|\arbcolor[red]{\arbnull{T}aN}}
%   \arb[trans]{^gAmi`aT|\arbcolor[red]{\arbnull{T}aN}}.
% \end{quote}
% \begin{quoting}
%   \textsc{Rem.} Note that in the last example
%   (\arb[trans]{^gAmi`aT|\arbcolor[red]{\arbnull{T}aN}}), the
%   \enquote*{pipe} character has been inserted before
%   \cs{arbcolor}. Otherwise, the |dmg| mode of the transliteration
%   rules would have interpreted the \arb[trans]{tA' marbU.taT} as
%   \emph{final} (e.g. \txtrans{h} instead of the expected
%   \txtrans{t}).\footnote{See also \vpageref{ref:ta-marbutah-pipe}
%   \enquote{Discarding the \arb[trans]{'i`rAb}} for more
%   information.}
% \end{quoting}
%
% The \arb[trans]{tanwIn} preceding a \arb[novoc]{_A} conveys even
% more intricate business to the rendering with the utmost accuracy in
% both romanized and non-romanized modes. First, a new Arabic mark
% needs to be defined.  It should print \arb[novoc]{_A} in Arabic
% script and not a thing in transliteration. It is to be appended after
% \cs{arbcolor}, like so:---
% \iffalse
%<*example>
% \fi
\begin{example}
  \newarbmark{Y}{^^^^0649}{}
  \arb{hud\arbcolor[red]{aN\arbnull{_A}}\arbmark{Y}}
  \arb[trans]{hud\arbcolor[red]{aN\arbnull{_A}}\arbmark{Y}}
\end{example}
% \iffalse
%</example>
% \fi%
%
% \paragraph{\texorpdfstring{\arb[trans]{wa.slaT} and
% \arb[trans]{maddaT}}{waṣlah and maddah}}
% Both can be generated with the help of \cs{arbnull}:---
% \begin{quote}
%   |wa-\arbcolor[red]{\arbnull{wa}i}stisqA'uN|
%   \arb[fullvoc]{wa-\arbcolor[red]{\arbnull{wa}i}stisqA'uN}
%   \arb[trans]{wa-\arbcolor[red]{\arbnull{wa}i}stisqA'uN}\footnote{To
%   the knowledge of the writer, the \arb[trans]{wa.slaT} alone is not
%   part of the Arabic Unicode block.}.
%
%   |fI| |"al".i-\arbcolor[red]{\arbnull{'l-}i}btidA'i|
%   \arb[fullvoc]{fI "al".i-\arbcolor[red]{\arbnull{'l-}i}btidA'i}
%   \linebreak
%   \arb[trans]{fI "al".i-\arbcolor[red]{\arbnull{'l-}i}btidA'i}.
%
%   |\arbcolor[red]{'a'\arbnull{k}}kulu|
%   \arb{\arbcolor[red]{'a'\arbnull{k}}kulu}
%   \arb[trans]{\arbcolor[red]{'a'\arbnull{k}}kulu},\\
%   |\arbcolor[red]{'A\arbnull{k}}kiluN|
%   \arb{\arbcolor[red]{'A\arbnull{k}}kiluN}
%   \arb[trans]{\arbcolor[red]{'A\arbnull{k}}kiluN}.
% \end{quote}
%
% The Unicode codepoint of the \arb[trans]{maddaT} is 0653, while bare
% \arb[trans]{'alif} is 0627. So:--- %
% \iffalse
%<*example>
% \fi
\begin{example}
  \newarbmark{alifmaddahred}{^^^^0627\arbcolor[red]{^^^^0653}}%
  {\arb[trans]{\arbcolor[red]{'a'\arbnull{k}}}}
  \arb{\arbmark{alifmaddahred}kulu}
  \arb[trans]{\arbmark{alifmaddahred}kulu}.
\end{example}
% \iffalse
%</example>
% \fi%
% 
% \begin{quoting}
%   \textsc{Rem.} In the preceding example, any consonant could have
%   been passed as argument to the \cs{arbnull} command.
% \end{quoting}
% 
% \paragraph{\texorpdfstring{\arb[trans]{^sad\-daT}}{šaddah}}
% In the following example, it is assumed that the
% \arb[trans]{^saddaT} above the letter \arb[novoc]{l} in
% \arb[fullvoc]{al-mu`allimUna}, \arb[trans]{al-mu`allimUna}, is to be
% rendered in red. Thus the Arabic mark must generate the
% \arb[trans]{^saddaT} alone---of which the Unicode codepoint is
% 0651---in Arabic script and the letter \enquote*{l} in
% transliteration:--- %
%\iffalse
%<*example>
% \fi
\begin{example}
  \newarbmark{lamshaddah}{^^^^0651}{l}
  \arb[fullvoc]{al-mu`al"\arbcolor[red]{\arbmark{lamshaddah}}.imUna}
  \arb[trans]{al-mu`al"\arbcolor[red]{\arbmark{lamshaddah}}.imUna}.
\end{example}
% \iffalse
%</example>
% \fi%
%
% \paragraph{\texorpdfstring{The definite article and the euphonic
% \arb[trans]{ta^sdId}}{The definite article and the euphonic tašdīd}}
% The intricate business of rendering in color the initial
% \arb[trans]{'alif al-wa.sl} of the definite article followed by a
% solar consonant must be unraveled.
%
% From the examples provided above, in |fI 'l-nAsi| \arb[fullvoc]{fI
% 'l-nAsi} \arb[trans]{fI 'l-nAsi}, the initial \arb[trans]{'alif-u
% 'l-wa.sl-i} can be rendered in red like so:
% |\arbcolor[red]{\arbnull{al-}a}|
% \arb[fullvoc]{\arbcolor[red]{\arbnull{al-}a}}. Then, the following
% two letters, namely |l-n|, must print the string \arb[trans]{lAm}
% $+$ \arb[trans]{nUn} $+$ \arb[trans]{^saddaT} in Arabic, and exactly
% \txtrans{n-n} in transliteration. Thus an Arabic mark is
% needed:--- %
% \iffalse
%<*example>
% \fi
\begin{example}
  \newarbmark{lnn}{^^^^0644^^^^0646^^^^0651}{n-n}
  \arb[fullvoc]{fI\arbnull{al-}
    \arbcolor[red]{\arbnull{al-}a}\arbmark{lnn}Asi}
  \arb[trans]{fI\arbnull{al-}
    \arbcolor[red]{\arbnull{al-}a}\arbmark{lnn}Asi}.
\end{example}
% \iffalse
%</example>
% \fi%
%
% \paragraph{\texorpdfstring{\arb[trans]{hamzaT}}{hamzah}}
% The \enquote*{quoting} technique provides an easy way to determine
% the carrier of the \arb[trans]{hamzaT}, as shown in
% \vref{tab:quoted-hamza}---:
% \begin{quote}
%   \verb+yatasA\arbnull{'a}\arbcolor[red]{|"'}.alUna+
%   \arb{yatasA\arbnull{'a}\arbcolor[red]{|"'}.alUna}
%   \arb[trans]{yatasA\arbnull{'a}\-\arbcolor[red]{|"'}.a\-lUna},
%   \verb+^say\arbcolor[red]{|"'}\arbnull{'}aN+
%   \arb{^say\arbcolor[red]{|"'}\arbnull{'}aN}
%   \arb[trans]{^say\arbcolor[red]{|"'}\arbnull{'}aN},
%   \verb+^say\ar+\allowbreak\verb+bcolor[red]{|"'}iN+
%   \arb{^say\arbcolor[red]{|"'}iN}
%   \arb[trans]{^say\arbcolor[red]{|"'}iN},
%   |\arbcolor[red]{a"'}.as\arbcolor|\allowbreak|[red]{y"'}.ilaTuN|
%   \arb{\arbcolor[red]{a"'}.as\arbcolor[red]{y"'}.ilaTuN}
%   \arb[trans]{\arbcolor[red]{a"'}.as\arbcolor[red]{y"'}.ilaTuN}.
% \end{quote}
% 
%
% \section{Transliteration}
% \label{sec:transliteration}
% It may be more appropriate to speak of \enquote{romanization} than
% \enquote{transliteration} of Arabic. As seen above in
% \cref{sec:options} \vpagerefrange{sec:options}{sec:local-options},
% the \enquote{transliteration mode} may be selected globally or locally.
%
% This mode transliterates the Arab\TeX\ input into one of the
% accepted standards. As said above \vpageref{ref:describe-trans},
% three standards are supported at present:
% \begin{description}
% \item[dmg] \emph{Deutsche Morgenländische Gesellschaft}, which was
% adopted by the International Convention of Orientalist Scholars in
% Rome in 1935.\footnote{See \textcite{dmg}.} |dmg| transliteration
% convention is selected by default;
% \item[loc] \emph{Library of Congress}: this standard is part of a
%   large set of standards for romanization of non-roman scripts
%   adopted by the American Library Association and the Library of
%   Congress;\footnote{See
%   \url{http://www.loc.gov/catdir/cpso/roman.html} for the
%   \href{http://www.loc.gov/catdir/cpso/romanization/arabic.pdf}{source
%   document concerning Arabic language}.}
% \item[arabica] \NEWfeature{v1.8}
%   \changes{v1.8}{2017/03/30}{\texttt{arabica} transliteration
%   standard is now supported} \emph{Journal of Arabic and Islamic
%   Studies}/\emph{Revue d'études arabes et islamiques}: this standard
%   is most widely used by scholars in the field of Arabic
%   studies.\footnote{See
%   \url{http://www.brill.nl/files/brill.nl/specific/authors_instructions/ARAB.pdf}.}
% \end{description}
% More standards will be included in future releases of
% \package{arabluatex}.
%
% \paragraph{Convention} \DescribeMacro{\SetTranslitConvention} The
% transliteration mode, which is set to |dmg| by default, may be
% changed at any point of the document by the
% \cs{SetTranslitConvention}\marg{mode} command , where \meta{mode}
% may be either |dmg|, |loc| or |arabica|. This command is also
% accepted in the preamble should one wish to set the transliteration
% mode globally, e.g.:---%
% \iffalse
%<*example>
% \fi
\begin{code}
  \usepackage{arabluatex}
  \SetTranslitConvention{loc}
\end{code}
% \iffalse
%</example>
% \fi
%
% \paragraph{Style} \DescribeMacro{\SetTranslitStyle} Any
% transliterated Arabic text is printed in italics by default. This
% also can be changed either globally in the preamble or locally at
% any point of the document by the \cs{SetTranslitStyle}\marg{style}
% command, where \meta{style} may be any font shape selection
% command, e.g. \cs{upshape}, \cs{itshape}, \cs{slshape}, and so forth.
%
% \paragraph{Font} \NEWfeature{v1.4} \DescribeMacro{\SetTranslitFont}
% \cs{SetTranslitFont}\marg{font selection command} allows any
% specific font to be selected for rendering transliterated text with
% the font-selecting commands of the \package{fontspec} or
% \package{luaotfload} package. Of course, this font must have been
% defined properly. To take one example, here is how the \emph{Gentium
% Plus} font can be used for rendering transliterated text:---
% \iffalse
%<*example>
% \fi
\begin{code}
  \newfontfamily\translitfont{Gentium Plus}[Ligatures=TeX]
  \SetTranslitFont{\translitfont}
\end{code}
% \iffalse
%</example>
% \fi
%
% \paragraph{Proper names} \DescribeMacro{\uc} Proper names or book
% titles that must have their first letters uppercased may be passed
% as arguments to the \cs{uc}\marg{word} command. \cs{uc} is a
% clever command, for it will give the definite article
% \arb[trans]{al-} in lower case in all positions. Moreover, if the
% inital letter, apart from the article, cannot be uppercased,
% viz. \arb[trans]{|"'} or \arb[trans]{`}, the letter next to it will be
% uppercased:---
% \begin{quote}
%   |\uc{.hunayn-u}| |bn-u| |\uc{'is.h_aq-a}|
%   \arb[voc]{\uc{.hunayn-u} bn-u \uc{'is.h_aq-a}}
%   \arb[trans]{\uc{.hunayn-u} bn-u \uc{'is.h_aq-a}},
%   |\uc{`u_tm_an-u}| \arb[voc]{\uc{`u_tm_an-u}}
%   \arb[trans]{\uc{`u_tm_an-u}}, |.daraba| |\uc{zayd-u}| |bn-u|
%   |\uc{_h_alidiN}| |\uc{sa`d-a}| |bn-a| |\uc{`awf-i}| |bn-i|
%   |\uc{|\allowbreak|`abd-i}| |\uc{'l-l_ah-i}|
%   \arb[fullvoc]{.daraba \uc{zayd-u} bn-u \uc{_h_alidiN}
%   \uc{sa`d-a} bn-a \uc{`awf-i} bn-i \uc{`abd-i} \uc{'l-l_ah-i}}
%   \arb[trans]{.daraba \uc{zayd-u} bn-u \uc{_h_alidiN} \uc{sa`d-a}
%   bn-a \uc{`awf-i} bn-i \uc{`abd-i} \uc{'l-l_ah-i}}.
% \end{quote}
% However, \cs{uc} must be used cautiously in some very particular
% cases, for the closing brace of its argument may prevent a rule from
% being applied. To take an example, as seen above
% \vpageref{ref:muhammaduni}, the transliteration of
% \arb[fullvoc]{\uc{m}u.hammaduN 'l-nabI} must be
% \arb[trans]{\uc{m}u.hammaduN 'l-nabI}, as nouns having the
% \arb[trans]{tanwIn} take a \arb[trans]{kasraT} in pronunciation
% before \arb[trans]{'alifu 'l-wa.sli}. In that case, encoding
% \arb[fullvoc]{mu.hammaduN} like so: |\uc{mu.hammaduN}| is wrong,
% because the closing brace would prevent \package{arabluatex} from
% detecting the sequence \meta{-uN} immediately followed by
% \meta{'l-}. Fortunately, this can be circumvented in a
% straightforward way by inserting only part of the noun in the
% argument of \cs{uc} vz. up to the first letter that is to be
% uppercased, like so: |\uc{m}u.hammaduN|.
%
% \paragraph{Hyphenation}
% In case transliterated Arabic words break the \TeX\ hyphenation
% algorithm, one may use the |\-| command to insert discretionary
% hyphens. This command will be discarded in all of the Arabic modes
% of \package{arabluatex}, but will be processed by any of the
% transliteration modes:---
% \begin{quote}
%   |\uc{'abU}| |\uc{bakriN}| |\uc{mu\-.ham\-madu}| |bnu|
%   |\uc{za\-ka| |\-riy\-yA'a}| |\uc{'l-rAziyyu}| \arb{\uc{'abU}
%   \uc{bakriN} \uc{mu\-.ham\-mad-u} bnu \uc{za\-ka\-riy\-yA'a}
%   \uc{'l-rAziyyu}} \arb[trans]{\uc{'abU} \uc{bakriN}
%   \uc{mu\-.ham\-mad-u} bn-u \uc{za\-ka\-riy\-yA'-a}
%   \uc{'l-rAziyyu}}.
% \end{quote}
%
% \paragraph{\texorpdfstring{\enquote*{Long} pro\-per
% names}{‘Long’ proper names}}
% \NEWfeature{v1.10} \cs{uc} is also able to process proper names
% consisting of several subsequent words:---
% \begin{quote}
%   |\arb[trans]{\uc{'abU| |zaydiN| |.hunaynu| |bnu| |'is.h_aqa|
%   |'l-`ibAdiyyu}}| \arb[trans]{\uc{'abU zaydiN .hunaynu bnu
%   'is.h_aqa 'l-`ibAdiyyu}}.
% \end{quote}
%
% \paragraph{Proper names outside Arabic environments}
% \changes{v1.10}{2018/01/03}{\cs{uc} supersedes \cs{cap}}
% \DescribeMacro{\prname}\NEWfeature{v1.10} Transliterated proper
% names inserted in paragraphs of English text should be printed in
% the same typeface as the surrounding text. \cs{prname}\marg{Arabic
% proper name} is provided to that effect:\footnote{Just as \cs{uc},
% \cs{prname} is also able to process proper names consisting of
% several subsequent words.}---
%\iffalse
%<*example>
% \fi
\begin{example}
  From \textcite[i. 23 C]{Wright}:--- If the name following
  \arb[fullvoc]{ibnuN} be that of the mother or the grandfather, the
  \arb[fullvoc]{"a} is retained; as \arb[fullvoc]{`Is_A ibnu maryama},
  \enquote{Jesus the son of Mary}; \arb[fullvoc]{`ammAru ibnu
    man.sUriN}, \enquote{\prname{`ammAr} the (grand)son of
    \prname{man.sUr}}.
\end{example}
% \iffalse
%</example>
% \fi
%
% The following example shows how \cs{prname} can be used in
% conjunction with the \package{nameauth} package to have Arabic
% proper names printed first in full then in partial
% forms:\footnote{See the documentation of \package{nameauth} for more
% details: \url{https://ctan.org/pkg/nameauth}}--- %
%\iffalse
%<*example>
% \fi
\begin{example}
  \begin{nameauth}
    \< Hunayn & \prname{'abU zayd} & \prname{.hunayn}, \prname{{i}bn
      'is.h_aq al-`ibAdiyy} & > %
    \< Razi & \prname{'abU bakr mu.hammad ibn zakariyyA'} &
    \prname{al-rAziyy} & > %
  \end{nameauth}

  On first occurrence, proper names are printed as \Hunayn, \Razi.
  Then as \Hunayn, \Razi.
\end{example}
% \iffalse
%</example>
% \fi
% 
% \begin{quoting}\label{ref:prname-star}
%   \textsc{Rem.} \DescribeMacro{\prname*} \package{arabluatex} also
%   provides \cs{prname*} which only renders in upright roman style
%   already transliterated proper names without applying any further
%   processing. It is mostly used internally and applied to proper
%   names exported in Unicode to an external selected
%   file.\footnote{See below \vref{sec:arabtex2utf} for more details.}
% \end{quoting}
% 
% \subsection{Additional note on \texttt{dmg} convention}
% \label{sec:additional-note-dmg}
% \NEWfeature{v1.3} According to \textcite[6]{dmg}, Arabic
% \arb[trans]{'i`rAb} may be rendered into |dmg| in three different
% ways:
% \begin{enumerate}
% \item \label{ref:dmg-full-rend}In full:
%   \NoArbUp\arb[trans]{\uc{`amruNU}}\ArbUpDflt\,;
% \item \label{ref:dmg-up-rend}As superscript text:
%   \arb[trans]{\uc{`amruNU}}\,;
% \item \label{ref:irab-discarded}Discarded: \arb[trans]{\uc{`amr}}.
% \end{enumerate}
% \DescribeMacro{\arbup} By default, \package{arabluatex} applies rule
% \ref{ref:dmg-up-rend}. Once delimited by a set of Lua functions,
% \arb[trans]{'i`rAb} is passed as an argument on to a \cs{arbup}
% command which is set to \cs{textsuperscript}.
%
% \DescribeMacro{\NoArbUp} \DescribeMacro{\ArbUpDflt} \cs{NoArbUp} may
% be used either in the preamble or at any point of the document in
% case one wishes to apply rule \ref{ref:dmg-full-rend}. The default
% rule \ref{ref:dmg-up-rend} can be set back with \cs{ArbUpDflt} at
% any point of the document.
%
% \DescribeMacro{\SetArbUp} Finally, \cs{SetArbUp}\marg{formatting
% directives} can be used to customize the way \arb[trans]{'i`rAb} is
% displayed. To take one example, here is how Arabic
% \arb[trans]{'i`rAb} may be rendered as subscript text:---
% \iffalse
%<*example>
% \fi
\begin{example}
  \SetArbUp{\textsubscript{#1}}
  Arabic |dmg| transliteration for \arb{ra'aytu ^gAmi`aN
    muhaddamaTaN mi'_danatu-hu}: \arb[trans]{ra'aytu
    ^gAmi`aN muhaddamaTaN mi'_danatu-hu.}
\end{example}
% \iffalse
%</example>
% \fi
%
% As shown in the above example, |#1| is the token that is replaced
% with the actual \arb[trans]{tanwIn} in the formatting directives of
% the \cs{SetArbUp} command.
%
% \paragraph{\texorpdfstring{\arb[trans]{'i`rAb} boundaries}{ʾiʿrāb
% boundaries}}
% Every declinable noun (\arb[trans]{mu`rab}) may be declined either
% with or without \arb[trans]{tanwIn}, viz. \arb[trans]{mun.sarifuN}
% or \arb[trans]{.gayr-u mun.sarifiN}. The former is automatically
% parsed by \package{arabluatex}, whereas the latter has to be
% delimited with an hyphen, like so:---
% \begin{quote}
%   \arb[trans]{\textbf{mun.sarif}}: |mu`allimuN|
%   \arb[voc]{mu`allimuN} \arb[trans]{mu`allimuN}, |kA'inuN|
%   \arb[voc]{kA'inuN} \arb[trans]{kA'inuN}, |kA'inAtuN|
%   \arb[voc]{kA'inAtuN} \arb[trans]{kA'inAtuN}, |\uc{`amraNU}|
%   \arb[voc]{\uc{`amraNU}} \arb[trans]{\uc{`amraNU}}, |fataN_A|
%   \arb[voc]{fataN_A} \arb[trans]{fataN_A}, |qA.diNI| \arb{qA.diNI}
%   \arb[trans]{qA.diNI}.
%
%   \arb[trans]{\textbf{.gayr mun.sarif}}: |al-mu`allim-u|
%   \arb[voc]{al-mu`allim-u} \arb[trans]{al-mu`allim-u}, |kitAb-Ani|
%   \arb[voc]{kitAb-Ani} \arb[trans]{kitAb-Ani}, |ra^sa'-Ani|
%   \arb[voc]{ra^sa'-Ani} \arb[trans]{ra^sa'-Ani}, |sAriq-Una|
%   \arb[voc]{sAriq-Una} \arb[trans]{sAriq-Una}, |qA.d-Una|
%   \arb[voc]{qA.d-Una} \arb[trans]{qA.d-Una}, |al-.zulm-Atu|
%   \arb[voc]{al-.zulm-Atu} \arb[trans]{al-.zulm-Atu}.
% \end{quote}
%
% \begin{quoting}
%   \textsc{Rem.}~\emph{a.} As the \arb[trans]{tanwIn} is passed over
%   in pronunciation when it is followed by the letters
%   \arb[novoc]{r}, \arb[novoc]{l}, \arb[novoc]{m}, \arb[novoc]{w},
%   \arb[novoc]{y} (see \vref{ref:assimilation}), it may be desirable
%   to further distinguish it by putting it above the line, but not to
%   do the same for \arb[trans]{.gayr mun.sarif} terminations. This
%   can be achieved by simply omitting the hyphen before any
%   \arb[trans]{.gayr mun.sarif} termination:---\\
%   |kAna| |.ganiyyaN| |l_akinna-hu| |labisa| |^gubbaTaN| |mumazzaqaN|
%   |'aydu-hA| \arb[voc]{kAna .ganiyyaN l_akinna-hu labisa ^gubbaTaN
%   mumazzaqaN 'aydu-hA} \arb[trans]{kAna .ganiyyaN l_akinna-hu labisa
%   ^gubbaTaN mumazzaqaN 'aydu-hA}.
%
%   \textsc{Rem.}~\emph{b.} Although the hyphen before the
%   \arb[trans]{tanwIn} is optional as \package{arabluatex} always
%   parses nouns with such termination, it may also be used to mark
%   better the inflectional endings:---\\
%   |mana`a| |'l-nAs-a| |kAffaT-aN| |min| |mu_hA.tabati-hi|
%   |'a.had-uN| |bi-sayyidi-nA| \arb[voc]{mana`a 'l-nAs-a kAffaT-aN
%   min mu_hA.tabati-hi 'a.had-uN bi-sayyidi-nA} \arb[trans]{mana`a
%   'l-nAs-a kAffaT-aN min mu_hA.tabati-hi 'a.had-uN bi-sayyidi-nA}.
% \end{quoting}
% 
% \paragraph{\texorpdfstring{Discar\-ding the
% \arb[trans]{'i`rAb}}{Discarding the ʾiʿrāb}}
% \label{ref:ta-marbutah-pipe}
% As said above (\vref{ref:irab-discarded}), the \arb[trans]{'i`rAb}
% may be discarded in some cases, as in transliterated proper names or
% book titles. \package{arabluatex} is able to render words ending
% with \arb[trans]{tA' marbU.taT} in different ways, depending on
% their function:---
% \begin{enumerate}
% \item Nouns followed by an adjective in apposition: |madInaT|
%   |kabIraT| \arb[trans]{madInaT kabIraT}, |al-madInaT| |al-kabIraT|
%   \arb[trans]{al-madInaT al-kabIraT}.
% \item Nouns followed by another noun in the genitive (contruct
%   state): |.hikmaT| |al-l_ah| \arb[trans]{.hikmaT| \uc{al-l_ah}},
%   |fi.d.daT| |al-darAhim| \arb[trans]{fi.d.daT al-darAhim}.
% \end{enumerate}
% \begin{quoting}
%   \textsc{Rem.} It may so happen, as in the absence of the article
%   before the annexed word, that \package{arabluatex} be unable to
%   determine which of the above two cases the word ending with
%   \arb[trans]{tA' marbU.taT} falls into. The \enquote*{pipe}
%   character (see \vref{sec:pipe}) may be appended to that word to
%   indicate that what follows is in the construct state:
%   |\uc{r}isAlaT| |fI| |tartIb| \verb+qirA'aT|+ |kutub|
%   |\uc{^g}AlInUs| \arb[trans]{\uc{r}isAlaT fI tartIb qirA'aT|
%   kutub \uc{^g}AlInUs}.
% \end{quoting}
% 
%
% \paragraph{Uncertain short vowels}
% In some printed books, it may happen that more than one short vowel
% be placed on a consonant in cases where the vocalization is
% uncertain or ambiguous, like so: \arb[voc]{fa`uaila}. In
% transliteration, the uncertain vowels go between slashes and are
% separated by commas: |fa`uaila| \arb[voc]{fa`uaila}
% \arb[trans]{fa`uaila}.
%
% \subsection{Examples}
% \label{sec:examples-translit}
% Here follows in transliteration the story of
% \arb[trans]{\uc{ju.hA}} and his donkey (\arb[voc]{\uc{ju.hA
% wa-.himAru-hu}}). See the code \vpageref{ref:juha-code}:---
%
% \SetTranslitConvention{dmg}
% \begin{arab}[trans]
%   \LR{\textbf{\emph{\enquote*{dmg}} standard:}} 'at_A .sadIquN 'il_A
%   \uc{ju.hA} ya.tlubu min-hu .himAra-hu li-yarkaba-hu fI safraTiN
%   qa.sIraTiN fa-qAla la-hu: \enquote{sawfa 'u`Idu-hu 'ilay-ka fI
%   'l-masA'-i wa-'adfa`u la-ka 'ujraTaN.} fa-qAla \uc{ju.hA}:
%   \enquote{'anA 'AsifuN jiddaN 'annI lA 'asta.tI`u 'an 'u.haqqiqa
%   la-ka ra.gbata-ka fa-'l-.himAr-u laysa hunA 'l-yawm-a.}  wa-qabla
%   'an yutimma \uc{ju.hA} kalAma-hu bada'a 'l-.himAr-u yanhaqu fI
%   'i.s.tabli-hi. fa-qAla la-hu .sadIqu-hu: \enquote{'innI 'asma`u
%   .himAra-ka yA \uc{ju.hA} yanhaqu.} fa-qAla la-hu \uc{ju.hA}:
%   \enquote{.garIbuN 'amru-ka yA .sadIqI 'a-tu.saddiqu 'l-.himAr-a
%   wa-tuka_d_diba-nI?}
% \end{arab}
% 
% \SetTranslitConvention{loc}
% \begin{arab}[trans]
%   \LR{\textbf{\emph{\enquote*{loc}} standard:}} 'at_A .sadIquN 'il_A
%   \uc{ju.hA} ya.tlubu min-hu .himAra-hu li-yarkaba-hu fI safraTiN
%   qa.sIraTiN fa-qAla la-hu: \enquote{sawfa 'u`Idu-hu 'ilay-ka fI
%   'l-masA'-i wa-'adfa`u la-ka 'ujraTaN.} fa-qAla \uc{ju.hA}:
%   \enquote{'anA 'AsifuN jiddaN 'annI lA 'asta.tI`u 'an 'u.haqqiqa
%   la-ka ra.gbata-ka fa-'l-.himAr-u laysa hunA 'l-yawm-a.}  wa-qabla
%   'an yutimma \uc{ju.hA} kalAma-hu bada'a 'l-.himAr-u yanhaqu fI
%   'i.s.tabli-hi. fa-qAla la-hu .sadIqu-hu: \enquote{'innI 'asma`u
%   .himAra-ka yA \uc{ju.hA} yanhaqu.} fa-qAla la-hu \uc{ju.hA}:
%   \enquote{.garIbuN 'amru-ka yA .sadIqI 'a-tu.saddiqu 'l-.himAr-a
%   wa-tuka_d_diba-nI?}
% \end{arab}
% \SetTranslitConvention{dmg}
%
% \SetTranslitConvention{arabica}
% \begin{arab}[trans]
%   \LR{\textbf{\emph{\enquote*{arabica}} standard:}} 'at_A .sadIquN
%   'il_A \uc{ju.hA} ya.tlubu min-hu .himAra-hu li-yarkaba-hu fI
%   safraTiN qa.sIraTiN fa-qAla la-hu: \enquote{sawfa 'u`Idu-hu
%   'ilay-ka fI 'l-masA'-i wa-'adfa`u la-ka 'ujraTaN.} fa-qAla
%   \uc{ju.hA}: \enquote{'anA 'AsifuN jiddaN 'annI lA 'asta.tI`u 'an
%   'u.haqqiqa la-ka ra.gbata-ka fa-'l-.himAr-u laysa hunA 'l-yawm-a.}
%   wa-qabla 'an yutimma \uc{ju.hA} kalAma-hu bada'a 'l-.himAr-u
%   yanhaqu fI 'i.s.tabli-hi. fa-qAla la-hu .sadIqu-hu: \enquote{'innI
%   'asma`u .himAra-ka yA \uc{ju.hA} yanhaqu.} fa-qAla la-hu
%   \uc{ju.hA}: \enquote{.garIbuN 'amru-ka yA .sadIqI 'a-tu.saddiqu
%   'l-.himAr-a wa-tuka_d_diba-nI?}
% \end{arab}
% \SetTranslitConvention{dmg}
%
% \section{Buckwalter input scheme}
% \label{sec:buckwalter-scheme}
% \NEWfeature{v1.4} Even though \package{arabluatex} is primarily
% designed to process the Arab\TeX\ notation, it can also process the
% Buckwalter input scheme to a large extent.\footnote{See
% \url{http://www.qamus.org/transliteration.htm}} The Buckwalter
% scheme is actually processed in two steps, as it is first converted
% into Arab\TeX. Then, once this is accomplished, the Arab\TeX\ scheme
% is processed through the above described functions. In this way, the
% Buckwalter input scheme can make the most of the
% \package{arabluatex} special features that are presented in
% \vref{sec:options}.
%
% \DescribeMacro{\SetInputScheme} The input scheme, which is set to
% |arabtex| by default, may be changed at any point of the document by
% the \cs{SetInputScheme}\marg{scheme} command, where \meta{scheme}
% may be either |arabtex| or |buckwalter|. This command is also
% accepted in the preamble should one wish to set the input scheme
% globally, like so:---
% \iffalse
%<*example>
% \fi
\begin{code}
  \usepackage{arabluatex}
  \SetInputScheme{buckwalter}
\end{code}
% \iffalse
%</example>
% \fi
%
% \paragraph{\texorpdfstring{\enquote*{base}, \enquote*{\texttt{xml}}
% and \enquote*{safe} schemes}{‘base’, ‘xml’ and ‘safe’ schemes}}
% \package{arabluatex} can use any of the so-called Buckwalter
% \enquote*{base}, \enquote*{\texttt{xml}} or \enquote*{safe} schemes
% as they are described in \textcite[25--26]{Habash}.\footnote{I am
% grateful to Graeme Andrews who suggested that the \enquote*{safe}
% scheme be included in \package{arabluatex}.} However, the following
% limitation apply to the \enquote*{base} and \enquote*{\texttt{xml}}
% schemes: the braces |{| and |}|, which are used to encode
% \arb[novoc]{"a} and \arb[novoc]{y"'}, must be replaced with square
% brackets viz. |[| and |]| respectively.
%
% It is therefore recommended to use the Buckwalter \enquote*{safe}
% scheme.
%
% \Cref{tab:buckwalter-scheme} gives the Buckwalter equivalents that
% are currently used by \package{arabluatex}. The additional
% characters that are defined in \vref{tab:additional-arabic-codings}
% are also available.
%
% \enlargethispage{\baselineskip}
% \begin{longtable}{llllll}
% \bottomrule
% \caption*{\Cref*{tab:buckwalter-scheme}: Buckwalter scheme}
% \endfoot
% \captionlistentry{Buckwalter scheme}\\[-1em]
% \toprule
% Letter & \multicolumn{3}{l}{Transliteration\footnotemark}
% & \multicolumn{2}{l}{Buckwalter notation} \\
% & \texttt{dmg} & \texttt{loc} & \texttt{arabica} & |base/xml| &
% |safe| \\ \midrule
% \endfirsthead
% \toprule
% Letter & \multicolumn{3}{l}{Transliteration}
% & \multicolumn{2}{l}{Buckwalter notation} \\
% & \texttt{dmg} & \texttt{loc} & \texttt{arabica} & |base/xml| &
% |safe| \\ \midrule
% \endhead \footnotetext{See \vref{sec:transliteration}.}
% \label{tab:buckwalter-scheme}
% \arb[novoc]{a} & \dmg{a} & \loc{a} & \brill{a} & |A| & |A| \\
% \arb[novoc]{b} & \dmg{b} & \loc{b} & \brill{b} & |b| & |b| \\
% \arb[novoc]{t} & \dmg{t} & \loc{t} & \brill{t} & |t| & |t| \\
% \arb[novoc]{_t} & \dmg{_t} & \loc{_t} & \brill{_t} & |v| & |v| \\
% \arb[novoc]{j} & \dmg{j} & \loc{j} & \brill{j} & |j| & |j| \\
% \arb[novoc]{.h} & \dmg{.h} & \loc{.h} & \brill{.h} & |H| & |H| \\
% \arb[novoc]{x} & \dmg{x} & \loc{x} & \brill{x} & |x| & |x| \\
% \arb[novoc]{d} & \dmg{d} & \loc{d} & \brill{d} & |d| & |d| \\
% \arb[novoc]{_d} & \dmg{_d} & \loc{_d} & \brill{_d} & |*| & |V| \\
% \arb[novoc]{r} & \dmg{r} & \loc{r} & \brill{r} & |r| & |r| \\
% \arb[novoc]{z} & \dmg{z} & \loc{z} & \brill{z} & |z| & |z| \\
% \arb[novoc]{s} & \dmg{s} & \loc{s} & \brill{s} & |s| & |s| \\
% \arb[novoc]{^s} & \dmg{^s} & \loc{^s} & \brill{^s} & |$| & |c| \\
% \arb[novoc]{.s} & \dmg{.s} & \loc{.s} & \brill{.s} & |S| & |S| \\
% \arb[novoc]{.d} & \dmg{.d} & \loc{.d} & \brill{.d} & |D| & |D| \\
% \arb[novoc]{.t} & \dmg{.t} & \loc{.t} & \brill{.t} & |T| & |T| \\
% \arb[novoc]{.z} & \dmg{.z} & \loc{.z} & \brill{.z} & |Z| & |Z| \\
% \arb[novoc]{`} & \dmg{`} & \loc{`} & \brill{`} & |E| & |E| \\
% \pagebreak[2]
% \arb[novoc]{.g} & \dmg{.g} & \loc{.g} & \brill{.g} & |g| & |g| \\
% \arb[novoc]{f} & \dmg{f} & \loc{f} & \brill{f} & |f| & |f| \\
% \arb[novoc]{q} & \dmg{q} & \loc{q} & \brill{q} & |q| & |q| \\
% \arb[novoc]{k} & \dmg{k} & \loc{k} & \brill{k} & |k| & |k| \\
% \arb[novoc]{l} & \dmg{l} & \loc{l} & \brill{l} & |l| & |l| \\
% \arb[novoc]{m} & \dmg{m} & \loc{m} & \brill{m} & |m| & |m| \\
% \arb[novoc]{n} & \dmg{n} & \loc{n} & \brill{n} & |n| & |n| \\
% \arb[novoc]{h} & \dmg{h} & \loc{h} & \brill{h} & |h| & |h| \\
% \arb[novoc]{w} & \dmg{w} & \loc{w} & \brill{w} & |w| & |w| \\
% \arb[novoc]{y} & \dmg{y} & \loc{y} & \brill{y} & |y| & |y| \\
% \arb[novoc]{Y} & \dmg{Y} & \loc{Y} & \brill{Y} & |Y| & |Y| \\
% \arb[novoc]{T} & \dmg{aT} & \loc{aT} & \brill{aT} & |p| & |p| \\
% \midrule
% \arb[novoc]{|"'} & \dmg{|"'} & \loc{|"'} & \brill{|"'} & \verb|'| & |C| \\
% \arb[novoc]{A"'} & \dmg{A"'} & \loc{A"'} & \brill{A"'} & \verb+|+ & |M| \\
% \arb[novoc]{a"'} & \dmg{a"'} & \loc{a"'} & \brill{a"'} & \verb|>| & |O| \\
% \arb[novoc]{w"'} & \dmg{w"'} & \loc{w"'} & \brill{w"'} & \verb|&| & |W| \\
% \arb[novoc]{i"'} & \dmg{i"'} & \loc{i"'} & \brill{i"'} & \verb|<| & |I| \\
% \arb[novoc]{y"'} & \dmg{y"'} & \loc{y"'} & \brill{y"'} & \verb|]| & |Q| \\
% \midrule
% \arb[novoc]{BB} & --- & --- & --- & \verb|~| & |~| \\
% \arb[novoc]{"a} & ' & ' & --- & |[| & |L| \\
% \midrule
% \arb[voc]{Ba} & \dmg{Ba} & \loc{Ba} & \brill{Ba} & \verb|a| & |a| \\
% \arb[voc]{Bu} & \dmg{Bu} & \loc{Bu} & \brill{Bu} & \verb|u| & |u| \\
% \arb[voc]{Bi} & \dmg{Bi} & \loc{Bi} & \brill{Bi} & \verb|i| & |i| \\
% \arb[voc]{BaN} & \dmg{BaN} & \loc{BaN} & \brill{BaN} & \verb|F| & |F| \\
% \arb[voc]{BuN} & \dmg{BuN} & \loc{BuN} & \brill{BuN} & \verb|N| & |N| \\
% \arb[voc]{BiN} & \dmg{BiN} & \loc{BiN} & \brill{BiN} & \verb|K| & |K| \\
% \arb[voc]{B"} & --- & --- & --- & \verb|o| & |o| \\
% \midrule
% \arb[novoc]{B_a} & \dmg{B_a} & \loc{B_a} & \brill{B_a} &  |`| & |e| \\
% \midrule
% \arb[novoc]{--} (\arb[trans]{ta.twIl}) & --- & --- & --- & |_| & |_| \\
% \end{longtable}
%
% \paragraph{Transliteration}
% The Buckwalter notation can also be transliterated into any accepted
% romanization standard of Arabic. See above
% \vref{sec:transliteration} for more information. However, it should
% be pointed out again that only accurate coding produces accurate
% transliteration. It is therefore at the very least highly advisable
% to use the hyphen for tying the definite article and the inseparable
% particles (viz. prepositions, adverbs and conjunctions) to words,
% like so:--- \SetInputScheme{buckwalter}
% \begin{quote}
%   |Al-EaAlamu| \arb{Al-EaAlam-u} \arb[trans]{Al-EaAlam-u},
%   |Al-camsu| \arb{Al-cams-u} \arb[trans]{Al-cams-u},
%   |bi-SinaAEapi| |Al-T~ib~i|, \arb{bi-SinaAEap-i Al-T~ib~-i}
%   \arb[trans]{bi-SinaAEap-i Al-T~ib~-i}.
%
%   |wa-Al-l~ehi| \arb{wa-Al-l~eh-i} \arb[trans]{wa-Al-l~eh-i},
%   |Al-Hamdu| |li-l~ehi| \arb{Al-Hamd-u li-l~eh-i}
%   \arb[trans]{Al-Hamd-u li-l~eh-i}.
% \end{quote}
% \SetInputScheme{arabtex}
%
% Similary, it is not advisable to use \verb+|+ and |[|
% (\enquote*{base} and \enquote*{\texttt{xml}} schemes) or |M| and |L|
% (\enquote*{safe} scheme) to encode the \arb[trans]{'alif-u
% 'l-mamdUdaT-i} and the \arb[trans]{'alif-u 'l-wa.sl-i} for such
% signs are supposed to be generated by \package{arabluatex} internal
% functions.  Besides, as they do not \emph{per se} convey any
% morphological information on what they are derived from, they cannot
% be transliterated accurately. To take one example, %
% \SetInputScheme{buckwalter}%
% |<ilY Al-LntiqaADi| gives \arb{>ilY Al-LntiqaADi} as expected, but
% only |<ilY Al-intiqADi| can be transliterated as \arb[trans]{<ilY
% Al-intiqaADi} with the correct vowel \meta{i} in place of the %
% \SetInputScheme{arabtex}%
% \arb[trans]{'alif-u 'l-wa.sl-i}.
%
% \section{Unicode Arabic input}
% \label{sec:unicode-input}
% \NEWfeature{v1.5} As said above in \vref{sec:buckwalter-scheme}
% about the Buckwalter input scheme, even though \package{arabluatex}
% is primarily designed to process the Arab\TeX\ notation, it also
% accepts Unicode Arabic input. It should be noted that
% \package{arabluatex} does in no way interfere with Unicode Arabic
% input: none of the |voc|, |fullvoc|, |novoc| or |trans| options will
% have any effect on plain Unicode Arabic for the time being.
%
% That said, there are two ways of inserting Unicode
% Arabic:
% \begin{enumerate}
% \item \DescribeMacro{\txarb} The \cs{txarb}\marg{Unicode Arabic}
%   command for inserting Unicode Arabic text in paragraphs;
% \item The \DescribeEnv{txarab} |txarab| environment for inserting
%   running paragraphs of Arabic text, like so:---\\%
% \iffalse
%<*example>
% \fi
\begin{code}
  \begin{txarab}
    <Unicode Arabic text>
  \end{txarab}
\end{code}
% \iffalse
%</example>
% \fi
% \end{enumerate}
%
% \needspace{4\baselineskip}
% \section{\LaTeX\ Commands in Arabic environments}
% \label{sec:commands-in-arb}
% \paragraph{General principle} \label{ref:cmd-inside-arabic}\LaTeX\
% commands are accepted in Arabic environments. The general principle
% which applies is that any single-argument command with up to
% \emph{two optional arguments}---that is:
% \cs{command}\oarg{opt1}\oarg{opt2}\marg{arg}---such as
% \cs{emph}\marg{text}, \cs{textbf}\marg{text} and the like, is
% assumed to have Arabic text in its mandatory argument:---
% \begin{quote}
%   |\abjad{45}| |kitAbu-hu| |\emph{fI| |'l-\uc{`AdAt-i}}|
%   \arb[voc]{\abjad{45} kitAbu-hu \emph{fI 'l-\uc{`AdAt-i}}}
%   \arb[trans]{\abjad{45} kitAbu-hu \emph{fI
%   'l-\uc{`AdAt-i}}}.\footnote{This is odd in Arabic script, but
%   using such features as \cs{emph} or \cs{textbf} is a matter of
%   personal taste.}
%   \NewDocumentCommand{\rlframebox}{o o m}{
%    \IfNoValueTF{#2}{\IfNoValueTF{#1}{
%    \framebox{\setRL#3}}{\framebox[#1]{\setRL#3}}
% }{\framebox[#1][#2]{\setRL#3}}}
%
%   |\arb{\abjad{45} \rlframebox[1in][s]{kitAbu-hu fI 'l-`AdAti}}|\\
%   \arb{\abjad{45} \rlframebox[1in][s]{kitAbu-hu fI
%   'l-`AdAti}}\,\footnote{\cs{rlframefox} has been adapted from
%   \cs{framebox} for insertions of right-to-left text.}
% \end{quote}
% 
% The same applies to footnotes:---
% \iffalse
%<*example>
% \fi
\begin{example}
  \renewcommand{\footnoterule}%
  {\hfill\noindent\rule[1mm]{.4\textwidth}{.15mm}}
  \begin{arab}
    'inna 'abI kAna mina 'l-muqAtilaT-i\footnote{al-muqAtilaT-i:
      al-muqAtil-Ina.}, wa-kAnat 'ummI min `u.zamA'-i buyUt-i
    'l-zamAzimaT-i\footnote{al-zamAzimaT-u: .tA'ifaT-u mina
      'l-furs-i.}.
  \end{arab}
\end{example}
% \iffalse
%</example>
% \fi
%
% Some commands, however, do not expect running text in their
% arguments, or one may wish to insert English text e.g. in footnotes
% or in marginal notes. \package{arabluatex} provides a set of
% commands to handle such cases.
%
% \DescribeMacro{\LR} \cs{LR}\marg{arg} is designed to typeset its
% argument from left to right. It may be used in an Arabic
% environment, either \cs{arb}\marg{Arabic text} or \cs{begin}|{arab}|
% \meta{Arabic text} \cs{end}|{arab}|, for short insertions of
% left-to-right text, or to insert any \LaTeX\ command that would
% otherwise be rejected by \package{arabluatex}, such as commands the
% argument of which is expected to be a dimension or a unit of
% measurement.
%
% \DescribeMacro{\RL} \cs{RL}\marg{arg} does the same as
% \cs{LR}\marg{arg}, but typesets its argument from right to left. Even
% in an Arabic environment, this command may be useful.
% 
% \DescribeMacro{\LRfootnote} \DescribeMacro{\RLfootnote}
% \cs{LRfootnote}\marg{text} and \cs{RLfootnote}\marg{text} typeset
% left-to-right and right-to-left footnotes respectively in Arabic
% environments. Unlike \cs{footnote}\marg{text}, the arguments of both
% \cs{LRfootnote} and \cs{RLfootnote} are not expected to be Arabic
% text. For example, \cs{LRfootnote} can be used to insert English
% footnotes in running Arabic text:---
% \iffalse
%<*example>
% \fi
\begin{example}
  \begin{arab}[fullvoc]
    \uc{z}ayd-uN\arbnull{ibnu}\LRfootnote{%
      \enquote{\arb[trans]{\uc{z}ayd} is the son of
        \arb[trans]{\uc{`a}mr}}: the second noun is not in
      apposition to the first, but forms part of the
      predicate\ldots} \arbnull{zayduN}ibn-u \uc{`a}mr-iNU
  \end{arab}
\end{example}
% \iffalse
%</example>
% \fi
%
% When footnotes are typeset from right to left, it may happen that
% the numbers of the footnotes that are at the bottom of the page be
% typeset in the wrong direction. For example, instead of an expected
% number 18, one may get 81. \package{arabluatex} is not responsible
% for that, but should it happen, it may be necessary to redefine in
% the preamble the \LaTeX\ macro \cs{thefootnote} like so:---\\
% \tcboxverb{\renewcommand*{\thefootnote}{\textsuperscript{\LR{\arabic{footnote}}}}}
% \DescribeMacro{\FixArbFtnmk} Another solution is to put in the
% preamble, below the line that loads \package{arabluatex}, the
% \cs{FixArbFtnmk} command. However, for more control over the layout
% of footnotes marks, it is advisable to use the \package{scrextend}
% package.\footnote{See \url{http://ctan.org/pkg/koma-script}; read
% the documentation of \package{KOMA-script} for details about the
% \cs{deffootnotemark} and \cs{deffootnote} commands.}
%
% \DescribeMacro{\LRmarginpar} The
% \cs{LRmarginpar}\oarg{left}\marg{right} command does for marginal
% notes the same as \cs{LRfootnote} does for footnotes. Of course, it
% is supposed to be used in Arabic environments. Note that
% \cs{marginpar} also works in Arabic environments, but it acts as any
% other single-argument command inserted in Arabic environments. The
% general principle laid \vpageref{ref:cmd-inside-arabic} applies.
%
% \label{ref:setrl-setlr}
% \DescribeMacro{\setRL} \DescribeMacro{\setLR} \cs{setRL} and
% \cs{setLR} can be used to change the direction of paragraphs, either
% form left to right or from right to left. As an example, an
% easy way to typeset a right-to-left sectional title follows:---
% \iffalse
%<*example>
% \fi
\begin{example}
  \setRL
  \section*{\arb{barzawayhi li-buzurjumihra bn-i 'l-buxtikAni}}
  \setLR
  \begin{arab}
    qAla barzawayhi bn-u 'azhar-a, ra's-u 'a.tibbA'-i fAris-a...
  \end{arab}
\end{example}
% \iffalse
%</example>
% \fi
%
% \subsection{New commands}
% \label{sec:declare-new-commands}
% \NEWfeature{v1.9}%
% In some particular cases, it may be useful to define new commands to
% be inserted in Arabic environments. From the general principle laid
% \vpageref{ref:cmd-inside-arabic}, it follows that any command that
% is found inside an Arabic environment is assumed to have Arabic text
% in its argument which \package{arabluatex} will process as such
% before passing it on to the command itself for any further
% processing. As a result of this feature, such a command as:\\
% \tcboxverb{\newcommand{\fvarabic}[1]{\arb[fullvoc]{#1}}}\\
% will work as expected, but will always output non-vocalized Arabic
% if it is inserted in a |novoc| Arabic environment because its
% argument will have been processed by the |novoc| rules before the
% command |\fvarabic| itself can see it.
%
% \DescribeMacro{\MkArbBreak} The \cs{MkArbBreak}\marg{csv list of
% commands} command can be used in the preamble to give any
% command---either new or already existing---the precedence over
% \package{arabluatex} inside Arabic environments. It takes as
% argument a comma-separated list of commands each of which must be
% stripped of its leading character
% |\|, like so:---\\
% \tcboxverb{\MkArbBreak{onecmd, anothercmd, yetanothercmd, ...}}
% 
% For example, here follows a way to define a new command |\fvred| to
% distinguish words with a different color and always print them in
% fully vocalized Arabic:--- %
% \iffalse
%<*example>
% \fi
\begin{example}
  \MkArbBreak{fvred}
  \newcommand{\fvred}[1]{\arbcolor[red]{\arb[fullvoc]{#1}}}
  \begin{arab}[voc]
    _tumma "intalaqa _dU 'l-qarn-ayni 'il_A 'ummaT-iN 'u_hr_A fI
    \fvred{((ma.tli`-i 'l-^sams-i))} wa-lA binA'-a la-hum
    yu'amminu-hum mina 'l-^sams-i.
  \end{arab}
\end{example}
% \iffalse
%</example>
% \fi
%
% It must be noted that the arguments, either optional or mandatory,
% of commands declared with \cs{MkArbBreak} are not to be processed by
% \package{arabluatex}. Therefore, as in the previous example, any of
% their argument to be rendered in Arabic must be inserted again in
% \cs{arb}. \NEWfeature{v1.12}These commands themselves may have up to
% two optional and/or mandatory arguments followed by one optional
% argument, like so:---
% \begin{enumerate}
% \item \cs{command} (no argument, lowermost combination)
% \item \cs{command}\oarg{opt1} (one optional argument)
% \item \cs{command}\marg{arg1} (one mandatory argument)
% \item \cs{command}\oarg{opt1}\marg{arg1} (one optional and one
%   mandatory argument)
% \item{} [\ldots]
% \item \cs{command}\oarg{opt1}\oarg{opt2}\marg{arg1}\marg{arg2}
% \item
%   \cs{command}\oarg{opt1}\oarg{opt2}\marg{arg1}\marg{arg2}\oarg{opt3}
%   (uppermost combination)
% \end{enumerate}
%
% \DescribeMacro{\MkArbBreak*}\NEWfeature{v1.12} As said above,
% \cs{MkArbBreak} prevents \package{arabluatex} from processing the
% arguments of \enquote*{declared} commands as Arabic text. This
% technique proves sufficient in most cases. However, a
% \enquote*{starred} version of this
% command---\cs{MkArbBreak*}\marg{csv list of commands}---is also
% provided. It goes a step further, as it directs \package{arabluatex}
% to \emph{close} the current Arabic environment before any of the
% \enquote*{declared} commands, then \emph{resume} it just after.
% 
% \iffalse
%<*example>
% \fi
\begin{tcblisting}{text only}
  It must be noted that \cs{MkArbBreak*} must be used with the utmost
  care and \emph{should never be used} if \cs{MkArbBreak} gives
  satisfaction. At any rate, the latter must always be tested before
  the former.
\end{tcblisting}
% \iffalse
%</example>
% \fi%
%
% \subsection{Environments}
% \label{sec:environments}
% \changes{v1.5}{2016/11/14}{Environments may be nested inside the
%   \texttt{arab} environment}
% \NEWfeature{v1.5} Environments such as
% \tcboxverb{\begin{quote} ... \end{quote}} may be nested inside the
% |arab| environment. Up to one optional argument may be passed to
% each nested environment, like so:---
% \iffalse
%<*example>
% \fi
\begin{code}
  \begin{arab}
    \begin{<environment>}[<options>]
      <Arabic text>
    \end{<environment>}
  \end{arab}
\end{code}
% \iffalse
%</example>
% \fi
%
% In the following example, the \package{quoting} package is used:---
% \iffalse
%<*example>
% \fi
\begin{example}
  \setquotestyle{arabic}
  \begin{arab}[fullvoc]
    kAna \uc{'abU} \uc{'l-hu_dayli} 'ahd_A 'il_A \uc{muwaysiN}
    dajAjaTaN. wa-kAnat dajAjatu-hu 'llatI 'ahdA-hA dUna mA kAna
    yuttaxa_du li-\uc{muwaysiN}. wa-l_akinna-hu bi-karami-hi
    wa-bi-.husni xuluqi-hi 'a.zhara 'l-ta`ajjuba min simani-hA
    wa-.tIbi la.hmi-hA. wa-kAna <\uc{'abU} \uc{'l-hu_dayli}>
    yu`rafu bi-'l-'imsAki 'l-^sadIdi. fa-qAla: \enquote{wa-kayfa
      ra'ayta yA \uc{'abA} \uc{`imrAna} tilka 'l-dajAjaTa?} qAla:
    \enquote{kAnat `ajabaN mina 'l-`ajabi!}  fa-yaqUlu:
    \begin{quoting}[begintext=\textquotedblright,
      endtext=\textquotedblleft]
      wa-tadrI mA jinsu-hA? wa-tadrI mA sinnu-hA?  fa-'inna
      'l-dajAjaTa 'inna-mA ta.tIbu bi-'l-jinsi wa-'l-sinni.
      wa-tadrI bi-'ayyi ^say'iN kunnA nusamminu-hA? wa-fI 'ayyi
      makAniN kunnA na`lifu-hA?
    \end{quoting}
    fa-lA yazAlu fI h_a_dA wa-'l-'A_haru ya.d.haku .da.hkaN
    na`rifu-hu na.hnu wa-lA ya`rifu-hu \uc{'abU} \uc{'l-hu_dayli}.
  \end{arab}
\end{example}
% \iffalse
%</example>
% \fi
%
% \subsubsection{Lists}
% \label{sec:lists}
% Lists environments are also accepted inside the |arab|
% environment. One may either use any of the three standard list
% environments, viz. |itemize|, |enumerate| and |description| or use
% packages that provide additional refinements such as
% \package{paralist} or \package{enumitem}.
%
% To take a first example, should one wish to typeset a list of
% manuscripts, the |description| environment can be used like so:---
% \iffalse
%<*example>
% \fi
\begin{example}
  \setRL\paragraph{\arb[novoc]{rumUzi 'l-kitAbi}}\setLR
  \begin{arab}[novoc]
    \begin{description}
    \item[b] max.tU.tu 'l-maktabaTi 'l-'ahliyyaTi bi-\uc{bArIs} 2860
      `arabiyyuN.
    \item[s] max.tU.tu 'l-maktabaTi 'l-'ahliyyaTi bi-\uc{bArIs} 2859
      `arabiyyuN.
    \item[m] max.tU.tu majlisi \arb[novoc]{^sUrAY malY} .tahrAna 521.
    \end{description}
  \end{arab}
\end{example}
% \iffalse
%</example>
% \fi
%
% As a second example, the contents of a treatise may be typeset with
% the standard list environments, like so:---
% \iffalse
%<*example>
% \fi
\begin{example}
  \setRL\centerline{\arb{\textbf{al-qAnUnu fI 'l-.tibbi}}}\setLR
  \begin{arab}
    \begin{itemize}
    \item \textbf{al-fannu 'l-'awwalu} fI .haddi 'l-.tibbi
      wa-maw.dU`Ati-hi mina 'l-'umUri 'l-.tabI`iyyaTi wa-ya^stamilu
      `al_A sittaTi ta`AlImiN
      \begin{itemize}
        \item \textbf{al-ta`lImu 'l-'awwalu} [wa-huwa fa.slAni]
          \begin{itemize}
          \item \textbf{al-fa.slu 'l-'awwalu}
          \end{itemize}
      \end{itemize}
    \end{itemize}
  \end{arab}
\end{example}
% \iffalse
%</example>
% \fi
%
% \label{ref:abjad-list}
% As a third example, abjad-numbered lists can be typeset in
% conjunction with the \package{enumitem} package,\footnote{See the
% documentation of \package{enumitem} for more details:
% \url{https://ctan.org/pkg/enumitem}} like so:---
% \iffalse
%<*example>
% \fi
\begin{code}
  % preamble:---
  \usepackage{enumitem}
  \newlist{enumabjad}{enumerate}{10}
  \setlist[enumabjad]{nosep, label={\abjad{\arabic*}}}
  \usepackage{multicol}
\end{code}
\begin{example}
  From \textcite[i. 29 B--C]{Wright}:--- The derived forms of the
  triliteral verb are usually reckoned fifteen in number, but the
  learner may pass over the last four, because (with the exception
  of the twelfth) they are of very rare occurrence.
  \RLmulticolcolumns
  \begin{multicols}{3}
    \begin{arab}[fullvoc]
      \begin{enumabjad}
      \item fa`ala
      \item fa``ala
      \item fA`ala
      \item 'af`ala
      \item tafa``ala
      \item tafA`ala
      \item infa`ala
      \item ifta`ala
      \item if`alla
      \item istaf`ala
      \item if`Alla
      \item if`aw`ala
      \item if`awwala
      \item if`anlala
      \item if`anl_A
      \end{enumabjad}
    \end{arab}
  \end{multicols}
\end{example}
% \iffalse
%</example>
% \fi
% 
% \paragraph{Caveat}
% The various French definition files of the \package{babel} package
% viz. |acadian|, |canadien|, |francais|, |frenchb| or |french| all
% redefine the list environments, which breaks the standard definition
% file that is used by \package{arabluatex}. Therefore,
% \package{babel-french} must be loaded with the |StandardLists=true|
% option, like so:---%
% \iffalse
%<*example>
% \fi
\begin{code}
  \usepackage[french]{babel}
  \frenchsetup{StandardLists=true}
\end{code}
% \iffalse
%</example>
% \fi%
% This option will prevent \package{babel-french} from interfering
% with the layout of the document. Then the \package{paralist} or
% \package{enumitem} packages can be used to make the lists
% \enquote*{compact} as \package{babel-french} do.
%
% \subsection{\package{csquotes}}
% \label{sec:csquotes}
% The recommended way of inserting quotation marks in running Arabic
% text is to use \package{csquotes}. With the help of the
% \cs{DeclareQuoteStyle} command, one can define an Arabic style, like
% so:---
% \iffalse
%<*example>
% \fi
\begin{code}
  \usepackage{csquotes}
  \DeclareQuoteStyle{arabic}
  {\textquotedblright}{\textquotedblleft}
  {\textquoteright}{\textquoteleft}
\end{code}
% \iffalse
%</example>
% \fi
% Then, use this newly defined style with \cs{setquotestyle}, like so:---
% \iffalse
%<*example>
% \fi
\begin{example}
  \setquotestyle{arabic}
  \begin{arab}
    fa-qAla la-hu ju.hA: \enquote{.garIb-uN 'amru-ka yA .sadIqI
      'a-tu.saddiqu 'l-.himAr-a wa-tuka_d_diba-nI?}
  \end{arab}
  \setquotestyle{english}
\end{example}
% \iffalse
%</example>
% \fi
% \begin{quoting}
%   \textsc{Rem.} Do not forget to set back the quoting style to its
%   initial state once the Arabic environment is closed. See the last
%   line in the code above.
% \end{quoting}
%
% \subsection{Two-argument special commands}
% \label{sec:two-arg-cmds}
%
% \paragraph{textcolor}
% \label{sec:textcolor}
% The two-argument command \cs{textcolor}\marg{color}\marg{Ara\-bic
% text} is supported inside \cs{begin}|{arab}| \ldots\
% \cs{end}|{arab}|. One simple example
% follows:\footnote{\package{arabluatex} provides its own
% \cs{arbcolor} command which is able to render syllabes or diacritics
% in colors. See \vref{sec:color}.}--- %
% \iffalse
%<*example>
% \fi
\begin{example}
  \begin{arab}
    \textcolor{red}{\uc{m}uha_d_dabu \uc{'l-d}Ini \uc{`a}bdu
      \uc{'l-r}a.hImi bnu \uc{`a}liyyiN} huwa ^say_hu-nA 'l-'imAmu
    'l-.sadru 'l-kabIru 'l-`Alimu 'l-fA.dilu \uc{m}uha_d_dabu
    \uc{'l-d}Ini \uc{'a}bU \uc{m}u.hammadiN \uc{`a}bdu
    \uc{'l-r}a.hImi bnu \uc{`a}liyyi bni \uc{.h}AmidiN wa-yu`rafu
    bi-\uc{'l-d}a_hwari.
  \end{arab}
  \begin{arab}[trans]
    \textcolor{red}{\uc{m}uha_d_dabu \uc{'l-d}Ini \uc{`a}bdu
      \uc{'l-r}a.hImi bnu \uc{`a}liyyiN} huwa ^say_hu-nA 'l-'imAmu
    'l-.sadru 'l-kabIru 'l-`Alimu 'l-fA.dilu \uc{m}uha_d_dabu
    \uc{'l-d}Ini \uc{'a}bU \uc{m}u.hammadiN \uc{`a}bdu
    \uc{'l-r}a.hImi bnu \uc{`a}liyyi bni \uc{.h}AmidiN wa-yu`rafu
    bi-\uc{'l-d}a_hwari.
  \end{arab}
\end{example}
% \iffalse
%</example>
% \fi
%
% \paragraph{\package{reledmac}}
% \label{sec:reledmac}
% The two-argument command \cs{edtext}\marg{lemma}\marg{commands} is
% supported inside \cs{begin}|{arab}| \ldots\
% \cs{end}|{arab}|.\footnote{\cs{pstart} and \cs{pend} are also
% supported inside the |arab| environment.} As an example, one may get
% \package{arabluatex} and \package{reledmac} to work together like
% so:--- %
% \iffalse
%<*example>
% \fi
\begin{code}
  \beginnumbering
  \pstart
  \begin{arab}
    wa-ya.sIru ta.hta 'l-jild-i
    \edtext{\arb{.sadId-uN}}{\Afootnote{M: \arb{.sadId-aN} E1}}
  \end{arab}
  \pend
  \endnumbering
\end{code}
% \iffalse
%</example>
% \fi
%
% \subsection{\package{quran}}
% \label{sec:pkg-quran}
% \changes{v1.5}{2016/11/14}{Compatibility with the \textsf{quran}
% package} \package{arabluatex} is compatible with the \package{quran}
% package so that both can be used in conjunction with one another for
% typesetting the \arb[trans]{\uc{qur'An}}. As \package{quran} draws
% the text of the \arb[trans]{\uc{qur'An}} from a Unicode encoded
% database, its commands have to be passed as arguments to the
% \cs{txarb} command for short insertions in left-to-right paragraphs,
% or inserted inside the \index{txarab=txarab (environment)}|txarab|
% environment for typesetting running paragraphs of
% \arb[trans]{\uc{qur'An}}\emph{ic} text (see above
% \vref{sec:unicode-input} for more details). Please note that
% \package{arabluatex} takes care of formatting the Arabic: therefore,
% it is recommended to load the \package{quran} package with the
% |nopar| option, after \package{arabluatex} itself has been loaded,
% like so:--- %
% \iffalse
%<*example>
% \fi
\begin{code}
  \usepackage{arabluatex}
  \usepackage[nopar]{quran}
\end{code}
% \iffalse
%</example>
% \fi
%
% As an example, the following code will typeset the \arb[trans]{sUraT
% al-\uc{fAti.haT}}:---
% \iffalse
%<*example>
% \fi
\begin{example}
  \begin{txarab}
    \quransurah[1]
  \end{txarab}
\end{example}
% \iffalse
%</example>
% \fi
%
% \section{Exporting Unicode Arabic to an external file}
% \label{sec:arabtex2utf}
% \NEWfeature{v.1.13}\package{arabluatex} is able to produce a
% duplicate of the original |.tex| source file in which all |arabtex|
% or |buckwalter| strings will have been replaced with Unicode
% equivalents, either in Arabic script or in any accepted standard of
% transliteration. Exporting \textsc{ascii} strings to Unicode while
% preserving the exact selected global or local options is a fairly
% complex operation which may require {\LuaLaTeX} to be run several
% times as will be explained below.
%
% \subsection{Commands and environments}
% \paragraph{\texttt{export} global option}
% \DescribeOption{export} First, \package{arabluatex} must be loaded
% with the |export| global option enabled,\footnote{See above
% \vpageref{ref:export-global-opt} for more information.} like
% so:--- %
% \iffalse
%<*example>
% \fi
\begin{code}
  % preamble
  \usepackage[export]{arabluatex}
  % or:
  \usepackage[export=true]{arabluatex}
\end{code}
% \iffalse
%</example>
% \fi
% 
% Once that is done, compiling the current file will produce a new
% empty external |.tex| file with the same preamble as the original
% file.
%
% \DescribeMacro{\SetArbOutSuffix} By default, |_out| is appended as a
% suffix to the external file name. Any other suffix may be set with
% the command \cs{SetArbOutSuffix}\marg{suffix}.
%
% \paragraph{Exporting running paragraphs}
% \DescribeEnv{arabexport} Then, the |arabexport| environment is
% provided to actually exporting running paragraphs with or without
% Arabic environments to the external selected file, like so:--- %
% \iffalse
%<*example>
% \fi
\begin{code}
  \begin{arabexport}
    <Running paragraphs of either Arabic or non-Arabic text>
  \end{arabexport}
\end{code}
% \iffalse
%</example>
% \fi
% 
% \package{arabluatex} converts to Unicode and writes to the external
% file what is found inside Arabic environments. As to non-Arabic
% text, it is appended untouched to this file, which is formatted as
% follows:---
% \begin{enumerate}
% \item Unicode Arabic text, either in Arabic script or in
%   transliteration, is inserted as argument of
%   \cs{txarb}\footnote{See above \vref{sec:unicode-input}.} or
%   \cs{txtrans}\footnote{\cs{txtrans} is used internally by several
%   Lua functions to format transliterated Arabic. Therefore, it is
%   not documented.} accordingly.
% \item \DescribeMacro{\arbpardir}Additionally, Arabic paragraphs may
%   receive \cs{arbpardir}, which \package{arabluatex} uses to
%   determine the direction of Arabic paragraphs to be set by default,
%   or either \cs{setRL} or \cs{setLR} depending on what may have been
%   set locally.\footnote{See above \vpageref{ref:setrl-setlr}.}
% \item \DescribeMacro{\prname*}Proper names are inserted as arguments
% of \cs{prname*}.\footnote{See above \vpageref{ref:prname-star}.}
% \end{enumerate}
% 
% \paragraph{Appending words or commands to the external file only}
% \DescribeMacro{\ArbOutFile}\DescribeMacro{\ArbOutFile*}
% \cs{ArbOutFile}\oarg{newline}\allowbreak\marg{argument} silently
% exports its argument to the external file. It may take the string
% |newline| as an optional argument, in which case a carriage return
% is appended to the contents of the
% argument. \cs{ArbOutFile*}\oarg{newline}\marg{argument} does the
% same as \cs{ArbOutFile}, but also inserts its argument into the
% current |.tex| source file.
%
% \paragraph{Exporting Arabic poetry}
% Lines of Arabic poetry are exported as described above
% \vpageref{ref:poetry-export} when the |export| option that is
% specific to the |arabverse| environment is set to |true|. As a
% result of this particular feature, |arabverse| environments must be
% left outside |\begin{arabexport}| \ldots\ |\end{arabexport}|.
%
% Please note that inside |arabverse| environments \cs{bayt} is
% replaced with \cs{bayt*}.\footnote{See above \vref{ref:bayt-star} for
% more information.}
%
% \subsection{Nested Arabic environments}
% The exporting mechanism described above converts only the outermost
% level of nested Arabic environments. This may be sufficient in some
% cases, but if nested Arabic environments be found in the original
% |.tex| source file, then the Unicode converted file must be opened
% and compiled in turn, and so on until the innermost Arabic
% environment be converted and exported. In such cases,
% \package{arabluatex} issues a warning, so that authors do not have
% to check the entire file that just has been exported:--- %
% \iffalse
%<*example>
% \fi
\begin{code}
  Package arabluatex Warning: There are still 'arabtex' strings
  to be converted. Please open <jobname><suffix>.tex and compile
  it one more time.
\end{code}
% \iffalse
%</example>
% \fi
% Where \meta{jobname} is the name of the original |.tex| source file,
% and \meta{suffix} the suffix appended to the file that is to be
% opened and compiled again.
%
% \subsection{Further processing of Unicode converted files}
% \label{sec:further-processing-utf-files}
% Unicode files can be further processed by document converters such
% as John McFarlane's |pandoc|\footnote{See
% \url{http://pandoc.org/}}. To take here one simple example, here is
% how |file_out.tex| can be converted from {\LuaLaTeX} into Open
% Document format (|.odt|):---%
% \iffalse
%<*example>
% \fi
\begin{code}
  pandoc file_out.tex -s -o file_out.odt
\end{code}
% \iffalse
%</example>
% \fi
% 
% However, specific commands such as \cs{txarb}, \cs{txtrans} or
% \cs{prname*}, which are not known to |pandoc|, must be redefined
% explicitly in the preamble to prevent the converter from gobbling
% their arguments, like so:---%
% \iffalse
%<*example>
% \fi
\begin{code}
  % preamble:
  \usepackage{arabluatex} % note that 'export' has been removed
  \renewcommand{\txarb}[1]{#1}
  \renewcommand{\txtrans}[1]{\emph{#1}}
  \renewcommand{\arbup}[1]{\textsuperscript{#1}}
  % now that \prname{} has been replaced with \prname*{} it should
  % be safe to say:
  \renewcommand{\prname}[2]{#2}
  % &c
\end{code}
% \iffalse
%</example>
% \fi
%
% \section{Future work}
% \label{sec:future-work}
% A short, uncommented, list of what is planned in the versions of
% \package{arabluatex} to come follows:
% \begin{enumerate}
% \item Short-term:
%   \begin{enumerate}
%   \item \texttt{TEI xml} support: \package{arabluatex} will
%     interoperate with \texttt{TEI xml} through new global and local
%     options that will output Arabic in a \texttt{TEI xml} compliant
%     file in addition to the usual PDF output: see
%     \vpageref{ref:tei-to-come}.
%   \end{enumerate}
% \item Medium-term:
%   \begin{enumerate}
%   \item More languages: the list of supported languages will
%     eventually be the same as \package{arabtex}: see
%     \vref{fn:arabtex-languages}.
%   \item Formulate propositions for extending the Arab\TeX\ notation
%     and the transliteration tables. Include them in
%     \package{arabluatex}. See \vref{sec:additional-characters}.
%   \end{enumerate}
% \end{enumerate}
%
% \StopEventually{}
%
% \section{Implementation}
%
% \iffalse
%<*package>
% \fi
%
% The most important part of \package{arabluatex} relies on Lua
% functions and tables. Read the |.lua| files that accompany
% \package{arabluatex} for more information.
%    \begin{macrocode}
\RequirePackage{iftex}
%    \end{macrocode}
% \package{arabluatex} requires {\LuaLaTeX} of course. Issue a warning
% if the document is processed with another engine.
%    \begin{macrocode}
\RequireLuaTeX
%    \end{macrocode}
% Declare the global options, and define them:
%    \begin{macrocode}
\RequirePackage{xkeyval}
\DeclareOptionX{voc}{\def\al@mode{voc}}
\DeclareOptionX{fullvoc}{\def\al@mode{fullvoc}}
\DeclareOptionX{novoc}{\def\al@mode{novoc}}
\DeclareOptionX{trans}{\def\al@mode{trans}}
\define@boolkey{arabluatex.sty}[@pkg@]{export}[true]{%
  \if@pkg@export%
  \AtBeginDocument{\luadirect{arabluatex.openstream()}%
    \MkArbBreak{@al@ob,@al@cb,@al@cb@sp}}
  \AtEndDocument{\luadirect{arabluatex.closestream()}}
  \else\fi}
\ExecuteOptionsX{voc}
\ProcessOptionsX\relax
\def\al@mode@voc{voc}
\def\al@mode@fullvoc{fullvoc}
\def\al@mode@novoc{novoc}
\def\al@mode@trans{trans}
%    \end{macrocode}
% Packages that are required by \package{arabluatex}:
%    \begin{macrocode}
\RequirePackage{xcolor}
\RequirePackage{luacolor}
\RequirePackage{etoolbox}
\RequirePackage{arabluatex-patch}
\RequirePackage{fontspec}
\RequirePackage{luacode}
\RequirePackage{xparse}
\RequirePackage{adjustbox}
\RequirePackage{xstring}
\RequirePackage{lua-ul}
%    \end{macrocode}
% The following boolean will be set to |true| in |RL| mode:
%    \begin{macrocode}
\providebool{al@rlmode}
%    \end{macrocode}
% Here begins the real work: load |arabluatex.lua|:
%    \begin{macrocode}
\luadirect{dofile(kpse.find_file("arabluatex.lua"))}
%    \end{macrocode}
% Font setup. If no Arabic font is selected, issue a warning message
% and attempt to load the Amiri font which is included in \TeX{}live:
%    \begin{macrocode}
\AtBeginDocument{\ifdefined\arabicfont\relax\else
  \PackageInfo{arabluatex}{%
    \string\arabicfont\ is not defined.\MessageBreak
    arabluatex will try to load Amiri}%
  \newfontfamily\arabicfont{Amiri}[Script=Arabic]\fi}%
%    \end{macrocode}
% \begin{macro}{\setRL}
%   This neutralizes what may be defined by other packages:
%    \begin{macrocode}
\AtBeginDocument{\def\setRL{\booltrue{al@rlmode}\pardir TRT%
    \textdir TRT}}
%    \end{macrocode}
% \end{macro}
% \begin{macro}{\setLR}
%   The same applies to \cs{setLR}:
%    \begin{macrocode}
\AtBeginDocument{\def\setLR{\boolfalse{al@rlmode}\pardir TLT%
    \textdir TLT}}
%    \end{macrocode}
% \end{macro}
% \begin{macro}{\LR} This command typesets its argument from left to
% right. As \cs{LR} may be already defined, we need to redefine for
% it to suit our purpose:
%    \begin{macrocode}
\AtBeginDocument{\ifdef{\LR}%
  {\RenewDocumentCommand{\LR}{m}{\bgroup\textdir TLT\rmfamily#1\egroup}}
  {\NewDocumentCommand{\LR}{m}{\bgroup\textdir TLT\rmfamily#1\egroup}}}
%    \end{macrocode}
% \end{macro}
% \begin{macro}{\RL} This one typesets its argument from right to
% left. Same remark as above regarding the need of redefinition.
%    \begin{macrocode}
\AtBeginDocument{\ifdef{\RL}%
  {\RenewDocumentCommand{\RL}{m}{\bgroup\textdir TRT\rmfamily#1\egroup}}
  {\NewDocumentCommand{\RL}{m}{\bgroup\textdir TRT#1\rmfamily\egroup}}}
%    \end{macrocode}
% \end{macro}
% \begin{macro}{\MkArbBreak}
% \changes{v1.9}{2017/07/05}{New \cs{MkArbBreak} command for
% inserting user-defined macros in Arabic environments} The
% \cs{MkArbBreak}\marg{csv list of commands} command can be used to
% give any command---either new or already existing---the precedence
% over \package{arabluatex} inside Arabic environments. It is
% actually coded in Lua.
% \begin{macro}{\MkArbBreak*}
% \changes{v1.12}{2018/06/24}{\enquote*{starred} version which
% closes Arabic environments before processing declared commands.}
% \cs{MkArbBreak*} goes a step further as it directs
% \package{arabluatex} to close the current Arabic environment before
% processing any \enquote*{declared} command then resume it just
% after.
%    \begin{macrocode}
\NewDocumentCommand{\MkArbBreak}{s m}{%
  \IfBooleanTF{#1}
  {\luadirect{arabluatex.mkarbbreak(\luastringN{#2}, "out")}}
  {\luadirect{arabluatex.mkarbbreak(\luastringN{#2}, "dflt")}}
}
%    \end{macrocode}
% \end{macro}
% \end{macro}
% \begin{macro}{\aemph} Arabic emphasis. Needs to be redefined as
%   well. The function is actually coded in Lua.
%   \changes{v1.16}{2018/11/06}{Now uses \textsf{ulem}}
%   \changes{v1.19}{2020/03/15}{Now uses \textsf{lua-ul}}
% \begin{macro}{\aemph*} The \enquote*{starred} version of this
%   command alway puts the stroke over its argument.
%   \changes{v1.9.2}{2017/10/24}{Starred version which always puts the
%   stroke over its argument}As of v1.19, \package{arabluatex} uses
%   \package{lua-ul} to render the strokes, thus allowing line breaks
%   and manual hyphenation for transliterated Arabic.
% \begin{macro}{\aoline}
% \begin{macro}{\aoline*}
% \begin{macro}{\auline}
% \changes{v1.19}{2020/03/15}{Non context-sensitive command to
% underline Arabic words is provided}\cs{aoline} and \cs{auline}
% derive from \cs{newunderlinetype} provided by the \package{lua-ul}
% package whereas \cs{aoline*}, which uses \cs{overline} in math-mode,
% is better suited for so-called \arb[trans]{'ab^gad} numbers.
%    \begin{macrocode}
\newunderlinetype\@aoverLine{\leaders\vrule height 3ex depth -2.9ex}
\def\aoline{\@ifstar\@aoline\@@aoline}
\def\@aoline#1{\ensuremath{\overline{\mbox{#1}}}}
\def\@@aoline#1{{\@aoverLine#1}}
\newunderlinetype\@aunderLine{\leaders\vrule height -.65ex depth .75ex}
\def\auline#1{{\@aunderLine#1}}
\AtBeginDocument{\ifdef{\aemph}%
  {\RenewDocumentCommand{\aemph}{s m}{%
      \IfBooleanTF{#1}{%
        \luadirect{tex.sprint(arabluatex.aemph(\luastringN{#2},
          "over"))}}
      {\luadirect{tex.sprint(arabluatex.aemph(\luastringN{#2},
          "dflt"))}}}}
  {\NewDocumentCommand{\aemph}{s m}{%
      \IfBooleanTF{#1}{%
        \luadirect{tex.sprint(arabluatex.aemph(\luastringN{#2},
          "over"))}}
      {\luadirect{tex.sprint(arabluatex.aemph(\luastringN{#2},
          "dflt"))}}}}}
%    \end{macrocode}
% \end{macro}
% \end{macro}
% \end{macro}
% \end{macro}
% \end{macro}
% \begin{macro}{\arbcolor}\changes{v1.12}{2018/06/24}{Standard color
% command for Arabic environments}
% \cs{arbcolor}\oarg{color}\marg{Arabic text} takes the Arabic text to
% be colored as argument.
%    \begin{macrocode}
\NewDocumentCommand{\arbcolor}{o m}{%
  \IfNoValueTF{#1}{#2}{\textcolor{#1}{#2}}}
%    \end{macrocode}
% 
% \end{macro}
% \begin{macro}{\SetInputScheme}
%   \changes{v1.4}{2016/07/05}{\cs{SetInputScheme} can be used to
%   process other input schemes such as \enquote*{Buckwalter}}
%   \package{arabluatex} is designed for processing Arab\TeX\ input
%   notation. \cs{SetInputScheme} may be used in the preamble or at
%   any point of the document should the user wish to use a different
%   notation such as the \enquote*{Buckwalter scheme}.
%    \begin{macrocode}
\def\al@input@scheme{arabtex}
\NewDocumentCommand{\SetInputScheme}{m}{\def\al@input@scheme{#1}}
%    \end{macrocode}
% \end{macro}
% \begin{macro}{\SetArbEasy}
%   \changes{v1.2}{2016/05/09}{New \cs{SetArbEasy}/\cs{SetArbDflt} for
%   \enquote*{modern} or \enquote*{classic} Arabic styles.}
% \begin{macro}{\SetArbEasy*}
%   \changes{v1.4.4}{2016/09/28}{this starred version discards the
%   \arb[trans]{sukUn} in addition to what is already discarded by
%   \cs{SetArbEasy}.}
% \begin{macro}{\SetArbDflt}
%   By default, \package{arabluatex} applies complex rules to generate
%   euphonic \arb[trans]{ta^sdId}, \arb[trans]{'alif mamdUdaT} and
%   \arb[trans]{sukUn} depending on the modes which are selected,
%   either |voc|, |fullvoc| or |trans|. Such refinements can be
%   discarded with \cs{SetArbEasy}, either globally in the preamble or
%   at any point of the document. Note that \cs{SetArbEasy} keeps the
%   \arb[trans]{sukUn} that is generated, while the starred version
%   \cs{SetArbEasy*} takes it away. Default complex rules can be set
%   back at any point of the document with \cs{SetArbDflt}.
% \begin{macro}{\SetArbDflt*}
%   \changes{v1.6}{2016/12/17}{This starred version applies the
%   assimilation rules in addition to what \cs{SetArbDflt} already
%   does.} As of v1.6, \package{arabluatex} does not applies any more
%   the assimilation rules that are laid on \vref{ref:assimilation}; a
%   new starred version \cs{SetArbDflt*} is now available to the user
%   should he wish to apply them.
%    \begin{macrocode}
\def\al@arb@rules{dflt}
\NewDocumentCommand{\SetArbEasy}{s}{%
  \IfBooleanTF{#1}
  {\def\al@arb@rules{easynosukun}}
  {\def\al@arb@rules{easy}}}
\NewDocumentCommand{\SetArbDflt}{s}{%
  \IfBooleanTF{#1}
  {\def\al@arb@rules{idgham}}
  {\def\al@arb@rules{dflt}}}
%    \end{macrocode}
% \end{macro}
% \end{macro}
% \end{macro}
% \end{macro}
% \begin{macro}{\SetTranslitFont}
%   \changes{v1.4}{2016/07/05}{For selecting a specific font for
%   transliterated texts} By default, the font that is used for
%   transliterated text is the main font of the document. Any other
%   font may also be selected with the font-selecting commands of the
%   \package{fontspec} package.
%    \begin{macrocode}
\def\al@trans@font{\rmfamily}%
\NewDocumentCommand{\SetTranslitFont}{m}{\def\al@trans@font{#1}}
%    \end{macrocode}
% \end{macro}
% \begin{macro}{\SetTranslitStyle} By default any transliterated
%   Arabic text is printed in italics. This can be changed either
%   globally in the preamble or at any point of the document:
%    \begin{macrocode}
\def\al@trans@style{\itshape}%
\NewDocumentCommand{\SetTranslitStyle}{m}{\def\al@trans@style{#1}}
%    \end{macrocode}
% \end{macro}
% \begin{macro}{\SetTranslitConvention}
%   \cs{SetTranslitConvention}\marg{convention} can be used to change
%   the transliteration convention, which is |dmg| by default:
%    \begin{macrocode}
\def\al@trans@convention{dmg}
\NewDocumentCommand{\SetTranslitConvention}{m}{%
  \def\al@trans@convention{#1}}
%    \end{macrocode}
% \end{macro}
% \begin{macro}{\arbup}
% \changes{v1.3}{2016/05/28}{\arb[trans]{'i`rAb} is now written as
% superscript text in \texttt{dmg} mode by default.}
% \begin{macro}{\NoArbUp}
% \begin{macro}{\ArbUpDflt}
% \begin{macro}{\SetArbUp}
%   By default, \cs{arbup} is set to \cs{textsuperscript}. This is how
%   the \arb[trans]{tanwIn} that takes place at the end of a word
%   should be displayed in |dmg| mode. \cs{NoArbUp} may be used
%   either in the preamble or at any point of the document in case one
%   wishes to have the \arb[trans]{tanwIn} on the line. The default
%   rule can be set back with \cs{ArbUpDflt} at any point of the
%   document. Finally \cs{SetArbUp} can be used to customize the way
%   \arb[trans]{tanwIn} is displayed: this command takes the
%   formatting directives as argument, like so:
%   \cs{SetArbUp}\marg{code}.
%    \begin{macrocode}
\NewDocumentCommand{\al@arbup@dflt}{m}{\textsuperscript{#1}}%
\NewDocumentCommand{\al@arbup}{m}{\al@arbup@dflt{#1}}
\NewDocumentCommand{\arbup}{m}{\al@arbup{#1}}
\NewDocumentCommand{\ArbUpDflt}{}{\let\al@arbup=\al@arbup@dflt}
\NewDocumentCommand{\NoArbUp}{}{\RenewDocumentCommand{\al@arbup}{m}{##1}}
\NewDocumentCommand{\SetArbUp}{m}{%
  \RenewDocumentCommand{\al@arbup}{m}{#1}}
%    \end{macrocode}
% \end{macro}
% \end{macro}
% \end{macro}
% \end{macro}
% \begin{macro}{\uc} Proper Arabic names or book titles should be
% passed to the \cs{uc} command so that they have their first letters
% uppercased. \cs{uc} is actually coded in Lua.
%    \begin{macrocode}
\NewDocumentCommand{\uc}{m}%
  {\luadirect{tex.sprint(arabluatex.uc(\luastringN{#1}))}}
%    \end{macrocode}
% \end{macro}
% \begin{macro}{\Uc} \cs{uc} can be used safely in all of the modes
%   that are provided by \package{arabluatex} as any of the |voc|,
%   |fullvoc| and |novoc| modes discard it on top of any other
%   functions to be run.  \cs{Uc} does the same as \cs{uc} except
%   that \emph{it is never discarded}. For that reason, \cs{Uc}
%   \emph{should never be used outside the} |trans|
%   \emph{mode}. \package{arabluatex} uses \cs{Uc} internally so as
%   to prevent \cs{uc} from being discarded in case words that are to
%   be transliterated are inserted into Arabic commands or
%   environments where transliteration is not required. Therefore, it
%   is not documented.
%    \begin{macrocode}
\let\Uc\uc
%    \end{macrocode}
% \end{macro}
% \begin{macro}{\prname}\changes{v1.10}{2018/01/03}{New command for
% typesetting Arabic proper names in transliteration} \cs{prname} is
% to be used outside Arabic environments for proper names. It takes as
% argument one or more Arabic words, each of which will be rendered in
% upright roman style with its first letter uppercased.
% \begin{macro}{\prname*}\changes{v1.13}{2018/08/27}{Renders proper
% names already converted to Unicode in upright roman style}
% Unlike \cs{prname}, \cs{prname*} does not take |arabtex| or
% |buckwalter| input as argument, but already Unicode converted
% names and renders them in upright roman style.
%    \begin{macrocode}
\NewDocumentCommand{\prname}{s m}{%
  \bgroup\SetTranslitStyle{\relax}%
  \IfBooleanTF{#1}{\txtrans{#2}}{\arb[trans]{\uc{#2}}}\egroup}
%    \end{macrocode}
% \end{macro}
% \end{macro}
% \begin{macro}{\txarb} \cs{txarb} sets the direction to right-to-left
%   and selects the Arabic font. It is used internally by several Lua
%   functions, but available to the user should he wish to insert
%   |utf8| Arabic text in his document.
% \begin{macro}{\txtrans} \cs{txtrans} is used internally by several
% Lua functions to insert transliterated Arabic text. Therefore, it is
% not documented.
%    \begin{macrocode}
\NewDocumentCommand{\txarb}{+m}{%
  \ifvmode\leavevmode\fi%
  \bgroup\textdir TRT\arabicfont#1\egroup}
\NewDocumentCommand{\txtrans}{+m}{%
  \bgroup\textdir TLT\al@trans@font\al@trans@style#1\egroup}
%    \end{macrocode}
% \end{macro}
% \end{macro}
% \begin{environment}{txarab}
%   \changes{v1.5}{2016/11/14}{New \texttt{txarab} environment for
%   typesetting running paragraphs in Unicode Arabic} The |txarab|
%   environment does for paragraphs the same as \cs{txarb} does for
%   short insertions of |utf8| Arabic text.
%    \begin{macrocode}
\NewDocumentEnvironment{txarab}{}{%
  \par%
  \booltrue{al@rlmode}%
  \pardir TRT\textdir TRT\arabicfont}{\par}
%    \end{macrocode}
% \end{environment}
% \begin{environment}{txarabtr}
% |txarabtr| environment is used internally by several Lua functions
% to insert running paragraphs of transliterated Arabic text
% Therefore, it is not documented.
%    \begin{macrocode}
\NewDocumentEnvironment{txarabtr}{}{%
  \par%
  \pardir TLT\textdir TLT%
  \al@trans@font\al@trans@style}{\par}
%    \end{macrocode}
% \end{environment}
% \begin{macro}{\arb}
%   The \cs{arb} command detects which Arabic mode is to be used,
%   either globally if no option is set, or locally, then passes its
%   argument to the appropriate Lua function.
%    \begin{macrocode}
\NewDocumentCommand{\arb}{O{\al@mode} +m}%
{\edef\@tempa{#1}%
  \ifx\@tempa\al@mode@voc%
  \ifvmode\leavevmode\fi%
  \bgroup\booltrue{al@rlmode}\textdir TRT\arabicfont%
  \luadirect{tex.sprint(arabluatex.processvoc(\luastringN{#2},
    \luastringO{\al@arb@rules}, \luastringO{\al@input@scheme}))}\egroup%
  \else%
  \ifx\@tempa\al@mode@fullvoc%
  \ifvmode\leavevmode\fi%
  \bgroup\booltrue{al@rlmode}\textdir TRT\arabicfont%
  \luadirect{tex.sprint(arabluatex.processfullvoc(\luastringN{#2},
    \luastringO{\al@arb@rules}, \luastringO{\al@input@scheme}))}\egroup%
  \else%
  \ifx\@tempa\al@mode@novoc%
  \ifvmode\leavevmode\fi%
  \bgroup\booltrue{al@rlmode}\textdir TRT\arabicfont%
  \luadirect{tex.sprint(arabluatex.processnovoc(\luastringN{#2},
    \luastringO{\al@arb@rules}, \luastringO{\al@input@scheme}))}\egroup%
  \else%
  \ifx\@tempa\al@mode@trans%
  \bgroup\textdir TLT\al@trans@font\al@trans@style%
  \luadirect{tex.sprint(arabluatex.processtrans(\luastringN{#2},
    \luastringO{\al@trans@convention},
    \luastringO{\al@arb@rules},
    \luastringO{\al@input@scheme}))}\egroup%
  \else%
  \fi\fi\fi\fi}
%    \end{macrocode}
% \end{macro}
% \begin{macro}{\arbmark}
%   \changes{v1.11}{2018/03/31}{New command for inserting additional
%   marks in Arabic environments}
%   \cs{arbmark}\oarg{rl\textbar{}lr}\marg{shorthand} takes one
%   argument from a list of defined elements.
%   \changes{v1.13}{2018/08/27}{New optional argument: either
%   \texttt{rl} or \texttt{lr}} The mark to be inserted is determined
%   by contextual analysis or by an optional argument, either |rl| or
%   |lr|. This command is coded in Lua.
%    \begin{macrocode}
\NewDocumentCommand{\arbmark}{O{} m}{%
  \bgroup%
  \SetInputScheme{arabtex}%
  \luadirect{tex.sprint(arabluatex.processarbmarks(\luastringN{#2},
    \luastringN{#1}))}%
  \egroup}
%    \end{macrocode}
% \end{macro}
% \begin{macro}{\newarbmark}
% \changes{v1.11}{2018/03/31}{Allows defining additional sets of Arabic
% marks} \cs{newarbmark} lets the user define additional Arabic
% marks. As \cs{arbmark}, this command is coded in Lua.  It takes
% three arguments: the abbreviated form to be used as argument of
% \cs{arbmark}, the rendition in Arabic script and the rendition in
% romanized Arabic.
%    \begin{macrocode}
\NewDocumentCommand{\newarbmark}{m m m}{%
  \luadirect{arabluatex.newarbmark(\luastringN{#1}, \luastringN{#2},
    \luastringN{#3})}}
%    \end{macrocode}
% \end{macro}
% \begin{environment}{arab}
% The |arab| environment does for paragraphs the same as \cs{arb} does
% for short insertions of Arabic text.
%    \begin{macrocode}
\NewDocumentEnvironment{arab}{!O{\al@mode} +b}%
{\par\edef\@tempa{#1}%
  \ifx\@tempa\al@mode@voc%
  \booltrue{al@rlmode}%
  \bgroup\pardir TRT\textdir TRT\arabicfont%
  \luadirect{tex.sprint(arabluatex.processvoc(\luastringN{#2},
    \luastringO{\al@arb@rules}, \luastringO{\al@input@scheme}))}\egroup%
  \else%
  \ifx\@tempa\al@mode@fullvoc%
  \booltrue{al@rlmode}%
  \bgroup\pardir TRT\textdir TRT\arabicfont%
  \luadirect{tex.sprint(arabluatex.processfullvoc(\luastringN{#2},
    \luastringO{\al@arb@rules}, \luastringO{\al@input@scheme}))}\egroup%
  \else%
  \ifx\@tempa\al@mode@novoc%
  \booltrue{al@rlmode}%
  \bgroup\pardir TRT\textdir TRT\arabicfont%
  \luadirect{tex.sprint(arabluatex.processnovoc(\luastringN{#2},
    \luastringO{\al@arb@rules}, \luastringO{\al@input@scheme}))}\egroup%
  \else%
  \ifx\@tempa\al@mode@trans%
  \bgroup\pardir TLT\textdir TLT\al@trans@font\al@trans@style%
  \luadirect{tex.sprint(arabluatex.processtrans(\luastringN{#2},
    \luastringO{\al@trans@convention},
    \luastringO{\al@arb@rules},
    \luastringO{\al@input@scheme}))}\egroup%
  \else \fi\fi\fi\fi}{\par}
%    \end{macrocode}
% \end{environment}
% \begin{environment}{arabverse}
%   \changes{v1.6}{2016/12/17}{New environment \texttt{arabverse} for
%   typesetting Arabic poetry} The |arabverse| environment may receive
%   different options: |mode|, |width|, |gutter|, |metre|, |color|,
%   |utf|, |delim| and |export|; all of them are defined here just
%   before the |arabverse|
%   environment. \changes{v1.13}{2018/08/27}{New options
%   \texttt{color} and \texttt{export} to \texttt{arabverse}
%   environment.}
%    \begin{macrocode}
\newlength{\al@bayt@width}
\newlength{\al@gutter@width}
\setlength{\al@bayt@width}{.3\textwidth}
\setlength{\al@gutter@width}{.15\al@bayt@width}
\define@key[al]{verse}{width}{\setlength{\al@bayt@width}{#1}}
\define@key[al]{verse}{gutter}{\setlength{\al@gutter@width}{#1}}
\define@key[al]{verse}{metre}{\arb{#1}}
\define@key[al]{verse}{color}[]{\color{#1}}
\define@boolkey[al]{verse}{utf}[true]{}
\define@boolkey[al]{verse}{delim}[true]{}
\define@boolkey[al]{verse}{export}[true]{}
\define@choicekey[al]{verse}{mode}{fullvoc, voc, novoc,
  trans}{\def\al@mode{#1}}
\presetkeys[al]{verse}{metre={}, utf=false,
  delim=false}{}
%    \end{macrocode}
% Then follows the environment itself:
%    \begin{macrocode}
\NewDocumentEnvironment{arabverse}{!O{}}%
{\bgroup\setkeys[al]{verse}[width, gutter, color, utf, delim,
  metre]{#1}%
  \if@pkg@export\ifal@verse@export%
  \ArbOutFile{\begin{arabverse}}%
    % \ifx\al@mode\al@mode@trans%
    %   \luadirect{arabluatex.tooutfile(\luastringN{[#1]})}%
    % \else%
      \IfSubStr[1]{#1}{utf}%
        {\luadirect{arabluatex.tooutfile(\luastringN{[#1]})}}%
        {\luadirect{arabluatex.tooutfile(\luastringN{[#1, utf]})}}%
    % \fi
  \else\fi\else\fi\egroup%
  \par\centering\noindent\bgroup\setkeys[al]{verse}[metre]{#1}%
  % \ifx\al@mode\al@mode@trans%
  %   \ifal@verse@utf\setRL\else\setLR\fi%
  % \else\setRL\fi%
  \ifal@verse@utf%
    \ifx\al@mode\al@mode@trans\setLR\else\setRL\fi%
    \else%
    \ifx\al@mode\al@mode@trans\setLR\else\setRL\fi%
    \fi%
  \arab@v@export[#1]
  }%
  {\endarab@v@export
    \hfill\setkeys[al]{verse}[width, gutter, color, utf, delim, mode,
    export]{#1}%
    \egroup\par%
    \bgroup\setkeys[al]{verse}[width, gutter, color, utf, delim, mode,
    metre]{#1}%
    \if@pkg@export\ifal@verse@export%
    \ArbOutFile{\end{arabverse}}
  \else\fi\else\fi\egroup}
%    \end{macrocode}
% \begin{macro}{\bayt}
%   \changes{v1.6}{2016/12/17}{New macro \cs{bayt} for typesetting
%   each verse inside the \texttt{arabverse} environment} Each verse
%   consists of two hemistichs; therefore the \cs{bayt} command takes
%   two arguments, the first receives the \arb[trans]{.sadr} and the
%   second the \arb[trans]{`ajuz}. That two subsequent hemistichs
%   should be connected with one another is technically named
%   \arb[trans]{tadwIr}. In some of these cases, the hemistichs may be
%   connected by a prominent horizontal flexible stroke which is drawn
%   by the \cs{al@verse@stroke} command.
% \begin{macro}{\StretchBayt}
% \changes{v1.20}{2020/03/23}{Optionally removes stretching from lines
% of poetry} \cs{StretchBayt}\oarg{true\textbar false} Allows to
% remove stretching and undesirable warping effect from Arabic lines of
% poetry. This command accepts one fixed optional argument, either
% |true| or |false|, and may be used either in the preamble or at any
% point of the document. By default, it is set to |true|.
% \end{macro}
% \begin{macro}{\SetHemistichDelim}
%   \changes{v1.6}{2016/12/17}{New \cs{SetHemistichDelim} command for
%   changing the default delimiter between hemistichs} A hemistich
%   delimiter also may be defined. By default, it is set to the
%   \enquote*{star} character: |*|. The
%   \cs{SetHemistichDelim}\marg{delimiter} command can be used at any
%   point of the document to change this default setting.
% \end{macro}
%    \begin{macrocode}
\newif\ifal@warp@bayt
\al@warp@bayttrue
\NewDocumentCommand{\StretchBayt}{O{true}}{
  \edef\oarg@true{true}
  \edef\oarg@false{false}
  \edef\@tempa{#1}
  \ifx\@tempa\oarg@true\al@warp@bayttrue
  \else
  \ifx\@tempa\oarg@false\al@warp@baytfalse
  \else
  \PackageError{arabluatex}{\string\StretchBayt\space must be
    either 'true' or 'false'}{}
  \fi
  \fi
}
\NewDocumentCommand{\arb@utf}{m}{%
  \ifal@verse@utf\txarb{#1}\else\arb{#1}\fi}
\def\al@hemistich@delim{*}
\NewDocumentCommand{\SetHemistichDelim}{m}{\def\al@hemistich@delim{#1}}
\def\al@verse@stroke{\leavevmode\xleaders\hbox{\arb{--}}\hfill\kern0pt}
\NewDocumentCommand{\bayt}{s m o m}{%
  \IfBooleanTF{#1}{\relax}{\relax}%
  \ifdefined\savenotes\savenotes\else\fi%
  \edef\al@tatweel{--}%
  \ifal@warp@bayt%
    \adjustbox{width=\al@bayt@width, height=\Height}{\arb@utf{#2}}%
  \else%
    \makebox[\al@bayt@width][s]{\arb@utf{#2}}%
  \fi%
  \IfNoValueTF{#3}{%
    \ifal@verse@delim\makebox[\al@gutter@width][c]{\al@hemistich@delim}%
    \else%
    \hspace{\al@gutter@width}%
    \fi
  }{%
    \edef\@tempa{#3}%
    \ifx\@tempa\al@tatweel%
    \ifx\al@mode\al@mode@trans%
    \hspace{\al@gutter@width}%
    \else%
    \makebox[\al@gutter@width][s]{\al@verse@stroke}%
    \fi%
    \else%
    \ifx\al@mode\al@mode@trans%
    \ifal@warp@bayt%
      \adjustbox{width=\al@gutter@width, height=\Height}{\arb@utf{#3}}%
    \else%
      \makebox[\al@gutter@width][s]{\arb@utf{#3}}%
    \fi%
    \else%
    \makebox[\al@gutter@width][s]{\arb@utf{#3}}%
    \fi\fi}%
  \ifal@warp@bayt%
    \adjustbox{width=\al@bayt@width, height=\Height}{\arb@utf{#4}}%
  \else%
    \makebox[\al@bayt@width][s]{\arb@utf{#4}}%
  \fi%
  \ifdefined\spewnotes\spewnotes\else\fi%
}
%    \end{macrocode}
% \end{macro}
% \end{environment}
% \begin{macro}{\arind}
% \changes{v1.18}{2020/02/29}{New command \cs{arind} for building
% indexes}
% \cs{arind}\marg{root} is a command specialized in the contruction of
% indexes. As a mandadory argument, it takes the Arabic root under
% which a given word is to be indexed. Additionally, it may receive
% three optional \enquote*{named} arguments: |index|, |root| and
% |form|.
%    \begin{macrocode}
\NewDocumentCommand{\SetDefaultIndex}{m}{
  \edef\@tempa{#1}
  \ifx\@tempa\empty
    \def\al@default@index{\jobname}
  \else
    \def\al@default@index{#1}
  \fi
}
%    \end{macrocode}
%    \begin{macrocode}
\def\al@index@mode{\al@mode}
\NewDocumentCommand{\SetIndexMode}{m}{
  \def\al@index@mode{#1}
}
%    \end{macrocode}
%    \begin{macrocode}
\define@cmdkeys[al]{index}[alind@]{index,root,form}
\NewDocumentCommand{\arind}{o m}{%
  \IfNoValueTF{#1}{%
    \ifdefined\al@default@index%
      \csname index\endcsname[\al@default@index]{#2}%
    \else%
      \csname index\endcsname{#2}%
    \fi%
  }{%
    \bgroup
    \setkeys[al]{index}{#1}%
    \def\al@one{%
      \ifdefined\alind@root!\LR{\alind@root}\else!\LR{1}\fi}%
    \def\al@two{%
      \ifdefined\alind@form @\arb[\al@index@mode]{\alind@form}\else\fi}%
    \ifdefined\alind@index%
      \csname index\endcsname[\alind@index]{#2\al@one\al@two}%
    \else%
      \ifdefined\al@default@index%
        \csname index\endcsname[\al@default@index]{#2\al@one\al@two}%
      \else%
        \csname index\endcsname{#2\al@one\al@two}%
      \fi%
    \fi%
    \egroup}}
%    \end{macrocode}
% \end{macro}
% \begin{macro}{\abjad} \cs{abjad}\marg{number} expresses its argument
%   in Arabic letters in accordance with the \arb[trans]{'abjad}
%   arrangement of the alphabet. \meta{number} must be between 1 and
%   1999. It is now coded in Lua so that \package{polyglossia} is no
%   longer needed. See |arabluatex.lua| for more information.
%   \changes{v1.1}{2016/04/26}{New and more flexible \protect\cs{abjad}
%   command.}
%    \begin{macrocode}
\AtBeginDocument{%
  \ifdefined\abjad%
  \RenewDocumentCommand{\abjad}{m}%
  {\ifbool{al@rlmode}%
    {\aoline*{%
        \luadirect{tex.sprint(arabluatex.abjadify(\luastring{#1}))}}}
    {\luadirect{tex.sprint(arabluatex.abjadify(\luastring{#1}))}}}
  \else%
  \NewDocumentCommand{\abjad}{m}%
  {\ifbool{al@rlmode}%
    {\aoline*{%
        \luadirect{tex.sprint(arabluatex.abjadify(\luastring{#1}))}}}
    {\luadirect{tex.sprint(arabluatex.abjadify(\luastring{#1}))}}}
  \fi}
%    \end{macrocode}
% \end{macro}
% \begin{macro}{\ayah}\changes{v1.15}{2018/10/09}{Prints End of Ayah
% sign}\cs{ayah}\marg{number} prints up to 3-digit numbers inside
% \enquote*{end of Ayah} sign (|U+06DD|) or inside parentheses
% depending on the mode which is selected.
%    \begin{macrocode}
\NewDocumentCommand{\ayah}{m}{%
  \luadirect{tex.sprint(arabluatex.ayah(\luastringN{#1}))}}
%    \end{macrocode}
% \end{macro}
% 
% \begin{macro}{\arbnull}
%   \changes{v1.7}{2016/12/24}{New \cs{arbnull} command for putting
%   back on any contextual analysis rule broken by other commands.}
%   The \cs{arbnull} command does nothing by itself. It is processed
%   only if it is found in Arabic context so as to put back on
%   contextual analysis in case it has been broken by other commands.
%    \begin{macrocode}
\NewDocumentCommand{\arbnull}{m}{\relax}
%    \end{macrocode}
% \end{macro}
% \begin{macro}{\abraces}
%   \cs{abraces}\marg{Arabic text} puts its argument between
%   braces. This macro is written in Lua and is dependent on the
%   current value of |tex.textdir|.
%   \changes{v1.4.3}{2016/09/14}{New \cs{abraces} command which
%   expresses its argument between braces.}
%    \begin{macrocode}
\NewDocumentCommand{\abraces}{+m}{%
  \luadirect{tex.sprint(arabluatex.abraces(\luastringN{#1}))}}
%    \end{macrocode}
% \end{macro}
% \begin{macro}{\LRmarginpar} \cs{LRmarginpar} is supposed to be
%   inserted in an Arabic environment. It typsets his argument in a
%   marginal note from left to right.
%    \begin{macrocode}
\DeclareDocumentCommand{\LRmarginpar}{o m}{%
  \IfNoValueTF{#1}
  {\marginpar{\textdir TLT #2}}
  {\marginpar[\textdir TLT #1]{\textdir TLT #2}}}
%    \end{macrocode}
% \end{macro}
% \begin{macro}{\LRfootnote} \cs{LRfootnote} and \cs{RLfootnote} are
%   supposed to be used in Arabic environments for insertions of non
%   Arabic text. \cs{LRfootnote} typesets its argument left-to-right\ldots
%   \begin{macro}{\RLfootnote} while \cs{RLfootnote} typesets its
%   argument left-to-right.
%    \begin{macrocode}
\DeclareDocumentCommand{\LRfootnote}{m}{\bgroup\pardir
  TLT\textdir TLT\footnote{#1}\egroup}
\DeclareDocumentCommand{\RLfootnote}{m}{\bgroup\pardir
  TRT\textdir TRT\footnote{#1}\egroup}
%    \end{macrocode}
%   \end{macro}
% \end{macro}
% \begin{macro}{\FixArbFtnmk} In the preamble, just below
%   \cs{usepackage}|{arabluatex}|, \cs{FixArbFtnmk} may be of some
%   help in case the footnote numbers at the bottom of the page are
%   printed in the wrong direction. This quick fix uses and loads
%   \package{scrextend} if it is not already loaded.
%    \begin{macrocode}
\NewDocumentCommand{\FixArbFtnmk}{}{%
  \@ifpackageloaded{scrextend}%
  {\AtBeginDocument{%
      \deffootnote{2em}{1.6em}{\LR{\thefootnotemark}.\enskip}}}%
  {\RequirePackage{scrextend}
    \AtBeginDocument{%
      \deffootnote{2em}{1.6em}{\LR{\thefootnotemark}.\enskip}}}}
%    \end{macrocode}
% \end{macro}
% 
% \paragraph*{Exporting Unicode Arabic to external file}
% \begin{macro}{\SetArbOutSuffix}
% \changes{v1.13}{2018/08/27}{Sets a suffix to be appended to the
% filename of the external Unicode file.}By default, |_out| is the
% suffix to be appended to the external file in which
% \package{arabluatex} exports Unicode in place of
% |arabtex| or |buckwalter| strings. Any other suffix may be
% set with \cs{SetArbOutSuffix}\marg{suffix}.
%    \begin{macrocode}
\NewDocumentCommand{\SetArbOutSuffix}{m}{
  \luadirect{arabluatex.utffilesuffix(\luastringN{#1})}}
%    \end{macrocode}
% \end{macro}
% \begin{macro}{\ArbOutFile}
%   \changes{v1.13}{2018/08/27}{Silently exports its argument in the
%   selected external file.}
%   \cs{ArbOutFile}\oarg{newline}\marg{string} silently exports
%   \meta{string} to the external selected file. It may take |newline|
%   as an optional argument in which case a carriage return is
%   appended to |string|.
% \begin{macro}{\ArbOutFile*}
%   \cs{ArbOutFile*}\oarg{newline}\marg{string} does the same as
%   \cs{ArbOutFile} but also inserts \meta{string} in the current
%   |.tex| source file.
%    \begin{macrocode}
\NewDocumentCommand{\ArbOutFile}{s O{no} +m}{%
  \if@pkg@export%
  \IfBooleanTF{#1}{%
    #3\luadirect{arabluatex.tooutfile(\luastringN{#3}, "#2")}}{%
    \luadirect{arabluatex.tooutfile(\luastringN{#3}, "#2")}}%
  \else\IfBooleanTF{#1}{#3}{}\fi}
%    \end{macrocode}
% \end{macro}
% \end{macro}
% \begin{environment}{arabexport}
%   \changes{v1.13}{2018/08/27}{Processes and print its argument in
%   the current file and exports it in full Unicode in the external
%   selected \texttt{.tex} file.} The |arabexport| environment
%   processes and prints its argument unchanged to the current |.pdf|
%   file. Additionally, if \package{arabluatex} is loaded with the
%   |export| option, this argument is exported to the external
%   selected |.tex| file with Unicode in place of the original
%   |arabtex| or |buckwalter| strings.
%    \begin{macrocode}
\NewDocumentEnvironment{arabexport}{+b}{%
  \if@pkg@export%
  \par
  #1
  \luadirect{arabluatex.doexport("yes")}
  \luadirect{tex.sprint(arabluatex.arbtoutf(\luastringN{#1}))}
  \luadirect{arabluatex.doexport("no")}
  \else\par#1\fi
  }{\par}
%    \end{macrocode}
% \end{environment}
% \begin{environment}{arab@v@export} The |arab@v@export| environment
%   does for |arabverse| the same as |arabexport|. It is used
%   internally by |arabverse|.
%    \begin{macrocode}
\NewDocumentEnvironment{arab@v@export}{O{} +b}{%
  \setkeys[al]{verse}[width, gutter, color, utf, delim, mode,
  metre]{#1}
  \if@pkg@export\ifal@verse@export%
  \par
  #2
  \luadirect{arabluatex.doexport("arabverse")}
  \luadirect{tex.sprint(arabluatex.arbtoutf(\luastringN{#2}))}
  \luadirect{arabluatex.doexport("no")}
  \else\par#2\fi\else\par#2\fi
}{\par}
%    \end{macrocode}
% \end{environment}
% \begin{macro}{\arbpardir}
% \changes{v1.13}{2018/08/27}{Sets the direction of Arabic paragraphs
% once they are converted to Unicode.} \cs{arbpardir} is automatically
% inserted by \package{arabluatex} at the beginning of Arabic
% paragraphs converted to Unicode so that they are printed in the
% right direction.
%    \begin{macrocode}
\NewDocumentCommand{\arbpardir}{}{%
  \ifx\al@mode\al@mode@trans\setLR\else\setRL\fi}
%    \end{macrocode}
% \end{macro}
% 
% \subsection*{Errors and Warnings}
%    \begin{macrocode}
\newcommand{\al@warning}[1]{\PackageWarning{arabluatex}{#1}}
\newcommand{\al@error}[2]{\PackageError{arabluatex}{#1}{#2}}
\newcommand{\al@wrong@nesting}{\al@error{%
    (RL/LR)\string\footnote\space is not allowed\MessageBreak inside
    \string\RL{} and \string\RL{} commands}{%
    Get rid of the surrounding \string\RL{} or \string\LR{} command.}}
\newcommand{\al@wrong@mark}{\al@warning{%
    Unknown Arabic mark in \string\arbmark{}. Replaced
    with\MessageBreak <??>. Please check your code}}
%    \end{macrocode}
% 
% That is it. Say goodbye before leaving.
%
% \iffalse
%</package>
% \fi
%
% \subsection*{Patches}
% \label{sec:patches}
%
% \iffalse
%<*patch>
% \fi
%    \begin{macrocode}
\NeedsTeXFormat{LaTeX2e}
\ProvidesPackage{arabluatex-patch}%
[2016/11/14 v1.0 patches for arabluatex]
%    \end{macrocode}
% I have put in a separate |.sty| file external lines of code that
% I had to patch for a good reason. I hate doing this, and hopefully,
% most of these lines will disappear as soon as they are not required
% anymore.
%
% The following is taken from |latex.ltx|. I had to make this patch
% for I could not find a way to process the list environments in
% right-to-left mode. The {\LuaTeX} primitives \cs{bodydir} and
% \cs{pagedir} will eventually allow us to get rid of this:
%    \begin{macrocode}
\def\list#1#2{%
  \ifnum \@listdepth >5\relax
    \@toodeep
  \else
    \global\advance\@listdepth\@ne
  \fi
  \rightmargin\z@
  \listparindent\z@
  \itemindent\z@
  \csname @list\romannumeral\the\@listdepth\endcsname
  \def\@itemlabel{#1}%
  \let\makelabel\@mklab
  \@nmbrlistfalse
  #2\relax
  \@trivlist
  \parskip\parsep
  \parindent\listparindent
  \advance\linewidth -\rightmargin
  \advance\linewidth -\leftmargin
%    \end{macrocode}
% patch begins:
%    \begin{macrocode}
  \ifbool{al@rlmode}{\advance\@totalleftmargin \rightmargin}%
  {\advance\@totalleftmargin \leftmargin}
%    \end{macrocode}
% patch ends.
%    \begin{macrocode}
  \parshape \@ne \@totalleftmargin \linewidth
  \ignorespaces}
\def\@item[#1]{%
  \if@noparitem
    \@donoparitem
  \else
    \if@inlabel
      \indent \par
    \fi
    \ifhmode
      \unskip\unskip \par
    \fi
    \if@newlist
      \if@nobreak
        \@nbitem
      \else
        \addpenalty\@beginparpenalty
        \addvspace\@topsep
        \addvspace{-\parskip}%
      \fi
    \else
      \addpenalty\@itempenalty
      \addvspace\itemsep
    \fi
    \global\@inlabeltrue
  \fi
  \everypar{%
    \@minipagefalse
    \global\@newlistfalse
    \if@inlabel
      \global\@inlabelfalse
      {\setbox\z@\lastbox
       \ifvoid\z@
         \kern-\itemindent
       \fi}%
      \box\@labels
      \penalty\z@
    \fi
    \if@nobreak
      \@nobreakfalse
      \clubpenalty \@M
    \else
      \clubpenalty \@clubpenalty
      \everypar{}%
    \fi}%
  \if@noitemarg
    \@noitemargfalse
    \if@nmbrlist
      \refstepcounter\@listctr
    \fi
  \fi
%    \end{macrocode}
% patch begins:
%    \begin{macrocode}
  \ifbool{al@rlmode}{\sRLbox\@tempboxa{\makelabel{#1}}}{%
  \sbox\@tempboxa{\makelabel{#1}}}%
  \ifbool{al@rlmode}{\global\setbox\@labels\hbox dir TRT}%
  {\global\setbox\@labels\hbox}{%
%    \end{macrocode}
% patch ends.
%    \begin{macrocode}
    \unhbox\@labels
    \hskip \itemindent
    \hskip -\labelwidth
    \hskip -\labelsep
    \ifdim \wd\@tempboxa >\labelwidth
      \box\@tempboxa
    \else
      \hbox to\labelwidth {\unhbox\@tempboxa}%
    \fi
    \hskip \labelsep}%
  \ignorespaces}
%    \end{macrocode}
% This is adapted from Vafa Khalighi's \package{bidi} package. Thanks
% to him.
%    \begin{macrocode}
\long\def\sRLbox#1#2{\setbox#1\hbox dir TRT{%
  \color@setgroup#2\color@endgroup}}
%    \end{macrocode}
%
% \iffalse
%</patch>
% \fi
%
% \Finale
\endinput
