\documentclass[12pt,a4paper]{article}
\usepackage{url}
\usepackage[colorlinks,bookmarks=false]{hyperref}
\usepackage{fancyvrb}
\usepackage[utf8]{vietnam}
\usepackage{charter}
\usepackage{ifpdf}
\usepackage{microtype}
% \usepackage{graphicx}
% \usepackage{array}

\parindent0pt
\parskip.4\baselineskip
\def\arraystretch{1.3}
\hyperlinkfileprefix{}


\advance\topmargin by -2cm
\advance\textheight by 3cm
\advance\footskip by .5cm
\advance\oddsidemargin by -0.5cm
\advance\textwidth by 1cm

\ifx\abbrloaded\relax
    \let\next=\endinput
\else
    \let\next=\relax
\fi

\next

\let\abbrloaded=y

\def\<#1>{%
    \expandafter\ifx\csname<#1>\endcsname\relax
        \errmessage{abbreviation <#1> undefined!}%
    \else
        \csname<#1>\endcsname
    \fi
}

\def\abbrA#1#2#3{%
    \expandafter\def\csname<#1>\endcsname{#2}%
}

\def\abbrB#1#2#3{%
    \def\abbrdef{#3}%
    \ifx\abbrdef\empty
        \expandafter\def\csname<#1>\endcsname{#2}%
    \else
        \expandafter\def\csname<#1>\endcsname{#3}%
    \fi
}

\ifx\HCode\undefined % tex4ht is not being used
    \let\abbr=\abbrA
\else 
    \let\abbr=\abbrB
\fi

\abbr{.}{.\,}{}
\abbr{3B2}{3B2}{}
\abbr{A2AC}{\texttt{a2ac}}{}
\abbr{ADOBE}{Adobe}{}
\abbr{AFM}{AFM}{}
\abbr{AFM}{AFM}{}
\abbr{AMIGA}{Amiga}{}
\abbr{AND}{\char038\relax}{&}
\abbr{APACHE}{Apache}{}
\abbr{AR}{Acrobat Reader}{}
\abbr{ASCII}{ASCII}{}
\abbr{BASH}{Bash}{}
\abbr{BLUESKY}{BlueSky}{}
\abbr{BS}{\char92}{}
\abbr{bull}{$\bullet$}{}
\abbr{C}{C}{}
\abbr{CMACTEX}{CMac\TeX}{CMacTeX}
\abbr{CM}{Computer Modern}{}
\abbr{CMR}{CMR}{}
\abbr{CMSS}{CMSS}{}
\abbr{CMSUPER}{CM-Super}{}
\abbr{CPP}{C++}{}
\abbr{CS}{CS}{}
\abbr{DANTE}{DANTE}{}
\abbr{DEBIAN}{Debian}{}
\abbr{DHSP}{�HSP}{}
\abbr{DJBDNS}{\textsf{djbdns}}{}
\abbr{DJGPP}{DJGPP}{}
\abbr{DNSCACHE}{\textsf{dnscache}}{}
\abbr{DOS}{DOS}{}
\abbr{DOTNET}{Visual~.NET}{}
\abbr{...}{\dots}{}
\abbr{DTP}{DTP}{}
\abbr{DVI}{DVI}{}
\abbr{DVIPDFM}{\textsf{dvipdfm}}{}
\abbr{DVIPDFMX}{\textsf{dvipdfmx}}{}
\abbr{DVIPS}{\textsf{dvips}}{}
\abbr{EC}{EC}{}
\abbr{EK}{\textit{{\Large$\varepsilon$\kern-.1em}k}}{}
\abbr{ELIB}{eLib}{}
\abbr{EMACS}{Emacs}{}
\abbr{EMAIL}{Email}{}
\abbr{EOF}{EOF}{}
\abbr{ET5}{ET5}{}
% \abbr{ETEX}{$\varepsilon$-\TeX}{eTeX}
\abbr{ETEX}{e\TeX}{eTeX}
\abbr{EUROTEX}{Euro\TeX}{EuroTeX}
\abbr{FLOITEX}{Floi\TeX}{}
\abbr{FMP}{FMP}{}
\abbr{FONTINST}{\texttt{fontinst}}{}
\abbr{FONTLAB}{FontLab}{}
\abbr{FPTEX}{fp\TeX}{}
\abbr{FTP}{Ftp}{}
\abbr{GCC}{GCC}{}
\abbr{GOOGLE}{Google}{}
\abbr{GS}{\textsf{ghostscript}}{}
\abbr{GUST}{GUST}{}
\abbr{GUT}{GUTenberg}{}
\abbr{GVSBK}{GVSBK}{}
\abbr{HCMUP}{HCMUP}{}
\abbr{HJ}{H\kern.1em<AND>\kern.1emJ}{}
\abbr{HREF}{hyperref}{}
\abbr{HSQL}{HSQL}{}
\abbr{HTML}{HTML}{}
\abbr{HZ}{\textit{hz}}{}
\abbr{ID}{InDesign}{}
\abbr{IIS}{IIS}{}
\abbr{INTERNET}{Internet}{}
\abbr{JAVA}{Java}{}
\abbr{JP}{\textit{jp}}{}
\abbr{KF}{\textit{kf\kern-.05em}}{}
\abbr{KR}{\textit{K$\varrho$}}{}
\abbr{LATEX}{\LaTeX}{LaTeX}
\abbr{LDAP}{LDAP}{}
\abbr{LF}{\textrm{\it letter\!\_\kern.1emfit}}{}
\abbr{LIBPNG}{LIBPNG}{}
\abbr{LIBTIFF}{LIBTIFF}{}
\abbr{LINUX}{Linux}{}
\abbr{LISP}{LISP}{}
\abbr{LM}{LM}{LM}
\abbr{M2}{\,m$^2$}{}
\abbr{MAC}{Macintosh}{}
\abbr{MF}{\MF}{}
\abbr{MG}{MetaFog}{}
\abbr{MIKTEX}{Mik\TeX}{MikTeX}
\abbr{MIRKA}{Miroslava Mis\'akov\'a}{}
\abbr{MMINSTANCE}{MMInstance}{}
\abbr{MM}{Multiple Master}{}
\abbr{MMTOOLS}{MMTOOLS}{}
\abbr{MP}{\MP}{}
\abbr{MVISCII}{Mac VISCII}{}
\abbr{MYSQL}{MySQL}{}
\abbr{NL}{\hfil\break}{}
\abbr{NTG}{NTG}{}
\abbr{NTS}{NTS}{}
\abbr{OMEGA}{$\Omega$}{}
\abbr{OPENTYPE}{OpenType}{}
\abbr{PASCAL}{Pascal}{}
\abbr{PDFETEX}{pdf\<ETEX>}{pdfeTeX}
\abbr{PDF}{PDF}{}
\abbr{PDFTEX}{pdf\TeX}{pdfTeX}
\abbr{PDFLATEX}{pdf\LaTeX}{pdfLaTeX}
\abbr{PDFXTEX}{pdfx\kern-.1em\TeX}{pdfxTeX}
\abbr{CONTEXT}{Con\TeX{}t}{ConTeXt}
\abbr{PERCENT}{\unskip\,\%}{}
\abbr{PERL}{Perl}{}
\abbr{PFA}{PFA}{}
\abbr{PFB}{PFB}{}
\abbr{PHP}{PHP}{}
\abbr{PK}{PK}{}
\abbr{PLAIN}{plain \TeX}{plain TeX}
\abbr{POSTGRESQL}{PostgreSQL}{}
\abbr{PROSPER}{Prosper}{}
\abbr{PS}{PS}{}
\abbr{RA}{$\longrightarrow$}{-->}
\abbr{RESIN}{Resin}{}
\abbr{SGML}{SGML}{}
\abbr{SP}{\hskip1cm}{}
\abbr{STL}{STL}{}
\abbr{T1}{Type\nobreak\,1}{Type1}
\abbr{T3}{Type\nobreak\,3}{}
\abbr{T5}{T5}{}
\abbr{TCVN}{TCVN1}{}
\abbr{TCX}{TCX}{}
\abbr{TETEX}{\textsf{te\TeX}}{teTeX}
\abbr{TEX4HT}{\TeX{}4ht}{TeX4ht}
\abbr{TEXINFO}{\texttt{texinfo}}{}
\abbr{TEXLIVE}{\TeX{}Live}{TeXLive}
\abbr{TEXME}{\TeX{}Me}{TeXMe}
\abbr{TEXMF}{\textsf{texmf}}{}
\abbr{TEXNICCENTER}{TeXnicCenter}{}
\abbr{TEX}{\TeX}{TeX}
\abbr{TEXTRACE}{\TeX{}trace}{TeXtrace}
\abbr{TFM}{TFM}{}
\abbr{TFTOPL}{TFtoPL}{}
% \abbr{THANH}{H\`an Th\^e\llap{\raise 0.5ex\hbox{\'{}}} Th\`anh}{Han The Thanh}
% \abbr{THANH}{H\`an Th\^e\llap{\raise 0.5ex\hbox{\'{}}} Th\`anh}{}
\abbr{THANH}{H\`an Th\'\ecircumflex{} Th\`anh}{}
\abbr{TINYDNS}{\textsf{tinydns}}{}
\abbr{TOMCAT}{Tomcat}{}
\abbr{TPHCM}{Tp.\,HCM}{}
\abbr{TRUETYPE}{True\kern-.1em Type}{TrueType}
\abbr{TUG}{TUG}{}
\abbr{UNICODE}{Unicode}{}
\abbr{UNIKEY}{Unikey}{}
\abbr{UNIX}{UNIX}{}
\abbr{UPORTAL}{uPortal}{}
\abbr{URW}{URW}{}
\abbr{URWVN}{URWVN}{}
\abbr{UTF8}{UTF8}{}
\abbr{VB}{Visual Basic}{}
\abbr{VC6}{Visual~C++~6.0}{}
\abbr{VIETLUG}{VietLUG}{}
\abbr{VIM}{Vim}{}
\abbr{VIM}{Vim}{}
\abbr{VISCII}{VISCII}{}
\abbr{VI}{Vi}{}
\abbr{VNCMR}{\textsf{vncmr}}{}
\abbr{VNI}{VNI}{}
\abbr{VNR}{VNR}{}
\abbr{VNTEX}{V\kern-.1em n\TeX}{VnTeX}
\abbr{VPS}{VPS}{}
\abbr{WC}{Windows Commander}{}
\abbr{WEB}{Web}{}
\abbr{WIN32}{Win32}{}
\abbr{WINDOWS}{Windows}{}
\abbr{WINEDT}{WinEdt}{}
\abbr{WWW}{WWW}{}
\abbr{XEMACS}{XEmacs}{}
\abbr{XEMTEX}{Xem\TeX}{XemTeX}
\abbr{XML}{XML}{}
\abbr{XPDF}{XPDF}{}
\abbr{YANDY}{Y<AND>Y}{}
\abbr{ZLIB}{ZLIB}{}
\abbr{EMAIL}{Email}{}
\abbr{WEBSITE}{Website}{}
\abbr{ASP}{ASP}{}
\abbr{FRONTPAGE}{FrontPage}{}
\abbr{DRW}{DreamWeaver}{}
\abbr{ABC}{ABC}{}
\abbr{VNI}{VNI}{}
\abbr{CTAN}{CTAN}{}
\abbr{PDFCPROT}{\textsf{pdfcprot}}{}
\abbr{PDFEXPAND}{\textsf{pdfexpand}}{}
\abbr{MICROTYPE}{\textsf{microtype}}{}
\abbr{PDFFONTS}{\textsf{pdffonts}}{}
\abbr{AFM2TFM}{\textsf{afm2tfm}}{}
\abbr{VF}{VF}{}
\abbr{TUGBOAT}{TUGboat}{}
\abbr{PSTRICKS}{\textsf{PStricks}}{}
\abbr{WEB2C}{\textsf{web2c}}{}
\abbr{IE}{i.\,e.\,\ignorespaces}{}
\abbr{ie}{i.\,e.\,\ignorespaces}{}
\abbr{GNU}{GNU}{}
\abbr{LIBAVL}{\textsf{libAVL}}{}
\abbr{TTF2AFM}{\textsf{ttf2afm}}{}
\abbr{PDFSYNC}{\textsf{pdfsync}}{}
\abbr{CFF}{CFF}{}
\abbr{GNOME}{GNOME}{}
\abbr{SGLUG}{SaigonLUG}{}
\abbr{SAIGON}{S�i G�n}{}
\abbr{SQL}{SQL}{}
\abbr{RVT}{RVT}{}
\abbr{XML2PDF}{XML2PDF}{}
\abbr{GUI}{GUI}{}
\abbr{NFSS}{NFSS}{}
\abbr{VNOSS}{VnOSS}{}
\abbr{GPL}{GPL}{}
\abbr{LPPL}{LPPL}{}
\abbr{AFPL}{AFPL}{}
\abbr{X11}{X11}{}
\abbr{XETEX}{Xe\TeX}{}
\abbr{EXTEX}{Ex\TeX}{}
\abbr{LUATEX}{Lua\TeX}{}
\abbr{XDVI}{XDvi}{}
\abbr{YAP}{Yap}{}
\abbr{DVIPSONE}{DVIPSONE}{}
\abbr{PLNFSS}{PLNFSS}{}
\abbr{MSWORD}{MS~Word}{}
\abbr{OOWRITER}{OpenOffice~Writer}{}
\abbr{THAIHOA}{Th\'ai Ph\'u Kh\'anh H\`oa}{}
\abbr{TEXMAKER}{TeXMaker}{}
\abbr{WINSHELL}{WinShell}{}
\abbr{XUNIKEY}{XUniKey}{}
\abbr{XVNKB}{Xvnkb}{}
\abbr{URL}{URL}{}
\abbr{KILE}{Kile}{}
\abbr{LM}{Latin Modern}{}
\abbr{SRC}{SRC}{}

\endinput


%% These abbrs are are not defined in abbr.tex:
\abbr{FD}{FD}{}
\abbr{MAKEINDEX}{MakeIndex}{}
\abbr{LATEXPIX}{LaTeXPIX}{}

\newcommand{\htmlheader}
{Hỗ trợ tiếng Việt cho \<TEX>}

\newcommand{\htmlfooter}{}

\begin{document}
\setcounter{tocdepth}{1}
\setcounter{secnumdepth}{1}
\DefineShortVerb{\|}

\title{\<VNTEX>}

% \noindent
% Chào bạn đã đến với trang web của \<VNTEX>.  

% \HCode{<h2>}
% Các tin đáng chú ý
% \HCode{</h2>}
% \begin{itemize}
% \item 06/10/2006: thêm phần hướng dẫn dịch bản tiếng Việt của 
% \href{http://www.tug.org/tex-archive/info/Free_Math_Font_Survey/survey.html}
% {Free Math Font Survey} ở phần \hyperlink{tai-lieu-vntex}{Tài liệu}

% \item 05/10/2006: cập nhật bản dịch Free Math Font Survey tại 
% \href{doc/survey-vn.pdf}{trang chủ} và
% \href{http://vntex.sarovar.org/survey-vn.pdf}{mirror tại sarovar}

% \item 04/09/2006: công bố \href{download/vntex-beta}{vntex-beta-20060904}

% \item 19/08/2006: công bố phiên bản \<TRUETYPE> của bộ font \<URWVN>-3.03
% tại \href{http://forum.vnoss.org/viewtopic.php?id=3334}{\<VNOSS>}. Xem 
% \href{fonts/urwvn-ttf}{hướng dẫn cài đặt}.

% \item 08/08/2006: hoàn thành việc Việt hóa các text font trong
% tài liệu
% \href{http://www.tug.org/tex-archive/info/Free_Math_Font_Survey/survey.html}
% {Free Math Font Survey} và công bố
% \href{fonts/samples/survey-vn.pdf}{bản dịch}

% % \item 18/07/2006: cập nhật mục \hyperlink{du-dinh-phat-trien}{dự
% % định phát triển}.

% \item 16/07/2006: thêm mẫu font mới: 
% VnGaramond
% (\href{http://vntex.sf.net/fonts/samples/garamondvn-test.pdf}{test},
%  \href{http://vntex.sf.net/fonts/samples/garamondvn-sample.pdf}{sample})
% và VnGrotesq
% (\href{http://vntex.sf.net/fonts/samples/grotesqvn-test.pdf}{test},
%  \href{http://vntex.sf.net/fonts/samples/grotesqvn-sample.pdf}{sample}). Các
%  font này sẽ được công bố để người dùng tải về trong thời gian tới.

% \item 13/07/2006: hướng dẫn cách tạo ``searchable'' \<PDF> với tiếng Việt. 
% Xem \hyperlink{cmap-vn}{chi tiết}.

% \item 24/06/2006: hỗ trợ \<UNICODE> bookmarks cho \<PDF>.
% Xem \hyperlink{bookmark-vn}{chi tiết}.

% \item 16/06/2006: công bố phiên bản chính thức mới của trang web \<VNTEX>

% \item 05/06/2006: công bố phiên bản thử nghiệm mới của trang web
% \<VNTEX>, cùng với một số gói mới: \hyperlink{more-extsizes}{more-extsizes}
% và \hyperlink{makeindex-vn}{makeindex-vn}.

% \item 03/06/2006: sửa lỗi các tập \<FD> cho font \<VNR>. Xem 
% \hyperlink{cap-nhat-vnr-fd}{chi tiết}.

% \item 26/05/2006: công bố phiên bản \<TRUETYPE> của bộ font \<URWVN>. Xem 
% \href{fonts/urwvn-ttf}{chi tiết}.

% \item 20/09/2005: công bố \<VNTEX>-3.02

% \item 20/09/2005: công bố \href{fonts/samples}{mẫu font}
% cho \<VNTEX>

% \item 01/12/2004: \<CONTEXT> đã
% \href{http://vnoss.org/forum/viewtopic.php?id=506}{hỗ trợ tiếng Việt}

% \end{itemize}

{\huge\centering\htmlheader\par}
%\vspace*{10pt}
\begin{center}
  Hàn Thế Thành
\end{center}

\section{Giới thiệu}
% \subsection{\<VNTEX>, \<LATEX>, \<CONTEXT>,\<...> là gì vậy?}
% If you are new to \<TEX>, see \url{http://en.wikipedia.org/wiki/TeX} for a
% basic explanation of \<TEX> and related programs. The \<TUG> (\<TEX> Users
        % roup) homepage at \url{http://tug.org} also contains useful introductory
% material to \<TEX>.

\subsection{\<VNTEX> là gì?}
\<VNTEX> là một gói chứa các hỗ trợ cần thiết cho việc sử dụng tiếng Việt
với \<TEX>. Trang web này chứa các thông tin về gói \<VNTEX> và các vấn đề
liên quan đến tiếng Việt trong \<TEX>. Chắc hẳn khi tìm đến với \<VNTEX>
bạn đã ít nhiều biết đến thế giới \<TEX>, và bạn cần \<VNTEX> cho việc dùng
tiếng Việt với \<TEX>.

Nếu bạn chưa từng sử dụng \<TEX>, có lẽ trang web này không
thích hợp lắm cho việc bắt đầu tìm hiểu về \<TEX>; bạn có thể
tham khảo một số địa chỉ sau:
\begin{enumerate}
\item Tiếng Việt:
\begin{itemize}
\item \href{http://forum.vnoss.org/viewforum.php?id=10}{Diễn đàn \<VNOSS>}
\item \href{http://viettug.org}{ViệtTUG}
\item \href{http://vi.wikipedia.org/wiki/TeX}{Định nghĩa \<TEX> tại
wikipedia bằng tiếng Việt} 
\end{itemize}
\item Tiếng Anh:
\begin{itemize}
\item \href{http://tug.org}{TUG}
\item \href{http://en.wikipedia.org/wiki/TeX}{Định nghĩa \<TEX> tại
wikipedia bằng tiếng Anh}
\end{itemize}
\end{enumerate}

Còn nếu bạn đang tìm câu trả lời cho một vấn đề trong \<TEX> nhưng không
liên quan đến tiếng Việt (ví dụ làm sao để gõ một công thức hay chèn một
hình ảnh vào văn bản) thì bạn có thể tìm hiểu tại các diễn đàn hay
đọc các tài liệu trong mục \hyperlink{tro-giup}{Tài liệu}.

Gói \<VNTEX> chứa các thành phần sau:
\begin{itemize}
\item các font tiếng Việt,
\item hỗ trợ tiếng Việt cho \<LATEX> (input encoding + font encoding)
\item một số (ít) tài liệu và ví dụ, cùng với các mẫu font hỗ trợ tiếng
Việt
\end{itemize}

\<VNTEX> được xây dựng tuân theo các qui ước chung của cộng đồng người dùng
\<TEX>. Việc này nhằm hạn chế bớt các xung đột có thể xảy ra khi dùng
\<VNTEX> với các ngôn ngữ hay các gói khác, cũng như giúp việc tích hợp
\<VNTEX> vào các hệ thống \<TEX> thuận lợi hơn.

\<VNTEX> đã được tích hợp vào một số hệ thống \<TEX> thông dụng như
\<TEXLIVE>, \<TETEX> và \<MIKTEX>.

\subsection{Tóm tắt lịch sử phát triển}
\begin{itemize}
\item 01/2000 -- phát hành phiên bản đầu tiên của \<VNTEX> (chưa được đánh số).
Phiên bản này chưa có các font \<T1>.

\item 08/2002 -- phát hành phiên bản 1.2 của \<VNTEX>. Bản này đã chứa dạng
\<T1> của các font \<VNR> (được tạo ra tự động bằng \<TEXTRACE>, chất lượng
tạm dùng được). Các font \<PS> thông dụng cũng được hỗ trợ tiếng Việt qua
cơ chế ``virtual font'' (dấu khá xấu do dùng các ký tự có sẵn để vẽ).

\item 03/2003 -- phát hành phiên bản 2.0 của \<VNTEX>. Phiên bản này đã
chứa định dạng \<T1> của các font \<VNR>, dựa trên các font \<T1> \<CMR> do
\<BLUESKY> thực hiện. Các font \<URWVN> cũng được phát hành trong phiên bản
này. Chi tiết về việc tạo ra các font này có thể xem tại
\href{http://www.tug.org/TUGboat/Articles/tb24-1/thanh.pdf}{đây}. Phiên bản
\<VNTEX> này sau đó đã được đưa vào \<TETEX>, \<MIKTEX> và \<TEXLIVE>.

\item 09/2005 -- phát hành phiên bản 3.02 của \<VNTEX>.
\end{itemize}

\subsection{Tác giả}
Phần lớn \<VNTEX> do \<THANH> viết, phần hỗ trợ cho \<LATEX> do Werner
Lemberg viết. Ngoài ra còn có sự đóng góp của nhiều người khác (xin xem
tiếp ở mục sau).

\subsection{Những người tham gia đóng góp cho \<VNTEX>}
Những người trong danh sách dưới đây (theo thứ tự ABC) đã tham gia đóng
góp cho sự phát triển của \<VNTEX> dưới các hình thức khác nhau: 

Huỳnh Kỳ Anh,
Nguyễn Đại Quý,
Nguyễn Phi Hùng,
Nguyễn Tân Khoa,
Reinhard Kotucha,
Thái Phú Khánh Hòa,
Ulrich Dirr,
Vladimir Volovich.

Nếu bạn thấy cần thêm ai vào danh sách này xin vui lòng góp ý
cho chúng tôi.

\subsection{Quản lý}
Hiện nay gói \<VNTEX> do \<THANH>, Werner Lemberg và Reinhard Kotucha quản
lý.

\subsection{Giấy phép và bản quyền}
Các font \<URWVN> và Bitstream được phân phối theo giấy phép
\<GPL>, các thứ còn lại theo giấy phép \<LPPL> ($\ge$\,1.3).

Vui lòng xem \href{http://www.gnu.org/licenses/gpl.txt}{\<GPL>} và
\href{http://www.latex-project.org/lppl.txt}{\<LPPL>} nếu cần biết thêm chi tiết.

\subsection{Về trang web này}
Trang web này do \<THANH> và \<THAIHOA> xây dựng và quản lý, cùng với
sự đóng góp của Tôn Nữ Thục Anh. Riêng phần
\href{http://vntex.sf.net/download}{download} do Reinhard Kotucha quản
lý. Mọi ý kiến đóng góp xin gởi đến |hanthethanh| hoặc |h2vnteam| tại
|gmail| chấm |com|.

\section{Cài đặt và cập nhật}
\subsection{Cài đặt}
Phiên bản chính thức mới nhất của \<VNTEX> được công bố tại
\href{http://vntex.sf.net/download/vntex}{đây}, có hướng dẫn cài đặt kèm theo.  Tuy nhiên
bạn nên kiểm tra kỹ xem hệ thống \<TEX> bạn đang sử dụng đã có sẵn
\<VNTEX> chưa. Nếu bạn dùng \<UNIX> thì nên dùng \<TEXLIVE>
($\ge$\,2005), còn nếu dùng \<WINDOWS> thì nên dùng \<MIKTEX>
($\ge$\,2.5). Làm như vậy bạn sẽ có sẵn \<VNTEX> mà không cần phải tự
cài đặt. Lưu ý là bản \<VNTEX> có trong \<TETEX>-3.0 có một số trục
trặc khi dùng mã \<UTF8>; nếu bạn dùng \<TETEX> thì nên tự cài đặt
thêm \<VNTEX> phiên bản $\ge$\,3.02, hoặc chuyển sang dùng \<TEXLIVE>.

Ngoài ra, để sử dụng được \<VNTEX> bạn cần phải có một trình soạn thảo để
gõ và hiển thị được tiếng Việt. Việc cài đặt những thứ này phụ thuộc vào
từng hệ thống nên chúng tôi không mô tả chi tiết ở đây được. Chúng tôi chỉ
có vài gợi ý cho người mới làm quen:
\begin{itemize}
\item Nếu bạn dùng \<WINDOWS>, hãy chọn:
\begin{itemize}
\item hệ thống \<TEX>: \<MIKTEX>
\item bộ gõ tiếng Việt: \<UNIKEY>
\item trình soạn thảo: \<TEXMAKER> hoặc \<WINSHELL>
\end{itemize}

\item Nếu bạn dùng \<UNIX>, hãy chọn:
\begin{itemize}
\item hệ thống \<TEX>: \<TEXLIVE>
\item bộ gõ tiếng Việt: \<XUNIKEY>  hoặc \<XVNKB>
\item trình soạn thảo: \<TEXMAKER> hoặc \<KILE>
\end{itemize}
\end{itemize}

Hướng dẫn cài đặt \<VNTEX> cụ thể cho từng hệ thống hiện chưa có. Nếu bạn
biết có tài liệu nào viết về vấn đề này, hoặc bạn muốn viết hướng dẫn cài
đặt \<VNTEX> cho 1 hệ thống cụ thể, xin vui lòng liên hệ với chúng tôi.

\subsection{Cập nhật}
\hypertarget{cap-nhat}{}
Đây là nơi công bố các gói sửa lỗi cho bản \<VNTEX> đã chính thức công bố,
hoặc những thành phần mới của \<VNTEX> chưa được tích hợp vào bản chính
thức. Những thứ công bố ở đây sẽ được đưa vào phiên bản \<VNTEX> kế tiếp.

\begin{description}
\item[Bản sửa lỗi cho các tập \<FD> của bộ font \<VNR>:]
  \hypertarget{cap-nhat-vnr-fd}{} sửa một số lỗi trong các tập \<FD>
  của font \<VNR>. Tải gói
  \href{http://vntex.sf.net/download/vntex-updates/vntex-update-20060603.zip}{này}
  về, sau đó bung nén và ghi đè các tập này lên các tập \<FD> của
  \<VNTEX>.
\end{description}


\section{Tài liệu}
\hypertarget{tai-lieu}{}
Đây là nơi thu thập các tài liệu tiếng Việt về \<TEX> và các  chủ đề liên
quan. Đang được cập nhật và sắp xếp dần.

\subsection{Tài liệu viết cho \<VNTEX>} 
\hypertarget{tai-lieu-vntex}{}
Bao gồm các tài liệu về các chủ đề liên quan trực tiếp đến \<VNTEX>.
\begin{description}
\item [Hướng dẫn sử dụng font với \<VNTEX>:]
Tài liệu hướng dẫn sử dụng font với \<VNTEX> do \<THANH> và \<THAIHOA>
viết. Xem:
\href{http://vntex.sf.net/doc/vn-fonts.html}{\<HTML>},
\href{http://vntex.sf.net/doc/vn-fonts.pdf}{\<PDF>},
\href{http://vntex.sf.net/doc/vn-fonts-src.zip}{\<SRC>}.

\item [Dịch ``Free Math Font Survey'' với \<MIKTEX>:]
Tài liệu mô tả các bước cần thiết để dịch bản tiếng Việt của tài liệu
\href{http://www.tug.org/tex-archive/info/Free_Math_Font_Survey/survey.html}
{Free Math Font Survey} với \<MIKTEX>-2.5. Xem:
\href{http://vntex.sf.net/doc/survey-vn-miktex.html}{\<HTML>},
\href{http://vntex.sf.net/doc/survey-vn-miktex.pdf}{\<PDF>}.
\end{description}


\subsection{Tài liệu viết cho \<LATEX>} 
Bao gồm các tài liệu được viết cho \<LATEX> và các vấn đề liên quan, được
dịch từ tiếng Anh.

\subsubsection{Các tài liệu hướng dẫn sử dụng \<LATEX>} 
\begin{description}
\item[Giới thiệu ngắn về \<LATEX>2e:]
Tài liệu ``Giới thiệu ngắn về \<LATEX>2e'' (A not so short
introduction to \<LATEX>) của Tobias Oetiker 
do Nguyễn Tân Khoa dịch. Tải về:
\href{http://vntex.sf.net/doc/lshort-vn.pdf}{\<PDF>},
\href{http://vntex.sf.net/doc/lshort-vn-src.zip}{\<SRC>}.

\item[Giáo trình \<LATEX>:]
Tài liệu ``Giáo trình \<LATEX>'' (A course of \<LATEX>) của Gary
L.~Gray do Nguyễn Phi Hùng dịch. Tải về:
\href{http://vntex.sf.net/doc/latex-course-vn.pdf}{\<PDF>},
\href{http://vntex.sf.net/doc/latex-course-vn-src.zip}{\<SRC>}.

\item[Hướng dẫn viết luận án bằng \<LATEX>:]
Tài liệu hướng dẫn thiết kế luận án tốt nghiệp bằng \<LATEX> được biên soạn bởi by
Dr.~Nicola Talbot và do \<THAIHOA> dịch. Có thể tải về từ
\href{http://theoval.cmp.uea.ac.uk/~nlct/latex/thesis_viet/index.html}{đây}.
\end{description}

% \subsubsection{Các tài liệu hướng dẫn sử dụng font với \<LATEX>2e} 
% \begin{description}
% \item[
% \end{description}

\subsection{Tài liệu cho một số công cụ liên quan}
\begin{description}
\item[Hướng dẫn sử dụng \<TEXMAKER>]
\<TEXMAKER> là một chương trình soạn thảo \<TEX> chạy trên \<UNIX>/\<LINUX>
và \<WINDOWS>, hỗ trợ \<UNICODE> và thích hợp với người dùng mới bắt đầu. Hướng dẫn
sử dụng \<TEXMAKER> do \<THAIHOA> dịch. Tải về:  
\href{http://vntex.sf.net/doc/texmaker-vn.pdf}{\<PDF>}.

\item[Hướng dẫn sử dụng \<LATEXPIX>]
\<LATEXPIX> một công cụ dùng để vẽ hình cho \<LATEX>. Phần mềm này 
thích hợp cho những ai mới bắt đầu với \<LATEX>. 
Tài liệu hướng dẫn sử dụng do \<THAIHOA> dịch. Tải về:
\href{http://vntex.sf.net/doc/LaTeXPiX-vn.pdf}{\<PDF>}, hoặc bản \<PDF>
\href{http://members.home.nl/nickvanbeurden/LaTeXPiX_vietnamese.pdf}{gốc}.
\end{description}


\section{Vấn đề liên quan}
\hypertarget{van-de-lien-quan}{}
\subsection{Sử dụng tiếng Việt với \<CONTEXT>}
\<CONTEXT> đã hỗ trợ tiếng Việt. Xem
\href{http://forum.vnoss.org/viewtopic.php?id=506}{tại đây} để biết thêm chi tiết

\hypertarget{more-extsizes}{}
\subsection{Sử dụng gói \texttt{extsizes} với \<VNTEX>}
Để sử dụng gói |extsizes| với \<VNTEX>, trước tiên cần phải cập nhật các
tập \<FD> cho các font \<VNR>. Xem thêm 
\href{http://vntex.sf.net/download/vntex-updates}{chi tiết}.

Để sử dụng được gói |extsizes| với các cỡ chữ khác như 13pt và 13.5pt,
tải và cài đặt gói
\href{http://vntex.sf.net/download/vntex-support/more-extsizes.zip}{more-extsizes}
do \<THAIHOA> viết. Ví dụ:

\begin{verbatim}
\documentclass[13pt]{extarticle}
\usepackage[utf8]{vietnam}
\usepackage{type1cm}

\begin{document}
Văn bản này dùng cỡ chữ 13pt.
\end{document}
\end{verbatim}

Lưu ý \emph{phải} dùng gói |type1cm| nếu văn bản có chứa các công thức
toán.

Các cỡ chữ được gói |more-extsizes| hỗ trợ thêm: 13, 13p5, 15 và 16.

\hypertarget{makeindex-vn}{}
\subsection{Sử dụng \<MAKEINDEX> với \<VNTEX>}
\<MAKEINDEX> là một chương trình sắp xếp chỉ mục cho
\<LATEX>. \<MAKEINDEX> không hỗ trợ tiếng Việt nên để dùng
\<MAKEINDEX> với \<VNTEX> ta cần một chút ``mẹo''.

Nếu bạn dùng \<UNIX>/\<LINUX>, hãy tải gói
\href{http://vntex.sf.net/download/vntex-support/makeindex-vn.zip}{makeindex-vn
  cho \<UNIX>/\<LINUX>}. Người dùng \<WINDOWS> có thể sử dụng gói
\href{http://vntex.sf.net/download/vntex-support/vnindexwin.zip}{makeindex-vn
  cho \<WINDOWS>} Sau khi tải về hãy bung nén và xem hướng dẫn đi kèm
để biết cách sử dụng.

Các ký tự Việt được sắp xếp theo qui tắc mô tả tại 
\href{http://www.vietlex.com/vietnamese/quytacABC.html}{đây}.

% Hướng dẫn chi tiết cho \<WINDOWS> hiện chưa có. Nếu bạn muốn viết phần này
% chúng tôi rất hoan nghênh.

\subsection{Chuyển đổi từ \<LATEX> sang \<HTML>}
Trước đây \<VNTEX> có chứa hỗ trợ cho \<TEX4HT>, sau đó các hỗ trợ này đã
được đưa vào bản phân phối chính thức của \<TEX4HT> nên đã được xóa khỏi
\<VNTEX>.

Để chuyển từ \<LATEX> sang \<HTML> bạn có thể dùng lệnh sau:
\begin{verbatim}
htlatex filename.tex "html,uni-html4,charset=utf-8" " -cunihtf -utf8"
\end{verbatim}

Bạn phải có bản \<TEX4HT> tương đối mới, ít nhất là như bản \<TEX4HT> trong
\<TEXLIVE>2005. File \<LATEX> có thể dùng một trong các bảng mã mà
\<VNTEX> hỗ trợ (ví dụ \<VISCII> hay \<UTF8>). \<TEX4HT> sẽ tự động ghi kết
quả dưới mã \<UTF8>. 

Trang web này cũng được viết bằng \<LATEX> và chuyển sang \<HTML> bằng
\<TEX4HT>.   

\hypertarget{bookmark-vn}{}
\subsection{Bookmark cho \<PDF> và tiếng Việt}
Gói |hyperref| cho phép tạo \<UNICODE> bookmark cho \<PDF>, tuy nhiên một
số ký tự Việt chưa được hỗ trợ. Để khắc phục ta làm như sau:
\begin{itemize}
\item tải về gói
  \href{http://vntex.sf.net/download/vntex-support/puenc-vn.zip}{puenc-vn}

\item bung nén và chép tập |puenc.def| đè lên tập |puenc.def| của
  |hyperref| (hoặc bạn có thể chép tập này vào thư mục chứa văn bản
  của bạn)

\item dùng gói |hyperref| trong văn bản \<LATEX> của bạn theo ví dụ sau:

\begin{verbatim}
\documentclass{article}
\usepackage[unicode]{hyperref}
\usepackage[utf8]{vietnam}
\begin{document}
\section{Tiếng Việt}
\end{document}
\end{verbatim}

(lưu ý để bookmark hiển thị đúng bạn phải dịch văn bản \<LATEX> của bạn ít
nhất là 2 lần)
\end{itemize}

\<UNICODE> bookmark không ``portable'' -- bạn phải dùng \<AR> phiên
bản $\ge$\,5.0 mới xem được đầy đủ các ký tự.

Nếu bạn không dùng \<UNICODE> bookmark thì \<PDF> của bạn sẽ portable hơn
và bookmark sẽ hiển thị tốt với đa số trình duyệt \<PDF>. Tuy nhiên  một
số ký tự Việt không được hỗ trợ trong bảng mã dùng cho \<PDF> bookmark
(PD1), do đó các ký tự này sẽ bị mất khi tạo bookmark. Bạn có thể hạn chế điều
này bằng cách thay thế các ký tự Việt bằng các ký tự ``gần giống'' trong
bảng mã PD1 (ví dụ `ắ' sẽ được thay thế bằng `á') theo cách sau:

\begin{verbatim}
\documentclass{article}
\usepackage{hyperref}
\usepackage[utf8]{vietnam}
% Copyright 2003-2005 Han The Thanh <hanthethanh@gmx.net>.
% This file is part of vntex.  License: LPPL, version 1.3 or newer,
% according to http://www.latex-project.org/lppl.txt

\DeclareTextAccent{\h}{PD1}{\texthookabove}
\DeclareTextAccent{\d}{PD1}{\textdotbelow}
\DeclareTextCompositeCommand{\h}{PD1}{\@empty}{\texthookabove}
\DeclareTextCompositeCommand{\d}{PD1}{\@empty}{\textdotbelow}
\DeclareTextCommand{\texthookabove}{PD1}{\@empty}
\DeclareTextCommand{\textdotbelow}{PD1}{\@empty}

\DeclareTextCommand{\ABREVE}{PD1}{A}
\DeclareTextCommand{\ACIRCUMFLEX}{PD1}{A}
\DeclareTextCommand{\ABREVE}{PD1}{A}
\DeclareTextCommand{\ACIRCUMFLEX}{PD1}{A}
\DeclareTextCommand{\ECIRCUMFLEX}{PD1}{E}
\DeclareTextCommand{\ECIRCUMFLEX}{PD1}{E}
\DeclareTextCommand{\OCIRCUMFLEX}{PD1}{O}
\DeclareTextCommand{\OHORN}{PD1}{O}
\DeclareTextCommand{\OCIRCUMFLEX}{PD1}{O}
\DeclareTextCommand{\OHORN}{PD1}{O}
\DeclareTextCommand{\UHORN}{PD1}{U}
\DeclareTextCommand{\UHORN}{PD1}{U}
\DeclareTextCommand{\abreve}{PD1}{a}
\DeclareTextCommand{\acircumflex}{PD1}{a}
\DeclareTextCommand{\dj}{PD1}{d}
\DeclareTextCommand{\ecircumflex}{PD1}{e}
\DeclareTextCommand{\ocircumflex}{PD1}{o}
\DeclareTextCommand{\ohorn}{PD1}{o}
\DeclareTextCommand{\uhorn}{PD1}{u}


\DeclareTextComposite{\'}{PD1}{\ABREVE}{`A}
\DeclareTextComposite{\'}{PD1}{\ACIRCUMFLEX}{`A}
\DeclareTextComposite{\'}{PD1}{\ABREVE}{`A}
\DeclareTextComposite{\'}{PD1}{\ACIRCUMFLEX}{`A}
\DeclareTextComposite{\'}{PD1}{\ECIRCUMFLEX}{`E}
\DeclareTextComposite{\'}{PD1}{\ECIRCUMFLEX}{`E}
\DeclareTextComposite{\'}{PD1}{\OCIRCUMFLEX}{`O}
\DeclareTextComposite{\'}{PD1}{\OHORN}{`O}
\DeclareTextComposite{\'}{PD1}{\OCIRCUMFLEX}{`O}
\DeclareTextComposite{\'}{PD1}{\OHORN}{`O}
\DeclareTextComposite{\'}{PD1}{\UHORN}{`U}
\DeclareTextComposite{\'}{PD1}{\UHORN}{`U}
\DeclareTextComposite{\'}{PD1}{\abreve}{`a}
\DeclareTextComposite{\'}{PD1}{\acircumflex}{`a}
\DeclareTextComposite{\'}{PD1}{\ecircumflex}{`e}
\DeclareTextComposite{\'}{PD1}{\ocircumflex}{`o}
\DeclareTextComposite{\'}{PD1}{\ohorn}{`o}
\DeclareTextComposite{\'}{PD1}{\uhorn}{`u}
\DeclareTextComposite{\`}{PD1}{Y}{`Y}
\DeclareTextComposite{\`}{PD1}{\ABREVE}{`A}
\DeclareTextComposite{\`}{PD1}{\ACIRCUMFLEX}{`A}
\DeclareTextComposite{\`}{PD1}{\ABREVE}{`A}
\DeclareTextComposite{\`}{PD1}{\ACIRCUMFLEX}{`A}
\DeclareTextComposite{\`}{PD1}{\ECIRCUMFLEX}{`E}
\DeclareTextComposite{\`}{PD1}{\ECIRCUMFLEX}{`E}
\DeclareTextComposite{\`}{PD1}{\OCIRCUMFLEX}{`O}
\DeclareTextComposite{\`}{PD1}{\OHORN}{`O}
\DeclareTextComposite{\`}{PD1}{\OCIRCUMFLEX}{`O}
\DeclareTextComposite{\`}{PD1}{\OHORN}{`O}
\DeclareTextComposite{\`}{PD1}{\UHORN}{`U}
\DeclareTextComposite{\`}{PD1}{\UHORN}{`U}
\DeclareTextComposite{\`}{PD1}{\abreve}{`a}
\DeclareTextComposite{\`}{PD1}{\acircumflex}{`a}
\DeclareTextComposite{\`}{PD1}{\ecircumflex}{`e}
\DeclareTextComposite{\`}{PD1}{\ocircumflex}{`o}
\DeclareTextComposite{\`}{PD1}{\ohorn}{`o}
\DeclareTextComposite{\`}{PD1}{\uhorn}{`u}
\DeclareTextComposite{\`}{PD1}{y}{`y}
\DeclareTextComposite{\d}{PD1}{A}{`A}
\DeclareTextComposite{\d}{PD1}{E}{`E}
\DeclareTextComposite{\d}{PD1}{I}{`I}
\DeclareTextComposite{\d}{PD1}{O}{`O}
\DeclareTextComposite{\d}{PD1}{U}{`U}
\DeclareTextComposite{\d}{PD1}{Y}{`Y}
\DeclareTextComposite{\d}{PD1}{\ABREVE}{`A}
\DeclareTextComposite{\d}{PD1}{\ACIRCUMFLEX}{`A}
\DeclareTextComposite{\d}{PD1}{\ECIRCUMFLEX}{`E}
\DeclareTextComposite{\d}{PD1}{\OCIRCUMFLEX}{`O}
\DeclareTextComposite{\d}{PD1}{\OHORN}{`O}
\DeclareTextComposite{\d}{PD1}{\UHORN}{`U}
\DeclareTextComposite{\d}{PD1}{\abreve}{`a}
\DeclareTextComposite{\d}{PD1}{\acircumflex}{`a}
\DeclareTextComposite{\d}{PD1}{\ecircumflex}{`e}
\DeclareTextComposite{\d}{PD1}{\ocircumflex}{`o}
\DeclareTextComposite{\d}{PD1}{\ohorn}{`o}
\DeclareTextComposite{\d}{PD1}{\uhorn}{`u}
\DeclareTextComposite{\d}{PD1}{a}{`a}
\DeclareTextComposite{\d}{PD1}{e}{`e}
\DeclareTextComposite{\d}{PD1}{i}{`i}
\DeclareTextComposite{\d}{PD1}{o}{`o}
\DeclareTextComposite{\d}{PD1}{u}{`u}
\DeclareTextComposite{\d}{PD1}{y}{`y}
\DeclareTextComposite{\h}{PD1}{A}{`A}
\DeclareTextComposite{\h}{PD1}{E}{`E}
\DeclareTextComposite{\h}{PD1}{I}{`I}
\DeclareTextComposite{\h}{PD1}{O}{`O}
\DeclareTextComposite{\h}{PD1}{U}{`U}
\DeclareTextComposite{\h}{PD1}{Y}{`Y}
\DeclareTextComposite{\h}{PD1}{\ABREVE}{`A}
\DeclareTextComposite{\h}{PD1}{\ACIRCUMFLEX}{`A}
\DeclareTextComposite{\h}{PD1}{\ECIRCUMFLEX}{`E}
\DeclareTextComposite{\h}{PD1}{\OCIRCUMFLEX}{`O}
\DeclareTextComposite{\h}{PD1}{\OHORN}{`O}
\DeclareTextComposite{\h}{PD1}{\UHORN}{`U}
\DeclareTextComposite{\h}{PD1}{\abreve}{`a}
\DeclareTextComposite{\h}{PD1}{\acircumflex}{`a}
\DeclareTextComposite{\h}{PD1}{\ecircumflex}{`e}
\DeclareTextComposite{\h}{PD1}{\ocircumflex}{`o}
\DeclareTextComposite{\h}{PD1}{\ohorn}{`o}
\DeclareTextComposite{\h}{PD1}{\uhorn}{`u}
\DeclareTextComposite{\h}{PD1}{a}{`a}
\DeclareTextComposite{\h}{PD1}{e}{`e}
\DeclareTextComposite{\h}{PD1}{i}{`i}
\DeclareTextComposite{\h}{PD1}{o}{`o}
\DeclareTextComposite{\h}{PD1}{u}{`u}
\DeclareTextComposite{\h}{PD1}{y}{`y}
\DeclareTextComposite{\~}{PD1}{E}{`E}
\DeclareTextComposite{\~}{PD1}{I}{`I}
\DeclareTextComposite{\~}{PD1}{U}{`U}
\DeclareTextComposite{\~}{PD1}{Y}{`Y}
\DeclareTextComposite{\~}{PD1}{\ABREVE}{`A}
\DeclareTextComposite{\~}{PD1}{\ACIRCUMFLEX}{`A}
\DeclareTextComposite{\~}{PD1}{\ABREVE}{`A}
\DeclareTextComposite{\~}{PD1}{\ACIRCUMFLEX}{`A}
\DeclareTextComposite{\~}{PD1}{\ECIRCUMFLEX}{`E}
\DeclareTextComposite{\~}{PD1}{\ECIRCUMFLEX}{`E}
\DeclareTextComposite{\~}{PD1}{\OCIRCUMFLEX}{`O}
\DeclareTextComposite{\~}{PD1}{\OHORN}{`O}
\DeclareTextComposite{\~}{PD1}{\OCIRCUMFLEX}{`O}
\DeclareTextComposite{\~}{PD1}{\OHORN}{`O}
\DeclareTextComposite{\~}{PD1}{\UHORN}{`U}
\DeclareTextComposite{\~}{PD1}{\UHORN}{`U}
\DeclareTextComposite{\~}{PD1}{\abreve}{`a}
\DeclareTextComposite{\~}{PD1}{\acircumflex}{`a}
\DeclareTextComposite{\~}{PD1}{\ecircumflex}{`e}
\DeclareTextComposite{\~}{PD1}{\i}{`i}
\DeclareTextComposite{\~}{PD1}{\ocircumflex}{`o}
\DeclareTextComposite{\~}{PD1}{\ohorn}{`o}
\DeclareTextComposite{\~}{PD1}{\uhorn}{`u}
\DeclareTextComposite{\~}{PD1}{e}{`e}
\DeclareTextComposite{\~}{PD1}{i}{`i}
\DeclareTextComposite{\~}{PD1}{u}{`u}
\DeclareTextComposite{\~}{PD1}{y}{`y}

\begin{document}
\section{Tiếng Việt}
\end{document}
\end{verbatim}

\hypertarget{cmap-vn}{}
\subsection{Tạo ``searchable'' \<PDF> với tiếng Việt}
Để có thể tìm kiếm hay cắt/dán tiếng Việt với các file \<PDF>, bạn có thể
dùng gói |cmap| của Vladimir Volovich như sau:

\begin{verbatim}
\documentclass{article}
\usepackage{cmap}
\usepackage[utf8]{vietnam}
\begin{document}
Tiếng Việt
\end{document}
\end{verbatim}

Lưu ý phải nạp (load) gói |cmap| trước khi nạp các gói khác. Nếu bạn quên
điều này thì |cmap| sẽ ghi ra một số cảnh báo (warning) và file \<PDF> của
bạn có thể sẽ không ``searchable'' (không thể tìm kiếm hoặc cắt/dán).

Một số hạn chế: gói |cmap| chỉ dùng với \<PDFTEX> và không có tác dụng đối
với ``virtual font''. Điều này có nghĩa là bạn phải dùng \<PDFTEX> để dịch
văn bản, và các đoạn text dùng font smallcap (chọn bằng lệnh |\textsc|) sẽ
không searchable.

\section{Trợ giúp}
\hypertarget{tro-giup}{}
Dưới đây là một số địa chỉ bạn có thể tham khảo khi gặp một vấn
đề nào đó với \<VNTEX> (để nhanh chóng có được câu trả lời, xin vui lòng
dành ít phút để đọc tài liệu
\href{http://forum.vnoss.org/pub/smart-question-vi.html}{Cách đặt một câu
hỏi thông minh} nếu bạn chưa đọc).

\begin{itemize}
\item \<VNOSS> có diễn dàn dành cho
\href{http://forum.vnoss.org/viewforum.php?id=10}{\<TEX>/\<LATEX>}. Ở đây
bạn có thể hỏi về \<VNTEX>, \<CONTEXT> và các vấn đề liên quan đến tiếng
Việt trong \<TEX>. 

\item Trang web \href{http://viettug.org/}{ViệtTUG} do Huỳnh Kỳ Anh quản
lý, chứa nhiều thông tin hữu ích cho người muốn học sử dụng \<LATEX>.

\item Mailing list của \<VNTEX> có tại
\href{http://lists.sourceforge.net/lists/listinfo/vntex-users}{đây}
\end{itemize}


\end{document}
