%% \CharacterTable
%%  {Upper-case    \A\B\C\D\E\F\G\H\I\J\K\L\M\N\O\P\Q\R\S\T\U\V\W\X\Y\Z
%%   Lower-case    \a\b\c\d\e\f\g\h\i\j\k\l\m\n\o\p\q\r\s\t\u\v\w\x\y\z
%%   Digits        \0\1\2\3\4\5\6\7\8\9
%%   Exclamation   \!     Double quote  \"     Hash (number) \#
%%   Dollar        \$     Percent       \%     Ampersand     \&
%%   Acute accent  \'     Left paren    \(     Right paren   \)
%%   Asterisk      \*     Plus          \+     Comma         \,
%%   Minus         \-     Point         \.     Solidus       \/
%%   Colon         \:     Semicolon     \;     Less than     \<
%%   Equals        \=     Greater than  \>     Question mark \?
%%   Commercial at \@     Left bracket  \[     Backslash     \\
%%   Right bracket \]     Circumflex    \^     Underscore    \_
%%   Grave accent  \`     Left brace    \{     Vertical bar  \|
%%   Right brace   \}     Tilde         \~}
%%
%\iffalse
%
% (c) Copyright 2007-2022 Apostolos Syropoulos 
% This program can be redistributed and/or modified under the 
% terms of the LaTeX Project Public License Distributed from 
% http://www.latex-project.org/lppl.txt; either
% version 1.3c of the License, or any later version.
%  
% This work has the LPPL maintenance status `maintained'.
%
% Please report errors or suggestions for improvement to
%
%    Apostolos Syropoulos  (asyropoulos@yahoo.com)
%
%\fi
% \CheckSum{1930}
% \iffalse This is a Metacommentxel	
%
%<xgreek, >\ProvidesFile{xgreek.sty}
%<xelistings, >\ProvidesFile{xelistings}
%
%<xgreek, > [2022/09/04 v3.2.0 Package `xgreek.sty']
%<xelistings, > [2022/09/04 v1.0 Package `xelistings.sty']
%
%    \begin{macrocode}
%<*driver>
\documentclass{ltxdoc}
\GetFileInfo{xgreek.drv}
\usepackage{xltxtra}
\begin{document}
%\newopentypefeature{Contextuals}{Alts}{+calt}
\setmainfont[Mapping=tex-text,Script=Greek,
             SmallCapsFeatures={Contextuals=Alternate}]{Universal Modern}
\setmonofont{UM Typewriter}
\setsansfont[Mapping=tex-text]{GFS Neohellenic}
   \DocInput{xgreek.dtx}
\end{document}
%</driver>
%    \end{macrocode}
% \fi
%\StopEventually{}
%\title{Greek Language Support for\\ \XeLaTeX\ and Lua\LaTeX}
%\author{Apostolos Syropoulos\\
%        Xanthi, Greece\\
%        \texttt{asyropoulos@yahoo.com}}
% \date{2009/11/23\\ Last Updated 2022/09/04}
%\maketitle
% \begin{abstract}
% The \textsf{xgreek} package provides rudimentary support for Greek language
% typesetting with \XeLaTeX and Lua\LaTeX. In particular, it provides support for modern 
% Greek (either monotonic or polytonic) and ancient Greek.  
%\end{abstract}
%
%\section{Introduction}
%
% The \textsf{xgreek} package provides rudimentary support for Greek language
% typesetting with \XeLaTeX and Lua\LaTeX. Users will be able to typeset documents in 
% either modern Greek (monotonic or polytonic) or ancient Greek by selecting the appropriate
% package option. The default ``language'' is monotonic Greek.  
%
% Support for Lua\LaTeX\ was provided by Javier Bezos. 
%
% \section{The Source Code of \textsf{xgreek}}
% According to the Unicode standard 
%\begin{center}
%|http://www.unicode.org/Public/UNIDATA/UnicodeData.txt|
%\end{center}
% the uppercase form of \textsc{greek small letter epsilon with tonos} is
% \textsc{greek capital letter eta with tonos}. This is certainly wrong. The main reason
% is that accents are not part of the letter as for example is the case with
% \textsc{latin small letter k with caron}. Since, \XeLaTeX\ blindly follows the Unicode
% standard, commands like |\MakeUppercase| produce wrong output. For this reason
% I first need to set up the correct |\uccode|s and |\lccode|s.  
%    \begin{macrocode}
%<*xgreek>
\message{Package 'xgreek' version 3.1.0 by Apostolos Syropoulos}
\global\lccode"0370="0371 \global\uccode"0370="0370
\global\lccode"0371="0371 \global\uccode"0371="0370
\global\lccode"0372="0373 \global\uccode"0372="0372
\global\lccode"0373="0373 \global\uccode"0373="0372
\global\lccode"0376="0377 \global\uccode"0376="0376
\global\lccode"0377="0377 \global\uccode"0377="0376
\global\lccode"03FD="037B \global\uccode"03FD="03FD
\global\lccode"037B="037B \global\uccode"037B="03FD
\global\lccode"03FE="037C \global\uccode"03FE="03FE
\global\lccode"037C="037C \global\uccode"037C="03FE
\global\lccode"03FF="037D \global\uccode"03FF="03FF
\global\lccode"037D="037D \global\uccode"037D="03FF
\global\lccode"0386="03AC \global\uccode"0386="0391
\global\lccode"0388="03AD \global\uccode"0388="0395
\global\lccode"0389="03AC \global\uccode"0389="0397
\global\lccode"038A="03AF \global\uccode"038A="0399
\global\lccode"038C="03CC \global\uccode"038C="039F
\global\lccode"038E="03CD \global\uccode"038E="03A5
\global\lccode"038F="03CE \global\uccode"038F="03A9
\global\lccode"0390="0390 \global\uccode"0390="03AA
\global\lccode"0391="03B1 \global\uccode"0391="0391
\global\lccode"0392="03B2 \global\uccode"0392="0392
\global\lccode"0393="03B3 \global\uccode"0393="0393
\global\lccode"0394="03B4 \global\uccode"0394="0394
\global\lccode"0395="03B5 \global\uccode"0395="0395
\global\lccode"0396="03B6 \global\uccode"0396="0396
\global\lccode"0397="03B7 \global\uccode"0397="0397
\global\lccode"0398="03B8 \global\uccode"0398="0398
\global\lccode"0399="03B9 \global\uccode"0399="0399
\global\lccode"039A="03BA \global\uccode"039A="039A
\global\lccode"039B="03BB \global\uccode"039B="039B
\global\lccode"039C="03BC \global\uccode"039C="039C
\global\lccode"039D="03BD \global\uccode"039D="039D
\global\lccode"039E="03BE \global\uccode"039E="039E
\global\lccode"039F="03BF \global\uccode"039F="039F
\global\lccode"03A0="03C0 \global\uccode"03A0="03A0
\global\lccode"03A1="03C1 \global\uccode"03A1="03A1
\global\lccode"03A3="03C3 \global\uccode"03A3="03A3
\global\lccode"03A4="03C4 \global\uccode"03A4="03A4
\global\lccode"03A5="03C5 \global\uccode"03A5="03A5
\global\lccode"03A6="03C6 \global\uccode"03A6="03A6
\global\lccode"03A7="03C7 \global\uccode"03A7="03A7
\global\lccode"03A8="03C8 \global\uccode"03A8="03A8
\global\lccode"03A9="03C9 \global\uccode"03A9="03A9
\global\lccode"03AA="03CA \global\uccode"03AA="03AA
\global\lccode"03AB="03CB \global\uccode"03AB="03AB
\global\lccode"03AC="03AC \global\uccode"03AC="0391
\global\lccode"03AD="03AD \global\uccode"03AD="0395
\global\lccode"03AE="03AE \global\uccode"03AE="0397
\global\lccode"03AF="03AF \global\uccode"03AF="0399
\global\lccode"03B0="03B0 \global\uccode"03B0="03AB
\global\lccode"03B1="03B1 \global\uccode"03B1="0391
\global\lccode"03B2="03B2 \global\uccode"03B2="0392
\global\lccode"03B3="03B3 \global\uccode"03B3="0393
\global\lccode"03B4="03B4 \global\uccode"03B4="0394
\global\lccode"03B5="03B5 \global\uccode"03B5="0395
\global\lccode"03B6="03B6 \global\uccode"03B6="0396
\global\lccode"03B7="03B7 \global\uccode"03B7="0397
\global\lccode"03B8="03B8 \global\uccode"03B8="0398
\global\lccode"03B9="03B9 \global\uccode"03B9="0399
\global\lccode"03BA="03BA \global\uccode"03BA="039A
\global\lccode"03BB="03BB \global\uccode"03BB="039B
\global\lccode"03BC="03BC \global\uccode"03BC="039C
\global\lccode"03BD="03BD \global\uccode"03BD="039D
\global\lccode"03BE="03BE \global\uccode"03BE="039E
\global\lccode"03BF="03BF \global\uccode"03BF="039F
\global\lccode"03C0="03C0 \global\uccode"03C0="03A0
\global\lccode"03C1="03C1 \global\uccode"03C1="03A1
\global\lccode"03C2="03C2 \global\uccode"03C2="03A3
\global\lccode"03C3="03C3 \global\uccode"03C3="03A3
\global\lccode"03C4="03C4 \global\uccode"03C4="03A4
\global\lccode"03C5="03C5 \global\uccode"03C5="03A5
\global\lccode"03C6="03C6 \global\uccode"03C6="03A6
\global\lccode"03C7="03C7 \global\uccode"03C7="03A7
\global\lccode"03C8="03C8 \global\uccode"03C8="03A8
\global\lccode"03C9="03C9 \global\uccode"03C9="03A9
\global\lccode"03CA="03CA \global\uccode"03CA="03AA
\global\lccode"03CB="03CB \global\uccode"03CB="03AB
\global\lccode"03CC="03CC \global\uccode"03CC="039F
\global\lccode"03CD="03CD \global\uccode"03CD="03A5
\global\lccode"03CE="03CE \global\uccode"03CE="03A9
\global\lccode"03D0="03D0 \global\uccode"03D0="0392
\global\lccode"03D1="03D1 \global\uccode"03D1="0398
\global\lccode"03D2="03C5 \global\uccode"03D2="03A5
\global\lccode"03D3="03CD \global\uccode"03D3="03A5
\global\lccode"03D4="03CB \global\uccode"03D4="03AB
\global\lccode"03D5="03C6 \global\uccode"03D5="03A6
\global\lccode"03D6="03C0 \global\uccode"03D6="03A0
\global\lccode"03DA="03DB \global\uccode"03DA="03DA
\global\lccode"03DB="03DB \global\uccode"03DB="03DA
\global\lccode"03DC="03DD \global\uccode"03DC="03DC
\global\lccode"03DD="03DD \global\uccode"03DD="03DC
\global\lccode"03DE="03DF \global\uccode"03DE="03DE
\global\lccode"03DF="03DF \global\uccode"03DF="03DE
\global\lccode"03E0="03E1 \global\uccode"03E0="03E0
\global\lccode"03E1="03E1 \global\uccode"03E1="03E0
\global\lccode"03F0="03BA \global\uccode"03F0="039A
\global\lccode"03F1="03C1 \global\uccode"03F1="03A1
\global\lccode"03F2="03F2 \global\uccode"03F2="03F9
\global\lccode"03F9="03F2 \global\uccode"03F9="03F9
\global\lccode"1F00="1F00 \global\uccode"1F00="0391
\global\lccode"1F01="1F01 \global\uccode"1F01="0391
\global\lccode"1F02="1F02 \global\uccode"1F02="0391
\global\lccode"1F03="1F03 \global\uccode"1F03="0391
\global\lccode"1F04="1F04 \global\uccode"1F04="0391
\global\lccode"1F05="1F05 \global\uccode"1F05="0391
\global\lccode"1F06="1F06 \global\uccode"1F06="0391
\global\lccode"1F07="1F07 \global\uccode"1F07="0391
\global\lccode"1F08="1F00 \global\uccode"1F08="0391
\global\lccode"1F09="1F01 \global\uccode"1F09="0391
\global\lccode"1F0A="1F02 \global\uccode"1F0A="0391
\global\lccode"1F0B="1F03 \global\uccode"1F0B="0391
\global\lccode"1F0C="1F04 \global\uccode"1F0C="0391
\global\lccode"1F0D="1F05 \global\uccode"1F0D="0391
\global\lccode"1F0E="1F06 \global\uccode"1F0E="0391
\global\lccode"1F0F="1F07 \global\uccode"1F0F="0391
\global\lccode"1F10="1F10 \global\uccode"1F10="0395
\global\lccode"1F11="1F11 \global\uccode"1F11="0395
\global\lccode"1F12="1F12 \global\uccode"1F12="0395
\global\lccode"1F13="1F13 \global\uccode"1F13="0395
\global\lccode"1F14="1F14 \global\uccode"1F14="0395
\global\lccode"1F15="1F15 \global\uccode"1F15="0395
\global\lccode"1F18="1F10 \global\uccode"1F18="0395
\global\lccode"1F19="1F11 \global\uccode"1F19="0395
\global\lccode"1F1A="1F12 \global\uccode"1F1A="0395
\global\lccode"1F1B="1F13 \global\uccode"1F1B="0395
\global\lccode"1F1C="1F14 \global\uccode"1F1C="0395
\global\lccode"1F1D="1F15 \global\uccode"1F1D="0395
\global\lccode"1F20="1F20 \global\uccode"1F20="0397
\global\lccode"1F21="1F21 \global\uccode"1F21="0397
\global\lccode"1F22="1F22 \global\uccode"1F22="0397
\global\lccode"1F23="1F23 \global\uccode"1F23="0397
\global\lccode"1F24="1F24 \global\uccode"1F24="0397
\global\lccode"1F25="1F25 \global\uccode"1F25="0397
\global\lccode"1F26="1F26 \global\uccode"1F26="0397
\global\lccode"1F27="1F27 \global\uccode"1F27="0397
\global\lccode"1F28="1F20 \global\uccode"1F28="0397
\global\lccode"1F29="1F21 \global\uccode"1F29="0397
\global\lccode"1F2A="1F22 \global\uccode"1F2A="0397
\global\lccode"1F2B="1F23 \global\uccode"1F2B="0397
\global\lccode"1F2C="1F24 \global\uccode"1F2C="0397
\global\lccode"1F2D="1F25 \global\uccode"1F2D="0397
\global\lccode"1F2E="1F26 \global\uccode"1F2E="0397
\global\lccode"1F2F="1F27 \global\uccode"1F2F="0397
\global\lccode"1F30="1F30 \global\uccode"1F30="0399
\global\lccode"1F31="1F31 \global\uccode"1F31="0399
\global\lccode"1F32="1F32 \global\uccode"1F32="0399
\global\lccode"1F33="1F33 \global\uccode"1F33="0399
\global\lccode"1F34="1F34 \global\uccode"1F34="0399
\global\lccode"1F35="1F35 \global\uccode"1F35="0399
\global\lccode"1F36="1F36 \global\uccode"1F36="0399
\global\lccode"1F37="1F37 \global\uccode"1F37="0399
\global\lccode"1F38="1F30 \global\uccode"1F38="0399
\global\lccode"1F39="1F31 \global\uccode"1F39="0399
\global\lccode"1F3A="1F32 \global\uccode"1F3A="0399
\global\lccode"1F3B="1F33 \global\uccode"1F3B="0399
\global\lccode"1F3C="1F34 \global\uccode"1F3C="0399
\global\lccode"1F3D="1F35 \global\uccode"1F3D="0399
\global\lccode"1F3E="1F36 \global\uccode"1F3E="0399
\global\lccode"1F3F="1F37 \global\uccode"1F3F="0399
\global\lccode"1F40="1F40 \global\uccode"1F40="039F
\global\lccode"1F41="1F41 \global\uccode"1F41="039F
\global\lccode"1F42="1F42 \global\uccode"1F42="039F
\global\lccode"1F43="1F43 \global\uccode"1F43="039F
\global\lccode"1F44="1F44 \global\uccode"1F44="039F
\global\lccode"1F45="1F45 \global\uccode"1F45="039F
\global\lccode"1F48="1F40 \global\uccode"1F48="039F
\global\lccode"1F49="1F41 \global\uccode"1F49="039F
\global\lccode"1F4A="1F42 \global\uccode"1F4A="039F
\global\lccode"1F4B="1F43 \global\uccode"1F4B="039F
\global\lccode"1F4C="1F44 \global\uccode"1F4C="039F
\global\lccode"1F4D="1F45 \global\uccode"1F4D="039F
\global\lccode"1F50="1F50 \global\uccode"1F50="03A5
\global\lccode"1F51="1F51 \global\uccode"1F51="03A5
\global\lccode"1F52="1F52 \global\uccode"1F52="03A5
\global\lccode"1F53="1F53 \global\uccode"1F53="03A5
\global\lccode"1F54="1F54 \global\uccode"1F54="03A5
\global\lccode"1F55="1F55 \global\uccode"1F55="03A5
\global\lccode"1F56="1F56 \global\uccode"1F56="03A5
\global\lccode"1F57="1F57 \global\uccode"1F57="03A5
\global\lccode"1F59="1F51 \global\uccode"1F59="03A5
\global\lccode"1F5B="1F53 \global\uccode"1F5B="03A5
\global\lccode"1F5D="1F55 \global\uccode"1F5D="03A5
\global\lccode"1F5F="1F57 \global\uccode"1F5F="03A5
\global\lccode"1F60="1F60 \global\uccode"1F60="03A9
\global\lccode"1F61="1F61 \global\uccode"1F61="03A9
\global\lccode"1F62="1F62 \global\uccode"1F62="03A9
\global\lccode"1F63="1F63 \global\uccode"1F63="03A9
\global\lccode"1F64="1F64 \global\uccode"1F64="03A9
\global\lccode"1F65="1F65 \global\uccode"1F65="03A9
\global\lccode"1F66="1F66 \global\uccode"1F66="03A9
\global\lccode"1F67="1F67 \global\uccode"1F67="03A9
\global\lccode"1F68="1F60 \global\uccode"1F68="03A9
\global\lccode"1F69="1F61 \global\uccode"1F69="03A9
\global\lccode"1F6A="1F62 \global\uccode"1F6A="03A9
\global\lccode"1F6B="1F63 \global\uccode"1F6B="03A9
\global\lccode"1F6C="1F64 \global\uccode"1F6C="03A9
\global\lccode"1F6D="1F65 \global\uccode"1F6D="03A9
\global\lccode"1F6E="1F66 \global\uccode"1F6E="03A9
\global\lccode"1F6F="1F67 \global\uccode"1F6F="03A9
\global\lccode"1F70="1F70 \global\uccode"1F70="0391
\global\lccode"1F71="1F71 \global\uccode"1F71="0391
\global\lccode"1F72="1F72 \global\uccode"1F72="0395
\global\lccode"1F73="1F73 \global\uccode"1F73="0395
\global\lccode"1F74="1F74 \global\uccode"1F74="0397
\global\lccode"1F75="1F75 \global\uccode"1F75="0397
\global\lccode"1F76="1F76 \global\uccode"1F76="0399
\global\lccode"1F77="1F77 \global\uccode"1F77="0399
\global\lccode"1F78="1F78 \global\uccode"1F78="039F
\global\lccode"1F79="1F79 \global\uccode"1F79="039F
\global\lccode"1F7A="1F7A \global\uccode"1F7A="03A5
\global\lccode"1F7B="1F7B \global\uccode"1F7B="03A5
\global\lccode"1F7C="1F7C \global\uccode"1F7C="03A9
\global\lccode"1F7D="1F7D \global\uccode"1F7D="03A9
\global\lccode"1F80="1F80 \global\uccode"1F80="1FBC
\global\lccode"1F81="1F81 \global\uccode"1F81="1FBC
\global\lccode"1F82="1F82 \global\uccode"1F82="1FBC
\global\lccode"1F83="1F83 \global\uccode"1F83="1FBC
\global\lccode"1F84="1F84 \global\uccode"1F84="1FBC
\global\lccode"1F85="1F85 \global\uccode"1F85="1FBC
\global\lccode"1F86="1F86 \global\uccode"1F86="1FBC
\global\lccode"1F87="1F87 \global\uccode"1F87="1FBC
\global\lccode"1F88="1F80 \global\uccode"1F88="1FBC
\global\lccode"1F89="1F81 \global\uccode"1F89="1FBC
\global\lccode"1F8A="1F82 \global\uccode"1F8A="1FBC
\global\lccode"1F8B="1F83 \global\uccode"1F8B="1FBC
\global\lccode"1F8C="1F84 \global\uccode"1F8C="1FBC
\global\lccode"1F8D="1F85 \global\uccode"1F8D="1FBC
\global\lccode"1F8E="1F86 \global\uccode"1F8E="1FBC
\global\lccode"1F8F="1F87 \global\uccode"1F8F="1FBC
\global\lccode"1F90="1F90 \global\uccode"1F90="1FCC
\global\lccode"1F91="1F91 \global\uccode"1F91="1FCC
\global\lccode"1F92="1F92 \global\uccode"1F92="1FCC
\global\lccode"1F93="1F93 \global\uccode"1F93="1FCC
\global\lccode"1F94="1F94 \global\uccode"1F94="1FCC
\global\lccode"1F95="1F95 \global\uccode"1F95="1FCC
\global\lccode"1F96="1F96 \global\uccode"1F96="1FCC
\global\lccode"1F97="1F97 \global\uccode"1F97="1FCC
\global\lccode"1F98="1F90 \global\uccode"1F98="1FCC
\global\lccode"1F99="1F91 \global\uccode"1F99="1FCC
\global\lccode"1F9A="1F92 \global\uccode"1F9A="1FCC
\global\lccode"1F9B="1F93 \global\uccode"1F9B="1FCC
\global\lccode"1F9C="1F94 \global\uccode"1F9C="1FCC
\global\lccode"1F9D="1F95 \global\uccode"1F9D="1FCC
\global\lccode"1F9E="1F96 \global\uccode"1F9E="1FCC
\global\lccode"1F9F="1F97 \global\uccode"1F9F="1FCC
\global\lccode"1FA0="1FA0 \global\uccode"1FA0="1FFC
\global\lccode"1FA1="1FA1 \global\uccode"1FA1="1FFC
\global\lccode"1FA2="1FA2 \global\uccode"1FA2="1FFC
\global\lccode"1FA3="1FA3 \global\uccode"1FA3="1FFC
\global\lccode"1FA4="1FA4 \global\uccode"1FA4="1FFC
\global\lccode"1FA5="1FA5 \global\uccode"1FA5="1FFC
\global\lccode"1FA6="1FA6 \global\uccode"1FA6="1FFC
\global\lccode"1FA7="1FA7 \global\uccode"1FA7="1FFC
\global\lccode"1FA8="1FA0 \global\uccode"1FA8="1FFC
\global\lccode"1FA9="1FA1 \global\uccode"1FA9="1FFC
\global\lccode"1FAA="1FA2 \global\uccode"1FAA="1FFC
\global\lccode"1FAB="1FA3 \global\uccode"1FAB="1FFC
\global\lccode"1FAC="1FA4 \global\uccode"1FAC="1FFC
\global\lccode"1FAD="1FA5 \global\uccode"1FAD="1FFC
\global\lccode"1FAE="1FA6 \global\uccode"1FAE="1FFC
\global\lccode"1FAF="1FA7 \global\uccode"1FAF="1FFC
\global\lccode"1FB0="1FB0 \global\uccode"1FB0="1FB8
\global\lccode"1FB1="1FB1 \global\uccode"1FB1="1FB9
\global\lccode"1FB2="1FB2 \global\uccode"1FB2="1FBC
\global\lccode"1FB3="1FB3 \global\uccode"1FB3="1FBC
\global\lccode"1FB4="1FB4 \global\uccode"1FB4="1FBC
\global\lccode"1FB6="1FB6 \global\uccode"1FB6="0391
\global\lccode"1FB7="1FB7 \global\uccode"1FB7="1FBC
\global\lccode"1FB8="1FB0 \global\uccode"1FB8="1FB8
\global\lccode"1FB9="1FB1 \global\uccode"1FB9="1FB9
\global\lccode"1FBA="1F70 \global\uccode"1FBA="0391
\global\lccode"1FBB="1F71 \global\uccode"1FBB="0391
\global\lccode"1FBC="1FB3 \global\uccode"1FBC="1FBC
\global\lccode"1FBD="1FBD \global\uccode"1FBD="1FBD
\global\lccode"1FC2="1FC2 \global\uccode"1FC2="1FCC
\global\lccode"1FC3="1FC3 \global\uccode"1FC3="1FCC
\global\lccode"1FC4="1FC4 \global\uccode"1FC4="1FCC
\global\lccode"1FC6="1FC6 \global\uccode"1FC6="0397
\global\lccode"1FC7="1FC7 \global\uccode"1FC7="1FCC
\global\lccode"1FC8="1F72 \global\uccode"1FC8="0395
\global\lccode"1FC9="1F73 \global\uccode"1FC9="0395
\global\lccode"1FCA="1F74 \global\uccode"1FCA="0397
\global\lccode"1FCB="1F75 \global\uccode"1FCB="0397
\global\lccode"1FCC="1FC3 \global\uccode"1FCC="1FCC
\global\lccode"1FD0="1FD0 \global\uccode"1FD0="1FD8
\global\lccode"1FD1="1FD1 \global\uccode"1FD1="1FD9
\global\lccode"1FD2="1FD2 \global\uccode"1FD2="03AA
\global\lccode"1FD3="1FD3 \global\uccode"1FD3="03AA
\global\lccode"1FD6="1FD6 \global\uccode"1FD6="0399
\global\lccode"1FD7="1FD7 \global\uccode"1FD7="03AA
\global\lccode"1FD8="1FD0 \global\uccode"1FD8="1FD8
\global\lccode"1FD9="1FD1 \global\uccode"1FD9="1FD9
\global\lccode"1FDA="1F76 \global\uccode"1FDA="0399
\global\lccode"1FDB="1F77 \global\uccode"1FDB="0399
\global\lccode"1FE0="1FE0 \global\uccode"1FE0="1FE8
\global\lccode"1FE1="1FE1 \global\uccode"1FE1="1FE9
\global\lccode"1FE2="1FE2 \global\uccode"1FE2="03AB
\global\lccode"1FE3="1FE3 \global\uccode"1FE3="03AB
\global\lccode"1FE4="1FE4 \global\uccode"1FE4="03A1
\global\lccode"1FE5="1FE5 \global\uccode"1FE5="03A1
\global\lccode"1FE6="1FE6 \global\uccode"1FE6="03A5
\global\lccode"1FE7="1FE7 \global\uccode"1FE7="03AB
\global\lccode"1FE8="1FE0 \global\uccode"1FE8="1FE8
\global\lccode"1FE9="1FE1 \global\uccode"1FE9="1FE9
\global\lccode"1FEA="1F7A \global\uccode"1FEA="03A5
\global\lccode"1FEB="1F7B \global\uccode"1FEB="03A5
\global\lccode"1FEC="1FE5 \global\uccode"1FEC="1FEC
\global\lccode"1FF2="1FF2 \global\uccode"1FF2="1FFC
\global\lccode"1FF3="1FF3 \global\uccode"1FF3="1FFC
\global\lccode"1FF4="1FF4 \global\uccode"1FF4="1FFC
\global\lccode"1FF6="1FF6 \global\uccode"1FF6="03A9
\global\lccode"1FF7="1FF7 \global\uccode"1FF7="1FFC
\global\lccode"1FF8="1F78 \global\uccode"1FF8="039F
\global\lccode"1FF9="1F79 \global\uccode"1FF9="039F
\global\lccode"1FFA="1F7C \global\uccode"1FFA="03A9
\global\lccode"1FFB="1F7D \global\uccode"1FFB="03A9
\global\lccode"1FFC="1FF3 \global\uccode"1FFC="1FFC
%    \end{macrocode}
% Next I define the various strings that correspond to the standard \LaTeX\ captions.
% I first define the strings for monotonic Greek. 
%    \begin{macrocode}
\def\prefacename{Πρόλογος}%
\def\refname{Αναφορές}%
\def\abstractname{Περίληψη}%
\def\bibname{Βιβλιογραφία}%
\def\chaptername{Κεφάλαιο}%
\def\appendixname{Παράρτημα}%
\def\contentsname{Περιεχόμενα}%
\def\listfigurename{Κατάλογος σχημάτων}%
\def\listtablename{Κατάλογος πινάκων}%
\def\indexname{Ευρετήριο}%
\def\figurename{Σχήμα}%
\def\tablename{Πίνακας}%
\def\partname{Μέρος}%
\def\enclname{Συνημμένα}%
\def\ccname{Κοινοποίηση}%
\def\headtoname{Προς}%
\def\pagename{Σελίδα}%
\def\seename{βλέπε}%
\def\alsoname{βλέπε επίσης}%
\def\proofname{Απόδειξη}%
\def\glossaryname{Γλωσσάρι}%
%    \end{macrocode}
% Macro |\polytonicn@mes| is invoked when polytonic Greek is the main language of the document.
%    \begin{macrocode} 
\def\polytonicn@mes{%
  \def\refname{Ἀναφορὲς}%
  \def\indexname{Εὑρετήριο}%
  \def\figurename{Σχῆμα}%
  \def\headtoname{Πρὸς}%
  \def\alsoname{βλέπε ἐπίσης}%
  \def\proofname{Ἀπόδειξη}%
}
%    \end{macrocode}
% Macro |\@ncientn@mes| is invoked when ancient Greek is the main language of the document.
%    \begin{macrocode} 
\def\@ncientn@mes{%
  \def\prefacename{Προοίμιον}%
  \def\abstractname{Περίληψις}%
  \def\bibname{Βιβλιογραφία}%
  \def\chaptername{Κεφάλαιον}%
  \def\appendixname{Παράρτημα}%
  \def\contentsname{Περιεχόμενα}%
  \def\listfigurename{Κατάλογος σχημάτων}%
  \def\listtablename{Κατάλογος πινάκων}%
  \def\indexname{Εὑρετήριον}%
  \def\tablename{Πίναξ}%
  \def\partname{Μέρος}%
  \def\enclname{Συνημμένως}%
  \def\ccname{Κοινοποίησις}%
  \def\headtoname{Πρὸς}%
  \def\pagename{Σελὶς}%
  \def\seename{ὃρα}%
  \def\alsoname{ὃρα ὡσαύτως}%
  \def\proofname{Ἀπόδειξις}%
  \def\glossaryname{Γλωσσάριον}%
  \def\refname{Ἀναφοραὶ}%
  \def\figurename{Σχῆμα}%
  \def\headtoname{Πρὸς}%
}
%    \end{macrocode}
% I redefine |\today| so as to produce dates in Greek. The
% names of months are defined by the macro |\gr@month|. 
%    \begin{macrocode}
\def\gr@month{%
  \ifcase\month\or Ιανουαρίου\or Φεβρουαρίου\or Μαρτίου\or Απριλίου\or
    Μαΐου\or Ιουνίου\or Ιουλίου\or Αυγούστου\or
     Σεπτεμβρίου\or Οκτωβρίου\or Νοεμβρίου\or Δεκεμβρίου\fi}
\def\today{\number\day \space \gr@month\space \number\year}
%    \end{macrocode}
% When either polytonic Greek or ancient Greek is the main language of the document,
% then the macro |\gr@c@month| becomes active.  
%    \begin{macrocode}
\def\gr@c@month{%
  \ifcase\month\or Ἰανουαρίου\or Φεβρουαρίου\or Μαρτίου\or Ἀπριλίου\or 
   Μαΐου\or Ἰουνίου\or Ἰουλίου\or Αὐγούστου\or Σεπτεμβρίου\or
  Ὀκτωβρίου\or Νοεμβρίου\or Δεκεμβρίου\fi}
%    \end{macrocode}
% Next, I define a few macros that allow one to access characters
% that are not usually easily accessible from the keyboard (e.g., the sampi or the
% koppa symbol). The list includes a command for the unicode symbol GREEK ANO TELEIA,
% which, in some systems, is confused with MIDDLE DOT. The use of command |\numer@lsign|
% will be explained later.
%    \begin{macrocode}
\def\anwtonos{ʹ} %GREEK NUMERAl SIGN
\let\numer@lsign\anwtonos
\def\katwtonos{͵} %GREEK LOWER NUMERAL SIGN
\def\koppa{\char"03DF\relax}
\def\sampi{\char"03E1\relax}
\def\Digamma{\char"03DC\relax}
\def\ddigamma{\char"03DD\relax}
\def\anoteleia{\char"0387\relax}
\def\euro{\char"20AC\relax}
\def\permill{\char"2030\relax}
%    \end{macrocode}
% Many users prefer the use of the letters sigma and tau instead of the stigma symbol in
% Greek numerals, therefore, by default the |\stigma| command expands to ``στ''. 
%    \begin{macrocode}
\def\stigma{στ\relax}
%    \end{macrocode}
% The following commands take care of the basic rules of typography. Note that the first command
% changes the way space is added after punctuation symbols and the last two commands force \LaTeX\
% to add indentation space to the first paragraph after a header. Since a number of users need, for 
% their own reasons, to be able to disable this particular feature I have introduced a new package
% option, namely |noindentfirst|, which restores the default behavior. In order to be able
% to do this I need the original value of the boolean variable |\@afterindentfalse|.
%    \begin{macrocode}
\frenchspacing
\let\@saveafterindentfalse\@afterindentfalse
\let\@afterindentfalse\@afterindenttrue
\@afterindenttrue
%    \end{macrocode}
% Lua\LaTeX\ and \XeLaTeX\ have different ways to load hyphenation patterns. Package 
% \textsf{luahyphenrules} by Javier Bezos facilitates this process for people who
% want to use Lua\LaTeX\ and the ``traditional'' way to load hyphenation patterns. 
% To ensure proper inclusion of LuaTeX staff, I use the following ``idiom'':
% \begin{center}
% |\ifx\directlua\undefined |\texttt{\textit{non Lua\LaTeX\ code}}|\else |%
% \texttt{\textit{Lua\LaTeX\ code}}|\fi|
% \end{center} 
%    \begin{macrocode}
\ifx\directlua\undefined\else\RequirePackage{luahyphenrules}\fi
%    \end{macrocode}
% The code that follows specifies which hyphenation patterns will be active. The \XeLaTeX\ code
% is quite standard and depends on the \textsf{babel} pattern loading mechanism, while the
% Lua\LaTeX\ code uses the |\HyphenRules| macro, which has essentially the functionality 
% of the |\selectlanguage| macro.
%    \begin{macrocode}
\DeclareOption{monogreek}{%
   \ifx\directlua\undefined%
   \language\l@monogreek\else\HyphenRules{monogreek}\fi%
}
\DeclareOption{polygreek}{%
   \ifx\directlua\undefined%
   \language\l@polygreek\else\HyphenRules{polygreek}\fi%
   \polytonicn@mes%
   \let\gr@month\gr@c@month%
}
\DeclareOption{ancientgreek}{%
   \ifx\directlua\undefined%
   \language\l@ancientgreek\else\HyphenRules{ancientgreek}\fi%
   \@ncientn@mes%
   \let\gr@month\gr@c@month%
}
%    \end{macrocode}
% If a user wants to use the stigma symbol in Greek numerals, she should use the
% |stigma| option.
%    \begin{macrocode}
\DeclareOption{stigma}{%
   \def\stigma{\char"03DB\relax}
}
%    \end{macrocode}
% As noted above, the new option |noindentfirst| restores the default \LaTeX\ behavior of adding no 
% indentation to the first paragraph after any header. 
%    \begin{macrocode}
\DeclareOption{noindentfirst}{%
   \let\@afterindentfalse\@saveafterindentfalse
}
%    \end{macrocode}
% Nowadays it is customary in Greece to use Greek numerals without the GREEK NUMERAL SIGN at the end
% of numeral. Thus, the |nonumeralsign| option disables the typesetting of the GREEK NUMERAL SIGN
% at the end of Greek numerals. 
%    \begin{macrocode}
\DeclareOption{nonumeralsign}{%
   \let\numer@lsign\relax
}
%    \end{macrocode}
%  Package \textsf{listings} does not work properly with UTF-8 encoded files. So this
% option should be used whenever one wants to use this package and see Greek text come out
% correctly. In versaion 3.1.0 of this package, I had included the code that modifies 
% package \textsf{listings} in the source code of this package. However, this decision
% was wrong since when one does not use the \texttt{listings} option, processing of the
% input file stops with an error message about a text line that contains an invalid character.
% So the best way to solve this problem is to move the code in a different file and load it
% when the user has specified \texttt{listings} option. To enable this option, I use a boolean 
% variable.
%    \begin{macrocode}
\newif\if@mylistings
\@mylistingsfalse
\DeclareOption{listings}{\@mylistingstrue}
%    \end{macrocode}
% By default the |monogreek| option is activated. 
%    \begin{macrocode}
\ExecuteOptions{monogreek}
\ProcessOptions
%    \end{macrocode}
% If the user has enabled the |listings| option, then it loads the package \textsf{xelistings}
%    \begin{macrocode}
\if@mylistings
\RequirePackage{xelistings}
\fi
%    \end{macrocode}
% Now I am going to define the macros that typeset alphabetic Greek numerals. The code 
% is borrowed from the greek option for the babel package.
%  \begin{macro}{\gr@ill@value}
%    When the argument of |\greeknumeral| has a value outside of the
%    acceptable bounds ($0 < x < 999999$) a warning will be issued
%    (and nothing will be printed).
%    \begin{macrocode}
\def\gr@ill@value#1{%
  \PackageWarning{xgreek}{Illegal value (#1) for greeknumeral}}
%    \end{macrocode}
%  \end{macro}
%  \begin{macro}{\anw@true}
%  \begin{macro}{\anw@false}
%  \begin{macro}{\anw@print}
%    When a large number with three \emph{trailing} zeros is to be
%    printed those zeros \emph{and} the numeric mark need to be
%    discarded. As each `digit' is processed by a separate macro
%    \emph{and} because the processing needs to be expandable we need
%    some helper macros that help remember to \emph{not} print the
%    numeric mark (|\numer@lsign|).
%
%    The command |\anw@false| switches the printing of the numeric
%    mark off by making |\anw@print| expand to nothing. The command
%    |\anw@true| (re)enables the printing of the numeric marc. These
%    macro's need to be robust in order to prevent improper expansion
%    during writing to files or during |\uppercase|.
%    \begin{macrocode}
\DeclareRobustCommand\anw@false{%
  \DeclareRobustCommand\anw@print{}}
\DeclareRobustCommand\anw@true{%
  \DeclareRobustCommand\anw@print{\numer@lsign}}
\anw@true
%    \end{macrocode}
%  \end{macro}
%  \end{macro}
%  \end{macro}
%
%  \begin{macro}{\@greeknumeral}
%    This command is used to get Greek numerals. The command uses
%    |\numer@lsign| to get the NUMERAL SIGN. Obviously, when the
%    user has specified the \texttt{no\-numeral\-sign} option, then numeral
%    comes out without the trailing NUMERAL SIGN. However, when a user
%    wants to typeset a Greek numeral, the numeral must come out correctly,
%    regardless of what appears in headers, etc. And that is exactly the
%    reason why this command is inaccessible to users.
%    The command |\@greeknumeral| needs to be \emph{fully} expandable
%    in order to get the right information in auxiliary
%    files. Therefore we use a big |\if|-construction to check the
%    value of the argument and start the parsing at the right level.
%    \begin{macrocode}
\def\@greeknumeral#1{%
%    \end{macrocode}
%    If the value is negative or zero nothing is printed and a warning
%    is issued.
%    \begin{macrocode}
  \ifnum#1<\@ne\space\gr@ill@value{#1}%
  \else
    \ifnum#1<10\expandafter\gr@num@i\number#1%
    \else
      \ifnum#1<100\expandafter\gr@num@ii\number#1%
      \else
%    \end{macrocode}
%    The available shorthands for 1.000 (|\@m|) and 10.000
%    (|\@M|) are used to save a few tokens.
%    \begin{macrocode}
        \ifnum#1<\@m\expandafter\gr@num@iii\number#1%
        \else
          \ifnum#1<\@M\expandafter\gr@num@iv\number#1%
          \else
            \ifnum#1<100000\expandafter\gr@num@v\number#1%
            \else
              \ifnum#1<1000000\expandafter\gr@num@vi\number#1%
              \else
%    \end{macrocode}
%    If the value is too large, nothing is printed and a warning
%    is issued.
%    \begin{macrocode}
                \space\gr@ill@value{#1}%
              \fi
            \fi
          \fi
        \fi
      \fi
    \fi
  \fi
}
%    \end{macrocode}
%  \end{macro}
%
%    What is left to make complete the definition of command |\greeknumeral| is a set of macros to produce
%    the various digits.
%  \begin{macro}{\gr@num@i}
%  \begin{macro}{\gr@num@ii}
%  \begin{macro}{\gr@num@iii}
%    As there is no ``digit'' representing $0$ in this system, the zeros
%    are simply discarded. When there is a large number with three
%    \emph{trailing} zeros also the numeric mark is discarded. 
%    Therefore these macros need to pass the information to each other
%    about the (non-)translation of a zero.
%    \begin{macrocode}
\def\gr@num@i#1{%
  \ifcase#1\or α\or β\or γ\or δ\or ε\or \stigma\or ζ\or η\or θ\fi
  \ifnum#1=\z@\else\anw@true\fi\anw@print}
\def\gr@num@ii#1{%
  \ifcase#1\or ι\or κ\or λ\or μ\or ν\or ξ\or ο\or π\or \koppa\fi
  \ifnum#1=\z@\else\anw@true\fi\gr@num@i}
\def\gr@num@iii#1{%
  \ifcase#1\or ρ\or σ\or τ\or υ\or φ\or χ\or ψ\or ω\or \sampi\fi
  \ifnum#1=\z@\anw@false\else\anw@true\fi\gr@num@ii}
%    \end{macrocode}
%  \end{macro}
%  \end{macro}
%  \end{macro}
%
%  \begin{macro}{\gr@num@iv}
%  \begin{macro}{\gr@num@v}
%  \begin{macro}{\gr@num@vi}
%    The first three ``digits'' always have the numeric mark, except
%    when one is discarded because it's value is zero.
%    \begin{macrocode}
\def\gr@num@iv#1{%
  \ifnum#1=\z@\else\katwtonos\fi
  \ifcase#1\or α\or β\or γ\or δ\or ε\or \stigma\or ζ\or η\or θ\fi
  \gr@num@iii}
\def\gr@num@v#1{%
  \ifnum#1=\z@\else\katwtonos\fi
  \ifcase#1\or ι\or κ\or λ\or μ\or ν\or ξ\or ο\or π\or \koppa\fi
  \gr@num@iv}
\def\gr@num@vi#1{%
  \katwtonos
  \ifcase#1\or ρ\or σ\or τ\or υ\or φ\or χ\or ψ\or ω\or \sampi\fi
  \gr@num@v}
%    \end{macrocode}
%  \end{macro}
%  \end{macro}
%  \end{macro}
%  \begin{macro}{\@Greeknumeral}
%    The command |\@Greeknumeral| prints uppercase Greek numerals. 
%    The parsing is performed by the macro |\@greeknumeral|. The printing
%    of the NUMERAL SIGN depends on the value of |\numer@lsign|.
%    \begin{macrocode}
\def\@Greeknumeral#1{%
  \expandafter\MakeUppercase\expandafter{\@greeknumeral{#1}}}
%    \end{macrocode}
%  \end{macro}
%  \begin{macro}{\greeknumeral}
%   This command prints lowercase Greek numerals and the NUMERAL SIGN
%   is always printed.
%    \begin{macrocode}
\def\greeknumeral#1{%
  \let\@numer@lsign\numer@lsign%
  \let\numer@lsign\anwtonos%
  \@greeknumeral{#1}
  \let\numer@lsign\@numer@lsign}
%    \end{macrocode}
%  \end{macro}
%  \begin{macro}{\Greeknumeral}
%   This command prints uppercase Greek numerals and the NUMERAL SIGN
%   is always printed.
%    \begin{macrocode}
\def\Greeknumeral#1{%
  \let\@numer@lsign\numer@lsign%
  \let\numer@lsign\anwtonos%
  \@Greeknumeral{#1}
  \let\numer@lsign\@numer@lsign}
%    \end{macrocode}
%  \end{macro}
%
% The alphabetic numbering system is not the only numbering system employed by Greeks.
% In fact, Greeks used various systems that are now known as {\em acrophonic} numbering
% systems. Many scholars are familiar with the acrophonic Attic numbering system and the
% the command |\atticnum| can be used to generate acrophonic Attic numerals.  
% The acrophonic Attic numbering system, like the Roman one, employs
% letters to denote important numbers. Multiple occurrence of a letter denote
% a multiple of the ``important'' number, e.g., the letter Ι denotes 1, so
% ΙΙΙ denotes 3. Here are the basic digits used in the acrophonic Attic numbering
% system:
% \begin{itemize}
%  \item Ι denotes the number one (1)
%  \item Π denotes the number five (5)
%  \item Δ denotes the number ten (10)
%  \item Η denotes the number one hundred (100)
%  \item Χ denotes the number one thousand (1000)
%  \item Μ denotes the number ten thousands (10000)
%\end{itemize}
% Moreover,  the letters Δ, Η, Χ, and Μ under the letter ^^^^^^010143 (a form of Π)
% denote five times their original value. In particular, the symbol ^^^^^10144, denotes 
% the number 50, the symbol ^^^^^10145 denotes the number 500, the symbol ^^^^^10146
% denotes the number 5000, and the symbol ^^^^^10147 denotes the number 50,000. It 
% must be noted that the numbering system does not provide negative numerals or a symbol for
% zero. 
%\begin{macro}{\@@atticnum}            
% Now, let me definite the macro 
% |\@@atticnum|. This macro uses one integer variable (or counter in 
% \TeX's jargon.)
%    \begin{macrocode}
\newcount\@attic@num
%    \end{macrocode}
% The macro |\@@atticnum| is also defined as a robust command.
%    \begin{macrocode}
\DeclareRobustCommand*{\@@atticnum}[1]{%
%    \end{macrocode}
% After assigning to variable |\@attic@num| the value of the macro's argument, 
% we make sure that the argument is in the expected range, i.e., it is greater
% than zero, and less or equal to $249999$.  In case it isn't, it simply 
% produces a |\space|, warns the user about it and quits. Although, the
% |\atticnum| macro is capable to produce an Athenian numeral for even greater
% intergers, the following argument by Claudio Beccari convised me to place
% this upper limit:
% \begin{quote} 
% According to psychological perception studies (that ancient Athenians
% and Romans perfectly knew without needing to study Freud and Jung)
% living beings (which includes at least all vertebrates, not only
% humans) can perceive up to four randomly set objects of the same kind   
% without the need of counting, the latter activity being a specific
% acquired ability of human kind; the biquinary numbering notation
% used by the Athenians and the Romans exploits this natural
% characteristic of human beings.
% \end{quote}
%    \begin{macrocode}
        \@attic@num#1\relax
        \ifnum\@attic@num<\@ne%
          \space%
          \PackageWarning{xgreek}{%
          Illegal value (\the\@attic@num) for acrophonic 
          Attic numeral}%
        \else\ifnum\@attic@num>249999%
          \space%
          \PackageWarning{xgreek}{%
          Value too large (\the\@attic@num) for acrophonic 
          Attic numeral}%
        \else
%    \end{macrocode}
% Having done all the necessary checks, it is possible to proceed with the actual
% computation. If the number is greater than $49999$, then it certainly
% has at least one ^^^^^10147 ``digit''. The macro finds all such digits by continuously
% subtracting $50000$ from |\@attic@num|, until |\@attic@num| becomes less than
% $50000$. 
%    \begin{macrocode}
            \@whilenum\@attic@num>49999\do{%
               ^^^^^^010147\advance\@attic@num-50000}%
%    \end{macrocode}
% Next the macro checks for tens of thousands.
%    \begin{macrocode}
            \@whilenum\@attic@num>9999\do{%
               M\advance\@attic@num-\@M}%
%    \end{macrocode}
% Since a number can have only one ^^^^^10146 ``digit'' (equivalent to 5000), it 
% is easy to check whether is should have one and produce the corresponding numeral when
% it does have one.
%    \begin{macrocode}
            \ifnum\@attic@num>4999%
               ^^^^^^010146\advance\@attic@num-5000%
            \fi\relax
%    \end{macrocode}
% The macro should also check for thousands, the same way it checked for tens of thousands.
%    \begin{macrocode}
            \@whilenum\@attic@num>999\do{%
               Χ\advance\@attic@num-\@m}%
%    \end{macrocode} 
% Since a numeral can have at most one ^^^^^10145 ``digit'' (equivalent to 500), this should be
% handled the way the macro handled the case of the five thousands ``digit''.
%    \begin{macrocode}
            \ifnum\@attic@num>499%
               ^^^^^^010145\advance\@attic@num-500%
            \fi\relax
%    \end{macrocode}
% It is time to check hundreds, which follow the same pattern as thousands.
%    \begin{macrocode}
            \@whilenum\@attic@num>99\do{%
               Η\advance\@attic@num-100}%
%    \end{macrocode}
% A numeral can have only one ^^^^^10144 ``digit'' (equivalent to 50).
%    \begin{macrocode} 
            \ifnum\@attic@num>49%
               ^^^^^^010144\advance\@attic@num-50%
            \fi\relax
%    \end{macrocode}
% The macro now checks now for tens digit.
%    \begin{macrocode}         
            \@whilenum\@attic@num>9\do{%
               Δ\advance\@attic@num by-10}%
%    \end{macrocode}
% Finally, it has to check for fives and the digits 1, 2, 3, and 4.
%    \begin{macrocode}
            \@whilenum\@attic@num>4\do{%
               Π\advance\@attic@num-5}%
            \ifcase\@attic@num\or Ι\or ΙΙ\or ΙΙΙ\or ΙΙΙΙ\fi%
   \fi\fi}
%    \end{macrocode}
%\end{macro}
% 
%\begin{macro}{\@atticnum}
% The command |\@atticnum| has one argument, which
% is a counter. It calls the command |\@@atticnum| to process the value of
% the counter.
%    \begin{macrocode}
\def\@atticnum#1{%
     \expandafter\@@atticnum\expandafter{\the#1}}
%    \end{macrocode}
%\end{macro}
%\begin{macro}{\atticnum}
% The command |\atticnum| is a wrapper that declares
% a new counter in a local scope, assigns to it the value of the argument of the command
% and calls the macro |\@atticnum|. This way the command can process correctly
% either a number or a counter. 
%    \begin{macrocode}
\def\atticnum#1{%
     \@attic@num#1\relax
     \@atticnum{\@attic@num}}
%    \end{macrocode}
%\end{macro}
%
%  \begin{macro}{\greek@alph}
%  \begin{macro}{\greek@Alph}
% Here I redefine the macros |\@alph| and |\@Alph|. First, I define some placeholders
%    \begin{macrocode}
\let\latin@alph\@alph
\let\latin@Alph\@Alph
%    \end{macrocode}
%    Then I define the Greek versions; the additional |\expandafter|s
%    are needed in order to make sure the table of contents will be
%    correct (e.g., when there are appendices). 
%    \begin{macrocode}
\def\greek@alph#1{\expandafter\@greeknumeral\expandafter{\the#1}}
\def\greek@Alph#1{\expandafter\@Greeknumeral\expandafter{\the#1}}
%    \end{macrocode}
% By default Greek alphabetic enumerals instaed of Latin numerals are used to enumerate items in an
% enumeration environment.
%    \begin{macrocode}
  \let\@alph\greek@alph
  \let\@Alph\greek@Alph
%    \end{macrocode}
% If for some reason, one needs to have the Latin numerals back, then she has to invoke command
% |\nogreekalph|. And if she wants to switch back, then she has to use the |\greekalph|
% command:
%    \begin{macrocode}   
\def\nogreekalph{%
  \let\@alph\latin@alph
  \let\@Alph\latin@Alph}
\def\greekalph{%
  \let\@alph\greek@alph
  \let\@Alph\greek@Alph}
%    \end{macrocode}
%  \end{macro}
%  \end{macro}
%
%  \begin{macro}{\setlanguage}
% We provide the |\setlanguage| command which 
% activates the hypehnation patterns of some other language. It is similar
% to babel's |\selectlanguage|, but we opted to use a new name to avoid possible name conflicts. 
% Valid arguments include |monogreek|, |polygreek|, |ancientgreek|, and |american|. As was noted
% previously, package \textsf{luahyphenrules} provides the command |\HyphenRules| which has exactly
% the same functionality as this command. So when using Lua\LaTeX\ users will actually use the
% |\HyphenRules| command.
%    \begin{macrocode}
\ifx\directlua\undefined%
  \def\setlanguage#1{%
     \expandafter\ifx\csname l@#1\endcsname\relax%
     \typeout{^^J Error: No hyphenation pattern for}%
     \typeout{ language #1 are loaded,}%
     \typeout{ default hyphenation patterns are used.^^J}%
     \language=0%
     \else\language=\csname l@#1\endcsname\fi}
\else
  \let\setlanguage\HyphenRules
\fi
%    \end{macrocode}
%  \end{macro}
%    The macros |\grtoday| and |\Grtoday| produces the current date, only that the
%    month and the day are shown as greek numerals instead of arabic
%    as it is usually the case. In addition, the two commands differ in that the
%    later produces the Greek numerals in uppercase.
%    \begin{macrocode}
\def\grtoday{%
  \expandafter\greeknumeral\expandafter{\the\day}\space
  \gr@c@month\space
  \expandafter\greeknumeral\expandafter{\the\year}}
\def\Grtoday{%
  \expandafter\Greeknumeral\expandafter{\the\day}\space
  \gr@c@month\space
  \expandafter\Greeknumeral\expandafter{\the\year}}
%</xgreek>
%    \end{macrocode}
% \section{The Source Code of \textsf{xelistings}}
%
% If the user has enabled the |listings| option, then the package loads the rudimentary package
% \textsf{xelistings}. This package loads the \textsf{listings} package and makes accessible to
% it all characters in the range 128--255 plus all Greek letters that belong to the 
% Greek and Coptic Unicode block. This is achieved by redefining the command |\lst@DefEC|.
%    \begin{macrocode}
%<*xelistings>
\RequirePackage{listings}
\lstset{inputencoding=utf8}
\lst@InputCatcodes
\gdef\lst@DefEC{%
 \lst@CCECUse \lst@ProcessLetter
  ^^80^^81^^82^^83^^84^^85^^86^^87^^88^^89^^8a^^8b^^8c^^8d^^8e^^8f%
  ^^90^^91^^92^^93^^94^^95^^96^^97^^98^^99^^9a^^9b^^9c^^9d^^9e^^9f%
  ^^a0^^a1^^a2^^a3^^a4^^a5^^a6^^a7^^a8^^a9^^aa^^ab^^ac^^ad^^ae^^af%
  ^^b0^^b1^^b2^^b3^^b4^^b5^^b6^^b7^^b8^^b9^^ba^^bb^^bc^^bd^^be^^bf%
  ^^c0^^c1^^c2^^c3^^c4^^c5^^c6^^c7^^c8^^c9^^ca^^cb^^cc^^cd^^ce^^cf%
  ^^d0^^d1^^d2^^d3^^d4^^d5^^d6^^d7^^d8^^d9^^da^^db^^dc^^dd^^de^^df%
  ^^e0^^e1^^e2^^e3^^e4^^e5^^e6^^e7^^e8^^e9^^ea^^eb^^ec^^ed^^ee^^ef%
  ^^f0^^f1^^f2^^f3^^f4^^f5^^f6^^f7^^f8^^f9^^fa^^fb^^fc^^fd^^fe^^ff%
  ^^^^0396^^^^0388^^^^0389^^^^038a^^^^038c% <--- Begin of Greek Letters
  ^^^^038e^^^^038f^^^^0390^^^^0391^^^^0392%
  ^^^^0393^^^^0394^^^^0395^^^^0396^^^^0397%
  ^^^^0398^^^^0399^^^^039a^^^^039b^^^^039c%
  ^^^^039d^^^^039e^^^^039f^^^^03a0^^^^03a1%
  ^^^^03a3^^^^03a4^^^^03a5^^^^03a6^^^^03a7%
  ^^^^03a8^^^^03a9^^^^03aa^^^^03ab^^^^03ac%
  ^^^^03ad^^^^03ae^^^^03af^^^^03b0^^^^03b1%
  ^^^^03b2^^^^03b3^^^^03b4^^^^03b5^^^^03b6%
  ^^^^03b7^^^^03b8^^^^03b9^^^^03ba^^^^03bb%
  ^^^^03bc^^^^03bd^^^^03be^^^^03bf^^^^03c0%
  ^^^^03c1^^^^03c2^^^^03c3^^^^03c4^^^^03c5%
  ^^^^03c6^^^^03c7^^^^03c8^^^^03c9^^^^03ca%
  ^^^^03cb^^^^03cc^^^^03cd^^^^03ce% <--- End of Greek Letters
  ^^00}%
\lst@RestoreCatcodes
%</xelistings>
%    \end{macrocode} 
% \Finale
