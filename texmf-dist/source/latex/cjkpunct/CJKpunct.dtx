% \iffalse
%<*internal>
\iffalse
%</internal>
%<*readme>
CJKpunct
========

`CJKpunct` is a LaTeX2e macro package to be the companion
of `CJK` package by Werner Lemberg for punctuation location
and width adjustments.

Contributing
------------

This package is a part of the [ctex-kit](https://github.com/CTeX-org/ctex-kit) project.

Issues and pull requests are welcome.

Copyright and Licence
---------------------

    Copyright (C) 2003--2010 by Linbo Zhang <zlb@lsec.cc.ac.cn>
    Copyright (C) 2003--2010 by Wenchang Sun <sunwch@nankai.edu.cn>
    Copyright (C) 2010 by Leo Liu <leoliu.pku@gmail.com>
    Copyright (C) 2016 by Qing Lee <sobenlee@gmail.com>
    ----------------------------------------------------------------------

    This work may be distributed and/or modified under the
    conditions of the LaTeX Project Public License, either
    version 1.3c of this license or (at your option) any later
    version. This version of this license is in
       http://www.latex-project.org/lppl/lppl-1-3c.txt
    and the latest version of this license is in
       http://www.latex-project.org/lppl.txt
    and version 1.3 or later is part of all distributions of
    LaTeX version 2005/12/01 or later.

    This work has the LPPL maintenance status `maintained'.

    The Current Maintainer of this work is Qing Lee.

    This package consists of the files CJKpunct.dtx,
                                       CJKpunct.spa,
                                       setpunct-main.tex,
                                       setpunct-macros.tex,
                                       example-CJKfntef.tex,
                                       example-gb.tex,
                                       example-gbk.tex,
                                       example-utf8.tex,
                                       README.txt,
                 and the derived files CJKpunct.sty,
                                       CJKpunct.pdf,
                                       CJKpunct.ins, and
                                       README.md (this file).
%</readme>
%<*internal>
\fi
\begingroup
  \def\temp{LaTeX2e}
\expandafter\endgroup\ifx\temp\fmtname\else
\csname fi\endcsname
%</internal>
%<*batchfile>

\input docstrip %

\keepsilent
\askforoverwritefalse

\preamble

    Copyright (C) 2003--2010 by Linbo Zhang <zlb@lsec.cc.ac.cn>
    Copyright (C) 2003--2010 by Wenchang Sun <sunwch@nankai.edu.cn>
    Copyright (C) 2010 by Leo Liu <leoliu.pku@gmail.com>
    Copyright (C) 2016 by Qing Lee <sobenlee@gmail.com>
--------------------------------------------------------------------------

    This work may be distributed and/or modified under the
    conditions of the LaTeX Project Public License, either
    version 1.3c of this license or (at your option) any later
    version. This version of this license is in
       http://www.latex-project.org/lppl/lppl-1-3c.txt
    and the latest version of this license is in
       http://www.latex-project.org/lppl.txt
    and version 1.3 or later is part of all distributions of
    LaTeX version 2005/12/01 or later.

    This work has the LPPL maintenance status `maintained'.

    The Current Maintainer of this work is Qing Lee.

--------------------------------------------------------------------------

\endpreamble
\postamble

    This package consists of the files CJKpunct.dtx,
                                       CJKpunct.spa,
                                       setpunct-main.tex,
                                       setpunct-macros.tex,
                                       example-CJKfntef.tex,
                                       example-gb.tex,
                                       example-gbk.tex,
                                       example-utf8.tex,
                                       README.txt,
                 and the derived files CJKpunct.sty,
                                       CJKpunct.pdf,
                                       CJKpunct.ins, and
                                       README.md (this file).
\endpostamble

\generate
  {
    \usedir{tex/latex/cjkpunct}
    \file{CJKpunct.sty} {\from{\jobname.dtx}{CJKpunct}}
    \usedir{doc/latex/cjkpunct}
    \nopreamble\nopostamble
    \file{README.md}    {\from{\jobname.dtx}{readme}}
  }
%</batchfile>
%<batchfile>\Msg{*************************************************************}
%<batchfile>\Msg{*}
%<batchfile>\Msg{* To finish the installation you have to move the following}
%<batchfile>\Msg{* files into a directory searched by LaTeX:}
%<batchfile>\Msg{*}
%<batchfile>\Msg{* \space\space\space  CJKpunct.sty}
%<batchfile>\Msg{*}
%<batchfile>\Msg{*************************************************************}
%<batchfile>\endbatchfile
%<*internal>
\generate{\file{CJKpunct.ins}{\from{\jobname.dtx}{batchfile}}}
\endbatchfile
\fi
%</internal>
%<*driver>
\documentclass[a4paper,12pt]{ltxdoc}
\usepackage{xcolor}
\usepackage{CJKutf8}
\usepackage{CJKspace}
\usepackage{CJKpunct}
\PassOptionsToPackage{driverfallback=dvipdfmx}{hyperref}
\usepackage{hypdoc}
\hypersetup{pdfencoding=unicode,bookmarksopen=true,pdfstartview=FitH}
\AtBeginShipoutFirst{{%
  \MakePercentComment
  \input{zhwindowsfonts}}}
\textheight 210mm
\textwidth 150mm
\oddsidemargin 0pt
\evensidemargin 0pt

% macros
{\catcode`\|=0 \catcode`\\=12
 |gdef|bslash{\}}

\newcommand{\defmacro}[1]{%                   % Define a macro.
 \textcolor{macrocolor}{$\backslash$#1}\index{\string\verb+\bslash#1+}%
}

\newcommand{\usemacro}[1]{%                   % Define a macro.
  \textcolor{macrocolor}{\string#1}%
  #1\index{\string\verb+\string#1+}%
}

\definecolor{parametercolor}{rgb}{1,0,1}
\definecolor{optioncolor}{rgb}{0,0,1}
\definecolor{macrocolor}{rgb}{0,0,0.63}

\newcommand{\usepmacro}[3][]{%
  \edef\tempa{#1}%
  \textcolor{macrocolor}{\string#2}%
  \string{\textcolor{parametercolor}{#3}\string}%
  \ifx\tempa\@empty\else (#1)\fi%
  #2{#3}\index{\string\verb+\string#2+}%
}

\newenvironment{decl}[1][]%
    {\par\small\addvspace{4.5ex plus 1ex}%
     \vskip -\parskip
     \ifx\relax#1\relax
        \def\@decl@date{}%
     \else
        \def\@decl@date{\NEWfeature{#1}}%
     \fi
     \noindent\hspace{-\leftmargini}%
     \begin{tabular}{|l|}\hline\ignorespaces}%
    {\\\hline\end{tabular}\nobreak\@decl@date\par\nobreak
     \vspace{2.3ex}\vskip -\parskip}

\renewcommand{\arg}[1]{{\tt\string{}\m{#1}{\tt\string}}}
\newcommand{\m}[1]{\mbox{\color{parametercolor}$\langle$\it #1\/$\rangle$}}
\setcounter{IndexColumns}{2}
\EnableCrossrefs
\CodelineIndex
\RecordChanges
\begin{document}
\begin{CJK*}{UTF8}{rm}
  \DocInput{CJKpunct.dtx}
  \PrintIndex
  \newpage
\end{CJK*}
\end{document}
%</driver>
%
%<*CJKpunct>
\def\fileversion{4.8.4}
\def\filedate{2016/05/14}
\ProvidesPackage{CJKpunct}[\filedate\space\fileversion]
%</CJKpunct>
%
% \fi
%
% \makeatletter                         ^^A% To document @-cmds
% \errorcontextlines=999                ^^A% Show up all my mistakes
%
% \CheckSum{936}
% \GetFileInfo{CJKpunct.sty}
%
% \title{CJKpunct 使用说明}
% \author{张林波 \quad 孙文昌}
% \date{2016 年 5 月 14 日}
% \maketitle
%
%
% \def\CJKpunct{{\textcolor{blue}{\texttt{CJKpunct}}}}
% \CJKpunct\ 是一个 \LaTeX\ 宏包, 用于排版中文标点,提供多种标点样式,以实现行末对齐、
% 标点挤压等效果。 其基本方法来自 CCT。
% \section{调用方式}
% \parindent 2em
% \parskip 5pt
% \begin{decl}
%   \defmacro{usepackage}\{\textcolor{parametercolor}{CJKpunct}\}
% \end{decl}
%
% \section{常用宏命令}
% \begin{decl}
%  \defmacro{punctstyle}\arg{punct style}
% \end{decl}
%
% 设置标点格式,有效值分别为
%
% \begin{tabular}{ll}
%   punct style & \\
%   \textcolor{parametercolor}{banjiao}       & 半角式 \\
%   \textcolor{parametercolor}{quanjiao}     & 全角式\\
%   \textcolor{parametercolor}{kaiming}      & 开明式\\
%   \textcolor{parametercolor}{hangmobanjiao}& 行末半角式\\
%   \textcolor{parametercolor}{CCT}          &    CCT 格式\\
%   \textcolor{parametercolor}{plain}          &   CJK 缺省格式
% \end{tabular}
%
%
% 注意:为了得到最好的排版效果,需要制作字体相关的 CJKpunct.spa 文件。请参考
% \nolinkurl{TDS/doc/latex/cjkpunct/setpunct/README.txt}。
%
%
% \begin{decl}
%   \defmacro{CJKpunctallowbreakbetweenpuncts} \\
%   \defmacro{CJKpunctnobreakbetweenpuncts}
% \end{decl}
%
% 缺省状态下,\CJKpunct\ 禁止在相邻的标点间换行(行末半角和 CJK 缺省格式除外)。 使用
%
% \defmacro{CJKpunctallowbreakbetweenpuncts}\newline
% 改变这一设置。注意:行末半角和 CJK 缺省格式总是允许相邻标点间换行。
%
% \begin{decl}
%  \defmacro{CJKpunctsetkern}\arg{标点1}\arg{标点2}\arg{间距}
% \end{decl}
%
%    设置标点 1 与标点 2 之间的距离。 例如,\defmacro{CJKpunctsetkern}\{:\}\{“\}\{0.4em\}
%
% \begin{decl}
%  \defmacro{CJKpunctmapfamily}\arg{font encoding}\arg{font family}\arg{font series}\arg{font shape}
%    \arg{punct family}
% \end{decl}
%
%  缺省状态下, \CJKpunct\ 根据 CJKfamily 确定当前标点符号的实际字体。 但这对C19com等组合字体
%  不适用。此命令提供一个解决方案:用户可以指定 \arg{font encoding}/ \arg{font family}/
% \arg{font series}/ \arg{font shape} 所对应的标点符号字体。例如,对于C19rm, 可以做如下设置:
%
% \begin{verbatim}
%   \CJKpunctmapfamily{C19}{rm}{m}{n}{song}
%   \CJKpunctmapfamily{C19}{rm}{bx}{n}{hei}
%   \CJKpunctmapfamily{C19}{rm}{m}{sl}{song}
%   \CJKpunctmapfamily{C19}{rm}{bx}{sl}{hei}
%   \CJKpunctmapfamily{C19}{rm}{m}{it}{kai}
%   \CJKpunctmapfamily{C19}{rm}{bx}{it}{kai}
% \end{verbatim}
%
% \section{制作 CJKpunct.spa 文件}
%
%  文件 CJKpunct.spa 保存了字体相关的标点符号宽度。制作方法请参考
%  \nolinkurl{TDS/doc/latex/cjkpunct/setpunct/README.txt}。
%
% \StopEventually{}
%
%
% \clearpage
% \part{CJKpunct.sty}
%
%
% \iffalse
%<*CJKpunct>
% \fi
% \fontsize{10pt}{10pt}\selectfont
% CJKpunct
%    \begin{macrocode}
\endlinechar \m@ne

\newif\if@CJKpunct
\newif\if@CJKpunct@dokerning
\newcount\CJKpunct@cnta
\newcount\CJKpunct@cntb
\newcount\CJKpunct@cntc
\newcount\CJKpunct@cntd
\newcount\CJKpunct@cnte
%    \end{macrocode}
%
% 为使 \CJKpunct\ 起作用,重定义一些 CJK 宏。
%
%    \begin{macrocode}
\let\CJKo@testLastCJK\CJK@testLastCJK
\def\CJKpunct@testLastCJK{
  \global\CJK@false
  \global\edef\CJKpunct@lastkern{\number\lastkern}}

\let\CJKo@testLastKern\CJK@testLastKern
\def\CJKpunct@testLastKern{
  \global\CJK@false}

\let\CJKo@testPrePunct\CJK@testPrePunct
\let\CJKo@testPostPunct\CJK@testPostPunct
\def\CJKpunct@testPrePunct#1#2#3{}
\def\CJKpunct@testPostPunct#1#2#3{}


\let\CJKo@nobreakglue\CJK@nobreakglue
%    \end{macrocode}
%
%  \defmacro{CJKsymbol} 的定义中需要三重括号以保证兼容 CJKfntef 宏包。
%
%    \begin{macrocode}

\let\CJKosymbol\CJKsymbol
\def\CJKpunct@CJKsymbol#1{
  {{{
  \ifnum\CJKpunct@lastkern>0\relax
    \ifnum\CJKpunct@lastcharclass=0\relax
      \CJKglue
    \else
      \CJKpunct@ULspecials
    \fi
  \fi
  \CJKosymbol{#1}
  \gdef\CJKpunct@lastcharclass{0}}}}}

\def\CJKpunct@lastcharclass{0}
\def\CJKpunct@lastkern{0}
%    \end{macrocode}
%
%  标点符号的排版规则:
%
%    \begin{macrocode}
\let\CJKopunctsymbol\CJKpunctsymbol
\def\CJKpunct@CJKpunctsymbol#1{
  \CJKpunct@setfamily
  \CJKpunct@setmarginkerning
  \edef\CJKpunct@currentpunct{\CJK@plane/\the#1}
  \ifcsname CJKpunct@\CJK@enc @\CJKpunct@currentpunct\endcsname
    \edef\CJKpunct@currentcharclass{
      \csname CJKpunct@\CJK@enc @\CJKpunct@currentpunct\endcsname}
    {{{%  We need three braces for CJKulem to work
    \@CJKpunctfalse
    \ifnum\CJKpunct@lastkern>0\relax
      \ifnum\CJKpunct@lastcharclass>0\relax
        \unkern
        \unkern
        \ifnum\CJKpunct@punctstyle>0\relax
          \@CJKpuncttrue
        \else
          \ifcsname CJKpunct@specialpunct\CJK@enc \CJKpunct@currentpunct\endcsname
            \@CJKpuncttrue
          \fi
        \fi
      \fi
    \fi
    \if@CJKpunct
      \CJKpunct@unskip
      \CJKpunct@setkern{\CJKpunct@lastpunct}{\CJKpunct@currentpunct}
      \kern \csname CJKpunct\CJKpunct@punctstyle @\CJK@enc @\CJKpunct@family
        @kern\CJKpunct@lastpunct @\CJKpunct@currentpunct\endcsname
      \CJKpunct@nobreak
    \else
      \CJKpunct@ULspecials
      \ifnum\CJKpunct@currentcharclass=1\relax
        \hskip \csname CJKpunct\CJKpunct@punctstyle @\CJK@enc @\CJKpunct@family
          @lglue@\CJKpunct@currentpunct\endcsname  plus 0.1em minus 0.1 em
      \else
        \ifcsname CJKpunct@specialpunct\CJK@enc \CJKpunct@currentpunct\endcsname
          \CJKglue  % breakable
        \else
          \nobreak
        \fi
      \fi
    \fi
    \global\edef\CJKpunct@lastpunct{\CJKpunct@currentpunct}

    \vrule width \csname CJKpunct\CJKpunct@punctstyle @\CJK@enc @\CJKpunct@family
      @lrule@\CJKpunct@currentpunct\endcsname depth \z@ height \z@

    \CJKopunctsymbol{#1}
    \vrule width \csname CJKpunct\CJKpunct@punctstyle @\CJK@enc @\CJKpunct@family
      @rrule@\CJKpunct@currentpunct\endcsname depth \z@ height \z@

    \ifnum\CJKpunct@currentcharclass=2\relax
      \hskip \csname CJKpunct\CJKpunct@punctstyle @\CJK@enc @\CJKpunct@family
        @rglue@\CJKpunct@currentpunct\endcsname  plus 0.1em minus 0.1 em
    \fi
    \global\let\CJKpunct@lastcharclass\CJKpunct@currentcharclass}}}
  \else
    \CJKsymbol{#1}
    \global\def\CJKpunct@lastcharclass{0}
  \fi}
%    \end{macrocode}
%
% 设置当前 font family.
%
%    \begin{macrocode}
\def\CJKpunct@setfamily{
  \ifcsname \CJK@enc @\CJK@family @\f@series @\f@shape\endcsname
    \global\edef\CJKpunct@family{\csname \CJK@enc @\CJK@family @\f@series @\f@shape\endcsname}
  \else
    \global\edef\CJKpunct@family{\CJK@family}
  \fi}

\def\CJKpunctmapfamily#1#2#3#4#5{
  \expandafter\edef\csname #1@#2@#3@#4\endcsname{#5}}

%    \end{macrocode}
%
%  CJK 缺省标点符号格式
%
%    \begin{macrocode}
\def\CJKpunct@plainpunctsymbol#1#2{
  \CJKpunctsymbol{#2}}
%    \end{macrocode}
%
% 设置标点符号边界宽度。
%
%    \begin{macrocode}
\def\CJKpunct@setmarginkerning{
  \ifcsname CJKpunct @\CJKpunct@punctstyle @\CJK@enc @\CJKpunct@family\endcsname
  \else
    \expandafter\gdef\csname CJKpunct @\CJKpunct@punctstyle @\CJK@enc
      @\CJKpunct@family\endcsname{}
    \ifcsname CJKpunct@\CJKpunct@family @spaces\endcsname
      \PackageInfo{CJKpunct}{use punctuation spaces for family '\CJKpunct@family'
        \space with punctstyle (\CJKpunct@currentpunctstyle)}\relax
      \edef\CJKpunct@spaces{\csname CJKpunct@\CJKpunct@family @spaces\endcsname}
    \else
      \ifcsname CJKpunct@spaces@\CJKpunct@family\endcsname
      \else
        \PackageInfo{CJKpunct}{punctuation spaces for family '\CJKpunct@family' do not exist.
          \space Use family 'def' instead.}\relax
         \global\expandafter\def\csname CJKpunct@spaces@\CJKpunct@family\endcsname{}
      \fi
      \edef\CJKpunct@spaces{\csname CJKpunct@def@spaces\endcsname}
    \fi
    \CJKpunct@cnta=0\relax
    \expandafter\CJKpunct@@setmarginkerning\CJKpunct@spaces
  \fi}

\def\CJKpunct@@setmarginkerning#1,#2,{
  \edef\CJKpunct@temp{#1}
  \ifx\CJKpunct@temp\@empty
    \def\CJKpunct@temp{}
  \else
    \def\CJKpunct@temp{\CJKpunct@@setmarginkerning}
    \ifnum\CJKpunct@cnta<12
      \def\CJKpunct@lr{l}
    \else
      \def\CJKpunct@lr{r}
    \fi
    \edef\CJKpunct@encpn{\csname CJKpunct@pn@\CJK@enc @\the\CJKpunct@cnta\endcsname}
    \if l\CJKpunct@lr
      \expandafter\gdef\csname CJKpunct@\CJK@enc @\CJKpunct@encpn\endcsname{1}
    \else
      \expandafter\gdef\csname CJKpunct@\CJK@enc @\CJKpunct@encpn\endcsname{2}
    \fi

    \@CJKpunct@dokerningtrue
    \ifnum\CJKpunct@punctstyle=\CJKpunct@ps@plain\relax
      \@CJKpunct@dokerningfalse
    \else
      \ifcsname CJKpunct@specialpunct\CJK@enc\CJKpunct@encpn\endcsname
        \@CJKpunct@dokerningfalse
      \fi
    \fi

    \ifnum\CJKpunct@punctstyle=\CJKpunct@ps@banjiao
      \def\CJKpunct@sidespaces{12}
    \else
      \def\CJKpunct@sidespaces{15}
    \fi

    \ifnum\CJKpunct@cnta=12\relax
      {\CJKpunct@cntb=#1\relax
      \advance\CJKpunct@cntb #2\relax
      \advance\CJKpunct@cntb 2\relax
      \CJKpunct@numtostring{\CJKpunct@cntb}
      \edef\CJKpunct@temp{\csname CJKpunct@pn@\CJK@enc @12\endcsname}
      \CJKpunct@cntc=0\relax
      \loop
        \global\expandafter\edef\csname CJKpunct\the\CJKpunct@cntc
          @\CJK@enc @\CJKpunct@family @kern\CJKpunct@temp @\CJKpunct@temp\endcsname{
            -0.\CJKpunct@decimal em}
        \advance \CJKpunct@cntc 1\relax
      \ifnum\CJKpunct@cntc<6\repeat}
    \fi
    \if@CJKpunct@dokerning
      \CJKpunct@cntb=#1\relax
      \advance\CJKpunct@cntb -\CJKpunct@sidespaces\relax
      \ifnum\CJKpunct@cntb<0\relax
        \CJKpunct@cntb=0\relax
      \fi
      \CJKpunct@cntc=#2\relax
      \advance\CJKpunct@cntc -\CJKpunct@sidespaces\relax
      \ifnum\CJKpunct@cntc<0\relax
        \CJKpunct@cntc=0\relax
      \fi

      \CJKpunct@cntd=\CJKpunct@cntb
      \advance\CJKpunct@cntd\CJKpunct@cntc\relax

      \ifcase\CJKpunct@punctstyle
          % hangmobanjiao
      \or % quanjiao
      \or % banjiao
        \advance\CJKpunct@cntd -50\relax
      \or % kaiming
        \ifcsname CJKpunct@kaiming\CJK@enc\CJKpunct@encpn\endcsname
        \else
          \advance\CJKpunct@cntd -50\relax
        \fi
      \or %CCT
        \advance\CJKpunct@cntd -20\relax
      \fi
      \CJKpunct@cnte=\CJKpunct@cntd
      \ifnum\CJKpunct@cntd<0\relax
        \CJKpunct@cntd=0\relax
      \fi
    \else
      \CJKpunct@cntb=0\relax
      \CJKpunct@cntc=0\relax
      \CJKpunct@cntd=0\relax
      \CJKpunct@cnte=0\relax
    \fi
    \CJKpunct@numtostring{\CJKpunct@cntb}
    \global\expandafter\edef\csname CJKpunct\CJKpunct@punctstyle
      @\CJK@enc @\CJKpunct@family @lrule@\CJKpunct@encpn\endcsname{
        -0.\CJKpunct@decimal em}
    \CJKpunct@numtostring{\CJKpunct@cntc}
    \global\expandafter\edef\csname CJKpunct\CJKpunct@punctstyle
      @\CJK@enc @\CJKpunct@family @rrule@\CJKpunct@encpn\endcsname{
        -0.\CJKpunct@decimal em}
    \CJKpunct@numtostring{\CJKpunct@cntd}
    \global\expandafter\edef\csname CJKpunct\CJKpunct@punctstyle
      @\CJK@enc @\CJKpunct@family @\CJKpunct@lr glue@\CJKpunct@encpn\endcsname{
        0.\CJKpunct@decimal em}
    \global\expandafter\edef\csname CJKpunct\CJKpunct@punctstyle
      @\CJK@enc @\CJKpunct@family @\CJKpunct@lr oglue@\CJKpunct@encpn\endcsname{
        \the\CJKpunct@cnte}
  \fi
  \advance \CJKpunct@cnta 1\relax
  \CJKpunct@temp}

\def\CJKpunct@numtostring#1{
  \edef\CJKpunct@decimal{\the#1}
  \ifnum\CJKpunct@decimal<10\relax
    \edef\CJKpunct@decimal{0\CJKpunct@decimal}
  \fi}
%    \end{macrocode}
%
%  设置相邻标点之间的距离
%
%    \begin{macrocode}
\def\CJKpunct@setkern#1#2{
  \ifcsname CJKpunct\CJKpunct@punctstyle @\CJK@enc @\CJKpunct@family @kern#1@#2\endcsname
  \else
    \CJKpunct@cnta=0\relax
    \ifcsname CJKpunct\CJKpunct@punctstyle @\CJK@enc @\CJKpunct@family @roglue@#1\endcsname
      \advance\CJKpunct@cnta\csname
        CJKpunct\CJKpunct@punctstyle @\CJK@enc @\CJKpunct@family @roglue@#1\endcsname
    \fi
    \ifcsname CJKpunct\CJKpunct@punctstyle @\CJK@enc @\CJKpunct@family @loglue@#2\endcsname
      \advance\CJKpunct@cnta\csname CJKpunct\CJKpunct@punctstyle @\CJK@enc @\CJKpunct@family
        @loglue@#2\endcsname
    \fi
    \relax
    \ifcase\CJKpunct@punctstyle
        % hangmobanjiao
    \or % quanjiao
      \advance\CJKpunct@cnta -50\relax
    \or % banjiao
    \or % kaiming
      \ifcsname CJKpunct@kaiming\CJK@enc#1\endcsname
        \advance\CJKpunct@cnta -\csname CJKpunct\CJKpunct@punctstyle @\CJK@enc
          @\CJKpunct@family @roglue@#1\endcsname
        \ifcsname CJKpunct\CJKpunct@punctstyle @\CJK@enc @\CJKpunct@family @loglue@#2\endcsname
          \advance\CJKpunct@cnta -50\relax
        \fi
      \fi
    \fi
    \ifnum\CJKpunct@cnta<0\relax
      \CJKpunct@cnta=0\relax
    \fi
    \CJKpunct@numtostring{\CJKpunct@cnta}
    \global\expandafter\edef\csname
      CJKpunct\CJKpunct@punctstyle @\CJK@enc @\CJKpunct@family @kern#1@#2\endcsname{
        0.\CJKpunct@decimal em}
  \fi}
%    \end{macrocode}
%
% CJKfntef 宏包兼容命令:
%
%    \begin{macrocode}

\let\CJKpunct@unskip\unskip
\def\CJKpunct@UL@unskip{
  \ifcsname CJKpunct\CJKpunct@punctstyle @\CJK@enc @\CJKpunct@family
    @rglue@\CJKpunct@lastpunct\endcsname
    \hskip -\csname CJKpunct\CJKpunct@punctstyle @\CJK@enc @\CJKpunct@family
      @rglue@\CJKpunct@lastpunct\endcsname \relax
  \fi}

\@ifundefined{UL@hskip}{\let\UL@hskip\relax}{}

\def\CJKpunct@punctUL@group{
  \ifx\hskip\UL@hskip
      \egroup
      \UL@stop
      \UL@start
      \bgroup
  \fi}

\def\CJKpunct@ULspecials{}

\AtBeginDocument{
  \ifcsname UL@hook\endcsname
    \addto@hook\UL@hook{\let\CJK@ignorespaces\ignorespaces
       \let\CJKpunct@unskip\CJKpunct@UL@unskip
       \let\CJKpunct@ULspecials\CJKpunct@punctUL@group}
  \fi}
%    \end{macrocode}
%
%  设置相邻标点间是否允许换行(缺省不允许)。
%
%    \begin{macrocode}
\def\CJKpunctallowbreakbetweenpuncts{
  \def\CJKpunct@nobreak{
    \ifnum\CJKpunct@lastcharclass=2
      \hskip 0pt
    \fi}}

\def\CJKpunctnobreakbetweenpuncts{
  \let\CJKpunct@nobreak\nobreak}
\CJKpunctnobreakbetweenpuncts
%    \end{macrocode}
%
% 标点符号样式:
%
%    \begin{macrocode}
\def\CJKpunctstyle#1{
  \ifcsname CJKpunct@ps@#1\endcsname
    \edef\CJKpunct@currentpunctstyle{#1}
    \edef\CJKpunct@punctstyle{\csname CJKpunct@ps@#1\endcsname}
    \ifnum\CJKpunct@punctstyle=\CJKpunct@ps@plain\relax
      \CJKpunctallowbreakbetweenpuncts
      \let\CJK@testLastCJK\CJKo@testLastCJK
      \let\CJK@testLastKern\CJKo@testLastKern
      \let\CJK@testPrePunct\CJKo@testPrePunct
      \let\CJK@testPostPunct\CJKo@testPostPunct
      \let\CJKpunct@punctsymbol\CJKpunct@plainpunctsymbol
      \let\CJKsymbol\CJKosymbol
      \let\CJKpunctsymbol\CJKopunctsymbol
      \let\CJK@nobreakglue\CJKo@nobreakglue
      \let\CJKpunct@utfsymbol\CJKpunct@utfbsymbol
    \else
      \let\CJK@testLastCJK\CJKpunct@testLastCJK
      \let\CJK@testLastKern\CJKpunct@testLastKern
      \let\CJK@testPrePunct\CJKpunct@testPrePunct
      \let\CJK@testPostPunct\CJKpunct@testPostPunct
      \let\CJKpunct@punctsymbol\CJKpunct@@punctsymbol
      \let\CJKsymbol\CJKpunct@CJKsymbol
      \let\CJKpunctsymbol\CJKpunct@CJKpunctsymbol
      \let\CJK@nobreakglue\relax
      \let\CJKpunct@utfsymbol\CJKpunct@utfasymbol
    \fi
  \else
    \PackageWarning{CJKpunct}{Punctstyle #1\space is not defined.}\relax
  \fi}

\let\punctstyle\CJKpunctstyle
\def\CJKpunct@ps@hangmobanjiao{0}
\def\CJKpunct@ps@marginkerning{0}
\def\CJKpunct@ps@quanjiao{1}
\def\CJKpunct@ps@fullwidth{1}
\def\CJKpunct@ps@banjiao{2}
\def\CJKpunct@ps@halfwidth{2}
\def\CJKpunct@ps@kaiming{3}
\def\CJKpunct@ps@mixedwidth{3}
\def\CJKpunct@ps@CCT{4}
\def\CJKpunct@ps@plain{5}
\AtBeginDocument{\punctstyle{quanjiao}}

\def\CJKplainout{\punctstyle{plain}}
\let\CJKnormalout\relax
%    \end{macrocode}
%
% 允许用户使用 \defmacro{CJKpunctsetkern} 调整相邻标点之间的距离
%
%    \begin{macrocode}
\def\CJKpunctsetkern#1#2#3{
  \CJKpunct@setplanenumber{#1}
  \edef\CJKpunct@pna{\CJKpunct@char@pn}
  \CJKpunct@setplanenumber{#2}
  \edef\CJKpunct@pnb{\CJKpunct@char@pn}
  \global\expandafter\edef\csname CJKpunct\CJKpunct@punctstyle
     @\CJK@enc @\CJKpunct@family @kern\CJKpunct@pna @\CJKpunct@pnb\endcsname{
       #3}}

\def\CJKpunct@setplanenumber#1{{
  \def\CJK@testPrePunct##1##2##3{
    \global\edef\CJKpunct@charplane{\CJK@plane}
    \global\edef\CJKpunct@charnumber{\the\@tempcnta}}
  \savebox\voidb@x{#1}
  \global\edef\CJKpunct@char@pn{\CJKpunct@charplane/\CJKpunct@charnumber}}}

%
% 标点符号表, 不能改变顺序!!
% pre ‘“「『〔([{〈《〖【
% post  —…、。,.:;!?%〕)]}〉》〗】’”」』
%    \end{macrocode}
%
%  设置标点符号所对应 CJKplane 和 字符序号, 与编码有关。
%
%    \begin{macrocode}
\def\CJKpunct@punctlist#1{
  \CJKpunct@cnta=0\relax
  \def\CJKpunct@enc{#1}
  \CJKpunct@setpunctfamilynumber}

\def\CJKpunct@setpunctfamilynumber#1,{
  \edef\CJKpunct@temp{#1}
  \ifx\CJKpunct@temp\@empty
    \def\CJKpunct@temp{}
  \else
    \expandafter\def\csname CJKpunct@pn@\CJKpunct@enc @\the\CJKpunct@cnta\endcsname{#1}
    \advance \CJKpunct@cnta 1\relax
    \def\CJKpunct@temp{\CJKpunct@setpunctfamilynumber}
  \fi
  \CJKpunct@temp}

\CJKpunct@punctlist{C70}20/24,20/28,30/12,30/14,30/20,ff/8,ff/59,ff/91,%
30/8,30/10,30/22,30/16,%
20/20,20/38,30/1,30/2,ff/12,ff/14,ff/26,ff/27,ff/1,ff/31,ff/5,30/21,ff/9,%
ff/61,ff/93,30/9,30/11,30/23,30/17,20/25,20/29,30/13,30/15,,

%gb
\CJKpunct@punctlist{C10}01/13,01/15,01/23,01/25,01/17,01/195,01/246,02/22,01/19,%
01/21,01/27,01/29,%
01/9,01/12,01/1,01/2,01/199,01/201,01/213,01/214,01/188,01/218,01/192,01/18,%
01/196,01/248,02/24,01/20,01/22,01/28,01/30,01/14,01/16,01/24,01/26,,

%gbk
\CJKpunct@punctlist{C19}25/45,25/47,25/55,25/57,25/49,26/163,26/214,26/246,25/51,%
25/53,25/59,25/61,%
25/41,25/44,25/33,25/34,26/167,26/169,26/181,26/182,26/156,26/186,26/160,%
25/50,26/164,26/216,26/248,25/52,25/54,25/60,25/62,25/46,25/48,25/56,25/58,,

\def\CJKpunct@totalpuncts{35}
%    \end{macrocode}
%
% 恢复 CJKutf8 重定义的引号
%
%    \begin{macrocode}
\ifcsname DeclareUnicodeCharacter\endcsname
  \DeclareUnicodeCharacter{2018}{\CJKpunct@utfsymbol{"80}{"98}}
  \DeclareUnicodeCharacter{2019}{\CJKpunct@utfsymbol{"80}{"99}}
  \DeclareUnicodeCharacter{201C}{\CJKpunct@utfsymbol{"80}{"9C}}
  \DeclareUnicodeCharacter{201D}{\CJKpunct@utfsymbol{"80}{"9D}}
  \DeclareUnicodeCharacter{2014}{\CJKpunct@utfsymbol{"80}{"94}}
  \DeclareUnicodeCharacter{2026}{\CJKpunct@utfsymbol{"80}{"A6}}
\fi
\def\CJKpunct@utfasymbol#1#2{
  \CJK@punctchar{\CJK@uniPunct}{0}{#1}{#2}}
\def\CJKpunct@utfbsymbol#1#2{
  \ifnum #2=148 %
    \textemdash
  \else
    \ifnum #2=166 %
      \textellipsis
    \else
      \ifnum #2=152 %
        \textquoteleft
      \else
        \ifnum #2=153 %
          \textquoteright
        \else
          \ifnum #2=156 %
            \textquotedblleft
          \else
            \ifnum #2=157 %
              \textquotedblright
            \fi
          \fi
        \fi
      \fi
    \fi
  \fi}
%    \end{macrocode}
%
% 省略号和破折号:
%
%    \begin{macrocode}
\def\CJKpunct@setspecialpunct#1#2{
  \expandafter\def\csname CJKpunct@specialpunct#1#2\endcsname{}}
\CJKpunct@setspecialpunct{C70}{20/20}
\CJKpunct@setspecialpunct{C70}{20/38}
\CJKpunct@setspecialpunct{C19}{25/41}
\CJKpunct@setspecialpunct{C19}{25/44}
\CJKpunct@setspecialpunct{C10}{01/9}
\CJKpunct@setspecialpunct{C10}{01/12}
%    \end{macrocode}
%
%  开明式中的全角标点。
%
%    \begin{macrocode}
\def\CJKpunct@setkaimingpunct#1#2{
  \expandafter\def\csname CJKpunct@kaiming#1#2\endcsname{}}
\CJKpunct@setkaimingpunct{C70}{30/02}
\CJKpunct@setkaimingpunct{C70}{ff/1}
\CJKpunct@setkaimingpunct{C70}{ff/31}
\CJKpunct@setkaimingpunct{C19}{25/34}
\CJKpunct@setkaimingpunct{C19}{26/156}
\CJKpunct@setkaimingpunct{C19}{26/186}
\CJKpunct@setkaimingpunct{C10}{01/2}
\CJKpunct@setkaimingpunct{C10}{01/188}
\CJKpunct@setkaimingpunct{C10}{01/218}
%    \end{macrocode}
%
% 缺省标点符号宽度
%
%    \begin{macrocode}
\def\CJKpunct@def@spaces{69,18,60,6,63,2,63,3,69,8,69,6,69,1,39,%
37,63,4,56,2,63,5,63,6,6,6,12,11,23,50,24,54,16,71,20,69,12,76,13,%
74,26,61,3,50,3,4,8,69,6,69,2,69,38,39,4,62,2,55,5,62,7,62,16,71,9,%
58,3,62,3,62,,,}
%    \end{macrocode}
%
% 调入字体相关的设置
%
%    \begin{macrocode}
\InputIfFileExists{CJKpunct.spa}{}{}

\endlinechar `\^^M
%    \end{macrocode}
%
% \iffalse
%</CJKpunct>
% \fi
%
% \Finale
\endinput
