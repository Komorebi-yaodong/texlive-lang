\ProvidesFile{dhua.tex}[2011/09/19 make dhua.pdf about dhua.sty]
\title{\textsf{dhua.sty}\\---\\German Abbreviations 
       Using Thin Space\thanks{This
       document describes version
       \textcolor{blue}{\UseVersionOf{\jobname.sty}}
       of \textsf{\jobname.sty} as of \UseDateOf{\jobname.sty}.}}
% \listfiles
{ \RequirePackage{makedoc} \ProcessLineMessage{}
  \MakeJobDoc{16}
  {\SectionLevelTwoParseInput}  }
\documentclass[fleqn]{article}      %% TODO paper dimensions!?
\input{makedoc.cfg}                 %% shared formatting settings
\providecommand*\pkg{\pkgnamefmt}   %% TODO should be in makedoc.cfg
%% 2011/09/09:
\AddQuotes                          %% niceverb
\hypersetup{%
    pdftitle=dhua.sty for German abbreviations using thin space,
    pdfsubject=documenting dhua.sty
}
\MDkeywords{German typography; 
            web typography, language support, macro programming}
\usepackage{dhua}
% \makeatletter
\newenvironment*{german}
                {\par
                 \it\frenchspacing\DontAddQuotes
                 \let\qtd\deqtd}
%                 {\par\@endpefalse}    %% strange: in vain
                {\par}
% \makeatother
% \usepackage[T1]{fontenc}    %% TODO just for ...!?
% \newcommand*{\deqtd}[1]{\guillemotleft#1\guillemotright}
\let\EnToday\today
\usepackage{german} \mdqoff \let\today\EnToday
\renewcommand*{\contentsname}{Contents} 
\renewcommand*{\abstractname}{Abstract/Zusammenfassung}
\newcommand*{\deqtd}[1]{\glqq#1\grqq}
\newcommand*{\HTML}{\acro{HTML}}
\sloppy 
\begin{document}
\maketitle
\begin{MDabstract}\DontAddQuotes
'dhua.sty' provides commands for German phrase abbreviations 
such as \qtd{\dh}\ that are recommended to use a 
\Wikienref{thin space}---set-up commands `\newdhua' and 
`\newtwopartdhua' %%% --- %% rm. 2011/09/16
as well as commands for single cases (e.g., `\zB' for \qtd{\zB}, 
to save you from typing `z.\,B.').
% %% 2011/09/15:
% Moreover, there are package options for (i)~web typography 
% and (ii)~automatical inclusion of `\xspace'.
Package options are intended to support generating \acro{PDF}
and \HTML\ from the same source, maybe automatically using 
`\xspace'.

\begin{german}                                  %% reworded 2011/09/16
  Das Paket 'dhua' bietet Befehle f\"ur sog.\ 
  mehrgliedrige Abk\"urzungen, f\"ur die
  \wikideref{Schmales Lehrzeichen}{schmale Leerzeichen}
  (\wikideref{Festabstand}{Festabst\"ande}) empfohlen werden.
  In die englische Paketdokumentation sind deutsche Hinweise 
  (kursiv) eingestreut.
\end{german}
\end{MDabstract}
\tableofcontents

%   \newpage
\section{Installing, Calling, Usage}
The file 'dhua.sty' is provided ready, installation only requires
putting it somewhere where \TeX\ finds it
(which may need updating the filename data
 base).\urlfoot{ukfaqref}{inst-wlcf}           %% corr. 2011/02/08

%% extended 2011/01/14:
Below the `\documentclass' line(s) and above `\begin{document}',
you load 'dhua.sty' (as usually) by
\[`\usepackage{dhua}'\]
or by 
\[`\usepackage[<option(s)>]{dhua}'\]
with the option(s) <option(s)> described in Section~\ref{sec:opt} 
(`[web]', `[xspace]'). A few macros for single abbreviations 
are described in Section~\ref{sec:single}, 
the macros `\newdhua' and `\newtwopartdhua' for defining such 
abbreviation macros are described in Section~\ref{sec:setup}.

\begin{german}
  Unten werden (i)~Paketoptionen `[web]' und `[xspace]', 
  (ii)~die Makros `\newdhua' und `\newtwopartdhua' 
  f\"ur die Definition einzelner Ab\-k\"ur\-zungs\-makros 
  sowie (iii)~einzelne vordefinierte Abk\"urzungsmakros beschrieben.
\end{german}

\section{Package File Header (Legalize)}
\ProvidesFile{dhua.cfg}[2011/09/19 local dhua.sty definitions]
%%
%% The user may want to use (some of) the following single macro 
%% names for a different purpose or so; to this end, a file 
%% `dhua.cfg' may contain a different set of definitions.
%% -- THIS HAS BEEN COPIED FROM `dhua.sty', 
%% YOU MAY EDIT IT!
%% 
%% |\idR| exemplifies multi-part abbreviations, 
%% where \qtd{multi} means ``more than two":
\newdhua{\idR}{i\DhuaSpace d\DhuaSpace R}
%   \show\idR
%% \begin{german}
%%   |\idR| erzeugt \qtd{\idR} -- ein Anwendungsfall 
%%   f\"ur |\newdhua|. Nachfolgend wird nur noch 
%%   |\newtwopartdhua| verwendet.
%% \end{german}
%%
%% \LaTeX\ actually defines |\dh| as something nordic 
%% (one of my earliest macro making experiences)
%% so we are \emph{re}defining it:
\PackageWarning{dhua}{Redefining \string\dh}
\let\dh\relax
\newtwopartdhua{\dh}{d}{h}
%   \show\dh
%% \begin{german}
%%   |\dh| wird hier \emph{umdefiniert}, um \qtd{\dh} zu bekommen.
%%   Die \"ubrigen Makros sind \qtd{normale} Anwendungs\-f\"alle 
%%   von `\newtwopartdhua', man achte aber noch auf die 
%%   Verwendung von `\protect'.
%% \end{german}
%%
%% \dqtd{Normal} cases                          %% TODO catchdq
%% (|\oae|  for \qtd{\oae}, 
%%  |\so|   for \qtd{\so},  |\su|   for \qtd{\su}, %% 2011/09/19
%%  |\uae|  for \qtd{\uae}, 
%%  |\ua|   for \qtd{\ua},  |\vglu| for \qtd{\vglu},
%%  |\vglo| for \qtd{\vglo}, 
%%  |\zB|   for \qtd{\zB},   %% was |\qtd{\zB}| until 2011/09/19
%%  |\zT|   for \qtd{\zT}):
\newtwopartdhua{\oae}{o}{\protect\"a} 
% \newtwopartdhua{\oae{o}{\"a} 
%   \show\oae
%% ---exemplifying use of |\protect| so the definition of `\oae'
%% has a single token `\"', not an expansion of `\"'. 
\newtwopartdhua{\so}  {s}{o} 
\newtwopartdhua{\su}  {s}{u} 
\newtwopartdhua{\ua}  {u}{a} 
\newtwopartdhua{\uae} {u}{\protect\"a} 
\newtwopartdhua{\vglu}{vgl}{u} 
\newtwopartdhua{\vglo}{vgl}{o} 
\newtwopartdhua{\zB}  {z}{B}
\newtwopartdhua{\zT}  {z}{T} 
\endinput

\section{Colophon}                      %% was `Coda' 2011/09/16
              %% rm. \enlargethispage{3\baselineskip} 2011/09/19
The English part of the documentation exemplifies a new 
(2011/09/09) function of 'niceverb.sty' v0.44: automatically 
enclose inline \TeX\ code in single quotation marks after 
`\AddQuotes'. 
I needed especially much time for this because group nesting 
spans several documentation pages. 

% On my Atari~ST, there must be another 'dhua.sty'. 
% I~guess it was quite worthless. 
% I think it is only about one year ago that I~became aware 
% of the Duden recommendation about thin unbreakable spaces. 
% I consider this older package 'dhua'~v0.\dots
% The present version number \qtd{v1.1} is an analogue 
% to my usual starting version number \qtd{v0.1}.
 
I spent much time with a special environment `{german}'
for the present purpose: the indent of the following 
paragraph was missing---until I added an empty 
documentation line. (Same with standard `{sloppypar}' 
environment, I don't understand it, tried 
`\@endpefalse' in vain.) 
%% modified 2011/09/16:
I don't like \ctanpkgref{babel}~\dots

%% 2011/09/16:
The German parts use 'niceverb''s `\DontAddQuotes' 
because of a different frequency of \TeX\ code. 
Even in the English parts I considered the single 
quotation marks bad and avoided them using \LaTeX's `\verb'.

And my terms \qtd{phrase abbreviation} and \qtd{abbreviation macro} 
may be bad, please help me~\dots

\end{document}

VERSION HISTORY

2011/09/14  for v0.1    renaming from `dhusw'
2011/09/15  for v0.1a   options in abstract
2011/09/16              more keywords, different \Provides..., 
                        Coda/Colophon extended
2011/09/19  for v0.11   Colophon on next page

