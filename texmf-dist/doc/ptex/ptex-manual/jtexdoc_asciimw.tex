\def\cs#1{$\backslash$\kern0pt#1}
\def\angle#1{$\langle$\kern0pt#1\kern0pt$\rangle$}
\documentstyle{jarticle}
\begin{document}
\title{日本語\TeX}
\author{倉沢 良一\\ASCII Corporation}
\date{昭和62年8月 \\ 昭和62年12月 \\ 平成元年10月}
\maketitle
\begin{abstract}
これは、日本語\TeX の使い方の注意点と、
\TeX の日本語化にともない変更あるいは追加された機能について
解説したドキュメントです。
基本的な\TeX の使い方については、\TeX bookや
\LaTeX: A document Preparation System. をご覧ください。
\end{abstract}

\section{日本語\TeX の概要}
日本語\TeX は、オリジナルの\TeX と完全なアッパーコンパチブルを保っています。
ですから、これまで欧文で作られたファイルはそのまま日本語\TeX に
かけられるはずです。
現在のバージョンで扱える漢字コードは、シフトJISおよびEUCです。
ただし、これらはソースファイルにパッチをあてて、makeし直さなければなりません。

日本語(2バイトコードキャラクタ)のハンドリングに関しては、
でき得る限り英字(1バイトコードキャラクタ)と同じように取り扱える
ようにしてあります。
したがって、特殊な使い方をしない限り英字と同じようにして、
原稿ファイルを作成していくことができます。

しかも、カレントフォントとして英字と漢字とを別々に持っているため、
英字と漢字の切り換えは特に気にする必要はなく、
そのまま混在して使うことができます。

和文と欧文の処理の違いとしてラインブレークのタイミングと
それに関係する禁則処理、さらに和文欧文が混在した場合のスペーシング
の処理などがあります。
日本語\TeX では、これらの処理を自動的に行っていますが、
柔軟性を高めるために、こうした処理に関係するパラメータを自由に再設定
できるようになっています。

フォントに関しては、1フォントでJISコードの第1水準、第2水準全てを
扱えるように拡張してありますから、英字フォントと同じように漢字フォントを扱え、
JFM(TFM)ファイルをつくることにより、自由に使用可能なフォントを増やして
いくことができます。

DVIファイルは、SET2 および PUT2 を使って2バイトコードを出力しているため、
これまでのものとの互換性は保たれています。
またDVIファイルには、JISコードを用いて出力しています。

\section{原稿を書く上での注意点}
基本的には、欧文と和文は全く同じようにして扱えます。
ただし、日本語特有の処理機能を持たせているため、
原稿を書くうえでは次のことに注意してください。
\begin{itemize}
\item コントロールシーケンスにも2バイトコードキャラクタ
	が使えるようになっています。
	したがって、コントロールシーケンスに続けて2バイトコードの
	文字を書き並べる場合は、必ず半角スペースやタブ等で間を区切ってください。
\item 欧文の場合、改行は単語間のスペースとして取り扱われますが、
	日本語の場合、原稿内では自由な箇所で改行が行えたほうが便利です。\\
	そこで日本語\TeX では、1行の終わりが1バイトコードの場合はスペーサー
	の挿入を行い、2バイトコードの場合は何も行わないようになっています。
\item 2バイトコードキャラクタと1バイトコードキャラクタが
	連続する場合、自動的に\cs{xkanjiskip}に設定されてているグルーを
	挿入します。ただしこの処理は、どの1バイトコードキャラクタとの
	間で行うか\cs{xspcode}によって指定でき、デフォルトではa--z,
	A--Z, 0--9との間で行われるように設定されています。
	この処理を特定の箇所で禁止したい場合は、
	そこに\cs{kern0pt}、\cs{hskip0pt}などを挿入してください。
\item 禁則処理は、禁則の対象となるキャラクタの前方あるいは
	後方にペナルティーを自動的に挿入することで実現しています。
	このペナルティーの設定は\cs{prebreakpenalty}、\cs{postbreakpenalty}
	によって行います。デフォルト値は、kinsoku.texに記述されています。
\item 現在、jtex、jlatexで扱えるフォントは明朝体とゴシック体です。
	これらのフォントの指定は、
	jtex, jlatexともに、\cs{mc}、\cs{gt}で行えます。
	これら以外のフォントは、英字フォントと同じように、
	対象のフォントにあわせてJFM(TFM)を作れば使用できます。
\item フォントの切り換えは、英字と漢字は独立して行われます。
	jlatexでは、\cs{large},\cs{small}などサイズ変更の
	コントロールシーケンスによって、
	連動してフォントをチェンジするようにしていますが、
	jtexではそのような操作は一切行っていませんので注意してください。
	ただし、指定したフォントが印字されるためには、それらの字体、サイズの
	フォントがそろっていなければ行えません。
	jlatexでは、次のフォントが存在するものと仮定しています。
	\begin{quote}
	min5, min6, min7, min8, min9, min10,
	min10 magstephalf, min10 magstep1,
	min10 magstep2, min10 magstep3,
	min10 magstep4, min10 magstep5, \\
	goth5, goth6, goth7, goth8, goth9, goth10,
	goth10 magstephalf, goth10 magstep1,
	goth10 magstep2, goth10 magstep3,
	goth10 magstep4, goth10 magstep5
	\end{quote}
	これらが、きちんと印字されるかどうかはプリンタドライバの責任です。
	また、これらのフォントにはカーニング・パラメータが設定されています。
	カーニングを行いたくない場合は、上記のフォントの代りに以下のものを
	使用してください。こちらのフォントは、カーニングのパラメータが設定
	されていないことを除いて上記のものと全く同じものです。
	\begin{quote}
	nmin5, nmin6, nmin7, nmin8, nmin9, nmin10,
	nmin10 magstephalf, nmin10 magstep1,
	nmin10 magstep2, nmin10 magstep3,
	nmin10 magstep4, nmin10 magstep5, \\
	ngoth5, ngoth6, ngoth7, ngoth8, ngoth9, ngoth10,
	ngoth10 magstephalf, ngoth10 magstep1,
	ngoth10 magstep2, ngoth10 magstep3,
	ngoth10 magstep4, ngoth10 magstep5
	\end{quote}
\item 現在のバージョンでは、全角スペースはほかの漢字キャラクタと同じように
	扱われます。
	半角スペースのような特別の処理は行っていませんので注意してください。
\item コントロールシーケンス名にも全角文字を使用することができますが
	次の点に注意してください。
	\begin{enumerate}
	\item 全角の1区、2区、7区に含まれる文字は、カテゴリーコードの
		12つまり `Other\_character' と同様の扱いになります。
		したがって、これらの文字は ``エスケープ文字$+$1文字'' の
		コントロールシーケンスとしてのみ扱えます。
	\item そのほかの文字に付いては、カテゴリーコードの11(letter)と
		同様に扱われます。
		また、これらの文字は半角文字と混在して使用することが
		できます。
	\end{enumerate}
\item 半角カナは使えません。原稿内に半角カナのコードが入らないようにし
	てください。
\end{itemize}

\section{追加されたプリミティブ}

日本語\TeX にはつぎのプリミティブが追加されています。
\begin{itemize}
\item \cs{{\bf kanjiskip}}
	\begin{quote}
	\cs{kanjiskip}$=$\angle{dimen}
	\end{quote}
	連続する2バイトコード間に自動的に挿入するグルーの値を
	格納するレジスタです。
	ただし、この処理は\cs{autospacing}、\cs{noautospacing}によって
	行うか行わないかの指定ができます。\\
	このレジスタの値は、パラグラフの終わりまたは\cs{hbox}の最後の
	時点で取り込まれ、処理されます。
	したがって、同一パラグラフ内、あるいは\cs{hbox}内で何度か値を
	変化させたとしても、最後に設定された値によって全て処理されます。

	\noindent
	使用例\\
	{\tt\cs{kanjiskip}$=$10pt plus 1pt minus 1pt}\\
	{\kanjiskip=10pt plus 1pt minus 1pt
	とするとこうなります。\par}
\item \cs{{\bf xkanjiskip}}
	\begin{quote}
	\cs{xkanjiskip}$=$\angle{dimen}
	\end{quote}
	連続する2バイトコードと\cs{xspcode}で指定された1バイトコードの間に
	自動的に挿入するグルーの値を格納するレジスタです。
	ただし、この処理は\cs{autoxspacing}、\cs{noautoxspacing}によって
	行うか行わないかの指定ができます。\\
	\cs{kanjiskip}と同様のタイミングで取り込まれ処理されます。

	\noindent
	使用例\\
	{\tt\cs{xkanjiskip}$=$10pt plus 1pt minus 1pt}\\
	{\xkanjiskip=10pt plus 1pt minus 1pt
	とするとalphabetやnumberとの間に挿入されるグルーが
	10pt plus 1pt minus 1ptになります。\par}
\item \cs{{\bf sjis}}
	\begin{quote}
	\cs{sjis}\angle{16--bit number}
	\end{quote}
	シフトJISコードから内部コードへの変換を行います。

	\noindent
	使用例\\
	{\tt\cs{char}\cs{sjis}"889F}\\
	とすれば、`\char\sjis"889F' となります。
\item \cs{{\bf jis}}
	\begin{quote}
	\cs{jis}\angle{16--bit number}
	\end{quote}
	JISコードから内部コードへの変換を行います。

	\noindent
	使用例\\
	{\tt\cs{char}\cs{jis}"3022}\\
	とすれば、`\char\jis"3022' となります。
\item \cs{{\bf kuten}}
	\begin{quote}
	\cs{kuten}\angle{16--bit number}
	\end{quote}
	区点コードから内部コードへの変換を行います。
	16進4桁の上2桁が区を下2桁が点であると解釈します。

	\noindent
	使用例\\
	{\tt\cs{char}\cs{kuten}"1003}\\
	とすれば、`\char\kuten"1003' となります。
\item \cs{{\bf xspcode}}
	\begin{quote}
	\cs{xspcode}\angle{8--bit number}$=$\angle{$0|1|2|3$}
	\end{quote}
	2バイトコードと1バイトコードの間に自動的にスペースが挿入されますが、
	このプリミティブによってどの1バイトコードとの間でこの処理を行うか
	どうかを指定できます。この指定は0--3のいずれかを選択することにより
	次のような動作を選択できます。
	\begin{itemize}
	\item 0を設定することによりこの1バイトコード文字と2バイトコード文字
		の間での処理を禁止します。
	\item 1を設定することでこの文字と直前の2バイトコード文字との間にのみ
		スペースを挿入することを許可します。
	\item 2を設定することにより直後の2バイトコード文字との間にのみ
		スペースを挿入することを許可します。
	\item 3を設定することで前後の2バイトコード文字との間でのスペースの
		挿入を許可します。
	\end{itemize}

	\noindent
	使用例\\
	{\tt\cs{xspcode}`1=0\\
	\cs{xspcode}`2=3\\
	\cs{xspcode}`;=2}\\
	とすることにより、`1' に対する処理を禁止し、
	`2' に対して前後の処理を許可します。
	また、`;' に対しては直後へのスペースの挿入を許可します。
\item \cs{{\bf inhibitxspcode}}
	指定した2バイトコードとそれに前後する1バイトコードの間に
	自動的に挿入されるスペースを抑制します。
	この指定は0--3のいずれかを選択することにより
	次のような動作を選択できます。
	\begin{itemize}
	\item 0を設定することによりこの2バイトコード文字と1バイトコード文字
		の間での処理を禁止します。
	\item 1を設定することでこの文字と直前の1バイトコード文字との間に
		スペースを挿入することを禁止します。
	\item 2を設定することにより直後の1バイトコード文字との間に
		スペースを挿入することを禁止します。
	\item 3を設定することで前後の1バイトコード文字との間でのスペースの
		挿入を許可します。
	\end{itemize}
	このプリミティブは、日本語\TeX のバージョン1.4で追加しました。

	\noindent
	使用例\\
	{\tt\cs{inhibitxspcode}`?=0\\
	\cs{inhibitxspcode}`(=2\\
	\cs{inhibitxspcode}`)=1}
\item \cs{{\bf jcharwidowpenalty}}
	\begin{quote}
	\cs{jcharwidowpenalty}$=$\angle{number}
	\end{quote}
	パラグラフの最後の全角文字が、孤立して改行されるのを防ぐための
	ペナルティです。またパラグラフの最後が1文字以上の1区、2区、7区
	に含まれる文字の場合は、その直前にある全角文字に対してこの
	ペナルティが使われます。
\item \cs{{\bf autospacing}}
	\begin{quote}
	\cs{autospacing}
	\end{quote}
	このプリミティブによって2バイトコード間へのグルーの自動挿入を許可
	します。
\item \cs{{\bf noautospacing}}
	\begin{quote}
	\cs{noautospacing}
	\end{quote}
	このプリミティブによって2バイトコード間へのグルーの自動挿入を禁止
	します。
\item \cs{{\bf autoxspacing}}
	\begin{quote}
	\cs{autoxspacing}
	\end{quote}
	このプリミティブによって2バイトコードと1バイトコード間へのグルーの
	自動挿入を許可します。
\item \cs{{\bf noautoxspacing}}
	\begin{quote}
	\cs{noautoxspacing}
	\end{quote}
	このプリミティブによって2バイトコードと1バイトコード間へのグルーの
	自動挿入を禁止します。
\item \cs{{\bf prebreakpenalty}}
	\begin{quote}
	\cs{prebreakpenalty}\angle{16--bit number}$=$\angle{number}
	\end{quote}
	指定する文字の前方に挿入するペナルティ値を設定します。
	この指定は、行頭禁則の指定にあたります。

	\noindent
	使用例\\
	{\tt\cs{prebreakpenalty}`あ$=$1000}\\
	とすることにより、`あ' の前方に1000のペナルティ値が付けられます。
\item \cs{{\bf postbreakpenalty}}
	\begin{quote}
	\cs{postbreakpenalty}\angle{16--bit number}$=$\angle{number}
	\end{quote}
	指定する文字の後方に挿入するペナルティ値を設定します。
	この指定は、行末禁則の指定にあたります。\\
	ただし、\cs{prebreakpenalty}と\cs{postbreakpenalty}を同一の
	文字に対して同時に設定することはできません。
	同一の文字に対して両方の指定を行った場合、後からの設定が有効になります。
\item \cs{{\bf jfont}} \\
	基本的な動作は、``\cs{font}'' と同じです。
	ただし、`\cs{showthe}' などのプリミティブと組み合せた場合に
	カレントの漢字フォントを返します。
	フォントの定義は、\cs{font}, \cs{jfont}のどちらを使っても
	英字フォント、漢字フォントの定義が行えます。
\item \cs{{\bf jfam}} \\
	`\cs{jfont}' と同様に ``\cs{fam}'' とほとんど同じ動作をします。
	`\cs{showthe}' などとの組み合せることで、カレントの漢字フォント
	ファミリーを返します。
	ただしこのプリミティブでは、英字フォントのファミリーを
	漢字ファミリーとして定義できてしまいますので注意してください。
\item {{\bf zw, zh}} \\
	プリミティブとは異なりますが、{\bf em, ex}と同じように
	zw、zhは、それぞれカレントの2バイトコードフォントの幅(全角幅)
	および高さの単位記号です。

	\noindent
	使用例\\
	{\tt\cs{baselineskip}$=$1.5zh\\
	\cs{hsize}=20zw}\\
	でカレントフォントの高さの1.5倍の値が\cs{baselineskip}に代入され、
	\cs{hsize}が全角20文字分に設定されます。
\end{itemize}

\subsection{自動挿入されるスペースについて}
\cs{kanjiskip}や\cs{xkanjiskip}の自動挿入は、単純に連続する文字列ばかりでなく
shift\_amountが0である\cs{hbox}内の文字との前後関係においても行われます。
これは、\cs{hbox}が入子状になっている場合でも同じです。
つまり、その\cs{hbox}内に現れる最初の文字と最後の文字とが、
boxの前後の文字に連続する文字列として解釈され、
そこに指定されたスペースが挿入されることになります。
\cs{vbox}はその対象とはなりません。具体的には以下の例を参考にしてください。

\begin{tabular}{l@{ $\longrightarrow$ }c}
\verb|あ\hbox{A}い| & あ\hbox{A}い \\
\verb|あ\hbox{\hbox{A}}い| & あ\hbox{\hbox{A}}い \\
\verb|あ\hbox{\hbox{\hbox{A}}}い| & あ\hbox{\hbox{\hbox{A}}}い \\
\verb|あ\hbox{\hbox{\hbox{\hbox{A}}}}い| & あ\hbox{\hbox{\hbox{\hbox{A}}}}い \\
\verb|あ\hbox{A\hbox{\vbox{\hbox{A}}}B}い| &
					あ\hbox{A\hbox{\vbox{\hbox{A}}}B}い \\
\verb|あ\hbox{\hbox{\vbox{\hbox{A}}}B}い| &
					あ\hbox{\hbox{\vbox{\hbox{A}}}B}い \\
\verb|あ\hbox{A\hbox{\vbox{\hbox{A}}}}い| &
					あ\hbox{A\hbox{\vbox{\hbox{A}}}}い \\
\verb|あfiい| & あfiい \\
\verb|あ\hbox{\hbox{\vbox{\hbox{A}}}}い| &
					あ\hbox{\hbox{\vbox{\hbox{A}}}}い \\
\verb|あ\hbox{\vbox{\hbox{\hbox{A}}}}い| &
					あ\hbox{\vbox{\hbox{\hbox{A}}}}い \\
\verb|あ\vbox{\hbox{\hbox{\hbox{A}}}}い| &
					あ\vbox{\hbox{\hbox{\hbox{A}}}}い \\
\verb|A\hbox{あ}A| & A\hbox{あ}A \\
\verb|A\hbox{\hbox{あ}}A| & A\hbox{\hbox{あ}}A \\
\verb|A\hbox{\hbox{\hbox{あ}}}A| & A\hbox{\hbox{\hbox{あ}}}A \\
\verb|A\hbox{\hbox{\hbox{\hbox{あ}}}}A| & A\hbox{\hbox{\hbox{\hbox{あ}}}}A \\
\verb|A\hbox{あ\hbox{\vbox{\hbox{あ}}}あ}A| &
					A\hbox{あ\hbox{\vbox{\hbox{あ}}}あ}A \\
\verb|A\hbox{\hbox{\vbox{\hbox{あ}}}あ}A| &
					A\hbox{\hbox{\vbox{\hbox{あ}}}あ}A \\
\verb|A\hbox{あ\hbox{\vbox{\hbox{あ}}}}A| &
					A\hbox{あ\hbox{\vbox{\hbox{あ}}}}A \\
\verb|A\hbox{\hbox{\vbox{\hbox{あ}}}}A| &
					A\hbox{\hbox{\vbox{\hbox{あ}}}}A \\
\verb|A\hbox{\vbox{\hbox{\hbox{あ}}}}A| &
					A\hbox{\vbox{\hbox{\hbox{あ}}}}A \\
\verb|A\vbox{\hbox{\hbox{\hbox{あ}}}}A| &
					A\vbox{\hbox{\hbox{\hbox{あ}}}}A \\
\end{tabular}

\section{日本語化されていないプリミティブ}
次のプリミティブについては、日本語化(2バイトコードに対応)されていません。

\begin{itemize}
\item \cs{catcode}, \cs{sfcode}, \cs{mathcode}, \cs{delcode},
	\cs{mathchardef}は、
	2バイトコードに対してそれぞれの定義を行うことはできません。
\item \cs{lccode}, \cs{uccode}は、2バイトコードをそのままとおし、
	ローワケース、アッパーケースの変更は行いません。
\end{itemize}


\section{問題点}
現在のバージョンの日本語\TeX には、次のような問題点があります。
使用の際には注意してください。

\begin{itemize}
\item 禁則処理および1バイトコードと2バイトコード間に挿入するグルーは、
	その処理の対象となる文字を単に文字コードでのみ判断しています。
	これは、2バイトコードの文字に対しては全く問題ないのですが、
	1バイトコードの文字で問題が生じます。
	それは、\TeX の文字がコードと一対一に対応していないためで、
	フォントにより、特定のコードに割り付けられている文字が
	変ってしまいます。
	たとえば、`\{' はフォントcmsyの66$_{16}$に対応付けられています。
	66$_{16}$は、アスキーコードで `f' を示すわけですから禁則の対象文字
	とはなっていないはずです。

	したがって、今のままでは `\{' のような文字に対して自動的に禁則処理や
	グルーの自動挿入を行うことは不可能です。
\end{itemize}
%% cmsy の \{ は 0x6E でなく 0x66 (2001/08/24)
%% texjporg: タイポだけ修正 (2017/03/13, 2017/04/12, 2017/04/16)
\end{document}
