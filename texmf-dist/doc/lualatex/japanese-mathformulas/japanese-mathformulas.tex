\documentclass[fleqn]{ltjsarticle}% !lualatex

\usepackage{mathformulas,framed}%
\usepackage[hiragino-pron,deluxe,expert,bold]{luatexja-preset}%
\usepackage[usetype1]{uline--}%
\title{\LARGE\uline{japanese-mathformulas.sty}\Large\\manual pdf\\(mainly for Japanese, Lua\LaTeX)}%
\author{\Large Hugh / Ponkichi}%
\date{\today}
\def\texttt#1{{\gtfamily #1}}
\def\auto#1#2{\noindent\leftline{\uline{\textgt{#2}}}}

\makeatletter
\newlength{\@tempdimi}
\let\@@vspace@@\vspace
\def\vspace{\@ifstar{\@@vspace@}{\@vspace@}}
\def\@vspace@#1{
\setlength{\@tempdimi}{#1}\@@vspace@@{\@tempdimi}}
\def\@@vspace@#1{
\setlength{\@tempdimi}{#1}\@@vspace@@*{\@tempdimi}}

\newlength{\pseprule} % 段仕切り線の太さ
\setlength{\pseprule}{.5truept}
\newlength{\psep} % 段の間隔
\setlength{\psep}{30pt}
\newenvironment{multicolparx}[1]{%
  \begin{trivlist}\item[]\hspace{\parindent}%
  \multicolumnparallelparagraphs{#1}{\psep}%
}{%
  \endmulticolumnparallelparagraphs
  \end{trivlist}%
}
\newcount\columnsleft   \newcount\totalcolumns   \newdimen\separation
\def\multicolumnparallelparagraphs#1#2{%
  \hbadness5000 \vbadness9999 \tolerance9999
  \totalcolumns=#1  \let\xpar=\par  \separation=#2  
  \vskip\parskip
  \columnsleft=#1\relax
  \hbox to\hsize\bgroup
  \let\par\nextmulticolumnparallelparagraph
  \dimen0=#2 \advance\hsize-\columnsleft\dimen0 \advance\hsize\dimen0
  \divide\hsize\columnsleft\relax
  \vtop\bgroup
}
\def\nextmulticolumnparallelparagraph{%
  \@@vspace@@{\baselineskip}\egroup
  \advance\columnsleft-1
  \ifnum\columnsleft>0\relax
    \hfil\vrule\@width\the\pseprule\hspace{.45\separation}\vtop\bgroup 
  \else
  \egroup
    \xpar\vskip-2pt\xpar
    \multicolumnparallelparagraphs\totalcolumns\separation
  \fi
}
\def\endmulticolumnparallelparagraphs{%
  \egroup
  \advance\columnsleft-1
  \ifnum\columnsleft>0\relax
    \hfil\vtop\bgroup\hbox to \hsize{}
    \endmulticolumnparallelparagraphs
  \else
  \egroup
    \xpar
  \fi
}
\makeatother

\begin{document}

\maketitle

\begin{multicolparx}{2}
\noindent\parbox[t]{\hsize}{\begin{center}%
-機能紹介と注記-
\end{center}}%

\noindent\parbox[t]{\hsize}{\begin{center}%
- Function Introduction and Notes -
\end{center}}%

\noindent\parbox[t]{\hsize}{\begin{center}%
中学高校で習う数学の定理や公式を出力するためのstyファイル。\\
\detokenize{\NewDocumentCommand}によって,インデント数式か別行立て数式かを指定できる。
\end{center}}%

\noindent\parbox[t]{\hsize}{\begin{center}%
This is a style file for compiling basic math formulas.\\
\detokenize{\NewDocumentCommand} allows you to specify whether the formula should be used within a sentence or on a new line.
\end{center}}%

\noindent\parbox[t]{\hsize}{\begin{center}%
後の例では記述がないが,$\Ttyuubracket{\mathrm{i}}$か$\Ttyuubracket{\mathrm{b}}$かの指定をしない場合は自動的に$\Ttyuubracket{\mathrm{i}}$とみなされる。
\end{center}}%

\noindent\parbox[t]{\hsize}{\begin{center}%
Although not shown in the examples below, if $\Ttyuubracket{\mathrm{i}}$ or $\Ttyuubracket{\mathrm{b}}$ is not specified, it is automatically assumed to be $\Ttyuubracket{\mathrm{i}}$.
\end{center}}%

\noindent\parbox[t]{\hsize}{\begin{center}%
二段組の文書を作成するときは,数式の上下間スペースを減らすために,以下をpreambleに記述するとよい。
\end{center}}%

\noindent\parbox[t]{\hsize}{\begin{center}%
When making two-column document, you are recommended to put these lines at preamble.\\
These reduce the space above and below math expressions.
\end{center}}%
\end{multicolparx}

\begin{framed}
\begin{verbatim}
\AtBeginDocument{
  \abovedisplayskip     =0\abovedisplayskip
  \abovedisplayshortskip=0\abovedisplayshortskip
  \belowdisplayskip     =0\belowdisplayskip
  \belowdisplayshortskip=0\belowdisplayshortskip}
\end{verbatim}
\end{framed}

\begin{multicolparx}{2}

\noindent\parbox[t]{\hsize}{\begin{center}%
以下が実例。
\end{center}}%


\noindent\parbox[t]{\hsize}{\begin{center}%
Now, here are the actual examples!
\end{center}}%
\end{multicolparx}

\auto{1}{\detokenize{\二次式展開{公式A}[i]}}

\二次式展開{公式A}[i]

\auto{2}{\detokenize{\二次式展開{公式A}[b]}}

\二次式展開{公式A}[b]


\auto{33}{\detokenize{\二次式因数分解{公式A}[i]}}

\二次式因数分解{公式A}[i]

\auto{34}{\detokenize{\二次式因数分解{公式A}[b]}}

\二次式因数分解{公式A}[b]

%\begin{simplesquarebox}{二次式展開}
%\begin{description}
\auto{1}{\detokenize{\二次式展開{公式A}[i]}}

\二次式展開{公式A}[i]

\auto{2}{\detokenize{\二次式展開{公式A}[b]}}

\二次式展開{公式A}[b]

\auto{3}{\detokenize{\二次式展開{公式B}[i]}}

\二次式展開{公式B}[i]

\auto{4}{\detokenize{\二次式展開{公式B}[b]}}

\二次式展開{公式B}[b]

\auto{5}{\detokenize{\二次式展開{公式C}[i]}}

\二次式展開{公式C}[i]

\auto{6}{\detokenize{\二次式展開{公式C}[b]}}

\二次式展開{公式C}[b]

\auto{7}{\detokenize{\二次式展開{公式D}[i]}}

\二次式展開{公式D}[i]

\auto{8}{\detokenize{\二次式展開{公式D}[b]}}

\二次式展開{公式D}[b]


%\end{description}
%\end{simplesquarebox}

%\begin{simplesquarebox}{二次式因数分解}
%\begin{description}
\auto{9}{\detokenize{\二次式因数分解{公式A}[i]}}

\二次式因数分解{公式A}[i]

\auto{10}{\detokenize{\二次式因数分解{公式A}[b]}}

\二次式因数分解{公式A}[b]

\auto{11}{\detokenize{\二次式因数分解{公式B}[i]}}

\二次式因数分解{公式B}[i]

\auto{12}{\detokenize{\二次式因数分解{公式B}[b]}}

\二次式因数分解{公式B}[b]

\auto{13}{\detokenize{\二次式因数分解{公式C}[i]}}

\二次式因数分解{公式C}[i]

\auto{14}{\detokenize{\二次式因数分解{公式C}[b]}}

\二次式因数分解{公式C}[b]

\auto{15}{\detokenize{\二次式因数分解{公式D}[i]}}

\二次式因数分解{公式D}[i]

\auto{16}{\detokenize{\二次式因数分解{公式D}[b]}}

\二次式因数分解{公式D}[b]


%\end{description}
%\end{simplesquarebox}

%\begin{simplesquarebox}{平方根}
%\begin{description}
\auto{17}{\detokenize{\平方根{定義}[i]}}

\平方根{定義}[i]

\auto{18}{\detokenize{\平方根{定義}[b]}}


\平方根{定義}[b]

\auto{19}{\detokenize{\平方根{性質A}[i]}}

\平方根{性質A}[i]

\auto{20}{\detokenize{\平方根{性質A}[b]}}


\平方根{性質A}[b]

\auto{21}{\detokenize{\平方根{性質B}[i]}}

\平方根{性質B}[i]

\auto{22}{\detokenize{\平方根{性質B}[b]}}

\平方根{性質B}[b]

\auto{23}{\detokenize{\平方根{性質C}[i]}}

\平方根{性質C}[i]

\auto{24}{\detokenize{\平方根{性質C}[b]}}

\平方根{性質C}[b]

\auto{25}{\detokenize{\平方根{性質D}[i]}}

\平方根{性質D}[i]

\auto{26}{\detokenize{\平方根{性質D}[b]}}

\平方根{性質D}[b]

\auto{27}{\detokenize{\平方根{性質E}[i]}}

\平方根{性質E}[i]

\auto{28}{\detokenize{\平方根{性質E}[b]}}

\平方根{性質E}[b]


%\end{description}
%\end{simplesquarebox}

%\begin{simplesquarebox}{一次不等式}
%\begin{description}
\auto{29}{\detokenize{\一次不等式{性質A}[i]}}

\一次不等式{性質A}[i]

\auto{30}{\detokenize{\一次不等式{性質A}[b]}}


\一次不等式{性質A}[b]

\auto{31}{\detokenize{\一次不等式{性質B}[i]}}

\一次不等式{性質B}[i]

\auto{32}{\detokenize{\一次不等式{性質B}[b]}}


\一次不等式{性質B}[b]

\auto{33}{\detokenize{\一次不等式{性質C}[i]}}

\一次不等式{性質C}[i]

\auto{34}{\detokenize{\一次不等式{性質C}[b]}}


\一次不等式{性質C}[b]


%\end{description}
%\end{simplesquarebox}

%\begin{simplesquarebox}{集合}
%\begin{description}
\auto{35}{\detokenize{\集合{積集合}[i]}}

\集合{積集合}[i]

\auto{36}{\detokenize{\集合{積集合}[b]}}

\集合{積集合}[b]

\auto{37}{\detokenize{\集合{和集合}[i]}}

\集合{和集合}[i]

\auto{38}{\detokenize{\集合{和集合}[b]}}

\集合{和集合}[b]

\auto{39}{\detokenize{\集合{補集合}[i]}}

\集合{補集合}[i]

\auto{40}{\detokenize{\集合{補集合}[b]}}

\集合{補集合}[b]


%\end{description}
%\end{simplesquarebox}

%\begin{simplesquarebox}{対偶}
%\begin{description}
\auto{41}{\detokenize{\対偶{定理}[i]}}

\対偶{定理}[i]

\auto{41}{\detokenize{\対偶{定理}[b]}}


\対偶{定理}[b]


\auto{41}{\detokenize{\対偶{証明}}}

\対偶{証明}


%\end{description}
%\end{simplesquarebox}

%\begin{simplesquarebox}{背理法}
%\begin{description}
\auto{42}{\detokenize{\背理法}}

\背理法

%\end{description}
%\end{simplesquarebox}

%\begin{simplesquarebox}{二次関数}
%\begin{description}
\auto{43}{\detokenize{\二次関数{標準形}[i]}}

\二次関数{標準形}[i]

\auto{44}{\detokenize{\二次関数{標準形}[b]}}

\二次関数{標準形}[b]

\auto{45}{\detokenize{\二次関数{一般形}[i]}}

\二次関数{一般形}[i]

\auto{46}{\detokenize{\二次関数{一般形}[b]}}

\二次関数{一般形}[b]

\auto{47}{\detokenize{\二次関数{切片形}[i]}}

\二次関数{切片形}[i]

\auto{48}{\detokenize{\二次関数{切片形}[b]}}

\二次関数{切片形}[b]

\auto{49}{\detokenize{\二次関数{平方完成}[i]}}

\二次関数{平方完成}[i]

\auto{50}{\detokenize{\二次関数{平方完成}[b]}}


\二次関数{平方完成}[b]


%\end{description}
%\end{simplesquarebox}

%\begin{simplesquarebox}{二次方程式の解の公式}
%\begin{description}
\auto{51}{\detokenize{\二次方程式の解の公式{公式}[i]}}

\二次方程式の解の公式{公式}[i]

\auto{52}{\detokenize{\二次方程式の解の公式{公式}[b]}}


\二次方程式の解の公式{公式}[b]

\auto{52}{\detokenize{\二次方程式の解の公式{証明A}[i]}}

\二次方程式の解の公式{証明A}[i]

\auto{52}{\detokenize{\二次方程式の解の公式{証明B}[i]}}

\二次方程式の解の公式{証明B}[i]

%\end{description}
%\end{simplesquarebox}

\auto{52}{\detokenize{\三角比の定義{定義A}[i]}}

\三角比の定義{定義A}[i]

\auto{52}{\detokenize{\三角比の定義{定義B}[i]}}

\三角比の定義{定義B}[i]




%\begin{simplesquarebox}{三角比の相互関係}
%\begin{description}
\auto{53}{\detokenize{\三角比の相互関係{公式A}[i]}}

\三角比の相互関係{公式A}[i]

\auto{54}{\detokenize{\三角比の相互関係{公式A}[b]}}

\三角比の相互関係{公式A}[b]

\auto{55}{\detokenize{\三角比の相互関係{公式B}[i]}}

\三角比の相互関係{公式B}[i]

\auto{56}{\detokenize{\三角比の相互関係{公式B}[b]}}

\三角比の相互関係{公式B}[b]

\auto{57}{\detokenize{\三角比の相互関係{公式C}[i]}}

\三角比の相互関係{公式C}[i]

\auto{58}{\detokenize{\三角比の相互関係{公式C}[b]}}

\三角比の相互関係{公式C}[b]

\auto{57}{\detokenize{\三角比の相互関係{証明}}}

\三角比の相互関係{証明}

%\end{description}
%\end{simplesquarebox}

%\begin{simplesquarebox}{正弦定理}
%\begin{description}
\auto{59}{\detokenize{\正弦定理{公式}[i]}}

\正弦定理{公式}[i]

\auto{60}{\detokenize{\正弦定理{公式}[b]}}

\正弦定理{公式}[b]


\auto{59}{\detokenize{\正弦定理{証明}}}

\正弦定理{証明}

%\end{description}
%\end{simplesquarebox}

%\begin{simplesquarebox}{余弦定理}
%\begin{description}
\auto{61}{\detokenize{\余弦定理{公式}[i]}}

\余弦定理{公式}[i]

\auto{62}{\detokenize{\余弦定理{公式}[b]}}

\余弦定理{公式}[b]

\auto{61}{\detokenize{\余弦定理{証明}}}

\余弦定理{証明}

%\end{description}
%\end{simplesquarebox}

%\begin{simplesquarebox}{三角形の面積}
%\begin{description}
\auto{63}{\detokenize{\三角比の三角形の面積公式{公式}[i]}}

\三角比の三角形の面積公式{公式}[i]

\auto{64}{\detokenize{\三角比の三角形の面積公式{公式}[b]}}

\三角比の三角形の面積公式{公式}[b]

\auto{63}{\detokenize{\三角比の三角形の面積公式{証明}}}

\三角比の三角形の面積公式{証明}

\auto{63}{\detokenize{\ヘロンの公式{公式}[i]}}

\ヘロンの公式{公式}[i]

\auto{63}{\detokenize{\ヘロンの公式{公式}[b]}}

\ヘロンの公式{公式}[b]

\auto{63}{\detokenize{\ヘロンの公式{証明}}}

\ヘロンの公式{証明}

\auto{63}{\detokenize{\外接円の半径と三角形の面積{公式}[i]}}

\外接円の半径と三角形の面積{公式}[i]

\auto{63}{\detokenize{\外接円の半径と三角形の面積{公式}[b]}}

\外接円の半径と三角形の面積{公式}[b]

\auto{63}{\detokenize{\外接円の半径と三角形の面積{証明}}}

\外接円の半径と三角形の面積{証明}

\auto{64}{\detokenize{\三角形の面積公式}}

\三角形の面積公式

%\end{description}
%\end{simplesquarebox}

%\begin{simplesquarebox}{場合の数と確率}
%\begin{description}
\auto{65}{\detokenize{\場合の数と確率{和集合の要素の個数}[i]}}

\場合の数と確率{和集合の要素の個数}[i]

\auto{66}{\detokenize{\場合の数と確率{和集合の要素の個数}[b]}}

\場合の数と確率{和集合の要素の個数}[b]

%\auto{67}{\detokenize{\場合の数と確率{積集合の要素の個数}[i]}}

%\場合の数と確率{積集合の要素の個数}[i]

%\auto{68}{\detokenize{\場合の数と確率{積集合の要素の個数}[b]}}

%\場合の数と確率{積集合の要素の個数}[b]

\auto{69}{\detokenize{\場合の数と確率{補集合の要素の個数}[i]}}

\場合の数と確率{補集合の要素の個数}[i]

\auto{70}{\detokenize{\場合の数と確率{補集合の要素の個数}[b]}}

\場合の数と確率{補集合の要素の個数}[b]

\auto{71}{\detokenize{\場合の数と確率{和の法則}[i]}}

\場合の数と確率{和の法則}[i]

\auto{72}{\detokenize{\場合の数と確率{和の法則}[b]}}

\場合の数と確率{和の法則}[b]

\auto{73}{\detokenize{\場合の数と確率{積の法則}[i]}}

\場合の数と確率{積の法則}[i]

\auto{74}{\detokenize{\場合の数と確率{積の法則}[b]}}

\場合の数と確率{積の法則}[b]

\auto{75}{\detokenize{\場合の数と確率{順列}[i]}}

\場合の数と確率{順列}[i]

\auto{76}{\detokenize{\場合の数と確率{順列}[b]}}

\場合の数と確率{順列}[b]

\auto{75}{\detokenize{\場合の数と確率{順列の証明}}}

\場合の数と確率{順列の証明}

\auto{77}{\detokenize{\場合の数と確率{円順列}[i]}}

\場合の数と確率{円順列}[i]

\auto{78}{\detokenize{\場合の数と確率{円順列}[b]}}

\場合の数と確率{円順列}[b]

\auto{77}{\detokenize{\場合の数と確率{円順列の証明}}}

\場合の数と確率{円順列の証明}

\auto{79}{\detokenize{\場合の数と確率{重複順列}[i]}}

\場合の数と確率{重複順列}[i]

\auto{80}{\detokenize{\場合の数と確率{重複順列}[b]}}

\場合の数と確率{重複順列}[b]

\auto{81}{\detokenize{\場合の数と確率{組み合わせ}[i]}}

\場合の数と確率{組み合わせ}[i]

\auto{82}{\detokenize{\場合の数と確率{組み合わせ}[b]}}

\場合の数と確率{組み合わせ}[b]

\auto{81}{\detokenize{\場合の数と確率{組み合わせの証明}}}

\場合の数と確率{組み合わせの証明}

\auto{83}{\detokenize{\場合の数と確率{同じものを含む順列}[i]}}

\場合の数と確率{同じものを含む順列}[i]

\auto{84}{\detokenize{\場合の数と確率{同じものを含む順列}[b]}}

\場合の数と確率{同じものを含む順列}[b]

\auto{83}{\detokenize{\場合の数と確率{同じものを含む順列の証明}}}

\場合の数と確率{同じものを含む順列の証明}

\auto{85}{\detokenize{\場合の数と確率{確率の定義}[i]}}

\場合の数と確率{確率の定義}[i]

\auto{86}{\detokenize{\場合の数と確率{確率の定義}[b]}}

\場合の数と確率{確率の定義}[b]

\auto{87}{\detokenize{\場合の数と確率{排反の定義}[i]}}

\場合の数と確率{排反の定義}[i]

\auto{88}{\detokenize{\場合の数と確率{排反の定義}[b]}}

\場合の数と確率{排反の定義}[b]


\auto{89}{\detokenize{\場合の数と確率{確率の性質A}[i]}}

\場合の数と確率{確率の性質A}[i]

\auto{90}{\detokenize{\場合の数と確率{確率の性質A}[b]}}

\場合の数と確率{確率の性質A}[b]

\auto{91}{\detokenize{\場合の数と確率{確率の性質B}[i]}}

\場合の数と確率{確率の性質B}[i]

\auto{92}{\detokenize{\場合の数と確率{確率の性質B}[b]}}

\場合の数と確率{確率の性質B}[b]

\auto{93}{\detokenize{\場合の数と確率{和事象の確率}[i]}}

\場合の数と確率{和事象の確率}[i]

\auto{94}{\detokenize{\場合の数と確率{和事象の確率}[b]}}

\場合の数と確率{和事象の確率}[b]

\auto{95}{\detokenize{\場合の数と確率{積事象の確率}[i]}}

%\場合の数と確率{積事象の確率}[i]

%\auto{96}{\detokenize{\場合の数と確率{積事象の確率}[b]}}

%\場合の数と確率{積事象の確率}[b]

%\auto{97}{\detokenize{\場合の数と確率{余事象の確率}[i]}}

\場合の数と確率{余事象の確率}[i]

\auto{98}{\detokenize{\場合の数と確率{余事象の確率}[b]}}

\場合の数と確率{余事象の確率}[b]

\auto{99}{\detokenize{\場合の数と確率{独立な事象の確率}[i]}}

\場合の数と確率{独立な事象の確率}[i]

\auto{100}{\detokenize{\場合の数と確率{独立な事象の確率}[b]}}

\場合の数と確率{独立な事象の確率}[b]

\auto{101}{\detokenize{\場合の数と確率{反復試行の確率}[i]}}

\場合の数と確率{反復試行の確率}[i]

\auto{102}{\detokenize{\場合の数と確率{反復試行の確率}[b]}}

\場合の数と確率{反復試行の確率}[b]

\auto{101}{\detokenize{\場合の数と確率{反復試行の確率の証明}}}

\場合の数と確率{反復試行の確率の証明}

\auto{103}{\detokenize{\場合の数と確率{条件付き確率}[i]}}

\場合の数と確率{条件付き確率}[i]

\auto{104}{\detokenize{\場合の数と確率{条件付き確率}[b]}}

\場合の数と確率{条件付き確率}[b]


%\end{description}
%\end{simplesquarebox}

%\begin{simplesquarebox}{図形の性質}
%\begin{description}

\auto{105}{\detokenize{\平行線と線分比の性質{公式A}}}

\平行線と線分比の性質{公式A}

\auto{105}{\detokenize{\平行線と線分比の性質{公式B}}}

\平行線と線分比の性質{公式B}

\auto{105}{\detokenize{\平行線と線分比の性質{証明}}}

\平行線と線分比の性質{証明}

\auto{105}{\detokenize{\図形の性質{内心}}}

\図形の性質{内心}

\auto{106}{\detokenize{\図形の性質{外心}}}

\図形の性質{外心}

\auto{107}{\detokenize{\図形の性質{垂心}}}

\図形の性質{垂心}

\auto{108}{\detokenize{\図形の性質{重心}}}

\図形の性質{重心}

\auto{109}{\detokenize{\図形の性質{傍心}}}

\図形の性質{傍心}

\auto{110}{\detokenize{\図形の性質{チェバの定理}}}

\図形の性質{チェバの定理}

\auto{110}{\detokenize{\図形の性質{チェバの定理の証明}}}

\図形の性質{チェバの定理の証明}

\auto{111}{\detokenize{\図形の性質{メネラウスの定理}}}

\図形の性質{メネラウスの定理}

\auto{111}{\detokenize{\図形の性質{メネラウスの定理の証明}}}

\図形の性質{メネラウスの定理の証明}
\auto{112}{\detokenize{\図形の性質{円周角の定理}}}

\図形の性質{円周角の定理}

\auto{112}{\detokenize{\図形の性質{円周角の定理の証明}}}

\図形の性質{円周角の定理の証明}

\auto{113}{\detokenize{\図形の性質{内接四角形の定理}}}

\図形の性質{内接四角形の定理}

\auto{113}{\detokenize{\図形の性質{内接四角形の定理の証明}}}

\図形の性質{内接四角形の定理の証明}

\auto{114}{\detokenize{\図形の性質{接弦定理}}}

\図形の性質{接弦定理}

\auto{114}{\detokenize{\図形の性質{接弦定理の証明}}}

\図形の性質{接弦定理の証明}

\auto{115}{\detokenize{\図形の性質{内角と外角の二等分線}}}

\図形の性質{内角と外角の二等分線}

\auto{116}{\detokenize{\図形の性質{方べきの定理A}}}

\図形の性質{方べきの定理A}

\auto{116}{\detokenize{\図形の性質{方べきの定理Aの証明}}}

\図形の性質{方べきの定理Aの証明}
\auto{117}{\detokenize{\図形の性質{方べきの定理B}}}

\図形の性質{方べきの定理B}

\auto{117}{\detokenize{\図形の性質{方べきの定理Bの証明}}}

\図形の性質{方べきの定理Bの証明}
\auto{118}{\detokenize{\図形の性質{方べきの定理C}}}

\図形の性質{方べきの定理C}

\auto{118}{\detokenize{\図形の性質{方べきの定理Cの証明}}}

\図形の性質{方べきの定理Cの証明}
%\end{description}
%\end{simplesquarebox}

%%%%%%%%%%%%%%%%%%%%ここから数\UTF{2161}B%%%%%%%%%%%%%%%%%%%%
%n-118=個数
%\begin{simplesquarebox}{展開}
%\begin{description}
\auto{119}{\detokenize{\三次式展開{公式A}[i]}}

\三次式展開{公式A}[i]

\auto{120}{\detokenize{\三次式展開{公式A}[b]}}

\三次式展開{公式A}[b]

\auto{121}{\detokenize{\三次式展開{公式B}[i]}}

\三次式展開{公式B}[i]

\auto{122}{\detokenize{\三次式展開{公式B}[b]}}

\三次式展開{公式B}[b]

\auto{123}{\detokenize{\三次式展開{公式C}[i]}}

\三次式展開{公式C}[i]

\auto{124}{\detokenize{\三次式展開{公式C}[b]}}

\三次式展開{公式C}[b]

\auto{125}{\detokenize{\三次式展開{公式D}[i]}}

\三次式展開{公式D}[i]

\auto{126}{\detokenize{\三次式展開{公式D}[b]}}

\三次式展開{公式D}[b]


%\end{description}
%\end{simplesquarebox}
       
%\begin{simplesquarebox}{因数分解}
%\begin{description}
\auto{127}{\detokenize{\三次式因数分解{公式A}[i]}}

\三次式因数分解{公式A}[i]

\auto{128}{\detokenize{\三次式因数分解{公式A}[b]}}

\三次式因数分解{公式A}[b]

\auto{129}{\detokenize{\三次式因数分解{公式B}[i]}}

\三次式因数分解{公式B}[i]

\auto{130}{\detokenize{\三次式因数分解{公式B}[b]}}

\三次式因数分解{公式B}[b]

\auto{131}{\detokenize{\三次式因数分解{公式C}[i]}}

\三次式因数分解{公式C}[i]

\auto{132}{\detokenize{\三次式因数分解{公式C}[b]}}

\三次式因数分解{公式C}[b]

\auto{133}{\detokenize{\三次式因数分解{公式D}[i]}}

\三次式因数分解{公式D}[i]

\auto{134}{\detokenize{\三次式因数分解{公式D}[b]}}

\三次式因数分解{公式D}[b]


%\end{description}
%\end{simplesquarebox}
       
%\begin{simplesquarebox}{二項定理}
%\begin{description}
\auto{135}{\detokenize{\二項定理{公式}[i]}}

\二項定理{公式}[i]

\auto{136}{\detokenize{\二項定理{公式}[b]}}

\二項定理{公式}[b]

\auto{137}{\detokenize{\二項定理{一般項}[i]}}

\二項定理{一般項}[i]

\auto{138}{\detokenize{\二項定理{一般項}[b]}}

\二項定理{一般項}[b]

\auto{135}{\detokenize{\二項定理{証明}}}

\二項定理{証明}

%\end{description}
%\end{simplesquarebox}
       
%\begin{simplesquarebox}{分数式}
%\begin{description}
\auto{139}{\detokenize{\分数式{公式A}[i]}}

\分数式{公式A}[i]

\auto{140}{\detokenize{\分数式{公式A}[b]}}

\分数式{公式A}[b]

\auto{141}{\detokenize{\分数式{公式B}[i]}}

\分数式{公式B}[i]

\auto{142}{\detokenize{\分数式{公式B}[b]}}

\分数式{公式B}[b]

\auto{143}{\detokenize{\分数式{公式C}[i]}}

\分数式{公式C}[i]

\auto{144}{\detokenize{\分数式{公式C}[b]}}

\分数式{公式C}[b]

\auto{145}{\detokenize{\分数式{公式D}[i]}}

\分数式{公式D}[i]

\auto{146}{\detokenize{\分数式{公式D}[b]}}

\分数式{公式D}[b]


%\end{description}
%\end{simplesquarebox}
       
%\begin{simplesquarebox}{相加相乗平均}
%\begin{description}
\auto{147}{\detokenize{\相加相乗平均{公式}[i]}}

\相加相乗平均{公式}[i]

\auto{148}{\detokenize{\相加相乗平均{公式}[b]}}

\相加相乗平均{公式}[b]

\auto{147}{\detokenize{\相加相乗平均{証明}}}

\相加相乗平均{証明}

%\end{description}
%\end{simplesquarebox}
       
%\begin{simplesquarebox}{虚数の定義}
%\begin{description}
\auto{149}{\detokenize{\虚数の定義{定義}[i]}}

\虚数の定義{定義}[i]

\auto{150}{\detokenize{\虚数の定義{定義}[b]}}

\虚数の定義{定義}[b]


%\end{description}
%\end{simplesquarebox}
       
%\begin{simplesquarebox}{複素数の定義}
%\begin{description}
\auto{151}{\detokenize{\複素数の定義{定義}[i]}}

\複素数の定義{定義}[i]

\auto{152}{\detokenize{\複素数の定義{定義}[b]}}


\複素数の定義{定義}[b]


%\end{description}
%\end{simplesquarebox}
       
%\begin{simplesquarebox}{二次方程式の解の判別}
%\begin{description}
\auto{153}{\detokenize{\二次方程式の解の判別}}

\二次方程式の解の判別

%\end{description}
%\end{simplesquarebox}
       
%\begin{simplesquarebox}{解と係数の関係}
%\begin{description}
\auto{154}{\detokenize{\解と係数の関係{二次方程式の解と係数の関係A}[i]}}

\解と係数の関係{二次方程式の解と係数の関係A}[i]

\auto{155}{\detokenize{\解と係数の関係{二次方程式の解と係数の関係A}[b]}}

\解と係数の関係{二次方程式の解と係数の関係A}[b]

\auto{156}{\detokenize{\解と係数の関係{二次方程式の解と係数の関係B}[i]}}

\解と係数の関係{二次方程式の解と係数の関係B}[i]

\auto{157}{\detokenize{\解と係数の関係{二次方程式の解と係数の関係B}[b]}}

\解と係数の関係{二次方程式の解と係数の関係B}[b]

\auto{154}{\detokenize{\解と係数の関係{二次方程式の解と係数の関係の証明}}}

\解と係数の関係{二次方程式の解と係数の関係の証明}

\auto{158}{\detokenize{\解と係数の関係{三次方程式の解と係数の関係A}[i]}}

\解と係数の関係{三次方程式の解と係数の関係A}[i]

\auto{159}{\detokenize{\解と係数の関係{三次方程式の解と係数の関係A}[b]}}

\解と係数の関係{三次方程式の解と係数の関係A}[b]

\auto{160}{\detokenize{\解と係数の関係{三次方程式の解と係数の関係B}[i]}}

\解と係数の関係{三次方程式の解と係数の関係B}[i]

\auto{161}{\detokenize{\解と係数の関係{三次方程式の解と係数の関係B}[b]}}

\解と係数の関係{三次方程式の解と係数の関係B}[b]

\auto{162}{\detokenize{\解と係数の関係{三次方程式の解と係数の関係C}[i]}}

\解と係数の関係{三次方程式の解と係数の関係C}[i]

\auto{163}{\detokenize{\解と係数の関係{三次方程式の解と係数の関係C}[b]}}

\解と係数の関係{三次方程式の解と係数の関係C}[b]

\auto{158}{\detokenize{\解と係数の関係{三次方程式の解と係数の関係の証明}}}

\解と係数の関係{三次方程式の解と係数の関係の証明}

%\end{description}
%\end{simplesquarebox}
       
%\begin{simplesquarebox}{剰余定理}
%\begin{description}
\auto{164}{\detokenize{\剰余定理{定理A}[i]}}

\剰余定理{定理A}[i]

\auto{165}{\detokenize{\剰余定理{定理A}[b]}}

\剰余定理{定理A}[b]

\auto{166}{\detokenize{\剰余定理{定理B}[i]}}

\剰余定理{定理B}[i]

\auto{167}{\detokenize{\剰余定理{定理B}[b]}}

\剰余定理{定理B}[b]

\auto{164}{\detokenize{\剰余定理{証明}}}

\剰余定理{証明}

%\end{description}
%\end{simplesquarebox}
       
%\begin{simplesquarebox}{因数定理}
%\begin{description}
\auto{168}{\detokenize{\因数定理{定理}[i]}}

\因数定理{定理}[i]

\auto{168}{\detokenize{\因数定理{定理}[b]}}

\因数定理{定理}[b]

\auto{168}{\detokenize{\因数定理{証明}}}

\因数定理{証明}

\auto{168}{\detokenize{\ユークリッド幾何の公理{公理A}}}

\ユークリッド幾何の公理{公理A}

\auto{168}{\detokenize{\ユークリッド幾何の公理{公理B}}}

\ユークリッド幾何の公理{公理B}

\auto{168}{\detokenize{\直線}}

\直線

\auto{168}{\detokenize{\線分}}

\線分

\auto{168}{\detokenize{\半直線}}

\半直線

\auto{168}{\detokenize{\距離}}

\距離

\auto{168}{\detokenize{\円}}

\円

\auto{168}{\detokenize{\弧}}

\弧

\auto{168}{\detokenize{\弦}}

\弦

\auto{168}{\detokenize{\中心角}}

\中心角

\auto{168}{\detokenize{\対頂角{定義}}}

\対頂角{定義}

\auto{168}{\detokenize{\対頂角{性質}}}

\対頂角{性質}

\auto{168}{\detokenize{\対頂角{証明}}}

\対頂角{証明}

\auto{168}{\detokenize{\錯角{定義}}}

\錯角{定義}

\auto{168}{\detokenize{\錯角{性質}}}

\錯角{性質}

\auto{168}{\detokenize{\錯角{証明}}}

\錯角{証明}

\auto{168}{\detokenize{\同位角{定義}}}

\同位角{定義}

\auto{168}{\detokenize{\同位角{公理}}}

\同位角{公理}

%\end{description}
%\end{simplesquarebox}
       
%\begin{simplesquarebox}{点の座標}
%\begin{description}
\auto{169}{\detokenize{\点の座標{二点間の距離}[i]}}

\点の座標{二点間の距離}[i]

\auto{170}{\detokenize{\点の座標{二点間の距離}[b]}}

\点の座標{二点間の距離}[b]

\auto{171}{\detokenize{\点の座標{内分点の座標}[i]}}

\点の座標{内分点の座標}[i]

\auto{172}{\detokenize{\点の座標{内分点の座標}[b]}}

\点の座標{内分点の座標}[b]

\auto{171}{\detokenize{\点の座標{内分点の座標の証明}}}

\点の座標{内分点の座標の証明}

\auto{173}{\detokenize{\点の座標{外分点の座標}[i]}}

\点の座標{外分点の座標}[i]

\auto{174}{\detokenize{\点の座標{外分点の座標}[b]}}

\点の座標{外分点の座標}[b]

\auto{173}{\detokenize{\点の座標{外分点の座標の証明}}}

\点の座標{外分点の座標の証明}

\auto{175}{\detokenize{\点の座標{中点の座標}[i]}}

\点の座標{中点の座標}[i]

\auto{176}{\detokenize{\点の座標{中点の座標}[b]}}

\点の座標{中点の座標}[b]

\auto{175}{\detokenize{\点の座標{中点の座標の証明}}}

\点の座標{中点の座標の証明}

\auto{177}{\detokenize{\点の座標{重心の座標}[i]}}

\点の座標{重心の座標}[i]

\auto{178}{\detokenize{\点の座標{重心の座標}[b]}}

\点の座標{重心の座標}[b]

\auto{177}{\detokenize{\点の座標{重心の座標の証明}}}

\点の座標{重心の座標の証明}

%\end{description}
%\end{simplesquarebox}
       
%\begin{simplesquarebox}{直線の方程式}
%\begin{description}
\auto{179}{\detokenize{\直線の方程式{公式A}[i]}}

\直線の方程式{公式A}[i]

\auto{180}{\detokenize{\直線の方程式{公式A}[b]}}

\直線の方程式{公式A}[b]

\auto{181}{\detokenize{\直線の方程式{公式B}[i]}}

\直線の方程式{公式B}[i]

\auto{182}{\detokenize{\直線の方程式{公式B}[b]}}

\直線の方程式{公式B}[b]

\auto{183}{\detokenize{\直線の方程式{公式C}[i]}}

\直線の方程式{公式C}[i]

\auto{184}{\detokenize{\直線の方程式{公式C}[b]}}

\直線の方程式{公式C}[b]

\auto{183}{\detokenize{\直線の方程式{公式Bの証明}}}

\直線の方程式{公式Bの証明}

%\end{description}
%\end{simplesquarebox}
       
%\begin{simplesquarebox}{二直線の関係}
%\begin{description}
\auto{185}{\detokenize{\二直線の関係{公式A}[i]}}

\二直線の関係{公式A}[i]

\auto{186}{\detokenize{\二直線の関係{公式A}[b]}}

\二直線の関係{公式A}[b]

\auto{187}{\detokenize{\二直線の関係{公式B}[i]}}

\二直線の関係{公式B}[i]

\auto{188}{\detokenize{\二直線の関係{公式B}[b]}}

\二直線の関係{公式B}[b]

\auto{185}{\detokenize{\二直線の関係{公式Bの証明}}}

\二直線の関係{公式Bの証明}

%\end{description}
%\end{simplesquarebox}
       
%\begin{simplesquarebox}{点と直線の距離}
%\begin{description}
\auto{189}{\detokenize{\点と直線の距離{公式}[i]}}

\点と直線の距離{公式}[i]

\auto{190}{\detokenize{\点と直線の距離{公式}[b]}}

\点と直線の距離{公式}[b]

\auto{189}{\detokenize{\点と直線の距離{証明}}}

\点と直線の距離{証明}

%\end{description}
%\end{simplesquarebox}
       
%\begin{simplesquarebox}{円の方程式}
%\begin{description}
\auto{191}{\detokenize{\円の方程式{公式}[i]}}

\円の方程式{公式}[i]

\auto{192}{\detokenize{\円の方程式{公式}[b]}}

\円の方程式{公式}[b]

\auto{191}{\detokenize{\円の方程式{証明}}}

\円の方程式{証明}

%\end{description}
%\end{simplesquarebox}
       
%\begin{simplesquarebox}{円と直線}
%\begin{description}
\auto{193}{\detokenize{\円と直線{公式}[i]}}

\円と直線{公式}[i]

\auto{194}{\detokenize{\円と直線{公式}[b]}}

\円と直線{公式}[b]

\auto{193}{\detokenize{\円と直線{証明}}}

\円と直線{証明}

%\end{description}
%\end{simplesquarebox}
       
%\begin{simplesquarebox}{三角関数の相互関係}
%\begin{description}
\auto{195}{\detokenize{\三角関数の相互関係{公式A}[i]}}

\三角関数の相互関係{公式A}[i]

\auto{196}{\detokenize{\三角関数の相互関係{公式A}[b]}}

\三角関数の相互関係{公式A}[b]

\auto{197}{\detokenize{\三角関数の相互関係{公式B}[i]}}

\三角関数の相互関係{公式B}[i]

\auto{198}{\detokenize{\三角関数の相互関係{公式B}[b]}}

\三角関数の相互関係{公式B}[b]

\auto{199}{\detokenize{\三角関数の相互関係{公式C}[i]}}

\三角関数の相互関係{公式C}[i]

\auto{200}{\detokenize{\三角関数の相互関係{公式C}[b]}}

\三角関数の相互関係{公式C}[b]

\auto{195}{\detokenize{\三角関数の相互関係{証明}}}

\三角関数の相互関係{証明}

%\end{description}
%\end{simplesquarebox}

\auto{201}{\detokenize{\三角関数の性質{性質A}[i]}}

\三角関数の性質{性質A}[i]

\auto{202}{\detokenize{\三角関数の性質{性質A}[b]}}

\三角関数の性質{性質A}[b]

\auto{203}{\detokenize{\三角関数の性質{性質B}[i]}}

\三角関数の性質{性質B}[i]

\auto{204}{\detokenize{\三角関数の性質{性質B}[b]}}

\三角関数の性質{性質B}[b]

\auto{205}{\detokenize{\三角関数の性質{性質C}[i]}}

\三角関数の性質{性質C}[i]

\auto{206}{\detokenize{\三角関数の性質{性質C}[b]}}

\三角関数の性質{性質C}[b]

\auto{207}{\detokenize{\三角関数の性質{性質D}[i]}}

\三角関数の性質{性質D}[i]

\auto{208}{\detokenize{\三角関数の性質{性質D}[b]}}

\三角関数の性質{性質D}[b]

\auto{209}{\detokenize{\三角関数の性質{性質E}[i]}}

\三角関数の性質{性質E}[i]

\auto{210}{\detokenize{\三角関数の性質{性質E}[b]}}

\三角関数の性質{性質E}[b]

\auto{211}{\detokenize{\三角関数の性質{性質F}[i]}}

\三角関数の性質{性質F}[i]

\auto{212}{\detokenize{\三角関数の性質{性質F}[b]}}

\三角関数の性質{性質F}[b]

\auto{213}{\detokenize{\三角関数の性質{性質G}[i]}}

\三角関数の性質{性質G}[i]

\auto{214}{\detokenize{\三角関数の性質{性質G}[b]}}

\三角関数の性質{性質G}[b]

\auto{215}{\detokenize{\三角関数の性質{性質H}[i]}}

\三角関数の性質{性質H}[i]

\auto{216}{\detokenize{\三角関数の性質{性質H}[b]}}

\三角関数の性質{性質H}[b]

\auto{217}{\detokenize{\三角関数の性質{性質I}[i]}}

\三角関数の性質{性質I}[i]

\auto{218}{\detokenize{\三角関数の性質{性質I}[b]}}

\三角関数の性質{性質I}[b]

\auto{219}{\detokenize{\三角関数の性質{性質J}[i]}}

\三角関数の性質{性質J}[i]

\auto{220}{\detokenize{\三角関数の性質{性質J}[b]}}

\三角関数の性質{性質J}[b]

\auto{221}{\detokenize{\三角関数の性質{性質K}[i]}}

\三角関数の性質{性質K}[i]

\auto{222}{\detokenize{\三角関数の性質{性質K}[b]}}

\三角関数の性質{性質K}[b]

\auto{223}{\detokenize{\三角関数の性質{性質L}[i]}}

\三角関数の性質{性質L}[i]

\auto{224}{\detokenize{\三角関数の性質{性質L}[b]}}

\三角関数の性質{性質L}[b]

\auto{225}{\detokenize{\三角関数の性質{性質M}[i]}}

\三角関数の性質{性質M}[i]

\auto{226}{\detokenize{\三角関数の性質{性質M}[b]}}

\三角関数の性質{性質M}[b]

\auto{227}{\detokenize{\三角関数の性質{性質N}[i]}}

\三角関数の性質{性質N}[i]

\auto{228}{\detokenize{\三角関数の性質{性質N}[b]}}

\三角関数の性質{性質N}[b]

\auto{229}{\detokenize{\三角関数の性質{性質O}[i]}}

\三角関数の性質{性質O}[i]

\auto{230}{\detokenize{\三角関数の性質{性質O}[b]}}

\三角関数の性質{性質O}[b]



%\begin{simplesquarebox}{三角関数の加法定理}
%\begin{description}
\auto{231}{\detokenize{\三角関数の加法定理{公式A}[i]}}

\三角関数の加法定理{公式A}[i]

\auto{232}{\detokenize{\三角関数の加法定理{公式A}[b]}}

\三角関数の加法定理{公式A}[b]

\auto{233}{\detokenize{\三角関数の加法定理{公式B}[i]}}

\三角関数の加法定理{公式B}[i]

\auto{234}{\detokenize{\三角関数の加法定理{公式B}[b]}}

\三角関数の加法定理{公式B}[b]

\auto{235}{\detokenize{\三角関数の加法定理{公式C}[i]}}

\三角関数の加法定理{公式C}[i]

\auto{236}{\detokenize{\三角関数の加法定理{公式C}[b]}}

\三角関数の加法定理{公式C}[b]

\auto{231}{\detokenize{\三角関数の加法定理{証明}}}

\三角関数の加法定理{証明}

%\end{description}
%\end{simplesquarebox}
       
\auto{237}{\detokenize{\三角関数の二倍角の公式{公式A}[i]}}

\三角関数の二倍角の公式{公式A}[i]

\auto{238}{\detokenize{\三角関数の二倍角の公式{公式A}[b]}}

\三角関数の二倍角の公式{公式A}[b]

\auto{239}{\detokenize{\三角関数の二倍角の公式{公式B}[i]}}

\三角関数の二倍角の公式{公式B}[i]

\auto{240}{\detokenize{\三角関数の二倍角の公式{公式B}[b]}}

\三角関数の二倍角の公式{公式B}[b]

\auto{241}{\detokenize{\三角関数の二倍角の公式{公式C}[i]}}

\三角関数の二倍角の公式{公式C}[i]

\auto{242}{\detokenize{\三角関数の二倍角の公式{公式C}[b]}}

\三角関数の二倍角の公式{公式C}[b]

\auto{243}{\detokenize{\三角関数の二倍角の公式{公式D}[i]}}

\三角関数の二倍角の公式{公式D}[i]

\auto{244}{\detokenize{\三角関数の二倍角の公式{公式D}[b]}}

\三角関数の二倍角の公式{公式D}[b]

\auto{245}{\detokenize{\三角関数の二倍角の公式{公式E}[i]}}

\三角関数の二倍角の公式{公式E}[i]

\auto{246}{\detokenize{\三角関数の二倍角の公式{公式E}[b]}}

\三角関数の二倍角の公式{公式E}[b]

\auto{237}{\detokenize{\三角関数の二倍角の公式{証明}}}

\三角関数の二倍角の公式{証明}

\auto{247}{\detokenize{\三角関数の三倍角の公式{公式A}[i]}}

\三角関数の三倍角の公式{公式A}[i]

\auto{248}{\detokenize{\三角関数の三倍角の公式{公式A}[b]}}

\三角関数の三倍角の公式{公式A}[b]

\auto{249}{\detokenize{\三角関数の三倍角の公式{公式B}[i]}}

\三角関数の三倍角の公式{公式B}[i]

\auto{250}{\detokenize{\三角関数の三倍角の公式{公式B}[b]}}

\三角関数の三倍角の公式{公式B}[b]

%\auto{251}{\detokenize{\三角関数の三倍角の公式{公式C}[i]}}

%\三角関数の三倍角の公式{公式C}[i]

%\auto{251}{\detokenize{\三角関数の三倍角の公式{公式C}[b]}}

%\三角関数の三倍角の公式{公式C}[b]

\auto{247}{\detokenize{\三角関数の三倍角の公式{証明}}}

\三角関数の三倍角の公式{証明}


\auto{252}{\detokenize{\三角関数の積和公式{公式A}[i]}}

\三角関数の積和公式{公式A}[i]

\auto{253}{\detokenize{\三角関数の積和公式{公式A}[b]}}

\三角関数の積和公式{公式A}[b]

\auto{254}{\detokenize{\三角関数の積和公式{公式B}[i]}}

\三角関数の積和公式{公式B}[i]

\auto{255}{\detokenize{\三角関数の積和公式{公式B}[b]}}

\三角関数の積和公式{公式B}[b]

\auto{256}{\detokenize{\三角関数の積和公式{公式C}[i]}}

\三角関数の積和公式{公式C}[i]

\auto{257}{\detokenize{\三角関数の積和公式{公式C}[b]}}

\三角関数の積和公式{公式C}[b]

\auto{252}{\detokenize{\三角関数の積和公式{証明}}}

\三角関数の積和公式{証明}


\auto{258}{\detokenize{\三角関数の和積公式{公式A}[i]}}

\三角関数の和積公式{公式A}[i]

\auto{259}{\detokenize{\三角関数の和積公式{公式A}[b]}}

\三角関数の和積公式{公式A}[b]

\auto{260}{\detokenize{\三角関数の和積公式{公式B}[i]}}

\三角関数の和積公式{公式B}[i]

\auto{261}{\detokenize{\三角関数の和積公式{公式B}[b]}}

\三角関数の和積公式{公式B}[b]

\auto{262}{\detokenize{\三角関数の和積公式{公式C}[i]}}

\三角関数の和積公式{公式C}[i]

\auto{263}{\detokenize{\三角関数の和積公式{公式C}[b]}}

\三角関数の和積公式{公式C}[b]

\auto{264}{\detokenize{\三角関数の和積公式{公式D}[i]}}

\三角関数の和積公式{公式D}[i]

\auto{265}{\detokenize{\三角関数の和積公式{公式D}[b]}}

\三角関数の和積公式{公式D}[b]

\auto{258}{\detokenize{\三角関数の和積公式{証明}}}

\三角関数の和積公式{証明}


%\begin{simplesquarebox}{三角関数の合成}
%\begin{description}
\auto{267}{\detokenize{\三角関数の合成{公式}[i]}}

\三角関数の合成{公式}[i]

\auto{268}{\detokenize{\三角関数の合成{公式}[b]}}

\三角関数の合成{公式}[b]


\auto{267}{\detokenize{\三角関数の合成{証明}}}

\三角関数の合成{証明}

%\end{description}
%\end{simplesquarebox}
       
%\begin{simplesquarebox}{有理数の指数}
%\begin{description}
\auto{269}{\detokenize{\有理数の指数{公式A}[i]}}

\有理数の指数{公式A}[i]

\auto{270}{\detokenize{\有理数の指数{公式A}[b]}}

\有理数の指数{公式A}[b]

\auto{271}{\detokenize{\有理数の指数{公式B}[i]}}

\有理数の指数{公式B}[i]

\auto{272}{\detokenize{\有理数の指数{公式B}[b]}}

\有理数の指数{公式B}[b]

\auto{273}{\detokenize{\有理数の指数{公式C}[i]}}

\有理数の指数{公式C}[i]

\auto{274}{\detokenize{\有理数の指数{公式C}[b]}}

\有理数の指数{公式C}[b]


%\end{description}
%\end{simplesquarebox}
       
%\begin{simplesquarebox}{指数法則}
%\begin{description}
\auto{275}{\detokenize{\指数法則{公式A}[i]}}

\指数法則{公式A}[i]

\auto{276}{\detokenize{\指数法則{公式A}[b]}}

\指数法則{公式A}[b]

\auto{277}{\detokenize{\指数法則{公式B}[i]}}

\指数法則{公式B}[i]

\auto{278}{\detokenize{\指数法則{公式B}[b]}}

\指数法則{公式B}[b]

\auto{279}{\detokenize{\指数法則{公式C}[i]}}

\指数法則{公式C}[i]

\auto{280}{\detokenize{\指数法則{公式C}[b]}}

\指数法則{公式C}[b]

\auto{281}{\detokenize{\指数法則{公式D}[i]}}

\指数法則{公式D}[i]

\auto{282}{\detokenize{\指数法則{公式D}[b]}}

\指数法則{公式D}[b]

\auto{283}{\detokenize{\指数法則{公式E}[i]}}

\指数法則{公式E}[i]

\auto{284}{\detokenize{\指数法則{公式E}[b]}}

\指数法則{公式E}[b]


%\end{description}
%\end{simplesquarebox}
       
%\begin{simplesquarebox}{対数の定義}
%\begin{description}
\auto{285}{\detokenize{\対数の定義{定義}[i]}}

\対数の定義{定義}[i]

\auto{286}{\detokenize{\対数の定義{定義}[b]}}

\対数の定義{定義}[b]


%\end{description}
%\end{simplesquarebox}
       
%\begin{simplesquarebox}{対数の性質}
%\begin{description}
\auto{287}{\detokenize{\対数の性質{公式A}[i]}}

\対数の性質{公式A}[i]

\auto{288}{\detokenize{\対数の性質{公式A}[b]}}

\対数の性質{公式A}[b]

\auto{289}{\detokenize{\対数の性質{公式B}[i]}}

\対数の性質{公式B}[i]

\auto{290}{\detokenize{\対数の性質{公式B}[b]}}

\対数の性質{公式B}[b]

\auto{291}{\detokenize{\対数の性質{公式C}[i]}}

\対数の性質{公式C}[i]

\auto{292}{\detokenize{\対数の性質{公式C}[b]}}

\対数の性質{公式C}[b]

\auto{293}{\detokenize{\対数の性質{公式D}[i]}}

\対数の性質{公式D}[i]

\auto{294}{\detokenize{\対数の性質{公式D}[b]}}

\対数の性質{公式D}[b]

\auto{295}{\detokenize{\対数の性質{公式E}[i]}}

\対数の性質{公式E}[i]

\auto{296}{\detokenize{\対数の性質{公式E}[b]}}

\対数の性質{公式E}[b]

\auto{297}{\detokenize{\対数の性質{公式F}[i]}}

\対数の性質{公式F}[i]

\auto{298}{\detokenize{\対数の性質{公式F}[b]}}

\対数の性質{公式F}[b]

\auto{287}{\detokenize{\対数の性質{証明}}}

\対数の性質{証明}

%\end{description}
%\end{simplesquarebox}
       
%\begin{simplesquarebox}{底の変換公式}
%\begin{description}
\auto{299}{\detokenize{\底の変換公式{公式}[i]}}

\底の変換公式{公式}[i]

\auto{300}{\detokenize{\底の変換公式{公式}[b]}}

\底の変換公式{公式}[b]

\auto{299}{\detokenize{\底の変換公式{証明}}}

\底の変換公式{証明}

%\end{description}
%\end{simplesquarebox}
       
%\begin{simplesquarebox}{導関数の微分}
%\begin{description}
\auto{301}{\detokenize{\導関数の定義{定義}[i]}}

\導関数の定義{定義}[i]

\auto{302}{\detokenize{\導関数の定義{定義}[b]}}

\導関数の定義{定義}[b]


%\end{description}
%\end{simplesquarebox}
       
%\begin{simplesquarebox}{べき乗関数と定数関数の導関数}
%\begin{description}
\auto{303}{\detokenize{\べき乗関数と定数関数の導関数{公式A}[i]}}

\べき乗関数と定数関数の導関数{公式A}[i]

\auto{304}{\detokenize{\べき乗関数と定数関数の導関数{公式A}[b]}}

\べき乗関数と定数関数の導関数{公式A}[b]

\auto{305}{\detokenize{\べき乗関数と定数関数の導関数{公式B}[i]}}

\べき乗関数と定数関数の導関数{公式B}[i]

\auto{306}{\detokenize{\べき乗関数と定数関数の導関数{公式B}[b]}}

\べき乗関数と定数関数の導関数{公式B}[b]

\auto{303}{\detokenize{\べき乗関数と定数関数の導関数{証明}}}

\べき乗関数と定数関数の導関数{証明}

%\end{description}
%\end{simplesquarebox}
       
%\begin{simplesquarebox}{導関数の性質}
%\begin{description}
\auto{307}{\detokenize{\導関数の性質{公式A}[i]}}

\導関数の性質{公式A}[i]

\auto{308}{\detokenize{\導関数の性質{公式A}[b]}}

\導関数の性質{公式A}[b]

\auto{309}{\detokenize{\導関数の性質{公式B}[i]}}

\導関数の性質{公式B}[i]

\auto{310}{\detokenize{\導関数の性質{公式B}[b]}}

\導関数の性質{公式B}[b]

\auto{311}{\detokenize{\導関数の性質{公式C}[i]}}

\導関数の性質{公式C}[i]

\auto{312}{\detokenize{\導関数の性質{公式C}[b]}}

\導関数の性質{公式C}[b]


%\end{description}
%\end{simplesquarebox}
       
%\begin{simplesquarebox}{接線の方程式}
%\begin{description}
\auto{313}{\detokenize{\接線の方程式{公式}[i]}}

\接線の方程式{公式}[i]

\auto{314}{\detokenize{\接線の方程式{公式}[b]}}

\接線の方程式{公式}[b]


%\end{description}
%\end{simplesquarebox}
       
%\begin{simplesquarebox}{不定積分の定義}
%\begin{description}
\auto{315}{\detokenize{\不定積分の定義{定義}[i]}}

\不定積分の定義{定義}[i]

\auto{316}{\detokenize{\不定積分の定義{定義}[b]}}

\不定積分の定義{定義}[b]

\auto{316}{\detokenize{\不定積分の性質{公式A}[i]}}

\不定積分の性質{公式A}[i]

\auto{316}{\detokenize{\不定積分の性質{公式A}[b]}}

\不定積分の性質{公式A}[b]

\auto{316}{\detokenize{\不定積分の性質{公式B}[i]}}

\不定積分の性質{公式B}[i]

\auto{316}{\detokenize{\不定積分の性質{公式B}[b]}}

\不定積分の性質{公式B}[b]

\auto{316}{\detokenize{\不定積分の性質{公式C}[i]}}

\不定積分の性質{公式C}[i]

\auto{316}{\detokenize{\不定積分の性質{公式C}[b]}}

\不定積分の性質{公式C}[b]

%\end{description}
%\end{simplesquarebox}
       
%\begin{simplesquarebox}{べき乗関数の不定積分}
%\begin{description}
%\auto{317}{\detokenize{\べき乗関数の不定積分{公式}[i]}}

%\べき乗関数の不定積分{公式}[i]

%\auto{318}{\detokenize{\べき乗関数の不定積分{公式}[b]}}

%\べき乗関数の不定積分{公式}[b]


%\end{description}
%\end{simplesquarebox}
       
%\begin{simplesquarebox}{不定積分の性質}
%\begin{description}
%\auto{319}{\texttt{\textbackslash 不定積分の性質\h{-0.1mm}$\lbrace$\h{公式A}\h{-0.1mm}$\rbrace$\kakkokukuri[[]{i}}}

\auto[1]{\不定積分の性質{公式A}[i]

\不定積分の性質{公式A}[i]


%\auto{320}{\texttt{\textbackslash 不定積分の性質\h{-0.1mm}$\lbrace$\h{公式A}\h{-0.1mm}$\rbrace$\kakkokukuri[[]{i}}}

\auto[1]{\不定積分の性質{公式A}[b]

\不定積分の性質{公式A}[b]

%\auto{321}{\texttt{\textbackslash 不定積分の性質\h{-0.1mm}$\lbrace$\h{公式B}\h{-0.1mm}$\rbrace$\kakkokukuri[[]{i}}}

\auto[1]{\不定積分の性質{公式B}[i]

\不定積分の性質{公式B}[i]


%\auto{322}{\texttt{\textbackslash 不定積分の性質\h{-0.1mm}$\lbrace$\h{公式B}\h{-0.1mm}$\rbrace$\kakkokukuri[[]{i}}}

\auto[1]{\不定積分の性質{公式B}[b]

\不定積分の性質{公式B}[b]

%\auto{323}{\texttt{\textbackslash 不定積分の性質\h{-0.1mm}$\lbrace$\h{公式C}\h{-0.1mm}$\rbrace$\kakkokukuri[[]{i}}}

\auto[1]{\不定積分の性質{公式C}[i]

\不定積分の性質{公式C}[i]


%\auto{324}{\texttt{\textbackslash 
%\不定積分の性質{公式C}\h{-0.1mm}$\rbrace$\kakkokukuri[[]{i}}}

\auto[1]{\不定積分の性質{公式C}[b]

\不定積分の性質{公式C}[b]


%\end{description}
%\end{simplesquarebox}
       
%\begin{simplesquarebox}{定積分の定義}
%\begin{description}
\auto{325}{\detokenize{\定積分の定義{定義}[i]}}

\定積分の定義{定義}[i]

\auto{326}{\detokenize{\定積分の定義{定義}[b]}}

\定積分の定義{定義}[b]


%\end{description}
%\end{simplesquarebox}
       
%\begin{simplesquarebox}{定積分の性質}
%\begin{description}
\auto{327}{\detokenize{\定積分の性質{公式A}[i]}}

\定積分の性質{公式A}[i]

\auto{328}{\detokenize{\定積分の性質{公式A}[b]}}

\定積分の性質{公式A}[b]

\auto{329}{\detokenize{\定積分の性質{公式B}[i]}}

\定積分の性質{公式B}[i]

\auto{330}{\detokenize{\定積分の性質{公式B}[b]}}

\定積分の性質{公式B}[b]

\auto{331}{\detokenize{\定積分の性質{公式C}[i]}}

\定積分の性質{公式C}[i]

\auto{332}{\detokenize{\定積分の性質{公式C}[b]}}

\定積分の性質{公式C}[b]

\auto{333}{\detokenize{\定積分の性質{公式D}[i]}}

\定積分の性質{公式D}[i]

\auto{334}{\detokenize{\定積分の性質{公式D}[b]}}

\定積分の性質{公式D}[b]

\auto{335}{\detokenize{\定積分の性質{公式E}[i]}}

\定積分の性質{公式E}[i]

\auto{336}{\detokenize{\定積分の性質{公式E}[b]}}

\定積分の性質{公式E}[b]


%\end{description}
%\end{simplesquarebox}
       
%\begin{simplesquarebox}{ベクトルの演算}
%\begin{description}
%n-336=個数
\auto{337}{\detokenize{\ベクトルの演算{公式A}[i]}}

\ベクトルの演算{公式A}[i]

\auto{338}{\detokenize{\ベクトルの演算{公式A}[b]}}

\ベクトルの演算{公式A}[b]

\auto{339}{\detokenize{\ベクトルの演算{公式B}[i]}}

\ベクトルの演算{公式B}[i]

\auto{340}{\detokenize{\ベクトルの演算{公式B}[b]}}

\ベクトルの演算{公式B}[b]

\auto{341}{\detokenize{\ベクトルの演算{公式C}[i]}}

\ベクトルの演算{公式C}[i]

\auto{342}{\detokenize{\ベクトルの演算{公式C}[b]}}

\ベクトルの演算{公式C}[b]

\auto{343}{\detokenize{\ベクトルの演算{公式D}[i]}}

\ベクトルの演算{公式D}[i]

\auto{344}{\detokenize{\ベクトルの演算{公式D}[b]}}

\ベクトルの演算{公式D}[b]

\auto{345}{\detokenize{\ベクトルの演算{公式E}[i]}}

\ベクトルの演算{公式E}[i]

\auto{346}{\detokenize{\ベクトルの演算{公式E}[b]}}

\ベクトルの演算{公式E}[b]

\auto{347}{\detokenize{\ベクトルの演算{公式F}[i]}}

\ベクトルの演算{公式F}[i]

\auto{348}{\detokenize{\ベクトルの演算{公式F}[b]}}

\ベクトルの演算{公式F}[b]

\auto{349}{\detokenize{\ベクトルの演算{公式G}[i]}}

\ベクトルの演算{公式G}[i]

\auto{350}{\detokenize{\ベクトルの演算{公式G}[b]}}

\ベクトルの演算{公式G}[b]

\auto{351}{\detokenize{\ベクトルの演算{公式H}[i]}}

\ベクトルの演算{公式H}[i]

\auto{352}{\detokenize{\ベクトルの演算{公式H}[b]}}

\ベクトルの演算{公式H}[b]

\auto{353}{\detokenize{\ベクトルの演算{公式I}[i]}}

\ベクトルの演算{公式I}[i]

\auto{354}{\detokenize{\ベクトルの演算{公式I}[b]}}

\ベクトルの演算{公式I}[b]

\auto{355}{\detokenize{\ベクトルの演算{公式J}[i]}}

\ベクトルの演算{公式J}[i]

\auto{356}{\detokenize{\ベクトルの演算{公式J}[b]}}

\ベクトルの演算{公式J}[b]

\auto{357}{\detokenize{\ベクトルの演算{公式K}[i]}}

\ベクトルの演算{公式K}[i]

\auto{358}{\detokenize{\ベクトルの演算{公式K}[b]}}

\ベクトルの演算{公式K}[b]

\auto{359}{\detokenize{\ベクトルの演算{公式L}[i]}}

\ベクトルの演算{公式L}[i]

\auto{360}{\detokenize{\ベクトルの演算{公式L}[b]}}

\ベクトルの演算{公式L}[b]

%\auto{361}{\detokenize{\ベクトルの演算{公式M}[i]}}

%\ベクトルの演算{公式M}[i]

%\auto{362}{\detokenize{\ベクトルの演算{公式M}[b]}}

%\ベクトルの演算{公式M}[b]


%\end{description}
%\end{simplesquarebox}
       
%\begin{simplesquarebox}{平面ベクトルの分解}
%\begin{description}
\auto{363}{\detokenize{\平面ベクトルの分解{公式}[i]}}

\平面ベクトルの分解{公式}[i]

\auto{364}{\detokenize{\平面ベクトルの分解{公式}[b]}}

\平面ベクトルの分解{公式}[b]


%\end{description}
%\end{simplesquarebox}
       
%\begin{simplesquarebox}{平面ベクトルの成分}
%\begin{description}
\auto{365}{\detokenize{\平面ベクトルの成分{公式A}[i]}}

\平面ベクトルの成分{公式A}[i]

\auto{366}{\detokenize{\平面ベクトルの成分{公式A}[b]}}

\平面ベクトルの成分{公式A}[b]

\auto{367}{\detokenize{\平面ベクトルの成分{公式B}[i]}}

\平面ベクトルの成分{公式B}[i]

\auto{368}{\detokenize{\平面ベクトルの成分{公式B}[b]}}

\平面ベクトルの成分{公式B}[b]

\auto{369}{\detokenize{\平面ベクトルの成分{公式C}[i]}}

\平面ベクトルの成分{公式C}[i]

\auto{370}{\detokenize{\平面ベクトルの成分{公式C}[b]}}

\平面ベクトルの成分{公式C}[b]

\auto{371}{\detokenize{\平面ベクトルの成分{公式D}[i]}}

\平面ベクトルの成分{公式D}[i]

\auto{372}{\detokenize{\平面ベクトルの成分{公式D}[b]}}

\平面ベクトルの成分{公式D}[b]

%\end{description}
%\end{simplesquarebox}
       
%\begin{simplesquarebox}{ベクトルの成分と大きさ}
%\begin{description}
\auto{373}{\detokenize{\ベクトルの成分と大きさ{公式A}[i]}}

\ベクトルの成分と大きさ{公式A}[i]

\auto{374}{\detokenize{\ベクトルの成分と大きさ{公式A}[b]}}

\ベクトルの成分と大きさ{公式A}[b]

\auto{375}{\detokenize{\ベクトルの成分と大きさ{公式B}[i]}}

\ベクトルの成分と大きさ{公式B}[i]

\auto{376}{\detokenize{\ベクトルの成分と大きさ{公式B}[b]}}

\ベクトルの成分と大きさ{公式B}[b]

\auto{373}{\detokenize{\ベクトルの成分と大きさ{証明}}}

\ベクトルの成分と大きさ{証明}

%\end{description}
%\end{simplesquarebox}
       
%\begin{simplesquarebox}{平面ベクトルの内積}
%\begin{description}
\auto{377}{\detokenize{\平面ベクトルの内積{公式}[i]}}

\平面ベクトルの内積{公式}[i]

\auto{378}{\detokenize{\平面ベクトルの内積{公式}[b]}}

\平面ベクトルの内積{公式}[b]


%\end{description}
%\end{simplesquarebox}
       
%\begin{simplesquarebox}{内積の性質}
%\begin{description}
\auto{379}{\detokenize{\内積の性質{公式A}[i]}}

\内積の性質{公式A}[i]

\auto{380}{\detokenize{\内積の性質{公式A}[b]}}

\内積の性質{公式A}[b]

\auto{381}{\detokenize{\内積の性質{公式B}[i]}}

\内積の性質{公式B}[i]

\auto{382}{\detokenize{\内積の性質{公式B}[b]}}

\内積の性質{公式B}[b]

\auto{383}{\detokenize{\内積の性質{公式C}[i]}}

\内積の性質{公式C}[i]

\auto{384}{\detokenize{\内積の性質{公式C}[b]}}

\内積の性質{公式C}[b]

\auto{385}{\detokenize{\内積の性質{公式D}[i]}}

\内積の性質{公式D}[i]

\auto{386}{\detokenize{\内積の性質{公式D}[b]}}

\内積の性質{公式D}[b]

\auto{387}{\detokenize{\内積の性質{公式E}[i]}}

\内積の性質{公式E}[i]

\auto{388}{\detokenize{\内積の性質{公式E}[b]}}

\内積の性質{公式E}[b]

\auto{389}{\detokenize{\内積の性質{公式F}[i]}}

\内積の性質{公式F}[i]

\auto{390}{\detokenize{\内積の性質{公式F}[b]}}

\内積の性質{公式F}[b]


%\end{description}
%\end{simplesquarebox}
       
%\begin{simplesquarebox}{平面ベクトルの平行条件}
%\begin{description}
\auto{391}{\detokenize{\平面ベクトルの平行条件{条件}[i]}}

\平面ベクトルの平行条件{条件}[i]

\auto{392}{\detokenize{\平面ベクトルの平行条件{条件}[b]}}

\平面ベクトルの平行条件{条件}[b]


%\end{description}
%\end{simplesquarebox}
       
%\begin{simplesquarebox}{平面ベクトルの垂直条件}
%\begin{description}
\auto{393}{\detokenize{\平面ベクトルの垂直条件{条件}[i]}}

\平面ベクトルの垂直条件{条件}[i]

\auto{394}{\detokenize{\平面ベクトルの垂直条件{条件}[b]}}

\平面ベクトルの垂直条件{条件}[b]


%\end{description}
%\end{simplesquarebox}
       
%\begin{simplesquarebox}{位置ベクトル}
%\begin{description}
\auto{395}{\detokenize{\位置ベクトル{公式A}[i]}}

\位置ベクトル{公式A}[i]

\auto{396}{\detokenize{\位置ベクトル{公式A}[b]}}

\位置ベクトル{公式A}[b]

\auto{395}{\detokenize{\位置ベクトル{公式A}[i]}}

\位置ベクトル{内分点の位置ベクトルの証明}

\auto{397}{\detokenize{\位置ベクトル{公式B}[i]}}

\位置ベクトル{公式B}[i]

\auto{398}{\detokenize{\位置ベクトル{公式B}[b]}}

\位置ベクトル{公式B}[b]

\auto{397}{\detokenize{\位置ベクトル{外分点の位置ベクトルの証明}}}

\位置ベクトル{外分点の位置ベクトルの証明}

\auto{399}{\detokenize{\位置ベクトル{公式C}[i]}}

\位置ベクトル{公式C}[i]

\auto{400}{\detokenize{\位置ベクトル{公式C}[b]}}

\位置ベクトル{公式C}[b]

\auto{401}{\detokenize{\位置ベクトル{公式D}[i]}}

\位置ベクトル{公式D}[i]

\auto{402}{\detokenize{\位置ベクトル{公式D}[b]}}

\位置ベクトル{公式D}[b]


%\end{description}
%\end{simplesquarebox}
       
%\begin{simplesquarebox}{ベクトル方程式}
%\begin{description}
\auto{403}{\detokenize{\ベクトル方程式{公式A}[i]}}

\ベクトル方程式{公式A}[i]

\auto{404}{\detokenize{\ベクトル方程式{公式A}[b]}}

\ベクトル方程式{公式A}[b]

\auto{405}{\detokenize{\ベクトル方程式{公式B}[i]}}

\ベクトル方程式{公式B}[i]

\auto{406}{\detokenize{\ベクトル方程式{公式B}[b]}}

\ベクトル方程式{公式B}[b]

\auto{407}{\detokenize{\ベクトル方程式{公式C}[i]}}

\ベクトル方程式{公式C}[i]

\auto{408}{\detokenize{\ベクトル方程式{公式C}[b]}}

\ベクトル方程式{公式C}[b]

\auto{409}{\detokenize{\ベクトル方程式{公式D}[i]}}

\ベクトル方程式{公式D}[i]

\auto{410}{\detokenize{\ベクトル方程式{公式D}[b]}}

\ベクトル方程式{公式D}[b]


%\end{description}
%\end{simplesquarebox}
       
%\begin{simplesquarebox}{等差数列}
%\begin{description}
\auto{411}{\detokenize{\等差数列{一般項}[i]}}

\等差数列{一般項}[i]

\auto{412}{\detokenize{\等差数列{一般項}[b]}}

\等差数列{一般項}[b]

\auto{413}{\detokenize{\等差数列{総和}[i]}}

\等差数列{総和}[i]

\auto{414}{\detokenize{\等差数列{総和}[b]}}

\等差数列{総和}[b]

\auto{411}{\detokenize{\等差数列{証明}}}

\等差数列{証明}

%\end{description}
%\end{simplesquarebox}
       
%\begin{simplesquarebox}{等比数列}
%\begin{description}
\auto{415}{\detokenize{\等比数列{一般項}[i]}}

\等比数列{一般項}[i]

\auto{416}{\detokenize{\等比数列{一般項}[b]}}

\等比数列{一般項}[b]

\auto{417}{\detokenize{\等比数列{総和}[i]}}

\等比数列{総和}[i]

\auto{418}{\detokenize{\等比数列{総和}[b]}}

\等比数列{総和}[b]

\auto{415}{\detokenize{\等比数列{証明}}}

\等比数列{証明}

%\end{description}
%\end{simplesquarebox}
       
%\begin{simplesquarebox}{シグマの公式}
%\begin{description}
\auto{419}{\detokenize{\シグマの公式{公式A}[i]}}

\シグマの公式{公式A}[i]

\auto{420}{\detokenize{\シグマの公式{公式A}[b]}}

\シグマの公式{公式A}[b]

\auto{421}{\detokenize{\シグマの公式{公式B}[i]}}

\シグマの公式{公式B}[i]

\auto{422}{\detokenize{\シグマの公式{公式B}[b]}}

\シグマの公式{公式B}[b]

\auto{423}{\detokenize{\シグマの公式{公式C}[i]}}

\シグマの公式{公式C}[i]

\auto{424}{\detokenize{\シグマの公式{公式C}[b]}}

\シグマの公式{公式C}[b]

\auto{425}{\detokenize{\シグマの公式{公式D}[i]}}

\シグマの公式{公式D}[i]

\auto{426}{\detokenize{\シグマの公式{公式D}[b]}}

\シグマの公式{公式D}[b]

\auto{427}{\detokenize{\シグマの公式{公式E}[i]}}

\シグマの公式{公式E}[i]

\auto{428}{\detokenize{\シグマの公式{公式E}[b]}}

\シグマの公式{公式E}[b]

\auto{419}{\detokenize{\シグマの公式{証明}}}

\シグマの公式{証明}

%\end{description}
%\end{simplesquarebox}
       
%\begin{simplesquarebox}{シグマの性質}
%\begin{description}
\auto{429}{\detokenize{\シグマの性質{性質}[i]}}

\シグマの性質{性質}[i]

\auto{430}{\detokenize{\シグマの性質{性質}[b]}}

\シグマの性質{性質}[b]


%\end{description}
%\end{simplesquarebox}
       
%\begin{simplesquarebox}{階差数列}
%\begin{description}
\auto{431}{\detokenize{\階差数列{一般項}[i]}}

\階差数列{一般項}[i]

\auto{432}{\detokenize{\階差数列{一般項}[b]}}

\階差数列{一般項}[b]


%\end{description}
%\end{simplesquarebox}
       
%\begin{simplesquarebox}{漸化式}
%\begin{description}
\auto{433}{\detokenize{\漸化式{等差型}[i]}}

\漸化式{等差型}[i]

\auto{434}{\detokenize{\漸化式{等差型}[b]}}

\漸化式{等差型}[b]

\auto{435}{\detokenize{\漸化式{等比型}[i]}}

\漸化式{等比型}[i]

\auto{436}{\detokenize{\漸化式{等比型}[b]}}

\漸化式{等比型}[b]


\auto{437}{\detokenize{\漸化式{階差型}[i]}}

\漸化式{階差型}[i]

\auto{438}{\detokenize{\漸化式{階差型}[b]}}

\漸化式{階差型}[b]


\auto{439}{\detokenize{\漸化式{特性方程式}[i]}}

\漸化式{特性方程式}[i]

\auto{440}{\detokenize{\漸化式{特性方程式}[b]}}

\漸化式{特性方程式}[b]


%\end{description}
%\end{simplesquarebox}
       
%\begin{simplesquarebox}{数学的帰納法}
%\begin{description}
\auto{441}{\detokenize{\数学的帰納法}}

\数学的帰納法

%\end{description}
%\end{simplesquarebox}

%%%%%%%%%%%%%%%%%%%%ここから数\UTF{2162}%%%%%%%%%%%%%%%%%%%%

%\begin{simplesquarebox}{共役複素数}
%\begin{description}
%n-441=個数
\auto{442}{\detokenize{\共役複素数{定義}[i]}}

\共役複素数{定義}[i]

\auto{443}{\detokenize{\共役複素数{定義}[b]}}

\共役複素数{定義}[b]

\auto{444}{\detokenize{\共役複素数{性質A}[i]}}

\共役複素数{性質A}[i]

\auto{445}{\detokenize{\共役複素数{性質A}[b]}}

\共役複素数{性質A}[b]


\auto{446}{\detokenize{\共役複素数{性質B}[i]}}

\共役複素数{性質B}[i]

\auto{447}{\detokenize{\共役複素数{性質B}[b]}}

\共役複素数{性質B}[b]

\auto{448}{\detokenize{\共役複素数{性質C}[i]}}

\共役複素数{性質C}[i]

\auto{449}{\detokenize{\共役複素数{性質C}[b]}}

\共役複素数{性質C}[b]


\auto{450}{\detokenize{\共役複素数{性質D}[i]}}

\共役複素数{性質D}[i]

\auto{451}{\detokenize{\共役複素数{性質D}[b]}}

\共役複素数{性質D}[b]

\auto{452}{\detokenize{\共役複素数{性質E}[i]}}

\共役複素数{性質E}[i]

\auto{453}{\detokenize{\共役複素数{性質E}[b]}}

\共役複素数{性質E}[b]

\auto{454}{\detokenize{\共役複素数{性質F}[i]}}

\共役複素数{性質F}[i]

\auto{455}{\detokenize{\共役複素数{性質F}[b]}}

\共役複素数{性質F}[b]

\auto{456}{\detokenize{\共役複素数{性質G}[i]}}

\共役複素数{性質G}[i]

\auto{457}{\detokenize{\共役複素数{性質G}[b]}}

\共役複素数{性質G}[b]

\auto{442}{\detokenize{\共役複素数{証明}}}

\共役複素数{証明}

%\end{description}
%\end{simplesquarebox}

%\begin{simplesquarebox}{複素数の絶対値}
%\begin{description}
\auto{458}{\detokenize{\複素数の絶対値{定義}[i]}}

\複素数の絶対値{定義}[i]

\auto{459}{\detokenize{\複素数の絶対値{定義}[b]}}

\複素数の絶対値{定義}[b]

\auto{460}{\detokenize{\複素数の絶対値{性質A}[i]}}

\複素数の絶対値{性質A}[i]

\auto{461}{\detokenize{\複素数の絶対値{性質A}[b]}}

\複素数の絶対値{性質A}[b]

\auto{462}{\detokenize{\複素数の絶対値{性質B}[i]}}

\複素数の絶対値{性質B}[i]

\auto{463}{\detokenize{\複素数の絶対値{性質B}[b]}}

\複素数の絶対値{性質B}[b]

%\end{description}
%\end{simplesquarebox}

%\begin{simplesquarebox}{極形式}
%\begin{description}
\auto{464}{\detokenize{\極形式{定義}[i]}}

\極形式{定義}[i]

\auto{465}{\detokenize{\極形式{定義}[b]}}

\極形式{定義}[b]


\auto{466}{\detokenize{\極形式{性質A}[i]}}

\極形式{性質A}[i]

\auto{467}{\detokenize{\極形式{性質A}[b]}}

\極形式{性質A}[b]

\auto{468}{\detokenize{\極形式{性質B}[i]}}

\極形式{性質B}[i]

\auto{469}{\detokenize{\極形式{性質B}[b]}}

\極形式{性質B}[b]

%\end{description}
%\end{simplesquarebox}

%\begin{simplesquarebox}{偏角}
%\begin{description}
\auto{470}{\detokenize{\偏角{定義}[i]}}

\偏角{定義}[i]

\auto{471}{\detokenize{\偏角{定義}[b]}}

\偏角{定義}[b]


\auto{472}{\detokenize{\偏角{性質A}[i]}}

\偏角{性質A}[i]

\auto{473}{\detokenize{\偏角{性質A}[b]}}

\偏角{性質A}[b]


\auto{474}{\detokenize{\偏角{性質B}[i]}}

\偏角{性質B}[i]

\auto{475}{\detokenize{\偏角{性質B}[b]}}

\偏角{性質B}[b]

\auto{476}{\detokenize{\偏角{性質C}[i]}}

\偏角{性質C}[i]

\auto{477}{\detokenize{\偏角{性質C}[b]}}

\偏角{性質C}[b]

%\end{description}
%\end{simplesquarebox}

%\begin{simplesquarebox}{ドモアブルの定理}
%\begin{description}
\auto{478}{\detokenize{\ドモアブルの定理{公式}[i]}}

\ドモアブルの定理{公式}[i]

\auto{479}{\detokenize{\ドモアブルの定理{公式}[b]}}

\ドモアブルの定理{公式}[b]

\auto{478}{\detokenize{\ドモアブルの定理{証明}}}

\ドモアブルの定理{証明}

%\end{description}
%\end{simplesquarebox}

%\begin{simplesquarebox}{放物線}
%\begin{description}
\auto{480}{\detokenize{\放物線{定義}[i]}}

\放物線{定義}[i]

\auto{481}{\detokenize{\放物線{定義}[b]}}

\放物線{定義}[b]

\auto{482}{\detokenize{\放物線{性質A}[i]}}

\放物線{性質A}[i]

\auto{483}{\detokenize{\放物線{性質A}[b]}}

\放物線{性質A}[b]


\auto{484}{\detokenize{\放物線{性質B}[i]}}

\放物線{性質B}[i]

\auto{485}{\detokenize{\放物線{性質B}[b]}}

\放物線{性質B}[b]

\auto{486}{\detokenize{\放物線{性質C}[i]}}

\放物線{性質C}[i]

\auto{487}{\detokenize{\放物線{性質C}[b]}}

\放物線{性質C}[b]

%\end{description}
%\end{simplesquarebox}

%\begin{simplesquarebox}{楕円}
%\begin{description}
\auto{488}{\detokenize{\楕円{定義}[i]}}

\楕円{定義}[i]

\auto{489}{\detokenize{\楕円{定義}[b]}}

\楕円{定義}[b]


\auto{490}{\detokenize{\楕円{性質A}[i]}}

\楕円{性質A}[i]

\auto{491}{\detokenize{\楕円{性質A}[b]}}

\楕円{性質A}[b]


\auto{492}{\detokenize{\楕円{性質B}[i]}}

\楕円{性質B}[i]

\auto{493}{\detokenize{\楕円{性質B}[b]}}

\楕円{性質B}[b]

\auto{494}{\detokenize{\楕円{性質C}[i]}}

\楕円{性質C}[i]

\auto{495}{\detokenize{\楕円{性質C}[b]}}

\楕円{性質C}[b]

%\end{description}
%\end{simplesquarebox}

%\begin{simplesquarebox}{双曲線}
%\begin{description}
\auto{496}{\detokenize{\双曲線{定義}[i]}}

\双曲線{定義}[i]

\auto{497}{\detokenize{\双曲線{定義}[b]}}

\双曲線{定義}[b]


\auto{498}{\detokenize{\双曲線{性質A}[i]}}

\双曲線{性質A}[i]

\auto{499}{\detokenize{\双曲線{性質A}[b]}}

\双曲線{性質A}[b]


\auto{500}{\detokenize{\双曲線{性質B}[i]}}

\双曲線{性質B}[i]

\auto{501}{\detokenize{\双曲線{性質B}[b]}}

\双曲線{性質B}[b]

\auto{502}{\detokenize{\双曲線{性質C}[i]}}

\双曲線{性質C}[i]

\auto{503}{\detokenize{\双曲線{性質C}[b]}}

\双曲線{性質C}[b]

\auto{504}{\detokenize{\双曲線{性質D}[i]}}

\双曲線{性質D}[i]

\auto{505}{\detokenize{\双曲線{性質D}[b]}}

\双曲線{性質D}[b]

%\end{description}
%\end{simplesquarebox}

%\begin{simplesquarebox}{連続な関数}
%\begin{description}
\auto{506}{\detokenize{\連続な関数{公式}[i]}}

\連続な関数{公式}[i]

\auto{507}{\detokenize{\連続な関数{公式}[b]}}

\連続な関数{公式}[b]

%\end{description}
%\end{simplesquarebox}

%\begin{simplesquarebox}{中間値の定理}
%\begin{description}
\auto{508}{\detokenize{\中間値の定理{公式}[i]}}

\中間値の定理{公式}[i]

\auto{509}{\detokenize{\中間値の定理{公式}[b]}}

\中間値の定理{公式}[b]

%\end{description}
%\end{simplesquarebox}

%\begin{simplesquarebox}{平均値の定理}
%\begin{description}
\auto{510}{\detokenize{\平均値の定理{公式}[i]}}

\平均値の定理{公式}[i]

\auto{511}{\detokenize{\平均値の定理{公式}[b]}}

\平均値の定理{公式}[b]

%\end{description}
%\end{simplesquarebox}

%\begin{simplesquarebox}{微分}
%\begin{description}
\auto{512}{\detokenize{\微分{定義}[i]}}

\微分{定義}[i]

\auto{513}{\detokenize{\微分{定義}[b]}}

\微分{定義}[b]

\auto{514}{\detokenize{\微分{積の微分公式}[i]}}

\微分{積の微分公式}[i]

\auto{515}{\detokenize{\微分{積の微分公式}[b]}}

\微分{積の微分公式}[b]

\auto{516}{\detokenize{\微分{商の微分公式}[i]}}

\微分{商の微分公式}[i]

\auto{517}{\detokenize{\微分{商の微分公式}[b]}}

\微分{商の微分公式}[b]

\auto{518}{\detokenize{\微分{合成関数の微分}[i]}}

\微分{合成関数の微分}[i]

\auto{519}{\detokenize{\微分{合成関数の微分}[b]}}

\微分{合成関数の微分}[b]

\auto{520}{\detokenize{\微分{初等関数の微分公式A}[i]}}

\微分{初等関数の微分公式A}[i]

\auto{521}{\detokenize{\微分{初等関数の微分公式A}[b]}}

\微分{初等関数の微分公式A}[b]

\auto{522}{\detokenize{\微分{初等関数の微分公式B}[i]}}

\微分{初等関数の微分公式B}[i]

\auto{523}{\detokenize{\微分{初等関数の微分公式B}[b]}}

\微分{初等関数の微分公式B}[b]

\auto{524}{\detokenize{\微分{初等関数の微分公式C}[i]}}

\微分{初等関数の微分公式C}[i]

\auto{525}{\detokenize{\微分{初等関数の微分公式C}[b]}}

\微分{初等関数の微分公式C}[b]

\auto{526}{\detokenize{\微分{初等関数の微分公式D}[i]}}

\微分{初等関数の微分公式D}[i]

\auto{527}{\detokenize{\微分{初等関数の微分公式D}[b]}}

\微分{初等関数の微分公式D}[b]

\auto{528}{\detokenize{\微分{初等関数の微分公式E}[i]}}

\微分{初等関数の微分公式E}[i]

\auto{529}{\detokenize{\微分{初等関数の微分公式E}[b]}}

\微分{初等関数の微分公式E}[b]

\auto{530}{\detokenize{\微分{初等関数の微分公式F}[i]}}

\微分{初等関数の微分公式F}[i]

\auto{531}{\detokenize{\微分{初等関数の微分公式F}[b]}}

\微分{初等関数の微分公式F}[b]

\auto{532}{\detokenize{\微分{初等関数の微分公式G}[i]}}

\微分{初等関数の微分公式G}[i]

\auto{533}{\detokenize{\微分{初等関数の微分公式G}[b]}}

\微分{初等関数の微分公式G}[b]

\auto{534}{\detokenize{\微分{初等関数の微分公式H}[i]}}

\微分{初等関数の微分公式H}[i]

\auto{535}{\detokenize{\微分{初等関数の微分公式H}[b]}}

\微分{初等関数の微分公式H}[b]

\auto{536}{\detokenize{\微分{初等関数の微分公式I}[i]}}

\微分{初等関数の微分公式I}[i]

\auto{537}{\detokenize{\微分{初等関数の微分公式I}[b]}}

\微分{初等関数の微分公式I}[b]

\auto{512}{\detokenize{\微分{三角関数の微分公式の証明}}}

\微分{三角関数の微分公式の証明}

\auto{512}{\detokenize{\微分{対数関数の微分公式の証明}}}

\微分{対数関数の微分公式の証明}

%\end{description}
%\end{simplesquarebox}

%\begin{simplesquarebox}{接線の方程式}
%\begin{description}
\auto{538}{\detokenize{\接線の方程式{公式}[i]}}

\接線の方程式{公式}[i]

\auto{539}{\detokenize{\接線の方程式{公式}[b]}}

\接線の方程式{公式}[b]

%\end{description}
%\end{simplesquarebox}

%\begin{simplesquarebox}{法線の方程式}
%\begin{description}
\auto{540}{\detokenize{\法線の方程式{公式}[i]}}

\法線の方程式{公式}[i]

\auto{541}{\detokenize{\法線の方程式{公式}[b]}}

\法線の方程式{公式}[b]

%\end{description
%\end{simplesquarebox}

%\begin{simplesquarebox}{不定積分}
%\begin{description}
\auto{542}{\detokenize{\不定積分{定義}[i]}}

\不定積分{定義}[i]

\auto{543}{\detokenize{\不定積分{定義}[b]}}

\不定積分{定義}[b]


\auto{544}{\detokenize{\不定積分{置換積分}[i]}}

\不定積分{置換積分}[i]

\auto{545}{\detokenize{\不定積分{置換積分}[b]}}

\不定積分{置換積分}[b]


\auto{546}{\detokenize{\不定積分{部分積分}[i]}}

\不定積分{部分積分}[i]

\auto{547}{\detokenize{\不定積分{部分積分}[b]}}

\不定積分{部分積分}[b]

\auto{548}{\detokenize{\不定積分{初等関数の積分公式A}[i]}}

\不定積分{初等関数の積分公式A}[i]

\auto{549}{\detokenize{\不定積分{初等関数の積分公式A}[b]}}

\不定積分{初等関数の積分公式A}[b]

\auto{550}{\detokenize{\不定積分{初等関数の積分公式B}[i]}}

\不定積分{初等関数の積分公式B}[i]

\auto{551}{\detokenize{\不定積分{初等関数の積分公式B}[b]}}

\不定積分{初等関数の積分公式B}[b]

\auto{552}{\detokenize{\不定積分{初等関数の積分公式C}[i]}}

\不定積分{初等関数の積分公式C}[i]

\auto{553}{\detokenize{\不定積分{初等関数の積分公式C}[b]}}

\不定積分{初等関数の積分公式C}[b]

\auto{554}{\detokenize{\不定積分{初等関数の積分公式D}[i]}}

\不定積分{初等関数の積分公式D}[i]

\auto{555}{\detokenize{\不定積分{初等関数の積分公式D}[b]}}

\不定積分{初等関数の積分公式D}[b]

\auto{556}{\detokenize{\不定積分{初等関数の積分公式E}[i]}}

\不定積分{初等関数の積分公式E}[i]

\auto{557}{\detokenize{\不定積分{初等関数の積分公式E}[b]}}

\不定積分{初等関数の積分公式E}[b]

\auto{558}{\detokenize{\不定積分{初等関数の積分公式F}[i]}}

\不定積分{初等関数の積分公式F}[i]

\auto{559}{\detokenize{\不定積分{初等関数の積分公式F}[b]}}

\不定積分{初等関数の積分公式F}[b]

%\end{description}
%\end{simplesquarebox}

%\begin{simplesquarebox}{定積分}
%\begin{description}
\auto{560}{\detokenize{\定積分{定義}[i]}}

\定積分{定義}[i]

\auto{561}{\detokenize{\定積分{定義}[b]}}

\定積分{定義}[b]

%\end{description}
%\end{simplesquarebox}
       
%\begin{simplesquarebox}{区分求積法}
%\begin{description}
\auto{562}{\detokenize{\区分求積法{公式}[i]}}

\区分求積法{公式}[i]

\auto{563}{\detokenize{\区分求積法{公式}[b]}}

\区分求積法{公式}[b]

%\end{description}
%\end{simplesquarebox}

%\begin{simplesquarebox}{体積の積分}
%\begin{description}
\auto{564}{\detokenize{\体積の積分{公式}[i]}}

\体積の積分{公式}[i]

\auto{565}{\detokenize{\体積の積分{公式}[b]}}

\体積の積分{公式}[b]

%\end{description}
%\end{simplesquarebox}

\end{document}