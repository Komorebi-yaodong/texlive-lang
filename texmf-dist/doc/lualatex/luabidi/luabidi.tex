% !TeX spellcheck = en_US
% !TeX TS-program = lualatex
\documentclass[11pt]{ltxdoc}
\usepackage{color}
\usepackage{xspace,fancyvrb,booktabs}
\usepackage[neverdecrease]{paralist}
\definecolor{myblue}{rgb}{0.02,0.04,0.48}
\definecolor{lightblue}{rgb}{0.61,.8,.8}
\definecolor{myred}{rgb}{0.65,0.04,0.07}
\usepackage[
unicode=true,
bookmarks=true,
colorlinks=true,
linkcolor=myblue,
urlcolor=myblue,
citecolor=myblue,
hyperindex=false,
hyperfootnotes=false,
pdftitle={luabidi: Bidirectional typesetting in LuaLaTeX},
pdfauthor={V Khalighi, A Reutenauer, J Spitzmüller},
pdfkeywords={luatex, lualatex, multilingual, bidi}
]{hyperref}
\usepackage{metalogo}
%\let\XeTeX\undefined
%\let\XeLaTeX\undefined
\usepackage{polyglossia}
\usepackage{luabidi}

\setmainlanguage{english}
\usepackage[protrusion]{microtype}
\newcommand*\Cmd[1]{\cmd{#1}\DescribeMacro{#1}\xspace}
\newcommand*{\cscmd}[1]{\texttt{\textbackslash #1}}
\newcommand*\pkg[1]{\textsf{\color{myblue}#1}}
\newcommand*\file[1]{\texttt{\color{myblue}#1}}
\newcommand*\TR[1]{\textcolor{myred}{#1}}
\newcommand*\TX[1]{\hyperref[#1]{\textcolor{myred}{#1}}}
\newcommand*\TB[1]{\textcolor{myblue}{\bf #1}}
\newcommand*\TA[1]{\textsc{\color{myblue}#1}}
\newcommand*\link[1]{\href{#1}{#1}}
\def\eg{\textit{e.g.,}\xspace}
\def\ie{\textit{i.e.,}\xspace}
\def\ca{\textit{ca.}\@\xspace}
\def\Eg{\textit{E.g.,}\xspace}
\def\Ie{\textit{I.e.,}\xspace}
\def\etc{\@ifnextchar.{\textit{etc}}{\textit{etc.}\@\xspace}}

%% Sidenotes  << copied from fontspec.dtx
\newcommand\new[1]{%
	\edef\thisversion{#1}%
	\ifhmode\unskip~\fi{\ifx\thisversion\fileversion\color{blue}\else\color[gray]{0.5}\fi
		$\leftarrow$}%
	\marginpar{\centering
		\small\ifx\thisversion\fileversion\color{blue}\else\color[gray]{0.5}\fi
		\textsf{#1}}}
\newcommand\displaycmd[2]{%
	\\\DescribeMacro{#2}\centerline{\cmd{#1}}}
\renewenvironment{itemize}{\begin{compactitem}[\char"2023]}%[{\fontspec{DejaVu Sans}\char"25BB}]}%
	{\end{compactitem}}

\def\condbreak#1{%
	\vskip 0pt plus #1\pagebreak[3]\vskip 0pt plus -#1\relax}

\newcommand*\lbd{\pkg{luabidi}}
\newcommand*\Lbd{\pkg{Luabidi}}

%% fontspec declarations:
\setmainfont{Linux Libertine O}
\setsansfont{Linux Biolinum O}
\setmonofont[Scale=MatchLowercase]{DejaVu Sans Mono}

\linespread{1.05}
\frenchspacing
\EnableCrossrefs
\CodelineIndex
\RecordChanges
% COMMENT THE NEXT LINE TO INCLUDE THE CODE
\AtBeginDocument{\OnlyDescription}


\begin{document}
\hyphenation{Kha-li-ghi Reu-ten-auer}
\GetFileInfo{luabidi.sty}

\title{\lbd\\\color{myblue}Bidirectional typesetting in \LuaLaTeX}

\date{\color{myblue}\filedate \qquad \fileversion\\
	\footnotesize (\textsc{pdf} file generated on \today)}

\author{\TA{Vafa Khalighi} \and \TA{Arthur Reutenauer}\thanks{%
		Current maintainer. Please submit bug reports and feature requests to \protect\url{https://github.com/bidi-tex/luabidi/issues}.}
	    \and \TA{Jürgen Spitzmüller}}

\maketitle

\DeleteShortVerb{\|}
\MakeShortVerb{\¦}

\tableofcontents

\condbreak{2\baselineskip}
\section{Objectives}
\Lbd\ is an attempt to provide bidirectional writing support for the \LuaTeX\ engine in the same vein as the \pkg{bidi} package\footnote{%
See \url{https://ctan.org/pkg/bidi}.} enables bidirectional writing with \XeTeX. The most prominent user of this package is \pkg{polyglossia}\footnote{%
See \url{https://ctan.org/pkg/polyglossia}.} which uses \lbd\ with RTL languages and \LuaTeX\ output (as opposed to \pkg{bidi} with \XeTeX). Note, however, that \lbd\ is much more limited than \textsf{bidi}. Currently, only a very basic
subset of the latter's features is supported.

\section{Package Options}

\begin{description}
	\item[\TB{arabmaths}] By default, \lbd\ generates left-to-right maths. If you would like to have right-to-left maths, use this option.
	\item[\TB{textwidthfootnoterule}] expands the footnote rule to the whole text width.
	\item[\TB{autofootnoterule}] sets the footnote rule right or left aligned, depending on the direction of the first footnote
	that follows the rule (\ie that comes on the current page).
\end{description}

\section{User Commands}

\subsection{Main Text Direction}

By default, the main directionality of the document is left-to-right. To change it to right-to-left, use the switch

\begin{quote}
	\Cmd\setRTLmain
\end{quote}
%
This is advisable if your document consists mainly of right-to-left text.

\subsection{Paragraph Text Direction}

To change directionality for paragraphs, you can use the following switches:
\begin{description}
	\item[\cmd{\setRTL}]\DescribeMacro{\setRTL} (alias: \Cmd{\setRL}, \Cmd{\unsetLTR}) changes paragraph direction to right-to-left.
	\item[\cmd{\setLTR}]\DescribeMacro{\setLTR} (alias: \Cmd{\setLR}, \Cmd{\unsetRTL}) changes paragraph direction to left-to-right.
\end{description}
%
\condbreak{2\baselineskip}
Alternatively,\DescribeMacro{RTL}\DescribeMacro{LTR} you can also use the environments

\begin{quote}
	\begin{verbatim}
	\begin{RTL}
		...
	\end{RTL}
	\end{verbatim}
\end{quote}
%
or

\begin{quote}
	\begin{verbatim}
	\begin{LTR}
	...
	\end{LTR}
	\end{verbatim}
\end{quote}

\subsection{Inline Text Direction}

To change directionality for text chunks inside paragraphs, use
\begin{description}
	\item[\cmd\RLE\{...\}]\DescribeMacro{\RLE} (alias: \Cmd\RL) changes the directionality locally to right-to-left.
	\item[\cmd\LRE\{...\}]\DescribeMacro{\LRE} (alias: \Cmd\LR) changes the directionality locally to left-to-right.
\end{description}

\subsection{Footnotes}

\subsubsection{Horizontal Footnote Position}

\Lbd\ provides two additional footnote commands in addition to standard \cmd{\footnote}:
\Cmd{\RTLfootnote}\new{v0.5} and \Cmd{\LTRfootnote}\new{v0.5}.
The standard \cmd{\footnote} command thereby places the footnote always on the side that is currently the origin of direction:
on the left side of the page in LTR paragraphs and on the right in RTL paragraphs.

\cmd{\LTRfootnote}, in contrast, always places the footnote on the left side, notwithstanding the current
directionality. \cmd{\RTLfootnote} always places it on the right side. Like \cmd{\footnote}, \cmd{\RTLfootnote}
and \cmd{\LTRfootnote} have an optional argument to customize the number.


\subsubsection{Footnote Rule Length and Position}

By default, the placement of the footnote rule depends on the main text directionality:

\begin{itemize}
	\item In default mode (\ie if \cmd{\setRTLmain} is not used), the footnote rule is always set left-aligned (as usual in LTR documents).
	\item If the main direction is RTL (\ie if \cmd{\setRTLmain} is used), the footnote rule is always set right-aligned (as usual in RTL documents)
\end{itemize}
%
However, with the switch \Cmd{\leftfootnoterule}\new{v0.5}, all subsequent footnote rules are always placed on the left.
Likewise, \Cmd{\rightfootnoterule}\new{v0.5} causes all subsequent footnote rules to be always placed on the right.

The switch \Cmd{\autofootnoterule}\new{v0.5} and the corresponding package option advise \lbd\ to automatically determine the rule position,
depending on the directionality of the first footnote on the page. Note that this automatic can fail with footnotes at page boundaries
that differ in directionality from the first footnote on the page. You can work around such cases by switching to \cmd{\rightfootnoterule}
or \cmd{\leftfootnoterule} on these pages.

If you want a footnote rule that spans the whole text width, you can use the switch \Cmd{\textwidth\-footnoterule}\new{v0.5}
or the respective package option.

The length of left and right footnote rules can be adjusted via

\begin{quote}
  ¦\setlength\footenoterulewidth{⟨length⟩}¦\DescribeMacro{\footenoterulewidth}
\end{quote}
%
The predefined ¦⟨length⟩¦ is ¦0.4\columnwidth¦.



\section{Commands for Package Authors}

The following tests are provided to be used in packages:

\begin{description}
	\item[\cscmd{if@RTL}]\DescribeMacro{\if@RTL} determines whether the current paragraph direction is right-to-left.
	\item[\cscmd{if@RTLmain}]\DescribeMacro{\if@RTLmain} determines whether the main direction is right-to-left.
\end{description}
%
The following macros are provided:
\begin{description}
	\item[\cscmd{@ensure@RTL\{...\}}]\DescribeMacro{\@ensure@RTL} Ensure that the argument is typeset RTL.
	\item[\cscmd{@ensure@LTR\{...\}}]\DescribeMacro{\@ensure@LTR} Ensure that the argument is typeset LTR.
	\item[\cscmd{@ensure@dir\{...\}}]\DescribeMacro{\@ensure@dir} (alias: \cmd{\@ensure@maindir\{...\}}\DescribeMacro{\@ensure@maindir}) If used in RTL mode, the argument is put inside \cmd{\RLE}, if used in LTR mode, the argument is output as is.
    \item[\cscmd{hboxR\{...\}}]\DescribeMacro{\hboxR}\new{v0.5} Puts its argument in a RTL \cmd{\hbox}.
\end{description}

\section{Revision Log}

\begin{description}
	\item[v. 0.5 (2019/10/27)] Add \cmd{\RTLfootnote}, \cmd{\LTRfootnote}, and \cmd{\hboxR}; fix \texttt{autofootnoterule} option; add \cmd{\autofootnoterule},
	     \cmd{\leftfootnoterule}, \cmd{\rightfootnoterule} and \cmd{\textwidthfootnoterule}; add manual.
	\item[v. 0.4 (2019/08/24)] Fix \cmd{\@ensure@RTL}.
	\item[v. 0.3 (2019/07/10)] Fix compatibility with recent \LuaTeX\ (this version was never released to CTAN).
	\item[v. 0.2 (2013/05/27)] Fix additional files.
	\item[v. 0.1 (2009/04/01)] Initial release.
\end{description}

\end{document}
