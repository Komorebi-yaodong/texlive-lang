% This file is encoded in UTF-8.
% (bxcjkjatype supports only UTF-8.)
% To be typeset with pdflatex, latex + dvipdfmx, or latex + dvips
\documentclass[a4paper]{article}
% hyperref may require an explicit driver option
% (bxcjkjatype does not.)
%\documentclass[a4paper,dvipdfmx]{article}

% The 'whole' option makes the whole document body wrapped with a
% CJK* environment.
\usepackage[whole]{bxcjkjatype}

% After loading the package, you can write 日本語 text, even in the
% preamble. You can define a macro which includes 日本語.
\newcommand\Nihongo{日本語}

% You can make PDF files which holds correct CJK text as document
% information, such as boookmarks.
\usepackage[unicode,% ←needed to make bookmark text right
  bookmarks=true,colorlinks=true]{hyperref}

% Here I use Standard fonts (IPAex fonts) for Mincho and Gothic
% families. For Maru-gothic family, I use "Rounded M+ 1c Regular".
% (available at http://d.hatena.ne.jp/itouhiro/20120226).
\setmarugothicfont{rounded-mplus-1c-regular.ttf}

\begin{document}
% (Now already in a CJK* environment.)

% You can safely put a section heading containing 日本語 letters!
\section{Preparing 文書 in \Nihongo\ using pdf\TeX}
The 文書 contains 日本語 and English.

\section{Font selection}
\begin{itemize}
% \rmfamily designates Mincho family.
\item \rmfamily Mincho (明朝) family.
  % If you have specified a bold Mincho font with
  % \setboldminchofont, then you will see it working.
  % \textbf{And bold (太字) version.}
% \sffamily designates Gothic family.
\item \sffamily Gothic (ゴシック) family.
% \mgfamily changes only CJK family, so the alphabetic font
% family remains unchanged (i.e. in \sffamily).
\item \mgfamily Maru-gothic (丸ゴシック) family.
\end{itemize}

\end{document}
