\documentclass{article}
\newcommand{\mtrx}[1]{\ensuremath{\mathsf{#1}}}
\newcommand{\mi}{\mathrm{i}}
\begin{document}
A matrix \mtrx{A} of type $(m,n)$ or shortly $\mtrx{A}_{(m,n)}$ is a structure
of $m\times n$ elements, e.\,g.\ real or complex numbers, functions,
differential quotients, vectors et al., that can be ordered into $m$ rows
and $n$ columns
\[ \mtrx{A} = (a_{ij}) = \left(\begin{array}{cccc}
     a_{11} & a_{12} & \cdots & a_{1n} \\
     a_{21} & a_{22} & \cdots & a_{2n} \\
     \vdots & \vdots & \vdots & \vdots \\
     a_{m1} & a_{m2} & \cdots & a_{mn}
   \end{array}\right) \]

Complex matrices consist of complex elements $a_{\mu\nu} + \mi b_{\mu\nu}$.
Such a complex matrix can be split into two matrices
\[ \mtrx{A} + \mi\mtrx{B} \]
which elements consist of pure real numbers. Between the elements of a
complex matrix \mtrx{A} and the corresponding conjugated complex matrix
\mtrx{A^*} the relation
\[ a^*_{\mu\nu} = \Re(a_{\mu\nu} - \mi\Im(a_{\mu\nu}) \]
exist.

From the original matrix \mtrx{A} of type $(m,n)$ the transposed matrix
$\mtrx{A}^T$ is created by exchanging the corresponding rows and columns
$(a_{\nu\mu})^T = (a_{\mu\nu})$. It is of type $(n,m)$.


\end{document}
