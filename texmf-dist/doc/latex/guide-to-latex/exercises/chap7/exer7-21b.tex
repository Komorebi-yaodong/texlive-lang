%========================================================================
% With an slightly increased \textwidth a better presentation for the
% whole math. table can be made together with an even simpler solution.
%========================================================================
\documentclass{article}
\setlength{\textwidth}{135mm}
\begin{document}
\newcommand{\D}{\displaystyle}
\newcommand{\bm}{\boldmath}
\[ \begin{array}{@{}|c|c|c|@{}}\hline
\multicolumn{3}{@{}|c|@{}}{\rule[-0.125cm]{0mm}{0.5cm}%
\mbox{Equations for the tangential plane and surface normal}}\\
\hline
\mbox{Equation}&&\\
\mbox{for the} & \mbox{Tangential plane} & \mbox{Surface normal}\\
\mbox{surface} & & \\ \hline
\rule{0mm}{0.583cm}F(x,y,z)=0
    & \begin{array}[t]{r@{\:+\:}l}
             \D\frac{\partial F}{\partial x}(X-x)
           & \D\frac{\partial F}{\partial y}(Y-y) \\[2ex]
           & \D\frac{\partial F}{\partial z}(Z-z) = 0
      \end{array}
    & \D\frac{X-x}{\D\frac{\partial F}{\partial x}} =
        \frac{Y-y}{\D\frac{\partial F}{\partial y}} =
        \frac{Z-z}{\D\frac{\partial F}{\partial z}}\\
\rule[-0.42cm]{0mm}{1cm}z=f(x,y)
    & Z-z = p(X-x) + q(Y-y)
    & \D\frac{X-x}{p} = \frac{Y-y}{q} = \frac{Z-z}{-1}\\
\begin{array}{c} x=x(u,v)\\y=y(u,v)\\z=z(u,v) \end{array}
    & \begin{array}{|ccc|}
            X-x & Y-y & Z-z\\[0.5ex]
            \D\frac{\partial x}{\partial u} &
            \D\frac{\partial y}{\partial u} &
            \D\frac{\partial z}{\partial u} \\[2.0ex]
            \D\frac{\partial x}{\partial v} &
            \D\frac{\partial y}{\partial v} &
            \D\frac{\partial z}{\partial v}
      \end{array} = 0
    & \D\frac{X-x}{\left|\begin{array}{c}
               \frac{\partial y}{\partial u}\;
               \frac{\partial z}{\partial u}\\[0.8ex]
               \frac{\partial y}{\partial v}\;\frac{\partial z}{\partial v}
                   \end{array}\right|}
      \D \frac{Y-y}{\left|\begin{array}{c}
               \frac{\partial z}{\partial u}\;
               \frac{\partial x}{\partial u}\\[0.8ex]
               \frac{\partial z}{\partial v}\;\frac{\partial x}{\partial v}
                   \end{array}\right|}
      \D \frac{Z-z}{\left|\begin{array}{c}
               \frac{\partial x}{\partial u}\;
               \frac{\partial y}{\partial u}\\[0.8ex]
               \frac{\partial x}{\partial v}\;\frac{\partial y}{\partial v}
                   \end{array}\right|} \\
\rule[-0.42cm]{0mm}{1.17cm}\mbox{\boldmath$r=r$}(u,v)
    & \begin{array}{r}
         \mbox{\boldmath$(R-r)(r_1\times r_2) = \mbox{\unboldmath$0$}$}\\
         \mbox{or\qquad\boldmath$(R-r)N = \mbox{\unboldmath$0$}$}
      \end{array}
    & \begin{array}{r@{\;=\;}l}
         \mbox{\boldmath$R$} & \mbox{\boldmath$r +
         \mbox{\unboldmath$\lambda$}(r_1\times r_2$)}\\
         \mbox{or\quad\boldmath$R$} &
         \mbox{\boldmath$r + \mbox{\unboldmath$\lambda$}N$}
      \end{array}\\ \hline
\multicolumn{3}{@{}|c|@{}}{\parbox{12.5cm}{\vspace*{0.5ex}In this table
   $x,\,y,\,z$ and
   \mbox{\boldmath$r$} are the coordinates and the radius vector of a fixed
   point $M$ on the curve; $X,\,Y,\,Z$, and \mbox{\boldmath$R$} are the
   coordinates and radius vector of a point on the tangential plane or surface
   normal with reference to $M$; furthermore,
   $p = \frac{\partial z}{\partial x}$, $q = \frac{\partial z}{\partial y}$
   and $\mbox{\boldmath$r_1$} = \partial\mbox{\boldmath$r$}/\partial u$,
       $\mbox{\boldmath$r_2$} = \partial\mbox{\boldmath$r$}/\partial v$.}}
\\[0.8ex] \hline
\end{array}  \]
\end{document}
