\documentclass{article}
\begin{document}

% First Solution
\footnotesize\noindent
\begin{minipage}[b]{85mm}
\parbox[t]{3.5cm}{The first line of this 3.5\,cm wide minipage or parbox
   is aligned with the first line of the neighboring minipage or parbox.}
\hfill
\parbox[t]{4.5cm}{This 4.5\,cm wide minipage or parbox is positioned so
   so that its top line is at the same level as that of the box on the
   left, while its bottom line is even with that of the box on the right.
   The na\"{\i}ve notation that this arrangement may be achieved with the
   positioning arguments \texttt{t}, \texttt{t}, and \texttt{b} is incorrect.
   Why? What would this selection really produce?} \\ \mbox{}
\end{minipage}\hfill
\parbox[b]{3.1cm}{The true solution involves the nesting of two of the three
   structures in an enclosed minipage, which is then seperated aligned
   with the third one.} 

% Second Solution 
\vspace{10mm}\noindent
\parbox[t]{3.5cm}{The first line of this 3.5\,cm wide minipage or parbox
   is aligned with the first line of the neighboring minipage or parbox.}
\hfill
\begin{minipage}[t]{81mm}\mbox{}\\
\parbox[b]{4.5cm}{This 4.5\,cm wide minipage or parbox is positioned so
   so that its top line is at the same level as that of the box on the
   left, while its bottom line is even with that of the box on the right.
   The na\"{\i}ve notation that this arrangement may be achieved with the
   positioning arguments \texttt{t}, \texttt{t}, and \texttt{b} is incorrect.
   Why? What would this selection really produce?}  
\hfill
\parbox[b]{3.1cm}{The true solution involves the nesting of two of the three
   structures in an enclosed minipage, which is then seperated aligned
   with the third one.} 
\end{minipage}
\end{document}

