\documentclass{article}
\setlength{\textwidth}{130mm}
\begin{document}
\begin{table}{\bfseries Primary Energy Consumption}\\[1ex]
\begin{tabular*}{130mm}%
   {@{}ll@{\extracolsep{\fill}}r@{\hspace{1em}}@{\extracolsep{1em}}rr@{}}
%  {@{}ll@{\extracolsep{\fill}}r@{\extracolsep{1em}}rr@{}}
% Try this column definition as alternative solution
\hline
\multicolumn{2}{@{}l}{Energy Source} & 1975 & 1980 & 1986\\ \hline
\multicolumn{2}{@{}l}{Total Consumption}&&& \\
\multicolumn{2}{@{}l}{(in million tons of BCU)\footnote{BCU =
    Bituminous Coal Unit (1 ton BCU corresponds to the heating
    equivalent of 1 ton of bituminous coal = 8140 kwh)}}
    & 347.7 & 390.2 & 385.0 \\
of which & (percentages)\\
& petroleum       & 52.1 & 47.6 & 43.2 \\
& bituminous coal & 19.1 & 19.8 & 20.0 \\
& brown coal      &  9.9 & 10.0 &  8.6 \\
& natural gas     & 14.2 & 16.5 & 15.1 \\
& nuclear Energy  &  2.0 &  3.7 & 10.1 \\
& other\footnote{Wind, water, solar energy, and so forth}
    &  2.7 &  2.3 &  3.0
\end{tabular*}

\emph{Source:} Energy Balance Study Group, Essen 1987.
\end{table}

\noindent \textbf{Answers to the questions from exercise 4.15}:

\begin{enumerate}
\item With \verb=@{}= at the beginning and end of the formatting
      definition the additional space of half of the column gap size
      in front and behind the table which occures by default, will
      be suppressed.
\item With \verb=@{\extracolsep{\hfill}}= at the beginning of the formatting
      definition additional space of equal width would be inserted
      between all following columns so that the total tabular with
      becomes exatly the ordered width of 118 mm. Try this modification
      as special exersice.
\item \verb=@{\extracolsep{\hfill}}= is to put behind the
      second column parameter and its countermanding
      \verb=@{\hspace{1em}}@\extracolsep{1em}}= is to put behind the
      third column parameter. With \verb=@{\extracolsep{1em}}= as
      countermanding expression the gap between the third and fourth
      column would be smaller than those for the following column.
      Try this as alternative exercise.
\end{enumerate}
\end{document}
