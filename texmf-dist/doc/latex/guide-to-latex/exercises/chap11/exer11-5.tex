\documentclass{article}
\begin{document}
\section*{Preface to the \LaTeX-Guide}

A new edition to ``A Guide to \LaTeX'' begs the fundamental question:
Has \LaTeX\ changed so much since the appearance of the third edition in 1999
that a new release of this manual is justified?

The simple answer to that question is `Well \dots.' In 1994, the \LaTeX\
world was in upheaval with the issue of the new version \LaTeXe, and the
second edition of the `Guide' came out just then to act as the bridge
between the old and new versions. By 1998, the initial teething problems had
been worked out and corrected through semi-annual releases, and the third
edition could describe an established, working system. However, homage was
still paid to the older 2.09 version since many users still employed its
familiar syntax, although they were most likely to be using it in a \LaTeXe\
environment. \LaTeX\ has now reached a degree of stability that since 2000
the regular updates have been reduced to annual events, which often appear
months after the nominal date, something that does not worry anyone. The old
version 2.09 is obsolete and should no longer play any role in such a manual.
In this fourth edition, it is reduced to an appendix just to document its
syntax and usage. 

The developement of \LaTeX\ between 1992 until now will be best demonstrated
by the four editions of the \LaTeX-Guide textbook. The first edition
\cite{dal92} describes and teaches the old version 2.09, which is now obsolete.
With the occurance of \LaTeXe\ the second edition \cite{dal95} was presented,
explainig and teaching both versions in parallel. In the third edition
\cite{dal99} the \LaTeX\ presentation was mainly restricted to \LaTeXe\
and still more restricted with the present edition \cite{dal03}
 
But if \LaTeX\ itself has not changed substantially since 1999, many of its
peripherals have. The rise of programs like `pdf\TeX'  and `dvipdfm' for
PDF output adds new possibilities, which are realized, not in \LaTeX\
directly, but by means of more modern `packages' to extend the basic
features. The distribution of \TeX/\LaTeX\ installations has changed, such
that most users are given a complete, ready-to-run setup, with all the
`extras' that one used to have to obtain oneself. Those extras include
user-contributed packages, many of which are now considered indispensable.
Today `the \LaTeX\ system' includes much more than the basic kernel by Leslie
Lamport, encompassing the contributions of hundreds of other people. This
edition reflects this increase in breadth.

The changes to the fourth edition are mainly those of emphasis.

1. The material has been reorganized into `Basics' and `Beyond the Basics'
  (`advanced' sounds too intimidating) while the appendices contain
  topics that really can be skipped by most everyday users. Exception:
  Appendix H is an alphabetized command summary that many
  people find extremely useful (including ourselves).

  This reorganizing is meant to stress certain aspects over others. For
  example, the section on graphics inclusion and color was originally
  treated as an exotic freak, relegated to an appendix on extensions; in the
  third edition, it moved up to be included in a front chapter along with the
  picture environment and floats; now it dominates
  Chapter 6 all on its own, the floats come in the following
  Chapter 7, and `picture' is banished to the later
  Chapter 13. This is not to say that the picture features are no good,
  but only that they are very specialized. We add
  descriptions of additional drawing possibilities there too.

2. It is stressed as much as possible that \LaTeX\ is a markup language,
  with separation of content and form. Typographical settings
  should be placed in the preamble, while the body contains only logical
  markup. This is in keeping with the modern ideas of XML, where form and
  content are radically segregated.

3. Throughout this edition, contributed packages are explained at that point
  in the text where they are most relevant. The `fancyhdr' package
  comes in the section on page styles, `natbib' where literature
  citations are explained. This stresses that these `extensions' are part of
  the \LaTeX\ system as a whole. However, to remind the user that they must
  still be explicitly loaded, a marginal note is placed at the start of their
  descriptions.

4. PDF output is taken for granted throughout the book, in addition to the
  classical DVI format. This means that the added possibilities of `pdf\TeX'
  and `dvipdfm' are explained where they are relevant. A separate
  Chapter 10 on PostScript and PDF is still necessary, and the
  best interface to PDF output, the `hyperref' package by Sebastian
  Rahtz, is explained in detail. PDF is also included in
  Chapter 15 on presentation material

  On the other hand, the other Web output formats, HTML and XML, are only
  dealt with briefly in Appendix E, since these are large topics
  treated in other books, most noticeably the `\LaTeX\ Web Companion'.

5. This book is being distributed with the \TeX Live CD, with the kind
  permission of Sebastian Rahtz who maintains it for the \TeX\ Users Group.
  It contains a full \TeX\ and \LaTeX\ installation for Windows, Macintosh,
  and Linux, plus many of the myriad extensions that exist.

We once again express our hope that this \textsl{Guide} will prove more than
useful to all those who wish to find their way through the intricate world of
\LaTeX. And with the addition of the \TeX Live CD, that world is brought even
closer to their doorsteps.

\newpage

\begin{thebibliography}{\hspace{1.5cm}}
\bibitem[abr90]{abr90} Abraham P. W., with Hargraves K. A. and Berry K. (1990).
   \textsl{\TeX\ for the Impatient.} Reading MA: Addison-Wesley
\bibitem[bec93]{bec93} Bechtolsheim, S. von (1993) \textsl{\TeX\ in Practice,
   Vol.\ I--IV,} New York: Springer
\bibitem[bec93a]{bec93a} Vol.\ I: Basics
\bibitem[bec93b]{bec93b} Vol.\ II: Paragraphs, Math, and Fonts
\bibitem[bec93c]{bec93c} Vol.\ III: Tokens, Macros
\bibitem[bec93d]{bec93d} Vol.\ IV: Output Routines, Tables
\bibitem[dal92]{dal92} Daly P. W. and Kopka H. (1992) {A Guide to \LaTeX.}
   Workingham, UK: Addison-Wesley
\bibitem[dal95]{dal95} Daly P. W. and Kopka H. (1995) {A Guide to \LaTeX,}
   2nd ed. Workingham, UK: Addison-Wesley
\bibitem[dal99]{dal99} Daly P. W. and Kopka H. (1999) {A Guide to \LaTeX,}
   3rd ed. Workingham, UK: Addison-Wesley
\bibitem[dal03]{dal03} Daly P. W. and Kopka H. (2003) {A Guide to \LaTeX,}
   4th ed. Workingham, UK: Addison-Wesley
\bibitem[eij92]{eij92} Eijkhout V. (1992). \textsl{\TeX\ by Topic, a 
   \TeX nician's Referenc.} Harlow: Addison-Wesley
\bibitem[goo94]{goo94} Goossens M., Mittelbach F. and Samarin A. (1994).
   \textsl{The \LaTeX\ Companion.} Reading MA: Adddison-Wesley
\bibitem[goo97]{goo97} Goossens M., Rahtz S. and Mittelbach F. (1997).
   \textsl{The \LaTeX\ Graphic Companion.} Reading MA: Addison-Wesley
\bibitem[goo99]{goo99} Goossens M. and Rahtz S. (1999).
   \textsl{The \LaTeX\ Web Companion.} Reading MA: Addison-Wesley
\bibitem[knu91]{knu91} Knuth D. E. (1991) \textsl{Computers and Typesetting
   Vol A--E.}  Reading, MA: Addison-Wesley
\bibitem[knu91a]{knu91a} Knuth D. E. (1991a) Vol. A: \textsl{The \TeX book.}
   11.~ed. Reading MA: Addison-Wesley
\bibitem[knu91b]{knu91b} Knuth D. E. (1991b) Vol. B: \textsl{The Program.}
   4.~ed. Reading MA: Addison-Wesley
\bibitem[lam94]{lam94} Lamport L. (1994) \textsl{\LaTeX---A Document
   Preparation System}, 2nd ed. Reading MA: Addison-Wesley
    Addison-Wesley-Longman (Deutschland) GmbH. Bonn 1995
\end{thebibliography}
\end{document}
