\documentclass[11pt]{article}
\begin{document}
CAN VENICE BE SAVED?

Water and Venice have always a complicated relationship. 
The world's most famously wet city is also one of its most
famously endangered ones, forever being flooded by its
signature canals. Even since the 14th century, Italian
engineers have dreamed of ways to control the water. Now
a solution may be at hand: the Moses project, a vast series
of sea gates that may finally the sodden city dry.

The need for Venetian water control has never been greater.
Especially high tides have caused major floods 10 times in
the past 67 years alone, most diastrously in 1966 when the
water in parts of the city climbed to more than 1.83 m.
Compression of sediment under the city, along with rising sea
levels, often causes smaller floods, shutting down businesses
and making sidewalks and squares impossible.

The source of the problem is geography. Venice is primarily a
small cluster of interlocked islands set in the northern end of
a 536-sq.-km lagoon. A long ridge of land seperates the lagoon
from the far larger Gulf of Venice except at three major inlets.
These openings allow high gulf tides to become high Venetian tides,
with the water sometimes climbing far enough to swamp the city's
seawalls.

In 1984 a commission composed of Italy's 50 largest engeneering and
construction firms was formed to find a way to control the water
flow through the inlets, and Moses is it. Moses, an acronym for the
plan's technical name as well as a lyrical reference to the parting of
the Red Sea, calls for 78 hollow sea gates---each up to 5 m thick, 20 m
wide and 27.5 m long---to be hinged to foundations, or caissons, in the
seabed and to lie flat there. The gates would usually be filled with
water, but when tides rises to a height of 1 m or more, compressse air
would pump the water out. The free end of the gates would then float
upwards, breaking the surface after about 30 minutes and sealing off 
the inlets. Sea locks would permit permit ships to pass while the gates
are up.

The project---which would take as long as ten years and cost at least
\$2.7 billion---could still run into obstacles, especially given the
fickle nature of Italian politics. Concerned that gates would be raised
so frequently and remain there so long that they would cause water in the
lagoon to grow stagnant, Greens are making that argument in an
environmental-impact review that could delay or even scuttle construction.
Even so, this is the closest Venice has come to a permanent solution to 
its water problems in 700 years. By local bureaucratic standards, that's
not too bad.

%---By Jeffrey Kluger. Reported by Jeff Israely/Rome---Time, June 2, 2003 
\end{document}
