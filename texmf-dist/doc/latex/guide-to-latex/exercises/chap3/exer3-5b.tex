\documentclass[11pt]{article}
\pagestyle{myheadings} \markright{Exercises}
\pagenumbering{Roman}
\setlength{\textwidth}{15cm} \setlength{\textheight}{23cm}
\setlength{\oddsidemargin}{7.5mm} \setlength{\topmargin}{-5mm}
%***********************************************************************
%  The chosen values for \oddsidemargin and \topmargin are appropriate *
%  for a4 paper format. If your printer uses a different papersize,    *
%  e.g. letter format, adjust this values to those, appropriate for    *
%  your printer's paper format.                                        *
%***********************************************************************
\begin{document}
CAN VENICE BE SAVED?

Water and Venice have always a complicated relationship. 
The world's most famously wet city is also one of its most
famously endangered ones, forever being flooded by its
signature canals. Even since the 14th century, Italian
engineers have dreamed of ways to control the water. Now
a solution may be at hand: the Moses project, a vast series
of sea gates that may finally the sodden city dry.

The need for Venetian water control has never been greater.
Especially high tides have caused major floods 10 times in
the past 67 years alone, most diastrously in 1966 when the
water in parts of the city climbed to more than 1.83 m.
Compression of sediment under the city, along with rising sea
levels, often causes smaller floods, shutting down businesses
and making sidewalks and squares impossible.

The source of the problem is geography. Venice is primarily a
small cluster of interlocked islands set in the northern end of
a 536-sq.-km lagoon. A long ridge of land seperates the lagoon
from the far larger Gulf of Venice except at three major inlets.
These openings allow high gulf tides to become high Venetian tides,
with the water sometimes climbing far enough to swamp the city's
seawalls.

In 1984 a commission composed of Italy's 50 largest engeneering and
construction firms was formed to find a way to control the water
flow through the inlets, and Moses is it. Moses, an acronym for the
plan's technical name as well as a lyrical reference to the parting of
the Red Sea, calls for 78 hollow sea gates---each up to 5 m thick, 20 m
wide and 27.5 m long---to be hinged to foundations, or caissons, in the
seabed and to lie flat there. The gates would usually be filled with
water, but when tides rises to a height of 1 m or more, compressse air
would pump the water out. The free end of the gates would then float
upwards, breaking the surface after about 30 minutes and sealing off 
the inlets. Sea locks would permit permit ships to pass while the gates
are up.

The project---which would take as long as ten years and cost at least
\$2.7 billion---could still run into obstacles, especially given the
fickle nature of Italian politics. Concerned that gates would be raised
so frequently and remain there so long that they would cause water in the
lagoon to grow stagnant, Greens are making that argument in an
environmental-impact review that could delay or even scuttle construction.
Even so, this is the closest Venice has come to a permanent solution to 
its water problems in 700 years. By local bureaucratic standards, that's
not too bad.

%---By Jeffrey Kluger. Reported by Jeff Israely/Rome---Time, June 2, 2003 

\bigskip
ADDITIONAL INFORMATION TO THE GATES.

The gates are made of steel covered with a resistant coating to prevent
building of algae and crustaceans. Every five years they're are scheduled 
for removal and cleaning.

The 78 hollow sea gates are filled with water most of the time and remain
out of sight in a foundation or cassion. During especially high tides,
compressed air flushes out the seawater. Within 30 min.\ the gates rise
to the surface and block the inlets. When the danger passes, water is
admitted back into the gates, causing them to sink within 15 min. 

\bigskip
PREFACE TO THE \LaTeX-GUIDE

\medskip
A new edition to ``A Guide to \LaTeX'' begs the fundamental question:
Has \LaTeX\ changed so much since the appearance of the third edition in 1999
that a new release of this manual is justified?

The simple answer to that question is `Well \dots.' In 1994, the \LaTeX\
world was in upheaval with the issue of the new version \LaTeXe, and the
second edition of the `Guide' came out just then to act as the bridge
between the old and new versions. By 1998, the initial teething problems had
been worked out and corrected through semi-annual releases, and the third
edition could describe an established, working system. However, homage was
still paid to the older 2.09 version since many users still employed its
familiar syntax, although they were most likely to be using it in a \LaTeXe\
environment. \LaTeX\ has now reached a degree of stability that since 2000
the regular updates have been reduced to annual events, which often appear
months after the nominal date, something that does not worry anyone. The old
version 2.09 is obsolete and should no longer play any role in such a manual.
In this fourth edition, it is reduced to an appendix just to document its
syntax and usage.

But if \LaTeX\ itself has not changed substantially since 1999, many of its
peripherals have. The rise of programs like `pdf\TeX'  and `dvipdfm' for
PDF output adds new possibilities, which are realized, not in \LaTeX\
directly, but by means of more modern `packages' to extend the basic
features. The distribution of \TeX/\LaTeX\ installations has changed, such
that most users are given a complete, ready-to-run setup, with all the
`extras' that one used to have to obtain oneself. Those extras include
user-contributed packages, many of which are now considered indispensable.
Today `the \LaTeX\ system' includes much more than the basic kernel by Leslie
Lamport, encompassing the contributions of hundreds of other people. This
edition reflects this increase in breadth.
\end{document}
