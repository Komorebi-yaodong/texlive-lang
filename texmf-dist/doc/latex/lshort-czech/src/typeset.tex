%%%%%%%%%%%%%%%%%%%%%%%%%%%%%%%%%%%%%%%%%%%%%%%%%%%%%%%%%%%%%%%%%
% Contents: Typesetting Part of LaTeX2e Introduction
% $Id: typeset.tex 199 2009-04-20 06:29:16Z oetiker $
%%%%%%%%%%%%%%%%%%%%%%%%%%%%%%%%%%%%%%%%%%%%%%%%%%%%%%%%%%%%%%%%%
\chapter{Sazba textu}

\begin{intro}
  Po přečtení předchozí kapitoly byste měli znát základní prvky
  dokumentu systému \LaTeXe. V~této kapitole doplním
  informace o~zbytku reálně potřebných věcí.
\end{intro}

\section{Struktura textu a~jazyka}
\secby{Hanspeter Schmid}{hanspi@schmid-werren.ch}
Hlavním cílem psaní textu (vyjma některou
  DAAC\footnote{Different At All Cost (Odlišný za každou cenu),
  překlad zkratky UVA (Um's Verrecken Anders) ze švýcarské němčiny.}
  moderní literaturu)
  je předat myšlenky, informace nebo znalosti čtenáři. Text
  je srozumitelnější, pokud jsou myšlenky dobře strukturované
  a~typografická forma odpovídá logické a~významové struktuře
  obsahu.

\LaTeX{} se od jiných sázecích systémů liší v~tom, že od uživatele
vyžaduje popis logické a~sémantické struktury textu. \LaTeX{} sám potom
vytvoří typografickou reprezentaci textu podle \emph{pravidel}
uvedených v~souboru s~třídou dokumentu a~různých stylových souborech.

Nejdůležitější \emph{jednotka textu} v~\LaTeX u (i~v~typografii) je
\wi{odstavec}. Odstavci se říká jednotka textu, protože 
je typografickou formou, která by měla vyjadřovat jednu
soudržnou myšlenku nebo ideu. V~následujících sekcích
se dozvíte, jak vynutit řádkový zlom (např. pomocí \texttt{\bs\bs})
a~konec odstavce (např. pomocí prázdného řádku v~zdrojovém souboru).
S~novou myšlenkou by měl začít i~nový odstavec. Pokud jste
ohledně dělení do odstavců na pochybách, představte si text
jako přepravce ideí a~myšlenek. Stejná myšlenka 
by neměla být rozdělena do několika odstavců, naopak objeví-li
se v~textu odstavce myšlenka nová, měla by být přesunuta
do zvláštního odstavce.

Většina lidí podceňuje význam správného dělení textu do odstavců.
Řada lidí dokonce neví, na základě čeho se toto dělení uskutečňuje
nebo, zvláště v~\LaTeX u, vytvářejí nové odstavce omylem. Tento
poslední případ je jednoduché udělat hlavně tehdy, když
jsou v~textu použity rovnice. Podívejte se na následující příklady
a~zkuste poznat, proč prázdné řádky (odstavcové zlomy) před a~za
rovnicí někdy uvedeny jsou a~někdy ne. (Pokud těmto příkladům ještě
úplně nerozumíte, přečtěte si nejdřív tuto a~následující kapitolu
a~k~této sekci se vraťte potom.)
  
\begin{code}
\begin{verbatim}
% Example 1
\ldots když Einstein přišel s~vzorcem
\begin{equation} 
  e = m \cdot c^2 \; , 
\end{equation} 
což je nejznámější a~zároveň nejméně pochopený
matematický vzorec.


% Example 2
\ldots z~čehož plyne Kirchhoffův zákon o~uzlech:
\begin{equation} 
  \sum_{k=1}^{n} I_k = 0 \; .
\end{equation} 

Kirchhoffův zákon o~napětí lze odvodit\ldots


% Example 3
\ldots což má několik výhod.

\begin{equation} 
  I_D = I_F - I_R
\end{equation} 
je jádrem velmi odlišného tranzistorového modelu.
\end{verbatim}
\end{code} 

Menší textovou jednotkou je věta.  V~anglických textech bývá větší
mezera za tečkou ukončující větu než za tečkou uvedenou za zkratkou.
\LaTeX{} zkouší odhadnout, který z~těchto dvou významů každá tečka má.
Pokud je jeho odhad špatný, musíte udělat opravu.  To je vysvětleno
níže v~této kapitole.

Strukturování textu se týká dokonce i~částí vět. Většina jazyků
má komplikovaná interpunkční pravidla, ale v~mnoha jazycích
(včetně němčiny a~angličtiny) se můžete řídit jednoduchým vodítkem:
krátká pauza v~toku jazyka. Pokud si s~čárkami nejste jistí,
přečtěte si větu nahlas a~krátce se zastavte na každé čárce.
Pokud je někde pauza nepřirozená, vymažte příslušnou čárku.
Cítíte-li, že byste se v~některém místě měli nadechnout (nebo
krátce zastavit), přidejte čárku.

Nakonec, odstavce textu by měly být logicky seskupeny do vyšších
celků (kapitol, sekcí, podsekcí atd.). Typografický efekt napsání např.
\verb|\section{| \texttt{Struktura textu a~jazyka}\verb|}|
je zřejmý a~je jasně vidět, jak se toto strukturování na vyšší
úrovni používá.

\section{Řádkový a~stránkový zlom}
 
\subsection{Zarovnané odstavce}

Knihy se často sází tak, že každý řádek má stejnou délku. Aby toho
dosáhl, vkládá \LaTeX{}
nezbytné \index{řádkový zlom}řádkové zlomy a~mezery mezi slova a~přitom optimalizuje
s~ohledem na obsah celého odstavce. Navíc dělí slova, která
nelze dobře umístit na jednu řádku. To, jak jsou odstavce vysázeny,
závisí na třídě dokumentu. Obyčejně se odsazuje první řádek odstavce
a~mezi dva odstavce se nevkládá žádná dodatečná mezera. Více informací
je uvedeno v~sekci~\ref{parsp}%.

Ve speciálních případech musíme \LaTeX u říct, aby rozdělil řádku:

\begin{lscommand}
\ci{\bs} nebo \ci{newline} 
\end{lscommand}
\noindent zahájí novou řádku (ale ne nový odstavec). 

\begin{lscommand}
\ci{\bs*}
\end{lscommand}
\noindent navíc zakáže stránkový zlom za specifikovaným řádkovým zlomem.

\begin{lscommand}
\ci{newpage}
\end{lscommand}
\noindent zahájí stránku.

\begin{lscommand}
\ci{linebreak}\verb|[|\emph{n}\verb|]|,
\ci{nolinebreak}\verb|[|\emph{n}\verb|]|, 
\ci{pagebreak}\verb|[|\emph{n}\verb|]|,
\ci{nopagebreak}\verb|[|\emph{n}\verb|]|
\end{lscommand}
\noindent indikuje místa, kde je možné provést řádkový, resp. stránkový
zlom. Hodnotou nepovinného parametru je možné upřesnit, jak se má
\LaTeX{} zachovat. Hodnota může být mezi nula a~čtyři. Hodnota čtyři
znamená, že \LaTeX{} nemusí zlom udělat, pokud by výsledek vypadal
velmi špatně. Tyto \emph{příkazy zlomu} ale nejsou to samé jako
\emph{příkazy new} (např. \ci{newline}). Použijete-li některý
z~\uv{příkazů zlomu}, \LaTeX{} se pokusí vyrovnat na řádku text 
před zlomem (jak je popsáno v~následující sekci); nepříjemným výsledkem
mohou být velké mezery na daném řádku. Pokud opravdu chcete začít
novou řádku, resp. novou stránku, použijte příslušný \uv{příkaz new}.

\LaTeX{} se vždy snaží najít nejlepší možné řádkové zlomy. Pokud se
mu nepodaří najít takové, které by splňovaly jeho vysoké standardy,
nechá jednu z~řádek odstavce přečnívat přes pravý okraj a~vypíše varování
(\index{varování overfull hbox}\texttt{overfull hbox}). K~tomu dochází hlavně tehdy, když
se \LaTeX u nepodaří najít vhodné místo pro rozdělení slova.%
\footnote{Ačkoliv je vypsáno varování (\uv{Overfull hbox}) a~vypsán
problematický řádek, najít tento řádek v~dokumentu nemusí být snadné.
Použijete-li v~příkazu \ci{documentclass} volbu \texttt{draft}, budou
takového řádku označeny tlustou černou čárou v~pravém okraji.} Pomocí
příkazu \ci{sloppy} je možné \LaTeX u říct, aby trochu snížil své
standardy. Dá se tím zabránit výše zmíněným přečnívajícím řádkám
(pomocí zvětšení mezislovních mezer) a~je zobrazeno varování
(\index{varování underfull hbox}\texttt{underfull hbox}).
Výsledek většinou nevypadá moc dobře.
Příkazem \ci{fussy} vrátíme \LaTeX{} zpět k~jeho \uv{vysokým standardům}.

\subsection{Dělení slov} \label{hyph}

\LaTeX{} se pokouší dělit slova vždy, když by to vedlo k~\uv{lépe
vypadajícímu} odstavci. Pokud v~daném slově existují místa, kde lze
slovo korektně rozdělit, ale \LaTeX ový algoritmus dělení slov tato
místa nenajde, můžete pomocí následujících příkazů tato místa
\LaTeX u označit \uv{ručně}.

Příkaz 
\begin{lscommand}
\ci{hyphenation}\verb|{|\emph{word list}\verb|}|
\end{lscommand}
\noindent zařídí, že pro slova uvedená jako argument bude \LaTeX{}
uvažovat rozdělení jen v~místech ve slově označených pomocí \verb|-|.
Argumenty příkazu by měla být slova obsahující jen normální písmena
a~znaky, které jsou \LaTeX em za normální písmena považovány. Argument
příkazu (\uv{tipy} pro dělení daných slov) je uložen pro jazyk, který
je aktivní ve chvíli, kdy je příkaz spuštěn. Umístíte-li tedy tento
příkaz do preambule svého dokumentu, bude se vztahovat na angličtinu.
Pokud příkaz umístíte za \verb|\begin{document}| a~používáte balík
podporující sazbu v~jiném jazyku (např. \pai{babel}), pak se bude
argument příkazu \ci{hyphenation} vztahovat na jazyk daným balíkem
aktivovaný.

Následující příkaz zařídí, že bude možno dělit jak \uv{hyphenation},
tak \uv{Hyphenation} a~zakáže dělení slov \uv{FORTRAN}, \uv{Fortran}
a~\uv{fortran}. Argument příkazu \ci{hyphenation} nesmí obsahovat 
žádné speciální znaky nebo symboly.

Příklad:
\begin{code}
\verb|\hyphenation{FORTRAN Hy-phen-a-tion}|
\end{code}

Příkaz \ci{-} vloží na dané místo do daného slova \uv{volitelné dělítko}.
Místa v~daném slově takto označená budou jedinými místy, kde smí
\LaTeX{} dané slovo rozdělit. Tento způsob popisu výjimek pro dělení
slov je obzvlášť užitečný u~slov, která obsahují speciální znaky
(např. znaky s~akcenty), protože \LaTeX{} sám by takováto slova
neumožnil dělit v~žádném místě.
%\footnote{Unless you are using the new
%\wi{DC fonts}.}.

\begin{example}
nej\-ne\-ob\-hos\-po\-da\-%
\v{r}o\-v\'{a}\-va\-tel\-%
n\v{e}j\-\v{s}\'{\i}mu
\end{example}

Chceme-li zabránit tomu, aby byla část textu rozdělena, můžeme
tento text zadat jako argument příkazu
\begin{lscommand}
\ci{mbox}\verb|{|\emph{text}\verb|}|
\end{lscommand}

\begin{example}
Brzy budu mít nové telefonní
číslo. Bude to
\mbox{0116 291 2319}.

Parametr
\mbox{\emph{filename}} by měl
obsahovat jméno daného souboru.
\end{example}

Příkaz \ci{fbox} je podobný jako \ci{mbox}, ale navíc bude okolo
obsahu nakreslen rámeček.


\section{Předpřipravené řetězce}

V~některých z~příkladů na předchozích stranách se objevily
jednoduché \LaTeX ové příkazy pro sazbu speciálních řetězců:

\vspace{2ex}

\noindent
\begin{tabular}{@{}lll@{}}
Příkaz&Možný výsledek&Popis\\
\hline
\ci{today} & \today   & Aktuální datum\\
\ci{TeX} & \TeX       & Váš oblíbený sazeč\\
\ci{LaTeX} & \LaTeX   & Jméno naší hry\\
\ci{LaTeXe} & \LaTeXe & Její aktuální inkarnace\\
\end{tabular}

\section{Zvláštní znaky a~symboly}
 
\subsection{Uvozovky}

\index{uvozovky}Uvozovky byste \emph{neměli} sázet pomocí znaku \verb|"|\index{""@\texttt{""}}.
Pro otevírací a~uzavírací uvozovku místo toho používáme v~\LaTeX u speciální symboly:
dva znaky~\textasciigrave~(obrácený apostrof) pro otevírací uvozovku
a~dva znaky~\textquotesingle~(apostrof) pro uzavírací uvozovku. Pro jednoduché
uvozovky používáme jen jeden z~této dvojice znaků.
\begin{example}
``Press the key `x', please.''
\end{example}
Vysázeno výše vpravo to nevypadá nejlépe, ale opravdu jsme použili zpětný
apostrof pro otevírací uvozovku a~normální apostrof pro zavírací.
Navzdory tomu, co zvolený font skutečně vysázel\ldots

\TODO{add text about quotes in Czech}

\subsection{Pomlčky a~spojovníky}

Spojovník do \LaTeX ového dokumentu vložíme, uvedeme-li ve vstupním souboru
jeden znak \texttt{-}. Uvedením dvou, resp. tří za sebou následujících znaků
\texttt{-} vložíme pomlčky dvou různých velikostí. Uvedením znaku \texttt{-}
uvnitř matematického vzorce vložíme do dokumentu matematický znak mínus:
\index{-}\index{--}\index{---}\index{-@$-$}\index{matematické!mínus}

\begin{example}
je-li, anglicko-německý\\
na stranách 13--67\\
yes---or no? \\
$0$, $1$ a~$-1$
\end{example}
Jména jednotlivých znaků jsou:
\wi{spojovník} (-), \wi{půlčtverčíková pomlčka} (--), \wi{čtverčíková pomlčka} (---)
a~\wi{matematický znak mínus} ($-$).

\subsection{Tilda ($\sim$)}
\index{www}\index{URL}\index{tilda}
Ve webových adresách se často objevuje znak tilda. V~\LaTeX u
ho vygenerujete pomocí \verb|\~|, ale výsledek \~{} asi není to,
co chcete získat. Srovnejte:

\begin{example}
http://www.rich.edu/\~{}bush \\
http://www.clever.edu/$\sim$demo
\end{example}  
 
\subsection{Symbol stupně \texorpdfstring{($\circ$)}{}}

Tisk symbolu \wi{stupeň} v~\uv{normálním} \LaTeX u.

\begin{example}
Je-li opravdu
$-30\,^{\circ}\mathrm{C}$,
je asi zima\ldots
\end{example}

Balík \pai{textcomp} generuje symbol stupně i~pomocí příkazu
\ci{textdegree} nebo -- v~kombinaci s~C -- pomocí \ci{textcelsius}.

\begin{example}
30 \textcelsius{} je
86 \textdegree{}F.
\end{example}

\subsection{Symbol Eura \texorpdfstring{(\officialeuro)}{}}

Mnoho současných fontů obsahuje symbol Eura. Po nahrání balíku \pai{textcomp}
v~preambuli dokumentu
\begin{lscommand}
\ci{usepackage}\verb|{textcomp}| 
\end{lscommand}
můžete tento symbol vysázet pomocí příkazu
\begin{lscommand}
\ci{texteuro}
\end{lscommand}

Pokud váš font znak Eura neobsahuje, nebo pokud obsažen je, ale nelíbí se vám,
můžete použít balík \pai{eurosym}, který poskytuje oficiální podobu symbolu:
\begin{lscommand}
\ci{usepackage}\verb|[|\emph{official}\verb|]{eurosym}|
\end{lscommand}
Pokud byste chtěli symbol Eura, který se hodí k~vašemu fontu, použijte
\texttt{gen} místo \texttt{official}.

%If the Adobe Eurofonts are installed on your system (they are available for
%free from \url{ftp://ftp.adobe.com/pub/adobe/type/win/all}) you can use
%either the package \pai{europs} and the command \ci{EUR} (for a Euro symbol
%that matches the current font).
% does not work
% or the package
% \pai{eurosans} and the command \ci{euro} (for the ``official Euro'').

%The \pai{marvosym} package also provides many different symbols, including a
%Euro, under the name \ci{EURtm}. Its disadvantage is that it does not provide
%slanted and bold variants of the Euro symbol.

\begin{table}[!htbp]
\caption{Balík plný symbolů Eura.}\label{eurosymb}
\begin{lined}{10cm}
\begin{tabular}{llccc}
LM+textcomp  &\verb+\texteuro+ & \huge\texteuro &\huge\sffamily\texteuro
                                                &\huge\ttfamily\texteuro\\
eurosym      &\verb+\euro+ & \huge\officialeuro &\huge\sffamily\officialeuro
                                                &\huge\ttfamily\officialeuro\\
$[$gen$]$eurosym &\verb+\euro+ & \huge\geneuro  &\huge\sffamily\geneuro
                                                &\huge\ttfamily\geneuro\\
%europs       &\verb+\EUR + & \huge\EURtm        &\huge\EURhv
%                                                &\huge\EURcr\\
%eurosans     &\verb+\euro+ & \huge\EUROSANS  &\huge\sffamily\EUROSANS
%                                             & \huge\ttfamily\EUROSANS \\
%marvosym     &\verb+\EURtm+  & \huge\mvchr101  &\huge\mvchr101
%                                               &\huge\mvchr101
\end{tabular}
\medskip
\end{lined}
\end{table}

\subsection{Výpustky (\texorpdfstring{\ldots}{...})}

Na psacím stroji má znak \index{čárka}{čárky} a~\index{tečka}tečky stejnou šířku jako
každý jiný znak. U~sázených textů jsou tyto znaky mnohem užší a~sází
se velmi blízko k~předcházejícímu písmenu. Napsáním tří teček za sebou
tedy \index{výpustka}výpustku nezískáme. Získáme ji ale pomocí příkazu:

\begin{lscommand}
\ci{ldots}
\end{lscommand}
\index{...@\ldots}


\begin{example}
Takto ne ... ale takto ano:\\
New York, Tokyo,
Budapest, \ldots
\end{example}
 
\subsection{Ligatury}

Některé znaky, zapíšeme-li je ve vstupním souboru za sebou, budou
vysázeny jako jediný speciální symbol (doporučujeme v~ukázce využít zvětšení).
\begin{code}
{\large ff fi fl ffi\ldots}\quad
místo \quad {\large f{}f f{}i f{}l f{}f{}i \ldots}
\end{code}
Jedná se o~tzv. \index{ligatura}ligatury a~spojení jednotlivých znaků ve vstupním
souboru do speciálního symbolu (ligatury) dělá \LaTeX{} automaticky. Pokud si
tyto speciální symboly nepřejeme, můžeme napsat \ci{mbox}\verb|{}|
mezi příslušné dva znaky. Potlačit ligatury můžeme chtít u~slov
skládajících se ze dvou slov.

\begin{example}
\Large Not shelfful\\
but shelf\mbox{}ful
\end{example}
 
\subsection{Akcenty a~speciální znaky}
 
\LaTeX{} podporuje používání \index{akcent}akcentů
a~\index{speciální znak}speciálních znaků
mnoha jazyků. Tabulka~\ref{accents} ukazuje všechny druhy akcentů
aplikované na písmeno o. Akcenty jde samozřejmě stejně použít
i~na ostatní písmena.

Při přidávání akcentů nad písmena i~a~j je navíc potřeba odstranit
tečky, které jsou součástí těchto písmen, aby s~těmito akcenty
nekolidovaly. Provede se to pomocí \verb|\i| a~\verb|\j|.

\begin{example}
H\^otel, na\"\i ve, \'el\`eve,\\ 
sm\o rrebr\o d, !`Se\~norita!,\\
Sch\"onbrunner Schlo\ss{} 
Stra\ss e
\end{example}

\begin{table}[bth]
\caption{Akcenty a~speciální znaky.} \label{accents}
\begin{lined}{10cm}
\begin{tabular}{*4{cl}}
\A{\`o} & \A{\'o} & \A{\^o} & \A{\~o} \\
\A{\=o} & \A{\.o} & \A{\"o} & \B{\c}{c}\\[6pt]
\B{\u}{o} & \B{\v}{o} & \B{\H}{o} & \B{\c}{o} \\
\B{\d}{o} & \B{\b}{o} & \B{\t}{oo} \\[6pt]
\A{\oe}  &  \A{\OE} & \A{\ae} & \A{\AE} \\
\A{\aa} &  \A{\AA} \\[6pt]
\A{\o}  & \A{\O} & \A{\l} & \A{\L} \\
\A{\i}  & \A{\j} & !` & \verb|!`| & ?` & \verb|?`| 
\end{tabular}
\index{\i{} a~\j{} bez tečky}\index{skandinávská písmena}%
\index{ae@\ae}\index{umlaut}\index{grave}\index{acute}%
\index{oe@\oe}\index{aa@\aa}%

\bigskip
\end{lined}
\end{table}



\section{Podpora pro neanglické jazyky}
\index{internacionalizace} Když píšete dokumenty v~\index{jazyk}jazycích
jiných než je angličtina, je potřeba nakonfigurovat tři aspekty
\LaTeX u:

\begin{enumerate}
\item Všechny automaticky generované textové řetězce%
  \footnote{Nadpis obsahu, seznamu obrázků, \ldots} je potřeba přizpůsobit
  danému jazyku. Pro mnoho jazyků tak lze učinit pomocí balíku
  \pai{babel} od Johanna Braamse.
\item Je potřeba nastavit pravidla pro dělení slov v~daném jazyce.
  Tento aspekt je trochu komplikovanější. Je potřeba znovu přeložit
  formátový soubor, který bude používat nové vzory dělení slov.
  Váš \guide{} by měl obsahovat informace o~tom, jak to provést.
\item Je potřeba nastavit typografická pravidla specifická pro daný jazyk. Např.
  ve francouzštině je povinné psát mezeru před každý výskyt znaku 
  dvojtečka (:).
\end{enumerate}

%Pokud je 
Je-li váš systém %řádně 
nakonfigurován, aktivujte balík
\pai{babel} pomocí příkazu
\begin{lscommand}
\ci{usepackage}\verb|[|\emph{language}\verb|]{babel}|
\end{lscommand}

\noindent za příkazem \verb|\documentclass|. Na začátku každého překladu
pomocí \LaTeX u pak bude zobrazen seznam \emph{jazyků} zabudovaných do
vašeho \LaTeX ového systému. \textsf{Babel} automaticky aktivuje příslušné vzory
dělení slov pro jazyk, který si vyberete. Pokud váš \LaTeX ový formát
nepodporuje dělení slov v~jazyce, s~kterým pracujete, \textsf{babel} bude fungovat
i~tak, ale dělení slov samozřejmě prováděno nebude, což mívá negativní
vliv na vzhled vysázeného dokumentu.

\textsf{Babel} také pro některé jazyky poskytuje příkazy, které zjednodušují
sazby speciálních znaků. Např. \wi{němčina} obsahuje mnoho přehlásek
(\"a\"o\"u). S~nahraným \textsf{babel}em vložíte znak \"o pomocí
\verb|"o| místo~\verb|\"o|.

Používáte-li \textsf{babel} s~více jazyky, pak po provedení následujícího řádku
vašeho dokumentu
\begin{lscommand}
\ci{usepackage}\verb|[|\emph{jazykA}\verb|,|\emph{jazykB}\verb|]{babel}| 
\end{lscommand}

\noindent bude aktivní poslední z~jazyků, uvedených jako argumenty
\ci{usepackage} (tj. \emph{jazykB}). Příkazem
\begin{lscommand}
\ci{selectlanguage}\verb|{|\emph{languageA}\verb|}|
\end{lscommand}

\noindent pak můžete změnit aktivní jazyk.

%Input Encoding
\newcommand{\ieih}[1]{%
\index{kódování!vstupní!#1@\texttt{#1}}%
\index{vstupní kódování!#1@\texttt{#1}}%
\index{#1@\texttt{#1}}}
\newcommand{\iei}[1]{%
\ieih{#1}\texttt{#1}}
%Font Encoding
\newcommand{\feih}[1]{%
\index{kódování!fontů!#1@\texttt{#1}}%
\index{font -- kódování!#1@\texttt{#1}}%
\index{#1@\texttt{#1}}}
\newcommand{\fei}[1]{%
\feih{#1}\texttt{#1}}

Většina moderních počítačů umožňuje zapsat písmena národních jazyků
přímo z~klávesnice. \LaTeX{} na zacházení s~vstupními kódováními
různých skupin jazyků na různých platformách používá balík \pai{inputenc}:
\begin{lscommand}
\ci{usepackage}\verb|[|\emph{encoding}\verb|]{inputenc}|
\end{lscommand}

Při používání tohoto kódování byste měli vzít v~úvahu, že čtenáři vašeho
dokumentu nemusí být schopni si správně zobrazit vaše vstupní soubory,
protože používají jiné kódování. Např. německá přehláska \"a je na OS/2
zakódována jako 132, na Unixových systémech používajících ISO-LATIN~2
jako 228, zatímco v~kódování podle cyriliky cp1251 na Windows tento znak
vůbec neexistuje. S~\pai{inputenc} byste proto měli zacházet obezřetně.
Následující kódování mohou být užitečná (v~závislosti na tom, který
systém používáte).\footnote{Více se o~podporovaných vstupních kódováních
pro latinkové a~cyrilické jazyky můžete dočíst v~dokumentaci
k~\texttt{inputenc.dtx} a~k~\texttt{cyinpenc.dtx}.
Sekce~\ref{sec:Packages} ukazuje, jak vyrobit dokumentaci k~balíku.}

\begin{center}
\begin{tabular}{l | r | r }
Operační & \multicolumn{2}{c}{kódování}\\
systém  & západní latinka      & cyrilika\\
\hline
Mac     &  \iei{applemac} & \iei{macukr}  \\
Unix    &  \iei{latin1}   & \iei{koi8-ru}  \\ 
Windows &  \iei{ansinew}  & \iei{cp1251}    \\
DOS, OS/2  &  \iei{cp850} & \iei{cp866nav}
\end{tabular}                
\end{center}                 

Obsahuje-li váš dokument texty ve více jazycích s~nekompatibilními
vstupními kódováními, můžete používat Unicode a~balík \pai{ucs}.


\begin{lscommand}
\ci{usepackage}\verb|{ucs}|\\ 
\ci{usepackage}\verb|[|\iei{utf8x}\verb|]{inputenc}| 
\end{lscommand}
\noindent vám umožní vytvářet \LaTeX ové vstupní soubory v~kódování
\iei{utf8x}, vícebytovém kódování kde každý znak je zakódován pomocí
jednoho až čtyř bytů.

Jiným případem je kódování fontů, které definuje, na jakých pozicích
v~\TeX ovém fontu jsou uložené jednotlivé znaky. Více vstupních kódování
může být namapováno do jednoho kódování fontu, což redukuje počet
potřebných sad fontů. Kódování fontů se zařizuje pomocí balíku \pai{fontenc}:\label{fontenc}
\begin{lscommand}
\ci{usepackage}\verb|[|\emph{encoding}\verb|]{fontenc}| \index{fonty -- kódování}
\end{lscommand}
\noindent kde \emph{encoding} je kódování fontu. Je možné nahrát více kódování.

Implicitním kódováním fontů v~\LaTeX u je \label{OT1}\fei{OT1}, což je kódování
původních \TeX o\-vých fontů Computer Modern. Tyto fonty obsahují jen 128 znaků
7-bitové znakové sady ASCII. Když chceme používat akcentované znaky, můžeme
je \TeX em vytvořit kombinací normálních znaků s~akcenty. Výsledek sice vypadá
dobře, ale slova s~akcentovanými znaky nelze dělit. Navíc ne všechny latinkové
znaky lze vytvořit kombinací normálních znaků s~akcenty, nemluvě o~písmenech
nelatinkových abeced, např. v~řečtině nebo cyrilice.

Za účelem odstranění těchto nedostatků bylo navrženo několik sad osmibitových
fontů odvozených z~fontů Computer Modern. Fonty \emph{Extended Cork} (EC)
v~kódování \fei{T1} obsahují písmena a~interpunkční znaky většiny evropských
jazyků založených na latince. Sada fontů LH obsahuje písmena potřebná k~sazbě
dokumentů v~jazycích používajících cyriliku. Kvůli velkému počtu znaků
cyriliky jsou tyto rozděleny do čtyř fontových kódování -- \fei{T2A}, \fei{T2B},
\fei{T2C}, a~\fei{X2}.\footnote{Seznam jazyků podporovaných každým z~těchto
kódování lze najít v~\cite{cyrguide}.} Sada CB obsahuje fonty v~kódování
\fei{LGR} a~lze ji použít pro řecké texty.

Používáním těchto fontů můžete zlepšit, resp. umožnit dělení slov v~neanglických
dokumentech. Další výhodou používání fontů odvozených z~CM je, že 
jsou dostupné ve všech váhách, tvarech a~opticky škálované pro různé velikosti.

\subsection{Podpora pro portugalštinu}

\secby{Demerson Andre Polli}{polli@linux.ime.usp.br}
Aktivování dělení slov a~změnu všech automatických textů do portugalštiny
\index{Portugu\^es} provedete příkazem:
\begin{lscommand}
\verb|\usepackage[portuguese]{babel}|
\end{lscommand}
Pokud jste v~Brazílii, uveďte \texttt{\wi{brazilian}} místo \texttt{\wi{portuguese}}.

Protože v~portugalštině existuje mnoho akcentů, může se vám hodit
\begin{lscommand}
\verb|\usepackage[latin1]{inputenc}|
\end{lscommand}
pomocí čehož je můžete snadno zadávat a~také
\begin{lscommand}
\verb|\usepackage[T1]{fontenc}|
\end{lscommand}
pomocí kterého získáte korektní dělení slov.

V~tabulce~\ref{portuguese} je uveden text preambule, který je 
pro práci s~portugalštinou potřeba použít. V~příkladech používáme vstupní kódování
\texttt{latin1}, takže nebudou fungovat na operačních systémech Mac nebo DOS.
Stačí ale nastavit stejné kódování, jako má váš systém.

\begin{table}[!bth]
\caption{Preambule pro dokumenty v~portugalštině.} \label{portuguese}
\begin{lined}{6.5cm}
\begin{verbatim}
\usepackage[portuguese]{babel}
\usepackage[latin1]{inputenc}
\usepackage[T1]{fontenc}
\end{verbatim}

\bigskip
\end{lined}
\end{table}

%\newpage
\subsection{Podpora pro francouzštinu}

\secby{Daniel Flipo}{daniel.flipo@univ-lille1.fr}
Několik rad pro ty, kteří pomocí \LaTeX u vytvářejí dokumenty ve
\index{francouzština}francouzštině: Nahrajte podporu francouzštiny pomocí příkazu:

\begin{lscommand}
\verb|\usepackage[frenchb]{babel}|
\end{lscommand}

Všimněte si, že z~historických důvodů je jméno \textsf{babel}ovského
nastavení pro francouzštinu buď \emph{frenchb} nebo \emph{francais},
nikoliv \emph{french}.

Tím jsme umožnili francouzské dělení slov (je-li váš \LaTeX ový systém
řádně nakonfigurován). Také všechny automatické texty jsou převedeny
do francouzštiny. \verb+\chapter+ vysází Chapitre, \verb+\today+
vysází aktuální datum ve francouzštině, atd. Je také aktivována sada
nových příkazů, pomocí kterých můžete vstupní soubory ve francouzštině
psát jednodušeji. Pro inspiraci se můžete podívat do tabulky
\ref{cmd-french}:

\begin{table}[!htbp]
\caption{Speciální příkazy pro francouzštinu.} \label{cmd-french}
\begin{lined}{9cm}
\selectlanguage{french}
\begin{tabular}{ll}
\verb+\og guillemets \fg{}+         \quad &\og guillemets \fg \\[1ex]
\verb+M\up{me}, D\up{r}+            \quad &M\up{me}, D\up{r}  \\[1ex]
\verb+1\ier{}, 1\iere{}, 1\ieres{}+ \quad &1\ier{}, 1\iere{}, 1\ieres{}\\[1ex]
\verb+2\ieme{} 4\iemes{}+           \quad &2\ieme{} 4\iemes{}\\[1ex]
\verb+\No 1, \no 2+                 \quad &\No 1, \no 2   \\[1ex]
\verb+20~\degres C, 45\degres+      \quad &20~\degres C, 45\degres \\[1ex]
\verb+\bsc{M.~Durand}+              \quad &\bsc{M.~Durand} \\[1ex]
\verb+\nombre{1234,56789}+          \quad &\nombre{1234,56789}
\end{tabular}

\selectlanguage{czech}%
\bigskip
\end{lined}
\end{table}

S~přepnutím do francouzštiny se změní i~vzhled 
seznamů. Více informací o~tom, co volba \texttt{frenchb} v~\textsf{babel}u
dělá a~jak lze přizpůsobit, spusťte \LaTeX{} na soubor \texttt{frenchb.dtx}
a~přečtěte si vytvořený soubor \texttt{frenchb.dvi}.

\subsection{Podpora pro němčinu}

Několik tipů pro ty, kteří v~\LaTeX u vytvářejí dokumenty v~\index{němčina}němčině\index{němčina}:
nahrajte podporu pro němčinu pomocí následujícího příkazu:

\begin{lscommand}
\verb|\usepackage[german]{babel}|
\end{lscommand}

Tím je umožněno německé dělení slov (je-li váš \LaTeX ový systém nakonfigurován
správně) a~změněny automatické texty do němčiny. Z~\uv{Chapter} se tak stává
\uv{Kapitel}. K~dispozici je navíc sada nových příkazů, pomocí kterých můžete
psát německé vstupní soubory rychleji dokonce i~tehdy, když nepoužíváte balík
\texttt{inputenc}. Pro inspiraci můžete nahlédnout do tabulky \ref{german}.
Jestliže ale balík \texttt{inputenc} používáte, je možná lepší psát speciální německé symboly
jednoduše vložením kódu daného znaku v~daném kódování. (Nevýhodou tohoto postupu by
snad mohly být komplikace při konverzi dokumentu do jiného kódování.)
% With inputenc, all this becomes moot, but your text also is locked in a particular encoding world.\TODO{what was meant?}

\shorthandon{"}
\begin{table}[!htbp]
\caption{Speciální německé znaky.} \label{german}
\begin{lined}{8cm}
\selectlanguage{german}
\begin{tabular}{*2{ll}}
\verb|"a| & "a \hspace*{1ex} & \verb|"s| & "s \\[1ex]
\verb|"`| & "` & \verb|"'| & "' \\[1ex]
\verb|"<| nebo \ci{flqq} & "<  & \verb|">| nebo \ci{frqq} & "> \\[1ex]
\ci{flq} & \flq & \ci{frq} & \frq \\[1ex]
\ci{dq} & " \\
\end{tabular}
\selectlanguage{czech}
\bigskip
\end{lined}
\end{table}
\shorthandoff{"}

V~německých knihách se často setkáte s~francouzskými uvozovkami (\flqq guil\-le\-mets\frqq).
Němečtí sazeči je ale používají jinak než francouzští. Citace v~německé knize
vypadá \frqq takto\flqq. V~německy mluvící části Švýcarska ale sazeči používají
\flqq guillemets\frqq~stejným způsobem jako francouzští.

Vážným problémem vyskytujícím se při používání příkazů typu \verb+\flq+ je, že
jestliže používáte font OT1 (což je implicitní font), guillemets
vypadají jako matematický symbol ``$\ll$'', což žádného sazeče nepotěší. Naproti
tomu fonty kódované pomocí T1 obsahují příslušné symboly. Pokud tedy tento typ
citací používáte, užijte fontů v~kódování T1, \verb|\usepackage[T1]{fontenc}|.

%\newpage
\subsection[Podpora pro korejštinu]{Podpora pro korejštinu\footnotemark}\label{support_korean}%
\footnotetext{%
Nastínění řady problémů, s~kterými se korejští uživatelé
\LaTeX u musejí vypořádat. Za korejský tým překladatelů
lshort tuto sekci sepsal Karnes KIM. Do angličtiny text
přeložil SHIN Jungshik a~zredukoval Tobi Oetiker.}

Abychom %pomocí \LaTeX u 
mohli sázet \index{korejština}korejštinu, musíme vyřešit tři problémy:

\begin{enumerate}
\item 
Musíme být schopni vytvořit \wi{vstupní soubory v~korejštině}. Tyto soubory musí mít
formát obyčejného textu, ale protože korejština používá vlastní znakovou
sadu odlišnou od US-ASCII, budou při prohlížení pomocí normálního ASCII
editoru tyto vstupní soubory vypadat \uv{divoce}. Dvě nejčastěji používaná
kódování korejštiny jsou EUC-KR a~shora kompatibilní rozšíření
používané na korejských MS-Windows, CP949/Windows-949/UHC. U~těchto
kódování kódy znaků US-ASCII reprezentují příslušné ASCII znaky,
podobně jako u~ostatních kompatibilních kódování jako např. 
ISO-8859-\textit{x}, EUC-JP, Big5 nebo Shift\_JIS. Naopak,
slabiky Hangul, Hanjas (čínské znaky použité v~korejštině), Hangul Jamos,
Hiraganas, Katakanas, Greek a~cyrilické znaky a~další symboly a~znaky
odvozené z~KS~X~1001 jsou reprezentovány dvěma byty. První z~nich má
nastaven svůj nejvýznamnější bit. Do poloviny devadesátých
let minulého století bylo velmi obtížné nastavit korejské národní prostředí
na nelokalizovaných (nekorejských) operačních systémech.
Chcete-li mít představu o~tom, jak se tehdy
korejština na nekorejských operačních systémech používala, podívejte
se do dnes už velmi zastaralého dokumentu \url{http://jshin.net/faq}.

\item \TeX{} a~\LaTeX{} byly původně navrženy pro písmové systémy
jejichž abecedy obsahovaly maximálně 256 znaků. Aby bylo možné sázet
i~jazyky s~mnohem více znaky, jako např. korejštinu%
 \footnote{
 Korejský Hangul je systém založený na abecedě s~čtrnácti základními
 souhláskami a~deseti základními samohláskami (\emph{jamos} [čamo]). Na rozdíl od latinky
 nebo cyriliky musí být jednotlivé znaky seřazeny do obdélníkových bloků
 velikých zhruba jako čínské znaky. Každý blok reprezentuje slabiku.
 Z~konečného množství samohlásek a~souhlásek je tedy možno vytvořit
 neomezeně mnoho slabik. Moderní korejské pravopisné standardy (pro
 Severní i~Jižní Koreu) ale pro tvorbu uvádějí určitá omezení. Důsledkem
 je existence pouze konečného počtu korektních slabik. Korejské znakové
 kódování pro každou z~těchto slabik definuje kódové pozice
 (KS~X~1001:1998 and KS~X~1002:1992). To znamená, že Hangul, i~když je
 abecedního typu, se zpracovává jako čínština a~japonština, s~jejich
 desetitisíci obrázkovými/slovními znaky. ISO~10646/Unicode
 nabízí dvě cesty, jak Hangul použitý pro \emph{moderní} korejštinu
 reprezentovat. První možností je pomocí zakódování spojených samohlásek Hangulu
 (abecedy: \url{http://www.unicode.org/charts/PDF/U1100.pdf}). \par Druhou
 možností je zakódování všech korektních slabik Hangulu \emph{moderní}
 korejštiny (\url{http://www.unicode.org/charts/PDF/UAC00.pdf}). Jednou
 z~nejobávanějších výzev sazby korejštiny pomocí \LaTeX u a~příslušného
 sázecího systému je podpora středověké korejštiny -- a~případně
 budoucích korejštin -- slabik, které lze reprezentovat jedině spojením
 \emph{jamo} v~Unicode. Lze doufat, že budoucí \TeX ové systémy,
 např. $\Omega$ a~$\Lambda$, přinesou řešení, takže i~korejští lingvisté
 a~historici budou moct přejít z~Microsoft Wordu, který má pro středověkou
 korejštinu slušnou podporu.
 }
nebo čínštinu, byl vytvořen mechanismus subfontů. Ten dělí jeden ČJK font
s~tisíci nebo desetitisíci znaků na množinu subfontů po 256 znacích.
Pro korejštinu se široce používají tři balíky; \wi{H\LaTeX}
od UN~Koaunghiho, \wi{h\LaTeX{}p} od CHA~Jaechoona a~\wi{balík CJK}
od Wernera~Lemberga.\footnote{%
\raggedright%
Je možné je získat z~\CTANref|language/korean/HLaTeX/|,
   \CTANref|language/korean/CJK/|
a~\texttt{http://knot.kaist.ac.kr/htex/}}
H\LaTeX{} a~h\LaTeX{}p jsou specifické pro korejštinu a~kromě podpory fontů
poskytují i~korejskou lokalizaci. Oba umějí zpracovat korejské vstupní soubory
zakódované v~EUC-KR. H\LaTeX{}, je-li použitý spolu se systémem $\Lambda$ nebo
$\Omega$, umí zpracovat dokonce i~vstupní soubory zakódované
v~CP949/Windows-949/UHC a~UTF-8.

Balík CJK není specifický pro korejštinu. Umí zpracovat vstupní soubory
v~kódování UTF-8, EUC-KR a~CP949/Windows-949/UHC a~je možné s~ním
sázet vícejazyčné dokumenty (zvláště čínštinu, japonštinu a~korejštinu).
Na rozdíl od h\LaTeX u neobsahuje balík CJK korejskou lokalizaci a~nemá
ani tolik speciálních korejských fontů.

Konečným cílem použití programů typu \TeX{} a~\LaTeX{} je vysázení dokumentů
,esteticky` uspokojivým způsobem. Lze tvrdit, že nejdůležitější částí
sazby je množina dobře navržených fontů. Distribuce H\LaTeX u obsahuje
fonty \index{korejský font!font UHC}UHC \PSi{} (deset rodin písem)
a~TrueType fonty Munhwabu\footnote{Korejské ministerstvo kultury.}
(pět rodin písem). Balík CJK pracuje s~množinou fontů používanou
dřívější verzí H\LaTeX u a~může používat cyberbit TrueType font
od Bitstreamu.
\end{enumerate}

Abyste mohli své korejské texty vysázet pomocí balíku H\LaTeX,
vložte následující řádek do preambule svého dokumentu:
\begin{lscommand}
\verb+\usepackage{hangul}+
\end{lscommand}

Tímto příkazem se zapne korejská lokalizace. Názvy kapitol, sekcí,
podsekcí, obsah a~tabulka obrázků budou všechny přeloženy do korejštiny
a~formátování dokumentu bude přizpůsobeno korejským konvencím. Navíc
je aktivován automatický výběr \uv{částic}. Korejština
má dvojice částic-přípon, které jsou gramaticky ekvivalentní, ale
liší se formou. To, kterou z~nich použít závisí na tom, zda předcházející
slabika končí samohláskou nebo souhláskou (zjednodušeně řečeno).
Rodilí mluvčí nemají problém vybrat správnou částici, ale u~odkazů
a~dalších automatických textů je tento výběr nemožný. Ruční upravování
částic po každém přidání nebo odebrání odkazu nebo dalších úpravách
je únavné. H\LaTeX{} vás této zátěže zbaví.

Pokud chcete sázet korejštinu, ale korejskou lokalizaci nepotřebujete,
stačí do preambule dokumentu vložit následující řádku:

\begin{lscommand}
\verb+\usepackage{hfont}+
\end{lscommand}

Více informací o~sazbě korejštiny pomocí H\LaTeX u lze najít
v~návodu k~H\LaTeX u. Dalším zdrojem informací jsou stránky
korejské skupiny uživatelů \TeX u (KTUG) na \url{http://www.ktug.or.kr/}.
K~dispozici je také korejský překlad tohoto manuálu.

\subsection{Sazba řečtiny}
\secby{Nikolaos Pothitos}{pothitos@di.uoa.gr}
Abyste mohli sázet \index{řečtina}řečtinu\index{Greek}, musíte do preambule svého dokumentu vložit
text uvedený v~tabulce~\ref{preamble-greek}. Potom bude umožněno dělení
řeckých slov a~změněny automatické texty.%
\footnote{Použijete-li volbu \texttt{utf8x}
balíku \texttt{inputenc}, \LaTeX{} bude umět číst řecké a~polytonické řecké
znaky Unicodu.}

\begin{table}[!bthp]
\caption{Preambule pro řecké dokumenty.} \label{preamble-greek}
\begin{lined}{7cm}
\begin{verbatim}
\usepackage[english,greek]{babel}
\usepackage[iso-8859-7]{inputenc}
\end{verbatim}
\bigskip
\end{lined}
\end{table}

Výše zmíněný text v~preambuli také zpřístupní množinu příkazů, pomocí které lze
jednodušeji psát vstup v~řečtině. Pro dočasné přepínání mezi řečtinou a~angličtinou
můžeme použít příkazy \verb|\textlatin{|\emph{anglický text}\verb|}|
a~\verb|\textgreek{|\emph{řecký text}\verb|}|, které oba vysázejí text
předaný jako argument pomocí požadovaného kódování. Pro \uv{dlouhodobé}
přepnutí použijte příkaz \verb|\selectlanguage| popsaný v~předchozí sekci.
Některé z~řeckých interpunkčních znaků jsou uvedené v~tabulce~\ref{sym-greek}
Použijte \verb|\euro| pro symbol Eura.

\begin{table}[!htbp]
\caption{Greek Special Characters.} \label{sym-greek}
\begin{lined}{4cm}
\selectlanguage{french}
\begin{tabular}{*2{ll}}
\verb|;| \hspace*{1ex}  &  $\cdot$ \hspace*{1ex}  &  \verb|?| \hspace*{1ex}&  ;   \\[1ex]
\verb|((|               &  \og                    &  \verb|))|&  \fg \\[1ex]
\verb|``|               &  `                      &  \verb|''| &  '   \\
\end{tabular}
\selectlanguage{czech}
\bigskip
\end{lined}
\end{table}


\subsection{Podpora cyriliky}

\secby{Maksym Polyakov}{polyama@myrealbox.com}
Verze~3.7h balíku \pai{babel} obsahuje podporu kódování \fei{T2*}
a~podporu sazby bulharských, ruských a~ukrajinských
textů pomocí písmen cyriliky.  

Podpora pro cyriliku je založena na mechanismech standardního \LaTeX u
a~balíků \pai{fontenc} a~\pai{inputenc}. Pokud ale chcete používat
cyriliku v~matematickém módu, budete potřebovat před balíkem \pai{fontenc}
nahrát balík \pai{mathtext}:\footnote{Používáte-li balíky \AmS-\LaTeX, 
nahrajte je také před \pai{fontenc} a~\paih{babel}\textsf{babelem}.}

\begin{lscommand}
\verb+\usepackage{mathtext}+\\
\verb+\usepackage[+\fei{T1}\verb+,+\fei{T2A}\verb+]{fontenc}+\\
\verb+\usepackage[+\iei{koi8-ru}\verb+]{inputenc}+\\
\verb+\usepackage[english,bulgarian,russian,ukranian]{babel}+
\end{lscommand}

Balík \pai{babel} obecně automaticky vybere implicitní kódování fontů,
což pro výše zmíněné tři jazyky je kódování \fei{T2A}. Dokumenty ale
mohou obsahovat text ve více kódováních. Vícejazyčné dokumenty používající
jak cyrilické tak latinkové jazyky je rozumné explicitně vložit v~kódování
latinkového fontu. \paih{babel}\textsf{Babel} se postará o~přepnutí do příslušného kódování
fontů ve chvíli, kde je v~dokumentu vybrán jiný jazyk.

Mimo zpřístupnění dělení slov, překladu automaticky generovaných
textů a~aktivování některých typografických pravidel týkajících se
konkrétních jazyků (podobně jako příkaz \ci{frenchspacing} ve standardním
\LaTeX u) \pai{babel} poskytuje příkazy umožňující sazbu
podle standardů bulharštiny, ruštiny nebo ukrajinštiny. 

K~dispozici jsou interpunkční symboly pro všechny tři jazyky: cyrilická
textová pomlčka (trochu užší než latinková a~obklopená drobnými mezerami),
pomlčka pro přímou řeč, uvozovky a~příkazy usnadňující dělení slov,
všechny jsou uvedeny v~tabulce~\ref{Cyrillic}.

% Table borrowed from Ukrainian.dtx
\begin{table}[htb]
  \begin{center}
  \index{""-@\texttt{""}\texttt{-}} 
  \index{""---@\texttt{""}\texttt{-}\texttt{-}\texttt{-}} 
  \index{""=@\texttt{""}\texttt{=}} 
  \index{""`@\texttt{""}\texttt{`}} 
  \index{""'@\texttt{""}\texttt{'}} 
  \index{"">@\texttt{""}\texttt{>}} 
  \index{""<@\texttt{""}\texttt{<}} 
  \caption[Bulharština, ruština a~ukrajinština.]{Speciální definice pro
           bulharské, ruské a~ukrajinské volby.\paih{babel}%\textsf{babelu}.
	}\label{Cyrillic}\shorthandon{"}
  \begin{tabular}{@{}p{.1\hsize}@{}p{.9\hsize}@{}}
   \hline
   \verb="|= & Zakaž ligaturu na tomto místě.               \\
   \verb|"-| & Explicitní pomlčka, která nebrání
               dělení v~daném slově.                         \\
   \verb|"---| & Pomlčka pro obyčejný text v~cyrilice.                      \\
   \verb|"--~| & Pomlčka pro složená jména (příjmení) v~cyrilice.       \\
   \verb|"--*| & Pomlčka pro přímou řeč v~cyrilice.         \\
   \verb|""| & Podobné jako \verb|"-|, ale nevytvoří žádný znak pomlčky
               (pro složená slova s~pomlčkou, např.\verb|x-""y|
               nebo některé jiné znaky jako \uv{zakaž/povol}.     \\
   \verb|"~| & Pro složené slovo bez možnosti zlomu.  \\
   \verb|"=| & Pro složené slovo s~možností zlomu, umožňující dělení
          v~jednotlivých slovech.                   \\
   \verb|",| & Úzká mezera pro iniciály s~místem zlomu
          v~následujícím příjmení.                              \\
   \verb|"`| & Pro německé levé dvojité uvozovky
               (vypadají podobně jako: \quotedblbase).                     \\
   \verb|"'| & Pro německé pravé dvojité uvozovky (vypadají takto: ``).       \\%''
   \verb|"<| & Pro francouzské levé dvojité uvozovky (vypadají jako: $<\!\!<$).  \\
   \verb|">| & Pro francouzské pravé dvojité uvozovky (vypadají jako: $>\!\!>$). \\
   \hline
  \end{tabular}
  \end{center}
\end{table}
\shorthandoff{"}

Ruské a~ukrajinské volby \paih{babel}\textsf{babelu} definují příkazy \ci{Asbuk} a~\ci{asbuk}, které
se chovají jako \ci{Alph} a~\ci{alph}, ale vytvoří velká a~malá písmena ruské nebo
ukrajinské abecedy (v~závislosti na tom, který z~těchto jazyků je právě aktivní).
Bulharská volba \paih{babel}\textsf{babelu} zpřístupňuje příkazy \ci{enumBul} a~\ci{enumLat} (\ci{enumEng}),
které předefinovávají \ci{Alph} a~\ci{alph} tak, že vytvoří písmena
bulharské nebo latinské (anglické) abecedy. Implicitní výstup příkazů
\ci{Alph} a~\ci{alph} pro bulharštinu jsou znaky z~bulharské abecedy.

%Finally, math alphabets are redefined and  as well as the commands for math
%operators according to Cyrillic typesetting traditions. 

\subsection{Podpora pro mongolštinu}

Pro sazbu mongolštiny pomocí \LaTeX u máte dvě možnosti: buď vícejazyčný Babel,
nebo Mon\TeX{} Olivera Corffa.

Mon\TeX{} podporuje jak cyrilici tak tradiční mongolské písmo. Abyste mohli použít
příkazy Mon\TeX u, přidejte:
\begin{lscommand}
\ci{usepackage}\verb|[|\emph{language},\emph{encoding}\verb|]{mls}|
\end{lscommand}
\noindent do preambule. Pro generování popisů a~dat v~moderní mongolštině
zadejte \pai{xalx} jako volbu \emph{language}. Pro psaní celého dokumentu
v~mongolštině je třeba jako \emph{language} zadat hodnotu \pai{bicig}.
Volba jazyka \pai{bicig} umožňuje vstupní metody \uv{zjednodušené transliterace}.

Latinkový transliterační mód lze povolit a~zakázat pomocí
\begin{lscommand}
\verb|\SetDocumentEncodingLMC|
\end{lscommand}
a~\begin{lscommand}
\verb|\SetDocumentEncodingNeutral|
\end{lscommand}

Na \CTANalt|tex-archive/language/mongolian/montex/doc| získáte další informace o~Mon\TeX u.

Mongolské cyrilické písmo je podporováno i~v~\paih{babel}\textsf{babelu}. Podporu
pro mongolštinu aktivujete pomocí příkazů:

\begin{lscommand}
\verb|\usepackage[T2A]{fontenc}|\\
\verb|\usepackage[mn]{inputenc}|\\
\verb|\usepackage[mongolian]{babel}|
\end{lscommand}

\noindent kde \iei{mn} je vstupní kódování \iei{cp1251}. Pro modernější přístup
místo toho použijte \iei{utf8}.

\section{Mezery mezi slovy}

Aby mohly být řádky vysázených odstavců zarovnané (tj. mít stejnou šířku),
\LaTeX{} mezi slova vkládá různě velké mezery. Na konci věty je obvykle
vložena větší mezera, čímž se zlepší čitelnost. \LaTeX{} předpokládá, že
věty končí buď tečkou, otazníkem nebo vykřičníkem. Pokud ale tečka
následuje za velkým písmenem, \LaTeX{} předpokládá, že se o~konec věty
\emph{nejedná} -- ve většině případů totiž tato situace opravdu nastává
u~zkratek (uvedených uprostřed věty).

Chceme-li jiné chování než to popsané v~předchozím odstavci, musíme se
o~ně sami postarat: Backslash před mezerou generuje mezeru, která se
nikdy \uv{nenatáhne}. Znak tilda~`\verb|~|' vygeneruje mezeru, která
také nemůže být roztažena a~navíc v~ní ani nemůže dojít k~řádkovému
zlomu. Příkaz \verb|\@| uvedený před tečkou specifikuje, že tato tečka
ukončuje větu, i~když předchází velké písmeno.
\cih{"@}\index{~@ \verb.~.}\index{tilda@tilda ( \verb.~.)}\index{. (mezera za .)}

\begin{example}
Prof.~Smith byl šťastný, že
ji vidí\\
srv.~Obr.~5\\
Mám rád BASIC\@. Co vy?
\end{example}

Dodatečné mezery za tečkami lze zakázat pomocí příkazu
\begin{lscommand}
\ci{frenchspacing}
\end{lscommand}
\noindent který \LaTeX u řekne, aby za tečku vložil \emph{stejné množství}
mezery bez ohledu na to, jestli se nachází na
konci věty. To je běžné v~neanglických jazycích
(s~výjimkou bibliografií). Použijete-li (jednou) \ci{frenchspacing}, příkaz
\verb|\@| už pak není nutné nikdy uvádět.

%\newpage
\section{Titulky, kapitoly, sekce}

Rozdělením dokumentu do kapitol, sekcí a~podsekcí usnadníme čitateli
orientaci v~textu. \LaTeX{} pro tento účel nabízí několik příkazů,
které jako svůj argument přijímají název dané kapitoly, sekce, \ldots
\ Jen je nesmíte \uv{pomíchat}.

Následující \emph{oddílové} příkazy jsou dostupné ve třídě dokumentů
\texttt{article}: \nopagebreak

\begin{lscommand}
\ci{section}\verb|{...}|\\
\ci{subsection}\verb|{...}|\\
\ci{subsubsection}\verb|{...}|\\
\ci{paragraph}\verb|{...}|\\
\ci{subparagraph}\verb|{...}|
\end{lscommand}

Chcete-li svůj dokument rozdělit do částí aniž byste ovlivnili číslování
sekcí nebo kapitol, použijte
\begin{lscommand}
\ci{part}\verb|{...}|
\end{lscommand}

Používáte-li třídu \texttt{report} nebo \texttt{book}, máte na nejvyšší úrovni
k~dispozici ještě oddílový příkaz
\begin{lscommand}
\cialt{chapter}{cghapter}\verb|{...}|
\end{lscommand}

Protože třída \texttt{article} neví nic o~kapitolách (chapters), je
jednoduché přidat články (articles) jako kapitoly do knihy
(book). Mezery mezi sekcemi, číslování a~velikost fontu nadpisů
\LaTeX{} přizpůsobí automaticky.

Dva oddílové příkazy se trochu liší:
\begin{itemize}
\item Příkaz \ci{part} neovlivňuje sekvenci čísel kapitol.
\item Příkaz \ci{appendix} nemá argumenty, pouze změní značení
  kapitol z~čísel na písmena.\footnote{Pro styl \emph{article}
  se změní číslování sekcí.}
\end{itemize}



Obsah je \LaTeX em vytvořen pomocí informací (jmen oddílů a~čísel stran,
kde jednotlivé oddíly začínají) shromážděných při předchozím překladu
dokumentu. Příkaz
\begin{lscommand} 
\ci{tableofcontents}
\end{lscommand} 
\noindent vytvoří obsah v~místě, kde je zapsán. Aby čísla stránek
uvedená v~obsahu byla správná, je obvykle potřeba nový dokument
\LaTeX em přeložit dvakrát. Někdy je dokonce potřeba
dokument přeložit třikrát -- \LaTeX{} vás na potřebu třetího přeložení
případně sám upozorní.

Všechny oddílové příkazy zmíněné výše existují i~ve \uv{hvězdičkovaných}
verzích. Tyto verze zapíšeme tak, že za jméno příkazu uvedeme hvězdičku.
Tyto verze se liší v~tom, že nejsou uvedeny v~obsahu a~nejsou číslovány.
Ukázka hvězdičkované verzi příkazu: \verb|\section*{Help}|.

Normálně jsou názvy oddílů v~obsahu zobrazeny stejně, jako jsou
vysázeny na začátku daného oddílu. Příliš dlouhé názvy by se do obsahu
nevešly a~v~takových případech je možné pomocí nepovinného argumentu
zadat kratší text, který pro danou kapitolu bude v~obsahu vysázen.

\begin{code}
\verb|\chapter[Title for the table of contents]{A long|\\
\verb|    and especially boring title, shown in the text}|
\end{code} 

\index{titul}Titul celého dokumentu se generuje pomocí příkazu
\begin{lscommand}
\ci{maketitle}
\end{lscommand}
\noindent Text popisující titul se vkládá pomocí příkazů
\begin{lscommand}
\ci{title}\verb|{...}|, \ci{author}\verb|{...}| 
a~případně \ci{date}\verb|{...}| 
\end{lscommand}
\noindent před zavoláním \verb|\maketitle|. V~příkazu \ci{author} může být seznam jmen
oddělených příkazy \ci{and}. 
%
Na obrázku~\ref{document} najdete ukázku použití některých z~příkazů
zmíněných výše.

Kromě právě zmíněných oddílových příkazů má \LaTeXe{} tři další
příkazy ve třídě \verb|book|. Jsou užitečné pro dělení publikací.
Tyto příkazy mění názvy kapitol a~číslování stránek tak, jak
je obvyklé u~knih:

\begin{description}
\item[\ci{frontmatter}] by měl být uveden jako úplně první příkaz
  na začátku těla dokumentu (\verb|\begin{document}|). Tento
  příkaz zapne číslování stránek římskými čísly a~sekce nebudou
  číslovány, jako kdybyste používali hvězdičkované verze příkazů
  (např. \verb|\chapter*{Preface}|). V~obsahu ale sekce uvedeny
  \emph{budou}.
\item[\ci{mainmatter}] se píše před první kapitolou knihy. Zapne
  číslování stránek arabskými čísly a~stránky začne číslovat
  znovu od jedné.
\item[\ci{appendix}] značí místo, kde ve vašem dokumentu začíná
  přídavný materiál. Případné následující kapitoly budou značeny
  písmeny (a~ne čísly).
\item[\ci{backmatter}] by mělo být uvedeno před posledními položkami
  vaší knihy (jako jsou seznam literatury a~rejstřík%
%bibliografie a~index
). V~standardních třídách
  dokumentů nemá tento příkaz žádný efekt.
\end{description}


\section{Křížové odkazy}

V~knihách, zprávách a~článcích se často odkazujeme (tj. používáme
\wi{křížové odkazy}) na obrázky, tabulky a~speciální části textu.
V~\LaTeX u jsou k~dispozici tyto příkazy pro křížové odkazy:

\begin{lscommand}
\ci{label}\verb|{|\emph{marker}\verb|}|, \ci{ref}\verb|{|\emph{marker}\verb|}| 
a~\ci{pageref}\verb|{|\emph{marker}\verb|}|
\end{lscommand}
\noindent kde \emph{marker} (též značka) je identifikátor vybraný uživatelem. \LaTeX{}
nahradí \verb|\ref| číslem sekce, podsekce, obrázku, tabulky nebo věty
za kterými se vyskytuje příslušný příkaz \verb|\label|. Příkaz \verb|\pageref| vypíše
číslo stránky, na které se vyskytl příkaz \verb|\label|.%
\footnote{Tyto příkazy neví, na co odkazují. \ci{label} jen uloží naposledy
vygenerované číslo.} Použijí se čísla vygenerovaná v~předchozím překladu
dokumentu, podobně jako u~názvů oddílů.

\begin{example}
Odkaz na tuto sekci
\label{sec:this} vypadá takto:
\uv{viz sekce~\ref{sec:this}
na straně~\pageref{sec:this}.}
\end{example}
 
\section{Poznámky pod čarou}
Příkazem
\begin{lscommand}
\ci{footnote}\verb|{|\emph{footnote text}\verb|}|
\end{lscommand}
\noindent vysázíme u~paty stránky poznámku (tzv. poznámku pod čarou).
Poznámky pod čarou vkládáme za\footnote{Upozornění: \emph{ne} před.} slovo
nebo větu, ke které se vztahují. Poznámky k~větě nebo její části bychom tedy 
měli napsat za čárku nebo tečku.\footnote{Poznámky pod čarou odvádějí
čtenářovu pozornost od hlavní části dokumentu -- jsme nakonec zvědaví tvorové,
proč tedy neříct všechno v~hlavní části dokumentu?\footnotemark}
\footnotetext{Ukazatel směru nemusí sám jít tam, kam ukazuje. \texttt{:-)}}

\begin{example}
Poznámky pod čarou\footnote{Toto
  je jedna z~nich.} jsou
  u~lidí používajících
\LaTeX{} časté.
\end{example}
 
\section{Zdůrazněná slova}

Na psacím stroji důležitá slova \texttt{zvýrazňujeme jejich \underline{podtržením}}.
\begin{lscommand}
\ci{underline}\verb|{|\emph{text}\verb|}|
\end{lscommand}
V~knihách se ale slova zvýrazňují vysázením pomocí \emph{italického} fontu.
\LaTeX{} ke zvýraznění textu poskytuje příkaz
\begin{lscommand}
\ci{emph}\verb|{|\emph{text}\verb|}|
\end{lscommand}
\noindent Záleží na kontextu, co tento příkaz se svým argumentem udělá:

\begin{example}
\emph{Budete-li
  zvýrazňovat uvnitř
  zvýrazňovaného textu,
  \LaTeX{} pro toto
  zvýraznění použije
  \emph{normální} font}
\end{example}

Všimněte si rozdílu mezi \emph{zvýrazněním} a~změnou \emph{fontu}:

\begin{example}
\textit{Máte možnost
  \emph{zvýraznit} text
  sázený pomocí italiky,} 
\textsf{pomocí
  \emph{bezserifového} fontu,}
\texttt{nebo pomocí
  stylu \emph{psací stroj}.}
\end{example}

\section{Prostředí} \label{env}

% To typeset special purpose text, \LaTeX{} defines many different
% \wi{environment}s for all sorts of formatting:
\begin{lscommand}
\ci{begin}\verb|{|\emph{environment}\verb|}|\quad
   \emph{text}\quad
\ci{end}\verb|{|\emph{environment}\verb|}|
\end{lscommand}
\noindent Kde \emph{environment} je jméno prostředí. Prostředí mohou být
vnořená, ale nesmí být \uv{pomíchaná}.
\begin{code}
\verb|\begin{aaa}...\begin{bbb}...\end{bbb}...\end{aaa}|
\end{code}

\noindent V~následujících podsekcích vysvětlíme všechna důležitá prostředí.

\subsection{Itemize, Enumerate a Description}

Prostředí \ei{itemize} je užitečné pro jednoduché seznamy (také se říká výčty), prostředí \ei{enumerate} pro
číslované seznamy a~\ei{description} pro popisy.
\cih{item}

\begin{example}
\flushleft
\begin{enumerate}
\item Seznamová prostředí můžete
míchat, jak se vám zlíbí:
\begin{itemize}
\item Ale výsledek nemusí
vypadat dobře.
\item[--] S~pomlčkou.
\end{itemize}
\item Proto vezměte v~úvahu:
\begin{description}
\item[Hloupé] věci
nezlepšíme tím, že je
dáme do seznamu.
\item[Chytré] věci ale
uvedením v~seznamu mohou získat.
\end{description}
\end{enumerate}
\end{example}
 
\subsection{Flushleft, Flushright a Center}

Prostředí \ei{flushleft} a~\ei{flushright} generují
odstavce s~řádky zarovnanými vlevo, resp. vpravo.\index{zarovnané
doleva} Prostředí \ei{center} generuje text s~vycentrovanými řádky.
Pokud nespecifikujete řádkový zlom sami, pomocí \ci{\bs}, \LaTeX{}
řádky zalomí automaticky.

\begin{example}
\begin{flushleft}
Tento text je\\ zarovnaný
doleva. \LaTeX{} se nesnaží
zajistit, aby každý řádek
byl stejně široký.
\end{flushleft}
\end{example}

\begin{example}
\begin{flushright}
Tento text je
zarovnaný\\doprava. 
\LaTeX{} se nesnaží zajistit,
aby každý řádek byl stejně
široký.
\end{flushright}
\end{example}

\begin{example}
\begin{center}
V~centru\\země
\end{center}
\end{example}

\subsection{Citace, citáty a~verše}

Prostředí \ei{quote} je užitečné pro citace, důležité fráze a~příklady.

\begin{example}
Obecné pravidlo pro
délku řádku je:
\begin{quote}
V~průměru by žádný řádek
neměl být delší než 66~znaků.
\end{quote}
Proto mají \LaTeX ové stránky
implicitně tak široké okraje
a~proto se v~novinách tiskne
do sloupců.
\end{example}

Existují dvě podobná prostředí: \ei{quotation} a~\ei{verse}.
Prostředí \texttt{quotation} je užitečné pro delší citace (přes několik odstavců),
protože odsadí první řádku každého z~těchto odstavců. Prostředí \texttt{verse} je
užitečné pro básně (jejichž rozdělení do řádek se nedá dělat automaticky).
Jejich řádky jsou odděleny pomocí \ci{\bs} na konci řádky a~prázdné
řádky oddělují sloky.

\begin{example}
Nazpaměť znám pouze jednu
anglickou báseň.
Je o~Humpty Dumptym.
\begin{flushleft}
\begin{verse}
Humpty Dumpty sat on a~wall:\\
Humpty Dumpty had a~great
   fall.\\ 
All the King's horses and all
the King's men\\
Couldn't put Humpty together
again.
\end{verse}
\end{flushleft}
\end{example}

\subsection{Abstrakty}

Vědecké publikace obvykle začínají abstraktem -- rychlým přehledem obsahu.
\LaTeX{} pro tento účel poskytuje prostředí \ei{abstract}, které
obvykle používáme v~dokumentech sázených pomocí třídy \texttt{article}.

\newenvironment{abstract}%
        {\begin{center}\begin{small}\begin{minipage}{0.8\textwidth}}%
        {\end{minipage}\end{small}\end{center}}
\begin{example}
\begin{abstract}
Abstraktní abstrakt.
\end{abstract}
\end{example}

\subsection{Sazba doslovně}

Text uzavřený mezi \verb|\begin{|\ei{verbatim}\verb|}|
a~\verb|\end{verbatim}| bude vysázen tak, jak je, jako kdyby byl napsán
na psacím stroji. Všechny řádkové zlomy a~mezery budou zachovány
a~jména \LaTeX ových příkazů budou vysázena, místo toho, aby se příslušné
příkazy vykonaly.

Vprostřed normálního textu můžeme podobného chování docílit pomocí
\begin{lscommand}
\ci{verb}\verb|+|\emph{text}\verb|+|
\end{lscommand}
\noindent \verb|+| je jedním z~možných oddělovačů. Můžete použít
libovolný znak kromě písmen, \verb|*| a~mezery. Mnoho příkladů v~této
knize je sázeno pomocí tohoto příkazu.

\begin{example}
Příkaz \verb|\ldots| \ldots

\begin{verbatim}
10 PRINT "HELLO WORLD ";
20 GOTO 10
\end{verbatim}
\end{example}

\begin{example}
\begin{verbatim*}
hvězdičkovaná verze
doslovného     prostředí
zdůrazňuje    mezery
v~textu
\end{verbatim*}
\end{example}

Příkaz \ci{verb} lze s~hvězdičkou použít také:

\begin{example}
\verb*|like   this :-) |
\end{example}

Prostředí \texttt{verbatim} a~příkaz \verb|\verb| někdy nejdou použít
uvnitř parametru příkazu.

 
\subsection{Tabular}

\newcommand{\mfr}[1]{\framebox{\rule{0pt}{0.7em}\texttt{#1}}}

Pomocí prostředí \ei{tabular} lze sázet nádherné \index{tabulka}tabulky
s~případnými horizontálními a~vertikálními linkami. Šířku sloupců
\LaTeX{} určí automaticky.

Argument \emph{table spec} příkazu
\begin{lscommand}
\verb|\begin{tabular}[|\emph{pos}\verb|]{|\emph{table spec}\verb|}|
\end{lscommand} 
\noindent definuje formát tabulky. Sloupec tabulky zarovnaný vlevo
se specifikuje pomocí \mfr{l}, sloupec zarovnaný vpravo pomocí \mfr{r}
a~vycentrovaný sloupec pomocí \mfr{c}; sloupec obsahující zarovnaný
text (výsledek automatického řádkového zlomu \LaTeX u) pomocí
\mfr{p\{\emph{width}\}} a~vertikální linka pomocí \mfr{|}.

Je-li text pro daný sloupec příliš široký, \LaTeX{} ho
\emph{nebude} automaticky lámat. Pro přizpůsobení textu šířce sloupce
je potřeba použít \mfr{p\{\emph{width}\}}, kde se pak s~textem
bude zacházet stejně, jako v~normálním odstavci.

Argument \emph{pos} specifikuje vertikální pozici tabulky vzhledem
k~účaří okolního textu. Můžete použít některý ze znaků \mfr{t}, \mfr{b}
a~\mfr{c} (pro zarovnání horní části, resp. spodní části, resp. středu
tabulky).
 
Uvnitř prostředí \texttt{tabular} znamená znak \texttt{\&} oddělovač
sloupců, \ci{\bs} začíná novou řádku a~\ci{hline} vloží horizontální
linku.  Neúplné linky můžeme vložit pomocí
\ci{cline}\texttt{\{}\emph{i}\texttt{-}\emph{j}\texttt{\}},
kde $i$~a~$j$~jsou čísla sloupců, mezi kterými by linka měla být natažena.
Při užití \texttt{czech} v~\textsf{babel}u vypneme navíc lokálně či globálně aktivní divis: \verb+\shorthandoff{-}+.

\index{"|@{\me\char'174}}

\begin{example}
\begin{tabular}{|r|l|}
\hline
7C0 & šestnáctkově \\
3700 & osmičkově \\ \cline{2-2}
11111000000 & binárně \\
\hline \hline
1984 & desítkově \\
\hline
\end{tabular}
\end{example}

\begin{example}
\begin{tabular}{|p{4.7cm}|}
\hline
Menší demonstrace odstavce
uvnitř tabulky. Výsledek snad
bude vypadat přijatelně.\\
\hline 
\end{tabular}
\end{example}

Oddělovač sloupců lze specifikovat pomocí konstrukce \mfr{@\{...\}}.
Tento příkaz potlačí veškeré mezisloupcové mezery a~nahradí je
materiálem uvedeným mezi složenými závorkami. Jedno běžné použití
tohoto příkazu je vysvětleno níže (zarovnání desetinných čísel).
Jinou možnou aplikací je potlačení úvodních mezer v~tabulce pomocí
\mfr{@\{\}}.

\begin{example}
\begin{tabular}{@{} l @{}}
\hline 
žádné úvodní mezery\\
\hline
\end{tabular}
\end{example}

\begin{example}
\begin{tabular}{l}
\hline
úvodní a~závěrečné mezery\\
\hline
\end{tabular}
\end{example}

%
% This part by Mike Ressler
%

\index{zarovnání desetinných čísel} V~\LaTeX u není vestavěna
podpora pro zarovnávání desetinných čísel\footnote{Pokud ale máte nainstalovánu
kolekci \textsf{tools}, podívejte se na balík \pai{dcolumn}.},
jednou z~možností, jak čísla zarovnat je použít tabulku se dvěma sloupci:
vpravo zarovnaná celá část desetinného čísla a~vlevo zarovnaná příslušná
desetinná část. Příkaz \verb|@{,}|
ve specifikaci sloupců \verb|\begin{tabular}| nahradí normální mezisloupcové
mezery čárkou a~tabulka bude vypadat jako jeden sloupec čísel zarovnaných
na desetinnou čárku. Nezapomeňte oddělit celou a~desetinnou část jednotlivých
čísel místo desetinou čárkou oddělovačem sloupců (\verb|&|)! Název sloupce umístíme
nad náš číselný \uv{sloupec} pomocí příkazu \ci{multicolumn}. Více o~možnostech tabulkové sazby balíček \textsf{tabu}.
 
\begin{example}
\begin{tabular}{c r @{,} l}
výraz obsahující Pi &
\multicolumn{2}{c}{Hodnota} \\
\hline
$\pi$               & 3&1416  \\
$\pi^{\pi}$         & 36&46   \\
$(\pi^{\pi})^{\pi}$ & 80662&7 \\
\end{tabular}
\end{example}

\begin{example}
\begin{tabular}{|c|c|}
\hline
\multicolumn{2}{|c|}{Ene} \\
\hline
Mene & Muh! \\
\hline
\end{tabular}
\end{example}

Materiál vysázený pomocí prostředí tabular vždy zůstane pohromadě na jedné
stránce. Chcete-li sázet vícestránkové tabulky, je třeba použít prostředí
\pai{longtable}.

\LaTeX ové tabulky mohou někdy vypadat trochu \uv{zhuštěně}. \uv{Rozvolnění}
lze provést nastavením vyšších hodnot \ci{arraystretch} a~\ci{tabcolsep}.

\begin{example}
\begin{tabular}{|l|}
\hline
Tyto řádky\\\hline
jsou namačkané\\\hline
\end{tabular}

{\renewcommand{\arraystretch}
              {1.5}
\renewcommand{\tabcolsep}{0.2cm}
\begin{tabular}{|l|}
\hline
méně namačkané\\\hline
rozložení tabulky\\\hline
\end{tabular}}

\end{example}

Pokud potřebujete zvětšit výšku jen jedné řádky tabulky, přidejte neviditelnou
vertikální podpěru\footnote{V~profesionální sazbě se jí často říká \wi{strut}.}
pomocí linky s~nulovou šířkou.

\begin{example}
\begin{tabular}{|c|}
\hline
\rule{1pt}{4ex}Pitprop \ldots\\
\hline
\rule{0pt}{4ex}Strut\\
\hline
\end{tabular}
\end{example}

\section{Plovoucí objekty}
Dnešní publikace často obsahují hodně obrázků a~tabulek. S~těmi se
musí zacházet zvláštním způsobem, protože na rozdíl od normálního
textu nemohou být vždy rozděleny na více stránek. Jednoduché ale
neuspokojivé řešení je začít novou stránku vždy, když je potřeba
vysázet obrázek nebo tabulku, které se nevlezou na aktuální stránku.
Problém tohoto přístupu je, že části stránek pak zůstávají prázdné,
což špatně vypadá.

Uspokojivým řešením je nechat obrázky a~tabulky, které se nevlezou
na aktuální stránku, \uv{doplavat} na některou z~následujících
stránek, přičemž aktuální stránka se zaplní textem, který ve vstupním
souboru následuje za daným obrázkem nebo tabulkou. \LaTeX{}
pro \emph{plovoucí objekty} nabízí dvě prostředí -- jedno pro tabulky
a~jedno pro obrázky. Abychom mohli plně využít možností těchto dvou
prostředí, je důležité, abychom zhruba věděli, jak \LaTeX{}
s~plovoucími objekty vnitřně zachází. Jinak bychom měli problémy
odhadnout, kam \LaTeX{} konkrétní plovoucí objekty umístí.


\bigskip
Začneme náš výklad s~\LaTeX ovými příkazy, které jsou pro plovoucí objekty
k~dispozici.

S~jakýmkoliv materiálem uvedeným v~prostředí \ei{figure} nebo \ei{table}
se bude zacházet jako s~plovoucím objektem. Obě prostředí pro
plovoucí objekty poskytují nepovinný parametr
\begin{lscommand}
\verb|\begin{figure}[|\emph{specifikace umístění}\verb|]| or
\verb|\begin{table}[|\ldots\verb|]|
\end{lscommand}
\noindent nazvaný \emph{specifikace umístění}. Tímto parametrem
můžeme \LaTeX u říct o~místech, do kterých smí daný plovoucí objekt
nechat doplavat. \emph{Specifikace umís\-tění} se skládá z~řetězce
\emph{povolenek pro umístění plovoucího objektu}. Viz tabulka~\ref{tab:permiss}.

\begin{table}[!bp]
\caption{Povolenky pro umístění plovoucího objektu.}\label{tab:permiss}
\noindent \begin{minipage}{\textwidth}
\medskip
\begin{center}
\begin{tabular}{@{}cp{8cm}@{}}
Pozice&Povolení umístit plovoucí objekt\ldots\\
\hline
\rule{0pt}{1.05em}\texttt{h} & \emph{Zde} (here) přesně na místo, kde byl daný
  text uveden ve vstupním souboru. Toto je užitečné zejména při použití malých fontů.\\[0.3ex]
\texttt{t} & Na \emph{vrcholu} (top) strany.\\[0.3ex]
\texttt{b} & Na \emph{spodu} (bottom) strany.\\[0.3ex]
\texttt{p} & Na zvláštní \emph{stránce} (page), která bude obsahovat jen plovoucí objekty.\\[0.3ex]
\texttt{!} & Bez braní v~úvahu většiny vnitřních parametrů\footnote{Např. maximální
  počet plovoucích objektů, který může být umístěn na jednu stranu.}, které by
  mohly zabránit v~umístění tohoto plovoucího objektu.
\end{tabular}
\end{center}
Poznámka: \texttt{pt} a~\texttt{em} jsou \TeX ové jednotky. Více informací je uvedeno
v~tabulce~\ref{units} na straně \pageref{units}.
\end{minipage}
\end{table}

Tabulku můžeme začít např. následujícím řádkem:
\begin{code}
\verb|\begin{table}[!hbp]|
\end{code}
\noindent \index{povolenka umístění}Povolenka umístění \verb|[!hbp]| říká \LaTeX u, že může
tabulku umístit buď přímo v~aktuálním místě (\texttt{h}), nebo naspodu nějaké
strany (\texttt{b}), nebo na zvláštní stránce obsahující jen plovoucí objekty
(\texttt{p}) -- to vše i~tehdy, když výsledek nebude vypadat nejlépe (viz \texttt{!}).
Neuvedeme-li žádnou povolenku, \LaTeX{} se chová, jako bychom uvedli \verb|[tbp]|.

\LaTeX{} umístí každý plovoucí objekt, na který při
zpracování dokumentu narazí, podle povolenky umístění zadané autorem.
Nejde-li objekt přidat do aktuálního místa na stránce, je 
přidán buď do fronty \emph{obrázků} nebo do fronty \emph{tabulek}%
\footnote{Fronty jsou klasické FIFO struktury -- \uv{Kdo první vejde,
první odejde.}}. Když zahájí novou stránku, \LaTeX{} nejdřív zkontroluje,
jestli je v~některé frontě \uv{na řadě} objekt, který má povoleno
být umístěn na zvláštní stránce (která obsahuje jen plovoucí objekty).
Jestliže takový objekt přítomen není, první objekt v~každé frontě
je uvažován jako kdyby se právě objevil ve vstupním souboru: \LaTeX{}
znovu zkouší umístit tento objekt podle příslušných povolenek
(kromě povolenky ,h`, která se už u~objektu ve frontě vyskytnout nemůže).
Jakýkoliv nový plovoucí objekt, který se objeví v~textu, je umístěn
do příslušné fronty. \LaTeX{} přidává plovoucí objekty do výstupu
přesně v~tom pořadí, v~kterém na ně ve vstupním
souboru narazí. Z~toho vyplývá, že obrázek, který se nepodaří umístit
na aktuální stranu, \uv{zablokuje} i~ostatní obrázky (protože ty nemohou být
umístěny dřív, než obrázek, který je ve frontě před nimi).

Proto:

\begin{quote}
Jestliže \LaTeX{} neumístí plovoucí objekt tak, jak jste očekávali,
často je to proto, že jeden z~plovoucích objektů zablokoval
jednu z~front plovoucích objektů.
\end{quote}                 

\LaTeX{} sice umožňuje zadat povolenku umístění o~jedné položce, ale
ta způsobuje problémy. Jestliže plovoucí objekt nelze pomocí dané
položky umístit, jsou on i~všechny následující zablokovány.
Obzvlášť \emph{nevhodná} je položka [h] -- je to tak špatná možnost,
že v~novějších verzích \LaTeX u je dokonce automaticky změněna na [ht].

\bigskip
\noindent Tím jsme probrali to obtížné. Nyní se zmíníme o~několika
věcech souvisejících s~prostředími \ei{table} a~\ei{figure}.
Příkazem

\begin{lscommand}
\ci{caption}\verb|{|\emph{caption text}\verb|}|
\end{lscommand}

\noindent můžeme definovat titulek daného plovoucího objektu. \LaTeX{}
připojí pořadové číslo a~řetězec \uv{Obrázek} nebo \uv{Tabulka}.

Příkazy

\begin{lscommand}
\ci{listoffigures} a~\ci{listoftables} 
\end{lscommand}

\noindent pracují podobně jako příkaz \verb|\tableofcontents| --
vysází seznam obrázků, resp. tabulek. Tyto seznamy zobrazí celý
text uvedený jako titulek jednotlivých objektů. Pokud tedy
používáte dlouhé titulky, měli byste zadat i~jejich kratší verze,
Krátkou verzi můžete uvést do hranatých závorek za příkazem
\verb|\caption|.
\begin{code}
\verb|\caption[Short]{LLLLLoooooonnnnnggggg}| 
\end{code}

Daný plovoucí objekt \uv{identifikujete} pomocí příkazu \ci{label}
a~můžete se pak na něj ve svém dokumentu odkazovat pomocí příkazu
\ci{ref}. Příkaz \ci{label} musí být uveden \emph{za} příkazem \ci{caption},
aby se vztahoval na daný plovoucí objekt.

Následující příklad nakreslí čtverec a~vloží ho do dokumentu. Takto
můžete v~dokumentu rezervovat místo pro obrázky, které budete později
do hotového dokumentu vkládat.

\begin{code}
\begin{verbatim}
Obrázek~\ref{white} je příkladem pop artu.
\begin{figure}[!hbtp]
\makebox[\textwidth]{\framebox[5cm]{\rule{0pt}{5cm}}}
\caption{Pět krát pět centimetrů.\label{white}}
\end{figure}
\end{verbatim}
\end{code}

\noindent V~předchozím příkladu \LaTeX{} udělá \emph{všechno možné}~(viz \texttt{!})
pro to, aby obrázek umístil \emph{v~aktuálním místě}~(\texttt{h}).%
\footnote{Za předpokladu, že fronta obrázků je prázdná.} Pokud se to ani tak nepodaří,
\LaTeX{} zkusí obrázek umístit \emph{naspod}~(\texttt{b}) strany.
Pokud se nepodaří ani to, \LaTeX{} zkusí obrázek umístit
\emph{nahoru}~(\texttt{t}) na aktuální straně. Nepodaří-li se ani to,
\LaTeX{} zjistí, jestli je možné vytvořit \uv{plovoucí stránku}
obsahující tento obrázek a~případně některé z~tabulek ve frontě
tabulek. Není-li pro speciální stránku obsahující jen plovoucí
objekty dost těchto objektů k~dispozici, \LaTeX{} zahájí novou stranu a~opět zachází
s~obrázkem stejným způsobem, jako kdyby se právě objevil v~textu.

Někdy je nutné použít příkaz

\begin{lscommand}
\ci{clearpage} nebo dokonce \ci{cleardoublepage} 
\end{lscommand}

\noindent Tento příkaz říká \LaTeX u, aby hned umístil všechny plovoucí
objekty přítomné ve frontách a~potom začal novou stránku.
Příkaz \ci{cleardoublepage} udělá to samé, ale navíc přejde na novou pravou stranu (nejbližší lichou).

Později se v~tomto dokumentu dočtete o~tom, jak do svého dokumentu \LaTeXe{}
vložit kresby ve formátu \PSi{}.

\section{Chránění \uv{zranitelných} příkazů}

Text, který zadáme jako argument příkazů typu \ci{caption} nebo \ci{section},
se může v~dokumentu objevit na více místech (např. v~obsahu i~v~těle
dokumentu). Pokud některé příkazy uvedeme jako argumenty příkazů
typu \ci{section}, zpracování našeho dokumentu selže. Těmto příkazům,
které jako parametry selžou, říkám \wi{zranitelné příkazy}. Patří
mezi ně např. \ci{footnote} nebo \ci{phantom}. Pokud ale zranitelné příkazy
\uv{ochráníme} tím, že před nimi uvedeme příkaz \ci{protect},
jako argumenty příkazů typu \ci{section} je už použít můžeme.

Příkaz \ci{protect} se vztahuje jen na příkaz, který za ním bezprostředně následuje,
ale ne na jeho argumenty. Ve většině případu neuškodí, když \ci{protect}
uvedeme navíc.

\begin{code}
\verb|\section{I am considerate|\\
\verb|      \protect\footnote{and protect my footnotes}}|
\end{code}

% Local Variables:
% TeX-master: "lshort2e"
% mode: latex
% mode: flyspell
% End:
