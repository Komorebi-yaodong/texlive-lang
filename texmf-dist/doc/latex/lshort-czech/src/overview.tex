%%%%%%%%%%%%%%%%%%%%%%%%%%%%%%%%%%%%%%%%%%%%%%%%%%%%%%%%%%%%%%%%%
% Contents: Who contributed to this Document
% $Id: overview.tex 169 2008-09-24 07:32:13Z oetiker $
%%%%%%%%%%%%%%%%%%%%%%%%%%%%%%%%%%%%%%%%%%%%%%%%%%%%%%%%%%%%%%%%%

% Because this introduction is the reader's first impression, I have
% edited very heavily to try to clarify and economize the language.
% I hope you do not mind! I always try to ask "is this word needed?"
% in my own writing but I don't want to impose my style on you... 
% but here I think it may be more important than the rest of the book.
% --baron

\chapter{Předmluva}

\LaTeX{} \cite{manual} je sázecí systém, který se výborně hodí na
tvorbu vědeckých a~matematických dokumentů vysoké typografické kvality.
\LaTeX{} se ale hodí i~na tvorbu všech možných dalších typů dokumentů,
od jednoduchých dopisů po knihy. \LaTeX{}
používá \TeX{} \cite{texbook} jako svůj formátovací stroj.

Tento krátký úvod popisuje \LaTeXe{} a~měl by pokrývat většinu aplikací
\LaTeX u. Úplný popis systému \LaTeX{} naleznete v~\cite{manual,companion}.

\bigskip
\noindent Tento úvod je rozdělen do šesti kapitol:
\begin{description}
\item[Kapitola 1] vás seznámí se základní strukturou dokumentů v~\LaTeXe{}.
  Dozvíte se také něco o~historii \LaTeX u. Po přečtení této kapitoly
  byste měli zhruba rozumět tomu, jak \LaTeX{} funguje.
\item[Kapitola 2] popisuje detaily sazby vašeho dokumentu.
  Vysvětluje většinu základních \LaTeX ových příkazů a~prostředí.
  Po přečtení této kapitoly budete schopni napsat své první dokumenty.
\item[Kapitola 3] vysvětluje, jak se v~\LaTeX u sází matematické vzorce.
  Řada příkladů ukazuje, jak se tato jedna z~největších předností \LaTeX u
  používá. Na konci kapitoly jsou tabulky všech matematických symbolů
  dostupných v~\LaTeX u.
\item[Kapitola 4] vysvětluje indexy, tvorbu bibliografií a~vkládání EPS
  grafiky. Dále představuje tvorbu PDF dokumentů pomocí pdf\LaTeX u
  a~některé užitečné balíky.
\item[Kapitola 5] ukazuje, jak lze pomocí \LaTeX u vytvářet grafiku
   jinak, než \uv{normálním} způsobem (místo toho, abyste obrázek nakreslili
   v~některém z~grafických programů, uložili do externího souboru a~vložili
   do \LaTeX ového dokumentu, vložíte do souboru s~textem \LaTeX ového dokumentu
   jeho popis a~necháte \LaTeX, aby vám podle něj obrázek nakreslil).
\item[Kapitola 6] obsahuje potenciálně nebezpečné informace týkající
  se toho, jak lze změnit standardní vzhled \LaTeX ového dokumentu.
  Přečtete si o~tom, jak udělat změny, které krásný vzhled \LaTeX ových
  dokumentů mohou zlepšit ještě víc, ale mohou ho také úplně pokazit.
\end{description}
\bigskip
\noindent Kapitoly je potřeba číst od první do poslední -- dokument
nakonec není tak rozsáhlý. Mnoho informací je obsaženo v~příkladech
a~bude proto nejlepší, když si je důkladně projdete.

\bigskip
\noindent \LaTeX{} je dostupný na většině počítačů, od PC a~Maců po
velké systémy typu UNIX a~VMS. Na počítačích mnoha univerzitních sítí
je \LaTeX{} nainstalován a~připraven k~použití. Informace o~tom, jak
získat přístup k~lokální \LaTeX ové instalaci, by měla být dostupná
v~\guide. Budete-li mít problémy s~prvními kroky, obraťte se
na osobu, od které jste tuto brožurku získali. Tento dokument vás
\emph{nenaučí}, jak nainstalovat a~zprovoznit systém \LaTeX, jeho
cílem je naučit vás, jak napsat dokumenty, které bude možno \LaTeX em
zpracovat.

\bigskip
\noindent Hledáte-li jakékoliv materiály týkající se \LaTeX u,
navštivte jedno z~míst, kde je uložen Comprehensive \TeX{}
Archive Network (\texttt{CTAN} -- Síť obsáhlých \TeX ových archivů).
Začít můžete na adrese \texttt{http://www.ctan.org}. Všechny balíky
lze také získat z~ftp archivu \texttt{ftp://www.ctan.org} a~jeho
\uv{zrcadel} umístěných po celém světě.

V~dokumentu budete narážet na odkazy do CTANu, zvláště na software
a~dokumenty, které si můžete stáhnout. Místo uvádění kompletních
URL vždy píšu jen \texttt{CTAN:} následované lokací uvnitř stromu CTANu,
kam se vydat.

Chcete-li \LaTeX{} používat na vašem vlastním počítači, poohlédněte se,
co je dostupné na \CTANalt|tex-archive/systems|.

\vspace{\stretch{1}}
\noindent Máte-li nápady, co by se do dokumentu mohlo přidat, odebrat
nebo změnit, dejte mi prosím vědět. Zvláště by mě zajímaly reakce od
\LaTeX ových začátečníků -- které části tohoto dokumentu jsou dobře
pochopitelné a~které by mohly být napsány lépe.

\bigskip
\begin{verse}
\contrib{Tobias Oetiker}{tobi@oetiker.ch}%
\noindent{OETIKER+PARTNER AG\\Aarweg 15\\4600 Olten\\Switzerland}
\end{verse}
\vspace{\stretch{1}}
\noindent Nejnovější verze tohoto dokumentu je dostupná na (český překlad):\\
\CTAN|lshort|

\endinput



%

% Local Variables:
% TeX-master: "lshort2e"
% mode: latex
% mode: flyspell
% End:
