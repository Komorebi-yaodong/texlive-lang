%\iffalse
% ecv.dtx generated using makedtx version 0.91b (c) Nicola Talbot
% Command line args:
%   -macrocode ".*"
%   -src "(ecv.cls|ecvNLS.sty|ecvGerman.ldf|ecvEnglish.ldf)=>\1"
%   -author "Christoph P. Neumann & Bernd Haberstumpf"
%   -dir "src"
%   -usedir "tex/latex/ecv"
%   -setambles ".*=>\nopreamble"
%   -doc "doc/ecv.tex"
%   -silent "1"
%   ecv
% Created on 2009/8/26 0:15
%\fi
%\iffalse
%<*package>
%% \CharacterTable
%%  {Upper-case    \A\B\C\D\E\F\G\H\I\J\K\L\M\N\O\P\Q\R\S\T\U\V\W\X\Y\Z
%%   Lower-case    \a\b\c\d\e\f\g\h\i\j\k\l\m\n\o\p\q\r\s\t\u\v\w\x\y\z
%%   Digits        \0\1\2\3\4\5\6\7\8\9
%%   Exclamation   \!     Double quote  \"     Hash (number) \#
%%   Dollar        \$     Percent       \%     Ampersand     \&
%%   Acute accent  \'     Left paren    \(     Right paren   \)
%%   Asterisk      \*     Plus          \+     Comma         \,
%%   Minus         \-     Point         \.     Solidus       \/
%%   Colon         \:     Semicolon     \;     Less than     \<
%%   Equals        \=     Greater than  \>     Question mark \?
%%   Commercial at \@     Left bracket  \[     Backslash     \\
%%   Right bracket \]     Circumflex    \^     Underscore    \_
%%   Grave accent  \`     Left brace    \{     Vertical bar  \|
%%   Right brace   \}     Tilde         \~}
%</package>
%\fi
% \iffalse
% Doc-Source file to use with LaTeX2e
% Copyright (C) 2009 Christoph P. Neumann & Bernd Haberstumpf, all rights reserved.
% \fi
% \iffalse
%<*driver>
%%
%% Copyright 2007 Bernd Haberstumpf
%%
%% Diese \LaTeX-Klasse bietet eine einfache Schnittstelle, um einen
%% originellen Lebenslauf zu kreieren. Momentan werden Lebensläufe nur
%% in der deutschen Sprache unterstützt.
%%
%% Diese Datei ist freies Eigentum; als spezielle Ausnahme gibt der Autor
%% unbegrenzte Erlaubnis sie zu kopieren und/oder mit oder ohne Modifikationen
%% zu verteilen, so lang diese Notiz erhalten bleibt. 
%% 
%%
%% Diese Datei wird in der Hoffnung verteilt, dass sie nützlich sein wird, 
%% aber OHNE JEGLICHE GARANTIE zur Ausdehnung der gesetzlichen Erlaubnis; 
%% ohne implizierte Garantie der Verkaufsfähigkeit oder EIGNUNG FÜR EINEN BESTIMMTEN ZWECK.
%% 
%%

\documentclass{ltxdoc} 
\usepackage[T1]{fontenc}
\usepackage[utf8]{inputenc}

\usepackage[ngerman]{babel}

\CodelineNumbered
\EnableCrossrefs 
%\DisableCrossrefs
\CodelineIndex 
%\PageIndex
\RecordChanges
%\OnlyDescription
\GetFileInfo{ecv.cls}

\parskip1.0ex
\parindent0.0ex

\begin{document} 
\DocInput{ecvDE.dtx}
\end{document}
%</driver>
%\fi
%
%\title{\textsf{ecv}\\
%Eine originelle Lebenslauf-Klasse} 
%\author{Christoph P.\ Neumann \texttt{$<$c.p.neumann+ecv@gmail.com$>$}, \\
%Bernd Haberstumpf \texttt{$<$poldi@kabatrinker.de$>$}\footnote{Übersetzt
%ins Deutsche von \textbf{Katrin Vogel} (FSU Jena), September 2011.}} 
%\maketitle 
%\PrintChanges
%
%\begin{abstract}
%Die \texttt{ecv}--Klasse bietet eine angenehme Umgebung, um einen originellen Lebenslauf zu kreieren. 
%Sie orientiert sich dabei am Europass. 
%(siehe: \texttt{http://europass.cedefop.eu.int}).
%\end{abstract}
%
%\section{Installation}
%
%Die \texttt{zip}- oder \texttt{tar.gz}-Datei kommt mit einem \texttt{ecv.ins}
%und einer \texttt{ecv.dtx} Datei, welche das \LaTeX\-Material beinhaltet.
%
%Um die Klassen-Dateien zu extrahieren, rufen Sie sie mit folgendem Befehl auf:
%
%\begin{verbatim}
%  $ latex ecv.ins
%\end{verbatim}
%
%Dieser Aufruf extrahiert alle \LaTeX\ spezifischen Dateien in das aktuelle
%Verzeichnis. Sie können die Dateien entweder für ein einzelnes 
%Lebenslauf-Projekt nutzen oder Sie können Ihre Dateien in Ihre \TeX\-Installation integrieren.
%
%
%Wenn Sie die \texttt{ecv} in Ihre \TeX-Installation integrieren wollen,
%erstellen Sie ein
%Verzeichnis \texttt{tex/latex/ecv} neben Ihrer \TeX-Installation (z.\,B.
%neben \texttt{/usr/share/texmf}\footnote{Bezieht sich auf Linux-Systeme.})
%und kopieren Sie alle Dateien des aktuellen Verzeichnisses dorthin. Rufen Sie nun folgendes auf:
%
%\begin{verbatim}
%  $ mktexlsr
%\end{verbatim}
%
%um den Datei-Zwischenspeicher von \LaTeX\ zu aktualisieren.
%
%
%Hinweis: Der \texttt{ecv-Verlauf} beinhaltet eine Auswahl von
%\texttt{docstrip}-Konfigurationen in \texttt{docstrip.cfg}, 
%mit denen die Dateien automatisch zu ihren korrekten Positionen innerhalb einer \LaTeX-Installation 
%zugeteilt werden. Sie können diese Datei gern an Ihre eigene Umgebung angleichen und hinterher
% \texttt{latex ecv.ins} aufrufen, um das Paket an seinem richtigen Platz zu installieren.
%
%\section{Linux und Windows}
%
%\textsf{Ecv} wurde auf Linux und Windows mit Hilfe von 
%MiXTeX und TeXnicCenter getestet.
%
%\section{Vorlagen}
%
%Für einen schnellen Start beinhaltet die \textsf{ecv}-Distribution Dokumentvorlagen
% für einen deutschen und einen englischen Lebenslauf. Die Vorlagen können in der Datei 
%\texttt{template.zip} gefunden werden.
%
%Das \texttt{template}-Verzeichnis beinhaltet die Vorlagen.
%
%\begin{itemize}
%\item \texttt{CV-template\_de.tex} für die deutsche Sprache,
%\item \texttt{CV-template\_en.tex} für die englische Sprache.
%\end{itemize}
%
%und ein \texttt{Makefile}, um die pdf-Datei zu erstellen. Rufen Sie auf:
%
%\begin{verbatim}
%  $ make
%\end{verbatim}
%
%um die pdf-Datei zu erstellen. Die Datei \texttt{porttrait.eps} beinhaltet ein Attrappen-Porträt
%für die erste Seite des Lebenslaufes.
%
%\section{Struktur}
%
%Die \texttt{tex}—Datei, die den Lebenslauf beinhaltet, wird ungefähr folgende Struktur haben:
%
%
%\begin{verbatim}
%\% Die englische und die deutsche Sprache werden unterstützt.
%\documentclass[german]{ecv}
%
%\% Porträt, das auf der ersten Seite genutzt wird.
%\ecvPortrait{images/myPortrait}
%
%\% Name, der auf der Fußzeile genutzt wird.
%\ecvName{Mein Name}
%
%\begin{document}
%
%\% Startet die Tabelle, welche den Lebenslauf beinhaltet (Das wird als Titel abgedruckt 
%\% und als Portrait)
%\begin{ecv}
%
%\% Gruppeneinträge mit Abschnitten
%\ecvSec{\ecvPerson}
%
%\% Einträge in der Tabelle
%\ecvEPR{Name}   {\textsc{Name}, Mein}
%\ecvEPR{Adresse}{Irgendwo 13, Musterstadt}
%\ecvEPR{Telefon}{(555) 555 / 555}
%\ecvEPR{E-Mail} {\ecvHyperEMail{nobody@nowhere.com}}
%\ecvEPR{Staatsangeh"origkeit}
%                {Deutsch}
%\ecvEPR{Geburtsdatum}
%                {12.34.5678}
%\end{ecv}
%
%\ecvSig{Name, Mein}{Irgendwo}
%
%\end{verbatim}
%
%Das Beispiel zeigt, dass manche Informationen, wie der Name oder die Fußlinie und das Porträt
%vor dem Dokumentstart bereitgestellt werden. Der eigentliche Lebenslauf wird dann in die 
%|ecv|-Umgebung
%geschrieben. Ein Lebenslauf kann mit einer Signatur abgeschlossen werden, wo der Ersteller 
%dann per Hand unterschreiben kann.
%
%\section{NLS-Unterstützung}
%
%Wie das Beispiel im letzten Kapitel vorschlägt, können Lebensläufe in Deutsch oder
%Englisch geschrieben werden. Eigentlich kann eine Lebenslauf—tex--Datei sowohl eine
%deutsche als auch eine englische Version beinhalten. Die meisten Befehle der Klasse 
%akzeptieren einen beliebigen Parameter, der definiert, welche Sprache der Befehl 
%applizieren wird. Wenn die Sprache nicht mit der Sprachendefinition in der Dokumentklasse
%zusammen passt, wird der Befehl ignoriert.
%
%Beispiel:
%
%\begin{verbatim}
%\ecvERP[german]{Staatsangeh"origkeit}{Deutsch}
%\ecvERP[english]{nationality}{german}
%\end{verbatim}
%
%Wenn das Dokument mit |\documentclass[german]{ecv}| erstellt wird, 
%wird die erste Zeile benutzt. Wird das Dokument mit |\documentclass[english]{ecv}|
%erstellt, wird die zweite Zeile abgebildet.
%
%Die Klasse bietet mit seinem Paket |ecvNLS.sty| ebenfalls einige Makros für
%nationalisierte Textfragmente wie |ecvPerson| welche in der deutschen Version als |Zur Person|
%und als |Personal Information| in der englischen Version ausgegeben werden.
%
%\section{Dokumentklasse}
%
%\DescribeMacro{documentclass ecv}
%Dieses Paket bietet die Dokumentklasse \texttt{ecv} an. Sie unterstützt 
%die folgenen Optionen:
%
%%\begin{itemize}
%\item |german| wählt deutsche Sprache,
%\item |english| wählt englische Sprache,
%\item |empty| bildet Kopf- oder Fußnote nicht ab.
%\end{itemize}
%
%\section{Kopfzeile}
%
%Zwischen der \texttt{documentclass}- und der
%\texttt{document}—Umgebung werden zwei Befehle unterstützt:
%
%\DescribeMacro{ecvName}
%\DescribeMacro{ecvPortrait}
%\begin{itemize}
%\item |\ecvName|\marg{name} Setzt den Namen des Autors des Lebenslaufes.
%      Der Name wird linker Hand in der Fußzeile abgebildet. Wenn der Name nicht
%      gesetzt wird, wird diese Fußzeile(\texttt{Curriculum Vitae} oder
%      \texttt{Lebenslauf}) nicht abgebildet.
%\item |ecvPortrait|\marg{image--name} Setzt den Namen des Bildes,
%      das als Porträt rechts neben dem Titel des Lebenslaufs benutzt werden soll.
%      Eine Datei—Vergrößerung von\texttt{jpg} wird an das
%      \texttt{image-name} angehängt. Das Bild wird mit den Maßen 40mm x 60mm abgebildet.
%       Wenn der |ecvPortrait|-Befehl nicht existiert, wird auch kein Bild angezeigt.
%\end{itemize}
%
%\section{Fußzeile}
%
%Nach dem Lebenslauf kann ein Feld für die Signatur hinzugefügt werden. Dieses
%Paket sieht den folgenden Befehl für diesen Zweck vor:
%
%\DescribeMacro{ecvSig}
%\begin{itemize}
%\item |\ecvSig|\marg{name}\marg{town} erstellt eine Signatur nahe dem Lebenslauf.
%\end{itemize}
%
%Die Signatur sieht dann folgendermaßen aus:
%
%\noindent
%MeineStadt \today \\[18pt]
%  
%Name, Mein
%
%\section{\texttt{ecv}-Umgebung}
%
%\DescribeEnv{ecv}
%Die |ecv|-Umgebung umschließt den Lebenslauf. Alle Eintragungen darin müssen
%innerhalb dieser |ecv|-Umgebung passieren. Die|ecv|-Umgebung erstellt einen Titel 
%(entweder |Lebenslauf| oder |Curriculum Vitae|)
%und das Porträt (wenn eines definiert wurde) vor den Einträgen.
%
%\DescribeEnv{ecv*}
%Neben der |ecv|-Umgebung ist die |ecv*| Umgebung vorgehalten. Diese Umgebung
%arbeitet exakt wie die |ecv|- Umgebung, sie aber lässt den Titel und das Bild weg.
%
%\section{Einträge}
%
%Der Lebenslauf ist eine Zusammenstellung von Einträgen, die durch Tags (linker Hand)
%und einem Wert (rechter Hand) arrangiert wurde. Beide können je nach Geschmack 
%variieren.
%
%Zum Beispiel haben Sie verschiedene Einträge für Ihren Beruf:
%"`Dauer"', "`Berufszweig"', "`Berufstitel"' und "`Job Beschreibung"'.
%Es wird empfohlen den Eintrag „Dauer“ mit einem 
%speziellen Symbol zu kennzeichnen, wie zum Beispiel einem blauen Dreieck, die anderen
%Einträge hingegen mit einem gebräuchlicheren Symbol wie einer blauen Kugel.
%
%Beachten Sie, dass in diesem Beispiel einige Berufe unter „Berufe“ gelistet werden
%würden. In der Beschreibung darunter bezieht sich die Bezeichnung Gruppe auf einen
%einzelnen Job mit seinen unterschiedlichen Einträgen.
%
%\DescribeMacro{\ecvTP}
%\DescribeMacro{\ecvTF}
%\DescribeMacro{\ecvTN}
%Tags können mit folgenden Makros geschrieben werden:
%
%\begin{itemize}
%\item |\ecvTF|\oarg{lang}\marg{text} oder |\ecvTagFirst|\oarg{lang}\marg{text}
%      Ein Tag, der eine Gruppe von Tags beginnt. Dem \texttt{text} ist ein kleines
%      blaues Dreieck vorangestellt.
%\item |\ecvTN|\oarg{lang}\marg{text} oder |\ecvTagNext|\oarg{lang}\marg{text}
%      Ein Tag innerhalb einer Gruppe von Tags. Dem \texttt{text} ist ein kleiner
%      Kreis vorangestellt.
%\item |\ecvTP|\oarg{lang}\marg{text} oder |\ecvTagPlain|\oarg{lang}\marg{text}
%      Die einfache Form eines Tags. Bildet nur \texttt{text} ab, ohne dass ein Symbol
%      vorangestellt wird.
%\end{itemize}
%
%
%\DescribeMacro{\ecvVR}
%\DescribeMacro{\ecvVB}
%Werte können flatternd rechts oder als ausgerichteter Text (Blocksatz) mit den folgenden Makros 
%geschrieben werden:
%
%\begin{itemize}
%\item |\ecvVR|\oarg{lang}\marg{text} oder |\ecvValueRagged|\oarg{lang}\marg{text}
%      Ein flatternder rechtsausgerichteter Wert.
%\item |\ecvVB|\oarg{lang}\marg{text} oder |\ecvValueBlocked|\oarg{lang}\marg{text}
%      Ein Wert mit justiertem Text.
%\end{itemize}
%
%\bigskip
%Tags und Werte sind getrennt durch ein \&:
%
%\begin{verbatim}
%\ecvTP{Name} & \ecvVR{Name, Mein}
%\end{verbatim}
%
%\DescribeMacro{\ecvEPR}
%\DescribeMacro{\ecvEPB}
%\DescribeMacro{\ecvEFR}
%\DescribeMacro{\ecvEFB}
%\DescribeMacro{\ecvENR}
%\DescribeMacro{\ecvENB}
%Normalerweise wollen wir ganze Einträge schreiben. Wir wollen keine separaten Tags 
%und Werte. Deshalb sind eine geeignete Form, um ganze Einträge zu
%schreiben, die folgenden
%Makros, welche eigentlich eine Kombination von Tag und Wertmakros sind. Der Name der 
%Makros besteht aus |ecvE| gefolgt von der Variante des Wertes:
%
%\begin{itemize}
%\item |\ecvEPR|\oarg{lang}\marg{tag}\marg{value} schreibt einen einfachen Tag mit 
%      einem Wert nach rechts flatternd.
%\item |\ecvEPB|\oarg{lang}\marg{tag}\marg{value} schreibt einen einfachen Tag mit
%      einem Wert nach rechts zentriert.
%\item |\ecvEFR|\oarg{lang}\marg{tag}\marg{value} schreibt einen ersten Tag mit
%      einem Wert  nach rechts flatternd.
%\item |\ecvEFB|\oarg{lang}\marg{tag}\marg{value} schreibt einen ersten Tag mit
%      einem Wert und nach rechts zentriert.
%\item |\ecvENR|\oarg{lang}\marg{tag}\marg{value} schreibt einen weiteren Tag mit
%      einem Wert und  nach rechts flatternd.
%\item |\ecvENB|\oarg{lang}\marg{tag}\marg{value} schreibt einen weiteren Tag mit
%      einem Wert und nach rechts zentriert.
%\end{itemize}
%
%Natürlich haben die Makros eine lange Form. 
%
%\begin{quote}
%|\ecvTagPlainValueRagged|\\
%|\ecvTagPlainValueBlocked| \\
%|\ecvTagFirstValueRagged|\\
%|\ecvTagFirstValueBlocked| \\
%|\ecvTagNextValueRagged|\\
%|\ecvTagNextValueBlocked|\\
%\end{quote}
%
%\DescribeMacro{\ecvOVR}
%\DescribeMacro{\ecvOnlyValueRagged}
%\DescribeMacro{\ecvOVB}
%\DescribeMacro{\ecvOnlyValueBlocked}
%Zwei spezielle Makros wurden entwickelt, die nur Teile drucken:
%
%\begin{itemize}
%\item |\ecvOVR|\oarg{lang}\marg{text} or |\ecvOnlyValueRagged|\oarg{lang}\marg{text}
%      Druckt nur die Werte nach rechts flatternd.
%\item |\ecvOVB|\oarg{lang}\marg{text} or |\ecvOnlyValueBlocked|\oarg{lang}\marg{text}
%      Druckt die Werte nach rechts zentiert.
%\end{itemize}
%
%\section{Abschnitte}
%
%Einträge im Lebenslauf können nach Abschnitten und Unterabschnitten gruppiert werden.
%Abschnitte werden in der linken Spalte mit blauen Buchstaben mit einer leicht größeren  
%Schriftart abgebildet. Unterabschnitte werden mit Großbuchstaben abgebildet.
%
%
%\DescribeMacro{\ecvSec}
%\DescribeMacro{\ecvSection}
%\DescribeMacro{\ecvBSec}
%\DescribeMacro{\ecvBreakSection}
%Den Abschnittsbefehl gibt es in zwei verschiedene Varianten: mit und ohne
%zusätzlichen vorangehenden vertikalen Abstand (6pt).
%
%%\begin{itemize}
%\item |\ecvSec|\oarg{lang}{text} oder |\ecvSection|\oarg{lang}{text} 
%      Bildet einen Abschnittstag ohne zusätzlich vorangehenden vertikalen Abstand.
%\item |\ecvBSec|\oarg{lang}{text} or  |\ecvBreakSection|\oarg{lang}{text}  
%      Bildet einen Abschnittstag mit zusätzlichen vorangehendem Abstand.
%\end{itemize}
%
%Im Moment haben wir keine Automatisierung des vertikalen
%Abstandes versucht, aber wir haben die 
%Erfahrung gemacht, dass es notwendig wäre, ihn hinzuzufügen.
%Sie können gern den \texttt{ecvB}-Befehl eliminieren, indem Sie eigene originelle
%Automatisierungsregeln benutzen. Und vergessen Sie nicht, Ihr
%anspruchsvolles \LaTeX-Makro-Wissen an uns weiterzuleiten.
%
%\DescribeMacro{\ecvSub}
%\DescribeMacro{\ecvBSub}
%\DescribeMacro{\ecvERSub}
%\DescribeMacro{\ecvBERSub}
%\DescribeMacro{\ecvEBSub}
%\DescribeMacro{\ecvBEBSub}
%Unterabschnittstags werden wie Abschnittstags in der ersten Spalte abgebildet,
%aber in einer anderen Schriftart. Im Kontrast zu den Abschnitten können sie einen 
%zugeordneten Wert haben. Unterabschnittskommandos werden deshalb in einer
%Version mit und einer Version ohne Wert bereitgestellt:
%
%%\begin{itemize}
%\item |\ecvSub|\oarg{lang}\marg{text} Standard-Unterabschnitte (ohne 
%      zusätzlichen vertikalen Abstand).
%\item |\ecvBSub|\oarg{lang}\marg{text} Unterabschnitte mit zusätzlichem
%      vertikalen Abstand (smallskip).
%\end{itemize}
%\begin{itemize}
%\item |\ecvERSub|\oarg{lang}\marg{text}\marg{value} Unterabschnitte mit einem 
%      flatternden rechten Wert (ohne zusätzlichen vertikalen Abstand).
%\item |\ecvBERSub|\oarg{lang}\marg{text}\marg{value} Unterabschnitte mit einem 
%      flatternden rechten Wert und zusätzlichem vertikalen Abstand.
%\end{itemize}
%\begin{itemize}
%\item |\ecvEBSub|\oarg{lang}\marg{text}\marg{value} Unterabschnitt mit einem 
%      zentrierten rechten Wert (ohne zusätzlichen vertikalen Abstand).
%\item |\ecvBEBSub|\oarg{lang}\marg{text}\marg{value} Unterabschnitt mit einem 
%      zentrierten rechten Wert und zusätzlichem vertikalen Abstand.
%\end{itemize}
%
%Auch diese Befehle kommen mit "`verbose"' Formen: 
%
%\begin{quote}
%|\ecvSubSection|\\
%|\ecvBreakSubSection|\\
%|\ecvEntryRaggedSubSection|\\
%|\ecvBreakEntryRaggedSubSection|\\
%|\ecvEntryBlockedSubSection|\\
%|\ecvBreakEntryBlockedSubSection|\\
%\end{quote}
%
%\section{Layouten}
%
%Die \texttt{ecv}-Klasse stellt einige Kommandos bereit, um das Layout des Lebenslaufes zu 
% optimieren.
%
%\DescribeMacro{\ecvPageBreak}
%\DescribeMacro{\ecvNewPage}
%Die folgenden zwei Befehle können als kontrollierte Seitenumbrüche genutzt werden:
%
%\begin{itemize}
%\item |ecvPageBreak| Schlägt einen Seitenumbruch vor.
%\item |ecvNewPage| Für eine neue Seite.
%\end{itemize}
%
%\DescribeMacro{\ecvBreakParagraphs}
%Der |\ecvBreakParagraphs|-Befehl kann auch genutzt werden, um einen vertikalen Abstand
%zwischen den Einträgen hinzuzufügen.
%
%\DescribeMacro{\ecvNewLine}
%Der Befehl |\ecvNewLine| kann eingesetzt werden, um eine Neu-Linie fortzusetzen.
%
%\DescribeMacro{\ecvNewPara}
%Der Befehl |\ecvNewPara| beginnt einen neuen Paragraphen mit zusätzlichem vertikalen Abstand
%(smallskip).
%
%\DescribeMacro{\ecvBold}
%Der Befehl |\ecvBold|\marg{text} kann genutzt werden, um einen fett gedruckten Text zu kreieren.
%
%\section{Hyperlinks}
%
%\DescribeMacro{\ecvURL}
%\DescribeMacro{\ecvEMail}
%\DescribeMacro{\ecvHyperLink}
%\DescribeMacro{\ecvHyperEMail}
%Die Lebenslauf-Klasse stellt einige Befehle bereit, um Hyperlinks zu erstellen:
%
%\begin{itemize}
%\item |\ecvHyperLink|\marg{url} Formatiert eine anklickbare URL.
%\item |\ecvHyperEMail|\marg{email} Formatiert eine anklickbare E-Mail.
%\end{itemize}
%
%Die oberen Befehle basieren auf den folgenden nicht anklickbaren Befehlen, 
%welche die Formatierung bereitstellen und welche ebenso genutzt werden können:
%
%\begin{itemize}
%\item |\ecvURL|\marg{url} Formatiert eine nicht anklickbare URL.
%\item |\ecvEMail|\marg{email} Formatiert eine nicht anklickbare E-Mail.
%\end{itemize}
%
%\section{Lokalisierte Zeichenfolgen}
%
%Die folgenden lokalisierten Zeichenfolgen werden für die deutsche und die englische Sprache bereit
%gestellt:
%
%\begin{itemize}
%\item |\ecvPerson| Entweder|Zur Person| oder |Personal Information|
%\DescribeMacro{ecvPerson}
%\item |\ecvProfession| Entweder |Beruf| oder |Profession|
%\DescribeMacro{ecvProfession}
%\item |\ecvResearch| Entweder |Forschung| oder |Research|
%\DescribeMacro{ecvResearch}
%\item |\ecvEducation| Entweder |Ausbildung| oder |Scholarship|
%\DescribeMacro{ecvEducation}
%\item |\ecvPublications| Entweder |Publikationen| oder |Publications|
%\DescribeMacro{ecvPublications}
%\item |\ecvAwards| Entweder |Auszeichungen| oder |Awards|
%\DescribeMacro{ecvAwards}
%\item |\ecvScholarships| Entweder |Stipendien| oder |Scholarships|
%\DescribeMacro{ecvScholarships}
%\item |\ecvJobs| Entweder |Arbeitserfahrung| oder |Jobs|
%\DescribeMacro{ecvJobs}
%\item |\ecvLanguages| Entweder |Sprachen| oder |Languages|
%\DescribeMacro{ecvLanguages}
%\item |\ecvLanguageTravels| Entweder |Sprachreisen| oder |Language Travels|
%\DescribeMacro{ecvLanguageTravels}
%\item |\ecvAbilities| Entweder |F"ahigkeiten| oder |Abilities|
%\DescribeMacro{ecvAbilities}
%\item |\ecvConferences| Entweder |Konferenzen| oder |Conferences|
%\DescribeMacro{ecvConferences}
%\item |\ecvSpeeches| Entweder|Vortr"age| oder |Speeches|
%\DescribeMacro{ecvSpeeches}
%\item |\ecvTraining| Entweder |Fortbildung| oder |Trainig|
%\DescribeMacro{ecvTrainig}
%\item |\ecvAttachements| Entweder|Anh"ange| oder |Attachements|
%\DescribeMacro{ecvAttachements}
%\end{itemize}
%
%\section{Anforderungen}
%
%Wir nutzen verschiedene andere \LaTeX\ Pakete für verschiedene Zwecke,
%die unter Ihrer Installation funktionieren.
%
%\begin{itemize}
%\item geometry
%\item longtable
%\item pgf
%\item paralist
%\item helvet
%\item color
%\item fancyhdr
%\item inputenc
%\item fontenc
%\item ae
%\item aecompl
%\item aeguill
%\item textcomp
%\item url
%\item hyperref
%\item babel
%\end{itemize}
%
%
%
%
%\StopEventually{}
%
%\section{Der Code}
%\iffalse
%    \begin{macrocode}
%<*ecv.cls>
%    \end{macrocode}
%\fi
%    \begin{macrocode}
%%
%% Copyright 2006-2007 Christoph Neumann, Bernd Haberstumpf
%%
%% Diese \LaTeX-Klasse bietet eine einfache Schnittstelle, um einen
%% originellen Lebenslauf zu kreieren. Momentan werden Lebensläufe nur
%% in der deutschen Sprache unterstützt.
%%
%% Diese Datei ist freies Eigentum; als spezielle Ausnahme gibt der Autor
%% unbegrenzte Erlaubnis sie zu kopieren und/oder mit oder ohne Modifikationen
%% zu verteilen, so lang diese Notiz erhalten bleibt. 
%%
%% Diese Datei wird in der Hoffnung verteilt, dass sie nützlich sein wird, 
%% aber OHNE JEGLICHE GARANTIE zur Ausdehnung der gesetzlichen Erlaubnis; 
%% ohne implizierte Garantie der Verkaufsfähigkeit oder EIGNUNG FÜR EINEN 
%%BESTIMMTEN ZWECK.
%%
%% Ein BESONDERER DANK geht an 
%% Alexander von Gernler, der mich in European Curriculum Vitae einführte.
%%

% \changes{v0.1}{2007/01/06}{Initial version}

\def\fileversion{0.1}
\def\filedate{2007/01/05}

\NeedsTeXFormat{LaTeX2e}

%
% Class definition
% 

\ProvidesClass{ecv}[\filedate %
  \space Version \fileversion\space by %
  Christoph P.\ Neumann & Bernd Haberstumpf %
]


%
% Option definition
%

\def\ecv@lang{german}
\def\ecv@german{german}
\def\ecv@english{english}
\DeclareOption{german}{\def\ecv@lang{\ecv@german}}
\DeclareOption{english}{\def\ecv@lang{\ecv@english}}
\DeclareOption{oneside}{\PassOptionsToClass{oneside}{scrartcl}}
\DeclareOption{twoside}{\PassOptionsToClass{twoside}{scrartcl}}
\DeclareOption{empty}{\def\ecv@empty{1}}
\ProcessOptions


%
% Load base class
% 

\LoadClass[a4paper,11pt]{article}

% define command to check for pdf
\RequirePackage{ifpdf}

\ifpdf
  \pdfcompresslevel=9           % compression level fortext and image;
\fi 

%
% Load packages
%

% NLS
\RequirePackage[\ecv@lang]{ecvNLS}

% Provides an ifthenelse command
\RequirePackage{ifthen}

% Pagelayout
\ifpdf
  \RequirePackage[a4paper, pdftex]{geometry}
\else
  \RequirePackage[a4paper, dvips ]{geometry}
\fi

% For the table that makes the text spanning multiple pages
\RequirePackage{longtable}

% Grafix package for the portrait
\RequirePackage{pgf}

% compact listings/enumerate environments
\RequirePackage{paralist}

% Font
\RequirePackage{helvet}

% Colors for the sections
\RequirePackage{xcolor}

% Needed for the footline to redefine the footline
\RequirePackage{fancyhdr}

% Inputencoding (latin1 with euro sign)
%\RequirePackage[latin9]{inputenc} % = latin1, but also with euro sign
% Better variant than inputenc:
\RequirePackage{selinput}
% SelectInputMappings seems not to be necessary? ...
% If it is used in the cls file it makes problems if used 
% in a Windows environment... ?!?
%\SelectInputMappings{
%	adieresis={ä},
%	germandbls={ß},
%	Euro={€},
%}

% Outputencoding
\RequirePackage[T1]{fontenc}

% Font tweaking
%\RequirePackage{ae}
%\RequirePackage{aecompl}
%\RequirePackage{aeguill}
% e.g. for \textcopyleft
%\RequirePackage{textcomp}

% Output of links
\RequirePackage{url}
\ifpdf
  \RequirePackage[pdftex]{hyperref} %,pdfstartpage=9
\else
  \RequirePackage[dvips]{hyperref} %ALT: colorlinks
\fi


%
% configuration (page setup, color setup etc.)
%

% page setup
\geometry{left=30mm, right=20mm, top=20mm, bottom=15mm}

% Footer
\ifx\ecv@empty\undefined
  \pagestyle{fancy}
\else
  \pagestyle{empty}
\fi

% Color setup (for the sections)
\definecolor{ecv@ColBlue}{rgb}{0.04,0.44,0.59}   % ANPA 732-0, but darker
\definecolor{ecv@ColRed}{rgb}{0.921,0.282,0.278} % ANPA 723-0

\ClassInfo{ecv}{used language is \ecv@lang}

% This variable holds the name set with ecvName
\newcommand{\ecv@name}{}

% \begin{document} preamble
\AtBeginDocument{%
  \sffamily
  \raggedbottom
  \fancyhead{}
  \fancyfoot{}
  \renewcommand{\headrulewidth}{0pt}
  \renewcommand{\footrulewidth}{0pt}
  \fancyfoot[R]{
    \begin{minipage}{5cm}\begin{flushright}
      \footnotesize{}\textsf{\ecvPage~\thepage}
    \end{flushright}\end{minipage}
  }
  \ifthenelse{\equal{\ecv@name}{}}{
    ~
  }{
    \fancyfoot[L]{
      \begin{minipage}{6cm}
        \footnotesize{}\textsf{\ecvTitle~\ecv@name}
      \end{minipage}
    }
  }
}

% Command to layout the portrait (must be 60mmx40mm) 
\newcommand\ecv@Portrait[1]{%
  %% A frame as placeholder (with  some 1mm inner padding):
  \pgfrect[stroke]{\pgfxy(6.85,0.65)}{\pgfxy(4.3,-6.3)}
  %% Actually a concrete digital image:
  \pgfdeclareimage[interpolate=true,height=60mm,width=40mm]{portrait}{#1}
  \pgfputat{\pgfxy(6.77,0.5)}{\pgfbox[left,top]{\pgfuseimage{portrait}}}
}   

% This variable holds the name of the portrait image
\newcommand\ecv@img{}

% title with image
\newcommand{\ecv@Title}{%
  \ifthenelse{\equal{\ecv@img}{}}{ %
    \ecvLeft{\textsc{\LARGE{\ecvTitle}}%
      \bigskip\bigskip\bigskip%
    } & \tabularnewline %
  }{ %
    \ecvLeft{\textsc{\LARGE{\ecvTitle}}%
      \bigskip\bigskip\bigskip%
    } & \ecv@Portrait{\ecv@img} %
    \tabularnewline %
  } %
}


%
% Define new commands
% 

% Hyperlink commands
\newcommand\ecvURL{\begingroup \urlstyle{sf}\Url} 
\newcommand\ecvEMail{\begingroup \urlstyle{sf}\Url}
\ifpdf
  \newcommand{\ecvHyperLink}[1]{%
    \href{#1}{\ecvURL{#1}}%
  }
  \newcommand{\ecvHyperEMail}[1]{%
    \href{mailto:#1}{\ecvEMail{#1}}%
  }
  \newcommand{\ecvHttp}[1]{%
    \href{http://#1}{\ecvURL{#1}}%
  }
\else
  \newcommand{\ecvHyperLink}[1]{%
    \ecvURL{#1}%
  }
  \newcommand{\ecvHyperEMail}[1]{%
    \ecvEMail{#1}%
  }
  \newcommand{\ecvHttp}[1]{%
    \ecvURL{#1}%
  }
\fi
\hypersetup{a4paper,pdfpagelayout={SinglePage},pdfstartview={Fit}}
\hyperbaseurl{http://}


% Name

% Defines the name of the issuer for the footline
\newcommand{\ecvName}[1]{\renewcommand{\ecv@name}{#1}}

% Portrait

% Defines the image (to be used before the \begin{ecv})
% \ecvPortrait{file-name}
\newcommand{\ecvPortrait}[1]{\renewcommand\ecv@img{#1}}


% Environment for layouting the cv

% Environment that prints title and portrait
\newenvironment{ecv}{%
  \begin{longtable}{p{.32\linewidth}|p{.68\linewidth}}
  \ecv@Title
}{%
  \end{longtable}
}
% Environment that skips title and portrait
\newenvironment{ecv*}{%
  \begin{longtable}{p{.32\linewidth}|p{.68\linewidth}}
}{%
  \end{longtable}
}

% Vertical Spacing to be used in left column

% begin new row: make an (optional) spacing before a section
\newcommand{\ecvBreaksections}[0]{& \tabularnewline[6pt]} 
% begin new row: make a break before a subsection
\newcommand{\ecvBreaksubsections}[0]{& \tabularnewline\smallskip}
% begin new row: make a break inside a section (for subsub grouping of entries)
\newcommand{\ecvBreakparagraphs}[0]{& \tabularnewline}

% Vertical spacing to be used inside a column

% start text on a new line
\newcommand{\ecvNewLine}[0]{\\}
% start a new paragraph
\newcommand{\ecvNewPara}[0]{\smallskip}

% Newpage

% Force a page break
\newcommand{\ecvNewPage}{
  \newpage
}
% Suggest a page break
\newcommand{\ecvPageBreak}{
  \pagebreak
}

% emphasizings text
\newcommand{\ecvBold}[2][\ecv@lang]{%
  \textbf{#2}%
}

% bullets for the left column entries
\newcommand{\ecvBulleted}[1]{$\circ$ #1}
\newcommand{\ecvBulletedFirst}[1]{%
  \textcolor{ecv@ColBlue}{$\triangleright$} #1%
}

% layout primitives for left and right column

% primitve for the left column
\newcommand{\ecvLeft}[1]{%
  \parbox[t]{\linewidth}{\raggedright #1}%
}
% primitive for the right column (raggedright)
\newcommand{\ecvRight}[1]{%
  %\parbox[t]{\linewidth}{
  {\raggedright #1}%
  \tabularnewline%
}
% primitive for the right column (block)
\newcommand{\ecvRightBlock}[1]{%
  \parbox[t]{0.9\linewidth}{#1}%
  \tabularnewline%
}


% left column commands

% tag without bullet (simple left column entry)
\newcommand{\ecvTP}[2][\ecv@lang]{%
  \ifthenelse{\equal{#1}{\ecv@lang}}{%
    \ecvLeft{#2}%
  }{}%
}
\newcommand{\ecvTagPlain}[2][\ecv@lang]{\ecvTP[#1]{#2}}
% tag with first line mark (triange bullet left column entry)
\newcommand{\ecvTF}[2][\ecv@lang]{%
  \ifthenelse{\equal{#1}{\ecv@lang}}{%
    \ecvBreakparagraphs
    \smallskip
    \ecvLeft{\ecvBulletedFirst{#2}}%
  }{}%
}
\newcommand{\ecvTagFirst}[2][\ecv@lang]{\ecvTF[#1]{#2}}
% tag with first line mark (triange bullet left column entry)
% but WITHOUT the line break!
\newcommand{\ecvTI}[2][\ecv@lang]{%
  \ifthenelse{\equal{#1}{\ecv@lang}}{%
    \ecvLeft{\ecvBulletedFirst{#2}}%
  }{}%
}
\newcommand{\ecvTagIntermediate}[2][\ecv@lang]{\ecvTI[#1]{#2}}
% tag with first follow line mark (circle bullet left column entry)
\newcommand{\ecvTN}[2][\ecv@lang]{%
  \ifthenelse{\equal{#1}{\ecv@lang}}{%
    \ecvLeft{\ecvBulleted{#2}}%
  }{}%
}
\newcommand{\ecvTagNext}[2]{\ecvTN[#1]{#2}}


% right column commands

% value with raggedright layout
\newcommand{\ecvVR}[2][\ecv@lang]{%
  \ifthenelse{\equal{#1}{\ecv@lang}}{%
    \ecvRight{#2}%
  }{}%
}
\newcommand{\ecvValueRagged}[2][\ecv@lang]{\ecvVR[#1]{#2}}
% value with block layout
\newcommand{\ecvVB}[2][\ecv@lang]{%
  \ifthenelse{\equal{#1}{\ecv@lang}}{%
    \ecvRightBlock{#2}%
  }{}%
}
\newcommand{\ecvValueBlocked}[2][\ecv@lang]{\ecvVB[#1]{#2}}


% Compound commands tag+value

% Plain tag with ragged value
\newcommand{\ecvEPR}[3][\ecv@lang]{%
  \ecvTP[#1]{#2} & \ecvVR[#1]{#3} %
}
\newcommand{\ecvTagPlainValueRagged}[3][\ecv@lang]{\ecvERP[#1]{#2}{#3}}
% Plain tag with blocked value
\newcommand{\ecvEPB}[3][\ecv@lang]{%
  \ecvTP[#1]{#2} & \ecvVB[#1]{#3} %
}
\newcommand{\ecvTagPlainValueBlocked}[3][\ecv@lang]{\ecvERB[#1]{#2}{#3}}
% bulleted first tag with ragged value
\newcommand{\ecvEFR}[3][\ecv@lang]{%
  \ecvTF[#1]{#2} & \ecvVR[#1]{#3} %
}
\newcommand{\ecvTagFirstValueRagged}[3][\ecv@lang]{\ecvEFR[#1]{#2}{#3}}
% bulleted first tag with blocked value
\newcommand{\ecvEFB}[3][\ecv@lang]{%
  \ecvTF[#1]{#2} & \ecvVB[#1]{#3} %
}
\newcommand{\ecvTagFirstValueBlocked}[3][\ecv@lang]{\ecvEFB[#1]{#2}{#3}}
% bulleted intermediate tag with ragged value
\newcommand{\ecvEIR}[3][\ecv@lang]{%
  \ecvTI[#1]{#2} & \ecvVR[#1]{#3} %
}
\newcommand{\ecvTagIntermediateValueRagged}[3][\ecv@lang]{\ecvEIR[#1]{#2}{#3}}
% bulleted intermediate tag with blocked value
\newcommand{\ecvEIB}[3][\ecv@lang]{%
  \ecvTI[#1]{#2} & \ecvVB[#1]{#3} %
}
\newcommand{\ecvTagIntermediateValueBlocked}[3][\ecv@lang]{\ecvEIB[#1]{#2}{#3}}
% bulleted next tag with ragged value
\newcommand{\ecvENR}[3][\ecv@lang]{%
  \ecvTN[#1]{#2} & \ecvVR[#1]{#3} %
}
\newcommand{\ecvTagNextValueRagged}[3][\ecv@lang]{\ecvENR[#1]{#2}{#3}}
% bulleted next tag with blocked value
\newcommand{\ecvENB}[3][\ecv@lang]{%
  \ecvTN[#1]{#2} & \ecvVB[#1]{#3} %
}
\newcommand{\ecvTagNextValueBlocked}[3][\ecv@lang]{\ecvENB[#1]{#2}{#3}}
% value only ragged
\newcommand{\ecvOVR}[2][\ecv@lang]{%
  & \ecvVR[#1]{#2} %
}
\newcommand{\ecvOnlyValueRagged}[2][\ecv@lang]{\ecvOVR[#1]{#2}}
% value only blocked
\newcommand{\ecvOVB}[2][\ecv@lang]{%
  & \ecvVB[#1]{#2} %
}
\newcommand{\ecvOnlyValueBlocked}[2][\ecv@lang]{\ecvOVB[#1]{#2}}




% Sections

% section: \ecvSection{name}
\newcommand{\ecvSec}[2][\ecv@lang]{%
  \ifthenelse{\equal{#1}{\ecv@lang}}{%
    \ecvLeft{\textsc{\Large{\textcolor{ecv@ColBlue}{#2}}} \bigskip } &%
    \tabularnewline%
  }{}%
}
\newcommand{\ecvSection}[2][\ecv@lang]{\ecvSec[#1]{#2}}
% section with breaksection: \ecvBreakSection{name}
\newcommand{\ecvBSec}[2][\ecv@lang]{%
  \ifthenelse{\equal{#1}{\ecv@lang}}{%
    \ecvBreaksections
    \ecvLeft{\textsc{\Large{\textcolor{ecv@ColBlue}{#2}}} \bigskip } &%
    \tabularnewline%
  }{}%
}
\newcommand{\ecvBreakSection}[2][\ecv@lang]{\ecvBSec[#1]{#2}}
% sub-section: \ecvSubSection{name}
\newcommand{\ecvSub}[2][\ecv@lang]{%
  \ifthenelse{\equal{#1}{\ecv@lang}}{%
    \ecvLeft{\textsc{\large{#2}}}%
    & \tabularnewline%
  }{}%
}
\newcommand{\ecvSubSection}[2][\ecv@lang]{\ecvSub[#1]{#2}}
\newcommand{\ecvBSub}[2][\ecv@lang]{%
  \ifthenelse{\equal{#1}{\ecv@lang}}{%
    \ecvBreaksubsections
    \ecvLeft{\textsc{\large{#2}}}%
    & \tabularnewline%
  }{}%
}
\newcommand{\ecvBreakSubSection}[2][\ecv@lang]{\ecvBSub[#1]{#2}}
% sub-section with a value
\newcommand{\ecvERSub}[3][\ecv@lang]{%
  \ifthenelse{\equal{#1}{\ecv@lang}}{%
    \ecvLeft{\textsc{\large{#2}}} & \ecvRight{#3}%
  }{}%
}
\newcommand{\ecvEntryRaggedSubSection}[3][\ecv@lang]{\ecvERSub[#1]{#2}{#3}}
\newcommand{\ecvBERSub}[3][\ecv@lang]{%
  \ifthenelse{\equal{#1}{\ecv@lang}}{%
    \ecvBreaksubsections
    \ecvLeft{\textsc{\large{#2}}} & \ecvRight{#3}%
  }{}%
}
\newcommand{\ecvBreakEntryRaggedSubSection}[3][\ecv@lang]{\ecvBERSub[#1]{#2}{#3}}
% sub-section with a value
\newcommand{\ecvEBSub}[3][\ecv@lang]{%
  \ifthenelse{\equal{#1}{\ecv@lang}}{%
    \ecvLeft{\textsc{\large{#2}}} & \ecvRightBlock{#3}%
  }{}%
}
\newcommand{\ecvEntryBlockedSubSection}[3][\ecv@lang]{\ecvEBSub[#1]{#2}{#3}}
\newcommand{\ecvBEBSub}[3][\ecv@lang]{%
  \ifthenelse{\equal{#1}{\ecv@lang}}{%
    \ecvBreaksubsections
    \ecvLeft{\textsc{\large{#2}}} & \ecvRightBlock{#3}%
  }{}%
}
\newcommand{\ecvBreakEntryBlockedSubSection}[3][\ecv@lang]{\ecvBEBSub[#1]{#2}{#3}}




% Signature

% \ecvSignature{name}{town}
\newcommand{\ecvSig}[2]{ %
  \vspace{1cm}
  \noindent
  #2, \today \\[18pt]

  #1
}
\newcommand{\ecvSignature}[2]{\ecvSig{#1}{#2}}
%    \end{macrocode}
%\iffalse
%    \begin{macrocode}
%</ecv.cls>
%    \end{macrocode}
%\fi
%\iffalse
%    \begin{macrocode}
%<*ecvEnglish.ldf>
%    \end{macrocode}
%\fi
%    \begin{macrocode}
%%
%% Copyright 2006-2007 Christoph Neumann, Bernd Haberstumpf
%%
%% This a language definition file for the ecv class.
%% THe file defines some NLS strings.
%%
%% This file is free property; as a special exception the author
%% gives unlimited permission to copy and/or distribute it, with
%% or without modifications, as long as this notice is 
%% preserved.
%%
%% This file is distributed in the hope that it will be useful, 
%% but WITHOUT ANY WARRANTY, to the extent permitted by law; 
%% without even the implied warranty of MERCHANTABILITY or 
%% FITNESS FOR A PARTICULAR PURPOSE.
%%
%% SPECIAL THANKS to
%% Alexander von Gernler, who introduced me to the European Curriculum Vitae
%%

\def\fileversion{0.1}
\def\filedate{2007/01/05}

\NeedsTeXFormat{LaTeX2e}

\ProvidesFile{ecvEnglish.ldf}[2007/01/05]

\def\ecvNLS@Page{Page}
\def\ecvNLS@Title{Curriculum Vitae}
\def\ecvNLS@Person{Personal Information}
\def\ecvNLS@Profession{Profession}
\def\ecvNLS@Education{Education}
\def\ecvNLS@Research{Research}
\def\ecvNLS@Awards{Awards}
\def\ecvNLS@Publications{Publications}
\def\ecvNLS@Scholarships{Scholarships}
\def\ecvNLS@Jobs{Jobs}
\def\ecvNLS@Languages{Languages}
\def\ecvNLS@LanguageTravels{Language Travels}
\def\ecvNLS@Abilities{Abilities}
\def\ecvNLS@Conferences{Conferences}
\def\ecvNLS@Speeches{Speeches}
\def\ecvNLS@Trainig{Training}
\def\ecvNLS@Attachements{Attachements}
%    \end{macrocode}
%\iffalse
%    \begin{macrocode}
%</ecvEnglish.ldf>
%    \end{macrocode}
%\fi
%\iffalse
%    \begin{macrocode}
%<*ecvGerman.ldf>
%    \end{macrocode}
%\fi
%    \begin{macrocode}
%%
%% Copyright 2006-2007 Christoph Neumann, Bernd Haberstumpf
%%
%% This a language definition file for the ecv class.
%% THe file defines some NLS strings.
%%
%% This file is free property; as a special exception the author
%% gives unlimited permission to copy and/or distribute it, with
%% or without modifications, as long as this notice is 
%% preserved.
%%
%% This file is distributed in the hope that it will be useful, 
%% but WITHOUT ANY WARRANTY, to the extent permitted by law; 
%% without even the implied warranty of MERCHANTABILITY or 
%% FITNESS FOR A PARTICULAR PURPOSE.
%%
%% SPECIAL THANKS to
%% Alexander von Gernler, who introduced me to the European Curriculum Vitae
%%


\def\fileversion{0.1}
\def\filedate{2007/01/05}

\NeedsTeXFormat{LaTeX2e}

\ProvidesFile{ecvGerman.ldf}[2007/01/05]

\def\ecvNLS@Page{Seite}
\def\ecvNLS@Title{Lebenslauf}
\def\ecvNLS@Person{Zur Person}
\def\ecvNLS@Profession{Beruf}
\def\ecvNLS@Education{Bildung}
\def\ecvNLS@Research{Forschung}
\def\ecvNLS@Awards{Auszeichnungen}
\def\ecvNLS@Publications{Publikationen}
\def\ecvNLS@Scholarships{Stipendien}
\def\ecvNLS@Jobs{Arbeitserfahrung}
\def\ecvNLS@Languages{Sprachen}
\def\ecvNLS@LanguageTravels{Sprachreisen}
\def\ecvNLS@Abilities{F\"ahigkeiten}
\def\ecvNLS@Conferences{Konferenzen}
\def\ecvNLS@Speeches{Vortr\"age}
\def\ecvNLS@Trainig{Fortbildung}
\def\ecvNLS@Attachements{Anlagen}
%    \end{macrocode}
%\iffalse
%    \begin{macrocode}
%</ecvGerman.ldf>
%    \end{macrocode}
%\fi
%\iffalse
%    \begin{macrocode}
%<*ecvNLS.sty>
%    \end{macrocode}
%\fi
%    \begin{macrocode}
%%
%% Copyright 2006-2007 Christoph Neumann, Bernd Haberstumpf
%%
%% This LaTeX package provides NLS support for the ecv class.
%%
%% This file is free property; as a special exception the author
%% gives unlimited permission to copy and/or distribute it, with
%% or without modifications, as long as this notice is 
%% preserved.
%%
%% This file is distributed in the hope that it will be useful, 
%% but WITHOUT ANY WARRANTY, to the extent permitted by law; 
%% without even the implied warranty of MERCHANTABILITY or 
%% FITNESS FOR A PARTICULAR PURPOSE.
%%
%% SPECIAL THANKS to
%% Alexander von Gernler, who introduced me to the European Curriculum Vitae
%%

\def\fileversion{0.1}
\def\filedate{2007/01/05}

\NeedsTeXFormat{LaTeX2e}

%
% Package definition
% 

\ProvidesPackage{ecvNLS}[\filedate %
  \space Version \fileversion\space by %
  Christoph Neumann & Bernd Haberstumpf %
]

%
% Option definition
%

\def\ecvNLS@lang{german}
\def\ecvNLS@german{1}
\def\ecvNLS@english{2}
\DeclareOption{german}{\def\ecvNLS@lang{\ecvNLS@german}\input{ecvGerman.ldf}}
\DeclareOption{english}{\def\ecvNLS@lang{\ecvNLS@english}\input{ecvEnglish.ldf}}
\ProcessOptions


%
% Load packages
%

% \selectlanguage{ngerman} will be called after \begin{document}
\RequirePackage[ngerman,english]{babel}   


% \begin{document} preamble
\AtBeginDocument{%
  \ifnum\ecvNLS@lang =\ecvNLS@german
    \selectlanguage{german}
  \else
    \selectlanguage{english}
  \fi
}


%
% Define NLS commands
%

\newcommand{\ecvPage}{\ecvNLS@Page}
\newcommand{\ecvTitle}{\ecvNLS@Title}
\newcommand{\ecvPerson}{\ecvNLS@Person}
\newcommand{\ecvProfession}{\ecvNLS@Profession}
\newcommand{\ecvEducation}{\ecvNLS@Education}
\newcommand{\ecvResearch}{\ecvNLS@Research}
\newcommand{\ecvAwards}{\ecvNLS@Awards}
\newcommand{\ecvPublications}{\ecvNLS@Publications}
\newcommand{\ecvScholarships}{\ecvNLS@Scholarships}
\newcommand{\ecvJobs}{\ecvNLS@Jobs}
\newcommand{\ecvLanguages}{\ecvNLS@Languages}
\newcommand{\ecvLanguageTravels}{\ecvNLS@LanguageTravels}
\newcommand{\ecvAbilities}{\ecvNLS@Abilities}
\newcommand{\ecvConferences}{\ecvNLS@Conferences}
\newcommand{\ecvSpeeches}{\ecvNLS@Speeches}
\newcommand{\ecvTrainig}{\ecvNLS@Trainig}
\newcommand{\ecvAttachements}{\ecvNLS@Attachements}
%    \end{macrocode}
%\iffalse
%    \begin{macrocode}
%</ecvNLS.sty>
%    \end{macrocode}
%\fi
%\Finale
\endinput
