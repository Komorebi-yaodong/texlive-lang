\ifx\printversion\undefined
\documentclass[11pt,leqno,titlepage,openany,oneside]{amsldoc}[1999/12/13]
\else
\documentclass[a4paper,11pt,leqno,titlepage,openany]{amsldoc}[1999/12/13]
\usepackage[monochrome]{color}
\advance \topmargin by -3\baselineskip
\advance \textheight by 5\baselineskip
\usepackage{lmodern}
\fi

% \ifx\UndEfiNed\url
%   \ClassError{amsldoc}{%
%     This version of amsldoc.tex must be processed\MessageBreak
%     with a newer version of amsldoc.cls (2.02 or later)}{}
% \fi

% ----------------------------------------------------------------------
\pdfminorversion=3
\usepackage[utf8]{vietnam}
\usepackage{amsxtra}
\usepackage{shortvrb}
\usepackage{afterpage}
\usepackage{amsfonts,euscript}

\MakeShortVerb{|}

% hack \url
% cannot use package hyperref. Donknow why?
% \makeatletter
% \let\old@url=\url
% \def\url#1{\textcolor{blue}{\old@url{#1}}}
% \makeatother

% ----------------------------------------------------------------------

\DeclareMathOperator{\tg}{tg}


% ----------------------------------------------------------------------


\newcommand{\entrylabel}[1]{\mbox{\textsf{#1:}}\hfil}
\newenvironment{entry}%
        {\begin{list}{}%
                {\renewcommand{\makelabel}{\entrylabel}%
                \setlength{\labelwidth}{100pt}%
                \setlength{\leftmargin}{\labelwidth}
                \addtolength{\leftmargin}{\labelsep}%
                }%
        }%
        {\end{list}}

\newlength{\mylen}
\newcommand{\lentrylabel}[1]{%
        \settowidth{\mylen}{\textsf{#1:}}%
        \ifthenelse{\lengthtest{\mylen >\labelwidth}}%
                {\parbox[b]{\labelwidth}%
                        {\makebox[0pt][l]{\textsf{#1\space:}}\\}}%
                {\hfil\textsf{#1\space:}}%
                \relax}
\newenvironment{lentry}%
        {\renewcommand{\entrylabel}{\lentrylabel}%
        \begin{entry}}%
        {\end{entry}}

% ----------------------------------------------------------------------
% :: hack heading


\makeatletter
\def\ps@empty{\let\@mkboth\@gobbletwo
  \let\@oddhead\@empty \let\@evenhead\@empty
  \let\@oddfoot\@empty
  \let\@evenfoot\@empty
  \global\topskip\normaltopskip}
\def\ps@plain{\ps@empty
  \def\@oddfoot{\normalfont\scriptsize \hfil\thepage\hfil}%
  \let\@evenfoot\@oddfoot}
%\newswitch{runhead}
\def\ps@headings{\ps@empty
%%  \def\@oddfoot{\hfil Vn\TeX\ (\url{http://vntex.org})\hfil}
  \def\@oddfoot{}
%%   \def\@evenhead{%
%%     \normalfont\small%\scriptsize
%% %%%    \hfil
%%     \thesection.
%%     \leftmark{}{}\hfil \llap{\thepage}}%
  \def\@evenhead{%
    \normalfont\small\thepage\hfil
    \leftmark{}{}}%
  \def\@oddhead{%
    \normalfont\small
    \rightmark{}{}\hfil \llap{\thepage}}%
%%%  \let\@evenhead\@oddhead
  \let\@mkboth\markboth
  \def\partmark{\@secmark\markboth\partrunhead\partname}%
  \def\chaptermark{%
    \@secmark\markboth\chapterrunhead{}}%
  \def\sectionmark{%
    \@secmark\markright\sectionrunhead\sectionname}%
}
\let\sectionname\@empty
\let\subsectionname\@empty
\let\subsubsectionname\@empty
\let\paragraphname\@empty
\let\subparagraphname\@empty
\def\leftmark{\expandafter\@firstoftwo\topmark{}{}}
\def\rightmark{\expandafter\@secondoftwo\botmark{}{}}
\long\def\@nilgobble#1\@nil{}
\def\@secmark#1#2#3#4{%
  \begingroup \let\protect\@unexpandable@protect
  \edef\@tempa{\endgroup \toks@{\protect#2{#3}{\@secnumber}}}%
  \@tempa
  \toks@\@xp{\the\toks@{#4}}%
  \afterassignment\@nilgobble\@temptokena\@themark{}\@nil
  \edef\@tempa{\@nx\@mkboth{%
    \ifx\markright#1\the\@temptokena\else\the\toks@\fi}{\the\toks@}}%
  \@tempa}
\let\@secnumber\@empty
\def\markboth#1#2{%
  \begingroup
    \@temptokena{{#1}{#2}}\xdef\@themark{\the\@temptokena}%
    \mark{\the\@temptokena}%
  \endgroup
  \if@nobreak\ifvmode\nobreak\fi\fi}
\newskip\normaltopskip
\normaltopskip=10pt \relax
\let\sectionmark\@gobble
\let\subsectionmark\@gobble
\let\subsubsectionmark\@gobble
\let\paragraphmark\@gobble

\def\partrunhead#1#2#3{%
  \@ifnotempty{#2}{\textsc{\ignorespaces#1 #2\unskip}\@ifnotempty{#3}{. }}%
  \def\@tempa{#3}%
  \ifx\@empty\@tempa\else
    \begingroup \def\\{ \ignorespaces}% defend against questionable usage
%    \MakeUppercase{\@tempa}
    \textsc{\@tempa}
    \endgroup
  \fi
}
\let\chapterrunhead\partrunhead
\let\sectionrunhead\partrunhead

\renewenvironment{thebibliography}[1]
     {\chapter*{\bibname}%
      \@mkboth{\textsc{\bibname}}{\textsc{\bibname}}%
      \list{\@biblabel{\@arabic\c@enumiv}}%
           {\settowidth\labelwidth{\@biblabel{#1}}%
            \leftmargin\labelwidth
            \advance\leftmargin\labelsep
            \@openbib@code
            \usecounter{enumiv}%
            \let\p@enumiv\@empty
            \renewcommand\theenumiv{\@arabic\c@enumiv}}%
      \sloppy
      \clubpenalty4000
      \@clubpenalty \clubpenalty
      \widowpenalty4000%
      \sfcode`\.\@m}
     {\def\@noitemerr
       {\@latex@warning{Empty `thebibliography' environment}}%
      \endlist}

\makeatother
% ----------------------------------------------------------------------

% :: title
%\title{User's Guide for the \pkg{amsmath} Package (Version~2.0)}
\title{Hướng dẫn sử dụng gói \pkg{amsmath} (phiên bản 2.0)}
%\author{American Mathematical Society}

\author{Tác giả: Hội Toán học Mỹ (AMS)\\[6pt]
        13/12/1999 (sửa đổi 25/02/2002)\\[1cm]
        Biên dịch: Ky Anh
        $\langle$\href{mailto:kyanh@o2.pl}{kyanh@o2.pl}$\rangle$\\[6pt]
  Bản dịch mới nhất 15/10/2005\\[5cm]
  \url{http://VietTUG.org}}

\date{}

%%\medskip \emph{Bản dịch số {$\mathbf{\the\buildnum}$}}}

%    Use the amsmath package and amscd package in order to print
%    examples.
\usepackage{amsmath}
\usepackage{amscd}

\makeindex % generate index data
\providecommand{\see}[2]{\textit{see} #1}

%    The amsldoc class includes a number of features useful for
%    documentation about TeX, including:
%
%    ---Commands \tex/, \amstex/, \latex/, ... for uniform treatment
%    of the various logos and easy handling of following spaces.
%
%    ---Commands for printing various common elements: \cn for command
%    names, \fn for file names (including font-file names), \env for
%    environments, \pkg and \cls for packages and classes, etc.

%    Many of the command names used here are rather long and will
%    contribute to poor linebreaking if we follow the \latex/ practice
%    of not hyphenating anything set in tt font; instead we selectively
%    allow some hyphenation.
%\allowtthyphens % defined in amsldoc.cls

\hyphenation{ac-cent-ed-sym-bol add-to-counter add-to-length align-at
  aligned-at allow-dis-play-breaks ams-art ams-cd ams-la-tex amsl-doc
  ams-symb ams-tex ams-text ams-xtra bmatrix bold-sym-bol cen-ter-tags
  eqn-ar-ray idots-int int-lim-its latex med-space neg-med-space
  neg-thick-space neg-thin-space no-int-lim-its no-name-lim-its
  over-left-arrow over-left-right-arrow over-right-arrow pmatrix
  qed-sym-bol set-length side-set small-er tbinom the-equa-tion
  thick-space thin-space un-der-left-arrow un-der-left-right-arrow
  un-der-right-arrow use-pack-age var-inj-lim var-proj-lim vmatrix
  xalign-at xx-align-at}

%    Prepare for illustrating the \vec example
\newcommand{\vect}[1]{\mathbf{#1}}

\newcommand{\booktitle}[1]{\textit{#1}}
\newcommand{\journalname}[1]{\textit{#1}}
\newcommand{\seriesname}[1]{\textit{#1}}

%    Command to insert and index a particular phrase. Doesn't work for
%    certain kinds of special characters in the argument.
\newcommand{\ii}[1]{#1\index{#1}}

\newcommand{\vstrut}[1]{\vrule width0pt height#1\relax}

%    An environment for presenting comprehensive address information:
\newenvironment{infoaddress}{%
  \par\topsep\medskipamount
  \trivlist\centering
  \item[]%
  \begin{minipage}{.7\columnwidth}%
  \raggedright
}{%
  \end{minipage}%
  \endtrivlist
}

\newenvironment{eqxample}{%
  \par\addvspace\medskipamount
  \noindent\begin{minipage}{.5\columnwidth}%
  \def\producing{\end{minipage}\begin{minipage}{.5\columnwidth}%
    \hbox\bgroup\kern-.2pt\vrule width.2pt%
      \vbox\bgroup\parindent0pt\relax
%    The 3pt is to cancel the -\lineskip from \displ@y
    \abovedisplayskip3pt \abovedisplayshortskip\abovedisplayskip
    \belowdisplayskip0pt \belowdisplayshortskip\belowdisplayskip
    \noindent}
}{%
  \par
%    Ensure that a lonely \[\] structure doesn't take up width less than
%    \hsize.
  \hrule height0pt width\hsize
  \egroup\vrule width.2pt\kern-.2pt\egroup
  \end{minipage}%
  \par\addvspace\medskipamount
}

%    The chapters are so short, perhaps we shouldn't call them by the
%    name `Chapter'. We make \chaptername read an argument in order to
%    remove a following \space or "{} " (both possibilities are present
%    in book.cls).

\renewcommand{\chaptername}[1]{}
\newcommand{\chapnum}[1]{\mdash #1\mdash }
\makeatletter
\def\@makechapterhead#1{%
  \vspace{1.5\baselineskip}%
  {\parindent \z@ \raggedright \reset@font
    \ifnum \c@secnumdepth >\m@ne
      \large\bfseries \chapnum{\thechapter}%
      \par\nobreak
      \vskip.5\baselineskip\relax
    \fi
    #1\par\nobreak
    \vskip\baselineskip
  }}
\makeatother

%    A command for ragged-right parbox in a tabular.
\newcommand{\rp}{\let\PBS\\\raggedright\let\\\PBS}

%    Non-indexed file name
\newcommand{\nfn}[1]{\texttt{#1}}

%    For the examples in the math spacing table.
%%\newcommand{\lspx}{\mbox{\rule{5pt}{.6pt}\rule{.6pt}{6pt}}}
%%\newcommand{\rspx}{\mbox{\rule[-1pt]{.6pt}{7pt}%
%%  \rule[-1pt]{5pt}{.6pt}}}
\newcommand{\lspx}{\mathord{\Rightarrow\mkern-1mu}}
\newcommand{\rspx}{\mathord{\mkern-1mu\Leftarrow}}
\newcommand{\spx}[1]{$\lspx #1\rspx$}

%    For a list of characters representing document input.
\newcommand{\clist}[1]{%
  \mbox{\ntt\spaceskip.2em plus.1em \xspaceskip\spaceskip#1}}

%    Fix weird \latex/ definition of rightmark.
\makeatletter
\def\rightmark{\expandafter\@rightmark\botmark{}{}}
%    Also turn off section marks.
\let\sectionmark\@gobble
\renewcommand{\chaptermark}[1]{%
%%  \MakeUppercase{\markboth{\rhcn#1}{\rhcn#1}}}
  \uppercase{\markboth{\rhcn#1}{\rhcn#1}}}
\newcommand{\rhcn}{\thechapter. }
\makeatother

%    Include down to \section but not \subsection, in toc:
\setcounter{tocdepth}{1}

\DeclareMathOperator{\ix}{ix}
\DeclareMathOperator{\nul}{nul}
\DeclareMathOperator{\End}{End}
\DeclareMathOperator{\xxx}{xxx}

\usepackage[colorlinks,draft=false,
pdftitle={AMSLaTeX Manual, v2.0, Vietnamese edition},
pdfauthor={American Mathematical Society, Translator: Ky Anh <kyanh@o2.pl>},
pdfsubject={Advanced Math Typesetting},
pdfkeywords={math, typesetting, formulas}]{hyperref} \input pd1supp.def
\ifx\printversion\undefined
\RequirePackage{thumbpdf}
\hypersetup{pdfpagemode=UseThumbs}
\fi
\hypersetup{plainpages=false}
\hypersetup{pdfpagelabels=false}
%\hypersetup{hypertexnames=true}
\hypersetup{naturalnames=true}

%\pagenumbering{roman}
\pagestyle{headings}
\begin{document}
%%%%%%%%%%%%%%%%%%%%%%%%%%%%%%%%%%%%%%%%%%%%%%%%%%%%%%%%%%%%%%%%%%%%%%%%
\pagestyle{empty}
\maketitle
%% \rightline{\emph{The only way to learn mathematics is to do mathematics.}}
%% \medskip
%% \rightline{{\rm P. R. HALMOS}}
\ifx\printversion\undefined
\newpage
\else
\cleardoublepage
\fi
\tableofcontents
\cleardoublepage % for better page number placement
\pagestyle{headings}
%\pagenumbering{arabic}

%%%%%%%%%%%%%%%%%%%%%%%%%%%%%%%%%%%%%%%%%%%%%%%%%%%%%%%%%%%%%%%%%%%%%%%%

\chapter{Thuật ngữ}

Dưới đây là một số thuật ngữ dùng trong tài liệu này.

\medskip
\begin{lentry}
\item[dimension] độ dài trong \latex/, ví dụ: |6pt|, |-2pc|, |5mm|,...
\item[font]kiểu chữ
\item[hyphen]tách một chữ (ở cuối dòng) với nhiều ký tự thành các phần nhỏ
        ngăn cách bởi dấu gạch ngang (dấu |hyphen|).
        Việc tách này giúp cho một chữ quá dài không tràn ra khỏi dòng.
\item[number] \emph{bla bla bla\ldots}
\item[preamble] phần nằm trước |\begin{document}| của tập tin nguồn \latex/.
\item[tag] chỉ số phương trình
\item[robust]\emph{bla bla bla\ldots}
\item[text] chuỗi các mã nguồn \latex/, ví dụ: "|xem Tiên đề~\ref{ax:1}|"
\item[typeset] sắp chữ nhờ \latex/
%       biên dịch tài liệu \latex/ nhờ chương trình |latex|, |pdflatex|,...
\item[wrap] tự động chia một dòng quá dài thành nhiều dòng nhỏ, để chúng
        bố trí vừa trên một chiều rộng cố định cho trước
\item[canh cột]bố trí các phần của phương trình theo cột (chiều đứng)
\item[dấu ngoặc](phân cách); thuật ngữ tiếng Anh là |delimiter|; là các dấu
        (,),[,],\{,\},$\vert$,...
\item[chỉ số phương trình]nhãn dùng để phân biệt các phương trình
\item[phương trình] biểu thức toán học bất kỳ được biểu diễn nhờ \latex/
\item[v-khoảng cách] khoảng cách theo chiều đứng
\end{lentry}

\chapter{Giới thiệu}

% The \pkg{amsmath} package is a \LaTeX{} package that provides
% miscellaneous enhancements for improving the information structure and
% printed output of documents that contain mathematical formulas. Readers
% unfamiliar with \LaTeX{} should refer to \cite{lamport}. If you have an
% up-to-date version of \LaTeX{}, the \pkg{amsmath} package is normally
% provided along with it. Upgrading when a newer version of the
% \pkg{amsmath} package is released can be done via
% \url{http://www.ams.org/tex/amsmath.html} or
% \url{ftp://ftp.ams.org/pub/tex/}.
Gói \pkg{amsmath} dành cho \LaTeX{} cung cấp nhiều tiện ích để |typeset|
các tài liệu Toán học phức tạp. Gói này có trong hầu hết các bản phân phối
mới của \LaTeX{}. Để lấy các thông tin cập nhật về gói này, bạn xem ở

\medskip
\rightline{\url{http://www.ams.org/tex/amsmath.html}}
\rightline{\url{ftp://ftp.ams.org/pub/tex/}}
\medskip

% This documentation describes the features of the \pkg{amsmath} package
% and discusses how they are intended to be used. It also covers some
% ancillary packages:
Tài liệu này mô tả các tính năng và của gói \pkg{amsmath} và thảo luận
về các hướng sử dụng chúng. Tài liệu cũng đề cập sơ lược về các gói
\begin{ctab}{ll}
\pkg{amsbsy}& \pkg{amstext}\\
\pkg{amscd}& \pkg{amsxtra}\\
\pkg{amsopn}
\end{ctab}
% These all have something to do with the contents of math formulas. For
% information on extra math symbols and math fonts, see \cite{amsfonts}
% and \url{http://www.ams.org/tex/amsfonts.html}. For documentation of the
% \pkg{amsthm} package or AMS document classes (\cls{amsart},
% \cls{amsbook}, etc.\@) see \cite{amsthdoc} or \cite{instr-l} and
% \url{http://www.ams.org/tex/author-info.html}.
Các gói này đều liên quan đến việc |typseset| biểu thức toán học.
Thông tin về các ký hiệu và |font| mở rộng, xem ở \cite{amsfonts} và

\medskip
\rightline{\url{http://www.ams.org/tex/amsfonts.html}}
\medskip
\noindent
Tài liệu về gói \pkg{amsthm}, các lớp\footnote{|documentclass|} AMS (\cls{amsart},
\cls{amsbook}, etc.\@) có thể tìm thấy trong \cite{amsthdoc}, \cite{instr-l}
và 

\medskip
\rightline{\url{http://www.ams.org/tex/author-info.html}}

% If you are a long-time \latex/ user and have lots of mathematics in what
% you write, then you may recognize solutions for some familiar problems
% in this list of \pkg{amsmath} features:
\bigskip
Nếu bạn đã làm việc lâu dài với \latex/ và phải |typeset| rất nhiều các biểu thức
toán học, thì với gói \pkg{amsmath}, bạn có thể tìm thấy giải pháp
cho những vấn đề hay gặp nhất:

\medskip
\begin{itemize}
\item
% A convenient way to define new `operator name' commands analogous
% to \cn{sin} and \cn{lim}, including proper side spacing and automatic
% selection of the correct font style and size (even when used in
% sub- or superscripts).
Dễ dàng định nghĩa toán tử, hàm toán học mới (tương tự như \cn{sin}, \cn{cos});
các toán tử mới sẽ tự động canh chỉnh kích thước, kiểu |font|
và khoảng cách tương quan với các phần tử khác trong biểu thức.

\item
% Multiple substitutes for the \env{eqnarray} environment to make
% various kinds of equation arrangements easier to write.
Nhiều biến thể của môi trường \env{eqnarray}
để sắp xếp nhiều loại (hệ) phương trình khác nhau.

\item
% Equation numbers automatically adjust up or down to avoid
% overprinting on the equation contents (unlike \env{eqnarray}).
Các số chỉ phương trình tự động chuyển dịch lên, xuống để tránh
tình trạng tràn trang (khắc phục nhược điểm của \env{eqnarray}).

\item
% Spacing around equals signs matches the normal spacing in the
% \env{equation} environment (unlike \env{eqnarray}).
Khoảng cách xung quanh dấu bằng ($=$) giống hệt khoảng cách bình thường
trong môi trường \env{equation} (không giống như \env{eqnarray}).

\item
% A way to produce multiline subscripts as are often used with
% summation or product symbols.
Có thể tạo chỉ số dưới, chỉ số trên với nhiều dòng (thường gặp khi 
làm việc với các ký hiệu tổng, tích)

\item
% An easy way to substitute a variant equation number for a given
% equation instead of the automatically supplied number.
Dễ dàng tạo các biến thể cho việc đánh số một phương trình cho trước
(nếu bạn không thích kiểu đánh số mặc định).

\item
% An easy way to produce subordinate equation numbers of the form
% (1.3a) (1.3b) (1.3c) for selected groups of equations.
Dễ dàng đánh số các phương trình con dạng (1.3a) (1.3b) (1.3c)
từ một nhóm các phương trình. Việc đánh số này là \emph{tự động.}
\end{itemize}

% The \pkg{amsmath} package is distributed together with some small
% auxiliary packages:
\medskip
Gói \pkg{amsmath} được phân phối cùng với một số gói bổ trợ

\begin{description}
\item[\pkg{amsmath}]
%       Primary package, provides various features for
%       displayed equations and other mathematical constructs.
        Gói chính; cung cấp rất nhiều tiện ích để biễu diễn phương trình
        và các biểu thức toán học từ đơn giản đến phức tạp.

\item[\pkg{amstext}]
%       Provides a \cn{text} command for
%       typesetting a fragment of text inside a display.
        Cung cấp lệnh \cn{text} để sắp xếp các đoạn văn bên trong
        biểu thức toán học.

\item[\pkg{amsopn}]
%       Provides \cn{DeclareMathOperator} for defining new
%       `operator names' like \cn{sin} and \cn{lim}.
        Cung cấp lệnh \cn{DeclareMathOperator} để định nghĩa
        các toán tử mới, như \cn{sin}, \cn{lim}.

\item[\pkg{amsbsy}]
% For backward compatibility this package continues
% to exist but use of the newer \pkg{bm} package that comes with \LaTeX{}
% is recommended instead.
        Gói này vẫn tồn tại để bảo đảm tính tương thích; tuy nhiên,
        bạn nên dùng gói \pkg{bm} để thay thế cho \pkg{amsbsy}.

\item[\pkg{amscd}]
% Provides a \env{CD} environment for simple
% commutative diagrams (no support for diagonal arrows).
        Cung cấp môi trường \env{CD} để biểu diễn các biểu đồ giao hoán
        đơn giản (với gói này, bạn không thể vẽ các mũi tên chéo).

\item[\pkg{amsxtra}]
% Provides certain odds and ends such as
% \cn{fracwithdelims} and \cn{accentedsymbol}, for compatibility with
% documents created using version 1.1.
        Gói bổ sung, nhằm bảo đảm tương thích với tài liệu dùng phiên bản 1.1
        của \pkg{amsmath}. Cung cấp: \cn{fracwithdelims}, \cn{accentedsymbol},...

\end{description}

% The \pkg{amsmath} package incorporates \pkg{amstext}, \pkg{amsopn}, and
% \pkg{amsbsy}. The features of \pkg{amscd} and \pkg{amsxtra}, however,
% are available only by invoking those packages separately.
\medskip
Gói \pkg{amsmath} đã bao gộp các gói \pkg{amstext}, \pkg{amsopn}, and
\pkg{amsbsy}; nghĩa là khi nạp gói \pkg{amsmath}, ba gói kia sẽ tự động
nạp theo. Còn để dùng các gói \pkg{amscd}, \pkg{amsxtra}, bạn phải nạp riêng chúng.

\chapter{Các tùy chọn của gói \pkg{amsmath}}\label{options}

Để dùng tùy chọn của gói, bạn để tên của tùy chọn vào trong phần tham
số bổ sung của lệnh nạp gói \cn{usepakage}. Các tùy chọn cách nhau bằng
dấu phảy. Ví dụ:

\medskip
\verb"\usepackage[intlimits]{amsmath}"\\
\indent\verb"\usepackage[sumlimits,intlimits]{amsmath}"

\medskip
% The \pkg{amsmath} package has the following options:
Gói \pkg{amsmath} có các tùy chọn sau đây:
\begin{description}

\item[\opt{centertags}] (mặc định)
%For a split equation, place equation
%numbers\index{equation numbers!vertical placement} vertically centered
%on the total height of the equation.
Đánh số phương trình\index{equation numbers!vertical placement}
bằng cách đặt chỉ số canh giữa theo chiều cao của phương trình.

\item[\opt{tbtags}]
%`Top-or-bottom tags': For a split equation, place
%equation numbers\index{equation numbers!vertical placement} level with
%the last (resp.\@ first) line, if numbers are on the right (resp.\@
%left).
`Top-or-bottom tags': Đặt chỉ số của phương trình ở phía bên phải,
dòng cuối cùng; hoặc ở phía bên trái, dòng đầu tiên.

\item[\opt{sumlimits}] (mặc định)
% Place the subscripts and
% superscripts\index{subscripts and superscripts!placement}\relax
% \index{limits|see{subscripts and superscripts}} of summation symbols
% above and below, in displayed equations. This option also affects other
% symbols of the same type\mdash $\prod$, $\coprod$, $\bigotimes$,
% $\bigoplus$, and so forth\mdash but excluding integrals (see below).
Đặt các chỉ số trên và dưới của các ký hiệu tổng ($\sum$) ở trên và
ở dưới (trong công thức riêng dòng). Tùy chọn này cũng ảnh hưởng
đến các ký hiệu cùng loại\mdash $\prod$, $\coprod$, $\bigotimes$,
$\bigoplus$,...\mdash (nhưng ký hiệu tích phân thì không; xem dưới đây)

\item[\opt{nosumlimits}]
% Always place the subscripts and superscripts of
% summation-type symbols to the side, even in displayed equations.
Luôn đặt chỉ số trên và chỉ số dưới của các ký hiệu dạng tổng ($\sum$, $\prod$,...)
ở bên cạnh, ngay cả trong công thức riêng dòng. Ví dụ $\sum_0^1$.

\item[\opt{intlimits}]
% Like \opt{sumlimits}, but for
% integral\index{integrals!placement of limits} symbols.
Tương tự như \opt{sumlimits}, nhưng cho ký hiệu tích phân.

\item[\opt{nointlimits}] (mặc định) Ngược với \opt{intlimits}.

\item[\opt{namelimits}] (mặc định)
% Like \opt{sumlimits}, but for certain
% `operator names' such as $\det$, $\inf$, $\lim$, $\max$, $\min$, that
% traditionally have subscripts \index{subscripts and
% superscripts!placement} placed underneath when they occur in a displayed
% equation.
Tương tự \opt{sumlimits}, nhưng cho một số toán tử như
$\det$, $\inf$, $\lim$, $\max$, $\min$;
các toán tử này theo truyền thống thường có chỉ số đặt bên dưới toán tử
(chế độ công thức riêng dòng).

\item[\opt{nonamelimits}] Ngược với \opt{namelimits}.

% To use one of these package options, put the option name in the optional
% argument of the \cn{usepackage} command\mdash e.g.,
% \verb"\usepackage[intlimits]{amsmath}".

% The \pkg{amsmath} package also recognizes the following options which
% are normally selected (implicitly or explicitly) through the
% \cn{documentclass} command, and thus need not be repeated in the option
% list of the \cn{usepackage}|{amsmath}| statement.

\item[\opt{leqno}]
% Place equation numbers on the left.\index{equation
% numbers!left or right placement}
Đặt chỉ số phương trình bên trái.

\item[\opt{reqno}]
% Place equation numbers on the right.
Đặt chỉ số phương trình bên phải.

\item[\opt{fleqn}]
% Position equations at a fixed indent from the left
% margin rather than centered in the text column.\index{displayed
% equations!centering}
Biểu diễn phương trình với lề trái cố định; theo mặc định, các phương
trình được canh giữa (do đó, lề trái của chúng \emph{thay đổi}).

\end{description}

% The \pkg{amsmath} package also recognizes the following options which
% are normally selected (implicitly or explicitly) through the
% \cn{documentclass} command, and thus need not be repeated in the option
% list of the \cn{usepackage}|{amsmath}| statement.
\medskip
Đối với ba tùy chọn cuối cùng (\opt{leqno}, \opt{reqno}, \opt{fleqn}),
bạn có thể để chúng vào phần tham số bổ sung của \cn{documentclass};
gói \pkg{amsmath} nhận biết điều này và do đó bạn không cần lặp lại
khi nạp gói bằng \cn{usepackage}|{amsmath}|:

\medskip
\verb"\documentclass[reqno]{report}"\par
\indent\verb"\usepackage{amsmath}"\% có tác dụng như \verb"\usepackage[reqno]{amsmath}"


%%%%%%%%%%%%%%%%%%%%%%%%%%%%%%%%%%%%%%%%%%%%%%%%%%%%%%%%%%%%%%%%%%%%%%%%
%\chapter{Displayed equations}
\chapter{Biểu diễn phương trình}

\section{Giới thiệu}
% The \pkg{amsmath} package provides a number of additional displayed
% equation structures\index{displayed
%   equations}\index{equations|see{displayed equations}} beyond the ones
% provided in basic \latex/. The augmented set includes:
Gói \pkg{amsmath} cung cấp thêm các môi trường biểu diễn phương trình sau
đây, bên cạnh các môi trường chuẩn của \latex/:
\begin{verbatim}
  equation     equation*     align       align*
  gather       gather*       flalign     flalign*
  multline     multline*     alignat     alignat*
  split
\end{verbatim}
% (Although the standard \env{eqnarray} environment remains available,
% it is better to use \env{align} or \env{equation}+\env{split} instead.)
(Mặc dù môi trường chuẩn \env{eqnarray} vẫn dùng được, nhưng tốt hơn hết
nên dùng môi trường \env{align} hoặc tổ hợp \env{equation}+\env{split}.)

\medskip

% Except for \env{split}, each environment has both starred and unstarred
% forms, where the unstarred forms have automatic numbering using
% \latex/'s \env{equation} counter.
Ngoại trừ \env{split}, mỗi môi trường đều có hai dạng:
\emph{có sao (*)} và \emph{không sao};
các môi trường không sao sẽ sử dụng bộ đếm \env{equation} của \latex/
để đánh số các phương trình một cách tự động (do đó, ta gọi
chúng là \emph{môi trường có đánh số}).
% You can suppress the number on any
% particular line by putting \cn{notag} before the \cn{\\};
Bạn có thể bỏ qua việc đánh số cho bất kỳ dòng phương trình nào
bằng cách đặt lệnh \cn{notag} trước khi dùng \cn{\\};
% you can also
% override\index{equation numbers!overriding} it with a tag of your own
% using \cn{tag}|{|\<label>|}|, where \<label> means arbitrary text such
% as |$*$| or |ii| used to \qq{number} the equation. 
cũng có thể thay đổi kiểu đánh số cho dòng phương trình cụ thể, bằng
cách dùng \cn{tag}|{|\<label>|}|, ở đây \<label> là |text| bất kỳ,
chẳng hạn |$*$| hoặc |ii|. Theo mặc định, \<label> của \cn{tag} sẽ được
đặt trong cặp dấu ngoặc đơn, ví dụ (3.32);
nếu không muốn điều này xảy ra, bạn có thể dùng \cn{tag*}.
% There is also a
% \cn{tag*} command that causes the text you supply to be typeset
% literally, without adding parentheses around it.
\emph{Để ý rằng,}
% \cn{tag} and \cn{tag*}
% can also be used within the unnumbered versions of all the \pkg{amsmath}
% alignment structures. Some examples of the use of \cn{tag} may be found
% in the sample files \fn{testmath.tex} and \fn{subeqn.tex} provided with
% the \pkg{amsmath} package.
\cn{tag} và \cn{tag*} có thể dùng với mọi môi trường đã liệt kê ở trên,
chứ không phải với chỉ các môi trường có đánh số (không sao).
Ví dụ về việc dùng \cn{tag} có thể tìm thấy trong \fn{testmath.tex}
và \fn{subeqn.tex} được phân phối cùng với tài liệu này.

% The \env{split} environment is a special subordinate form that is used
% only \emph{inside} one of the others. It cannot be used inside
% \env{multline}, however.
\medskip
Môi trường \env{split} là một dạng đặc biệt, chỉ có thể được dùng
bên trong các môi trường khác. Tuy nhiên, nó lại không thể dùng bên
trong \env{multline}.

% In the structures that do alignment (\env{split}, \env{align} and
% variants), relation symbols have an \verb'&' before them but not
% after\mdash unlike \env{eqnarray}. Putting the \verb'&' after the
% relation symbol will interfere with the normal spacing; it has to go
% before.
\medskip
Với các môi trường có chức năng canh cột (\env{split}, \env{align},...),
các ký hiệu quan hệ (dấu $=$, $>$, $\le$, ...) phải được đặt sau dấu \verb'&'.
Đây là điểm khác biệt so với \env{eqnarray}.

\begin{table}[p]
\caption[]{
% Comparison of displayed equation environments (vertical lines
% indicating nominal margins)
So sánh các môi trường biểu diễn phương trình (đường thẳng đứng
dùng để chỉ lề trái, phải của tờ giấy tưởng tượng)
}
\label{displays}
\renewcommand{\theequation}{\arabic{equation}}
\begin{eqxample}
\begin{verbatim}
\begin{equation*}
a=b
\end{equation*}
\end{verbatim}
\producing
\begin{equation*}
a=b
\end{equation*}
\end{eqxample}

\begin{eqxample}
\begin{verbatim}
\begin{equation}
a=b
\end{equation}
\end{verbatim}
\producing
\begin{equation}
a=b
\end{equation}
\end{eqxample}

\begin{eqxample}
\begin{verbatim}
\begin{equation}\label{xx}
\begin{split}
a& =b+c-d\\
 & \quad +e-f\\
 & =g+h\\
 & =i
\end{split}
\end{equation}
\end{verbatim}
\producing
\begin{equation}\label{xx}
\begin{split}
a& =b+c-d\\
 & \quad +e-f\\
 & =g+h\\
 & =i
\end{split}
\end{equation}
\end{eqxample}

\begin{eqxample}
\begin{verbatim}
\begin{multline}
a+b+c+d+e+f\\
+i+j+k+l+m+n
\end{multline}
\end{verbatim}
\producing
\begin{multline}
a+b+c+d+e+f\\
+i+j+k+l+m+n
\end{multline}
\end{eqxample}

\begin{eqxample}
\begin{verbatim}
\begin{gather}
a_1=b_1+c_1\\
a_2=b_2+c_2-d_2+e_2
\end{gather}
\end{verbatim}
\producing
\begin{gather}
a_1=b_1+c_1\\
a_2=b_2+c_2-d_2+e_2
\end{gather}
\end{eqxample}

\begin{eqxample}
\begin{verbatim}
\begin{align}
a_1& =b_1+c_1\\
a_2& =b_2+c_2-d_2+e_2
\end{align}
\end{verbatim}
\producing
\begin{align}
a_1& =b_1+c_1\\
a_2& =b_2+c_2-d_2+e_2
\end{align}
\end{eqxample}

\begin{eqxample}
\begin{verbatim}
\begin{align}
a_{11}& =b_{11}&
  a_{12}& =b_{12}\\
a_{21}& =b_{21}&
  a_{22}& =b_{22}+c_{22}
\end{align}
\end{verbatim}
\producing
\begin{align}
a_{11}& =b_{11}&
  a_{12}& =b_{12}\\
a_{21}& =b_{21}&
  a_{22}& =b_{22}+c_{22}
\end{align}
\end{eqxample}

\begin{eqxample}
\begin{verbatim}
\begin{flalign*}
a_{11}& =b_{11}&
  a_{12}& =b_{12}\\
a_{21}& =b_{21}&
  a_{22}& =b_{22}+c_{22}
\end{flalign*}
\end{verbatim}
\producing
\begin{flalign*}
a_{11}& =b_{11}&
  a_{12}& =b_{12}\\
a_{21}& =b_{21}&
  a_{22}& =b_{22}+c_{22}
\end{flalign*}
\end{eqxample}
\end{table}

\afterpage{\clearpage}

%\section{Single equations}
\section{Phương trình đơn}

% The \env{equation} environment is for a single equation with an
% automatically generated number. The \env{equation*} environment is the
% same except for omitting the number.%
Môi trường \env{equation} dùng biểu diễn phương trình đơn
và tự động đánh số cho phương trình đó. Môi trường \env{equation*}
có tác dụng tương tự, nhưng không đánh số.%
%%%%%%%%%%%%%%%%%%%%%%%%%%%%%%%%%%%%%%%%%%%%%%%%%%%%%%%%%%%%%%%%%%%%%%%%
% \footnote{Basic \latex/ doesn't provide an \env{equation*} environment,
% but rather a functionally equivalent environment named
% \env{displaymath}.}
\footnote{\latex/ chuẩn không có môi trường \env{equation*}, mà
chỉ có dạng tương đương là \env{displaymath}.}

\medskip

Hai môi trường này chỉ biểu diễn phương trình trên đúng một dòng;
bạn không thể dùng \cn{\\} bên trong hai môi trường đó.
Nếu biểu thức quá dài, sẽ xảy tràn trang. Hãy xem mục tiếp theo.

%\section{Split equations without alignment}
\section{Chia nhỏ phương trình nhưng không canh cột}

% The \env{multline} environment is a variation of the \env{equation}
% environment used for equations that don't fit on a single line. The
% first line of a \env{multline} will be at the left margin and the last
% line at the right margin, except for an indention on both sides in the
% amount of \cn{multlinegap}. Any additional lines in between will be
% centered independently within the display width (unless the \opt{fleqn}
% option is in effect).
Môi trường \env{multline} là biến thể của \env{equation}, cho phép
biểu diễn các phương trình quá dài, không thể bố trí vừa khít trên một dòng.
Trong môi trường này, bạn dùng \cn{\\} để tách các dòng.
Dòng đầu tiên sẽ canh ở lề trái, dòng cuối cùng được canh ở lề phải;
ngay trước dòng đầu tiên và ngay sau dòng cuối cùng là khoảng trắng |indent|
được cung cấp bởi biến độ dài \cn{multlinegap}.
Các dòng còn lại sẽ được canh giữa trang, trừ khi bạn dùng tùy chọn \opt{fleqn}.

% Like \env{equation}, \env{multline} has only a single equation number
% (thus, none of the individual lines should be marked with \cn{notag}).
% The equation number is placed on the last line (\opt{reqno} option) or
% first line (\opt{leqno} option); vertical centering as for \env{split}
% is not supported by \env{multline}.
\medskip
Giống như \env{equation}, môi trường \env{multline} chỉ đánh số tất cả
các dòng của nó bởi đúng một chỉ số (do đó, bạn không thể dùng \cn{notag}
cho riêng dòng nào trong môi trường). Chỉ số được đánh sẽ được ở dòng
cuối cùng (tùy chọn \opt{reqno}) hoặc dòng đầu tiên (tùy chọn \opt{leqno});
kiểu đánh số canh giữ như \env{split} không được hỗ trợ.

% It's possible to force one of the middle lines to the left or right with
% commands \cn{shoveleft}, \cn{shoveright}. These commands take the entire
% line as an argument, up to but not including the final \cn{\\}; for
% example
\medskip
Có thể làm cho các dòng giữa của môi trường dịch chuyển qua trái hoặc
qua phải bằng cách lệnh \cn{shoveleft} hay \cn{shoveright}.
Các lệnh này sẽ nhận cả dòng cần dịch chuyển làm tham số
(nhưng trừ ra \cn{\\} ở cuối dòng)
\begin{multline}
\framebox[.65\columnwidth]{A}\\
\framebox[.5\columnwidth]{B}\\
\shoveright{\framebox[.55\columnwidth]{C}}\\
\framebox[.65\columnwidth]{D}
\end{multline}
\begin{verbatim}
\begin{multline}
\framebox[.65\columnwidth]{A}\\
\framebox[.5\columnwidth]{B}\\
%
\shoveright{\framebox[.55\columnwidth]{C}}\\
%
\framebox[.65\columnwidth]{D}
\end{multline}
\end{verbatim}

% The value of \cn{multlinegap} can be changed with the usual \latex/
% commands \cn{setlength} or \cn{addtolength}.
Giái trị của biến độ dài \cn{multlinegap} có thể thay đổi nhờ các
lệnh của \latex/ là \cn{setlength} và \cn{addtolength}.

%\section{Split equations with alignment}
\section{Chia nhỏ phương trình và canh cột}

% Like \env{multline}, the \env{split} environment is for \emph{single}
% equations that are too long to fit on one line and hence must be split
% into multiple lines.
Giống như \env{multline}, môi trường \env{split} dùng biểu diễn
\emph{các phương trình đơn} quá dài, không bố trí vừa trên riêng một dòng
và do đó phải chia chúng thành nhiều dòng.
% Unlike \env{multline}, however, the \env{split}
% environment provides for alignment among the split lines, using |&| to
% mark alignment points.
Nhưng không như \env{multline}, môi trường \env{split}
cho phép canh cột các dòng nhờ sử dụng |&| để đánh dấu cột.
% Unlike the other \pkg{amsmath} equation
% structures, the \env{split} environment provides no numbering, because
% it is intended to be used \emph{only inside some other displayed
%   equation structure}, usually an \env{equation}, \env{align}, or
% \env{gather} environment, which provides the numbering. For example:
Không như các môi trường phương trình khác của gói \pkg{amsmath},
\env{split} không đánh số phương trình, bởi \emph{nó chỉ có thể dùng
bên trong khác môi trường khác}, thường là
\env{equation}, \env{align}, hay \env{gather}
(các môi trường vừa nhắc đến thực hiện đánh số). Ví dụ:
\begin{equation}\label{e:barwq}\begin{split}
H_c&=\frac{1}{2n} \sum^n_{l=0}(-1)^{l}(n-{l})^{p-2}
\sum_{l _1+\dots+ l _p=l}\prod^p_{i=1} \binom{n_i}{l _i}\\
&\quad\cdot[(n-l )-(n_i-l _i)]^{n_i-l _i}\cdot
\Bigl[(n-l )^2-\sum^p_{j=1}(n_i-l _i)^2\Bigr].
\kern-2em % adjust equation body to the right [mjd,13-Nov-1994]
\end{split}\end{equation}

\begin{verbatim}
\begin{equation}\label{e:barwq}
\begin{split}
H_c &=\frac{1}{2n} \sum^n_{l=0}(-1)^{l}(n-{l})^{p-2}
            \sum_{l _1+\dots+ l _p=l}\prod^p_{i=1}\binom{n_i}{l _i}
\\  &\quad\cdot[(n-l )-(n_i-l _i)]^{n_i-l _i}
        \cdot\Bigl[(n-l )^2-\sum^p_{j=1}(n_i-l _i)^2\Bigr].
\end{split}
\end{equation}
\end{verbatim}

% The \env{split} structure should constitute the entire body of the
% enclosing structure, apart from commands like \cn{label} that produce no
% visible material.
Phần phương trình biểu diễn nhờ \env{split}
nên được xem là một phần riêng biệt;
lệnh \cn{label} đặt bên trong môi trường sẽ không có tác dụng.

%\section{Equation groups without alignment}
\section{Nhóm phương trình không canh cột}

% The \env{gather} environment is used for a group of consecutive
% equations when there is no alignment desired among them; each one is
% centered separately within the text width (see Table~\ref{displays}).
% Equations inside \env{gather} are separated by a \cn{\bslash} command.
% Any equation in a \env{gather} may consist of a \verb'\begin{split}'
%  \dots\ \verb'\end{split}' structure\mdash for example:
Môi trường \env{gather} thường dùng biểu diễn nhóm các phương trình
mà không quan tâm đến việc canh cột giữa chúng; mỗi phương trình trong nhóm
sẽ nằm ở dòng riêng (xem Bảng~\ref{displays}). Các phương trình bên trong
môi trường \env{gather} được phân biệt nhờ \cn{\\}. Bất kỳ phương trình
nào bên trong \env{gather} cũng có thể là ở dạng
\verb'\begin{split}' \dots\ \verb'\end{split}'
\begin{verbatim}
\begin{gather}
  phương trình thứ nhất\\
  \begin{split}
    phương trình & thứ hai\\
           & với hai dòng
  \end{split}
  \\
  phương trình thứ ba
\end{gather}
\end{verbatim}

%\section{Equation groups with mutual alignment}
\section{Nhóm phương trình có canh cột}

% The \env{align} environment is used for two or more equations when
% vertical alignment is desired; usually binary relations such as equal
% signs are aligned (see Table~\ref{displays}).
Môi trường \env{align} biểu diễn nhóm các phương trình và cho phép
canh cột giữa các phương trình. Xem Bảng~\ref{displays}.

% To have several equation columns side-by-side, use extra ampersands
% to separate the columns:
\medskip
Để có vài cột phương trình đi cùng nhau, hãy \emph{dùng thêm} các dấu |&|
\begin{align}
x&=y       & X&=Y       & a&=b+c\\
x'&=y'     & X'&=Y'     & a'&=b\\
x+x'&=y+y' & X+X'&=Y+Y' & a'b&=c'b
\end{align}
%
\begin{verbatim}
\begin{align}
x&=y       & X&=Y       & a&=b+c\\
x'&=y'     & X'&=Y'     & a'&=b\\
x+x'&=y+y' & X+X'&=Y+Y' & a'b&=c'b
\end{align}
\end{verbatim}
% Line-by-line annotations on an equation can be done by judicious
% application of \cn{text} inside an \env{align} environment:
Có thể thêm chú thích cho phương trình (nằm cùng dòng với phương trình)
bằng cách dùng lệnh \cn{text} bên trong môi trường \env{align}:
\begin{align}
x& = y_1-y_2+y_3-y_5+y_8-\dots
                    && \text{theo \eqref{eq:C}}\\
 & = y'\circ y^*    && \text{theo \eqref{eq:D}}\\
 & = y(0) y'        && \text {theo Tiên đề 1.}
\end{align}
%
\begin{verbatim}
\begin{align}
x& = y_1-y_2+y_3-y_5+y_8-\dots
                    && \text{theo \eqref{eq:C}}\\
 & = y'\circ y^*    && \text{theo \eqref{eq:D}}\\
 & = y(0) y'        && \text {theo Tiên đề 1.}
\end{align}
\end{verbatim}
% A variant environment \env{alignat} allows the horizontal space between
% equations to be explicitly specified. This environment takes one argument,
% the number of \qq{equation columns}: count the maximum number of \verb'&'s
% in any row, add 1 and divide by 2.
Biến thể \env{alignat} của môi trường \env{align} cho phép xác định khoảng
cách (theo chiều ngang), nhờ đó sẽ tiết kiệm không gian hơn. Môi trường này
nhận một tham số, là \emph{"số cột phương trình"}. Để có số này, bạn
đếm số dấu |&| trong các phương trình, chọn ra số lớn nhất, cộng số đó
với 1 rồi chia kết quả  cho 2. So sánh ví dụ dưới đây với ví dụ ở ngay trên.
\begin{alignat}{2}
x& = y_1-y_2+y_3-y_5+y_8-\dots
                  &\quad& \text{theo \eqref{eq:C}}\\
 & = y'\circ y^*  && \text{theo \eqref{eq:D}}\\
 & = y(0) y'      && \text {theo Tiên đề 1.}
\end{alignat}
%
\begin{verbatim}
\begin{alignat}{2}
x& = y_1-y_2+y_3-y_5+y_8-\dots
                  &\quad& \text{theo \eqref{eq:C}}\\
 & = y'\circ y^*  && \text{theo \eqref{eq:D}}\\
 & = y(0) y'      && \text {theo Tiên đề 1.}
\end{alignat}
\end{verbatim}

%\section{Alignment building blocks}
\section{Canh cột các khối phương trình}

% Like \env{equation}, the multi-equation environments \env{gather},
% \env{align}, and \env{alignat} are designed to produce a structure
% whose width is the full line width. This means, for example, that one
% cannot readily add parentheses around the entire structure. 
Giống như \env{equation}, các môi trường phương trình nhiều dòng
\env{gather}, \env{align} và \env{alignat} bố trí phương trình \emph{trên
cả chiều rộng của dòng.} Điều này dẫn tới, chẳng hạn, ta không thể
thêm các dấu ngoặc để bao quanh phương trình.

% But variants \env{gathered}, \env{aligned}, and \env{alignedat} are provided whose
% total width is the actual width of the contents; thus they can be used
% as a component in a containing expression. E.g.,
\medskip
Các môi trường  \env{gathered}, \env{aligned} và \env{alignedat} khắc phục
nhược điểm trên; chúng biểu diễn phương trình \emph{trên một chiều rộng đúng bằng chiều
rộng của nội dung phương trình.} Nhờ đó, chúng có thể xem như một phần của biểu thức,
ví dụ:
\begin{equation*}
\left.% dùng \left. để bỏ đi dấu ngoặc bên trái
\begin{aligned}
  B'&=-\partial\times E,\\
  E'&=\partial\times B - 4\pi j,
\end{aligned}
\right\}
\qquad \text{Phương trình Maxwell}
\end{equation*}
\begin{verbatim}
\begin{equation*}
\left.% dùng \left. để bỏ đi dấu ngoặc bên trái
\begin{aligned}
  B'&=-\partial\times E,\\
  E'&=\partial\times B - 4\pi j,
\end{aligned}
\right\}
\qquad \text{Phương trình Maxwell}
\end{equation*}
\end{verbatim}
% Like the \env{array} environment, these \texttt{-ed} variants also take
% an optional \verb'[t]' or \verb'[b]' argument to specify vertical
% positioning.

\medskip
Giống như \env{array}, các biến thể \texttt{-ed} nhận tham số tùy chọn (bổ sung)
\verb'[t]' hoặc \verb'[b]' để chỉ vị trí biểu diễn (`t' cho top-ở trên;
`b' cho bottom-ở dưới)

% \qq{Cases} constructions like the following are common in
% mathematics:
\medskip
Biến thể \env{cases} dùng để \emph{chia các trường hợp}:
\begin{equation}\label{eq:C}
P_{r-j}=
  \begin{cases}
    0&  \text{nếu $r-j$ là số lẻ},\\
    r!\,(-1)^{(r-j)/2}&  \text{nếu $r-j$ là số chẵn}.
  \end{cases}
\end{equation}
% and in the \pkg{amsmath} package there is a \env{cases} environment to
% make them easy to write:
\begin{verbatim}
P_{r-j}=\begin{cases}
    0&  \text{nếu $r-j$ là số lẻ},\\
    r!\,(-1)^{(r-j)/2}&  \text{nếu $r-j$ là số chẵn}.
  \end{cases}
\end{verbatim}
% Notice the use of \cn{text} (cf.~\secref{text}) and the nested
% math formulas.
Để ý đến việc dùng lệnh \cn{text} (xem Mục~\secref{text}) và biểu thức
toán |$...$| bên trong \cn{text}.

\medskip
Cần nhớ: các biến thể \env{cases} và \env{-ed} chỉ được dùng bên trong
các môi trường phương trình khác\mdash tương tự như \env{split}.


%\section{Adjusting tag placement}
\section{Thay đổi vị trí chỉ số phương trình}

% Placing equation numbers can be a rather complex problem in multiline
% displays. The environments of the \pkg{amsmath} package try hard to
% avoid overprinting an equation number on the equation contents, if
% necessary moving the number down or up to a separate line.
Thay đổi vị trí đặt chỉ số phương trình là vấn đề khá phức tạp trong
biểu diễn phương trình nhiều dòng. Các môi trường của gói \pkg{amsmath}
luôn cố gắng để không đặt chỉ số phương trình \emph{đè} lên nội dung phương
trình, và nếu cần thiết thì đặt chỉ số ở trên hoặc dưới một chút,
thậm chí, ở riêng một dòng.
% Difficulties
% in accurately calculating the profile of an equation can occasionally
% result in number movement that doesn't look right. 
Nhưng việc tính chính xác thuộc tính của phương trình là điều khó khăn;
do đó, đôi khi việc dịch chuyển chỉ số phương trình do gói \pkg{amsmath}
thực hiện sẽ không mang lại kết quả đẹp mắt.
% There is a \cn{raisetag} command provided to adjust the vertical position of the
% current equation number, if it has been shifted away from its normal
% position.
Trong trường hợp đó, bạn phải \emph{làm bằng tay} nhờ lệnh \cn{raisetag}.
%To move a particular number up by six points, write |\raisetag{6pt}|.
Để dịch chuyển chỉ số lên phía trên một quãng $6pt$, dùng |\raisetag{6pt}|;
còn để dịch xuống, dùng chẳng hạn |\raisetag{-6pt}|.
% This kind of adjustment is fine tuning like line
% breaks and page breaks, and should therefore be left undone until your
% document is nearly finalized, or you may end up redoing the fine tuning
% several times to keep up with changing document contents.

\medskip
Việc điều chỉnh ví trí chỉ số bằng \cn{raisetag} là công việc tỉ mỉ, cũng như việc
ngắt dòng, ngắt trang bằng \cn{linebreak}, \cn{pagebreak}. Bạn \emph{chỉ nên}
thực hiện điều chỉnh khi tài liệu của bạn gần như hoàn tất, và cần làm
lại mỗi khi bạn thay đổi nội dung tài liệu.

%\section{Vertical spacing and page breaks in multiline displays}
\section{V-khoảng cách và ngắt trang trong biểu diễn nhiều dòng}

% You can use the \cn{\\}|[|\<dimension>|]| command to get extra vertical
% space between lines in all the \pkg{amsmath} displayed equation
% environments, as is usual in \latex/. 
Bạn có thể dùng \cn{\\}|[|\<dimension>|]| để thêm các v-khoảng cách giữa
các dòng trong mọi môi trường biểu diễn phương trình của gói \pkg{amsmath};
điều này cũng tương tự trong \latex/.
% When the \pkg{amsmath} package is
% in use \ii{page breaks} between equation lines are normally disallowed;
% the philosophy is that page breaks in such material should receive
% individual attention from the author.


\medskip
Khi dùng gói \pkg{amsmath}, việc \ii{ngắt trang} giữa các dòng của phương trình
không (tự động) xảy ra; bởi việc ngắt trang như vậy sẽ làm cho phương trình trở nên
khó theo dõi đối với độc giả.
% To get an individual page break
% inside a particular displayed equation, a \cn{displaybreak} command is
% provided.
Để ngắt trang bên trong phương trình, bạn phải làm bằng tay nhờ lệnh
\cn{displaybreak}.
% \cn{displaybreak} is best placed immediately before the
% \cn{\\} where it is to take effect.  Like \latex/'s \cn{pagebreak},
% \cn{displaybreak} takes an optional argument between 0 and 4 denoting
% the desirability of the pagebreak. |\displaybreak[0]| means \qq{it is
%   permissible to break here} without encouraging a break;
% \cn{displaybreak} with no optional argument is the same as
% |\displaybreak[4]| and forces a break.
Nơi đặt \cn{displaybreak} tốt nhất là ngay trước \cn{\\} ở dòng cần ngắt trang.
Giống như lệnh \cn{pagebreak} của \latex/, lệnh \cn{displaybreak} nhận
tham số bổ sung (tùy chọn) là một trong các số 0, 1, 2, 3, 4;
tham số này cho biết mức độ ngắt trang. Dùng |\displaybreak[0]| để hàm ý
\qq{không được ngắt trang ở đây},...; trong khi |\displaybreak|
(không có tham số bổ sung), như |\displaybreak[4]|, hàm ý
\qq{ngắt trang ở đây}.

% If you prefer a strategy of letting page breaks fall where they may,
% even in the middle of a multi-line equation, then you might put
% \cn{allowdisplaybreaks}\texttt{[1]} in the preamble of your document. An
% optional argument 1\ndash 4 can be used for finer control: |[1]| means
% allow page breaks, but avoid them as much as possible; values of 2,3,4
% mean increasing permissiveness. When display breaks are enabled with
% \cn{allowdisplaybreaks}, the \cn{\\*} command can be used to prohibit a
% pagebreak after a given line, as usual.
\medskip
Việc ngắt trang dùng \cn{displaybreak} như trên chỉ có thể làm đối với
phương trình cụ thể. Nếu muốn cung cấp cho gói \pkg{amsmath} \qq{giấy phép ngắt trang}
một cách toàn cục (cho mọi phương trình nhiều dòng),
bạn dùng \cn{allowdisplaybreaks}\texttt{[1]} trong phần |preamble| của tài liệu.
Tham số bổ sung của lệnh này nhận giá trị từ 1 đến 4: |[1]| có nghĩa là
cho phép ngắt trang, nhưng tránh việc đó nếu có thể được; các giá trị 2,3,4
càng tăng khả năng ngắt trang. Khi dùng \cn{allowdisplaybreaks}, thì lệnh
\cn{\\*} dùng để cấm xảy ra ngắt trang sau một dòng cụ thể.

\medskip
\begin{bfseries}
% Note: Certain equation environments wrap their contents in an
% unbreakable box, with the consequence that neither \cn{displaybreak} nor
% \cn{allowdisplaybreaks} will have any effect on them. These include
Các môi trường \env{split}, \env{aligned}, \env{gathered} và \env{alignedat}
có khả năng |wrap| nội dung phương trình trong các hộp không cho phép ngắt
(|unbreakable box|); hệ quả là trong trường hợp đó, cả hai lệnh
\cn{displaybreak} và \cn{allowdisplaybreaks} đều không có tác dụng.
\end{bfseries}

%\section{Interrupting a display}
\section{Xen liên từ vào giữa các phương trình}

% The command \cn{intertext} is used for a short interjection of one or
% two lines of text\index{text fragments inside math} in the middle of a
% multiple-line display structure (see also the \cn{text} command in
% \secref{text}). Its salient feature is preservation of the alignment,
% which would not happen if you simply ended the display and then started
% it up again afterwards. \cn{intertext} may only appear right after a
% \cn{\\} or \cn{\\*} command. Notice the position of the word \qq{and} in
% this example.
Lệnh \cn{intertext} dùng để xen liên từ\footnote{hay cái khác, tùy bạn!}
vào giữa các dòng của biểu diễn nhiều dòng (xem thêm về \cn{text}
ở Mục~\secref{text}). Nhờ đó, để bảm đảm việc canh cột các phương trình,
bạn đỡ phải tốn công kết thúc một biểu diễn rồi sau đó bắt đầu lại.
Lệnh \cn{intertext} chỉ có thể đặt ngay sau \cn{\\} hoặc \cn{\\*}.
Hãy chú ý đến sự xuất hiện của liên từ \qq{và} trong ví dụ sau.
\begin{align}
A_1&=N_0(\lambda;\Omega')-\phi(\lambda;\Omega'),\\
A_2&=\phi(\lambda;\Omega')-\phi(\lambda;\Omega),\\
\intertext{và}
A_3&=\mathcal{N}(\lambda;\omega).
\end{align}
\begin{verbatim}
\begin{align}
A_1&=N_0(\lambda;\Omega')-\phi(\lambda;\Omega'),\\
A_2&=\phi(\lambda;\Omega')-\phi(\lambda;\Omega),\\
\intertext{và}
A_3&=\mathcal{N}(\lambda;\omega).
\end{align}
\end{verbatim}

%\section{Equation numbering}
\section{Đánh số phương trình}

%\subsection{Numbering hierarchy}
\subsection{Hệ thống chỉ số phương trình}
% In \latex/ if you wanted to have equations numbered within
% sections\mdash that is, have
% equation numbers (1.1), (1.2), \dots, (2.1), (2.2),
% \dots, in sections 1, 2, and so forth\mdash you could redefine
% \cn{theequation} as suggested in the \latex/ manual \cite[\S6.3, \S C.8.4]{lamport}:
Với \latex/, nếu bạn muốn đánh số phương trình theo mục\mdash
nghĩa là chỉ số phương trình có dạng (1.1), (1.2), \dots, (2.1), (2.2), \dots,
trong chương 1, 2, v.v...\mdash bạn có thể định nghĩa lại lệnh
\cn{theequation} như lời khuyên trong sổ tay (manual) \latex/ 
\cite[\S6.3, \S C.8.4]{lamport}:

\medskip
\begin{verbatim}
\renewcommand{\theequation}{\thesection.\arabic{equation}}
\end{verbatim}

% This works pretty well, except that the equation counter won't be reset
% to zero at the beginning of a new section or chapter, unless you do it
% yourself using \cn{setcounter}.
\medskip
Cách làm như trên cho kết quả như ý, nhưng có điều bất tiện là chỉ số
phương trình sẽ không tự động đặt về 0 khi chuyển%
\footnote{ở mục 1, các phương trình được đánh số (1.1), (1.2), \ldots;
còn qua mục 2, chỉ số phải bắt đầu từ (2.1) rồi đến (2.2), \ldots;
nghĩa là ta phải đặt bộ đếm về 0 khi vừa qua mục 2.}
% ----------------------------------------------------------------------
từ mục này qua mục khác, và bạn phải làm điều đó bằng tay nhờ lệnh \cn{setcounter}.
% To make this a little more convenient,
% the \pkg{amsmath} package provides a command\index{equation
% numbers!hierarchy} \cn{numberwithin}. To have equation numbering tied to
% section numbering, with automatic reset of the equation counter, write
Để cuộc sống dễ dàng hơn, gói \pkg{amsmath} cung cấp lệnh \cn{numberwithin};
nhờ lệnh này, bạn có thể đánh số phương trình theo mục, với chỉ số phương trình
\emph{tự động đặt về 0} khi sang mục mới.

\medskip
\begin{verbatim}
\numberwithin{equation}{section}
\end{verbatim}

\medskip
% As its name implies, the \cn{numberwithin} command can be applied to
% any counter, not just the \texttt{equation} counter.
Để ý rằng, lệnh \cn{numberwithin} có tác dụng đối với mọi bộ đếm, chứ
không riêng gì bộ đếm \texttt{equation}.

%\subsection{Cross references to equation numbers}
\subsection{Tham khảo chéo đến chỉ số phương trình}

% To make cross-references to equations easier, an \cn{eqref}
% command\index{equation numbers!cross-references} is provided. This
% automatically supplies the parentheses around the equation number. I.e.,
% if \verb'\ref{abc}' produces 3.2 then \verb'\eqref{abc}' produces (3.2).
Việc tham khảo chéo đến phương trình được thực hiện dễ dàng hơn với lệnh
\cn{eqref}. Lệnh này tự động đặt chỉ số phương trình vào cặp dấu ngoặc đơn.
Ví dụ: dùng \verb'\ref{abc}' ta thu được 3.2,
trong khi \verb'\eqref{abc}' sẽ cho (3.2).

%\subsection{Subordinate numbering sequences}
\subsection{Đánh số các phương trình con của nhóm phương trình}

% The \pkg{amsmath} package provides also a \env{subequations}
% environment\index{equation numbers!subordinate numbering} to make it
% easy to number equations in a particular group with a subordinate
% numbering scheme. For example
Gói \pkg{amsmath} cung cấp môi trường \env{subequations} cho phép
bạn đánh số các phương trình con của một nhóm các phương trình
\emph{một cách tự động}. Ví dụ

\medskip
\begin{verbatim}
\begin{subequations}
...
\end{subequations}
\end{verbatim}

% causes all numbered equations within that part of the document to be
% numbered (4.9a) (4.9b) (4.9c) \dots, if the preceding numbered
% equation was (4.8).
\medskip
\noindent
sẽ khiến các chỉ số của các phương trình bên trong môi trường
\env{subequations} sẽ (tự động) có dạng (4.9a) (4.9b) (4.9c) \dots,
nếu phương trình đi trước nhóm phương trình này có chỉ số (4.8).
% A \cn{label} command immediately after
% \verb/\begin{subequations}/ will produce a \cn{ref} of the parent
% number 4.9, not 4.9a. 
Chú ý là, nếu bạn đặt nhãn tham chiếu (|label|) ngay sau
\verb/\begin{subequations}/, việc tham chiếu bằng \cn{ref}
đến nhãn đó sẽ cho ra, chẳng hạn, 4.9, chứ không phải là 4.9a.
Vui lòng xem tài liệu \fn{subeqn.pdf} và mã nguồn \fn{subeqn.tex}
để có các ví dụ về việc dùng môi trường này.

% The counters used by the subequations
% environment are \verb/parentequation/ and \verb/equation/ and
% \cn{addtocounter}, \cn{setcounter}, \cn{value}, etc., can be applied
% as usual to those counter names. 
\medskip
Các bộ đếm dùng bởi môi trường \env{subequations} là \verb/parentequation/
và \verb/equation/; các lệnh thay đổi bộ đếm
\cn{addtocounter}, \cn{setcounter}, \cn{value}, \ldots
đều có tác dụng bình thường với những bộ đếm đó.
% To get anything other than lowercase
% letters for the subordinate numbers, use standard \latex/ methods for
% changing numbering style \cite[\S6.3, \S C.8.4]{lamport}. For example,
% redefining \cn{theequation} as follows will produce roman numerals.
Chẳng hạn, các phương trình con được đánh số nhờ chữ cái |alphabet|
(a,b,c,\ldots); để thay đổi điều này, sử dụng phương pháp chuẩn
của \latex/ để thay đổi kiểu đánh số (xem trong \cite[\S6.3, \S C.8.4]{lamport}).
Trong ví dụ dưới đây, các phương trình con sẽ được đánh chỉ số La Mã:

\medskip
\begin{verbatim}
\begin{subequations}
\renewcommand{\theequation}{\theparentequation \roman{equation}}
...
\end{verbatim}

%%%%%%%%%%%%%%%%%%%%%%%%%%%%%%%%%%%%%%%%%%%%%%%%%%%%%%%%%%%%%%%%%%%%%%%%

%\chapter{Miscellaneous mathematical features}
\chapter{Linh tinh, nhưng quan trọng}

\section{Ma trận}\label{ss:matrix}

% The \pkg{amsmath} package provides some environments for
% matrices\index{matrices} beyond the basic \env{array} environment of
% \latex/.
Gói \pkg{amsmath} cung cấp một số môi trường để biểu diễn ma trận
tiện lợi hơn\footnote{|beyond|} môi trường \env{array} của \latex/.
% The \env{pmatrix}, \env{bmatrix}, \env{Bmatrix}, \env{vmatrix}
% and \env{Vmatrix} have (respectively) $(\,)$, $[\,]$, $\lbrace\,\rbrace$,
% $\lvert\,\rvert$, and
% $\lVert\,\rVert$ delimiters built in.
Các môi trường \env{pmatrix}, \env{bmatrix}, \env{Bmatrix}, \env{vmatrix}
và \env{Vmatrix} tự động có (tương tứng) cặp dấu ngoặc $(\,)$, $[\,]$,
$\lbrace\,\rbrace$,
$\lvert\,\rvert$, và $\lVert\,\rVert$ bao xung quanh.
% For naming consistency there is a \env{matrix} environment sans delimiters. 
Ngoài ra, môi trường \env{matrix} cho một ma trận không có kèm dấu ngoặc nào.
% This is not entirely redundant
% with the \env{array} environment; the matrix environments all use more
% economical horizontal spacing than the rather prodigal spacing of the
% \env{array} environment.

\medskip
Các môi trường ma trận của \pkg{amsmath} sử dụng thuật toán
điều chỉnh khoảng cách và canh cột tinh tế, tiết kiệm hơn cách làm hoang phí%
\footnote{|prodigal|} của môi trường \env{array}.
% Also, unlike the \env{array} environment, you
% don't have to give column specifications for any of the matrix
% environments; by default you can have up to 10 centered columns.%
Hơn nữa, không giống như \env{array}, bạn không cần chỉ ra \emph{số cột}
khi dùng các môi trường \env{-matrix}; theo mặc định, số cột tối đa của
các ma trận là 10\mdash con số này có thể thay đổi được%
\footnote{%%%%%%%%%%%%%%%%%%%%%%%%%%%%%%%%%%%%%%%%%%%%%%%%%%%%%%%%%%%%%%
% More precisely: The maximum number of columns in a matrix is determined
% by the counter |MaxMatrixCols| (normal value = 10), which you can change
% if necessary using \latex/'s \cn{setcounter} or \cn{addtocounter}
% commands.%
\emph{Cụ thể hơn:} số tối đa các cột của ma trận được xác định bởi bộ đếm |MaxMatrixCols|
(mặc định là 10); để thay đổi số này, bạn dùng các lệnh của \latex/ là
\cn{setcounter} hoặc \cn{addtocounter}.
}.
% (If you need left or right alignment in a column or other special
% formats you must resort to \env{array}.)
(Nếu bạn muốn canh phải, trái các cột hoặc muốn tinh chỉnh ma trận theo ý bạn,
không còn cách nào khác hơn là bạn phải quay lại dùng môi trường \env{array}.)

% To produce a small matrix suitable for use in text, there is a
% \env{smallmatrix} environment (e.g.,
\medskip
Các ma trận cỡ nhỏ đặt xen vào biểu thức chung dòng được cho bởi
môi trường \env{smallmatrix}. Ví dụ, ma trận
\begin{math}
\bigl( \begin{smallmatrix}
  a&b\\ c&d
\end{smallmatrix} \bigr)
\end{math}
% that comes closer to fitting within a single text line than a normal
% matrix.
được bố trí vừa vẹn trên dòng này, tốt hẳn hơn so
với khi bạn thay thế \env{smallmatrix} bởi \env{matrix}:
% Delimiters must be provided; there are no |p|,|b|,|B|,|v|,|V|
% versions of \env{smallmatrix}. The above example was produced by

\medskip
\begin{verbatim}
\bigl( \begin{smallmatrix}
  a&b\\ c&d
\end{smallmatrix} \bigr)
\end{verbatim}

\medskip
Chú ý rằng các dấu ngoặc phải được chỉ ra khi dùng \env{smallmatrix};
không có các biến thể |p|,|b|,|B|,|v|,|V| của \env{smallmatrix}.

% \cn{hdotsfor}|{|\<number>|}| produces a row of dots in a
% matrix\index{matrices!ellipsis dots}\index{ellipsis dots!in
% matrices}\index{dots|see{ellipsis dots}} spanning the given number of
% columns. For example,
\medskip
Có một lệnh hết sức tiện lợi, là \cn{hdotsfor}|{|\<number>|}|.
Lệnh này sẽ sinh ra dãy các dấu chấm lửng trong matrix. Tham số
của lệnh là số cột cần bỏ qua. Quan sát ví dụ dưới đây

\medskip
\begin{center}
\begin{minipage}{.3\columnwidth}
\noindent$\begin{matrix} a&b&c&d\\
e&\hdotsfor{3} \end{matrix}$
\end{minipage}%
\qquad
\begin{minipage}{.45\columnwidth}
\begin{verbatim}
\begin{matrix} a&b&c&d\\
e&\hdotsfor{3} \end{matrix}
\end{verbatim}
\end{minipage}%
\end{center}

% The spacing of the dots can be varied through use of a square-bracket
% option, for example, |\hdotsfor[1.5]{3}|.  The number in square brackets
% will be used as a multiplier (i.e., the normal value is 1.0).
\medskip
Khoảng cách giữa các dấu chấm cho bởi \cn{hdotsfor} có thể thay thế nhờ
tham số bổ sung của lệnh; ví dụ: |\hdotsfor[1.5]{3}|. Con số trong ngoặc
vuông (|[1.5]|) được dùng như một thừa số nhân; giá trị mặc định là 1.0.
\begin{equation}\label{eq:D}
\begin{pmatrix} D_1t&-a_{12}t_2&\dots&-a_{1n}t_n\\
-a_{21}t_1&D_2t&\dots&-a_{2n}t_n\\
\hdotsfor[2]{4}\\
-a_{n1}t_1&-a_{n2}t_2&\dots&D_nt\end{pmatrix}
\end{equation}

\begin{verbatim}
\begin{pmatrix} D_1t&-a_{12}t_2&\dots&-a_{1n}t_n\\
-a_{21}t_1&D_2t&\dots&-a_{2n}t_n\\
%
\hdotsfor[2]{4}\\% tăng gấp đôi khoảng cách giữa các dấu chấm
%
-a_{n1}t_1&-a_{n2}t_2&\dots&D_nt\end{pmatrix}
\end{verbatim}

%\section{Math spacing commands}
\section{Điều chỉnh khoảng cách}

% The \pkg{amsmath} package slightly extends the set of math
% spacing\index{horizontal space!in math mode} commands, as shown below.
% Both the spelled-out and abbreviated forms of these commands are robust,
% and they can also be used outside of math.
Gói \pkg{amsmath} cung cấp tập hợp các lệnh điều chỉnh khoảng cách
trong chế độ toán. Phiên bản rút gọn (lệnh ngắn) hoặc đầy đủ (lệnh dài) của
các lệnh này đều có tính |robust| và có thể dùng bên ngoài chế độ toán.
\begin{ctab}{lll|lll}
Lệnh ngắn& Lệnh dài& Ví dụ & Lệnh ngắn& Lệnh dài& Ví dụ\\
\hline
\vstrut{2.5ex}
& {\tiny (không khoảng cách)}& \spx{}& & {\tiny (không khoảng cách)} & \spx{}\\
\cn{\,}& \cn{thinspace}& \spx{\,}&
  \cnbang& \cn{negthinspace}& \spx{\!}\\
\cn{\:}& \cn{medspace}& \spx{\:}&
  & \cn{negmedspace}& \spx{\negmedspace}\\
\cn{\;}& \cn{thickspace}& \spx{\;}&
  & \cn{negthickspace}& \spx{\negthickspace}\\
& \cn{quad}& \spx{\quad}\\
& \cn{qquad}& \spx{\qquad}
\end{ctab}
% For the greatest possible control over math spacing, use \cn{mspace}
% and `math units'. One math unit, or |mu|, is equal to 1/18 em. Thus to
% get a negative \cn{quad} you could write |\mspace{-18.0mu}|.
Tổng quát hơn, để điều chỉnh khoảng cách trong chế độ toán, bạn dùng
lệnh \cn{mspace} và \emph{đơn vị toán}. \emph{Đơn vị toán},
ký hiệu là |mu| (|math unit|), là đơn vị độ dài của \latex/ trong chế độ toán,
bằng 1/18 em\footnote{Thế |em| là cái chi?}. Ví dụ,
để có khoảng \cn{quad} âm, bạn có thể dùng |\mspace{-18.0mu}|.

\section{Các dấu chấm}

% For preferred placement of ellipsis dots (raised or on-line) in various
% contexts there is no general consensus. It may therefore be considered a
% matter of taste. By using the semantically oriented commands
Vị trí của dấu ba chấm (phía trên dòng hay ở ngay chân dòng) trong các
ngữ cảnh khác nhau thường không nhất quán, và có thể là vấn đề sở thích
cá nhân. Bằng cách sử dụng các lệnh
\begin{itemize}
\item \cn{dotsc} cho \qq{dấu ba chấm đi với dấu phảy}
\item \cn{dotsb} cho \qq{dấu ba chấm đi với toán tử nhị phân/quan hệ}
\item \cn{dotsm} cho \qq{dấu ba chấm đi trong biểu thức nhân}
\item \cn{dotsi} cho \qq{dấu ba chấm đi trong biểu thức tích phân}
\item \cn{dotso} cho \qq{dấu ba chấm cho mọi trường hợp còn lại}
\end{itemize}
% instead of \cn{ldots} and \cn{cdots},
thay vì dùng \cn{ldots} và \cn{cdots},
% you make it possible for your
% document to be adapted to different conventions on the fly, in case (for
% example) you have to submit it to a publisher who insists on following
% house tradition in this respect. 
bạn có thể làm cho mã nguồn \TeX\ của tài liệu trở nên trong sáng hơn,
dễ chỉnh sửa hơn khi cần thiết (chẳng hạn theo yêu cầu của nhà xuất bản).
Định nghĩa của các lệnh ở trên là theo quy ước của Hội Toán học Mỹ (AMS).
% The default treatment for the various
% kinds follows American Mathematical Society conventions:

\medskip
\begin{center}
\begin{tabular}{@{}l@{}l@{}}
\begin{minipage}[t]{.54\textwidth}
\begin{verbatim}
Ta có chuỗi $A_1, A_2, \dotsc$,
tổng vô hạn $A_1 +A_2 +\dotsb $,
tích vô hạn $A_1 A_2 \dotsm $,
và tích phân không xác định
\[\int_{A_1}\int_{A_2}\dotsi\]
\end{verbatim}
\end{minipage}
&
\begin{minipage}[t]{.45\textwidth}
\noindent
Ta có chuỗi $A_1, A_2, \dotsc$,
tổng vô hạn $A_1 +A_2 +\dotsb $,
tích vô hạn $A_1 A_2 \dotsm $,
và tích phân không xác định
\[\int_{A_1}\int_{A_2}\dotsi\]
\end{minipage}
\end{tabular}
\end{center}

%\section{Nonbreaking dashes}
\section{Gạch ngang không vỡ}

% A command \cn{nobreakdash} is provided to suppress the possibility
% of a linebreak after the following hyphen or dash.
Lệnh \cn{nobreakdash} giúp ta hạn chế sự ngắt dòng sau một dấu |hyphen|
hoặc sau dấu gạch ngang.
% For example, if you
% write `pages 1\ndash 9' as |pages 1\nobreakdash--9| then a linebreak will
% never occur between the dash and the 9. You can also use
% \cn{nobreakdash} to prevent undesirable hyphenations in combinations
% like |$p$-adic|. For frequent use, it's advisable to make abbreviations,
% e.g.,
Ví dụ, nếu bạn viết `các trang 1\ndash 9' bằng cách dùng |các trang 1\nobreakdash--9|
thì việc ngắt dòng sẽ không bao giờ xảy ra giữa dấu gạch ngang và số 9.
Bạn cũng có thể dùng \cn{nobreakdash} để hạn chế việc tách chữ (|hyphenation|)
ngoài ý muốn, ví dụ trong tổ hợp |$p$-adic|. Để tiện lợi, bạn có thể định
nghĩa các lệnh tắt như sau:

\medskip
\begin{verbatim}
\newcommand{\p}{$p$\nobreakdash}% cho "\p-adic"
\newcommand{\Ndash}{\nobreakdash--}% cho "các trang 1\Ndash 9"
%    Cho không gian "\n chiều" ("n-chiều"):
\newcommand{\n}[1]{$n$\nobreakdash-\hspace{0pt}}
\end{verbatim}
% The last example shows how to prohibit a linebreak after the hyphen but
% allow normal hyphenation in the following word. (It suffices to add a
% zero-width space after the hyphen.)

\medskip
Ví dụ cuối cùng ở trên cho biết cách làm thế nào để ngăn cản ngắt dòng sau dấu
|hyphen| nhưng cho phép tách chữ bình thường ở từ tiếp theo dấu |hyphen| đó.
(Chỉ cần thêm một độ rộng 0 sau dấu |hyphen|.)

%\section{Accents in math}
\section{Dấu nhấn trong toán học}

% In ordinary \LaTeX{} the placement of the second accent in doubled math
% accents is often poor. With the \pkg{amsmath} package you
% will get improved placement of the second accent:
Thông thường, \latex/ khó có thể biểu diễn cho đẹp cùng lúc hai dấu nhấn
trên một ký hiệu. Với gói \pkg{amsmath}, việc này khá hơn rất nhiều; ví dụ
$\hat{\hat{A}}$ (cho bởi \cn{hat}|{\hat{A}}|).

% The commands \cn{dddot} and \cn{ddddot} are available to produce triple
% and quadruple dot accents in addition to the \cn{dot} and \cn{ddot}
% accents already available in \latex/.
\medskip
Các lệnh \cn{dddot} và \cn{ddddot} biểu diễn hai và ba dấu chấm
như là dấu nhấn, ví dụ $\dddot C$. Đây là các lệnh do \pkg{amsmath} cung cấp;
còn \cn{dot} và \cn{ddot} là các lệnh của \latex/.

% To get a superscripted hat or tilde character, load the \pkg{amsxtra}
% package and use \cn{sphat} or \cn{sptilde}. Usage is \verb'A\sphat'
% (note the absence of the \verb'^' character).
\medskip
Để có được chỉ số trên là dấu mũ hoặc dấu ngã, dùng gói \pkg{amsxtra}
và các lệnh của gói này là \cn{sphat}, \cn{sptilde}. Ví dụ,
$A\sphat$\space\space cho bởi \verb'A\sphat' (để ý ở đây ta không dùng dấu |^|).

% To place an arbitrary symbol in math accent position, or to get under
% accents, see the \pkg{accents} package by Javier Bezos.
\medskip
Để đặt ký hiệu bất kỳ vào vị trí của dấu nhấn, hoặc để có các dấu nhấn
bên dưới, vui lòng xem tài liệu hướng dẫn gói \pkg{accents}(viết bởi Javier Bezos).

%\section{Roots}
\section{Căn số}

% In ordinary \latex/ the placement of root indices is sometimes not so
% good: $\sqrt[\beta]{k}$ (|\sqrt||[\beta]{k}|). 
Cách biểu diễn căn số của \latex/ nhiều khi không thật đẹp mắt, như
trong ví dụ $\sqrt[\beta]{k}$ (cho bởi |\sqrt||[\beta]{k}|). 
% In the \pkg{amsmath} package \cn{leftroot} and \cn{uproot} allow you to adjust
% the position of the root:
Gói \pkg{amsmath} cung cấp các lệnh \cn{leftroot} (dịch qua trái) và \cn{uproot}
(dịch lên trên)
cho phép bạn điều chỉnh vị trí của bậc căn số%
\footnote{bậc của căn số, ví dụ $\beta$ trong $\sqrt[\leftroot{-2}\uproot{2}\beta]{k}$.}.
% ----------------------------------------------------------------------
Chẳng hạn

\medskip
\begin{verbatim}
\sqrt[\leftroot{-2}\uproot{2}\beta]{k}
\end{verbatim}
%will move the beta up and to the right:
% $\sqrt[\leftroot{-2}\uproot{2}\beta]{k}$. The negative argument used
% with \cn{leftroot} moves the $\beta$ to the right. The units are a small
% amount that is a useful size for such adjustments.

\medskip\noindent
sẽ cho $\sqrt[\leftroot{-2}\uproot{2}\beta]{k}$. Nếu bạn dùng |dimension|
âm làm tham số của \cn{leftroot} (như trong ví dụ trên),
thì bậc căn số sẽ dịch chuyển qua phải.

%\section{Boxed formulas}
\section{Đóng khung biểu thức}

% The command \cn{boxed} puts a box around its
% argument, like \cn{fbox} except that the contents are in math mode:
Biểu thức \cn{boxed} sẽ đóng khung chữ nhật quanh nội dung là tham số của nó;
lệnh này tương tự như lệnh \cn{fbox}, nhưng nội dung của lệnh ở chế độ toán.
\begin{equation}
\boxed{\eta \leq C(\delta(\eta) +\Lambda_M(0,\delta))}
\end{equation}
\begin{verbatim}
  \boxed{\eta \leq C(\delta(\eta) +\Lambda_M(0,\delta))}
\end{verbatim}

%\section{Over and under arrows}
\section{Mũi tên bên trên, bên dưới}

% Basic \latex/ provides \cn{overrightarrow} and \cn{overleftarrow}
% commands. Some additional over and under arrow commands are provided
% by the \pkg{amsmath} package to extend the set:
\latex/ chuẩn chỉ cung cấp các mũi tên bên-trên-qua-trái hoặc bên-trên-qua-phải
nhờ các lệnh \cn{overrightarrow}, \cn{overleftarrow}. Dưới đây là các dạng
mũi tên khác do gói \pkg{amsmath} cung cấp.

\begin{tabbing}
\qquad\=\ncn{overleftrightarrow}\qquad\=\kill
\> \cn{overleftarrow}    \> \cn{underleftarrow} \+\\
   \cn{overrightarrow}    \> \cn{underrightarrow} \\
   \cn{overleftrightarrow}\> \cn{underleftrightarrow}
\end{tabbing}

Ví dụ, \cn{overleftarrow}|{XYZ}| sẽ cho bạn $\overleftarrow{XYZ}$.

%\section{Extensible arrows}
%\section{Mũi tên có chỉ số. Bổ sung của ccc}
\section{Mũi tên có chỉ số.}

% \cn{xleftarrow} and \cn{xrightarrow} produce
% arrows\index{arrows!extensible} that extend automatically to accommodate
% unusually wide subscripts or superscripts. These commands take one
% optional argument (the subscript) and one mandatory argument (the
% superscript, possibly empty):
Các lệnh \cn{xleftarrow} và \cn{xrightarrow} cho ta các mũi tên \emph{tự động
điều chỉnh độ dài cho bằng với nội dung của chỉ số trên hoặc dưới}.
Tham số bổ sung của hai lệnh này là chỉ số dưới, còn tham số bắt buộc
là chỉ số trên. Cả chỉ số trên và chỉ số dưới của hai lệnh này đều có thể rỗng.
\begin{equation}
A\xleftarrow{n+\mu-1}B \xrightarrow[T]{n\pm i-1}C
\end{equation}
\begin{verbatim}
A\xleftarrow{n+\mu-1}B \xrightarrow[T]{n\pm i-1}C
\end{verbatim}

%% \medskip
%% Để có chẳng hạn,
%% \begin{equation}
%% A\xlongequal[ccc]{\text{theo (1)}}B,\qquad
%% A\xLongrightarrow{\text{theo Tiên đề 2}}B,
%% \end{equation}
%% bạn có thể dùng gói \pkg{ccc-arrows} (của |ccc|;
%% tải về: \url{http://vntex.org/download}).

%\section{Affixing symbols to other symbols}
\section{Gắn các ký hiệu với nhau}

% \latex/ provides \cn{stackrel} for placing a
% superscript\index{subscripts and superscripts} above a binary relation.
% In the \pkg{amsmath} package there are somewhat more general commands,
% \cn{overset} and \cn{underset}, that can be used to place one symbol
% above or below another symbol, whether it's a relation or something
% else. The input |\overset{*}{X}| will place a superscript-size $*$ above
% the $X$: $\overset{*}{X}$; \cn{underset} is the analog for adding a
% symbol underneath.
\latex/ cung cấp lệnh \cn{stackrel} để đặt chỉ số trên cho các quan hệ (toán tử)
nhị phân. Với gói \pkg{amsmath}, các lệnh \cn{overset} và \cn{underset}
còn làm được việc tổng quát hơn: đặt một ký hiệu bên trên hoặc bên dưới
một ký hiệu khác, bất kể ký hiệu đó có phải là toán tử nhị phân hay không.
Ví dụ, với |\overset{*}{X}| sẽ đặt dấu sao $*$ bên trên chữ $X$:
$\overset{*}{X}$; lệnh \cn{underset} cũng tương tự, nhưng nó đặt chỉ số
bên dưới ký hiệu.

\medskip
%See also the description of \cn{sideset} in \secref{sideset}.
Xem thêm cách dùng \cn{sideset} ở Mục~\secref{sideset}.

%\section{Fractions and related constructions}
\section{Phân số và cấu trúc liên quan}

\subsection{Các lệnh \cn{frac}, \cn{dfrac}, và \cn{tfrac}}

% The \cn{frac} command, which is in the basic command set of
% \latex/,\index{fractions} takes two arguments\mdash numerator and
% denominator\mdash and typesets them in normal fraction form. The
% \pkg{amsmath} package provides also \cn{dfrac} and \cn{tfrac} as
% convenient abbreviations for |{\displaystyle\frac| |...| |}|
% and\index{textstyle@\cn{textstyle}}\relax
% \index{displaystyle@\cn{displaystyle}} |{\textstyle\frac| |...| |}|.
Lệnh \cn{frac}, là lệnh cơ bản của \latex/, nhận hai tham số\mdash tử số
và mẫu số\mdash và biểu diễn phân số theo kiểu thông thường trong toán học.
Để ý rằng, các phân số ở chế độ toán chung dòng thường nhỏ; để khiến chúng
to hơn, bạn dùng |\displaystyle\frac...|. Để tiện lợi,
gói \pkg{amsmath} cung cấp thêm các lệnh: \cn{dfrac}, \cn{tfrac} lần lượt
là dạng viết tắt của |{\displaystyle\frac| |...| |}| và
|{\textstyle\frac| |...| |}|. Ví dụ
\begin{equation}
\frac{1}{k}\log_2 c(f)\quad\tfrac{1}{k}\log_2 c(f)\quad
\sqrt{\frac{1}{k}\log_2 c(f)}\quad\sqrt{\dfrac{1}{k}\log_2 c(f)}
\end{equation}
\begin{verbatim}
\begin{equation}
\frac{1}{k}\log_2 c(f)\;\tfrac{1}{k}\log_2 c(f)\;
\sqrt{\frac{1}{k}\log_2 c(f)}\;\sqrt{\dfrac{1}{k}\log_2 c(f)}
\end{equation}
\end{verbatim}

%\subsection{The \cn{binom}, \cn{dbinom}, and \cn{tbinom} commands}
\subsection{Ký hiệu Tổ hợp:  \cn{binom}, \cn{dbinom}, \cn{tbinom}}

% For binomial expressions\index{binomials} such as $\binom{n}{k}$
% \pkg{amsmath} has \cn{binom}, \cn{dbinom} and \cn{tbinom}:
Để biểu diễn tổ hợp dạng $\binom{n}{k}$, bạn dùng các lệnh
\cn{binom}, \cn{dbinom} và \cn{tbinom}. Các tiếp vĩ ngữ |d| và |t|
được hiểu tương tự như ở các lệnh về phân số.
\begin{equation}
2^k-\binom{k}{1}2^{k-1}+\dbinom{k}{2}2^{k-2}
\end{equation}
\begin{verbatim}
2^k-\binom{k}{1}2^{k-1}+\binom{k}{2}2^{k-2}
\end{verbatim}
%
%\subsection{The \cn{genfrac} command}
\subsection{Tạo phân số tổng quát với \cn{genfrac}}

% The capabilities of \cn{frac}, \cn{binom}, and their variants are
% subsumed by a generalized fraction command \cn{genfrac} with six
% arguments.
Khả năng của các lệnh \cn{frac}, \cn{binom} và các biến thể có thể có
được nhờ lệnh \cn{genfrac}. Đây là lệnh tạo phân số tổng quát với
sáu tham số.
% The last two correspond to \cn{frac}'s numerator and
% denominator; 
Hai tham số cuối cùng của \cn{genfrac} lần lượt là \emph{tử số} và \emph{mẫu số};
% the first two are optional delimiters (as seen in
% \cn{binom});
hai tham số bổ sung đầu tiên là các dấu ngoặc trái và phải (như trong
kết quả của \cn{binom});
% the third is a line thickness override (\cn{binom} uses
% this to set the fraction line thickness to 0\mdash i.e., invisible);
% and
tham số thứ ba là độ dày của đường kẻ ngang giữa tử số và mẫu số (lệnh
\cn{binom} dùng độ dày 0, nghĩa là \emph{không thấy được});
% the fourth argument is a mathstyle override: integer values 0\ndash 3 select
% respectively \cn{displaystyle}, \cn{textstyle}, \cn{scriptstyle}, and
% \cn{scriptscriptstyle}. If the third argument is left empty, the line
% thickness defaults to `normal'.
tham số thứ tư xác định chế độ toán trong đó phân số được biểu diễn:
tham số này nhận giá trị 0,1,2,3 tương ứng với các chế độ
 \cn{displaystyle}, \cn{textstyle}, \cn{scriptstyle}, \cn{scriptscriptstyle}.

\medskip
\begin{center}\begin{minipage}{\columnwidth}
\raggedright \normalfont\ttfamily \exhyphenpenalty10000
\newcommand{\ma}[1]{%
  \string{{\normalfont\itshape#1}\string}\penalty9999 \ignorespaces}
\string\genfrac \ma{left-delim} \ma{right-delim} \ma{thickness}
\ma{mathstyle} \ma{numerator} \ma{denominator}
\end{minipage}\end{center}

% To illustrate, here is how \cn{frac}, \cn{tfrac}, and
% \cn{binom} might be defined.
\medskip\noindent
Ví dụ, các lệnh \cn{frac}, \cn{tfrac} và \cn{binom} có thể được định nghĩa như sau:

\medskip
\begin{verbatim}
\newcommand{\frac}[2]{\genfrac{}{}{}{}{#1}{#2}}
\newcommand{\tfrac}[2]{\genfrac{}{}{}{1}{#1}{#2}}
\newcommand{\binom}[2]{\genfrac{(}{)}{0pt}{}{#1}{#2}}
\end{verbatim}
% If you find yourself repeatedly using \cn{genfrac} throughout a document
% for a particular notation, you will do yourself a favor (and your
% publisher) if you define a meaningfully-named abbreviation for that
% notation, along the lines of \cn{frac} and \cn{binom}.

% The primitive generalized fraction commands \cs{over}, \cs{overwithdelims},
% \cs{atop}, \cs{atopwithdelims}, \cs{above}, \cs{abovewithdelims} produce
% warning  messages if used with the \pkg{amsmath} package, for reasons
% discussed in \fn{technote.tex}.
\medskip
Các lệnh nguyên thủy \cn{over}, \cs{overwithdelims},
\cs{atop}, \cs{atopwithdelims}, \cs{above}, \cs{abovewithdelims}
khi dùng với gói \pkg{amsmath} sẽ sinh ra cảnh báo lỗi. Vui lòng
xem thêm tài liệu \fn{technote.tex} có trong |texmf/source/latex/amsmath/|

%\section{Continued fractions}
\section{Phân số liên tục}

Phân số liên tục\index{phân số liên tục}
\begin{equation}
\cfrac{1}{\sqrt{2}+
 \cfrac{1}{\sqrt{2}+
  \cfrac{1}{\sqrt{2}+\cdots
}}}
\end{equation}
%can be obtained by typing
có thể thu đựơc nhờ
{\samepage
\begin{verbatim}
\cfrac{1}{\sqrt{2}+
 \cfrac{1}{\sqrt{2}+
  \cfrac{1}{\sqrt{2}+\dotsb
}}}
\end{verbatim}
}% End of \samepage
% This produces better-looking results than straightforward use of
% \cn{frac}. Left or right placement of any of the numerators is
% accomplished by using \cn{cfrac}|[l]| or \cn{cfrac}|[r]| instead of
% \cn{cfrac}.
\medskip\noindent
Việc dùng \cn{cfrac} cho kết quả dễ nhìn hơn so với khi bạn tạo phân số
liên tục bằng \cn{frac}. Bạn có thể đặt vị trí của tử số ở bên trái hoặc
bên phải tùy ý, bằng cách dùng \cn{cfrac}|[l]| hoặc \cn{cfrac}|[r]| thay vì
\cn{cfrac}.

%\section{Smash options}
\section{Lệnh \cn{smash} đặt độ cao/sâu về 0}

% The command \cn{smash} is used to typeset a subformula with an effective height and depth of zero, which is sometimes
% useful in adjusting the subformula's position with respect to adjacent
% symbols.
Lệnh \cn{smash} khiến cho các biểu thức con có chiều cao (|heigth|) và chiều
sâu (|depth|) bằng 0; việc dùng biểu thức con như vậy rất có hữu ích,
chẳng hạn khi điều chỉnh vị trí của biểu thức con theo các ký hiệu.
% With the \pkg{amsmath} package \cn{smash} has optional
% arguments |t| and |b|, because occasionally it is advantageous to be
% able to \qq{smash} only the top or only the bottom of something while
% retaining the natural depth or height.
Gói \pkg{amsmath} cung cấp lệnh \cn{smash} với tham số tùy chọn là |t|
và |b|, bởi đôi khi ta chỉ cần làm cho chiều cao (ứng với |t|) hoặc
chiều sâu (ứng với |b|) trở về 0.
% For example, when adjacent
% radical symbols are unevenly sized or positioned because of differences
% in the height and depth of their contents, \cn{smash} can be employed to
% make them more consistent. Compare

\medskip
Ví dụ, chiều cao của các dấu căn bậc hai trong một biểu thức
không bằng nhau vì chiều cao đó tùy thuộc vào nội dung bên dưới dấu căn.
(Xem ở độ phóng đại lớn,) bạn sẽ thấy dấu căn trong $\sqrt y$ trong biểu thức
$\sqrt{x}+\sqrt{y}+\sqrt{z}$ đi xuống phía dưới chân dòng hơn so với
hai dấu căn còn lại. Ta sẽ dùng \cn{smash[b]} để đặt độ sâu của chữ $y$
về 0, nhờ đó, dấu căn trong $\sqrt y$ sẽ giống hệt dấu căn trong $\sqrt x$:

\medskip
\verb"$\sqrt{x}" \verb"+"
\verb"\sqrt{"\5\verb"\smash[b]{y}}" \verb"+" \verb"\sqrt{z}$".

\medskip\noindent
Bây giờ, bạn có $\sqrt{x}+\sqrt{\smash[b]{y}}+\sqrt{z}$. Bạn có thấy
sự khác biệt ở biểu thức này với biểu thức trước kia không?

%\section{Delimiters}
%
\section{Dấu ngoặc}

\subsection{Kích cỡ dấu ngoặc}\label{bigdel}

% The automatic delimiter sizing done by \cn{left} and \cn{right} has two
% limitations: First, it is applied mechanically to produce delimiters
% large enough to encompass the largest contained item, and second, the
% range of sizes is not even approximately continuous but has fairly large
% quantum jumps. 
Việc dùng dấu ngoặc tự động điều chỉnh kích cỡ với \cn{left} và \cn{right}
có hai hạn chế: \emph{thứ nhất,} các lệnh này máy móc điều chỉnh cho dấu ngoặc
có chiều cao vừa bằng với chiều cao của nội dung cao nhất,
và \emph{thứ hai,} tập hợp các kích cỡ do chúng tạo ra không biến thiên
đều đặn, mà có những bước nhảy quá lớn\footnote{\emph{Nghĩa là:} hai dấu ngoặc kế tiếp nhau
trong cùng biểu thức tạo bởi \cn{left} hoặc \cn{right} có thể có chiều cao
chênh nhau đến 3pt hoặc nhiều hơn, và như thế sẽ quá xấu!}.
% This means that a math fragment that is infinitesimally
% too large for a given delimiter size will get the next larger size, a
% jump of 3pt or so in normal-sized text.

% There are two or three
% situations where the delimiter size is commonly adjusted, using a set of
% commands that have `big' in their names.
\medskip
Trong thực tế, thường có hai hoặc ba trường hợp ta cần điều chỉnh kích cỡ
của dấu ngoặc nhờ vào các lệnh `big' của gói \pkg{amsmath}.
\begin{ctab}{l|llllll}
Cỡ&
Cỡ& \ncn{left}& \ncn{bigl}& \ncn{Bigl}& \ncn{biggl}& \ncn{Biggl}\\
ngoặc&
|text|& \ncn{right}& \ncn{bigr}& \ncn{Bigr}& \ncn{biggr}& \ncn{Biggr}\\
\hline
Kết quả\vstrut{5ex}&
  $\displaystyle(b)(\frac{c}{d})$&
  $\displaystyle\left(b\right)\left(\frac{c}{d}\right)$&
  $\displaystyle\bigl(b\bigr)\bigl(\frac{c}{d}\bigr)$&
  $\displaystyle\Bigl(b\Bigr)\Bigl(\frac{c}{d}\Bigr)$&
  $\displaystyle\biggl(b\biggr)\biggl(\frac{c}{d}\biggr)$&
  $\displaystyle\Biggl(b\Biggr)\Biggl(\frac{c}{d}\Biggr)$
\end{ctab}
% The first kind of situation is a cumulative operator with limits above
% and below. With \cn{left} and \cn{right} the delimiters usually turn out
% larger than necessary, and using the |Big| or |bigg|
% sizes\index{big@\cn{big}, \cn{Big}, \cn{bigg}, \dots\ delimiters}
% instead gives better results:

\emph{Trường hợp đầu tiên,} là khi xảy ra sự chồng chất (|cumulative|) các
toán tử với chỉ số trên và dưới. Với \cn{left} và \cn{right}, các dấu ngoặc
tạo ra thường lớn hơn nhiều so với ý muốn; việc dùng |Big| or |bigg| thay
thế cho \cn{left} và \cn{right} cho kết quả khả quan hơn hẳn:
\begin{equation*}
\left[\sum_i a_i\left\lvert\sum_j x_{ij}\right\rvert^p\right]^{1/p}
\quad\text{so với}\quad
\biggl[\sum_i a_i\Bigl\lvert\sum_j x_{ij}\Bigr\rvert^p\biggr]^{1/p}
\end{equation*}
\begin{verbatim}
\biggl[\sum_i a_i\Bigl\lvert\sum_j x_{ij}\Bigr\rvert^p\biggr]^{1/p}
\end{verbatim}
% The second kind of situation is clustered pairs of delimiters where
% \cn{left} and \cn{right} make them all the same size (because that is
% adequate to cover the encompassed material) but what you really want
% is to make some of the delimiters slightly larger to make the nesting
% easier to see.

\medskip
\emph{Trường hợp thứ hai,} là khi các dấu ngoặc lồng nhau liên tiếp.
Trong trường hợp này, lệnh \cn{left} và \cn{right} cho ra các dấu ngoặc
cùng một kích cỡ\footnote{%
        because that is adequate to cover the encompassed material.},
trong dấu ngoặc càng ở bên ngoài càng cần có kích thước lớn hơn.
\begin{equation*}
\left((a_1 b_1) - (a_2 b_2)\right)
\left((a_2 b_1) + (a_1 b_2)\right)
\quad\text{versus}\quad
\bigl((a_1 b_1) - (a_2 b_2)\bigr)
\bigl((a_2 b_1) + (a_1 b_2)\bigr)
\end{equation*}
\begin{verbatim}
\left((a_1 b_1) - (a_2 b_2)\right)
\left((a_2 b_1) + (a_1 b_2)\right)
\quad\text{so với}\quad
\bigl((a_1 b_1) - (a_2 b_2)\bigr)
\bigl((a_2 b_1) + (a_1 b_2)\bigr)
\end{verbatim}
% The third kind of situation is a slightly oversize object in running
% text, such as $\left\lvert\frac{b'}{d'}\right\rvert$ where the
% delimiters produced by \cn{left} and \cn{right} cause too much line
% spreading. In that case \ncn{bigl} and \ncn{bigr}\index{big@\cn{big},
% \cn{Big}, \cn{bigg}, \dots\ delimiters} can be used to produce
% delimiters that are slightly larger than the base size but still able to
% fit within the normal line spacing:

\medskip
\emph{Trường hợp thứ ba,} là khi biểu diễn các biểu thức quá khổ trong một
dòng, chẳng hạn biểu thức $\left\lvert\frac{b'}{d'}\right\rvert$ trong dòng này.
Các dấu ngoặc tạo bởi \cn{left} và \cn{right}, như bạn thấy, sẽ làm cho dòng
bị dãn quá nhiều theo chiều đứng. Khi đó, việc dùng \ncn{bigl} và \ncn{bigr}
chỉ cho ra dấu ngoặc vừa đủ cao và nhờ đó dòng không bị v-dãn nhiều quá,
như trong $\bigl\lvert\frac{b'}{d'}\bigr\rvert$.

% In ordinary \latex/ \ncn{big}, \ncn{bigg}, \ncn{Big}, and \ncn{Bigg}
% delimiters aren't scaled properly over the full range of \latex/ font
% sizes.  With the \pkg{amsmath} package they are.
\medskip
Các lệnh \ncn{big}, \ncn{bigg}, \ncn{Big} và \ncn{Bigg} mà \latex/ cung cấp
không thay đổi kích cỡ dấu ngoặc một cách chính xác trong các kích cỡ |font|
khác nhau của \latex/. Gói \pkg{amsmath} khắc phục nhược điểm đó.

%\subsection{Vertical bar notations}
\subsection{Ký hiệu $\rvert$}

% The \pkg{amsmath} package provides commands \cn{lvert}, \cn{rvert},
% \cn{lVert}, \cn{rVert} (compare \cn{langle}, \cn{rangle}) to address the
% problem of overloading for the vert bar character \qc{\|}. 
Gói \pkg{amsmath} cung cấp các lệnh \cn{lvert}, \cn{rvert},
\cn{lVert}, \cn{rVert} (so sánh với \cn{langle}, \cn{rangle}) nhằm
giải quyết vấn đề trùng hợp ý nghĩa của ký tự \qc{\|}.
% This character is currently used in \latex/ documents to represent a wide
% variety of mathematical objects: the `divides' relation in a
% number-theory expression like $p\vert q$, or the absolute-value
% operation $\lvert z\rvert$, or the `such that' condition in set
% notation, or the `evaluated at' notation $f_\zeta(t)\bigr\rvert_{t=0}$.
Ký tự này được dùng trong tài liệu \latex/ để chỉ nhiều đối tượng
toán học khác nhau: chỉ quan hệ `ước số' trong lý thuyết số, ví dụ như
trong biểu thứ $p\vert q$; chỉ giá trị tuyệt đối $\lvert z\rvert$; hay
là viết tắt của `sao cho' khi biểu diễn tập hợp; hoặc hàm ý `lấy giá trị tại'
trong ký hiệu $f_\zeta(t)\bigr\rvert_{t=0}$.
% The multiplicity of uses in itself is not so bad; what is bad, however,
% is that fact that not all of the uses take the same typographical
% treatment, and that the complex discriminatory powers of a knowledgeable
% reader cannot be replicated in computer processing of mathematical
% documents. 
Tính đa nghĩa của ký hiệu $\vert$ không phải là điều gì xấu; cái không
hay là ở chỗ: không phải tất cả cách dùng của $\vert$ đều có cùng cách thể hiện
xét về mặt in ấn (|typographical|), và sự phân biệt phức tạp của độc giả
có kiến thức không thể nào thể hiện được trong |typeset|.
% It is recommended therefore that there should be a one-to-one
% correspondence in any given document between the vert bar character
% \qc{\|} and a selected mathematical notation, and similarly for the
% double-bar command \cn{\|}.
Vì vậy, trong bất kỳ tài liệu nào, bạn cũng nên cho tương ứng ký tự \qc{\|}
với một khái niệm toán học cụ thể; và cũng nên làm điều tương tự với ký tự
\cn{\|}.
% This immediately rules out the use of \qc{|}
% and \ncn{\|}\index{"|@\cn{"\"|}} for delimiters, because left and right
% delimiters are distinct usages that do not relate in the same way to
% adjacent symbols;
Điều này không bao gồm cách dùng \qc{|} và \ncn{\|}\index{"|@\cn{"\"|}}
như các dấu ngoặc, bởi vì cách dùng dấu ngoặc trái hoặc phải khác hẳn với
các cách dùng của $\vert$ như kể trên.
% recommended practice is therefore to define suitable
% commands in the document preamble for any paired-delimiter use of vert
% bar symbols:

\medskip
Ví dụ, bạn nên định nghĩa
\begin{verbatim}
\providecommand{\abs}[1]{\lvert#1\rvert}
\providecommand{\norm}[1]{\lVert#1\rVert}
\end{verbatim}
% whereupon the document would contain |\abs{z}| to produce $\lvert
% z\rvert$ and |\norm{v}| to produce $\lVert v\rVert$.
và suốt trong tài liệu của mình, bạn dùng |\abs{z}| để có
$\lvert z\rvert$ và dùng |\norm{v}| để có $\lVert v\rVert$.

%\chapter{Operator names}
\chapter{Tên toán tử}

%\section{Defining new operator names}
\section{Định nghĩa toán tử mới}\label{s:opname}

% Math functions\index{operator names}\relax \index{function
% names|see{operator names}} such as $\log$, $\sin$, and $\lim$ are
% traditionally typeset in roman type to make them visually more distinct
% from one-letter math variables, which are set in math italic. The more
% common ones have predefined names, \cn{log}, \cn{sin}, \cn{lim}, and so
% forth, but new ones come up all the time in mathematical papers, so the
% \pkg{amsmath} package provides a general mechanism for defining new
% `operator names'. To define a math function \ncn{xxx} to work like
% \cn{sin}, you write
Các toán tử toán học như $\log$, $\sin$, $\lim$ theo truyền thống được
|typeset| với kiểu chữ |roman| để có thể dễ dàng phân biệt chúng với
các tên biến toán học (các tên biến được |typeset| với |font| nghiêng).
Các toán tử thường gặp nhất là \cn{log}, \cn{sin}, \cn{lim},... đã được
định nghĩa; còn các toán tử mới có thể định nghĩa dễ dàng nhờ lệnh
\cn{DeclareMathOperator}. Lệnh này chỉ có thể đặt trong phần |preamble|
của tài liệu. Chẳng hạn, để định nghĩa hàm số \ncn{tg} (là
cách viết hàm số |tangement| theo kiểu Việt Nam), có thể dùng

\begin{verbatim}
\DeclareMathOperator{\tg}{tg}
\end{verbatim}
% whereupon ensuing uses of \ncn{xxx} will produce {\upshape xxx} in the
% proper font and automatically add proper spacing\index{horizontal
% space!around operator names} on either side when necessary, so that you
% get $A\xxx B$ instead of $A\mathrm{xxx}B$. In the second argument of
% \cn{DeclareMathOperator} (the name text), a pseudo-text mode prevails:
% the hyphen character \qc{\-} will print as a text hyphen rather than a
% minus sign and an asterisk \qc{\*} will print as a raised text asterisk
% instead of a centered math star. (Compare
% \textit{a}-\textit{b}*\textit{c} and $a-b*c$.) But otherwise the name
% text is printed in math mode, so that you can use, e.g., subscripts and
% superscripts there.
nhờ đó, việc dùng \ncn{tg} sẽ cho ra $\tg$ với |font| chính xác
đồng thời tự động thêm cách khoảng cách xung quanh toán tử {\upshape tg}
khi cần thiết, nhờ đó, ta thu được $A(\tg x)B$ thay vì $A(\mathrm{tg}x)B$.

% If the new operator should have subscripts and superscripts placed in
% `limits' position above and below as with $\lim$, $\sup$, or $\max$, use
\medskip
Nếu toán tử mới cần các chỉ số trên và chỉ số dưới, như trong các toán tử
$\lim$, $\sup$, or $\max$, thì bạn dụng dạng \qc{\*} của lệnh \cn{DeclareMathOperator}:
% the \qc{\*} form of the \cn{DeclareMathOperator} command:

\medskip
\begin{verbatim}
\DeclareMathOperator*{\Lim}{Lim}
\end{verbatim}
%See also the discussion of subscript placement in

\medskip\noindent
Xem thêm ở Mục~\ref{subplace} về việc đặt chỉ số dưới.

\medskip
%The following operator names are predefined:
Các toán tử được định nghĩa trước là
\begin{ctab}{rlrlrlrl}
\cn{arccos}& $\arccos$ &\cn{deg}& $\deg$ &      \cn{lg}& $\lg$ &        \cn{projlim}& $\projlim$\\
\cn{arcsin}& $\arcsin$ &\cn{det}& $\det$ &      \cn{lim}& $\lim$ &      \cn{sec}& $\sec$\\
\cn{arctan}& $\arctan$ &\cn{dim}& $\dim$ &      \cn{liminf}& $\liminf$ &\cn{sin}& $\sin$\\
\cn{arg}& $\arg$ &      \cn{exp}& $\exp$ &      \cn{limsup}& $\limsup$ &\cn{sinh}& $\sinh$\\
\cn{cos}& $\cos$ &      \cn{gcd}& $\gcd$ &      \cn{ln}& $\ln$ &        \cn{sup}& $\sup$\\
\cn{cosh}& $\cosh$ &    \cn{hom}& $\hom$ &      \cn{log}& $\log$ &      \cn{tan}& $\tan$\\
\cn{cot}& $\cot$ &      \cn{inf}& $\inf$ &      \cn{max}& $\max$ &      \cn{tanh}& $\tanh$\\
\cn{coth}& $\coth$ &    \cn{injlim}& $\injlim$ &\cn{min}& $\min$\\
\cn{csc}& $\csc$ &      \cn{ker}& $\ker$ &      \cn{Pr}& $\Pr$
\end{ctab}
\begin{ctab}{rlrl}
\cn{varlimsup}&  $\displaystyle\varlimsup$&
  \cn{varinjlim}&  $\displaystyle\varinjlim$\\
\cn{varliminf}&  $\displaystyle\varliminf$&
  \cn{varprojlim}& $\displaystyle\varprojlim$
\end{ctab}

% There is also a command \cn{operatorname} such that using
\medskip
Ngoài ra, còn có lệnh \cn{operatorname}: ví dụ

\medskip
\begin{verbatim}
\operatorname{abc}
\end{verbatim}
%in a math formula is equivalent to a use of \ncn{abc} defined by

\medskip\noindent
trong biểu thức toán học tương đương với việc khai báo nhờ
\cn{DeclareMathOperator} rồi dùng \ncn{abc}. Lệnh \cn{operatorname}
chỉ nên dùng nếu toán tử thi thoảng gặp trong tài liệu;
còn nếu toán tử xuất hiện nhiều lần, bạn nên khai báo toán tử mới 
nhờ \cn{DeclareMathOperator}.
% This may be occasionally useful for
% constructing more complex notation or other purposes.

\medskip
%(Use the variant \cn{operatorname*} to get limits.)
Để ý dùng \cn{operatorname*} trong trường hợp toán tử cần chỉ số.

%\section{\cn{mod} and its relatives}
\section{Ký hiệu Đồng dư}

% Commands \cn{mod}, \cn{bmod}, \cn{pmod}, \cn{pod} are provided to deal
% with the special spacing conventions of \qq{mod} notation. \cn{bmod} and
% \cn{pmod} are available in \latex/, but with the \pkg{amsmath} package
% the spacing of \cn{pmod} will adjust to a smaller value if it's used in
% a non-display-mode formula. \cn{mod} and \cn{pod} are variants of
% \cn{pmod} preferred by some authors; \cn{mod} omits the parentheses,
% whereas \cn{pod} omits the \qq{mod} and retains the parentheses.
Các lệnh \cn{mod}, \cn{bmod}, \cn{pmod}, \cn{pod} biểu diễn các ký hiệu
đồng dư. Lệnh \cn{bmod} và \cn{pmod} đã có trong \latex/, nhưng trong gói
\pkg{amsmath}, chúng được tinh chỉnh để cho ra khoảng cách tốt hơn trong
các biểu thức chung dọng. Các lệnh \cn{mod} và \cn{pod} là biến thể của
\cn{pmod}; lệnh \cn{mod} không tạo ra cặp dấu ngoặc đơn, trong khi lệnh 
\cn{pod} lại bỏ \qq{mod} và các thứ khác vào cặp dấu ngoặc.
\begin{equation}
\gcd(n,m\bmod n);\quad x\equiv y\pmod b;
\quad x\equiv y\mod c;\quad x\equiv y\pod d
\end{equation}
\begin{verbatim}
\gcd(n,m\bmod n);\quad x\equiv y\pmod b;
\quad x\equiv y\mod c;\quad x\equiv y\pod d
\end{verbatim}

%\chapter{The \cn{text} command}
\chapter{Lệnh \cn{text} chèn \texttt{text} vào biểu thức}\label{text}

% The main use of the command \cn{text} is for words or
% phrases\index{text fragments inside math} in a display. It is very
% similar to the \latex/ command \cn{mbox} in its effects, but has a
% couple of advantages. If you want a word or phrase of text in a
% subscript, you can type |..._{\text{word or phrase}}|, which is slightly
% easier than the \cn{mbox} equivalent: |..._{\mbox{\scriptsize| |word|
% |or| |phrase}}|. The other advantage is the more descriptive name.
Mục đích chính của lệnh \cn{text} là để chèn văn bản, |text| vào giữa
các biểu thức toán học. Lệnh có tác dụng tương tự như lệnh \cn{mbox}
của \latex/, nhưng dùng \cn{text} hay hơn ở nhiều điểm.
Nếu bạn muốn văn bản thêm vào ở cỡ chữ |sub-script|, bạn có thể dùng
|..._{\text{từ hoặc câu}}|, trong khi với \cn{mbox}, bạn phải dùng
|..._{\mbox{\scriptsize| |từ| |hoặc| |câu}}|. Điểm hay khác, là tên của
lệnh \cn{text} phản ánh được đầy đủ tác dụng của nó hơn lệnh \cn{mbox}.
\begin{equation}
f_{[x_{i-1},x_i]} \text{ là đơn điệu,}
\quad i = 1,\dots,c+1
\end{equation}
\begin{verbatim}
f_{[x_{i-1},x_i]} \text{ là đơn điệu,}
\quad i = 1,\dots,c+1
\end{verbatim}

% \chapter{Integrals and sums}
\chapter{Tích phân và Tổng}

%\section{Multiline subscripts and superscripts}
\section{Chỉ số trên/dưới nhiều dòng}

% The \cn{substack} command can be used to produce a multiline subscript
% or superscript:\index{subscripts and superscripts!multi-line}\relax
% \index{superscripts|see{subscripts and superscripts}} for example
Nhờ lệnh \cn{substack}, bạn biểu diễn các chỉ số trên, dưới với nhiều dòng.
Ví dụ
\begin{ctab}{ll}
\begin{minipage}[t]{.6\columnwidth}
\begin{verbatim}
\sum_{\substack{
         0\le i\le m\\
         0<j<n}}
  P(i,j)
\end{verbatim}
\end{minipage}
&
$\displaystyle
\sum_{\substack{0\le i\le m\\ 0<j<n}} P(i,j)$
\end{ctab}
% A slightly more generalized form is the \env{subarray} environment which
% allows you to specify that each line should be left-aligned instead of
% centered, as here:
Tổng quát hơn, môi trường \env{subarray} còn cho phép bạn canh trái các
dòng của chỉ số trên/dưới, thay vì canh giữa như mặc định.
Hãy để ý đến tham số |{l}| của \env{subaray} trong ví dụ dưới đây.
\begin{ctab}{ll}
\begin{minipage}[t]{.6\columnwidth}
\begin{verbatim}
\sum_{\begin{subarray}{l}
        i\in\Lambda\\ 0<j<n
      \end{subarray}}
 P(i,j)
\end{verbatim}
\end{minipage}
&
$\displaystyle
  \sum_{\begin{subarray}{l}
        i\in \Lambda\\ 0<j<n
      \end{subarray}}
 P(i,j)$
\end{ctab}

%\section{The \cn{sideset} command}
\section{Lệnh \cn{sideset}}\label{sideset}

% There's also a command called \cn{sideset}, for a rather special
% purpose: putting symbols at the subscript and
% superscript\index{subscripts and superscripts!on sums} corners of a
% large operator symbol such as $\sum$ or $\prod$. \emph{Note: this
% command is not designed to be applied to anything other than sum-class symbols.} The prime
% example is the case when you want to put a prime on a sum symbol. If
% there are no limits above or below the sum, you could just use
% \cn{nolimits}: here's
Lệnh \cn{sideset} dùng cho mục đích đặc biệt, là đặt các ký hiệu vào
vị trí của chỉ số trên và chỉ số dưới của các toán tử lớn như $\sum$ hay $\prod$.
\emph{Chú ý rằng, lệnh này không có mục đích áp dụng vào các ký hiệu
không thuộc lớp ký hiệu tổng.}
Ví dụ dễ thấy nhất là khi bạn muốn có một dấu phảy để chỉ biến thể
của ký hiệu tổng:
%%%%%%%%%%%%%%%%%%%%%%%%%%%%%%%%%%%%%%%%%%%%%%%%%%%%%%%%%%%%%%%%%%%%%%%%
% |\sum\nolimits' E_n| in display mode:
\begin{equation}
\sum\nolimits' E_n
\end{equation}
Để được như trên, bạn có thể dùng |\sum\nolimits' E_n|; ở đây, \cn{nolimits}
phải được dùng, bởi nếu không, dấu phảy sẽ được đặt ở bên trên chứ không phải
là góc trên-bên-phải của ký hiệu tổng. Nếu bạn làm như vậy, thì nảy ra
vấn đề nan giải là, bây giờ khó có thể đặt thứ gì vào bên trên và bên dưới
của $\sum$ nữa (bởi bạn đã dùng \cn{nolimits}). Cách giải quyết vấn đề này
nhờ lệnh \cn{sideset} được minh họa trong ví dụ sau:
% If, however, you want not only the prime but also something below or
% above the sum symbol, it's not so easy\mdash indeed, without
% \cn{sideset}, it would be downright difficult. With \cn{sideset}, you
% can write

\medskip
\begin{ctab}{ll}
\begin{minipage}[t]{.6\columnwidth}
\begin{verbatim}
\sideset{}{'}
  \sum_{n<k,\;\text{$n$ odd}} nE_n
\end{verbatim}
\end{minipage}
&$\displaystyle
\sideset{}{'}\sum_{n<k,\;\text{$n$ odd}} nE_n
$
\end{ctab}
% The extra pair of empty braces is explained by the fact that
% \cn{sideset} has the capability of putting an extra symbol or symbols at
% each corner of a large operator; to put an asterisk at each corner of a
% product symbol, you would type
Phải dùng cặp dấu ngoặc rỗng (|{}|) như trên, là bởi lệnh \cn{sideset}
có khả năng đặt các ký hiệu vào bốn góc của toán tử lớn. Hãy học hỏi ví
dụ rất thú vị sau đây:

\medskip\begin{ctab}{ll}
\begin{minipage}[t]{.6\columnwidth}
\begin{verbatim}
\sideset{_*^*}{_*^*}\prod_{n=1}^{\infty}
\end{verbatim}
\end{minipage}
&$\displaystyle
\sideset{_*^*}{_*^*}\prod_{n=1}^{\infty}
$
\end{ctab}

%\section{Placement of subscripts and limits}
\section{Vị trí của chỉ số dưới và `limit'}\label{subplace}

% The default positioning for subscripts depends on the
% base symbol involved. The default for sum-class symbols is
% `displaylimits' positioning: When a sum-class symbol appears
% in a displayed formula, subscript and superscript are placed in `limits'
% position above and below, but in an inline formula, they are placed to
% the side, to avoid unsightly and wasteful spreading of the
% surrounding text lines.
Vị trí mặc định của chỉ số dưới của ký hiệu phụ thuộc vào tính chất
ký hiệu đó. Với các ký hiệu dạng tổng ($\sum$, $\prod$), vị trí đó
là `displaylimits': nghĩ là, khi ký hiệu dạng tổng xuất hiện trong
biểu thức riêng dòng, chỉ số trên và chỉ số dưới được bố trí bên trên
hoặc bên dưới ký hiệu; còn nếu ký hiệu đó xuất hiện trong chế độ chung dòng,
thì các chỉ số trên, dưới được đặt ở bên cạnh ký hiệu, nhằm tránh việc
dãn dòng qua mức cần thiết.
% The default for integral-class symbols is to have sub- and
% superscripts always to the side, even in displayed formulas.
% (See the discussion of the \opt{intlimits} and related options in
% Section~\ref{options}.)
Đối với các ký hiệu dạng tích phân, vị trí mặc định của các chỉ số luôn
là ở bên cạnh ký hiệu, ngay cả trong các công thức riêng dòng.
(Nhưng xem về tùy chọn \opt{intlimits} trong Mục~\ref{options}.)

% Operator names such as $\sin$ or $\lim$ may have either `displaylimits'
% or `limits' positioning depending on how they were defined. The standard
% operator names are defined according to normal mathematical usage.
\medskip
Các toán tử như $\sin$, $\lim$ đều có vị trí chỉ số thuộc loại
`displaylimits' hay `limits' tùy theo cách chúng được định nghĩa,
và điều đã được chọn theo thói quen phổ biến trong giới toán học.

% The commands \cn{limits} and \cn{nolimits} can be used to override the
% normal behavior of a base symbol:
\medskip
Các lệnh \cn{limits} và \cn{nolimits} dùng để bật qua lại giữa hai chế
độ đặt vị trí chỉ số:
\begin{equation*}
\sum\nolimits_X,\qquad \iint\limits_{A},
\qquad\varliminf\nolimits_{n\to \infty}
\end{equation*}
\begin{verbatim}
\sum\nolimits_X,\qquad \iint\limits_{A},
\qquad\varliminf\nolimits_{n\to \infty}
\end{verbatim}

% To define a command whose subscripts follow the
% same `displaylimits' behavior as \cn{sum}, put
% \cn{displaylimits} at the tail end of the definition. When multiple
% instances of \cn{limits}, \cn{nolimits}, or \cn{displaylimits} occur
% consecutively, the last one takes precedence.
\medskip
Để định nghĩa một lệnh mới mà chỉ số dưới của nó có thuộc tính
`displaylimits', bạn đặt lệnh \cn{displaylimits} và cuối định nghĩa
của lệnh. Trong dãy gồm các lệnh \cn{limits}, \cn{nolimits}, \cn{displaylimits}
tiếp nhau, thì lệnh cuối cùng có tác dụng ưu tiên.

%\section{Multiple integral signs}
\section{Dấu tích phân bội}

% \cn{iint}, \cn{iiint}, and \cn{iiiint} give multiple integral
% signs\index{integrals!multiple} with the spacing between them nicely
% adjusted, in both text and display style. \cn{idotsint} is an extension
% of the same idea that gives two integral signs with dots between them.
Các lệnh \cn{iiint}, \cn{iiint} và \cn{iiiint} cho ta các dấu tích phân bội
với khoảng cách giữa các dấu tích phân được tinh chỉnh tốt hơn.
Lệnh \cn{idotsint} cho ta dấu tích phân bội với số lớp tùy ý.
\begin{gather}
\iint\limits_A f(x,y)\,dx\,dy\qquad\iiint\limits_A
f(x,y,z)\,dx\,dy\,dz\\
\iiiint\limits_A
f(w,x,y,z)\,dw\,dx\,dy\,dz\qquad\idotsint\limits_A f(x_1,\dots,x_k)
\end{gather}

%\chapter{Commutative diagrams}
\chapter{Biểu đồ giao hoán}\label{s:commdiag}

% Some commutative diagram commands like the ones in \amstex/ are
% available as a separate package, \pkg{amscd}. For complex commutative
% diagrams authors will need to turn to more comprehensive packages like
% \pkg{kuvio} or \xypic/, but for simple diagrams without diagonal
% arrows\index{arrows!in commutative diagrams} the \pkg{amscd} commands
% may be more convenient. Here is one example.
Một số lệnh tạo biểu đồ giao hoán như trong \amstex/ được gộp thành 
gói riêng biệt, là \pkg{amscd}. Với các biểu đồ phức tạp, bạn nên dùng gói
chuyên dụng hơn là \pkg{kuvio} hoặc \pkg{xypic}; nhưng với các biểu đồ
đơn giản, không có các mũi tên chéo, thì dùng gói \pkg{amscd} sẽ nhanh
chóng cho kết quả. Dưới đây là ví dụ.
\begin{equation*}
\begin{CD}
S^{{\mathcal{W}}_\Lambda}\otimes T   @>j>>   T\\
@VVV                                    @VV{\End P}V\\
(S\otimes T)/I                  @=      (Z\otimes T)/J
\end{CD}
\end{equation*}
\begin{verbatim}
\begin{CD}
S^{{\mathcal{W}}_\Lambda}\otimes T   @>j>>   T\\
@VVV                                    @VV{\End P}V\\
(S\otimes T)/I                  @=      (Z\otimes T)/J
\end{CD}
\end{verbatim}
% In the \env{CD} environment the commands |@>>>|,
% |@<<<|, |@VVV|, and |@AAA| give respectively right, left, down, and up
% arrows. For the horizontal arrows, material between the first and second
% |>| or |<| symbols will be typeset as a superscript, and material
% between the second and third will be typeset as a subscript. Similarly,
% material between the first and second or second and third |A|s or |V|s
% of vertical arrows will be typeset as left or right \qq{sidescripts}.
% The commands |@=| and \verb'@|' give horizontal and vertical double lines.
% A \qq{null arrow} command |@.| can be used instead of a visible arrow
% to fill out an array where needed.
Trong môi trường \env{CD}, các lệnh |@>>>|, |@<<<|, |@VVV| và |@AAA|
lần lượt cho các mũi tên qua phải, trái, xuống, lên. Với các mũi tên ngang,
mọi thứ đặt giữa hai ký hiệu |>| hoặc |<| đầu tiên sẽ được xem là chỉ số trên,
các đối tượng đặt giữa ký hiệu thứ hai và thứ ba được xem là chỉ số dưới.
Tương tự, đối tượng đặt giữa ký hiệu |A| hay |V| thứ nhất và thứ hai
sẽ là chỉ số bên trái, còn đối tượng đặt giữa ký hiệu |A| hay |V| thứ nhì và
thứ ba sẽ là chỉ số bên phải. Các lệnh |@=| và \verb'@|'
sẽ cho các đường-ngang-kép và đường-đứng-kép. Một \qq{mũi tên rỗng}
được cho bởi lệnh |@.|.


%%%%%%%%%%%%%%%%%%%%%%%%%%%%%%%%%%%%%%%%%%%%%%%%%%%%%%%%%%%%%%%%%%%%%%%%
%\chapter{Using math fonts}
\chapter{Sử dụng `font' toán}

\section{Giới thiệu}

% For more comprehensive information on font use in \latex/, see the
% \latex/ font guide (\fn{fntguide.tex}) or \booktitle{The \latex/
% Companion} \cite{tlc}. The basic set of math font commands\index{math
% fonts}\relax \index{math symbols|see{math fonts}} in \latex/ includes
% \cn{mathbf}, \cn{mathrm}, \cn{mathcal}, \cn{mathsf}, \cn{mathtt},
% \cn{mathit}. Additional math alphabet commands such as
% \cn{mathbb} for blackboard bold, \cn{mathfrak} for Fraktur, and
% \cn{mathscr} for Euler script are available through the packages
% \pkg{amsfonts} and \pkg{euscript} (distributed separately).
Mọi vấn đề về |font| có thể tìm thấy trong hướng dẫn (\fn{fntguide.tex})
hoặc \booktitle{The \latex/ Companion} \cite{tlc}.

\medskip
Các lệnh cơ bản (của \latex/)
điều chỉnh |font| trong chế độ toán bao gồm:
\cn{mathbf}, \cn{mathrm}, \cn{mathcal}, \cn{mathsf}, \cn{mathtt}, \cn{mathit}.
Ngoài ra, gói \pkg{amsfonts} cung cấp các lệnh đặc biệt:
\cn{mathbb} cho kiểu `blackboard bold' ($\mathbb R$),
\cn{mathfrak} cho kiểu `Fraktur' ($\mathfrak R$).
% và \cn{mathscr} cho kiểu `Euler-script' ($\mathscr r$).

%\section{Recommended use of math font commands}
\section{Lời khuyên}

% If you find yourself employing math font commands frequently in your
% document, you might wish that they had shorter names, such as \ncn{mb}
% instead of \cn{mathbf}. Of course, there is nothing to keep you from
% providing such abbreviations for yourself by suitable \cn{newcommand}
% statements. But for \latex/ to provide shorter names would actually be a
% disservice to authors, as that would obscure a much better alternative:
% defining custom command names derived from the names of the underlying
% mathematical objects, rather than from the names of the fonts used to
% distinguish the objects. For example, if you are using bold to indicate
% vectors, then you will be better served in the long run if you define a
% `vector' command instead of a `math-bold' command:
Nếu bạn phải thường xuyên thay đổi kiểu |font| trong chế độ toán,
bạn có thể nghĩ đến các định nghĩa vắn tắt, chẳng hạn, dùng \ncn{mb}
thay vì \cn{mathbf}. Tất nhiên, không gì ngăn cản bạn làm điều đó,
vì bạn có trong tay lệnh-tạo-định-nghĩa \cn{newcommand}. Nhưng với \latex/,
việc đưa ra các dạng định nghĩa vắn tắt như vậy cần nên tránh, cần thay thế
bởi giải pháp hay hơn: \emph{định nghĩa một lệnh dựa trên \emph{thuộc tính của
đối tượng toán học}, hơn là dựa trên tên của |font| dùng để phân biệt
các đối tượng.} Ví dụ, nếu bạn muốn dùng |font| đậm chỉ các véctơ, bạn
nên định nghĩa một lệnh tạo véctơ với |font| đậm, hơn là định nghĩa một
lệnh tạo |font| đậm để tạo véctơ; nhờ

\medskip
\begin{verbatim}
  \newcommand{\vect}[1]{\mathbf{#1}}
\end{verbatim}

\medskip\noindent
bạn có thể viết |\vect{a} + \vect{b}| để có $\vect{a}+\vect{b}$;
bạn không nên dùng, chẳng hạn \cn{mb}|{a}|+\ncn{mb}|{b}| với lệnh \ncn{mb}
là dạng viết tắt của \cn{mathbf}.
% If you decide several months down the road that you want to use the bold
% font for some other purpose, and mark vectors by a small over-arrow
% instead, then you can put the change into effect merely by changing the
% definition of \ncn{vect}; otherwise you would have to replace all
% occurrences of \cn{mathbf} throughout your document, perhaps even
% needing to inspect each one to see whether it is indeed
% an instance of a vector.
Sau một vài tháng chẳng hạn, bạn thấy cần phải dùng |font| đậm cho mục đích khác,
chứ không phải để chỉ các véctơ, và muốn các véctơ bây giờ có thêm các mũi tên,
thì bạn có thể dễ dàng thực hiện được ý đồ của mình nếu bạn dùng định nghĩa
\cn{vect} để tạo véctơ; còn nếu ngược lại, bạn phải thay thế toàn bộ các lệnh
\cn{mb} trong tài liệu của bạn\mdash điều này nguy hiểm ở chỗ, việc thay thế
có thể ảnh hưởng đến các đối tượng không là véctơ nhưng được biểu diễn bằng
\cn{mb}.

% It can also be useful to assign distinct
% command names for different letters of a particular font:
\medskip
Sẽ có ích nếu bạn tạo các lệnh mới để lấy ra các ký tự đặc biệt trong
một |font| cụ thể nào đó, ví dụ

\medskip
\begin{verbatim}
\DeclareSymbolFont{AMSb}{U}{msb}{m}{n}% or use amsfonts package
\DeclareMathSymbol{\C}{\mathalpha}{AMSb}{"43}
\DeclareMathSymbol{\R}{\mathalpha}{AMSb}{"52}
\end{verbatim}

% These statements would define the commands \cn{C} and \cn{R} to produce
% blackboard-bold letters from the `AMSb' math symbols font. If you refer
% often to the\break complex numbers or real numbers in your document, you
% might find this method more convenient than (let's say) defining a
% \ncn{field} command and writing\break |\field{C}|, |\field{R}|. But for
% maximum flexibility and control, define such a \ncn{field} command and
% then define \ncn{C} and \ncn{R} in terms of that
% command:\index{mathbb@\cn{mathbb}}
\medskip\noindent
Các dòng ở trên định nghĩa các lệnh \cn{C}, \cn{R} để lấy ra ký hiệu
`blackboard' từ bộ `AMSb' là |font| các ký hiệu. Nếu bạn thường xuyên
làm việc với các trường số thực, phức, bạn có thể có một cách tiện lợi, là
định nghĩa lệnh \ncn{field} và sau đó dùng |\field{C}|, |\field{R}|,...
để có các trường số mong muốn. Nhưng để \emph{tăng tối đa tính uyển chuyển
và khả năng điều khiển}, bạn hãy định nghĩa lệnh \ncn{field} rồi sau đó
định nghĩa các lệnh \ncn{C}, \ncn{R} dựa trên \ncn{field} như sau:

\medskip
\begin{verbatim}
\usepackage{amsfonts}% to get the \mathbb alphabet
\newcommand{\field}[1]{\mathbb{#1}}
\newcommand{\C}{\field{C}}
\newcommand{\R}{\field{R}}
\end{verbatim}

%\section{Bold math symbols}
\section{Các ký hiệu in đậm}

% The \cn{mathbf} command is commonly used to obtain bold Latin letters in
% math, but for most other kinds of math symbols it has no effect, or its
% effects depend unreliably on the set of math fonts that are in use. For
% example, writing
Lệnh \cn{mathbf} thường được dùng để có được phiên bản đậm của các chữ cái
trong chế độ toán, nhưng với hầu hết các ký hiệu toán khác (không phải là chữ cái),
lệnh này không có tác dụng, hoặc có tác dụng nhưng kết quả
của nó không có liên hệ chặt chẽ với |font| đang được dùng. Ví dụ, viết

\medskip
\begin{verbatim}
\Delta \mathbf{\Delta}\mathbf{+}\delta \mathbf{\delta}
\end{verbatim}

\medskip\noindent
sẽ cho bạn $\Delta \mathbf{\Delta}\mathbf{+}\delta \mathbf{\delta}$;
%the \cn{mathbf} has no effect on the plus sign or the small delta.
để ý rằng, trong kết quả thu được, các dấu cộng và dấu $\delta$ không
bị ảnh hưởng bởi \cn{mathbf}.

% The \pkg{amsmath} package therefore provides two additional commands,
% \cn{boldsymbol} and \cn{pmb}, that can be applied to other kinds of math
% symbols. \cn{boldsymbol} can be used for a math symbol that remains
% unaffected by \cn{mathbf} if (and only if) your current math font set
% includes a bold version of that symbol. \cn{pmb} can be used as a last
% resort for any math symbols that do not have a true bold version
% provided by your set of math fonts; \qq{pmb} stands for \qq{poor man's
% bold} and the command works by typesetting multiple copies of the symbol
% with slight offsets. The quality of the output is inferior, especially
% for symbols that contain any hairline strokes. When the standard default set of
% \latex/ math fonts are in use (Computer Modern), the only symbols that
% are likely to require \cn{pmb} are large operator symbols like \cn{sum},
% extended delimiter symbols, or the extra math symbols provided by
% the \pkg{amssymb} package \cite{amsfonts}.
\medskip
Gói \pkg{amsmath} vì vậy cung cấp thêm hai lệnh mới, là
\cn{boldsymbol} và \cn{pmb}; các lệnh này có tác dụng làm đậm mọi loại
ký hiệu. Lệnh \cn{boldsymbol} có thể được dùng cho các ký hiệu toán không
bị ảnh hưởng bởi lệnh \cn{mathbf} \emph{nếu và chỉ nếu} |font| toán
đang dùng có phiên bản |đậm| tương ứng của ký hiệu đó. Lệnh \cn{pmb}
được dùng, như là giải pháp cuối cùng, cho trường hợp lệnh \cn{boldsymbol}
bất lực; \qq{pmb} là viết tắt của \qq{poor man's bold};
nguyên lý làm việc của lệnh \cn{pmb} là |typeset| nhiều bản sao của
ký hiệu và sắp xếp các bản sao đó sát bên nhau (giống như khi bạn
dùng bút kẻ nhiều đường liên tiếp nhau trên giấy để có một đường đậm nét).
Rõ ràng, chất lượng kết quả của lệnh \cn{pmb} là không tốt lắm, nhất là
độ nét của ký hiệu. Trong khi bộ |font| chuẩn của \latex/ được dùng để
|typeset| các biểu thức toán, là bộ |font| CM (Computer Modern), thì cần đến
lệnh \cn{pmb} chỉ là các ký hiệu toán tử lớn, như \cn{sum},
các dấu ngoặc quá cỡ, hay là các ký hiệu toán học bổ sung bởi gói
\pkg{amssymb} (xem \cite{amsfonts}).

% The following formula shows some of the results that are possible:
\medskip
Các biểu thức sau cho thấy một số kết quả khả dĩ:
\begin{verbatim}
A_\infty + \pi A_0
\sim \mathbf{A}_{\boldsymbol{\infty}} \boldsymbol{+}
  \boldsymbol{\pi} \mathbf{A}_{\boldsymbol{0}}
\sim\pmb{A}_{\pmb{\infty}} \pmb{+}\pmb{\pi} \pmb{A}_{\pmb{0}}
\end{verbatim}
\begin{equation*}
A_\infty + \pi A_0
\sim \mathbf{A}_{\boldsymbol{\infty}} \boldsymbol{+}
  \boldsymbol{\pi} \mathbf{A}_{\boldsymbol{0}}
\sim\pmb{A}_{\pmb{\infty}} \pmb{+}\pmb{\pi} \pmb{A}_{\pmb{0}}
\end{equation*}
% If you want to use only the \cn{boldsymbol} command without loading the
% whole \pkg{amsmath} package, the \pkg{bm} package is recommended (this
% is a standard \latex/ package, not an AMS package; you probably have it
% already if you have a 1997 or newer version of \latex/).

\medskip\noindent
Nếu bạn muốn dùng lệnh \cn{boldsymbol} độc lập mà không tải gói \pkg{amsmath},
thì hãy dùng gói \pkg{bm}; đây là gói thuộc chuẩn \latex/, không thuộc bộ
phân phối của các gói AMS. Gói này tích hợp trong bộ \latex/ phiên
bản năm 1997 hoặc cao hơn.

%\section{Italic Greek letters}
\section{Các chữ cái Hy Lạp in nghiêng}

% For italic versions of the capital Greek letters, the following commands
% are provided:
\begin{ctab}{rlrl}
\cn{varGamma}& $\varGamma$& \cn{varSigma}& $\varSigma$\\
\cn{varDelta}& $\varDelta$& \cn{varUpsilon}& $\varUpsilon$\\
\cn{varTheta}& $\varTheta$& \cn{varPhi}& $\varPhi$\\
\cn{varLambda}& $\varLambda$& \cn{varPsi}& $\varPsi$\\
\cn{varXi}& $\varXi$& \cn{varOmega}& $\varOmega$\\
\cn{varPi}& $\varPi$
\end{ctab}

%\chapter{Error messages and output problems}
\newpage
\chapter{Lỗi thường gặp khi dùng gói \pkg{amsmath}}

\section{General remarks}

This is a supplement to Chapter~8 of the \latex/ manual \cite{lamport} (first
edition: Chapter~6). For the reader's convenience, the set of error
messages discussed here overlaps somewhat with the set in that chapter,
but please be aware that we don't provide exhaustive coverage here.
The error messages are arranged in alphabetical order, disregarding
unimportant text such as |! LaTeX Error:| at the beginning, and
nonalphabetical characters such as \qc{\\}. Where examples are given, we
show also the help messages that appear on screen when you respond to an
error message prompt by entering |h|.

There is also a section discussing some output errors, i.e., instances
where the printed document has something wrong but there was no \latex/
error during typesetting.

\section{Error messages}

\begin{error}{\begin{split} won't work here.}
\errexa
\begin{verbatim}
! Package amsmath Error: \begin{split} won't work here.
 ...

l.8 \begin{split}

? h
\Did you forget a preceding \begin{equation}?
If not, perhaps the `aligned' environment is what you want.
?
\end{verbatim}
\errexpl
The \env{split} environment does not construct a stand-alone displayed
equation; it needs to be used within some other environment such as
\env{equation} or \env{gather}.

\end{error}

\begin{error}{Extra & on this line}
\errexa
\begin{verbatim}
! Package amsmath Error: Extra & on this line.

See the amsmath package documentation for explanation.
Type  H <return>  for immediate help.
 ...

l.9 \end{alignat}

? h
\An extra & here is so disastrous that you should probably exit
 and fix things up.
?
\end{verbatim}
\errexpl
In an \env{alignat} structure the number of alignment points per line
is dictated by the numeric argument given after |\begin{alignat}|.
If you use more alignment points in a line it is assumed that you
accidentally left out a newline command \cn{\\} and the above error is
issued.
\end{error}

\begin{error}{Improper argument for math accent}
\errexa
\begin{verbatim}
! Package amsmath Error: Improper argument for math accent:
(amsmath)                Extra braces must be added to
(amsmath)                prevent wrong output.

See the amsmath package documentation for explanation.
Type  H <return>  for immediate help.
 ...

l.415 \tilde k_{\lambda_j} = P_{\tilde \mathcal
                                               {M}}
?
\end{verbatim}
\errexpl
Non-simple arguments for any \LaTeX{} command should be enclosed in
braces. In this example extra braces are needed as follows:
\begin{verbatim}
... P_{\tilde{\mathcal{M}}}
\end{verbatim}
\end{error}

\begin{error}{Font OMX/cmex/m/n/7=cmex7 not loadable ...}
\errexa
\begin{verbatim}
! Font OMX/cmex/m/n/7=cmex7 not loadable: Metric (TFM) file not found.
<to be read again>
                   relax
l.8 $a
      b+b^2$
? h
I wasn't able to read the size data for this font,
so I will ignore the font specification.
[Wizards can fix TFM files using TFtoPL/PLtoTF.]
You might try inserting a different font spec;
e.g., type `I\font<same font id>=<substitute font name>'.
?
\end{verbatim}
\errexpl
Certain extra sizes of some Computer Modern fonts that were formerly
available mainly through the AMSFonts\index{AMSFonts collection}
distribution are considered part of standard \latex/ (as of June 1994):
\fn{cmex7}\ndash \texttt{9}, \fn{cmmib5}\ndash \texttt{9}, and
\fn{cmbsy5}\ndash \texttt{9}. If these extra sizes are missing on your
system, you should try first to get them from the source where you
obtained \latex/. If that fails, you could try getting the fonts from
CTAN (e.g., in the form of Metafont\index{Metafont source files} source
files, directory \nfn{/tex-archive/fonts/latex/mf}, or in PostScript
Type 1 format, directory
\nfn{/tex-archive/fonts/cm/ps-type1/bakoma}\index{BaKoMa fonts}\relax
\index{PostScript fonts}).

If the font name begins with \fn{cmex}, there is a special option
\fn{cmex10} for the \pkg{amsmath} package that provides a temporary
workaround. I.e., change the \cn{usepackage} to
\begin{verbatim}
\usepackage[cmex10]{amsmath}
\end{verbatim}
This will force the use of the 10-point size of the \fn{cmex} font in
all cases. Depending on the contents of your document this may be
adequate.
\end{error}

\begin{error}{Math formula deleted: Insufficient extension fonts}
\errexa
\begin{verbatim}
! Math formula deleted: Insufficient extension fonts.
l.8 $ab+b^2$

?
\end{verbatim}
\errexpl
This usually follows a previous error |Font ... not loadable|; see the
discussion of that error (above) for solutions.
\end{error}

\begin{error}{Missing number, treated as zero}
\errexa
\begin{verbatim}
! Missing number, treated as zero.
<to be read again>
                   a
l.100 \end{alignat}

? h
A number should have been here; I inserted `0'.
(If you can't figure out why I needed to see a number,
look up `weird error' in the index to The TeXbook.)

?
\end{verbatim}
\errexpl
There are many possibilities that can lead to this error. However, one
possibility that is relevant for the \pkg{amsmath} package is that you
forgot to give the number argument of an \env{alignat} environment, as
in:
\begin{verbatim}
\begin{alignat}
 a&  =b&    c& =d\\
a'& =b'&   c'& =d'
\end{alignat}
\end{verbatim}
where the first line should read instead
\begin{verbatim}
\begin{alignat}{2}
\end{verbatim}

Another possibility is that you have a left bracket character |[|
following a linebreak command \cn{\\} in a multiline construction such
as \env{array}, \env{tabular}, or \env{eqnarray}. This will be
interpreted by \latex/ as the beginning of an `additional vertical
space' request \cite[\S C.1.6]{lamport}, even if it occurs on the following
line and is intended to be part of the contents. For example
\begin{verbatim}
\begin{array}
a+b\\
[f,g]\\
m+n
\end{array}
\end{verbatim}
To prevent the error message in such a case, you can
add braces as discussed in the \latex/ manual \cite[\S C.1.1]{lamport}:
\begin{verbatim}
\begin{array}
a+b\\
{[f,g]}\\
m+n
\end{array}
\end{verbatim}

\end{error}

\begin{error}{Missing \right. inserted}
\errexa
\begin{verbatim}
! Missing \right. inserted.
<inserted text>
                \right .
l.10 \end{multline}

? h
I've inserted something that you may have forgotten.
(See the <inserted text> above.)
With luck, this will get me unwedged. But if you
really didn't forget anything, try typing `2' now; then
my insertion and my current dilemma will both disappear.
\end{verbatim}
\errexpl
This error typically arises when you try to insert a linebreak inside a
\cn{left}-\cn{right} pair of delimiters in a \env{multline} or
\env{split} environment:
\begin{verbatim}
\begin{multline}
AAA\left(BBB\\
  CCC\right)
\end{multline}
\end{verbatim}
There are two possible solutions: (1)~instead of using \cn{left} and
\cn{right}, use `big' delimiters of fixed size (\cn{bigl} \cn{bigr}
\cn{biggl} \cn{biggr} \dots; see \secref{bigdel}); or (2)~use null
delimiters to break up the \cn{left}-\cn{right} pair into parts for each
line:
\begin{verbatim}
AAA\left(BBB\right.\\
  \left.CCC\right)
\end{verbatim}
The latter solution may result in mismatched delimiter sizes;
ensuring that they match requires using \cn{vphantom} in the line
that has the smaller delimiter (or possibly \cn{smash} in the line that
has the larger delimiter). In the argument of \cn{vphantom} put a copy
of the tallest element that occurs in the other line, e.g.,
\begin{verbatim}
xxx \left(\int_t yyy\right.\\
  \left.\vphantom{\int_t} zzz ... \right)
\end{verbatim}
\end{error}

\begin{error}{Paragraph ended before \xxx was complete}
\errexa
\begin{verbatim}
Runaway argument?

! Paragraph ended before \multline was complete.
<to be read again>
                   \par
l.100

? h
I suspect you've forgotten a `}', causing me to apply this
control sequence to too much text. How can we recover?
My plan is to forget the whole thing and hope for the best.
?
\end{verbatim}
\errexpl
This might be produced by a misspelling in the |\end{multline}| command,
e.g.,
\begin{verbatim}
\begin{multline}
...
\end{multiline}
\end{verbatim}
or by using abbreviations for certain environments, such as |\bal| and
|\eal| for |\begin{align}| and |\end{align}|:
\begin{verbatim}
\bal
...
\eal
\end{verbatim}
For technical reasons that kind of abbreviation does not work with
the more complex displayed equation environments of the \pkg{amsmath} package
(\env{gather}, \env{align}, \env{split}, etc.; cf.\@ \fn{technote.tex}).
\end{error}

\begin{error}{Runaway argument?}
See the discussion for the error message
\texttt{Paragraph ended before \ncn{xxx} was complete}.
\end{error}

\begin{error}{Unknown option `xxx' for package `yyy'}
\errexa
\begin{verbatim}
! LaTeX Error: Unknown option `intlim' for package `amsmath'.
...
? h
The option `intlim' was not declared in package `amsmath', perhaps you
misspelled its name. Try typing  <return>  to proceed.
?
\end{verbatim}
\errexpl
This means that you misspelled the option name, or the package simply
does not have an option that you expected it to have. Consult the
documentation for the given package.
\end{error}

\begin{error}{Old form `\pmatrix' should be \begin{pmatrix}.}
\errexa
\begin{verbatim}
! Package amsmath Error: Old form `\pmatrix' should be
                         \begin{pmatrix}.

See the amsmath package documentation for explanation.
Type  H <return>  for immediate help.
 ...

\pmatrix ->\left (\matrix@check \pmatrix
                                         \env@matrix
l.16 \pmatrix
             {a&b\cr c&d\cr}
? h
`\pmatrix{...}' is old Plain-TeX syntax whose use is
ill-advised in LaTeX.
?
\end{verbatim}
\errexpl
When the \pkg{amsmath} package is used, the old forms of \cn{pmatrix},
\cn{matrix}, and \cn{cases} cannot be used any longer because of naming
conflicts. Their syntax did not conform with standard \LaTeX{} syntax
in any case.
\end{error}

\begin{error}{Erroneous nesting of equation structures}
\errexa
\begin{verbatim}
! Package amsmath Error: Erroneous nesting of equation structures;
(amsmath)                trying to recover with `aligned'.

See the amsmath package documentation for explanation.
Type  H <return>  for immediate help.
 ...

l.260 \end{alignat*}
                    \end{equation*}
\end{verbatim}
\errexpl
The structures \env{align}, \env{alignat}, etc., are designed
for top-level use and for the most part cannot be nested inside some
other displayed equation structure. The chief exception is that
\env{align} and most of its variants can be used inside the
\env{gather} environment.
\end{error}

\section{Warning messages}

\begin{error}{Foreign command \over [or \atop or \above]}
\errexa
\begin{verbatim}
Package amsmath Warning: Foreign command \over; \frac or \genfrac
(amsmath)                should be used instead.
\end{verbatim}
\errexpl The primitive generalized fraction commands of \tex/\mdash
\cs{over}, \cs{atop}, \cs{above}\mdash are deprecated when the
\pkg{amsmath} package is used because their syntax is foreign to \latex/
and \pkg{amsmath} provides native \latex/ equivalents. See
\fn{technote.tex} for further information.
\end{error}

\begin{error}{Cannot use `split' here}
\errexa
\begin{verbatim}
Package amsmath Warning: Cannot use `split' here;
(amsmath)                trying to recover with `aligned'
\end{verbatim}
\errexpl The \env{split} environment is designed to serve as the entire
body of an equation, or an entire line of an \env{align} or \env{gather}
environment. There cannot be any printed material before or
after it within the same enclosing structure:
\begin{verbatim}
\begin{equation}
\left\{ % <-- Not allowed
\begin{split}
...
\end{split}
\right. % <-- Not allowed
\end{equation}
\end{verbatim}
\end{error}

\section{Wrong output}

\subsection{Section numbers 0.1, 5.1, 8.1 instead of 1, 2, 3}
\label{numinverse}

This most likely means that you have the arguments for \cn{numberwithin}
in reverse order:
\begin{verbatim}
\numberwithin{section}{equation}
\end{verbatim}
That means `print the section number as \textit{equation
number}.\textit{section number} and reset to 1 every time an equation
occurs' when what you probably wanted was the inverse
\begin{verbatim}
\numberwithin{equation}{section}
\end{verbatim}

\subsection{The \cn{numberwithin} command had no effect on equation
numbers}

Are you looking at the first section in your document? Check the section
numbers elsewhere to see if the problem is the one described in
\secref{numinverse}.

%%%%%%%%%%%%%%%%%%%%%%%%%%%%%%%%%%%%%%%%%%%%%%%%%%%%%%%%%%%%%%%%%%%%%%%%

%%%% \chapter{Additional information}
%%%% 
%%%% \section{Converting existing documents}
%%%% 
%%%% \subsection{Converting from plain \LaTeX{}}
%%%% 
%%%% A \LaTeX{} document will typically continue to work the same in most
%%%% respects if \verb'\usepackage{amsmath}' is added in the document
%%%% preamble. By default, however, the \pkg{amsmath} package suppresses page
%%%% breaks inside multiple-line displayed equation structures such as
%%%% \env{eqnarray}, \env{align}, and \env{gather}. To continue allowing page
%%%% breaks inside \env{eqnarray} after switching to \pkg{amsmath}, you will
%%%% need to add the following line in your document preamble:
%%%% \begin{verbatim}
%%%% \allowdisplaybreaks[1]
%%%% \end{verbatim}
%%%% To ensure normal spacing around relation symbols, you might also want to
%%%% change \env{eqnarray} to \env{align}, \env{multline}, or
%%%% \env{equation}\slash\env{split} as appropriate.
%%%% 
%%%% Most of the other differences in \pkg{amsmath} usage can be considered
%%%% optional refinements, e.g., using
%%%% \begin{verbatim}
%%%% \DeclareMathOperator{\Hom}{Hom}
%%%% \end{verbatim}
%%%% instead of \verb'\newcommand{\Hom}{\mbox{Hom}}'.
%%%% 
%%%% \subsection{Converting from \amslatex/ 1.1}
%%%% See \fn{diffs-m.txt}.
%%%% 
%%%% \section{Technical notes}
%%%% The file \fn{technote.tex} contains some remarks on miscellaneous
%%%% technical questions that are less likely to be of general interest.
%%%% 
%%%% \section{Getting help}
%%%% 
%%%% Questions or comments regarding \pkg{amsmath} and related packages
%%%% should be sent to:
%%%% \begin{infoaddress}
%%%% American Mathematical Society\\
%%%% Technical Support\\
%%%% Electronic Products and Services\\
%%%% P. O. Box 6248\\
%%%% Providence, RI 02940\\[3pt]
%%%% Phone: 800-321-4AMS (321-4267) \quad or \quad 401-455-4080\\
%%%% Internet: \mail{tech-support@ams.org}
%%%% \end{infoaddress}
%%%% If you are reporting a problem you should include
%%%% the following information to make proper investigation possible:
%%%% \begin{enumerate}
%%%% \item The source file where the problem occurred, preferably reduced
%%%%   to minimum size by removing any material that can be removed without
%%%%   affecting the observed problem.
%%%% \item A \latex/ log file showing the error message (if applicable) and
%%%%   the version numbers of the document class and option files being used.
%%%% \end{enumerate}
%%%% 
%%%% \section{Of possible interest}\label{a:possible-interest}
%%%% Information about obtaining AMSFonts or other \tex/-related
%%%% software from the AMS Internet archive \fn{e-math.ams.org}
%%%% can be obtained by sending a request through electronic mail to:
%%%% \mail{webmaster@ams.org}.
%%%% 
%%%% Information about obtaining the \pkg{amsmath} distribution on diskette
%%%% from the AMS is available from:
%%%% \begin{infoaddress}
%%%% American Mathematical Society\\
%%%% Customer Services\\
%%%% P. O. Box 6248\\
%%%% Providence, RI 02940\\[3pt]
%%%% Phone: 800-321-4AMS (321-4267) \quad or \quad 401-455-4000\\
%%%% Internet: \mail{cust-serv@ams.org}
%%%% \end{infoaddress}
%%%% 
%%%% The \tex/ Users Group\index{TeX Users@\tex/ Users Group} is a nonprofit
%%%% organization that publishes a journal
%%%% (\journalname{TUGboat}\index{TUGboat@\journalname{TUGboat}}), holds
%%%% meetings, and serves as a clearing-house of general information about
%%%% \tex/ and \tex/-related software.
%%%% \begin{infoaddress}
%%%% \tex/ Users Group\\
%%%% PO Box 2311\\
%%%% Portland, OR 97208-2311\\
%%%% USA\\[3pt]
%%%% Phone: +1-503-223-9994\\
%%%% Email: \mail{office@tug.org}
%%%% \end{infoaddress}
%%%% Membership in the \tex/ Users Group is a good way to support continued
%%%% development of free \tex/-related software. There are also many local
%%%% \tex/ user groups in other countries; information about contacting a
%%%% local user group can be gotten from the \tex/ Users Group.
%%%% 
%%%% There is a Usenet newsgroup called \fn{comp.text.tex} that is a fairly
%%%% good source of information about \latex/ and \tex/ in general. If you
%%%% don't know about reading newsgroups, check with your local system
%%%% administrator to see if newsgroup service is available at your site.

\newpage

\begin{thebibliography}{9}
\addcontentsline{toc}{chapter}{Tài liệu tham khảo}

\bibitem{amsfonts}\booktitle{AMSFonts version \textup{2.2}\mdash user's guide},
Amer. Math. Soc., Providence, RI, 1994; distributed
with the AMSFonts package.

\bibitem{instr-l}\booktitle{Instructions for preparation of
papers and monographs\mdash \amslatex/},
Amer. Math. Soc., Providence, RI, 1996, 1999.

\bibitem{amsthdoc}\booktitle{Using the \pkg{amsthm} Package},
Amer. Math. Soc., Providence, RI, 1999.

\bibitem{tlc} Michel Goossens, Frank Mittelbach, and Alexander Samarin,
\booktitle{The \latex/ companion}, Addison-Wesley, Reading, MA, 1994.
  [\emph{Note: The 1994 edition is not a reliable guide for the
    \pkg{amsmath} package unless you refer to the errata for Chapter
    8\mdash file \fn{compan.err}, distributed with \LaTeX{}.}]

% Deal with a line breaking problem
\begin{raggedright}
\bibitem{mil} G. Gr\"{a}tzer,
\emph{Math into \LaTeX{}: An Introduction to \LaTeX{} and AMS-\LaTeX{}}
  \url{http://www.ams.org/cgi-bin/bookstore/bookpromo?fn=91&arg1=bookvideo&itmc=MLTEX},
Birkh\"{a}user, Boston, 1995.\par
\end{raggedright}

\bibitem{kn} Donald E. Knuth, \booktitle{The \tex/book},
Addison-Wesley, Reading, MA, 1984.

\bibitem{lamport} Leslie Lamport, \booktitle{\latex/: A document preparation
system}, 2nd revised ed., Addison-Wesley, Reading, MA, 1994.

\bibitem{msf} Frank Mittelbach and Rainer Sch\"opf,
\textit{The new font family selection\mdash user
interface to standard \latex/}, \journalname{TUGboat} \textbf{11},
no.~2 (June 1990), pp.~297\ndash 305.

\bibitem{jt} Michael Spivak, \booktitle{The joy of \tex/}, 2nd revised ed.,
Amer. Math. Soc., Providence, RI, 1990.

\end{thebibliography}

%\begin{theindex}

%\end{theindex}

\end{document}

% Local Variables:
% coding: utf-8
% eval: (set-input-method "vietnamese-viqr")
% End:
