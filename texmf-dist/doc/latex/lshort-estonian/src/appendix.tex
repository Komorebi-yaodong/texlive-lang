\appendix
\chapter{\LaTeX i installimine}
\begin{intro}
\index{Knuth, Donald E.}Knuth avaldas \TeX i lähteteksti ajal, mil
avatud kood ja/või vaba tarkvara olid veel tundmatud mõisted. \TeX iga
kaasasolev litsents lubab lähtetekstiga teha mida iganes, kuid töö
tulemust võib nimetada \TeX iks ainult siis, kui programm läbib testid,
mille \index{Knuth, Donald E.}Knuth samuti avaldas. See on viinud
olukorrani, kus vaba \TeX isüsteem on olemas peaaegu iga
operatsioonisüsteemi jaoks päikese all. Käesolev peatükk annab nõu, mida
tuleks installida Linuxis, OS X-s ja Windowsis, et \TeX{} seal tööle
panna.
\end{intro}

\section{Mida installida}

Igasuguses arvutisüsteemis läheb \LaTeX i kasutamiseks vaja mitut
programmi, mis peaksid seega olema süsteemis kättesaadavad.

\begin{enumerate}

\item \TeX i / \LaTeX i programm, mis teisendab \LaTeX i lähtetekstid
PDF- või DVI-failideks.

\item Tekstiredaktor \LaTeX i lähtefailide redigeerimiseks. Mõned
süsteemid lubavad käivitada \LaTeX i isegi otse tekstiredaktori seest.

\item PDF-/DVI-failide vaatamisprogramm, millega saab dokumente
ekraanil vaadata ja printida.

\item Programm \PSi i failidega ja dokumenti lisatavate piltidega
ümberkäimiseks.

\end{enumerate}

Iga platvormi jaoks on olemas mitu programmi, mis nendele nõuetele
vastavad. Siin tutvustame ainult neid, mida meie tunneme, armastame ning
millega meil on kogemusi.

\section{Platvormiülene redaktor}
\label{sec:texmaker}

Kuigi \TeX{} on olemas paljude erinevate arvutiplatvormide jaoks, on
\LaTeX i redaktorid pikka aega olnud väga platvormispetsiifilised.

Mõne viimase aasta jooksul on mulle väga meeldima hakanud programm
\wi{Texmaker}. Olles väga kasulik tekstiredaktor integreeritud
PDF-vaaturiga ning süntaksi esiletõstmisega, on tal lisaks veel see
eelis, et ta töötab ühtviisi hästi nii Windowsi, Maci kui ka
Unixi/Linuxi all. Täpsemat infot saab aadressilt
\url{http://www.xm1math.net/texmaker}. Texmakerist on olemas ka
haruversioon nimega \wi{TeXstudio}
\url{http://texstudio.sourceforge.net}. Seda redaktorit hoitakse samuti
hästi korras ja ta on saadaval kõigi kolme peamise platvormi jaoks.

Mõned platvormispetsiifilised redaktorisoovitused leiab altpoolt
operatsioonisüsteemide jaotistest.

\section{\TeX{} ja Mac OS X}

\subsection{\TeX i distributsioon}

Lihtsalt laadi alla \index{mactex@Mac\TeX}Mac\TeX. See on
valmiskompileeritud \LaTeX i distributsioon OS X jaoks. Mac\TeX sisaldab
täielikku \LaTeX i installatsiooni ja hulka lisatööriistu. Mac\TeX i
leiab aadressilt \url{http://www.tug.org/mactex}.

\subsection{OS X \TeX i redaktor}

Kui platvormiülene soovitus \wi{Texmaker} (jaotis \ref{sec:texmaker}) ei
rahulda, siis kõige populaarsem vaba lähtekoodiga redaktor \LaTeX i
jaoks paistab olevat \index{TeXShop}TeX\-Shop, mille saab aadressilt
\url{http://www.uoregon.edu/~koch/texshop}. See sisaldub ka
\index{mactex@Mac\TeX}Mac\TeX i distributsioonis.

Hilisemad \index{texlive@\TeX{} Live}\TeX{} Live'i distributsioonid
sisaldavad redaktorit \wi{TeXworks} \url{http://www.tug.org/texworks},
mis on mitmeplatvormiline redaktor ja põhineb TeXShopi mudelil. Kuna
TeXworks kasutab Qt tööriistakomplekti, saab seda kasutada igal
platvormil, mis seda tööriistakomplekti toetab (Mac OS X, Windows,
Linux).

\subsection{Naudi \wi{PDFView}-d}

\LaTeX iga genereeritud PDF-failide vaatamiseks võib kasutada programmi
\wi{PDFView}, mis integreerub tihedalt \LaTeX i tekstiredaktoriga.
PDFView on vaba lähtekoodiga rakendus ning kättesaadav veebilehelt
\url{http://pdfview.sourceforge.net}. Pärast installimist tuleks avada
PDFView eelistuste dialoogiaken ja veenduda, et suvand \emph{Laadi
dokumendid automaatselt uuesti} oleks aktiveeritud ja et PDFSynci toetus
oleks seatud sobivalt.

\section{\TeX{} ja Windows}

\subsection{\TeX i hankimine}

Kõigepealt laadi alla suurepärane MiK\TeX i\index{miktex@MiK\TeX}
distributsioon aadressilt \url{http://www.miktex.org}. See sisaldab
kõiki vajalikke põhiprogramme ja faile \LaTeX i dokumentide
kompileerimiseks. Arvatavasti kõige toredam on see, et dokumendi
kompileerimise ajal laadib MiK\TeX{} puuduvad \LaTeX i paketid jooksvalt
alla ja installib need maagiliselt.

Teise võimalusena võib kasutada \index{texlive@\TeX{} Live}\TeX{} Live'i
distributsiooni \url{http://www.tug.org/texlive}, mis on olemas
Windowsi, Unixi ja Mac OS-i jaoks ning millega saab põhisüsteemi tööle.

\subsection{\LaTeX i redaktor}

Kui platvormiülene soovitus \wi{Texmaker} (jaotis \ref{sec:texmaker}) ei
rahulda, siis võib soovitada programmi \wi{TeXnicCenter}, mis loob
Windowsis kena ja efektiivse \LaTeX i kirjutamiskeskkonna, kasutades
mitmeid programmeerimismaailma mõisteid. Hankida saab seda aadressilt
\url{http://www.texniccenter.org}. TeXnicCenter töötab hästi koos
\index{miktex@MiK\TeX}MiK\TeX iga.

Viimased \index{texlive@\TeX{} Live}\TeX{} Live'i distributsioonid
sisaldavad redaktorit \wi{TeXworks} \url{http://www.tug.org/texworks},
mis toetab \wi{Unicode}'i ja nõuab vähemalt Windows XP-d.

\subsection{Dokumendi vaatamine}

Kõige tõenäolisemalt kasutad DVI vaatamiseks programmi \wi{Yap}, sest
see installitakse koos \index{miktex@MiK\TeX}MiK\TeX iga. PDF-failide
jaoks võib uurida programmi \wi{Sumatra PDF}
\url{http://www.sumatrapdfreader.org}. Nimetan programmi Sumatra PDF
sellepärast, et see lubab PDF-dokumendis hüpata igast positsioonist
vastavale positsioonile lähtedokumendis.

\subsection{Graafikaga töötamine}

\LaTeX is kõrge kvaliteediga graafikaga töötamine tähendab, et
joonisevorminguks peaks olema \ePSi{} (EPS) või PDF.
Programm, mis aitab sellega hakkama saada, on \wi{Ghostscript}, mille
võib koos juurdekuuluva liidesprogrammiga \wi{GhostView} alla laadida
aadressilt \url{http://pages.cs.wisc.edu/~ghost}.

Tegutsedes rastergraafikaga (fotod ja skannitud materjal), võib vaadata
Photoshopi avatud lähtetekstiga alternatiivi \wi{Gimp}, mis on saadaval
aadressilt \url{http://www.gimp.org}.

\section{\TeX{} ja Linux}

Linuxis töötades on suur tõenäosus, et \LaTeX{} on süsteemis juba
installitud või vähemalt olemas installimismeediumil, mida
süsteemi ülesseadmiseks kasutati. Paketihalduri abil tuleks installida
järgmised paketid:

\begin{itemize}
\item texlive -- \TeX i/\LaTeX i põhiinstallatsioon;
\item emacs (koos AUCTeXiga) -- redaktor, mis integreerub tihedalt
\LaTeX iga läbi AUCTeXi lisandpaketi;
\item ghostscript -- \PSi i vaatamisprogramm;
\item xpdf ja acrobat -- PDFi vaatamisprogramm;
\item imagemagick -- vaba programm rasterpiltide konvertimiseks;
\item gimp -- Photoshopiga sarnanev vaba graafikaprogramm;
\item inkscape -- Illustratori / Corel Draw'ga sarnanev
vaba graafikaprogramm.
\end{itemize}

Kui vaja oleks rohkem Windowsiga sarnanevat graafilist
redigeerimiskeskkonda, siis võib vaadata \wi{Texmaker}i poole, vt
jaotist \ref{sec:texmaker}.

Enamikus Linuxi distributsioonides on \TeX i keskkond tükeldatud suureks
arvuks valikulisteks pakettideks, nii et kui pärast esimest
installatsiooni on midagi puudu, siis tuleks kontrollida uuesti.