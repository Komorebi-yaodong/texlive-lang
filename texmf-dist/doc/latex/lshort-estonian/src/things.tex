%%%%%%%%%%%%%%%%%%%%%%%%%%%%%%%%%%%%%%%%%%%%%%%%%%%%%%%%%%%%%%%%%
% Contents: Things you need to know
% $Id: things.tex 536 2015-06-26 06:41:33Z oetiker $
%%%%%%%%%%%%%%%%%%%%%%%%%%%%%%%%%%%%%%%%%%%%%%%%%%%%%%%%%%%%%%%%%

\chapter{Asjad, mida tuleks teada}
\begin{intro}
Peatüki esimeses pooles anname lühikese ülevaate \LaTeXe{} filosoofiast
ja ajaloost. Teises pooles keskendume \LaTeX i-dokumendi
põhistruktuurile. Selle peatüki läbilugemisel peaks tekkima \LaTeX i
töötamisest üldine arusaam, mida läheb vaja raamatu ülejäänud osa
mõistmiseks.
\end{intro}

\section{Nimed}
\subsection{\TeX}

\TeX{} on \index{Knuth, Donald E.}Donald E. Knuthi kirjutatud
arvutiprogramm \cite{texbook} teksti ja valemite ladumiseks. Knuth
alustas trükiladumisprogrammi \TeX{} loomist aastal 1977, et uurida
võimalusi, mida pakkusid tol ajal kirjastamistööstuses levima hakanud
digitaalsed trükiseadmed, iseäranis lootuses pöörata ümber
trüki"-kvaliteedi järkjärguline langus, mida ta nägi omaenda raamatute ja
artiklite peal. \TeX{} sellisel kujul, nagu me teda tänapäeval kasutame,
valmis aastal 1982. Väikesi täiendusi tehti veel 1989. aastal, kui
parandati 8-bitiste märkide ja mitmekeelsuse tuge. \TeX i kuulsus
põhineb sellel, et ta on äärmiselt stabiilne, töötab paljudel
arvutitüüpidel ja on sama hästi kui veavaba. \TeX i versiooninumber
läheneb arvule $\pi$ ja on praegu $3{,}141592653$.

Nime \TeX{} hääldatakse kui "`tehh"', kus "`hh"' hääldub nagu saksa
sõnas \emph{ach}\footnote{Saksa keeles on \emph{ch} hääldamiseks õieti
kaks viisi ja võiks arvata, et \emph{ch} pehme hääldus sõna \emph{Pech}
moodi on sobivam. Sellekohasele küsimusele vastas Knuth saksa Vikipeedia
andmetel: "`Ma ei pahanda, kui inimesed hääldavad sõna \TeX{} nii, nagu
neile meeldib [\,-\,-\,-\,] ning Saksamaal ütlevad paljud pehme "`ch"',
sest X järgneb vokaalile e, mitte kõva "`ch"', mis järgneb vokaalile a.
Vene keeles on sõna \emph{tex} väga tavaline ja hääldub nagu "`tjeh"'.
Kuid ma usun, et kõige sobivam hääldus on kreeka keeles, kus "`ch"' on
kaledam nagu sõnades \emph{ach} ja \emph{loch}."'} või \v{s}oti sõnas
\emph{loch}. Hääldus "`hh"' tuleneb kreeka tähestikust, kus X on täht h
ehk hii. \TeX{} on ka kreeka sõna {\greektext teqnik'h} `tehnika'
esimene silp. ASCII-keskkonnas kirjutatakse \TeX{} kujul \texttt{TeX}.

\subsection{\LaTeX}

\LaTeX{} on makrode pakett, mis võimaldab autoritel oma kirjatööd
vormistada ja trükkida kõige kõrgemal tüpograafilisel kvaliteeditasemel,
rakendades eeldefineeritud professionaalset kujundust. \LaTeX i on
loonud \index{Lamport, Leslie}Leslie Lamport~\cite{manual} ning see
kasutab ladumismootorina küljendusprogrammi \TeX. Praegusel ajal haldab
\LaTeX i \index{Mittelbach, Frank}Frank Mittelbach.

%In 1994 the \LaTeX{} package was updated by the \index{LaTeX3@\LaTeX
%  3}\LaTeX 3 team, led by \index{Mittelbach, Frank}Frank Mittelbach,
%to include some long-requested improvements, and to re\-unify all the
%patched versions which had cropped up since the release of
%\index{LaTeX 2.09@\LaTeX{} 2.09}\LaTeX{} 2.09 some years earlier. To
%distinguish the new version from the old, it is called \index{LaTeX
%2e@\LaTeXe}\LaTeXe. This documentation deals with \LaTeXe. These days you
%might be hard pressed to find the venerable \LaTeX{} 2.09 installed
%anywhere.

\LaTeX{} hääldub kui "`La-tehh"' või "`Lei-tehh"'. ASCIIs
kirjutatakse \LaTeX{} kujul \texttt{LaTeX}. \LaTeXe{}
hääldatakse "`La-tehh kaks e"' ja kirjutatakse \texttt{LaTeX2e}.

%Figure~\ref{components} above % on page \pageref{components}
%shows how \TeX{} and \LaTeXe{} work together. This figure is taken from
%\texttt{wots.tex} by Kees van der Laan.

%\begin{figure}[btp]
%\begin{lined}{0.8\textwidth}
%\begin{center}
%\input{kees.fig}
%\end{center}
%\end{lined}
%\caption{Components of a \TeX{} System.} \label{components}
%\end{figure}

\section{Põhialused}

\subsection{Autor, kujundaja ja laduja}

Käsikirja avaldamiseks annab autor selle kirjastamisfirmale. Firma
kujundaja paneb seejärel paika teose kujunduse (veerulaius, kirjatüübid,
vahed enne ja pärast pealkirja,~\ldots). Kujundaja kirjutab oma juhised
käsikirjale ja annab selle siis ladujale, kes teose vastavalt nendele
juhistele valmis laob.

Inimkujundaja püüab aru saada, mida autor käsikirja kirjutamise ajal
mõtles. Peatükkide pealkirjade, viidete, näidete, valemite jne üle
otsustab ta oma professionaalsete teadmiste ja käsikirja sisu põhjal.

\LaTeX i keskkonnas on kujundaja rollis \LaTeX{} ja ladujaks \TeX. Kuid
et \LaTeX{} on "`ainult"' programm, vajab ta seetõttu rohkem abi. Autor
peab talle andma täiendavat informatsiooni, kirjeldades teose loogilist
struktuuri. See informatsioon kirjutatakse teksti sisse "`\LaTeX i
käskudena"'.

Selline lähenemine erineb üsnagi
\index{visuaalredaktorid}visuaalredaktorite\footnote{\wi{WYSIWYG}
(\emph{What you see is what you get} `Mida näed, seda saad').} omast,
mida järgib enamik tänapäeva tekstitöötlusprogramme, nagu \wi{MS Word}
ja \wi{LibreOffice}. Nendes programmides määrab autor dokumendi
kujunduse interaktiivselt teksti sisestamise käigus. Autor näeb
ekraanil, kuidas teos prindituna paistab.

\LaTeX i puhul autor lõppväljundit tavaliselt teksti kirjutamise ajal ei
näe, kuid lõppväljundit saab vaadata ekraanil pärast faili töötlemist
\LaTeX iga. Siis on võimalik enne printimist teha dokumendis parandusi.

\subsection{Küljenduse kujundus}

Tüpograafiline kujundamine on oskustöö. Oskusteta autorid teevad tihti
tõsiseid vormistamisvigu eeldades, et teose kujundamine on eeskätt
esteetika küsimus: "`Kui dokument näeb ilus välja, siis on ta hästi
kujundatud"'. Aga kuivõrd dokument on mõeldud lugemiseks, mitte seinale
riputamiseks, on loetavus ja arusaadavus palju olulisemad kui ilus
välimus. Näiteks:\enlargethispage{1.3\baselineskip}
\begin{itemize}
\item pealkirjade kirjasuurus ja nummerdus tuleb valida nii, et
  peatükkide ja jaotiste struktuur oleks lugejale selge;
\item reapikkus peab olema piisavalt väike, et mitte lugeja silmi
  kurnata, samas aga piisavalt suur, et lehekülg kenasti täita.
\end{itemize}

Visuaalredaktoritega\index{visuaalredaktorid} loovad kasutajad tihti
esteetiliselt kauneid dokumente, millel struktuur peaaegu puudub või
pole kooskõlaline. \LaTeX{} ennetab selliseid vormistusvigu, sest sunnib
autorit kirjeldama dokumendi \emph{loogilist} struktuuri ja valib selle
järgi ise kõige sobivama kujunduse.

\subsection{Eelised ja puudused}

Kui visuaalredaktorite ja \LaTeX i kasutajad omavahel kokku saavad,
tekib tihti arutelu teemal "`\index{LaTeXi eelised@\LaTeX i
eelised}\LaTeX i eelised tavalise tekstitöötlusprogrammi ees"' või
vastupidi. Kui selline arutelu käivitub, siis on kõige parem hoida
madalat profiili, sest sageli kipuvad need mõttevahetused käest ära
minema. Kuid mõnikord ei ole pääsu \ldots

Seega on siin natuke laskemoona. \LaTeX i peamised eelised tavaliste
tekstitöötlusprogrammide ees on järgmised.
\begin{itemize}
\item Saab kasutada professionaalseid kujundusi, tänu millele näeb
  dokument välja tõesti nagu "`trükitud"'.
\item Valemite vormistamine on mugav.
\item Vaja on selgeks õppida ainult mõned lihtsasti arusaadavad
  käsud, mis määravad ära dokumendi loogilise struktuuri. Peaaegu mitte
  kunagi pole vaja jännata dokumendi tegeliku kujundusega.
\item Lihtsasti saab luua ka keerulisi struktuure nagu allmärkusi,
  ristviiteid, sisukordi ja kirjandusnimestikke.
\item Tüpograafiliste ülesannete jaoks, mida baas-\LaTeX{} ei toeta, on
  olemas vabalt kasutatavad lisapaketid. Näiteks on olemas paketid
  dokumenti \PSi i graafika lisamiseks ja kindlat standardit järgivate
  kirjandusnimestike vormistamiseks. Paljusid neist pakettidest on
  kirjeldatud raamatus \companion.
\item \LaTeX{} soodustab hea struktuuriga tekstide kirjutamist, sest see
  on viis, kuidas \LaTeX{} töötab -- struktuuri määrates.
\item \LaTeXe{} ladumismootor \TeX{} on väga portatiivne ja vaba.
  Seetõttu töötab süsteem peaaegu igal riistvaraplatvormil.
%
% Add examples ...
%
\end{itemize}

\LaTeX il on samuti mõningaid puudusi. Ma arvan, et minul on veidi raske
leida ühtki mõistlikku, kuid olen kindel, et teised suudavad neid välja
tuua sadu \texttt{;-)}
\begin{itemize}
\item \LaTeX{} ei aita eriti inimesi, kes on müünud oma hinge
  \ldots
\item Kuigi valmis dokumendikujundustes saab sättida
  mõningaid parameetreid, on terve uue kujunduse loomine raske ja võtab
  palju aega.\footnote{Kuuldused räägivad, et see on üks
  peamistest küsimustest, mida puudutab valmiv \LaTeX 3
  süsteem.}\index{LaTeX3@\LaTeX 3}
\item Väga raske on kirjutada struktureerimata ja organiseerimata
  tekste.
\item Lubavatest esimestest sammudest hoolimata ei tarvitse sinu
  lemmikhamster kunagi täielikult mõista loogilise märgendamise
  põhimõtet.
\end{itemize}

\section{\LaTeX i sisendfailid}

\LaTeX i sisendiks\index{sisendfail} on tavaline tekstifail.
Unixis/Linuxis on tekstifailid üsna tavalised. Windowsis saab
tekstifaile moodustada Notepadiga. Sisendfail sisaldab nii teose teksti
kui ka käske, mis ütlevad \LaTeX ile, kuidas teksti vormistada. Kui
tegutseda \LaTeX i integreeritud keskkonnas, siis on seal olemas
vahendid tekstivormingus sisendfaili loomiseks \LaTeX i jaoks.

\subsection{Tühikud}

"`Tühisümboleid"' nagu tühikut ja tabulatsioonimärki käsitleb \LaTeX{}
ühtviisi \wi{tühik}una. \emph{Mitu järjestikust} \wi{tühisümbol}it
loetakse \emph{üheks} tühikuks. Rea alguses olevat tühikut üldiselt
ignoreeritakse ja ühte reavahetust loetakse samuti
tühikuks.\index{tühik!rea alguses}

Tühi rida kahe tekstirea vahel märgib lõigu lõppu. \emph{Mitu} tühja
rida on sama mis \emph{üks} tühi rida. Seda illustreerib järgmine näide.
Vasakul on sisendfaili tekst ja paremal vormindatud väljund.

\begin{example}
Pole oluline, kas
sõna järele lisada üks
või mitu       tühikut.

Tühi rida alustab uut
lõiku.
\end{example}

\subsection{Erimärgid}

Järgmised märgid on \wi{reserveeritud sümbolid}, millel on \LaTeX is kas
eritähendus või pole nad kõigis kirjades kättesaadavad. Kui sisestada
need märgid otse teksti, siis tavaliselt neid ei trükita, vaid nad
panevad \LaTeX i tegema asju, mida kasutajal ilmselt polnud plaanis.
\begin{code}
\verb.#  $  %  ^  &  _  {  }  ~  \ . %$
\end{code}
Nagu edaspidi näeme, saab neid märke siiski teksti lisada, kui kirjutada
nende ette \wi{langjoon}:
\begin{example}
\# \$ \% \^{} \& \_ \{ \} \~{}
\textbackslash
\end{example}

Teisi sümboleid ja palju muud saab trükkida erikäskudega
valemire\v{z}iimis või diakriitiliste märkidena.
\index{langjoon}Langjoone märki \textbackslash{} \emph{ei saa} sisestada
teist langjoont selle ette lisades (\verb|\\|); see märgijärjend on
mõeldud rea murdmiseks. Selle asemel võib kasutada käsku
\ci{textbackslash}.

\subsection{\LaTeX i käsud}

\LaTeX i \index{käsk}käsud on tõstutundlikud ning nad esinevad
emmal-kummal järgmisest kahest kujust.

\begin{itemize}
\item Käsk algab \wi{langjoon}ega \verb|\| ja sellele järgneb ainult
tähtedest koosnev nimi. Käsu nime lõpetab tühik, number või ükskõik
milline muu mittetäht.
\item Käsk koosneb langjoonest ja täpselt ühest mittetähest.
\end{itemize}

Paljudel käskudel on olemas ka tärnkuju, mille puhul käsu nime järele on
lisatud tärn.\index{käsk!tärniga}\index{tärniga käsk}

%
% \\* doesn't comply !
%

%
% Can \3 be a valid command ? (jacoboni)
%
\label{whitespace}

\LaTeX{} ignoreerib tühikuid käskude järel. Kui on vaja panna käsu
järele \index{tühik!käsu järel}tühik, tuleb käsu nime järele kirjutada
kas tühi argument \verb|{}| ja tühik või siis spetsiaalne tühja vahe
käsk. Tühi argument \verb|{}| ei lase \LaTeX il pärast käsu nime
tulevaid tühikuid ära süüa.

\begin{example}
Algajatel võib \TeX tühikud käsu
järel vahele jätta. % valesti
Edasijõudnutele \TeX{} sobib, sest
nemad on \TeX perdid ja teavad,
kuidas tühikuid lisada. % õigesti
\end{example}

Mõned käsud nõuavad \index{argument}argumenti, mis tuleb anda
\index{looksulud}looksulgudes \verb|{ }| pärast käsu nime. Mõned käsud
tunnistavad ka \index{valikuline
argument}\index{argument!valikuline}valikulist argumenti, mis lisatakse
käsu nime järele \index{nurksulud}nurksulgudes~\verb|[ ]|.
\begin{code}
\verb|\|\textit{käsk}\verb|[|\textit{valikuline argument}\verb|]{|\textit{argument}\verb|}|
\end{code}
Järgmistes näidetes on kasutatud mõningaid \LaTeX i käske. Nende pärast
pole vaja muretseda, neid selgitatakse hiljem.

\begin{example}
Sa võid mulle \textsl{toetuda}!
\end{example}
\begin{example}
Palun alusta uut rida
just siit!\newline
Tänan!
\end{example}

\subsection{Kommentaarid}
\index{kommentaarid}

Kui \LaTeX{} kohtab sisendfaili töödeldes protsendimärki
\index}\verb|%|, siis ignoreerib ta käsiloleva rea
ülejäänud osa, reavahetust ja kõiki tühisümboleid järgmise rea alguses.
Nii saab sisendfaili kirjutada märkusi, mis trükiversioonis ei ilmu.

\begin{example}
See on % rumal
% Parem: õpetlik <----
näide: kuulilen%
              nuteetun%
    neliluuk
\end{example}

Märgi \verb|%| abil saab ka tükeldada pikki sisendridu, kus tühikud ega
reavahetused pole lubatud.

Pikemate kommentaaride jaoks on olemas keskkond \ei{comment} paketist
\pai{verbatim}. Selle keskkonna kasutamiseks tuleb dokumendi
preambulisse lisada rida \verb|\usepackage{verbatim}|, nagu selgitatakse
edaspidi.

\begin{example}
See on üks teine
\begin{comment}
üsna rumal,
kuid kasulik
\end{comment}
näide kommentaaride
lisamisest dokumenti.
\end{example}
\noindent See ei tööta keerulisemate keskkondade sees, nagu
valemikeskkond.

\section{Sisendfaili struktuur}
\label{sec:structure}\index{sisendfail}
Kui \LaTeXe{} töötleb sisendfaili, siis eeldab ta, et see järgib
teatavat \index{sisendfaili struktuur}struktuuri. Sellest tulenevalt
peab iga sisendfail algama käsuga\cih{documentclass}
\begin{code}
\verb|\documentclass{...}|
\end{code}
See määrab, mis liiki dokumendiga on tegu. Selle järel tulevad käsud,
mis mõjutavad kogu dokumendi välimust, või loevad sisse \wi{pakett}e,
mis lisavad \LaTeX i süsteemile uusi võimalusi. Pakett loetakse sisse
käsuga\cih{usepackage}
\begin{code}
\verb|\usepackage{...}|
\end{code}

Kui kogu seadistustöö on tehtud,\footnote{Käskude \texttt{\bs
documentclass} ja \texttt{\bs begin$\mathtt{\{}$document$\mathtt{\}}$}
vahele jäävat dokumendi osa nimetatakse \emph{\wi{preambul}iks}.} siis
algab dokumendi põhisisu käsuga
\begin{code}
\verb|\begin{document}|
\end{code}
Nüüd võib sisestada teksti vaheldumisi igasuguste kasulike \LaTeX i
käskudega. Dokumendi lõppu pannakse käsk
\begin{code}
\verb|\end{document}|
\end{code}
mis ütleb \LaTeX ile, et töö on läbi. Kõike, mis veel järgneb, \LaTeX{}
ignoreerib.

Joonisel~\ref{mini} on kujutatud minimaalse \LaTeX i faili sisu. Veidi
keerukam \wi{sisendfail} on joonisel~\ref{document}.\index{sisendfail}

\begin{figure}[tp]
\begin{lined}{6cm}
\begin{verbatim}
\documentclass{article}
\begin{document}
Minimaalne on ilus.
\end{document}
\end{verbatim}
\end{lined}
\caption{Minimaalne \LaTeX i fail} \label{mini}
\end{figure}

\begin{figure}[tp]
\begin{lined}{10cm}
\begin{verbatim}
\documentclass[a4paper,11pt]{article}
% seadistused eesti keele jaoks
\usepackage[estonian]{babel}
\usepackage[utf8]{inputenc}
\usepackage[T1]{fontenc}
% määra pealkiri
\author{H.~Partl}
\title{Minimalism}
\begin{document}
% moodustab pealkirja
\maketitle
% lisab sisukorra
\tableofcontents
\section{Mõned huvitavad sõnad}
Nii, siin algab minu armas artikkel.
\section{Nägemiseni}
\ldots{} ja siin ta lõpeb.
\end{document}
\end{verbatim}
\end{lined}
\caption[Realistliku ajakirjaartikli näide]{Realistliku ajakirjaartikli
näide. Kõiki selles näites esinevaid käske tutvustame hiljem.}
\label{document}

\end{figure}

\section{Tüüpiline käsureasessioon}

Arvatavasti on nüüd tekkinud suur tahtmine leheküljel \pageref{mini}
olevat kena väikest \LaTeX i sisendfaili ise järele proovida. Siin on
veidi juhiseid: \LaTeX il endal puuduvad graafiline liides ja peened
vajutatavad nupud. Ta on lihtsalt programm, mis töötleb sisendfaili.
Mõnes \LaTeX i installatsioonis on olemas graafiline kasutajaliides, kus
sisendfaili kompileerimiseks on olemas nupp \LaTeX. Teistes süsteemides
võib olla vaja midagi klaviatuurilt trükkida, seega näitame siin, kuidas
meelitada \LaTeX i kompileerima sisendfaili tekstipõhises süsteemis.
Tähelepanu: see kirjeldus eeldab, et arvutis on olemas töötav \LaTeX i
installatsioon.\footnote{See on nii enamikus hästi hallatud Unixi
süsteemides ning \ldots{} Tõelised Mehed kasutavad Unixit, nii et
\ldots{} \texttt{;-)}}

\begin{enumerate}
\item

Ava/loo \LaTeX i sisendfail. See fail peab olema lihtne ASCII tekst.
Unixis teevad kõik tekstiredaktorid just seda. Windowsis tuleks hoolt
kanda, et fail salvestatakse ASCII või lihtteksti vormingus. Faili nime
valides tuleks jälgida, et laiendiks saaks \eei{tex}.

\item

Ava käsurida või \texttt{cmd} aken, mine kataloogi, kus sisendfail asub,
ja käivita \LaTeX{} sisendfailil.
\begin{lscommand}
\verb+latex foo.tex+
\end{lscommand}
Õnnestumise korral tekib töö tulemusena fail laiendiga \eei{dvi}.
Vajalik võib olla sisendfail \LaTeX ist läbi lasta mitu korda, et
sisukord ja kõik ristviited õigeks muutuksid. Kui sisendfailis on mõni
viga, siis teatab \LaTeX{} sellest ja peatab faili töötlemise. Vajuta
Ctrl+D, et käsureale tagasi saada.

\item

Nüüd võib \wi{DVI-fail}i vaadata. Selleks on mitu võimalust. Faili
vaatamiseks ekraanil on käsk\index{Xdvi}
\begin{lscommand}
\verb+xdvi foo.dvi &+
\end{lscommand}
See töötab ainult Unixis X11-ga. Windowsis võib proovida programmi
\wi{Yap} (\emph{Yet another previewer}).

Võib ka teisendada DVI-faili \PSi iks, mida saab printida või vaadata
\wi{Ghostscript}iga, andes
käsu\index{dvips@\texttt{dvips}}\index{PS-fail}
\begin{lscommand}
\verb+dvips -Pcmz foo.dvi -o foo.ps+
\end{lscommand}

Kui veab, siis võib \LaTeX i süsteem sisaldada isegi tööriista
\index{dvipdf@\texttt{dvipdf}}\texttt{dvipdf}, millega saab DVI-faili
teisendada otse PDF-iks.\index{PDF-fail}
\begin{lscommand}
\verb+dvipdf foo.dvi+
\end{lscommand}

\end{enumerate}

\section{Dokumendi kujundus}

\subsection {Dokumendiklassid}\label{sec:documentclass}

Esimene informatsioon, mida \LaTeX{} sisendfaili töötlemisel vajab, on
loodava dokumendi liik. See määratakse käsuga \ci{documentclass}.
\begin{lscommand}
\ci{documentclass}\verb|[|\emph{suvandid}\verb|]{|\emph{klass}\verb|}|
\end{lscommand}
\noindent Argument \emph{klass} määrab dokumendi liigi.
Tabelis~\ref{documentclasses} on loetletud dokumendiklassid, mida
käesolevas sissejuhatuses mainitakse. \LaTeXe{} distributsiooni
kuulub muidki dokumendiklasse, sealhulgas klassid kirjade ja esitluste
jaoks. Argument \index{suvand}\emph{suvandid} täpsustab dokumendiklassi
käitumist. Suvandid tuleb üksteisest eraldada komadega. Standardsete
dokumendiklasside kõige tavalisemad suvandid on kirjas
tabelis~\ref{options}.

\begin{table}[!bp]
\caption{Dokumendiklassid} \label{documentclasses}
\begin{lined}{\textwidth}
\begin{description}

\item [\normalfont\pai{article}] teadusajakirjade artiklid,
  ettekanded, lühiaruanded, programmidokumentatsioon, infolehed, \ldots{}
  \index{klass!article@\textsf{article}}
\item [\normalfont\pai{proc}] artikliklassil põhinevate dokumentide
  kogumikud (toimetised). \index{klass!proc@\textsf{proc}}
\item [\normalfont\pai{minimal}] nii väike kui saab olla. Määrab
  ainult tekstiploki mõõtmed ja põhikirja tüübi. Kasutatakse peamiselt
  silumise eesmärgil. \index{klass!minimal@\textsf{minimal}}
\item [\normalfont\pai{report}] pikemad mitmepeatükilised aruanded,
  väiksemad raamatud, väitekirjad, \ldots{}
  \index{klass!report@\textsf{report}}
\item [\normalfont\pai{book}] päris raamatud.
  \index{klass!book@\textsf{book}}
\item [\normalfont\pai{slides}] slaidid. Tekstikirjaks on suur
  seriifideta kiri. Slaidide jaoks on eelistatum klass \pai{beamer}.
  \index{klass!slides@\textsf{slides}}\index{klass!beamer@\textsf{beamer}}
\end{description}
\end{lined}
\end{table}

\begin{table}[tp]
\caption{Dokumendiklasside suvandid} \label{options}
\begin{lined}{\textwidth}
% \begin{flushleft}
\begin{description}
\item[\normalfont\texttt{10pt}, \texttt{11pt}, \texttt{12pt}] \quad
  Määrab dokumendi põhikirja suuruse. Kui suvandit pole antud, siis
  võetakse selleks \texttt{10pt}.%
  \index{kirjasuurus}\index{dokumendi kirjasuurus}\index{põhikirja suurus}%
  \index{10pt@\texttt{10pt}}\index{11pt@\texttt{11pt}}\index{12pt@\texttt{12pt}}

\item[\normalfont\texttt{a4paper}, \texttt{letterpaper}, \ldots] \quad
  Määrab paberi formaadi. Vaikeformaat on \texttt{letterpaper}. Veel on
  olemas \texttt{a5paper}, \texttt{b5paper}, \texttt{executivepaper} ja
  \texttt{legalpaper}.
  \index{paberi formaat}%
  \index{formaat!A4}\index{formaat!letter}\index{formaat!A5}%
  \index{formaat!B5}\index{formaat!executive}\index{formaat!legal}%
  \index{a4paper@\texttt{a4paper}}\index{letterpaper@\texttt{letterpaper}}%
  \index{a5paper@\texttt{a5paper}}\index{b5paper@\texttt{b5paper}}%
  \index{executivepaper@\texttt{executivepaper}}\index{legalpaper@\texttt{legalpaper}}

\item[\normalfont\texttt{fleqn}] \quad Rajastab eraldi real olevad
  valemid vasakule, mitte keskele.
  \index{fleqn@\texttt{fleqn}}

\item[\normalfont\texttt{leqno}] \quad Paneb valeminumbrid valemist
  vasakule, mitte paremale.
  \index{leqno@\texttt{leqno}}

\item[\normalfont\texttt{titlepage}, \texttt{notitlepage}] \quad Määrab,
  kas pärast dokumendi \index{tiitel}tiitlit alustada uut lehekülge või
  mitte. Klass \pai{article} vaikimisi ei alusta uut lehekülge,
  klassid \pai{report} ja \pai{book} alustavad. \index{tiitel}
  \index{titlepage@\texttt{titlepage}}\index{notitlepage@\texttt{notitlepage}}

\item[\normalfont\texttt{onecolumn}, \texttt{twocolumn}] \quad Küljendab
  dokumendi teksti \index{üheveeruline trükk}ühes veerus või
  \index{kaheveeruline trükk}kahes veerus.
  \index{onecolumn@\texttt{onecolumn}}\index{twocolumn@\texttt{twocolumn}}

\item[\normalfont\texttt{twoside}, \texttt{oneside}] \quad Määrab, kas
  genereerida kahepoolselt või ühepoolselt trükitavate lehekülgedega
  väljund. Vaikimisi on klassid \pai{article} ja \pai{report}
  \index{ühepoolne trükk}ühepoolsed, klass \pai{book} aga
  \index{kahepoolne trükk}{kahepoolne}. See suvand puudutab ainult
  dokumendi stiili. Suvand \texttt{twoside} \emph{ei anna} kasutatavale
  printerile korraldust printida dokument välja kahepoolselt.
  \index{oneside@\texttt{oneside}}\index{twoside@\texttt{twoside}}

\item[\normalfont\texttt{landscape}] \quad Muudab dokumendi kujunduse
  sobivaks rõhtpaigutuses printimise jaoks.
  \index{landscape@\texttt{landscape}}\index{rõhtpaigutus}

\item[\normalfont\texttt{openright}, \texttt{openany}] \quad Seab peatükid algama
  kas ainult parempoolsel leheküljel või järgmisel vabal leheküljel. Ei
  tööta klassiga \pai{article}, mis peatükke ei tunne. Vaikimisi
  algavad peatükid klassis \pai{report} järgmisel vabal leheküljel ja
  klassis \pai{book} parempoolsel leheküljel.
  \index{openright@\texttt{openright}}\index{openany@\texttt{openany}}

\end{description}
% \end{flushleft}
\end{lined}
\end{table}

Näiteks võib \LaTeX i sisendfail alata reaga
\begin{code}
\ci{documentclass}\verb|[11pt,twoside,a4paper]{article}|
\end{code}
Sellega vormistab \LaTeX{} dokumendi \emph{artiklina} põhikirja
suurusega \emph{11 punkti} küljenduses, mis sobib
\emph{kahepoolseks} printimiseks \emph{A4-lehele}.

\subsection{Paketid}
\index{pakett}Dokumenti kirjutades võib mõnes valdkonnas ilmneda
probleeme, mida baas-\LaTeX{} lahendada ei suuda. Kui on vaja dokumenti
lisada graafikat, moodustada värvilist teksti või lugeda failist sisse
programmi lähtekoodi, siis tuleb \LaTeX i võimeid laiendada. Selliseid
laiendusi nimetatakse pakettideks. Pakett võetakse kasutusele käsuga
\begin{lscommand}
\ci{usepackage}\verb|[|\emph{suvandid}\verb|]{|\emph{pakett}\verb|}|
\end{lscommand}
\noindent kus \emph{pakett} on paketi nimi ja \emph{suvandid} nimekiri
võtmesõnadest, mis käivitavad paketis spetsiaalseid funktsioone. Käsk
\ci{usepackage} pannakse dokumendi preambulisse. Täpsemalt vaadeldi seda
jaotises \ref{sec:structure}.

Mõned paketid tulevad kaasa \LaTeXe{} baasdistributsiooniga (vt
tabelit~\ref{packages}), teised on saadaval eraldi. Oma arvutisse
installitud pakettide kohta peaks rohkem infot andma \guide. Põhiline
infoallikas \LaTeX i pakettide kohta on \companion, mis sisaldab sadade
pakettide kirjeldusi, samuti juhiseid, kuidas \LaTeXe{} jaoks ise
laiendusi kirjutada.

Kaasaegses \TeX i distributsioonis on suur hulk pakette juba
eelinstallitud. Unixis saab paketi dokumentatsiooni kätte käsuga
\index{texdoc@\texttt{texdoc}}\texttt{texdoc}.

\begin{table}[tp]
\caption{Mõned \LaTeX iga kaasatulevad paketid} \label{packages}
\begin{lined}{\textwidth}
\begin{description}
\item[\normalfont\pai{doc}] Võimaldab \LaTeX is koostatud programme
  dokumenteerida. Kirjeldatud failis \texttt{doc.dtx}\footnote{See
  fail peaks olema süsteemis installitud ning \wi{DVI-fail}i
  peaks saama genereerida käsuga \texttt{latex doc.dtx} ükskõik millises
  kataloogis, kus kasutajal on kirjutamisõigus. Sama kehtib kõigi
  teiste selles tabelis nimetatud failide kohta.} ja raamatus
  \companion.

\item[\normalfont\pai{exscale}] Teeb kättesaadavaks laiendatud
  valemikirjade skaleeritud variandid. Kirjeldatud failis
  \texttt{ltexscale.dtx}.

\item[\normalfont\pai{fontenc}] Määrab, millist \wi{kirjakodeering}ut
  \LaTeX{} peaks kasutama. Kirjeldatud failis \texttt{ltoutenc.dtx}.

\item[\normalfont\pai{ifthen}] Teeb kättesaadavaks käsud kujul
  "`kui \ldots{} siis \ldots{} muidu \ldots"'. Kirjeldatud failis
  \texttt{ifthen.dtx} ja raamatus \companion.

\item[\normalfont\pai{latexsym}] \LaTeX i sümbolikirja kasutamiseks
  tuleks sisse lugeda pakett \pai{latexsym}. Kirjeldatud
  failis \texttt{latexsym.dtx} ja raamatus \companion.

\item[\normalfont\pai{makeidx}] Muudab kättesaadavaks aineregistri
  moodustamise käsud. Kirjeldatud jaotises~\ref{sec:indexing} ja
  raamatus \companion.

\item[\normalfont\pai{syntonly}] Töötleb dokumenti ilma seda ladumata.

\item[\normalfont\pai{inputenc}] Lubab määrata sisendkodeeringut,
  nagu ASCII, ISO Latin-1, ISO Latin-2, 437/850 IBM koodileheküljed,
  Apple Macintosh, Next, ANSI-Windows või kasutaja defineeritud
  kodeering. Kirjeldatud failis \texttt{inputenc.dtx}.
\end{description}
\end{lined}
\end{table}

\subsection{Leheküljestiilid}

\LaTeX{} toetab kolme eeldefineeritud \wi{päis}e/\wi{jalus}e
kombinatsiooni ehk nn \wi{leheküljestiil}i. Käsu
\begin{lscommand}
\ci{pagestyle}\verb|{|\emph{stiil}\verb|}|
\end{lscommand}
\noindent argument \emph{stiil} määrab, millist stiili kasutada.
Eeldefineeritud stiilid on loetletud tabelis~\ref{pagestyle}.

\begin{table}[tp]
\caption{\LaTeX i eeldefineeritud leheküljestiilid} \label{pagestyle}
\begin{lined}{\textwidth}
\begin{description}

\item[\normalfont\texttt{plain}] trükib lehekülje alaäärde, jaluse
keskele, leheküljenumbrid. Vaikestiil.
\index{leheküljestiil!plain@\texttt{plain}}\index{plain@\texttt{plain}}

\item[\normalfont\texttt{headings}] trükib iga lehekülje päisesse
jooksva peatüki pealkirja ja leheküljenumbri, jalus jääb tühjaks. (See
on käesolevas dokumendis kasutatav stiil.)
\index{leheküljestiil!headings@\texttt{headings}}\index{headings@\texttt{headings}}

\item[\normalfont\texttt{empty}] jätab nii päise kui ka jaluse tühjaks.
\index{leheküljestiil!empty@\texttt{empty}}\index{empty@\texttt{empty}}

\end{description}
\end{lined}
\end{table}

Jooksva lehekülje stiili on võimalik muuta käsuga
\begin{lscommand}
\ci{thispagestyle}\verb|{|\emph{stiil}\verb|}|
\end{lscommand}
Kirjelduse, kuidas luua oma päiseid ja jaluseid, leiab raamatust
\companion{} ning jaotisest~\ref{sec:fancy}
leheküljel~\pageref{sec:fancy}.
%
% Pointer to the Fancy headings Package description !
%

\section{Esineda võivad failid}

\LaTeX iga töötades võib kasutaja kiiresti leida end eri
\index{laiendid}laienditega failide labürindist ilma juhtlõngata.
Järgmises loendis on kirjas mitmesugused \wi{failitüübid}, mis \TeX iga
töötades võivad ette tulla. See tabel ei ole kindlasti täielik laiendite
nimekiri, kuid kui puudu on midagi olulist, siis võiks mulle teada anda.

\begin{description}

\item[\eei{tex}] \LaTeX i või \TeX i sisendfail. Saab kompileerida
käsuga \texttt{latex}.
\item[\eei{sty}] \LaTeX i makropakett. Saab käsuga \ci{usepackage}
\LaTeX i dokumenti sisse lugeda.
\item[\eei{dtx}] Dokumenteeritud \TeX{}. See on \LaTeX i stiilifailide
peamine distributsioonivorming. Kui DTX-fail kompileerida, siis on
tulemuseks DTX-failis sisalduva \LaTeX i paketi dokumenteeritud makrokood.
\item[\eei{ins}] Vastavas DTX-failis sisalduvate failide
installija. Laadides \LaTeX i paketi võrgust alla, saab tavaliselt
DTX-faili ja INS-faili. Käivitades \LaTeX i INS-failil, saab DTX-faili
lahti pakkida.
\item[\eei{cls}] Klassifail, mis määrab, kuidas dokument välja näeb.
Klassifail valitakse käsuga \ci{documentclass}.
\item[\eei{fd}] Kirjadefinitsioonide fail, mis tutvustab \LaTeX ile uusi
kirju.
\end{description}
Järgmised failid genereerib \LaTeX{} sisendfaili töötlemisel.

\begin{description}
\item[\eei{dvi}] Seadmest sõltumatu fail (\emph{Device Independent
File}). See on \LaTeX i kompileerimistöö põhitulemus. Faili sisu saab
vaadata \wi{DVI-fail}ide vaatamisprogrammiga või saata printerile
programmiga \index{dvips@\texttt{dvips}}\texttt{dvips} või muu sarnase
programmiga.
\item[\eei{log}] Sisaldab detailset aruannet sellest, mis viimase
kompileerimise jooksul juhtus.
\item[\eei{toc}] Säilitab kõigi jaotiste pealkirju. Loetakse sisse
järgmise kompileerimise käigus, kui moodustatakse sisukord.
\item[\eei{lof}] Nagu TOC, aga jooniste loetelu jaoks.
\item[\eei{lot}] Sama tabelite loetelu jaoks.
\item[\eei{aux}] Veel üks fail, mis kannab informatsiooni ühelt
kompileerimiskorralt järgmisele. Muu hulgas säilitatakse AUX-failis
ristviidetega seotud informatsiooni.
\item[\eei{idx}] Kui dokument sisaldab aineregistrit, siis salvestab
\LaTeX{} kõik registrisse minevad sõnad sellesse faili. Seda faili
tuleb töödelda programmiga \wi{MakeIndex}. Aineregistri kohta leiab
rohkem infot jaotisest \ref{sec:indexing} leheküljel
\pageref{sec:indexing}.
\item[\eei{ind}] Töödeldud IDX-fail, valmis järgmises
kompileerimistsüklis dokumenti sisselugemiseks.
\item[\eei{ilg}] Logifail, mis ütleb, mida \wi{MakeIndex} tegi.
\end{description}


% Package Info pointer
%
%



%
% Add Info on page-numbering, ...
% \pagenumbering

\section{Suured projektid}

Suuri dokumente luues võib tekkida soov jaotada \wi{sisendfail} mitmeks
osaks. Selleks on \LaTeX is kaks käsku.
\begin{lscommand}
\ci{include}\verb|{|\emph{failinimi}\verb|}|
\end{lscommand}
\noindent Selle käsuga saab lisada faili \emph{failinimi}\verb|.tex|
sisu käsiloleva dokumendi sisse. Enne faili \emph{failinimi}\verb|.tex|
materjali töötlemist alustab \LaTeX{} uut lehekülge.

Teist käsku saab kasutada preambulis ning selle toimel loeb \LaTeX i
sisse ainult mõned käskude \verb|\include| argumentideks olevad failid.
\begin{lscommand}
\ci{includeonly}\verb|{|\emph{failinimi}\verb|,|\emph{failinimi}%
\verb|,|\ldots\verb|}|
\end{lscommand}
\noindent Pärast selle käsu täitmist dokumendi preambulis täidetakse
\ci{include}-kä\-sud ainult nende failinimede puhul, mis on loetletud
käsu \ci{includeonly} argumendis.

Käsk \ci{include} alustab sisseloetava teksti ladumist uuelt
leheküljelt. See sobib hästi käskude \ci{includeonly} jaoks, sest
leheküljepiirid ei muutu, isegi kui mõned sisseloetavad failid välja
jäävad. Kuid mõnikord pole see soovitav. Sel juhul võib kasutada käsku
\begin{lscommand}
\ci{input}\verb|{|\emph{failinimi}\verb|}|
\end{lscommand}
\noindent See käsk lihtsalt loeb antud faili sisse. Ei mingeid kirjusid
kostüüme ega kuljuseid.

Paketi \pai{syntonly} abil saab lasta \LaTeX il kiiresti dokumendi üle
kontrollida: \LaTeX{} vaatab dokumendi läbi, kontrollib ainult süntaksit
ja käskude kasutamise korrektsust, aga ei moodusta (DVI) väljundit. Kuna
selles re\v{z}iimis töötab \LaTeX{} kiiremini, võib see hoida kokku
väärtuslikku aega. Kasutamine on väga lihtne:
\begin{code}
\begin{verbatim}
\usepackage{syntonly}
\syntaxonly
\end{verbatim}
\end{code}
Soovides saada tegelikke lehekülgi, tuleb lihtsalt teine rida välja
kommenteerida (lisades selle ette protsendimärgi).


%

% Local Variables:
% TeX-master: "lshort2e"
% mode: latex
% mode: flyspell
% End:
