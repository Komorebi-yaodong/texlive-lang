\chapter{论文几个主要部分的写法}

研究生学位论文的书写,除表达形式上需要遵循一定的格式要求外,内容上也要符合一定的要求。\\
通常研究生学位论文只能有一个主题(不能是几块工作拼凑在一起),该主题应针对某学科领域中的一个具体问题展开深入、系统的研究,并得出有价值的研究结论。学位论文的研究主题切忌过大,例如,“中国国有企业改制问题研究”这样的研究主题过大,因为“国企改制”涉及的问题范围太广,很难在一本研究生学位论文中完全研究透彻。

\section{论文的语言及表述}

除留学生外,学位论文一律须用汉语书写。学位论文应当用规范汉字进行撰写,除古汉语研究中涉及的古文字和参考文献中引用的外文文献之外,均采用简体汉字撰写。\\
留学生一般应以中文或英文书写学位论文,格式要求同上。用英文书写的学位论文,应有不少于6000字的详细中文摘要,英文摘要500-800单词。论文须用中文封面。\\
研究生学位论文是学术作品,因此其表述要严谨简明,重点突出,专业常识应简写或不写,做到层次分明、数据可靠、文字凝练、说明透彻、推理严谨、立论正确,避免使用文学性质的或带感情色彩的非学术性语言。\\
论文中如出现一个非通用性的新名词、新术语或新概念,需随即解释清楚。

\section{论文题目的写法}
论文题目应简明扼要地反映论文工作的主要内容,切忌笼统。由于别人要通过你论文题目中的关键词来检索你的论文,所以用语准确是非常重要的。论文题目应该是对研究对象的准确具体的描述,这种描述一般要在一定程度上体现研究结论,因此,我们的论文题目不仅应告诉读者这本论文研究了什么问题,更要告诉读者这个研究得出的结论。例如:“在事实与虚构之间:梅乐、卡彭特、沃尔夫的新闻观”就比“三个美国作家的新闻观研究”更专业更准确。 

\section{摘要的写法}

论文的摘要,是对论文研究内容的高度概括,其他人会根据摘要检索一篇研究生学位论文,因此摘要应包括:对问题及研究目的的描述、对使用的方法和研究过程进行的简要介绍、对研究结论的简要概括等内容。摘要应具有独立性、自明性,应是一篇简短但意义完整的文章。 \\
通过阅读论文摘要,读者应该能够对论文的研究方法及结论有一个整体性的了解,因此摘要的写法应力求精确简明。论文摘要切忌写成全文的提纲,尤其要避免“第1章……;第2章……;……”这样的陈述方式。 

\section{引言的写法}

一篇学位论文的引言大致包含如下几个部分:1、问题的提出;2、选题背景及意义;3、文献综述;4、研究方法;5、论文结构安排。\\
问题的提出:要清晰地阐述所要研究的问题“是什么”。选题时切记要有“问题意识”,不要选不是问题的问题来研究。\\
选题背景及意义:论述清楚为什么选择这个题目来研究,即阐述该研究对学科发展的贡献、对国计民生的理论与现实意义等。\\
文献综述:对本研究主题范围内的文献进行详尽的综合述评,“述”的同时一定要有“评”,指出现有研究状态,仍存在哪些尚待解决的问题,讲出自己的研究有哪些探索性内容。\\
研究方法:讲清论文所使用的科学研究方法。\\
论文结构安排:介绍本论文的写作结构安排。

\section{“第2章,第3章,……,结论前的一章”的写法}

本部分是论文作者的研究内容,不能将他人研究成果不加区分地掺和进来。已经在引言的文献综述部分讲过的内容,这里不需要再重复。\\
各章之间要存在有机联系,符合逻辑顺序。

\section{结论的写法}

结论是对论文主要研究结果、论点的提炼与概括,应准确、简明,完整,有条理,使人看后就能全面了解论文的意义、目的和工作内容。主要阐述自己的创造性工作及所取得的研究成果在本学术领域中的地位、作用和意义。同时,要严格区分自己取得的成果与导师及他人的科研工作成果。\\
在评价自己的研究工作成果时,要实事求是,除非有足够的证据表明自己的研究是“首次”的,“领先”的,“填补空白”的,否则应避免使用这些或类似词语。