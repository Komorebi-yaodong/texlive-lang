\documentclass{csbulletin}
\selectlanguage{english}
\usepackage[utf8]{inputenc}
\usepackage[all]{nowidow}
\usepackage{csquotes}
\usepackage[
  backend=biber,
  style=iso-numeric,
  sortlocale=en,
  autolang=other,
  bibencoding=UTF8,
  mincitenames=2,
  maxcitenames=2,
]{biblatex}
\addbibresource{example.bib}
\usepackage[
  implicit=false,
  hidelinks,
]{hyperref}
\begin{document}

\title{English Title}
\EnglishTitle{English Title}
\author{Name Surname}
\podpis{Name Surname, e-mail address}
\maketitle

\begin{abstract}
English abstract.
\keywords: first English keyword, second English keyword, third English …
\end{abstract}

\section{Section Title}
\TeX{} is a digital typesetting system. The book \citetitle{knuth-ttp}~\cite{knuth-ttp} is a detailed account of \TeX's source code, whereas \textcite{knuth-tb} focuses more on the end user.

\printbibliography

\section*{Czech Title}
\begin{otherlanguage}{czech}
Czech abstract.
\end{otherlanguage}
\klicovaslova: first Czech keyword, second Czech keyword, third Czech …
\end{document}
