%% $Id$
\documentclass{csbulletin}
\DeclareRobustCommand\version{\unskip~2022/12/12}
\let\pkg\textsc

\usepackage[pdftitle=LaTeX\ Class\ for\ CSTUG\ Bulletin, pdfauthor={Zdenek\ Wagner, Vit\ Novotny},
            pdfkeywords=CSTUG\ Bulletin,bookmarks=false]{hyperref}

% section numbering
\setcounter{secnumdepth}{2}

\begin{document}
\title{Styl pro Zpravodaj Československého sdružení uživatelů \TeX{}u,\version}
\EnglishTitle{\LaTeX\ Class for \CSTUG\ Bulletin,\version}
\author{Zdeněk Wagner, Vít Novotný}
\podpis{Zdeněk Wagner, Vít Novotný\\\url{http://bulletin.cstug.cz}}
\maketitle

\begin{abstract}
Tento dokument popisuje třídu pro psaní článků do Zpravodaje Československého sdružení uživatelů
\TeX{}u. Dokument je určen zejména pro české a slovenské autory, proto je psán česky. Je psán 
ve stylu pro Zpravodaj, autoři se tedy mohou podívat přímo do zdrojového kódu tohoto manuálu a
napsat svůj článek obdobně.
\end{abstract}
\klicovaslova: \LaTeX, styl, dokumentová třída, Zpravodaj \CSTUG, dokumentace

\section{English introduction}
\begin{otherlanguage}{english}
This manual describes the \LaTeX\ class used for articles for the Bulletin of the Czechoslovak
\TeX\ Users Group (\cstug). Since it is asumed that its audience will will be mainly the Czech and
Slovak users, the manual is written in Czech. We do publish articles in other languages
occasionally. In such a case you can prepare the article just in the standard \pkg{article} class.
The \pkg{csbulletin} class is modelled after it with just a few modifications. You can also look
into the source text of this manual. It is written intentionally using the \pkg{csbulletin} class.
You can therefore write your article according to it.
\end{otherlanguage}

\section{Úvod}
Od roku 2008 je styl pro Zpravodaj Československého sdružení uživatelů \TeX{}u implementován jako
třída (class). Makra jsou psána tak, aby hlavní soubor dokázal načíst celý článek včetně příkazů
\cmd{documentclass} a případných \cmd{usepackage}. Třída \pkg{csbulletin} je odvozena ze standardní
třídy \pkg{article}. Je přidáno pouze několik maker. Tento manuál tedy lze spíše chápat jako
instrukce pro autory.

\section{Použité balíčky}
Třída \pkg{csbulletin} nějaké balíčky načítá automaticky. Jejich seznam je uveden v~následujících
podsekcích.

\subsection{Povinné balíčky}
Balíčky, uvedené v~následujícím seznamu, se načítají vždy. Musíte je tedy mít instalovány.

\begin{itemize}
\item Balíček \pkg{csbulacronym} definuje běžné akronymy robustním způsobem pomocí \cmd{DeclareRobustCommand} a
některé speciální akronymy, jako je např. makro \cmd{cstug} pro \cstug. Soubor je dodáván společně s~třídou
\pkg{csbulletin}. Definice se používají i mimo Zpravodaj, proto jsou v~samostatném souboru.

\item \pkg{fontenc} s~parametrem T1

\item \pkg{lmodern}

\item \pkg{ifpdf}

\item \pkg{mflogo}

\item \pkg{babel}, viz kapitola~\ref{babel}.
\end{itemize}

\subsection{Nepovinné balíčky}
Některé balíčky jsou využívány v~redakci při sazbě finální verze Zpravodaje, ale pro pracovní verzi
článku nejsou nezbytné. Jejich instalace na počítači autorů tedy není vyžadována. Třída jejich
přítomnost detekuje a načte je, pokud jsou v~počítači přítomny.

\begin{itemize}
\item \pkg{microtype}

\item \pkg{array}

\item \pkg{fancyvrb}

\item \pkg{verbatim}
\end{itemize}

\subsection{Babel}\label{babel}
Třída \pkg{csbulletin} vyžaduje \pkg{babel} s~moduly pro češtinu a slovenštinu ve verzi~3.2 nebo
novější. Tato verze je distribuována na \TeX Live~2008, ale dělení slov funguje pouze
v~kódování~T1. Pokud máte starší verzi těchto modulů, bude \LaTeX\ hlásit chybu na dvou příkazech
\cmd{languageattribute}. Chyby odstraníte použitím parametru \texttt{oldcsbabel} v~hranatých
závorkách v~příkazu \cmd{documentclass}.

\section{Začátek souboru}
Prvním příkazem souboru, kde lze (viz sekce~\ref{babel}) použít nepovinný parametr, je:

\medskip
\verb;\documentclass{csbulletin};
\medskip

Pak můžete načíst nezbytné balíčky, definovat vlastní makra a prostředí. Všechny definice budou
lokální pro váš článek, nemusíte se proto obávat kolizí s~články jiných autorů.

Implicitním jazykem pro Zpravodaj je čeština. Pokud je článek psán v~jiném jazyce, vložte hned za
\cmd{begin\{document\}} přepínač \cmd{selectlanguage} se správným jazykem.

Název článku zapište do makra \cmd{title}, jména autorů do makra \cmd{author} a anglický název do
makra \cmd{EnglishTitle}. Nepovinně můžete použít makro \cmd{podpis}, jehož parametr se objeví
kurzívou na konci článku. Poté použijte makro \cmd{maketitle}.

Abstrakt článku v~jazyce článku zapište v~prostředí \texttt{abstract}. Ve výjimečných případech lze
abstrakt vynechat. Za prostředí \texttt{abstract} volitelně uveďte jedno a více klíčových slov
oddělených čárkou a předznamenaných příkazem \cmd{klicovaslova:}, \cmd{klucoveslova:} nebo
\cmd{keywords:} podle jazyka článku.

Na konci článku uveďte anglický souhrn v~prostředí \texttt{summary}. Anglický název bude vzat
z~příkazu \cmd{EnglishTitle}, přepnutí jazyka se též provede automaticky. Anglický souhrn nemusí být
přesným překladem abstraktu. Na závěr prostředí \texttt{summary} volitelně uveďte jedno a více
anglických klíčových slov oddělených čárkou a předznamenaných příkazem \cmd{keywords:}.

Redakce se postará o~jazykovou korekturu anglického souhrnu. V~případě, že jej autor nedodá,
postará se o~překlad redakce.

\section{Číslování kapitol}
Implicitně nejsou kapitoly ve Zpravodaji číslovány. Složitější články jsou však bez číslování
nepřehledné. Číslování zapnete vložením kladné hodnoty do čítače \texttt{secnumdepth}. V~tomto
manuálu je před \verb;\begin{document}; použit příkaz \verb;\setcounter{secnumdepth}{2};.

\section{Tabulky a obrázky}
Všechny
tabulky a obrázky musí být v~plovoucích prostředích. Musí mít název v~makru \cmd{caption} a
případně symbolický název definovaný v~makru \cmd{label}.

\section{Chyby}
Třída \pkg{csbulletin} nemá žádné známé zjevné chyby.
V~každém případě je třeba si uvědomit, že třída samotná pracuje na hranicích možností \LaTeX{}u.
Použití různých balíčků může vést k~dalším problémům. Redakce se bude snažit o~vyřešení všech
nahlášených potíží.

\section{Licence}
Balíček může být používán a šířen podle \LaTeX\ Project Public License verze~1.3c nebo novější, jejíž
text najdete v~souboru \texttt{LICENSE.txt} v~adresáři \texttt{doc}, nebo na
\url{http://www.latex-project.org/lppl.txt}

\begin{summary}
The package provides the class for articles for the \cstug\ Bulletin (Zpravodaj Československého
sdružení uživatelů \TeX u). You can see the structure of a document by looking to the source file
of this manual. The package can be used and distributed according to the \LaTeX\ Project Public
License.
  \keywords: \LaTeX, style, document class, \CSTUG{} Bulletin, documentation
\end{summary}

\end{document}
