%# -*- coding: utf-8 -*-
\ifdefined\epTeXinputencoding % defined in e-pTeX (> TL2016)
  \epTeXinputencoding utf8    % ensure utf-8 encoding for platex
\fi

%
%  日本語論文用 後註生成マクロ for pLaTeX2e 付属ドキュメント
%
%  endnotesj.sty v3.0 系列は、熊本学園大学経済学部の
%  小川弘和さんによる endnotesj.sty 2003/02/12 v2.1 に対し、
%  許可を得て山下弘展がいくつかの修正・拡張を施したものです。
%
%  この文書は、原著者である小川さんの文書
%    aboutendnotesj.sjis
%  を山下が LaTeX ソース化し、適宜記述を追加したものです。
%
%  元のファイル (v2.1) は、2016 年 9 月 8 日に
%    JIS X 0212 for pTeX
%      http://www2.kumagaku.ac.jp/teacher/herogw/
%  から取得しました。著作表記は、以下のとおりでした。
%
%  =============================================================
%  改造者:小川弘和(熊本学園大学経済学部)
%    mail:dokuroishi@mac.com,herogw@kumagaku.ac.jp
%     web:http://homepage.mac.com/dokuroryokan/index.html
%         http://www2.kumagaku.ac.jp/teacher/herogw/index.html
%
%  based on "endnotes.sty" written by John Lavagnino
%                                  lav@brandeis.bitnet, 9/23/88
%   Department of English and American Literature,
%                                          Brandeis University
%  =============================================================
%

\documentclass[a4paper]{jsarticle}
\usepackage{doc}
\usepackage{endnotesj}
\GetFileInfo{endnotesj.sty}
\def\Lpack#1{\textsf{#1}}
\def\Lopt#1{\texttt{#1}}
\title{\Lpack{\filename} \fileversion\\
       日本語論文用 後註生成マクロ}
\author{小川 弘和(熊本学園大学経済学部)\\
        modified by 山下 弘展}
\date{\filedate}
\begin{document}

\maketitle

日本史・国文学関係論文作成に必要な、縦型の“(連数字)”形式
註番号および、複数行にわたる註部分テキストの2行目以降を、先頭行
テキスト開始位置に揃える表記を可能とした、後註生成マクロです。
縦組論文での使用を意図して作成していますが、横組論文でもその
まま使用可能です。

このバージョン(\fileversion)は、旧版との互換性を極力維持しつつ、
内部マクロを本家(\Lpack{endnotes.sty})最新版に合わせて更新し、さらに
機能拡張を施したものです。旧版同様にp\LaTeX/up\LaTeX{}で動作する
ことに加え、Lua\LaTeX\ (Lua\TeX-ja)にも対応しています。
最新版はGitHubリポジトリ
\begin{verbatim}
  https://github.com/aminophen/endnotesj
\end{verbatim}
で管理しています。

\section{コマンド・マニュアル}

コマンド自体はオリジナルである\Lpack{endnotes.sty}と同じです。

\begin{itemize}
\item \verb+\endnote{註}+\\
  本文中に挿入することで自動的に、\verb+{}+に挟まれたテキストを、
  連番を付した後註として扱います。つまり、このコマンドの埋め込
  まれた位置の本文肩に註番号を生成するとともに、後に解説する
  \verb+\theendnotes+コマンドによって、\verb+{}+に挟まれたテキ
  ストの冒頭に同様の番号を付したうえで、後註として列挙するわけ
  です。
\item \verb+\endnote[数字]{註}+\\
  \verb+[数字]+部分に任意の数字を埋め込むことで、数字で指定した
  数を註番号とする後註を生成します。
  このコマンドによって生成された註は、上記\verb+\endnote{註}+に
  よって生成される註とは別グループとして扱われるため、そちら
  の番号に影響を与えることはありません。
  よって表記を改良すれば補注用として利用可能ですが、現状では
  通常の註と同スタイルのため紛らわしいので、使用は勧められません。
\item \verb+\endnotemark[数字]+\\
  実際には後註を生成せぬままで、本文肩に任意の註番号を生成します。
  また、\verb+[数字]+オプションを用いた場合、以降の註番号は任意
  の番号にスキップしたうえで生成されていくことになります。
\item \verb+\endnotetext[数字]{註}+\\
  番号を表立って表記せぬ後註を生成します。このコマンドの場合
  でも、内部的には註番号はカウントされています。
  なお[数字]オプションの機能は\verb+\endnotemark+と同様です。
\item \verb+\theendnotes+\\
  本文中に挿入することでその位置に、それまで\verb+\endnote+等の
  コマンドによって生成された後註を列挙表示します。
  なお、\verb+\setcounter{endnote}{0}+コマンドを用いると、その
  後の註番号が再び(1)より開始されますので、本コマンドとの組合せ
  によって、容易に章・節ごとの後註が作成できます。
\end{itemize}

\section{パッケージ・オプション}

次に、パッケージのオプションを説明します。
これらは\Lpack{endnotesj.sty}独自のものです。

\subsection{注釈印の書式}

プリアンブルで
\begin{verbatim}
  \usepackage{endnotesj}
\end{verbatim}
とすることで、縦組用の“(連数字)”型注(『日本史研究』等の形式。
本文中では、注挿入指定位置直前の文字の直上に注No.を配置する。)が
使用可能に、
\begin{verbatim}
  \usepackage[yoko]{endnotesj}
\end{verbatim}
とすることで、横組用の“(数字)”型注(本文中では、注挿入指定
位置直前の文字の直後上方に、ほぼ1/4倍角で注No.を配置する。)が
使用可能に、
\begin{verbatim}
  \usepackage[single]{endnotesj}
\end{verbatim}
とすることで、横組用の“数字)”型注(『歴史学研究』等の形式。
本文中では、注挿入指定位置直前の文字の直後上方に、ほぼ1/4倍角で
注No.を配置する。)が使用可能になります。

\subsection{\Lpack{otf}パッケージを用いた注釈印の書式}

\Lpack{otf}パッケージをインストールしてある環境であれば、
注番号の数字に詰数字を用い、より奇麗に表示可能です。
この機能を使うには
\begin{verbatim}
  \usepackage[otf]{endnotesj}
\end{verbatim}
と指定します。\Lopt{otf}はv3.0で新設されたオプションで、
旧バージョンv2.1にあった\Lopt{utf}も同じ意味になります。

このオプションを指定すると、
\Lpack{otf.sty}および\Lpack{ajmacros.sty}で定義されている
\verb+\UTF+・\verb+\ajTumesuji+コマンドに動作が依存します。
そのため、これらのパッケージを自動で読み込みます
\footnote{古い環境で\Lpack{utf.sty}および\Lpack{utfmacro.sty}しか
利用できない場合は、これらにフォールバックします。}。このため、
別途これらのパッケージを\verb+\usepackage+する必要はありませんが、
\Lpack{otf.sty}の各種オプション(\Lopt{expert}など)を用いたい場合は
\Lpack{endnotesj.sty}より\emph{前}に指定してください。

\subsection{後注列挙部分のタイトル}

ここまでの節で紹介したオプション以外が\verb+\usepackage+の
オプションに指定された場合は、それを後注列挙部分のタイトルと
して用います。たとえば、
\begin{verbatim}
  \usepackage[注]{endnotesj}
\end{verbatim}
オプションで、文末の後注列挙部分の先頭に「注」、
\begin{verbatim}
  \usepackage[註]{endnotesj}
\end{verbatim}
オプションで「註」と表記されます(v2.1以前は「注」と「註」のみ
サポートしていましたが、v3.0以降は「注釈」や「後註」なども
自由に指定できます)。オプション無指定の場合には、何も表記せぬ
まま、注が列挙されていきます。

\section{行数・桁数指定マクロ}

\Lpack{endnotesj.sty}には、
『\LaTeX{}スタイル・マクロ ポケットリファレンス』(技術評論社)、
いわゆるポケリで紹介されている行数・桁数指定マクロも組み込んで
あり、使用することが可能です。用紙サイズにあわせて自動的に字間
配置を調整する\TeX{}には本来は、行数・桁数指定は馴染まないので
すが、投稿規定上、行数・文字数を固定する必要がある場合に用いて
ください。

基本的な記述法は以下の通りです。
\begin{verbatim}
  \kcharparline{30}
  \begin{document}
  \linesparpage{20}
\end{verbatim}

\verb+\kcharparline{30}+が、用紙縦方向の文字数指定。
この場合、30文字に指定しています。
なお、このコマンドはプリアンプルで指定します。

\verb+\linesparpage{20}+が、用紙横方向の文字数指定。
ここでは、20文字に指定しています。こちらは本文で指定するか、
\verb+\AtBeginDocument{}+に入れて使用します。

\section{旧版(v2.1以前)との違い}

% ---- 細かい挙動なので削除 ----
% v2.1以前とv3.0以降の唯一の違いとして、v2.1では「注」と「註」を
% 同時に指定した場合には「注」が優先されましたが、v3.0以降は
% 「複数指定されたうちの最後の一つ」が優先されます。同時に指定
% するという使い方は全く無意味でしたから、影響はないと思います。

% ----- 2018 年となっては 15 年以上前の話なので削除 -----
% なお、v2よりもっと古く配布していた、横組用“数字)”型表記用
% マクロendnotesjs.styは、endnotesj.styに機能が改良統合された
% ため、廃止しました。また、以前は本文肩註番号表記の実現に際し、
% 金水敏氏が作成された訓点資料表記用マクロkunten2e.styに定義され
% ている\verb+\MigiNakaTn+コマンドに依存しておりましたが、記述を
% 見直すことにより、非依存となりました。

旧版(v2.1)からv3.0での変更点は、以下のとおりです。
\begin{itemize}
\item パッケージのコード本体からASCII文字以外を排除。
\item 後註内部マクロを最新の\Lpack{endnotes.sty}%
  \footnote{Date of this version: 15 January 2003.}ベースに更新。
  これにより、例えば本文中で合印直前で行分割することがあった問題が
  解決しました。
\item  パッケージのオプションを拡張。
  組方向(縦・横)に応じて適切な注の印が出る\Lopt{auto-tateyoko}を
  新設し、これをデフォルトに設定しました。
  また、「注」「註」以外のタイトル形式を可能になりました。
\item もし本家パッケージ(\Lpack{endnotes.sty})が存在する場合で、
  \Lpack{endnotesj.sty}が読み込まれた時点で未読み込みならば、
  読み込み済み扱いするようにした(そうしないと、後で意図せず
  読み込まれて、日本語対応コードが上書きされるかもしれないため。)
\item \verb+\linesparpage+の修正(行間の数ではなく行数で割って
  いたため、行数が合わないことがありました。また、\verb+\topskip+の
  分を差し引いておらず、正しい行間隔になっていませんでした。)
\end{itemize}

\section{実際のスタイル見本(日本語版)}

スタイル見本はGitHubのsamplesディレクトリを参照してください。

\end{document}
