%%%%%%%%%%%%%%%%%%%%%%%%%%%%%%%%%%%%%%%%%%%%%%%%%%%%%%%%%%%%%%%%
%
% LaTeX me first!!!
%
% This is mls-diag.tex of the MonTeX package. LaTeX this file
% and it will reveal everything... Software you need, software
% you have, etc.
%
%   Author: Oliver Corff
%     Date: October 1st, 2001
%
% Contents: The MonTeX package provides an environment for
%           writing Mongolian texts in a number of writings.
%
%%%%%%%%%%%%%%%%%%%%%%%%%%%%%%%%%%%%%%%%%%%%%%%%%%%%%%%%%%%%%%%%
\documentclass[11pt,a4paper]{article}
\newif\ifBadNews
\IfFileExists{diagnose.sty}{\usepackage{diagnose}}{\BadNewstrue}

\title{Mon\TeX\ Diagnostics:\\
	System Compatibility Report}
\author{Oliver Corff}

\begin{document}
\maketitle

\ifBadNews
	Please install \texttt{diagnose.sty} from CTAN before you
	proceed with the installation of Mon\TeX. Without
	\texttt{diagnose.sty}, Mon\TeX\ is not able to determine
	your system capabilities. Recompile this document once
	\texttt{diagnose.sty} is installed!
\else
\section{Introduction}

Mon\TeX\ relies on numerous functions supplied by the existing
\LaTeXe\ installation for offering advanced services. Before installing
Mon\TeX\ it might be interesting to find out which support is offered
by the system and which additional packages should be supplied in order
to raise Mon\TeX\ to its complete functionality.

This document will tell you which functionality is really available on
your system and
what you should do in case the desired support is missing. Whenever the
text informs you that something from CTAN (the Comprehensive TeX Archive
Network) should be installed then you have the choice to obtain the
required software from CTAN via the Internet or from a CD-ROM with the 
contents of the CTAN site. In short, this will usually be the same source
from which you obtained Mon\TeX.


\section{RL Line Orientation\label{RLSupport}}

If you want to be able to print complete paragraphs of Uighur Mongolian 
texts, and not limit yourself to individual words and phrases, then you
will most certainly require the correct writing direction for Uighur 
Mongolian which starts at the right end of line rather than the left end,
as supplied by Implementation Levels III and IV of Mon\TeX.
Apart from its vertical orientation, Uighur Mongolian is a RL writing
much like Arab and Hebrew. The RL mechanism is provided by modern
\LaTeXe\ installations.

\TokenDiagnostics%
	{\TeXXeTstate}				% Token tested for
	{RLSupportAvailable}			% Associated counter
	{RL Support Available!}			% Yes message
	{RL Support Not Available!}		% No! message

\textbf{Your system:}
\ifnum\theRLSupportAvailable=1
	Your system is up to date and provides support for RL writings.
	No further action is required.
\else
	During this \LaTeXe\ run, your system did not reveal that it is
	capable of typesetting RL writings correctly. Please make sure
	\marginpar{\huge!!!}
	you started \texttt{elatex} instead of \texttt{latex} for calling
	e-\LaTeXe. If this message persists then you should install a
	recent \TeX\ distribution from CTAN.
\fi


\section{Vertical Text\label{VerticalText}}

If you want to mix horizontal and vertical text on one page your
system needs PostScript support. This implies that your \TeX\ 
installation can handle insertions of PostScript commands in
documents (this is done automatically without your interaction);
it does not imply that you can print the results on your printer
if your printing equipment is not PostScript-compatible.

\PackageDiagnostics%
	{rotating.sty}				% Package tested for
	{PostScriptAvailable}			% Associated counter
	{PostScript Support Available!}		% Yes message
	{PostScript Support Not Available!}	% No! message
	
\textbf{Your system:}
\ifnum\thePostScriptAvailable=1
	Your \TeX\ installation can handle vertical text; if your
	system does not display or print vertical text properly
	then you should check whether you do have a PostScript
	printer or interpreter like GhostScript and GhostView.
\else
	Your \TeX\ installation cannot handle vertical text;
	you should either install a \TeX\ system with pre-configured
	\marginpar{\huge!!!}
	PostScript support or should install PostScript support
	from CTAN.
\fi
%

\section{Uighur Mongolian Text Pages}

\ifnum\theRLSupportAvailable=0
	Since your system does not offer RL support (see above
	\ref{RLSupport}) this section is meaningless and should
	be skipped.
\else
	If you want to put Uighur text in vertical pages which
	show the proper orientation without physically rotating
	the paper they are printed on (like in books with a
	mixture of Uighur Mongolian and Modern Mongolian articles)
	you must make sure that your installation has so-called
	landscape orientation support.

	\PackageDiagnostics%
		{lscape.sty}				% Package tested for
		{LandscapeAvailable}			% Associated counter
		{Mongol Bicig Page Support Available!}	% Yes message
		{Mongol Bicig Page Support Not Available!}% No! message

	\ifnum\theLandscapeAvailable=1
		Your system has landscape orientation support and
		does not require any action.
	\else
		You should install landscape orientation support
		(usually this is done together with support for
		vertical text material, see above \ref{VerticalText}).
	\fi
\fi


%%  \section{Extended Cyrillic Support}
%%  
%%  Beyond the basic Cyrillic support which is provided by Mon\TeX\ for
%%  writing Modern Xalx Mongolian, Buryat and last but not least, Russian,
%%  Kazakh requires additional Cyrillic characters which are not supplied
%%  by Mon\TeX.
%%  
%%  \PackageDiagnostics%
%%  	{x2enc.def}				% Package tested for
%%  	{LHCyrillicAvailable}			% Associated counter
%%  	{Cyrillic Support for Kazakh available!}% Yes message
%%  	{Cyrillic Support for Kazakh Not Available!}% No! message
%%  
%%  \ifnum\theLHCyrillicAvailable=1
%%  	Since your installation offers the full range of extended
%%  	Cyrillic characters, it is possible to write Kazakh on your 
%%  	system. No further measures are required.
%%  \else
%%  	Your installation does not offer the extended Cyrillic character
%%  	set necessary for writing Kazakh. You should install the LH
%%  	Cyrillic package from CTAN. Even without this support, your
%%  	Mon\TeX\ package as such is fully functional.
%%  \fi
%%  
\section{Gamma Support}

The Greek gamma $\gamma$ is used in transliterations for Uighur 
Mongolian Script. Without support of Modern Greek fonts, only
the slim, italic $\gamma$ used in mathematics is available.
If you want to match its typeface with all Latin and Cyrillic
letters used in Mon\TeX\ then you should make sure the Modern
Greek fonts are installed.
%
\PackageDiagnostics%
	{lgrenc.def}				% Package tested for
	{GreekAvailable}			% Associated counter
	{Greek Support for Gammas available!}% Yes message
	{Greek Support for Gammas Not Available!}% No! message

\ifnum\theGreekAvailable=1
	On this system, the Modern Greek fonts \textit{are} installed.
	There is no further action necessary.
\else
	If you use transliterations extensively and need a matching
	\marginpar{\huge!!!}
	typeface for the $\gamma$ then you should get the Modern
	Greek fonts from CTAN and install them now.
\fi

\section{Tibetan Support}

Though Tibetan, strictly speaking, is no part of Mon\TeX, it is
suggested to have Tibetan on your system in order to properly
display the Manju transliterations for Tibetan in the documentation.

\PackageDiagnostics%
	{lctenc.def}				% Package tested for
	{TibetanAvailable}			% Associated counter
	{Tibetan Support available!}		% Yes message
	{Tibetan Support Not Available!}	% No! message

\ifnum\theTibetanAvailable=1
	On this system, a suitable Tibetan system  \textit{is} installed.
	There is no further action necessary.
\else
	At least for properly printing the section on Tibetan-Manju
	\marginpar{\huge!!!}
	transliterations, it is highly recommended to install
	\texttt{ctibtex} from CTAN:language/tibetan.
\fi


\fi
\end{document}

