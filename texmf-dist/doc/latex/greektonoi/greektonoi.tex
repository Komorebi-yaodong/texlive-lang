\documentclass{article}
\usepackage{fontspec} 
\usepackage{polyglossia}
\setmainlanguage{english}
\setotherlanguage[variant=polytonic]{greek}
\setmainfont % load font from path
    [
        %Path = C:/Windows/Fonts/,
        Extension = .otf,
        UprightFont = LinLibertine_R,
        BoldFont = LinLibertine_RZ, % Linux Libertine O Regular Semibold
        ItalicFont = LinLibertine_RI,
        BoldItalicFont = LinLibertine_RZI, % Linux Libertine O Regular Semibold Italic
    ]
    {libertine}
\newfontfamily\greekfont{Linux Libertine O}
\newfontfamily\greekGFS{GFS Neohellenic}
\newfontfamily\baske[Mapping=greektonoi]{GFS Baskerville}
\setmonofont[Scale=0.75]{DejaVu Sans Mono}
\newfontfamily\ariall{Arial}
\usepackage{hyperref}
\usepackage{metalogo}
\usepackage{booktabs}
\usepackage{longtable} 
\usepackage{greektonoi}
\title{Documentation for the {\sf greektonoi.sty} package and {\sf greektonoi.map} mapping.}
\author{Charalampos Milt. Cornaros\thanks{ Questions, commentary and bug reports to \texttt{kornaros@aegean.gr} please.}}
\date{2016/01/01}
\let\tb=\textbackslash
\newcommand*{\cellpar}[1]{\parbox[b]{0.50\textwidth}{#1}}
\begin{document}
\maketitle
\begin{abstract}
The \textsf{greektonoi} package offers the possibility to directly type or paste in ancient Greek texts with diacritical marks and transforming monotonic texts to polytonic ones. To insert accents and breathings we use a method similar to the common \texttt{Beta Code} convention.

This file documents version 0.2 of \textsf{greektonoi.sty}.
\end{abstract}

\tableofcontents


\begin{greek}
\addtocontents{toc}{\par\vspace{1em}}
\addtocontents{toc}{\noindent\textbf{\textgreek{Περιεχόμενα}}\par}
\addtocontents{toc}{ ~\hfill\textbf{\textgreek{Σελ.}}\par}
\setcounter{section}{0}
 \section{\textgreek{Εισαγωγή}}
 
 Από το 1982 και ύστερα καταργήθηκε το πολυτονικό σύστημα γραφής στην Ελλάδα επίσημα από το Ελληνικό κοινοβούλιο. Όμως η ανάγκη της χρήσεως τους δεν έχει εξαλειφθεί. Πολλοί εκδοτικοί οίκοι, η Ορθόδοξη Εκκλησία της Ελλάδος, συγγραφείς, φιλόλογοι  και άλλοι ιδιώτες συνεχίζουν να γράφουν πολυτονικά. Στο www.unicode.org έχουν ευτυχώς συμπεριληφθεί σχεδόν όλοι οι μονοτονικοί και πολυτονικοί ελληνικοί χαρακτήρες των \texttt{UnicodeGreek and Coptic code set}  (με κώδικες που έχουν αρίθμηση U+ 0370 - U+03FF) , \texttt{Greek Extended code set} (από U+ 1F00 - U+1FFF), όπως επίσης και το \texttt{Combining Diacritical Marks code set} (U+0300 - U+036F). )  Ο ελληνικός δίφθογγος ου είναι ένας εκ των  ελαχίστων εξαιρέσεων γραμμάτων που δεν έχουν συμπεριληφθεί (δείτε
 \url{ https://en.wikipedia.org/wiki/Ou_(ligature) }
). Γενικά η πληκτρολόγηση των πολυτονικών ελληνικών παραμένει   ένα δύσκολο πρόβλημα. Το πρόβλημα έχει αντιμετωπιστεί επιτυχώς στο \TeX{} και στα παρακλάδια του. Κατασκευάσαμε ένα πακέτο({\sf greektonoi.sty}) και ένα map αρχείο ({\sf greektonoi.map}) που δίνει την δυνατότητα  εισαγωγής πολυτονικών (κυρίως) στο περιβάλλον \XeTeX. Μπορούμε επίσης να τα χρησιμοποιήσουμε για την εύκολη (αλλά όχι αυτόματη) μετατροπή των μονοτικών κειμένων σε πολυτονικά.

\section{\textgreek{Λεπτομέρειες σχεδιασμού.}}

H συγγραφή των πολυτονικών στο \LaTeX{} γίνεται συνήθως με χρήση της ελληνικής επιλογής του πακέτου(package) \textsf{babel}. Έτσι με την εντολή 
\begin{center} \verb'\usepackage[polutonikogreek]{babel}' \end{center}
 στον πρόλογο του αρχείου μας μπορούμε να αρχίσουμε να γράφουμε κατ’ ευθείαν πολυτονικό κείμενο. 
 
Η εισαγωγή των τόνων(accents) και των πνευμάτων(breathings) γίνεται πολύ απλά χωρίς την χρήση κάποιων συνδυασμών πλήκτρων κατ’ ευθείαν  από το πληκτρολόγιο. Οι αντιστοιχίες είναι: 
Για να θέσουμε τόνο (accent) χρησιμοποιούμε  το σύμβολο ' (U+0027),  θέτουμε τα διαλυτικά (dialytika  ή diaeresis) με το σύμβολο " (U+0022),   για την  οξεία (acute , οξεία) χρησιμοποιήσουμε  το ίδιο σύμβολο όπως με το τόνο, η  βαρεία(grave, vareia ή βαρεία) εισάγεται με το σύμβολο ` (U+0060) και η περισπωμένη (περισπωμένη, perispomeni) με το σύμβολο  \char"007E{}(U+007E). 

Για τα πνεύματα οι αντιστοιχίες είναι :
Για την ψιλή (smooth,  ψιλή – lenis – thin) το  σύμβολο > (U+003E), ενώ για την δασεία (rough, δασεία – asper – dense) χρησιμοποιούμε το σύμβολο < (U+003C). Τέλος η υπογεγραμμένη (ι υπογεγραμμένηum) εισάγεται με το σύμβολο |(U+007C). Για να γράψουμε, για παράδειγμα,  την λέξη  ἁμαρτία γράφουμε <amart'ia (δείτε και  \url{http://tex.stackexchange.com/questions/210843/how-do-i-write-amartia-in-polytonic-greek } )

Μια άλλη δυνατότητα είναι η χρήση του πακέτου \textsf{betababel} (δείτε \url{http://ftp.yzu.edu.tw/CTAN/macros/latex/contrib/betababel/betatest.pdf }). Και εδώ η φιλοσοφία είναι παρόμοια απλά αλλάζουν οι παραπάνω αντιστοιχίες. Έτσι παίρνουμε την  ψιλή με  ), την 
δασεία με ( , τα διαλυτικά(diaeresis) με  +, το τόνο( acute accent) με /, την βαρεία (grave accent) με  \tb, τη περισπωμένη (περισπωμένη accent) = και τέλος την υπογεγραμμένη (ι υπογεγραμμένη) με  |.


 Το βασικό πρόβλημα με τις παραπάνω μεθοδολογίες είναι ότι η έξοδος \emph{δεν} είναι {\sf unicode} και άρα δεν μπορούμε να επαναχρησιμοποιήσουμε το τελικό αποτέλεσμα.  Επίσης δεν έχουν αρκετή ευελιξία π.χ. το *)/ANDRA θα δώσει το σωστό αποτέλεσμα με το betababel package αλλά το */)ANDRA δεν θα δώσει. Ένα άλλο πρόβλημα που διαπιστώσαμε είναι ότι όλοι οι παραπάνω συνδυασμοί απαιτούν τις περισσότερες φορές  την χρήση του πλήκτρου {\sf Shift} πράγμα που πιστεύουμε καθυστερεί και δυσκολεύει την πληκτρολόγηση όταν χρησιμοποιούμε μόνο το ένα χέρι. 


\section{\textgreek{Το αρχείο κωδικοποίησης {\sf greektonoi.map}}}

Το \XeTeX{}  δίνει φανταστικές δυνατότητες στην εύκολη χρήση Unicode γραμματοσειρών ttf και otf φορμά χωρίς να χρειάζεται να τις έχουμε εγκαταστήσει στον υπολογιστή μας. Τα παραπάνω πακέτα \textsf{babel} και \textsf{betababel} δεν μπορούν δυστυχώς να χρησιμοποιηθούν στο περιβάλλον \XeTeX.  Για αυτό το λόγο σκεφτήκαμε να σχεδιάσουμε απ’ την αρχή ένα κατάλληλο αρχείο κωδικοποίησης για την εύκολη χρήση με την επιλογή \verb'\setmainlanguage[variant=polytonic]{greek}'
 του πακέτου \textsf{polygossia}. 
Στο \textsf{greektonoi.map} υπάρχουν σχεδόν τρεις χιλιάδες συνδυασμοί για την εύκολη εισαγωγή των πολυτονικών συμβόλων. Οι αντιστοιχίες είναι οι παρακάτω. Τα σύμβολα  ` (U+0060) και '(U+0027)  για την βαρεία,  τα -(U+002D) και |(U+007C) για την υπογεγραμμένη,  τα  ) (U+0029) και ] (U+005D) για την ψιλή,  τα  ((U+0028) και [(U+005B) για την δασεία,  τα  "  (U+0022) και + (U+002B) για τα διαλυτικά. Τον τόνο μπορούμε να τον εισάγουμε με το γνωστό τρόπο(συνδυασμός του ; με το αντίστοιχο φωνήεν)  με το  ελληνικό πληκτρολόγιο είτε με το σύμβολο /(U+002F). Τα σύμβολα |, ( , ) , +, /, = και ` αντιστοιχούν στα συνήθη σύμβολα |, ( , ) , +, /, = και \tb{}  που συναντάμε στο πακέτο \textsf{betababel}.

  H  χρήση του \textsf{Shift} στα περισσότερα απ’ αυτά μπορεί να ελαχιστοποιηθεί με την εναλλακτική χρήση των αντίστοιχων συμβόλων -, [, ], ", /, = και ` ή '  αντίστοιχα, τα οποία δεν χρειάζονται το πλήκτρο \textsf{Shift} για να εισαχθούν εκτός βέβαια για  το  "(U+0022). Το τελευταίο σύμβολο " συνήθως δεν εισάγεται σωστά στο κείμενο με κάποιους editors (π.χ. στο Word διότι με το ελληνικό πληκτρολόγιο εισάγεται λανθασμένα το «(U+00AB)). Σε αυτές τις περιπτώσεις μπορούμε εναλλακτικά να χρησιμοποιήσουμε το σύμβολο + Γενικά υπάρχει μεγάλη  ελευθερία με την θέση των παραπάνω βοηθητικών συμβόλων. Ο μόνος μνημονικός κανόνας που έχουμε θέσει για την τοποθέτηση των χαρακτήρων είναι: όλα τα παραπάνω σύμβολα μπαίνουν πάντα από τα αριστερά του φωνήεντος με οποιαδήποτε σειρά. Η υπογεγραμμένη ( | ή - ) πρέπει να μπαίνει οπωσδήποτε ακριβώς πριν (από τα αριστερά) ή ακριβώς μετά(από τα δεξιά) του φωνήεντος.  
Για να γράψουμε, για παράδειγμα,  την λέξη  ἁμαρτίᾳ,  γράφουμε [αμαρτία- ή (αμαρτία| ή [αμαρτ/ια|  ή (αμαρτ/ι-α . Οι δυο πρώτες γραφές είναι πολύ χρήσιμες στην περίπτωση που έχουμε στην διάθεσή μας ήδη μια μονοτονική λέξη ή ένα ολόκληρο κείμενο και μας ενδιαφέρει να το μετατρέψουμε εύκολα σε πολυτονικό με χρήση του greektonoi.map. Απλά προσθέτουμε τα κατάλληλα  σημάδια απ' τα αριστερά στα φωνήεντα που μας ενδιαφέρουν και θα πάρουμε έξοδο πολυτονικό κείμενο. 
\vspace{1em}\\ \noindent
{\bf Παράδειγμα:} Για να πάρουμε το πολυτονικό κείμενο 

\begin{quote}
Τῃ πάντα διδούσῃ καὶ ἀπολαμβανούσῃ φύσει ὁ πεπαιδευμένος καὶ αἰδήμων λέγει· «δὸς, ὃ θέλεις, ἀπόλαβε, ὃ θέλεις». Λέγει δὲ τοῦτο οὐ καταθρασυνόμενος, ἀλλὰ πειθαρχῶν μόνον καὶ εὐνοῶν αὐτῇ.\end{quote}


απο το αντίστοιχο ελληνικό μονοτονικό γράφουμε:

\begin{quote}
Τη- πάντα διδούση| κα`ι  ]απολαμβανούση| φύσει [ο πεπαιδευμένος κα`ί α]ιδήμων λέγει· «δ`ος, `(ο θέλεις, ]απόλαβε, [`ο θέλεις». Λέγει δ`ε το=υτο ο)υ καταθρασυνόμενος, ]αλλ`ά πειθαρχ=ών μόνον κα`ι ε]υνο=ων α]υτ=|η.
\end{quote}

Κάποιο πρόβλημα παρατηρήσαμε με τη χρήση των ] και [ μετά την αλλαγή γραμμής δηλ. μετά από την χρήση του διπλού backslash \tb\tb{} Για παράδειγμα, το \tb\tb [`υ =]υ δημιουργεί πρόβλημα ενώ το \tb\tb (`υ =]υ ή το \tb\tb\{\} [`υ =]υ κανένα. Η χρήση των διπλών \{ και \} είναι μια απλή λύση για να διαχωρίζουμε την εντολή \tb\tb{} απ’ αυτά που θα ακολουθήσουν. 
Παρακάτω δίνουμε ένα ολοκληρωμένο παράδειγμα της χρήσης του greektonoi encoding με \XeLaTeX.

\begin{verbatim}

\documentclass[a4paper]{article}
\usepackage{fontspec}
\setmainfont{Arial}
\usepackage{polyglossia}
\setmainlanguage[variant=polytonic]{greek}
\newfontfamily\greekfont{GFS Neohellenic}
\newfontfamily\baske[Mapping=greektonoi]{GFS Baskerville}
\begin{document} 
{\baske Τη- πάντα διδούση| κα`ι ]απολαμβανούση| φύσει [ο πεπαιδευμένος 
κα`ι α]ιδήμων λέγει· «δ`ος, `[ο θέλεις, ]απόλαβε, `[ο θέλεις». Λέγει δ`ε 
το=υτο ο]υ καταθρασυνόμενος, ]αλλ'α πειθαρχ=ών μόνον κα'ι ε]υνο=ων α]υτ=η-.
}\par
{\greekfont \addfontfeature{Mapping=greektonoi} Τη- πάντα διδούση| κα`ι 
]απολαμβανούση| φύσει (ο πεπαιδευμένος κα`ί α)ιδήμων λέγει·
 «δ`ος, [`ο θέλεις, ]απόλαβε, [`ο θέλεις». Λέγει δ`ε το=υτο ο]υ 
 καταθρασυνόμενος,  ]αλλ`ά 
πειθαρχ=ών μόνον κα`ι ε]υνο=ων α]υτ=η-.}
\end{document}

\end{verbatim} 

Η έξοδος είναι:\vspace{1em}

{\baske Τη- πάντα διδούση| κα`ι ]απολαμβανούση| φύσει [ο πεπαιδευμένος 
κα`ι α]ιδήμων λέγει· «δ`ος, `[ο θέλεις, ]απόλαβε, `[ο θέλεις». Λέγει δ`ε 
το=υτο ο]υ καταθρασυνόμενος, ]αλλ'α πειθαρχ=ών μόνον κα'ι ε]υνο=ων α]υτ=η-.
}\par
{\greekGFS \addfontfeature{Mapping=greektonoi} Τη- πάντα διδούση| κα`ι 
]απολαμβανούση| φύσει (ο πεπαιδευμένος κα`ί α)ιδήμων λέγει·
 «δ`ος, [`ο θέλεις, ]απόλαβε, [`ο θέλεις». Λέγει δ`ε το=υτο ο]υ 
 καταθρασυνόμενος,  ]αλλ`ά 
πειθαρχ=ών μόνον κα`ι ε]υνο=ων α]υτ=η-.}


Στο παράδειγμα χρησιμοποιήσαμε δυο νέες  οικογένειες την {\sf GFS Neohellenic} και την {\sf GFS Baskerville} από την Εταιρεία Ελληνικών Τυπογραφικών Στοιχείων 
  (δείτε \url{http://www.greekfontsociety.gr/pages/en_typefaces20th.html }) .  Στην πρώτη  γραμματοσειρά  χρησιμοποιήσαμε το greektonoi μόνο τοπικά δηλ. μόνο σε ένα κομμάτι του κειμένου μας, που περιλαμβάνεται μέσα στα\begin{center} \verb'{\greekfont  \addfontfeature{Mapping=greektonoi} ……},'\end{center} ενώ στην δεύτερη γραμματοσειρά που την ονομάσαμε \verb'\baske' χρησιμοποιήσαμε μόνιμα (σε όλη  την έκταση του κειμένου που χρησιμοποιεί την \verb'\baske') την κωδικοποίηση (mapping) {\sf greektonoi.map}.
Για να πάρουμε έξοδο(σε μορφή pdf) από το παραπάνω κώδικα θα πρέπει φυσικά να έχουμε τοποθετήσει το αρχείο {
\sf greektonoi.tec} στο ίδιο φάκελο που έχουμε σώσει(με κατάλληξη tex) τον παραπάνω κώδικα. 

\section{\textgreek{Το πακέτο} greektonoi}

Η παραπάνω κωδικοποίηση {\sf greektonoi.map} είναι χρήσιμη μόνο σε μεγάλα κείμενα αρχαίων ελληνικών χωρίς να παρεμβάλλονται μαθηματικά ή άλλες εντολές.  Η χρήση των [ και ] είναι εντελώς ακατάλληλη μέσα σε κείμενα με εντολές διότι αυτά χρησιμοποιούνται συνήθως για να οριοθετήσουν τα ορίσματα των εντολών. Συνεπώς θα έπρεπε να κατασκευαστεί ένα άλλο κατάλληλο πακέτο εντολών για τα πολυτονικά γράμματα με το \XeLaTeX{} για τέτοιες περιπτώσεις.  Για αυτό το λόγο, κατασκευάσαμε το πακέτο {\sf greektonoi.sty}. Μέσα στο πακέτο αυτό δίνονται  σχεδόν εννιακόσιες εντολές με τις οποίες μπορούμε να εισαγάγουμε πολύ εύκολα πολυτονικά, αρχαϊκά ελληνικά, αριθμητικά και άλλα σύμβολα.  Οι εντολές έχουν ορισθεί με τα γράμματα του ελληνικού αλφαβήτου(π.χ. β, ψ, δ, π κοκ.)  αν και σε μεταγενέστερη έκδοση σκοπεύουμε να χρησιμοποιήσουμε και λατινικούς χαρακτήρες για όσους χρήστες δεν διαθέτουν ελληνικό πληκτρολόγιο(greek keyboard). Στο {\sf greektonoi.sty} υπάρχει η ίδια φιλοσοφία με το {\sf greektonoi.map}. Η ψιλή παριστάνεται με το γράμμα ψ, η υπογεγραμμένη με το μ, ο τόνος με το τ, η βαρεία με το β, η δασεία με το δ, η περισπωμένη με το π και τα διαλυτικά με το λ.  Έτσι για  να γράψουμε το παραπάνω κείμενο με το πακέτο {\sf greektonoi.sty} μπορούμε να γράψουμε τον εξής κώδικα:

\begin{verbatim}
\documentclass[a4paper]{article}
\usepackage{fontspec}
\setmainfont{Arial}
\usepackage{polyglossia}
\setmainlanguage[variant=polytonic]{greek}
\newfontfamily\baske{GFS Baskerville}
\usepackage{greektonoi}
\begin{document}
{\baske Τ\μη πάντα διδούσ\μη κα\βι \ψα{}πολαμβανούσ\ημ φύσει \δο πεπαιδευμένος
 κα\βί α\ψι{}δήμων λέγει\; \<<δ\βος, \δβο θέλεις, \ψα{}πόλαβε, \βδο θέλεις\>>. 
 Λέγει δ\βε το\πυ{}το ο\ψυ καταθρασυνόμενος, \ψα{}λλ\βά πειθαρχ\πών μόνον κα\βι 
 ε\ψυ{}νο\πων α\ψυ{}τ\πμη.
}
\end{document}
\end{verbatim}


Θα πρέπει να επισημάνουμε ότι η χρήση του διπλού  διαχωριστικού \{\} είναι απαραίτητη για να χωρίσουμε μια εντολή από το υπόλοιπο μέρος της λέξης. Για παράδειγμα: το \tb ψαπολαμβανούσ\tb ημ δεν είναι σωστό διότι δεν υπάρχει εντολή με όνομα \tb ψαπολαμβανούσ Tο \{\} χωρίζει στη λέξη ἀπολαμβανούσῃ το   αρχικό ἀ (που γράφεται με την εντολή \tb ψα) από το πολαμβανούσῃ που γράφεται φυσικά ως πολαμβανούσ\tb ημ (όπου στο τέλος δεξιά έχει φυσικά χρησιμοποιηθεί η εντολή \tb ημ ή ισοδύναμα η \tb μη για  να τοπθετηθεί η υπογεγραμμένη κάτω από το η).


Επίσης πρέπει να επισημάνουμε ότι εάν ακολουθεί κενό ή τελικός χαρακτήρας (κόμμα, τελεία, άνω τελεία, θαυμαστικό, εισαγωγικά κοκ.) η χρήση του \{\} δεν είναι απαραίτητη. Μπορούμε εναλλακτικά, αν θέλουμε να αποφύγουμε την χρήση του {\sf Shift} για την πληκτρολόγηση των \{\}, να χρησιμοποιήσουμε τις  εντολές \tb / ή  \tb ] ή τέλος τον ειδικό χαρακτήρα
\char"007E (tilde, U+007E).
Βέβαια, ο τελευταίος χρησιμοποιεί το {\sf Shift} για την πληκτρολόγησή του οπότε ο χρόνος που απαιτείται για την πλητρολόγηση π.χ. του \tb ](δύο σύμβολα)  είναι  μάλλον  ο ίδιος με τον χρόνο που απαιτείται για να πατήσουμε τον συνδυασμό {\sf Shift +tilde}. Αν οπωσδήποτε προτιμάμε το tilde ως ενωτικό σύμβολο στο πολυτονικό κειμενό μας τότε θα πρέπει να βάλουμε κάπου στην αρχή του  κυρίως μέρους του κώδικά μας (μετά το \verb'\begin{document}') την εντολή \verb'\tildeON'. Η εντολή \verb'\tildeOFF' επαναφέρει το \char"007E  {} στην συνηθισμένη χρήση του στο LaTeX.
\vspace{1em}\\ \noindent
{\bf Παράδειγμα:}
 \tb ψα\tb ]πόλαβε ή εναλλακτικά \tb tildeON ....  \tb ψα\char"007E πόλαβε ....\tb tildeOFF

\section{\textgreek{Τελευταία σχόλια. }}


Το πακέτο {\sf greektonoi.sty} πρέπει να υπάρχει στο ίδιο φάκελο που έχουμε σώσει τον κώδικά μας ή να το έχουμε εγκαταστήσει στο \TeX{} σύστημα μας.  Μπορεί να χρησιμοποιηθεί επικουρικά μαζί με το {\sf greektonoi.tec} ή ανεξάρτητα απ' αυτό.  Προσφέρει μεγάλες δυνατότητες για την εύκολη συγγραφή πολυτονικού ή την μετατροπή μονοτονικού σε πολυτονικό με χρήση εντολών. Μέσα σε αυτό έχουν εισαχθεί με κατάλληλες ευκολομνημόνευτες εντολές πολλοί χαρακτήρες ακόμα και από αρχαϊκά ελληνικά.   Για παράδειγμα για να εισάγουμε τα διπλά αριστερά εισαγωγικά (U+00AB, left-pointing double angle quotation mark 
« ) μπορούμε να χρησιμοποιήσουμε την εντολή \tb <<,  ενώ για τα αντίστοιχα δεξιά(U+00BB, ») την εντολή \tb >>.  Για να πάρουμε την άνω τελεία (U+0387, Greek ano teleia) χρησιμοποιούμε την εντολή \tb ; Υπάρχει ακόμα η δυνατότητα να εισάγουμε   αριθμητικά σύμβολα του ελληνικού συστήματος (όπως τα σαμπι \σαμ και \Σαμ με τις εντολές \tb σαμ και  \tb Σαμ για να πάρουμε  τα σύμβολα με unicode κώδικες U+ 03E1 και U+03E0 δηλ. το greek small letter sampi και το letter sampi αντίστοιχα). Υπάρχουν και κάμποσοι αρχαϊκοί χαρακτήρες όπως τα   δίγγαμμα \γγ και \Γγ που έχουν ορισθεί με τις εντολές \tb γγ (U+03DD, small digamma) και \tb Γγ (U+03DC, Digamma) αλλά και πολλές άλλες εντολές για  (σχεδόν) όλα τα σύμβολα που περιλαμβάνονται στα unicode sets που αναφέραμε στην εισαγωγή. Μπορούμε να τις  εντοπίσουμε με ένα απλό έλεγχο του πίνακα που ακολουθεί.

\section{\textgreek{Ευχαριστίες. }}

Ευχαριστώ θερμά τον αναπληρωτή καθηγητή κ. Αντώνιο Τσολομύτη  του τμήματος μαθηματικών του Πανεπιστημίου Αιγαίου για την επισήμανση ενός σχεδιαστικού λάθους στην αρχική κατασκευή του {\sf greektonoi.map}. 
\section{Πίνακας με τις {\sf greektonoi} εντολές}

\begin{longtable}{@{}lll@{}}\toprule
\multicolumn{1}{l}{Command}&\multicolumn{1}{l}{Output}&\multicolumn{1}{l}{Example}
%\hline
\\\midrule
\multicolumn{3}{c}{{\bf greektonoi.map}}\endfirsthead
\multicolumn{1}{l}{Command}&\multicolumn{1}{l}{Output}&\multicolumn{1}{l}{Example}\\\midrule
\endhead
\midrule
\char"0028{} ή [ για δασεία & ῾ & [α \char"21D2{} ἁ\\
\char"0029{}  ή ] για ψιλή & ᾿  & )α \char"21D2{} ἀ\\
|  ή - για ι υπογεγραμμένη &   ͅ  & η| ή η- ή |η ή -η  \char"21D2{} ῃ\\
 "  ή + για διαίρεση &   ¨  &   "υ  \char"21D2{} ϋ, "ύ  \char"21D2{} ΰ \\
  = για περισπωμένη &   ͂  &   =ω  \char"21D2{} ῶ\\
  ` ή ' για grave (βαρεία) &   ` &   `o ή 'ο ή 'ό ή `ό \char"21D2{} ὸ \\
  / ή ; (στο Greek Keyboard) για accute (οξεία) &  ´ &   /ο ή ό  \char"21D2{} ό \\
  
\midrule
\multicolumn{3}{c}{{\bf greektonoi.sty}}\\
\midrule
\tb -- &  – &   \\
\tb --- &  — &   \\
\tb  / ή \tb ]  ή \{ \} &  (για το  \{\}) &   \\
\tb << &  \char"00AB &   \\
\tb >> &  \char"00BB &   \\ 
\tb (( & \char"201C &   \\
\tb )) &  \char"201D &   \\
\tb ; &  \; &   \\
\tb \char"007E &  \~ &   \\
 \tb  ααπ (left(α) single quotation(απ)  mark) &  \ααπ & \\
\tb δαπ (right(δ) single quot. mark)	 &  \δαπ & κα\tb βι \{\}\tb δαπ\tb  δαπ \char"21D2{} κα\βι{}\δαπ\δαπ  \\\midrule 
δ για δασεία & ῾ & \tb δα \char"21D2{} ἁ\\
ψ για ψιλή & ᾿  & \tb ψα \char"21D2{} ἀ\\
μ για ι υπογεγραμμένη &   ͅ  & \tb μη ή \tb ημ \char"21D2{} ῃ\\ 
 λ για διαίρεση &   ¨  &   \tb λυ  \char"21D2{} ϋ, \tb λύ  \char"21D2{} \λύ\\
 π για περισπωμένη &   ͂  &   \tb πω  \char"21D2{} \πω \\
 β για grave (βαρεία) &   ` &   
 \tb βι \char"21D2{} \βι \\
  τ  για accute (οξεία) &  ´ &  \tb τι  \char"21D2{} \τι\\
  \tb κρν (κορωνίς) & ᾽& \\\midrule 
  \multicolumn{3}{c}{{\bf Μακρά και βραχέα φωνέντα}}\\\midrule
\tb βρχα (alpha with vrachy)& ᾰ& \\ 
\tb μκρα (alpha with macron)& ᾱ& \\ 
\tb βρχΑ (capital alpha with vrachy) & \βρχΑ & \\ 
\tb μκρΑ (capital alpha with macron) & \μκρΑ& \\ 
\tb βρχι (iota with vrachy) & ῐ & \\ 
\tb μκρι (iota with macron) & ῑ & \\ 
\tb βρχΙ (capital iota with vrachy) & \βρχΙ & \\ 
\tb μκρΙ (capital iota with macron) & \μκρΙ& \\ 
\tb βρχυ (upsilon with vrachy)& ῠ& \\ 
\tb μκρυ (upsilon with macron)& ῡ& \\ 
\tb βρχΥ (capital upsilon with vrachy) & Ῠ& \\ 
\tb μκρY (capital upsilon with macron) & Ῡ& 
\\\midrule
\multicolumn{3}{c}{{\bf Εναλλακτικές μορφές}}\\\midrule
  \tb   εβ (beta symbol)&\εβ  & \εβ vs.  β\\  
 \tb   εθ(theta symbol)&   \char"03D1 & \εθ vs.  θ\\ 
   \tb   εΘ(capital theta symbol) &   \εΘ &  \εΘ vs.  Θ \\  
 \tb   εφ (phi symbol) &   \char"03D5 & \εφ vs.  φ\\ 
 \tb   επ  (pi symbol) &   \char"03D6 & \επ vs.  π\\
 \tb   ερ  (rho symbol)&   \char"03F1 & \ερ vs.  ρ\\ 
 \tb   εκ  (kappa symbol)&   \char"03F0 & \εκ vs.  κ \\ 
\tb εε ή \tb ηε(lunate epsilon symbol) & \char"03F5 & \εε vs.  ε
\\\midrule
\multicolumn{3}{c}{{\bf Αρχα\λι\]κά}}\\\midrule
 \tb   γρ  (rho ρ with stroke  i.e.  {\bf γ}ραμμή) &  {\ariall  \char"03FC} & \\ 
 \tb ησ  (small lunate sigma  {\bf η}μισέλινο  {\bf σ}ίγμα) & \char"03F2 & \\
\tb Ησ ή \tb ηΣ(Capital lunate sigma  {\bf Η}μισέλινο  {\bf Σ}ίγμα)& \char"03F9 & \\ 
\cellpar{\tb Ητσ ή \tb ητΣ \\(Capital dotted lunate sigma -
{\bf Η}μισέλινο Σίγμα {\bf τ}ελίτσα)} & \char"03FE & \\ 
\tb ητσ (small dotted lunate sigma) & \char"037C & \\ 
\tb αηε (reversed(α) lunate(η) epsilon(ε) symbol)& \αηε & \\
\tb αησ (small reversed  lunate sigma)& \char"037B & \\ 
\tb Αησ ή \tb αηΣ  (Capital  reversed(Α)  lunate(η) sigma(σ)) & \char"03FD & \\ 
\cellpar{\tb Αητσ  ή \tb αητΣ \\(Capital reversed dotted lunate sigma)}& \char"03FF & \\ 
\tb ατσ (reversed dotted sigma)& \char"037D & \\ 
\tb γΥ (Upsilon with hook(γ) symbol)& \char"03D2 & \\ 
\tb γΎ (Upsilon with acute and hook symbol)& \char"03D3 & \\ 
\cellpar{\tb λγΥ ή \tb γΫ \\(Upsilon with diaeresis(λ) and hook(γ) symbol)}& \char"03D4 & \\ 
\tb ιωτ(greek letter yot) & \char"03F3 & \\ 
\tb σαν (small letter san)& \char"03FB & \\ 
\tb Σαν (capital letter San)& {\ariall \char"03FA} & \\ 
\tb σχω (greek small letter sho)& \char"03F8 & \\ 
\tb Σχω (capital letter Sho)& {\ariall \char"03F7} & 
\\\midrule
\multicolumn{3}{c}{{\bf Ου Δίφθογγος}}\\\midrule
\tb Ου  ή \tb Λου (Latin Capital Script Ou)& \char"0222 & \\
\tb ου ή \tb λου (Latin small Script ou)& \char"0223 & \\ 
\tb Κου (Cyrillic uppercase letter monograph Uk) &  \char"A64A  & \\ 
\tb κου (Cyrillic lower letter monograph Uk ) &  \char"A64B & \\ 
\midrule
\multicolumn{3}{c}{{\bf Αριθμητικά σημάδια και Ελληνικοί  Αριθμοί}}\\\midrule
\cellpar{\tb καχ ή ακ\\ (lower(κ) numeral(α) sign για thousands(χ) ή left(α) keraia(κ))} &   \καχ &\\
\cellpar{\tb ααμ ή δκ \\(upper(α) numeral(α) sign για smaller values(μ) ή right(δ) keraia)} &   \ααμ & \tb ακ εωοε\tb δκ \char"21D2{} \ακ εωοε\δκ\\
\tb διγ ή \tb γγ ή  \tb δγ (small digamma)& \char"03DD & \\ 
\tb Διγ ή \tb Γγ ή \tb Δγ  (Capital digamma)& \char"03DC & \\ 
\tb κοπ (small letter koppa)& \char"03DF & \\ 
\tb Κοπ (Capital letter koppa)& \char"03DE & \\  
\tb Ακοπ ή  \tb αΚοπ  ή \tb Κοφ (archaic(A) koppa) & \char"03D8 & \\ 
\tb ακοπ ή \tb κοφ (small archaic koppa)& \char"03D9 & \\
\tb σαμ (sampi)& \char"03E1 & \\ 
\tb Σαμ (Sampi)& \char"03E0 & \\ 
\tb στ (small stigma)& \char"03DB & \\ 
\tb Στ (capital stigma) & \char"03DA & \\\midrule 
\multicolumn{3}{c}{{\bf Χρήσιμοι τελικοί συνδυασμοί}}\\\midrule
\tb ός ή \tb τος & \τος& \\ 
\tb άν ή \tb ταν & άν& \\ 
\tb έν ή \tb τεν & έν& \\ 
\tb ήν ή \tb την & ήν& \\ 
\tb όν ή \tb τον& όν& \\ 
\tb ύν ή \tb τυν& \ύν& \\ 
\tb ών ή \tb των& ών& \\ 
\tb άς  ή \tb τας& άς& \\ 
\tb έρ ή \tb τερ& έρ& \\ 
\tb μώω ή \tb τμωω & \τμωω & \\ 
\tb μώων ή \tb τμωων & \τμωων& \\ 
\tb ές ή \tb τες & ές& \\ 
\tb ής ή \tb της& ής& \\ 
\tb ίς ή \tb τις& ίς& \\ 
\tb ύς ή \tb τυς& ύς& \\ 
\tb ώς ή \tb τως& ώς& \\ 
\tb έρ ή \tb τερ& έρ& \\ 
\tb ίο ή \tb τιο& ίο& \\ 
\tb ήρ ή \tb τηρ & ήρ& \\ 
\tb ύω ή \tb τυω & ύω& \\ 
\tb ίων ή \tb τιων& ίων& \\ 
\tb ίως ή \tb τιως & ίως& \\ 
\tb ίω ή \tb τιω & ίω& \\ 
\tb έω ή \tb τεω & \τεω& \\ 
\tb έων ή \tb τεων& έων& \\ 
\tb έως ή \tb τεως& έως& \\ 
\tb άω ή \tb ταω & άω& \\ 
\tb βάς ή \tb βας & ὰς& \\ 
\tb βεν ή \tb βέν  & ὲν& \\ 
\tb βερ ή \tb βέρ & ὲρ& \\  
\tb βες ή \tb βές& ὲς& \\  
\tb βόν ή \tb βον & ὸν& \\ 
\tb βός ή \tb βος& ὸς& \\ 
\tb βις ή \tb βίς& ὶς& \\ 
\tb βιν ή \tb βίν& ὶν& \\ 
\tb βυν ή \tb βύν & ὺν& \\ 
\tb βής ή \tb βης & ὴς& \\ 
\tb βηρ ή \tb βήρ & ὴρ& \\  
\tb βήν ή \tb βην& ὴν& \\ 
\tb βυς ή \tb βύς & ὺς& \\ 
\tb βώς ή \tb βως & ὼς& \\  
\tb βών ή \tb βων& ὼν& \\  
\tb δβας ή \tb δβάς  & ἃς& \\ 
\tb δβος ή \tb δβός & ὃς& \\ 
\tb δβυς ή \tb δβύς& ὓς& \\ 
\tb δβεν ή \tb δβέν& ἓν& \\ 
\tb δβην ή \tb δβήν& ἣν& \\ 
\tb δβον ή \tb δβόν& ὃν& \\ 
\tb δτεν ή \tb δέν& ἕν& \\ 
\tb δτως ή \tb δώς& ὥς& \\ 
\tb δτος ή \tb δός & ὅς& \\ 
\tb δπων ή \tb δπών & ὧν& \\ 
\tb δπις ή \tb δπίς& ἷς& \\ 
\tb δπια ή \tb δπία& ἷα& \\ 
\tb δπυς ή \tb δπύς& ὗς& \\  
\tb δτυω ή \tb δύω& ὕω& \\ 
\tb δτια ή \tb δία& ἵα& \\  
\tb δως & ὡς& \\ 
\tb πδις ή \tb πδίς & ἷς& \\ 
\tb πδιος ή \tb πδίος& ἷος& \\ 
\tb πδιον ή \tb πδίον& ἷον& \\ 
\tb πδιοί ή \tb πδίοί & ἷοί& \\ 
\tb πδων ή \tb πδών & ὧν& \\ 
\tb πας ή \tb πάς& ᾶς& \\ 
\tb πμαον ή \tb πμάον& ᾷον& \\ 
\tb παν ή \tb πάν & ᾶν& \\ 
\tb πυν ή \tb πύν& ῦν& \\   
\tb πυξ ή \tb πύξ& ῦξ& \\  
\tb πυς ή \tb πύς & ῦς& \\  
\tb πψυς ή \tb πψύς& ὖς& \\  
\tb πψυν ή \tb πψύν& ὖν& \\  
\tb πψην ή \tb πψήν& ἦν& \\ 
\tb πδυς ή \tb πδύς& ὗς& \\  
\tb πις ή \tb πίς & ῖς& \\ 
\tb πιος ή \tb πίος& ῖος& \\ 
\tb πιο ή \tb πίο & ῖο& \\ 
\tb πια ή \tb πία& ῖα& \\ 
\tb πδιά & ἷά& \\ 
\tb πδίς ή \tb πδις& ἷς& \\ 
\tb πιν ή \tb πίν & ῖν& \\ 
\tb πμωα ή \tb πμώα& ῷα& \\ 
\tb πων ή \tb πών& ῶν& \\ 
\tb πως ή \tb πώς& ῶς& \\  
\tb πμας ή \tb πμάς& ᾷς& \\ 
\tb πμώος ή \tb πμωος& ῷος& \\ 
\tb πμώα ή \tb πμωα& ῷα& \\ 
\tb πμώω ή \tb πμωω& ῴω& \\ 
\tb πής ή \tb πης& ῆς& \\ 
\tb πμης ή \tb πμής& ῇς& \\ 
\tb μης ή \tb μής& ῃς& \\ 
\tb πήρ ή \tb πηρ& ῆρ& \\ 
\tb πήν ή \tb πην & ῆν& \\ 
\tb ψβαν ή \tb ψβάν & ἂν& \\ 
\tb ψταν ή \tb ψάν& ἄν& \\ 
\tb ψτον ή \tb ψόν & ὄν& \\ 
\tb ψπυς ή \tb ψπύς& ὖς& \\  
\tb ψπυν ή \tb ψπύν& ὖν& \\ 
\tb ψτεκ ή \tb ψέκ& ἔκ& \\ 
\tb ψβην ή \tb ψβήν & ἢν& \\ 
\tb ψβων ή \tb ψβών& ὢν& \\ 
\tb ψτων ή \tb ψών& ὤν& \\ 
\tb ψβον ή \tb ψβόν& ὂν& \\ 
\tb ψτιη ή \tb ψίη & ἴη& \\ 
\tb ψτιης ή \tb ψίης& ἴς& \\ 
\tb ψτυω ή \tb ψύω& ὔω& \\  
\tb ψπυς ή \tb ψπύς& ὖς& \\ 
\tb ψτεν ή \tb ψέν& ἔν& \\ 
\tb ψτην ή \tb ψήν& ἤν& \\ 
\tb ψπην ή \tb ψπήν& ἦν& \\ 
\tb ψις & ἰς& \\  
\tb ψεν & ἐν& \\ 
\tb ψες & ἐς& \\ 
\tb ψεπ & ἐπ& \\ 
\tb μής & ῄς& \\ 
\tb μάς ή \tb μτας& ᾴς& \\ 
\tb μώα & ῴα& \\  
\tb μης & ῃς& \\ 
\tb ρρ & ῤῥ& \\
\tb και (small kai symbol) & \char"03D7 &\\
\tb Και (capital kai symbol) &  \char"03CF(most  fonts haven't it) &\\
\end{longtable}


%\section{Version history}
%\subsection{Changes in v0.1}
%
%
\end{greek}
\end{document}