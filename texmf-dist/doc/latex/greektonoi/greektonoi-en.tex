\documentclass{article}
\usepackage{fontspec} 
\usepackage{polyglossia}
\setmainlanguage{english}
\setotherlanguage[variant=polytonic]{greek}
\setmainfont % load font from path
    [
        %Path = C:/Windows/Fonts/,
        Extension = .otf,
        UprightFont = LinLibertine_R,
        BoldFont = LinLibertine_RZ, % Linux Libertine O Regular Semibold
        ItalicFont = LinLibertine_RI,
        BoldItalicFont = LinLibertine_RZI, % Linux Libertine O Regular Semibold Italic
    ]
    {libertine}
\newfontfamily\greekfont{Linux Libertine O}
\newfontfamily\greekGFS{GFS Neohellenic}
\newfontfamily\baske[Mapping=greektonoi]{GFS Baskerville}
\setmonofont[Scale=0.75]{DejaVu Sans Mono}
\newfontfamily\ariall{Arial}
\usepackage{hyperref}
\usepackage{metalogo}
\usepackage{booktabs}
\usepackage{longtable} 
\usepackage{greektonoi}
\title{Documentation for the {\sf greektonoi.sty} package and {\sf greektonoi.map} mapping.}
\author{Charalampos Milt. Cornaros\thanks{ Questions, commentary and bug reports to \texttt{kornaros@aegean.gr} please.}}
\date{2016/01/01}
\let\tb=\textbackslash
\newcommand*{\cellpar}[1]{\parbox[b]{0.50\textwidth}{#1}}
\begin{document}
\maketitle
\begin{abstract}
The \textsf{greektonoi} package offers the possibility to directly type or paste in ancient Greek texts with diacritical marks and transforming monotonic texts to polytonic ones. To insert accents and breathings we use a method similar to the common \texttt{Beta Code} convention.

This file documents version 0.2 of \textsf{greektonoi.sty}.
\end{abstract}

\tableofcontents

\begin{english}
%\addtocontents{toc}{\textbf{English Version}\par}
\addtocontents{toc}{ ~\hfill\textbf{Page}\par}
\section{Introduction}

The polytonic(multi-accents) writing system in Greece was abolished by the Greek parliament since 1982. However, the need for using it  has not been eliminated. Many publishing houses, the  Orthodox Church of Greece, writers, scholars and other individuals continue to write by using  the polytonic system. Fortunatelly,  almost all the monotonic and polytonic  Greek characters are included in the Unicode sets by the \url{www.unicode.org}: \texttt{Unicode Greek and Coptic code set} (with codes  U + 0370 - U+03FF), \texttt{Greek Extended code set} (from U  + 1F00 - U+1FFF)  and \texttt{Combining Diacritical Marks code set} ( U + 0300 - U+036F) as well. The Greek diphthong (δίφθογος) ‘ου’ is one of the few  letter combinations that are not included  (see  \url{https://en.wikipedia.org/wiki/Ou_(ligature)} ). In general, writing in the polytonic system (in electronic devices) remains a difficult problem. The problem has been successfully confronted with \TeX{} and its branches. We created a package {\sf greektonoi.sty } and a map file {\sf greetonoi.map} that gives us the opportunity to include pollytonic letters in our texts by using the \XeTeX{} engine. They can also be used in order to easily (though not automatically) convert monotonic texts into polytonic ones.

\section{Design Details}

Writing polytonic texts in   \LaTeX{}  can be usually performed by  using the Greek option  of \textsf{babel} package. So with the instruction: 
\begin{center} \verb'\usepackage[polutonikogreek]{babel}' \end{center} 
in the preamble of our file, we can immediately start writing   a polytonic text. The insertion of   accents and breathings is an easy task without  using any special combinations of keys and dead keys directly  from the keyboard. The encodings are: to insert the greek accent(tonos) we simply use the symbol '( U+0027), we put the diaeresis using the symbol " (U+0022). For the acute accent we use the same character as for the greek accent, we insert the vareia(grave) with the symbol `(U+ 0060) and the circumflex (perispomeni) with the symbol \char"007E  {}(U+007E). 
The encodings for the breathings are:  we use the symbol >(U+003E) for the psili(smooth, lenis),  we use the symbol <(U+003C)  for the daseia (asper, dense) and for the iota subscript (υπογεγραμμένη) we use the symbol |(U+OO7C). In order to write, for example, the word ἁμαρτία we  simply  insert <amart’ia (see \url{http://tex.stackexchange.com /questions/210843/how-do-i-write-amartia-in-polytonic-greek} ).


Another possibility is to  use  the \textsf{betababel} package (see \url{http://ftp.yzu.edu.tw/CTAN/macros/latex/contrib/betababel/betatest.pdf }).
This uses  the same concepts  with the exception that  the encodings are different from the ones previously used. So we insert  the psili with ), the daseia with (, the diaeresis with +, the acute accent with /, the vareia (grave accent) with \tb, the circumflex accent with the symbol = and  the iota subscript with the symbol |. 

The main problem with the  above  methodologies is that the output is \emph{not} unicode and as a result it can not used again. Also, there is not enough flexibility: For example, the */)ANDRA  will provide the correct result with the  \textsf{betababel} package but the */)ANDRA will not. Another problem that  emerges       is that most of the above combinations are inserted using the {\sf Shift}  key which is regarded as time consuming and makes typing more difficult in case of one hand typing.

\section{The mapping file {\sf greektonoi.map} }

The \XeTeX{} engine  gives us endless possibilities  for  the easy use of  Unicode fonts of ttf and otf font format without installing them in our computer. The  packages, babel and betababel, mentioned  above, cannot be used, unfortunately, with the \XeTeX{} engine. Because of that, we decided to design from scratch a proper mapping file to work with  the option \verb'\setmainlanguage[variant=polytonic]{greek}' of the \textsf{polygossia}  package. In \textsf{greektonoi.map} there are  almost 3,000 accent combinations in order to easily insert the polytonic symbols. The new encodings are : `(U+0027) and  '(U+0060) for  bareia, - (U+002D) and |(U+007C) for the iota subscript, the ) (U+0029) and ] (U+005D) for the psili, the =(U+003D) for circumflex,  the ((U+0028) and [(U+005B) for the daseia and  the "(U+0022) or +(U+002B) for the diaeresis. We can put the usual tonos(acute accent) with the normal way  by simply using the Greek keyboard (using, for example, a combination of ; with a vowel in PCs) or with the symbol /(U+002F). The symbols |, (, ), +, /, = and  `correspond to the  symbols |, ( , ) , +, /, = and  \tb{} that we see in the \textsf{betababel} package.


The use of the \textsf{Shift} key in most of the above cases could be minimized with the use of the symbols: -, [, ], ",  /,  = and ` or ' respectively that doesn’t  use  the \textsf{Shift} key in order to  insert them (except for the  "(U+0022) ). The last symbol " is not usually imported properly in the text with some  editors such as Word, for example(MS Word  wrongly imports the character  «(U+00AB) ).
In this case, we could alternatively use the symbol +. We could insert  these auxiliary symbols in our text in any possible  order. The only mnemonic rule we have set is: all of the diacritical marks are placed to the left of the vowel in any order. The iota subscript ( | or - ) should definitely come just before  or just after  the vowel. To taking, for example, the word ἁμαρτίᾳ, we write [αμαρτία- or (αμαρτία| or [αμαρτ/ια| or (αμαρτ/ι-α. The first two combinations [αμαρτία- or (αμαρτία|, are very useful in case we have already the (monotonic) word αμαρτία (or a whole monotonic text that we would like to convert  to a polytonic one using \textsf{greektonoi.map}).  We simply add the proper symbols on the left of the vowels that we are interested in and we get the pollytonic text in the output.\vspace{1em}\\ \noindent
{\bf Example:} In order to have the pollytonic text

\begin{quote}
Τῃ πάντα διδούσῃ καὶ ἀπολαμβανούσῃ φύσει ὁ πεπαιδευμένος καὶ αἰδήμων λέγει· «δὸς, ὃ θέλεις, ἀπόλαβε, ὃ θέλεις». Λέγει δὲ τοῦτο οὐ καταθρασυνόμενος, ἀλλὰ πειθαρχῶν μόνον καὶ εὐνοῶν αὐτῇ.\end{quote}

from the corresponding  monotonic we could type:

\begin{quote}
Τη- πάντα διδούση| κα`ι  ]απολαμβανούση| φύσει [ο πεπαιδευμένος κα`ί α]ιδήμων λέγει· «δ`ος, `(ο θέλεις, ]απόλαβε, [`ο θέλεις». Λέγει δ`ε το=υτο ο)υ καταθρασυνόμενος, ]αλλ`ά πειθαρχ=ών μόνον κα`ι ε]υνο=ων α]υτ=|η.
\end{quote}

Throughout this process, we noticed a problem with the use of the symbols ] and [ after the use of the double backslash \tb\tb{}. For example,  the \tb\tb [`υ =]υ causes a problem but the \tb\tb (`υ =]υ or \tb\tb\{\} [`υ =]υ does not. The use of the double \{ and \} is a simple solution in order to separate the \tb\tb{}  from the subsequent words. We will provide a complete example of   greektonoi enconding  using \XeLaTeX{} engine below.


\begin{verbatim}

\documentclass[a4paper]{article}
\usepackage{fontspec}
\setmainfont{Arial}
\usepackage{polyglossia}
\setmainlanguage[variant=polytonic]{greek}
\newfontfamily\greekfont{GFS Neohellenic}
\newfontfamily\baske[Mapping=greektonoi]{GFS Baskerville}
\begin{document} 
{\baske Τη- πάντα διδούση| κα`ι ]απολαμβανούση| φύσει [ο πεπαιδευμένος 
κα`ι α]ιδήμων λέγει· «δ`ος, `[ο θέλεις, ]απόλαβε, `[ο θέλεις». Λέγει δ`ε 
το=υτο ο]υ καταθρασυνόμενος, ]αλλ'α πειθαρχ=ών μόνον κα'ι ε]υνο=ων α]υτ=η-.
}\par
{\greekfont \addfontfeature{Mapping=greektonoi} Τη- πάντα διδούση| κα`ι 
]απολαμβανούση| φύσει (ο πεπαιδευμένος κα`ί α)ιδήμων λέγει·
 «δ`ος, [`ο θέλεις, ]απόλαβε, [`ο θέλεις». Λέγει δ`ε το=υτο ο]υ 
 καταθρασυνόμενος,  ]αλλ`ά 
πειθαρχ=ών μόνον κα`ι ε]υνο=ων α]υτ=η-.}
\end{document}
\end{verbatim} 
The output is\vspace{1em}

{\baske Τη- πάντα διδούση| κα`ι ]απολαμβανούση| φύσει [ο πεπαιδευμένος 
κα`ι α]ιδήμων λέγει· «δ`ος, `[ο θέλεις, ]απόλαβε, `[ο θέλεις». Λέγει δ`ε 
το=υτο ο]υ καταθρασυνόμενος, ]αλλ'α πειθαρχ=ών μόνον κα'ι ε]υνο=ων α]υτ=η-.
}\par
{\greekGFS \addfontfeature{Mapping=greektonoi} Τη- πάντα διδούση| κα`ι 
]απολαμβανούση| φύσει (ο πεπαιδευμένος κα`ί α)ιδήμων λέγει·
 «δ`ος, [`ο θέλεις, ]απόλαβε, [`ο θέλεις». Λέγει δ`ε το=υτο ο]υ 
 καταθρασυνόμενος,  ]αλλ`ά 
πειθαρχ=ών μόνον κα`ι ε]υνο=ων α]υτ=η-.}
 \vspace{0.3em}\\

In the above example  we used two new font families, the {\sf GFS Neohellenic} and {\sf GFS Baskerville} from the Greek Font Society (see \url{http://www.greekfontsociety.gr/pages/en_typefaces.html}). For the first typeface we used  {\sf greektonoi} mapping only locally ie. only for the part of our text that is included inside the \begin{center} \verb'{\greekfont  \addfontfeature{Mapping=greektonoi} ...},'\end{center} 
  We used the {\sf  greektonoi.map} mapping and  the second typeface named \verb'\baske' for a whole text area (that uses the \verb'\baske' typeface). In order to receive the  pdf output from the above code  we have to install the {\sf greektonoi} mapping or  alternatively, put the file {\sf greektonoi.tec} in the same folder  that we have saved  the above code(with the {\sf .tex} extention).

\section{The greektonoi package}
 
The  {\sf greektonoi.map} mapping is useful only in large polytonic texts written without any math symbols or other \TeX{} commands in the middle. The use of [ and ] to insert breathings is completely improper in texts that include  \TeX{} commands because these characters  are used to delimit the set of variables from the body of the command. Therefore,  we  created {\sf greektonoi.sty}  package which is  more suitable  for such cases.  There are almost nine hundred commands stored in the package to easily and effectively facilitate the import of  greek numerals, polytonic archaic Greek, accents, breathings and  other symbols. The commands have been using letters from the Greek alphabet (β, ψ, δ, π  and so on) although in a revised future version we intend to extend the package with commands that use only  ascii characters for those users that do not have the greek keyboard. In {\sf greektonoi.sty} underlie the same concepts as in  {\sf greektonoi.map}. The  vareia(\textbf{β}αρεία) is inserted using the letter  β, for psili(\textbf{ψ}ιλή) we use ψ, daseia(\textbf{δ}ασεία) is inserted with δ and the circumflex(\textbf{π}ερισπωμένη) is inserted  by typing  π.  For iota subscript (υπογεγραμ\textbf{μ}ένη)  we use the letter μ, for tonos the letter τ  and for  diaeresis(δια\textbf{λ}υτικά)  the λ character. So, in order  to write the text above    by using exclusively  the  {\sf greektonoi.sty} package
we could type the following: 

\begin{verbatim}
\documentclass[a4paper]{article}
\usepackage{fontspec}
\setmainfont{Arial}
\usepackage{polyglossia}
\setmainlanguage[variant=polytonic]{greek}
\newfontfamily\baske{GFS Baskerville}
\usepackage{greektonoi}
\begin{document}
{\baske Τ\μη πάντα διδούσ\μη κα\βι \ψα{}πολαμβανούσ\ημ φύσει \δο πεπαιδευμένος
 κα\βί α\ψι{}δήμων λέγει\; \<<δ\βος, \δβο θέλεις, \ψα{}πόλαβε, \βδο θέλεις\>>. 
 Λέγει δ\βε το\πυ{}το ο\ψυ καταθρασυνόμενος, \ψα{}λλ\βά πειθαρχ\πών μόνον κα\βι 
 ε\ψυ{}νο\πων α\ψυ{}τ\πμη.
}
\end{document}
\end{verbatim}


It should be noted that the use of the double  \{\} is necessary to separate a command from the rest of the word. For example: \tb ψαπολαμβανούσ\tb ημ is not correct because there is no command named \tb ψαπολαμβανούσ The \{\} combination separates  the initial ἀ (written
with the command \tb ψα) of  the word ἀπολαμβανούσῃ from  πολαμβανούσ\ημ (which uses the command \tb ημ  or equivalently the \tb μη  to put the iota subscript below η). It should  be also noted that if a space or other any final character (comma, semicolon, colon, quotation mark and so on)  follows then the use of \{\} is not nessesary. We can alternatively use the commands \tb  / or \tb ]  to avoid in this case the use of  {\sf Shift} key to type the \{\}.   We could also use  the tilde character \verb'~' (U + 007E) for this case. Certainly, the latter uses the {\sf Shift} key  so  the time required for  typing of \tb  ] (two characters), for example,  is the same as the time required for the combination {\sf Shift + tilde}. If we prefer the tilde symbol for separation purposes then we should  put  the command \verb'\tildeON' somewhere in the beginning of main part of the code. The command \verb'\tildeOFF' restores the \char"007E (tilde) to its  normal  \LaTeX{} use.
 \vspace{1em}\\ \noindent
 {\bf Example:}
 \tb ψα\tb ]πόλαβε or equivalently \tb tildeON ....  \tb ψα\char"007E πόλαβε ....\tb tildeOFF

\section{Final  Comments}

The {\sf greektonoi.sty} package must be stored at  the same folder in which we save our code or it must be  installed in our \TeX{} system. It can be used with the {\sf greektonoi.tec} or independently. It offers tremendous possibilities concerning typing polytonic greek  easily or converting monotonic to polytonic using simple commands. Many commands are stored within the package through which typing even archaic greek letters could be performed.  For example, to insert the left double quotes
(U + 00AB,  «) we can use the command \tb << while the corresponding right  quotes(U + 00BB, ») are inserted with the command \tb >>.
To obtain the semicolon (U + 0387, Greek ano teleia) we use the command \tb ; There is also the possibility to introduce numerals  of the Greek system such as sambi ϡ and Ϡ (ie. the greek small letter sampi and letter sampi respectively) with  the commands \tb σαμ and \tb Σαμ  respectively. There are quite a few archaic symbols like digamma  \Γγ(U + 03DC, Digamma) and \γγ(U + 03DD, small digamma) using the commands \tb Γγ  and \tb γγ respectively and many other commands for almost every  greek symbol included in the unicode code sets  mentioned in the introduction. We can find them by an inspection of   the table below.

\section{Acknowledgements}

 
I would like to thank the associate professor  Antonis Tsolomitis of the Department of Mathematics of the University of Aegean for pointing out  an  error during the design process of the first version of {\sf greektonoi.map} and also miss Margaret Christoforatou student of the Department of Mathematics in the University of Aegean who helped me  translate  this documentation at English.

\section{List of {\sf greektonoi} commands}

\begin{longtable}{@{}lll@{}}\toprule
\multicolumn{1}{l}{Command}&\multicolumn{1}{l}{Output}&\multicolumn{1}{l}{Example}
%\hline
\\\midrule
\multicolumn{3}{c}{{\bf greektonoi.map}}\endfirsthead
\multicolumn{1}{l}{Command}&\multicolumn{1}{l}{Output}&\multicolumn{1}{l}{Example}\\\midrule
\endhead
\midrule
\char"0028{} or [ for daseia & ῾ & [α \char"21D2{} ἁ\\
\char"0029{}  or ] for psili & ᾿  & )α \char"21D2{} ἀ\\
|  or - for iota subscript &   ͅ  & η| or η- or |η or -η  \char"21D2{} ῃ\\
 "  or + for diairesis &   ¨  &   "υ  \char"21D2{} ϋ, "ύ  \char"21D2{} ΰ \\
  = for circumflex &   ͂  &   =ω  \char"21D2{} ῶ\\
  ` or ' for grave (bareia) &   ` &   `o or 'ο or 'ό or `ό \char"21D2{} ὸ \\
  / or ; (in Greek Keyboard) for accute (oxeia) &  ´ &   /ο or ό  \char"21D2{} ό \\
  
\midrule
\multicolumn{3}{c}{{\bf greektonoi.sty}}\\
\midrule
\tb -- &  – &   \\
\tb --- &  — &   \\
\tb  / or \tb ]  or \{ \} &  (to insert \{\}) &   \\
\tb << &  \char"00AB &   \\
\tb >> &  \char"00BB &   \\ 
\tb (( & \char"201C &   \\
\tb )) &  \char"201D &   \\
\tb ; &  \; &   \\
\tb \char"007E &  \~ &   \\
 \tb  ααπ (left(α) single quotation(απ)  mark) &  \ααπ & \\
\tb δαπ (right(δ) single quot. mark)	 &  \δαπ & κα\tb βι \{\}\tb δαπ\tb  δαπ \char"21D2{} κα\βι{}\δαπ\δαπ  \\\midrule 
δ for daseia & ῾ & \tb δα \char"21D2{} ἁ\\
ψ for psili & ᾿  & \tb ψα \char"21D2{} ἀ\\
μ for iota subscript &   ͅ  & \tb μη or \tb ημ \char"21D2{} ῃ\\ 
 λ for diairesis &   ¨  &   \tb λυ  \char"21D2{} ϋ, \tb λύ  \char"21D2{} \λύ\\
 π for circumflex &   ͂  &   \tb πω  \char"21D2{} \πω \\
 β for grave (bareia) &   ` &   
 \tb βι \char"21D2{} \βι \\
  τ  for accute (oxeia) &  ´ &  \tb τι  \char"21D2{} \τι\\
  \tb κρν (koronis) & ᾽& \\\midrule 
  \multicolumn{3}{c}{{\bf Macron and Vrachy vowels}}\\\midrule
\tb βρχα (alpha with vrachy)& ᾰ& \\ 
\tb μκρα (alpha with macron)& ᾱ& \\ 
\tb βρχΑ (capital alpha with vrachy) & \βρχΑ & \\ 
\tb μκρΑ (capital alpha with macron) & \μκρΑ& \\ 
\tb βρχι (iota with vrachy) & ῐ & \\ 
\tb μκρι (iota with macron) & ῑ & \\ 
\tb βρχΙ (capital iota with vrachy) & \βρχΙ & \\ 
\tb μκρΙ (capital iota with macron) & \μκρΙ& \\ 
\tb βρχυ (upsilon with vrachy)& ῠ& \\ 
\tb μκρυ (upsilon with macron)& ῡ& \\ 
\tb βρχΥ (capital upsilon with vrachy) & Ῠ& \\ 
\tb μκρY (capital upsilon with macron) & Ῡ& 
\\\midrule
\multicolumn{3}{c}{{\bf Rounded forms}}\\\midrule
  \tb   εβ (beta symbol)&\εβ  & \εβ vs.  β\\  
 \tb   εθ(theta symbol)&   \char"03D1 & \εθ vs.  θ\\ 
   \tb   εΘ(capital theta symbol) &   \εΘ &  \εΘ vs.  Θ \\  
 \tb   εφ (phi symbol) &   \char"03D5 & \εφ vs.  φ\\ 
 \tb   επ  (pi symbol) &   \char"03D6 & \επ vs.  π\\
 \tb   ερ  (rho symbol)&   \char"03F1 & \ερ vs.  ρ\\ 
 \tb   εκ  (kappa symbol)&   \char"03F0 & \εκ vs.  κ \\ 
\tb εε or \tb ηε(lunate epsilon symbol) & \char"03F5 & \εε vs.  ε
\\\midrule
\multicolumn{3}{c}{{\bf Archaic letters}}\\\midrule
 \tb   γρ  (rho ρ with stroke  i.e.  {\bf γ}ραμμή) &  {\ariall  \char"03FC} & \\ 
 \tb ησ  (small lunate sigma  {\bf η}μισέλινο  {\bf σ}ίγμα) & \char"03F2 & \\
\tb Ησ or \tb ηΣ(Capital lunate sigma  {\bf Η}μισέλινο  {\bf Σ}ίγμα)& \char"03F9 & \\ 
\cellpar{\tb Ητσ or \tb ητΣ \\(Capital dotted lunate sigma -
{\bf Η}μισέλινο Σίγμα {\bf τ}ελίτσα)} & \char"03FE & \\ 
\tb ητσ (small dotted lunate sigma) & \char"037C & \\ 
\tb αηε (reversed(α) lunate(η) epsilon(ε) symbol)& \αηε & \\
\tb αησ (small reversed  lunate sigma)& \char"037B & \\ 
\tb Αησ or \tb αηΣ  (Capital  reversed(Α)  lunate(η) sigma(σ)) & \char"03FD & \\ 
\cellpar{\tb Αητσ  or \tb αητΣ \\(Capital reversed dotted lunate sigma)}& \char"03FF & \\ 
\tb ατσ (reversed dotted sigma)& \char"037D & \\ 
\tb γΥ (Upsilon with hook(γ) symbol)& \char"03D2 & \\ 
\tb γΎ (Upsilon with acute and hook symbol)& \char"03D3 & \\ 
\cellpar{\tb λγΥ or \tb γΫ \\(Upsilon with diaeresis(λ) and hook(γ) symbol)}& \char"03D4 & \\ 
\tb ιωτ(greek letter yot) & \char"03F3 & \\ 
\tb σαν (small letter san)& \char"03FB & \\ 
\tb Σαν (capital letter San)& {\ariall \char"03FA} & \\ 
\tb σχω (greek small letter sho)& \char"03F8 & \\ 
\tb Σχω (capital letter Sho)& {\ariall \char"03F7} & 
\\\midrule
\multicolumn{3}{c}{{\bf Ου Diphthongs}}\\\midrule
\tb Ου  or \tb Λου (Latin Capital Script Ou)& \char"0222 & \\
\tb ου or \tb λου (Latin small Script ou)& \char"0223 & \\ 
\tb Κου (Cyrillic uppercase letter monograph Uk) &  \char"A64A  & \\ 
\tb κου (Cyrillic lower letter monograph Uk ) &  \char"A64B & \\ 
\midrule
\multicolumn{3}{c}{{\bf Greek Numeral Signs and Numerals}}\\\midrule
\cellpar{\tb καχ or ακ\\ (lower(κ) numeral(α) sign for thousands(χ) or left(α) keraia(κ))} &   \καχ &\\
\cellpar{\tb ααμ or δκ \\(upper(α) numeral(α) sign for smaller values(μ) or right(δ) keraia)} &   \ααμ & \tb ακ εωοε\tb δκ \char"21D2{} \ακ εωοε\δκ\\
\tb διγ or \tb γγ or  \tb δγ (small digamma)& \char"03DD & \\ 
\tb Διγ or \tb Γγ or \tb Δγ  (Capital digamma)& \char"03DC & \\ 
\tb κοπ (small letter koppa)& \char"03DF & \\ 
\tb Κοπ (Capital letter koppa)& \char"03DE & \\  
\tb Ακοπ or  \tb αΚοπ  or \tb Κοφ (archaic(A) koppa) & \char"03D8 & \\ 
\tb ακοπ or \tb κοφ (small archaic koppa)& \char"03D9 & \\
\tb σαμ (sampi)& \char"03E1 & \\ 
\tb Σαμ (Sampi)& \char"03E0 & \\ 
\tb στ (small stigma)& \char"03DB & \\ 
\tb Στ (capital stigma) & \char"03DA & \\\midrule 
\multicolumn{3}{c}{{\bf Some useful  combinations}}\\\midrule
\tb ός or \tb τος & \τος& \\ 
\tb άν or \tb ταν & άν& \\ 
\tb έν or \tb τεν & έν& \\ 
\tb ήν or \tb την & ήν& \\ 
\tb όν or \tb τον& όν& \\ 
\tb ύν or \tb τυν& \ύν& \\ 
\tb ών or \tb των& ών& \\ 
\tb άς  or \tb τας& άς& \\ 
\tb έρ or \tb τερ& έρ& \\ 
\tb μώω or \tb τμωω & \τμωω & \\ 
\tb μώων or \tb τμωων & \τμωων& \\ 
\tb ές or \tb τες & ές& \\ 
\tb ής or \tb της& ής& \\ 
\tb ίς or \tb τις& ίς& \\ 
\tb ύς or \tb τυς& ύς& \\ 
\tb ώς or \tb τως& ώς& \\ 
\tb έρ or \tb τερ& έρ& \\ 
\tb ίο or \tb τιο& ίο& \\ 
\tb ήρ or \tb τηρ & ήρ& \\ 
\tb ύω or \tb τυω & ύω& \\ 
\tb ίων or \tb τιων& ίων& \\ 
\tb ίως or \tb τιως & ίως& \\ 
\tb ίω or \tb τιω & ίω& \\ 
\tb έω or \tb τεω & \τεω& \\ 
\tb έων or \tb τεων& έων& \\ 
\tb έως or \tb τεως& έως& \\ 
\tb άω or \tb ταω & άω& \\ 
\tb βάς or \tb βας & ὰς& \\ 
\tb βεν or \tb βέν  & ὲν& \\ 
\tb βερ or \tb βέρ & ὲρ& \\  
\tb βες or \tb βές& ὲς& \\  
\tb βόν or \tb βον & ὸν& \\ 
\tb βός or \tb βος& ὸς& \\ 
\tb βις or \tb βίς& ὶς& \\ 
\tb βιν or \tb βίν& ὶν& \\ 
\tb βυν or \tb βύν & ὺν& \\ 
\tb βής or \tb βης & ὴς& \\ 
\tb βηρ or \tb βήρ & ὴρ& \\  
\tb βήν or \tb βην& ὴν& \\ 
\tb βυς or \tb βύς & ὺς& \\ 
\tb βώς or \tb βως & ὼς& \\  
\tb βών or \tb βων& ὼν& \\  
\tb δβας or \tb δβάς  & ἃς& \\ 
\tb δβος or \tb δβός & ὃς& \\ 
\tb δβυς or \tb δβύς& ὓς& \\ 
\tb δβεν or \tb δβέν& ἓν& \\ 
\tb δβην or \tb δβήν& ἣν& \\ 
\tb δβον or \tb δβόν& ὃν& \\ 
\tb δτεν or \tb δέν& ἕν& \\ 
\tb δτως or \tb δώς& ὥς& \\ 
\tb δτος or \tb δός & ὅς& \\ 
\tb δπων or \tb δπών & ὧν& \\ 
\tb δπις or \tb δπίς& ἷς& \\ 
\tb δπια or \tb δπία& ἷα& \\ 
\tb δπυς or \tb δπύς& ὗς& \\  
\tb δτυω or \tb δύω& ὕω& \\ 
\tb δτια or \tb δία& ἵα& \\  
\tb δως & ὡς& \\ 
\tb πδις or \tb πδίς & ἷς& \\ 
\tb πδιος or \tb πδίος& ἷος& \\ 
\tb πδιον or \tb πδίον& ἷον& \\ 
\tb πδιοί or \tb πδίοί & ἷοί& \\ 
\tb πδων or \tb πδών & ὧν& \\ 
\tb πας or \tb πάς& ᾶς& \\ 
\tb πμαον or \tb πμάον& ᾷον& \\ 
\tb παν or \tb πάν & ᾶν& \\ 
\tb πυν or \tb πύν& ῦν& \\   
\tb πυξ or \tb πύξ& ῦξ& \\  
\tb πυς or \tb πύς & ῦς& \\  
\tb πψυς or \tb πψύς& ὖς& \\  
\tb πψυν or \tb πψύν& ὖν& \\  
\tb πψην or \tb πψήν& ἦν& \\ 
\tb πδυς or \tb πδύς& ὗς& \\  
\tb πις or \tb πίς & ῖς& \\ 
\tb πιος or \tb πίος& ῖος& \\ 
\tb πιο or \tb πίο & ῖο& \\ 
\tb πια or \tb πία& ῖα& \\ 
\tb πδιά & ἷά& \\ 
\tb πδίς or \tb πδις& ἷς& \\ 
\tb πιν or \tb πίν & ῖν& \\ 
\tb πμωα or \tb πμώα& ῷα& \\ 
\tb πων or \tb πών& ῶν& \\ 
\tb πως or \tb πώς& ῶς& \\  
\tb πμας or \tb πμάς& ᾷς& \\ 
\tb πμώος or \tb πμωος& ῷος& \\ 
\tb πμώα or \tb πμωα& ῷα& \\ 
\tb πμώω or \tb πμωω& ῴω& \\ 
\tb πής or \tb πης& ῆς& \\ 
\tb πμης or \tb πμής& ῇς& \\ 
\tb μης or \tb μής& ῃς& \\ 
\tb πήρ or \tb πηρ& ῆρ& \\ 
\tb πήν or \tb πην & ῆν& \\ 
\tb ψβαν or \tb ψβάν & ἂν& \\ 
\tb ψταν or \tb ψάν& ἄν& \\ 
\tb ψτον or \tb ψόν & ὄν& \\ 
\tb ψπυς or \tb ψπύς& ὖς& \\  
\tb ψπυν or \tb ψπύν& ὖν& \\ 
\tb ψτεκ or \tb ψέκ& ἔκ& \\ 
\tb ψβην or \tb ψβήν & ἢν& \\ 
\tb ψβων or \tb ψβών& ὢν& \\ 
\tb ψτων or \tb ψών& ὤν& \\ 
\tb ψβον or \tb ψβόν& ὂν& \\ 
\tb ψτιη or \tb ψίη & ἴη& \\ 
\tb ψτιης or \tb ψίης& ἴς& \\ 
\tb ψτυω or \tb ψύω& ὔω& \\  
\tb ψπυς or \tb ψπύς& ὖς& \\ 
\tb ψτεν or \tb ψέν& ἔν& \\ 
\tb ψτην or \tb ψήν& ἤν& \\ 
\tb ψπην or \tb ψπήν& ἦν& \\ 
\tb ψις & ἰς& \\  
\tb ψεν & ἐν& \\ 
\tb ψες & ἐς& \\ 
\tb ψεπ & ἐπ& \\ 
\tb μής & ῄς& \\ 
\tb μάς or \tb μτας& ᾴς& \\ 
\tb μώα & ῴα& \\  
\tb μης & ῃς& \\ 
\tb ρρ & ῤῥ& \\
\tb και (small kai symbol) & \char"03D7 &\\
\tb Και (capital kai symbol) &  \char"03CF(most  fonts haven't it) &\\
\end{longtable}
\end{english}


\end{document}