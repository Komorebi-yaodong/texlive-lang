%%%%%%%%%%%%%%%%%%%%%%%%%%%%%%%%%%%%%%%%%%%%%%%%%%%%%%%%%%%%%%%%%
% Contents: Who contributed to this Document
% $Id: overview.tex,v 1.1.1.1 2002/02/26 10:04:21 oetiker Exp $
%%%%%%%%%%%%%%%%%%%%%%%%%%%%%%%%%%%%%%%%%%%%%%%%%%%%%%%%%%%%%%%%%

% Because this introduction is the reader's first impression, I have
% edited very heavily to try to clarify and economize the language.
% I hope you do not mind! I always try to ask "is this word needed?"
% in my own writing but I don't want to impose my style on you...
% but here I think it may be more important than the rest of the book.
% --baron

%\chapter{Lời giới thiệu}
\chapter{Lời mở đầu}
\LaTeX{} \cite{manual} là một hệ thống soạn thảo rất phù hợp với
việc tạo ra các tài liệu khoa học và toán học với chất lượng bản
in rất cao. Đồng thời, nó cũng rất phù hợp với các công việc soạn
thảo các tài liệu khác từ thư từ cho đến những cuốn sách hoàn
chỉnh. \LaTeX{} sử dụng \TeX{}~\cite{texbook} làm bộ máy định
dạng.\\

Tài liệu ngắn gọn này sẽ giới thiệu về \LaTeXe{} và nó sẽ giới
thiệu hầu hết các ứng dụng của \LaTeX. Bạn có thể tham khảo
thêm~\cite{manual,companion} để biết thêm chi tiết về hệ thống
\LaTeX{}.

\bigskip
\noindent Tài liệu này được chia làm 7 chương (6 chương được dịch
từ tài liệu gốc và 1 chương hướng dẫn sử dụng \LaTeX{} để soạn thảo tài liệu tiếng Việt):

\begin{description}
\item[Chương 1] giới thiệu cấu trúc cơ bản của các tài liệu được soạn thảo bằng \LaTeXe{}. Ngoài ra, chương này cũng giới
thiệu sơ lược về lịch sử phát triển của \LaTeX{}. Kết thúc chương,
bạn sẽ hiểu được cơ chế làm việc của \LaTeX{}. Đây sẽ là nền tảng
quan trọng mà từ đó bạn có thể kết hợp với các kiến thức ở các chương
sau để có được một cái nhìn sâu hơn về \LaTeX{}.

\item[Chương 2] đi sâu vào việc soạn thảo các tài liệu. Bạn sẽ được giới thiệu về những lệnh cơ bản, phổ biến cùng với những môi trường định dạng trong \LaTeX{}. Sau khi kết thúc
chương, bạn sẽ có thể tự soạn thảo một số kiểu tài liệu đơn giản.

\item[Chương 3] hướng dẫn cách soạn thảo các công thức bằng \LaTeX.
Chúng tôi sẽ cung cấp cho các bạn rất nhiều ví dụ minh hoạ cách sử
dụng sức mạnh này của \LaTeX{}. Chương này sẽ được kết thúc bằng
một bảng liệt kê tất cả các kí hiệu toán học được hỗ trợ trong
\LaTeX{}.

\item[Chương 4] nói về việc tạo chỉ mục, danh mục tài liệu tham
khảo và thêm hình ảnh dạng EPS vào tài liệu. Chương này cũng nói về việc
tạo một tài liệu dạng PDF với pdf\LaTeX{}, giới thiệu một số gói
mở rộng hữu dụng như XY-pic, pdfscreen, \ldots .

\item[Chương 5] hướng dẫn tạo các tập tin hình ảnh với \LaTeX{}. Bên cạnh việc sử dụng các công cụ vẽ hình bên ngoài để thiết kế hình ảnh rồi thêm vào tài liệu, bạn có thể mô tả hình ảnh và \LaTeX{} sẽ trực tiếp vẽ cho bạn.

\item[Chương 6] nói về những ``nguy hiểm tìm ẩn'' của việc thay đổi
định dạng chuẩn của \LaTeX{}. Bạn sẽ biết được những thay đổi
không nên làm vì nó sẽ khiến cho \LaTeX{} xuất ra tài liệu kết quả
không đẹp.

\item[Chương 7] hướng dẫn cài đặt và sử dụng gói \pai{VnTeX} để
soạn thảo tài liệu bằng tiếng Việt với \LaTeX{}.
\end{description}

\bigskip
\noindent Bạn nên đọc tài liệu theo thứ tự các chương bởi vì tài
liệu này không quá dài. Hãy tìm hiểu thật kỹ các ví dụ bởi lẽ chúng chứa đựng rất nhiều thông tin và sẽ được sử dụng xuyên suốt trong toàn bộ tài liệu.

\bigskip
\noindent \LaTeX{} có thể được sử dụng gần như trên mọi hệ thống máy tính, mọi hệ điều hành,
từ máy PC, Mac đến các hệ thống máy tính lớn như UNIX và VMS. Tại các mạng máy tính trong các trường đại học, bạn có thể thấy rằng \LaTeX{} đã được cài đặt sẵn. Thông tin hướng dẫn cách thức truy cập và sử dụng được cung cấp trong phần \guide. Nếu bạn gặp khó khăn trong việc sử dụng thì hãy liên hệ với người đã đưa cho bạn quyển sách này! Việc hướng dẫn cài đặt và cấu hình \LaTeX{} không thuộc vào phạm vi giới thiệu ngắn gọn của tài liệu. Ở đây, chúng tôi chỉ tập trung giới thiệu những kiến thức cơ bản  để soạn thảo tài liệu bằng \LaTeX{}.

\bigskip
\noindent Nếu bạn có nhu cầu liên quan đến \LaTeX{}, hãy tham khảo thêm tài liệu ở trang web của Comprehensive \TeX{}
Archive Network (\texttt{CTAN}). Trang chủ được đặt tại \texttt{http://www.ctan.org}. Bạn có thể tải về tất cả các gói dữ
liệu thông qua các chương trình FTP ở địa chỉ \texttt{ftp://www.ctan.org} hay rất nhiều địa chỉ liên kết phụ khác trên thế giới như \texttt{ftp://ctan.tug.org} (US), \texttt{ftp://ftp.dante.de} (Germany), \texttt{ftp://ftp.tex.ac.uk} (UK). Nếu bạn không ở các nước trên thì hãy lựa chọn địa chỉ nào gần bạn nhất.\\

\noindent Bạn sẽ thấy những phần cần tham khảo thêm ở CTAN trong suốt tài liệu này, đặc biệt là các tham chiếu đến phần mềm và tài liệu bạn có thể tải về. Thay vì phải viết toàn bộ địa chỉ URL, chúng tôi sẽ chỉ viết \texttt{CTAN:} sau đó là vị trí  trong cây thư mục ở CTAN.\\

\noindent Nếu bạn muốn sử dụng \LaTeX trên máy tính cá nhân, hãy xem qua
những thông tin ở địa chỉ \texttt{CTAN:/tex-archive/systems}.

\vspace{\stretch{1}}

\noindent Nếu bạn thấy rằng tài liệu này cần được bổ sung, thay đổi
thì hãy liên hệ với chúng tôi.

\bigskip
\begin{verse}
\contrib{Tobias Oetiker}{oetiker@ee.ethz.ch}%
\noindent{Department of Information Technology and\\ Electrical Engineering,
Swiss Federal Institute of Technology}
\end{verse}

\vspace{\stretch{1}}
\noindent Tài liệu hiện thời đang có ở địa chỉ: \\
\texttt{CTAN:/tex-archive/info/lshort}

\endinput
%
% Local Variables:
% TeX-master: "lshort2e"
% mode: latex
% coding: utf-8
% End: