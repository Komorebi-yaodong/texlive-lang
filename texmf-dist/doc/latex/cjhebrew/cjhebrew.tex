% This file is part of the cjhebrew package
%
% cjhebrew is subject to the LaTeX Project Public License (LPPL).
% A copy of the LPPL can be found in lppl.txt.
% For the most recent version of this license have a look at
%
%    http://www.latex-project.org/lppl.txt
%
\documentclass[a4paper,10pt]{article}
\usepackage[english]{babel}
\usepackage[T1]{fontenc}
\usepackage[utf8]{inputenc}
\usepackage{cjhebrew}
\usepackage{mathpazo}
\usepackage[scaled=0.9]{helvet}
\usepackage[scaled=0.8]{luximono}
\usepackage[%
        pdfauthor={Christian Justen},
        pdftitle={The CJHebrew manual},
        pdfborder={0 0 0},
        bookmarksnumbered=true,
        pdfpagemode=None,
        pdfstartview=FitH,
        ]%
    {hyperref}
\usepackage{booktabs}

\frenchspacing

\def\cjh{\textsf{cjhebrew}}

\def\cjhp{\texttt{cjhebrew}}

\def\!#1!{\texttt{#1}}

\newcommand{\bs}{\textbackslash}

\def\showmacro#1{\marginpar{\texttt{{\bs}#1}}}

\def\showother#1{\marginpar{\texttt{#1}}}

\def\cjhebversion{0.2a}

\def\hinweis#1{\mbox{}\marginpar{\textsf{#1}}}

\def\command#1{\medskip\par\texttt{#1}\medskip\par}

\newcommand{\eTeX}{$\varepsilon$-\TeX}
\newcommand{\eLaTeX}{$\varepsilon$-\LaTeX}

\begin{document}

\title{\cjh%
    \footnote{\cjh\ is subject to the \textit{\LaTeX\ Project Public
    License}. The most recent version of this license can be found at
    \href{http://www.latex-project.org/lppl.txt}{\!www.latex-project.org/lppl.txt!}.}%
}

\author{Christian Justen\\
\href{mailto:mail@christian-justen.de}{{\small\!mail@christian-justen.de!}}}

\date{March 6\textsuperscript{th}, 2017 -- Version \cjhebversion}

\maketitle


\section{Overview}

\cjh\ is a package which allows the typesetting of Hebrew text in
\LaTeX\ documents. Hebrew text can be vocalised, also a few
accents are available. \cjh\ makes it easy to insert Hebrew words,
sentences or paragraphs into non-Hebrew text; so \cjh\ is quite
appropriate e.\,g. for theological papers.

When running \LaTeX, \cjh\ uses the extensions of \eTeX\ in order to
typeset Hebrew text from right to left.  Most modern \TeX\ systems use
\eLaTeX\ by default when you run \LaTeX, so there should be nothing
special you have to do (unless, of course, you fiddled with your
system).  This version of \cjh\ can also be used with lua\LaTeX.


\section{Installation}

\cjh\ is part of most modern \TeX\ distributions.  Please, use your
distribution's package manager to install \cjh.  If no package manager
is available, then have a look at your distribution's documentation,
as it should tell you how to install packages manually.



\section{Usage}

To use \cjh\ simply put \verb!\usepackage{cjhebrew}! in the
preamble of your document.

\cjh\ provides the text font command
\showmacro{textcjheb}\verb!\textcjheb! which switches to the hebrew
font, but does not change the direction of typesetting. Thus, if you
type \verb+\textcjheb{'bgd}+, the result will be \textcjheb{'bgd},
which is in most cases not what you want. Instead you will normally
use the command \showmacro{cjRL}\verb+\cjRL+, that also switches to
the right direction of typesetting. The input \verb+\cjRL{'bgd}+ will
have as result the ouput \cjRL{'bgd}. There is also an even shorter
form: \showmacro{<>}\verb+\<>+; so, you can type \verb+\<'bgd>+
instead of \verb+\cjRL{'bgd}+.

If you want to typeset a complete passage of Hebrew text, you
ought to use the \showother{cjhebrew}\!cjhebrew! environment.

Finally, \cjh\ provides also the \verb+\cjLR+\showmacro{cjLR}
command, which switches (inside a Hebrew piece of text) back to
the ``normal'' direction of typesetting. Be careful: this command
does not switch to a non-Hebrew font!

\subsection{The consonants}

\begin{table}

\centering

\begin{tabular}{ccccccccccccccc}

\toprule\midrule

\<'> & \<b> & \<g> & \<d> & \<h> & \<w> & \<z> & \<.h> &
\<.t> & \<y> & \<k|> & \<k> & \<l> & \<m|> & \<m>\\

\!'! & \!b! & \!g! & \!d! & \!h! & \!w! & \!z! & \!.h! & \!.t! &
\!y! & \!k! & \!K! & \!l! & \!m! & \!M!\\

\midrule

\<n|> & \<n> & \<s> & \<`> & \<p|> & \<p> & \<.s|> & \<.s> & \<q>
& \<r> & \</s>
& \<,s> & \<+s> & \<t> &\\

\!n! & \!N! & \!s! & \!`! & \!p! & \!P! & \!.s! & \!.S! & \!q! &
\!r! & \!/s!
& \!,s! & \!+s! & \!t! &\\

\midrule\bottomrule

\end{tabular}

\medskip

\textit{Note: \!'!~= semicolon, \!`!~= grave accent}

\caption{Coding of the consonants}

\label{coding:consonants}

\end{table}

How the Hebrew consonants are coded in your input file, is shown
in table~\ref{coding:consonants}. Normally the final letters are
used automatically; \verb+\<mlk>+ will become \<mlk>. Sometimes it
is necessary to use final letters in places where they will not be
set automatically, e.\,g. in the middle of a word. To do this you
either use the coding according to table~\ref{coding:consonants}
or you put an exclamation mark (\showother{!}\verb+!+) after the
consonant; alternatively you can use the
\showmacro{endofword}\verb+\endofword+ command. For example, a
\textit{final mem} could be achieved by typing \verb+\<M>+,
\verb+\<m!>+ or \verb+\<m\endofword>+.

On the other hand, sometimes you will not want this automatic
replacement. In these cases you put \showother{|}\!|! after the
consonant or use the \showmacro{zeronojoin}\verb+\zeronojoin+
command. Both \verb+\<m|>+ and \verb+\<m\zeronojoin>+ will give
you a normal \textit{mem}.

\subsection{The vowels}

\begin{table}

\def\dc{\verb+\dottedcircle+}

\centering

\begin{tabular}{ccccccccccccccc}

\toprule\midrule

\<\dottedcircle i> & \<\dottedcircle e> & \<\dottedcircle E> &
\<\dottedcircle E:> & \<\dottedcircle a> & \</a\dottedcircle> &
\<\dottedcircle a:> & \<\dottedcircle A> & \<\dottedcircle A:> &
\<\dottedcircle o> & \<\dottedcircle u> &
\<\dottedcircle *> & \<\dottedcircle :> & \<O> & \<U>\\

\!i! & \!e! & \!E! & \!E:! & \!a! & \!/a! & \!a:! & \!A! & \!A:! &
\!o! & \!u! & \!*! & \!:! & \!O! / \!wo! & \!U!
/ \!w*!\\

\midrule

\<;> & \<--> & \<\dottedcircle> \\
\!;! & \!-\/-! & \multicolumn{6}{l}{\texttt{\bs dottedcircle}}\\

\midrule\bottomrule

\end{tabular}

\caption{Coding of the vowels, accents and symbols}

\label{coding:vowels}

\end{table}

How to code the vowels in your input file is shown in
table~\ref{coding:vowels}. The vowels have to be typed
\textit{after} the consonant they belong to (for example
\<'E:lohiym> is coded as \verb+\<'E:lohiym>+). The only exception
is the \textit{pata\d{h} furtivum} as in \<rU/a.h>
(\verb+\<rU/a.h>+). Always use \verb+*+ for \textit{dage\v{s}};
\cjh\ does not distinguish between \textit{dage\v{s} lene} and
\textit{dage\v{s} forte}. The \textit{dage\v{s}} has to follow its
consonant \textit{immediately} (\verb+\<b*:>+ becomes \<b*:>),
otherwise you will get a wrong result (\verb+\<b:*>+ becomes
\<b:*>).

\subsection{More accents and symbols}

There are some more accents and symbols available, as shown in
table~\ref{coding:vowels}. I hope to add some more in the future.


\section{An example}

Here is the beginning of the Bible:

\bigskip

\begin{cjhebrew}

b*:re'+siyt b*ArA' 'E:lohiym 'et ha+s*Amayim w:'et hA'ArE.s;
w:hA'ArE.s hAy:tAh tohU wAbohU w:.ho+sEk: `al--p*:ney t:hOm
w:rU/a.h 'E:lohiym m:ra.hEpEt `al--p*:ney ham*Ayim;

\end{cjhebrew}

\bigskip

\bgroup

\raggedright

\verb+\begin{cjhebrew}+

\!b*:re'+siyt b*ArA' 'E:lohiym 'et ha+s*Amayim w:'et hA'ArE.s;
w:hA'ArE.s hAy:tAh tohU wAbohU w:.ho+sEk: `al-\/-p*:ney t:hOm
w:rU/a.h 'E:lohiym m:ra.hEpEt `al-\/-p*:ney ham*Ayim;!

\verb+\end{cjhebrew}+

\egroup

\section{What is new in this version?}

\subsection*{v0.2a}

\begin{itemize}
\item Fixed a stupid typo which created a warning.
\end{itemize}

\subsection*{v 0.2}

\begin{itemize}
\item \cjh\ can now be used with lua\LaTeX.  The neccessary code was
  provided by Axel Kielhorn.
\end{itemize}

\subsection*{v 0.1a}

\begin{itemize}

    \item A typo in \!cjhebrew.sty! was corrected. It would cause
    trouble if you used a Hebrew font in 7pt size or smaller.

\end{itemize}


\subsection*{v 0.1}

\begin{itemize}

    \item You can use \cjh\ now, even if \eLaTeX\ is not available
        (but the  Hebrew text will be typeset in the ``wrong''
        direction, i.\,e. from left to right).%
            \footnote{This feature was suggested by Malte
            Rosenau.}

    \item The documentation was rewritten. Instead of a German and
        an English documentation there is now only this English
        manual.

    \item Some bugs in \!cjhebrew.sty! are fixed.\footnote{Thanks to Walter Schmidt.}

    \item A problem with the letter ``qof'' is fixed (\<q|A:q|E:q|a:>
        does not look \textit{that} good \ldots\ \<qA:qE:qa:> is much better).

    \item There are also some changes regarding the font names and
        the encoding. If you use an older version of
        \cjh\ make sure that all old files are removed before you
        install the new version!

\end{itemize}

Probably, some new bugs have crept into any new version. Please do
report them as you discover them!


\section{Finally}

The version number of \cjh\ is \cjhebversion, it is far from being
finished. Especially the fonts still need a lot of work; many
letters look rather imperfect. If you have any ideas how to
improve \cjh, please do send me an email.

\end{document}

% Local Variables:
% coding: utf-8-unix
% End: