\documentclass[%
candidate,   % тип документа
%natbib,      % использовать пакет natbib для "сжатия" цитирований
subf,        % использовать пакет subcaption для вложенной нумерации рисунков
href,        % использовать пакет hyperref для создания гиперссылок
colorlinks,  % цветные гиперссылки
%fixint,     % включить прямые знаки интегралов
%classified, % гриф секретности
%facsimile,  % отображать факсимиле диссертанта
]{disser}

\usepackage[
  a4paper, mag=1000,
  left=2.5cm, right=1cm, top=2cm, bottom=2cm, headsep=0.7cm, footskip=1cm
]{geometry}

\usepackage[intlimits]{amsmath}
\usepackage{amssymb,amsfonts}

\usepackage[T2A]{fontenc}
\usepackage[utf8]{inputenc}
\usepackage[english,russian]{babel}
\ifpdf\usepackage{epstopdf}\fi
\usepackage[autostyle]{csquotes}

% Список сокращений и условных обозначений
\usepackage[intoc,nocfg,russian]{nomencl}
\newcommand{\nomencl}[2]{#1 --- #2\nomenclature{#1}{#2}}
\setlength{\nomlabelwidth}{3em}
\setlength{\nomitemsep}{-\parsep}
\renewcommand{\nomlabel}[1]{#1 ---}
\makenomenclature

% Шрифт Times в тексте как основной
%\usepackage{tempora}
% альтернативный пакет из дистрибутива TeX Live
%\usepackage{cyrtimes}

% Шрифт Times в формулах как основной
%\usepackage[varg,cmbraces,cmintegrals]{newtxmath}
% альтернативный пакет
%\usepackage[subscriptcorrection,nofontinfo]{mtpro2}

% Плавающие рисунки "в оборку".
\usepackage{wrapfig}

\usepackage[style=gost-numeric,
  backend=biber,
  language=auto,
  hyperref=auto,
  autolang=other,
  sorting=none
]{biblatex}

\addbibresource{thesis.bib}

% Номера страниц снизу и по центру
%\pagestyle{footcenter}
%\chapterpagestyle{footcenter}

% Точка с запятой в качестве разделителя между номерами цитирований
%\setcitestyle{semicolon}

% Ссылки на работы соискателя включаются в общий список литературы
\let\citeown=\cite

% Использовать полужирное начертание для векторов
\let\vec=\mathbf

% Путь к файлам с иллюстрациями
\graphicspath{{fig/}}

\begin{document}

% Переопределение стандартных заголовков
%\def\contentsname{Содержание}
%\def\conclusionname{Выводы}
%\def\bibname{Литература}

% Включение файла с общим текстом диссертации и автореферата
% (текст титульного листа и характеристика работы).
% ����� ���� ���������� ����� ����������� � ������������
\institution{�������� �����������}

\topic{���� �����������}

\author{��� ������}

\specnum{01.04.05}
\spec{������}
%\specsndnum{01.04.07}
%\specsnd{������ ����������������� ���������}

\scon{��� ������������}
\sconstatus{�.~�.-�.~�., ����.}
%\sconsnd{��� ������� ������������}
%\sconsndstatus{�.~�.-�.~�., ����.}

\city{�����-���������}
\date{\number\year}

% ����� ������� ������������ � �����������
\mkcommonsect{actuality}{������������ ���� ������������.}{%
����� �� ������������. ������~\cite{Yoffe_1993_AP_42_173}.
}

\mkcommonsect{development}{������� ��������������� ���� ������������.}{
����� � ������� ��������������� ����.
}

\mkcommonsect{objective}{���� � ������ ��������������� ������:}{%
������ �����.

��� ���������� ������������ ����� ���� ������ ��������� ������:
}

\mkcommonsect{novelty}{������� �������.}{%
����� � �������.
}

\mkcommonsect{value}{������������� � ������������ ����������.}{%
����������, ���������� � �����������, ����� ���� ������������ ��� ...
}

\mkcommonsect{methods}{����������� � ������ ������������.}{%
����� � ������� ������������.
}

\mkcommonsect{results}{���������, ��������� �� ������:}{%
����� � ���������� � �����������.
}

\mkcommonsect{approbation}{������� ������������� � ��������� �����������.}{%
�������� ���������� ����������� ������������� �� ��������� ������������:
}

\mkcommonsect{pub}{����������.}{%
��������� ����������� ������������ � $N$ �������� �������, �� ��� $n_1$
������ � ������������� ��������~\citeown{Ivanov_1999_Journal_17_173,
Petrov_2001_Journal_23_12321,Sidorov_2002_Journal_32_1531}, $n_2$ ������ �
��������� ������ ����������� � $n_3$ ������� ��������.
}

\mkcommonsect{contrib}{������ ����� ������.}{%
���������� ����������� � �������� ���������, ��������� �� ������, �������� ������������ ����� ������ � �������������� ������.
���������� � ���������� ���������� ����������� ����������� ��������� � ����������, ������ ����� ����������� ��� ������������. ��� �������������� � ����������� ���������� �������� ����� �������.
}

\mkcommonsect{struct}{��������� � ����� �����������.}{%
����������� ������� �� ��������, ������ ����������, $n$ ����, ���������� � ������������.
����� ����� ����������� $P$ �������, �� ��� $p_1$ �������� ������, ������� $f$ ��������.
������������ �������� $B$ ������������ �� $p_2$ ���������.
}


% номер копии для грифа секретности
%\copynum{1}
% класс доступа
%\classlabel{Для служебного пользования}

% номер УДК
\libcatnum{12345}

\title{ДИССЕРТАЦИЯ\\
на соискание ученой степени\\
кандидата физико-математических наук}

\maketitle

%%
%% Titlepage in English
%%
%
%\institution{Name of Organization}
%
%\title{PhD Thesis}
%
%% Topic
%\topic{Dummy Title}
%
%% Author
%\author{Author's Name}
%
%\specnum{01.04.05}
%\spec{Optics}
%
%%\specsndnum{01.04.07}
%%\specsnd{Condensed matter physics}
%
%\sa{I.\,I.~Ivanov}
%\sastatus{Professor}
%%\sasnd{P.\,P.~Petrov}
%%\sasndstatus{Professor}
%
%% Scientific consultant
%%\scon{B.\,B.~Baranov}
%%\sconstatus{Professor}
%
%% City & Year
%\city{Saint Petersburg}
%\date{\number\year}
%
%\maketitle[en]

% Содержание
\tableofcontents

% Введение
\intro

%
% Используемые далее команды определяются в файле common.tex.
%

% Актуальность работы
\actualitysection
\actualitytext

% Цели и задачи диссертационной работы
\objectivesection
\objectivetext

% Научная новизна
\noveltysection
\noveltytext

% Теоретическая и практическая значимость
\valuesection
\valuetext

% Результаты и положения, выносимые на защиту
\resultssection
\resultstext

% Степень достоверности и апробация результатов
\approbationsection
\approbationtext

% Публикации
\pubsection
\pubtext

% Личный вклад автора
\contribsection
\contribtext

% Структура и объем диссертации
\structsection
\structtext


% Обзор литературы
%\review


% Основная часть
%% Глава 1
\chapter{Название главы}
\section{Название секции}

Внутритекстовая формула $\frac{1}{\epsilon^*}=\frac{1}{\epsilon_\infty}-\frac{1}{\epsilon_0}$.
\nomenclature{$\epsilon_\infty$}{высокочастотная диэлектрическая проницаемость}
\nomenclature{$\epsilon_0$}{статическая диэлектрическая проницаемость}
Внутритекстовая формула в стиле выделенной $\dfrac{1}{\epsilon_\infty}$.
Ссылки на литературу~\cite{Yoffe_1993_AP_42_173,Efros_1982_FTP_16_7_1209,%
Anselm_1978,Segall_1968,Agranovich_1983,InP,Mishchenko_1996,Skvortsov_2008,%
Perelman_2003_math:0307245,Nielsen_2010_1006.2735,patent1,patent2}.
Ссылка на формулу~\eqref{e:Coulomb}
\begin{equation}\label{e:Coulomb}
  \frac{1}{|\vec r_1 - \vec r_2|} =
  4\pi \int \frac{d^3 q}{(2\pi)^3}\,
  \frac{e^{i\vec q(\vec r_1 - \vec r_2)}}{q^2},
\end{equation}
где \nomencl{$\vec r_i$}{координата $i$-й частицы}.

\section{Выводы к первой главе}
%% Глава 2
%\input{2}

% Заключение
\conclusion


% Список сокращений и условных обозначений
\printnomenclature

% Словарь терминов
%% ������� ��������
\dict

\textbf{������} "--- �����������.


% Список литературы
\printbibliography[heading=bibintoc]

% Список иллюстративного материала
%\listoffigures

% Приложения
%\appendix
%\chapter{�������� ����������}


\end{document}
