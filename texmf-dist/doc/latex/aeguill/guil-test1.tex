\documentclass[a4paper]{article}
\usepackage[T1]{fontenc}
\usepackage[latin1]{inputenc}
\usepackage[pl]{aeguill}
%\usepackage{french}
%\usepackage[frenchb]{babel}

\begin{document}
\title{The \texttt{aeguill} package}
\author{Denis Roegel\\
        \texttt{roegel@loria.fr}}
\date{2 August 2003}
\maketitle

When using the T1 encoding and the \texttt{ae} fonts, one is
faced with the problem that the \texttt{ae} fonts do not contain
the guillemets, which are necessary for the French.

This package proposes a solution to this problem.

Example of french guillemets: �~test~�.  
  % the input << test >> works only with \usepackage{french}

The guillemets in the previous example 
are taken from the \texttt{plr} fonts (Polish CMR),
which are similar to EC guillemets and
of which there is a Type~1 version. Hence, you will get
a nice output with \texttt{pdflatex}.

There are five options to the \texttt{aeguill} package:

\begin{itemize}
\item `\texttt{lm}' (default): with this, the guillemets
are taken from the \texttt{lmr} fonts.
\item `\texttt{pl}': with this, the guillemets
are taken from the \texttt{plr} fonts.
\item `\texttt{cyr}': with this, the guillemets
are taken from the \texttt{wncyr} fonts.
\item `\texttt{cm}': with this, the guillemets are built with the
\texttt{lasy} fonts.
\item `\texttt{ec}': with this, the guillemets are those in the 
\texttt{ec} fonts.
\end{itemize}


\textbf{Caution:}

\begin{itemize}
\item If the \texttt{babel} package is used with the \texttt{french} option, 
      do not use \verb|\CyrillicGuillemets|.
\item If the \texttt{french} package is used, 
      you may write ``\verb|<<|'' and ``\verb|>>|''
      instead of ``\verb|�|'' and ``\verb|�|'' (i.e., use 7-bit encoding
      instead of 8-bit encoding).
\end{itemize}

\end{document}