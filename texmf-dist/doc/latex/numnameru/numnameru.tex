\documentclass[pagesize=auto, parskip=full, fontsize=14pt, DIV=9]{scrartcl}

\usepackage{fixltx2e}
\usepackage{etex}
\usepackage{xspace}
\usepackage{cmap}
\usepackage[T1,T2A]{fontenc}
\usepackage[utf8]{inputenc}
\usepackage[english,main=russian]{babel}
\usepackage{textcomp}
\usepackage{microtype}
\usepackage[pdflang=russian,%
unicode,%
pdfencoding=unicode%
]{hyperref}

\usepackage{numnameru}

\newcommand*{\pkg}[1]{\textsf{#1}}
\newcommand*{\cls}[1]{\textsf{#1}}

\addtokomafont{title}{\rmfamily}

\title{The \pkg{numnameru} package}
\author{Vit}
\date{2017/07/26}


\begin{document}
\maketitle
\par This package converts a numerical number to the russian spelled out
name of the number. For example, 1~$\to$~\numnameru{1}, 2~$\to$~\numnameru{2}, 12~$\to$~\numnameru{12}.%
\par Attention!%
\par Max number to converts: 2\,147\,483\,647~$\to$~\numnameru{2147483647}.%
\par Этот пакет позволяет преобразовать число в число прописью на русском языке.
Пример: 1~$\to$~\numnameru{1}, 2~$\to$~\numnameru{2}, 12~$\to$~\numnameru{12}.%
\par Внимание!%
\par Максимальное число для преобразования: 2\,147\,483\,647~$\to$~\numnameru{2147483647}.%
\par This code was based on `numname.sty' by Peter R. Wilson.%

\end{document}

%%% Local Variables:
%%% mode: latex
%%% coding: utf-8
%%% compile-command: "pdflatex numnameru.tex"
%%% TeX-master: t
%%% End:
