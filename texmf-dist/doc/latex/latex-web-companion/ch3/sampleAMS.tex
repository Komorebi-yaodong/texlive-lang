\documentclass[a4paper,twoside]{article}
\usepackage{html}
\usepackage{amsmath}

\renewcommand{\d}{\partial}\providecommand{\bm}[1]{\mathbf{#1}}
\providecommand{\Range}{\mathcal{R}}\providecommand{\Ker}{\mathcal{N}}
\providecommand{\Quat}{\vec{\mathbf{Q}}}

\newcommand{\StAndrews}{\htmlurl{http://www-groups.dcs.st-and.ac.uk/~history}}%
\newcommand{\Pythagorians}{\htmladdnormallink
 {Pythagorians}{\StAndrews/Mathematicians/Pythagoras.html}}
\newcommand{\Fermat}{\htmladdnormallink
 {Fermat, c.1637}{\StAndrews/HistTopics/Fermat's_last_theorem.html}}
\newcommand{\Wiles}{\htmladdnormallink
 {Wiles, 1995}{\htmlurl{http://www.pbs.org:80/wgbh/nova/proof}}}

\begin{document}
\htmlhead[center]{section}{Math examples}
\begin{eqnarray}
 \phi(\lambda) & = &  \frac{1} {2 \pi i}\int^{c+i\infty}_{c-i\infty}
  \exp \left( u \ln u + \lambda u \right ) du \hspace{1cm}\mbox{for } c \geq 0 \\
 \lambda       & = &  \frac{\epsilon  -\bar{\epsilon} }{\xi}
                   - \gamma' - \beta^2 - \ln \frac{\xi} {E_{\rm max}}          \\
 \gamma        & = &  0.577215\dots \mathrm{\hspace{5mm}(Euler's\ constant)}   \\
 \gamma'       & = &  0.422784\dots = 1 - \gamma                               \\
   \epsilon , \bar{\epsilon} & = & \mbox{actual/average energy loss}
\end{eqnarray}

Since~\eqref{eqn:stress-sr} or~\eqref{gdef} should hold for arbitrary $\delta\bm{c}$%
-vectors, it is clear that $\Ker(A) = \Range(B)$ and that when $y=B(x)$ one has...\\
...the \Pythagorians{} knew infinitely many solutions in integers to $a^2+b^2=c^2$. 
That no non-trivial integer solutions exist for $a^n+b^n=c^n$ with integers $n>2$ has long 
been suspected (\Fermat). Only during the current decade has this been proved (\Wiles).

\begin{eqnarray}\label{eqn:stress-sr}
 V \bm{\pi}^{sr} & = & \left<  \sum_i M_i \bm{V}_i \bm{V}_i
  + \sum_i \sum_{j>i} \bm{R}_{ij} \bm{F}_{ij}\right> \\ \nonumber
                 & = & \left< \sum_i M_i \bm{V}_i \bm{V}_i
  + \sum_{i}\sum_{j>i}\sum_\alpha\sum_\beta \bm{r}_{i\alpha j\beta}\bm{f}_{i\alpha j\beta}
  - \sum_i \sum_\alpha \bm{p}_{i\alpha} \bm{f}_{i\alpha}   \right>
\end{eqnarray}

\begin{subequations}\label{bgdefs}
\begin{align} B_{ij}^\alpha     & =
        \left(B_{ij}^\alpha\right)_0 + \left(B_{ij}^\alpha\right)_a \label{bdef}  \\
  \left(B_{ij}^\alpha\right)_0  & = \frac{1}{2}\left(\frac{\d N_i^\alpha}{\d X_j}
        + \frac{\d N_j^\alpha} {\d X_i} \right)                     \label{b0def} \\
  \left(B_{ij}^\alpha\right)_a  & = H_{ij}^{\alpha \beta} a^\beta   \label{budef} \\
  H_{ij}^{\alpha \beta}         & = 
    \frac{1}{2}\left( \frac{\d N_k^\alpha}{\d X_i} \frac{\d N_k^\beta}{\d X_j} 
    + \frac{\d N_k^\beta}{\d X_i} \frac{\d N_k^\alpha}{\d X_j} \right)  \label{gdef}
\end{align}
\end{subequations}
\end{document}
