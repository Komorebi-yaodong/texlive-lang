\begin{htmlonly}
\usepackage{makeidx,html}
\usepackage{francais}
\input ex32.ptr
\end{htmlonly}
\newcommand{\Lcs}[1]{\texttt{\symbol{'134}#1}}
\startdocument
\index{section!tableau}

Le \hyperref{tableau}{tableau }{}{tab-exa}
montre l'utilisation de l'environnement \texttt{table}.
\begin{table}[h]
\centering
 \begin{tabular}{cccccc}
  \Lcs{primo}  & \primo & \Lcs{secundo} & \secundo
  & \Lcs{tertio} & \tertio \\
  \Lcs{quatro} & \quatro& 1\Lcs{ier}    & 1\ier 
  & 1\Lcs{iere}  & 1\iere  \\
  \Lcs{fprimo)}&\fprimo)& \Lcs{No} 10   & \No 10 
  & \Lcs{no} 15  & \no 15  \\
  \Lcs{og} a \Lcs{fg}&\og a \fg&3\Lcs{ieme}&3\ieme
  & 10\Lcs{iemes}& 10\iemes 
 \end{tabular}
\caption{Quelques commandes de l'option \texttt{french} 
         de \texttt{babel}}\label{tab-exa}
\index{tableau}
\end{table}
