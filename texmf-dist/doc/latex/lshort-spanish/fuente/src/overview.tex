%%%%%%%%%%%%%%%%%%%%%%%%%%%%%%%%%%%%%%%%%%%%%%%%%%%%%%%%%%%%%%%%%
% Contents: Who contributed to this Document
%%%%%%%%%%%%%%%%%%%%%%%%%%%%%%%%%%%%%%%%%%%%%%%%%%%%%%%%%%%%%%%%%

% Because this introduction is the reader's first impression, I have
% edited very heavily to try to clarify and economize the language.
% I hope you do not mind! I always try to ask "is this word needed?"
% in my own writing but I don't want to impose my style on you... 
% but here I think it may be more important than the rest of the book.
% --baron

\chapter{Prefacio}

\LaTeX{} \cite{manual} es un sistema de composición muy adecuado para realizar documentos científicos y matemáticos de alta calidad tipográfica.  Es también adecuado para producir documentos de cualquier otro tipo, desde simples cartas a libros enteros.  \LaTeX{} emplea \TeX{} \cite{texbook} como motor de formato.

Esta breve introducción describe \LaTeXe{} y debería bastar para la mayoría de las aplicaciones de \LaTeX. Consulte~\cite{manual,companion} para una descripción exhaustiva del sistema \LaTeX{}.

%\bigskip
Esta introducción se divide en 6 capítulos:
\begin{description}
  \item[El capítulo 1] trata sobre la estructura básica de documentos \LaTeXe{}.  Aprenderá un poco sobre la historia de \LaTeX{}. Tras leer este capítulo, debería tener un conocimiento somero de cómo trabaja \LaTeX{}.
  \item[El capítulo 2] profundiza en los detalles como componer los documentos.  Explica la mayoría de las órdenes y entornos esenciales de \LaTeX{}.  Tras leer este capítulo, debería ser capaz de escribir sus primeros documentos.
  \item[El capítulo 3] explica cómo componer fórmulas con \LaTeX.  Con muchos ejemplos se muestra cómo usar uno de los puntos fuertes de \LaTeX{}.  Al final del capítulo hay tablas con todos los símbolos matemáticos disponibles en \LaTeX{}.
  \item[El capítulo 4] explica los índices, generación de bibliografías e inclusión de gráficos EPS.  Presenta la creación de documentos PDF mediante pdf\LaTeX{} y varios paquetes adicionales interesantes.
  \item[El capítulo 5] muestra cómo usar \LaTeX{} para crear gráficos. En lugar de dibujar una figura con algún progama gráfico, grabarla en un \filenomo{} y después incluirla en \LaTeX{}, podrá describir directamente el dibujo \LaTeX{} lo dibujará por usted.
  \item[El capítulo 6] contiene información potencialmente peligrosa sobre cómo alterar la presentación normal del documento producido con \LaTeX{}.  Le indicará cómo cambiar cosas de forma que la salida hermosa de \LaTeX{} se volverá horrible o deslumbrante, según sus habilidades.
\end{description}
%\bigskip
%
Es importante leer los capítulos en orden ---el libro no es tan largo, después de todo---.  Asegúrese de leer con cuidado los ejemplos, porque mucha información está en los ejemplos dispersos a lo largo del libro.

%\bigskip
%
\LaTeX{} está disponible para la mayor parte de \computernomo{}es, desde PC y Mac a grandes sistemas UNIX y VMS.  En muchos \computernomo{}es universitarios encontrará una instalación de \LaTeX{} disponible y lista para usar.  Habrá información sobre cómo acceder la instalación local de \LaTeX{} en la \guide.  Si tiene problemas para comenzar, pregunte a la persona que le proporcionó este libro.  El objetivo de este documento \emph{no} es contarle cómo instalar y configurar un sistema \LaTeX{}, sino enseñarle cómo escribir documentos para que pueda procesarlos con~\LaTeX{}.

%\bigskip
%
Si necesita conseguir cualquier material relativo a \LaTeX{}, eche un vistazo a las páginas de la Red Integral de \Filenomo{}s \TeX{} (\texttt{CTAN}).  La página de internet se encuentra en \texttt{http://www.ctan.org}.  Todos los paquetes pueden conseguirse desde la dirección ftp \texttt{ftp://www.ctan.org} y sus espejos en todo el mundo.

Encontrará otras referencias a CTAN a lo largo del libro, especialmente indicaciones a programas y documentos que podría querer descargar.  En lugar de escribir direcciones completas, sólo escribí \texttt{CTAN:} seguido del lugar dentro de CTAN al que debería acceder.

Si quiere ejecutar \LaTeX{} es su propio \computernomo{}, busque qué hay disponible en \CTAN|systems|.

\vspace{\stretch{1}}
%
Si se le ocurre qué puede añadirse, eliminarse o cambiarse en este documento, por favor hágamelo saber. Estoy especialmente interesado en opiniones de novatos en \LaTeX{} sobre qué partes de esta intro son fáciles de entender y cuáles deberían explicarse mejor.

\bigskip
\begin{verse}
\contrib{Tobias Oetiker}{oetiker@ee.ethz.ch}%
\noindent{Departmento de Tecnología de la Información e\\ Ingeniería Eléctrica,\\ Instituto Federal Suizo de Tecnología}
\end{verse}
\vspace{\stretch{1}}
La versión actual de este documento está disponible en\\
\CTAN|info/lshort|

\endinput