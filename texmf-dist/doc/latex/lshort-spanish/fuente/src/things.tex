%%%%%%%%%%%%%%%%%%%%%%%%%%%%%%%%%%%%%%%%%%%%%%%%%%%%%%%%%%%%%%%%%
% Contents: Things you need to know
%%%%%%%%%%%%%%%%%%%%%%%%%%%%%%%%%%%%%%%%%%%%%%%%%%%%%%%%%%%%%%%%%
 
\chapter{Cosas que debe saber}
\begin{intro}
La primera parte de este capítulo presenta un vistazo breve de la filosofía e historia de \LaTeXe.  La segunda parte se centra en la estructura básica de un documento \LaTeX{}.  Tras leer este capítulo, debería tener un conocimiento básico de cómo funciona \LaTeX{}, que necesitará para entender el resto de este libro.
\end{intro}

\section{El nombre del hombre}
\subsection{\TeX}
 
\TeX{} es un programa de \computernomo{} creado por \index{Knuth, Donald E.}Donald E. Knuth \cite{texbook}. Sirve para componer texto y fórmulas matemáticas.  Knuth empezó a escribir el motor de composición \TeX{} en 1977 para investigar el potencial de los equipos de impresión digital que estaban empezando a usarse en la industria tipográfica en aquel tiempo; en concreto tenía la esperanza de poder revertir la tendencia de calidad tipográfica en declive que él vio afectar a sus propios libros y artículos.  El programa \TeX{} tal como lo conocemos hoy día fue publicado en 1982, con algunas sutiles mejoras añadidas en 1989 para soportar caracteres de 8 bites y múltiples lenguajes.  \TeX{} tiene fama de ser muy estable, muy portable y prácticamente sin errores.  El número de versión de \TeX{} converge hacia $\pi$ y es ahora $3.1415926$.
                                                                       
\TeX{} se pronuncia ``Tej''.  La ``j'' surge del alfabeto griego donde X es la letra ``j'' o ``ji''.  \TeX{} es también la primera sílaba de la palabra griega $\tau\epsilon\xi\nu\eta$ (arte).  En un entorno ASCII, \TeX{} se convierte en \texttt{TeX}.

\subsection{\LaTeX}
 
\LaTeX{} es un paquete de macros que permite a los autores componer e imprimir su trabajo con la mayor calidad tipográfica posible, usando un formato profesional predefinido. \LaTeX{} fue escrito originalmente por \index{Lamport, Leslie}Leslie Lamport~\cite{manual}.  Emplea el formateador \TeX{} como motor de composición.  Actualmente \index{\LaTeX team} un equipo de programadores da mantenimiento a \LaTeX{}.

\LaTeX{} se pronuncia ``Látej''.  Si quiere referirse a \LaTeX{} en un entorno \texttt{ASCII}, escriba \texttt{LaTeX}. \LaTeXe{} se pronuncia ``Látej dos e'' y se escribe \texttt{LaTeX2e}.


\section{Lo básico}
 
\subsection{Autor, maquetador y compositor}\label{sec:basics}

Para publicar algo, los autores dan su manuscrito mecanografiado a una editorial.  Uno de sus maquetadores decide el aspecto del documento (anchura de columna, tipografías, espacio ante y tras cabeceras, \ldots).  El maquetador escribe sus instrucciones en el manuscrito y luego se lo da al compositor o cajista, quien compone el libro siguiendo esas instrucciones.

Un maquetador humano suele interpretar qué pretendía el autor mientras escribía el manuscrito.  Decide sobre las cabeceras de los capítulos, las citas, los ejemplos, las fórmulas, etc.\ basándose es su conocimiento profesional y en el contenido del manuscrito.

En un entorno \LaTeX{}, \LaTeX{} representa el papel del maquetador y usa \TeX{} como su compositor.  Pero \LaTeX{} es ``sólo'' un programa, y por tanto necesita más supervisión.  El autor tiene que proporcionar información adicional para describir la estructura lógica de su trabajo.  Tal información se escribe entre el texto como ``órdenes \LaTeX{}''.

Esto es bastante diferente del enfoque visual o \wi{WYSIWYG}\footnote{What you see is what you get: lo que ve es lo que consigue.} que sigue la mayoría de los procesadores de texto modernos, como \emph{Abiword} u \emph{Open/Libre\-Office Writer}.  Con estos programas, los autores especifican el aspecto del documento interactivamente mientras escriben texto en el \computernomo{}.  Así pueden ver en la pantalla cómo aparecerá el trabajo final cuando se imprima.

Cuando use \LaTeX{} no suele ser posible ver el aspecto final del texto mientras lo escribe, pero tal aspecto puede verse en pantalla tras procesar el \filenomo{} mediante \LaTeX.  Entonces pueden hacerse correcciones antes de enviar el documento a la impresora para tener una copia en papel.

\subsection{Maquetación}

La maquetación (diseño tipográfico) es un arte.  Los autores sin habilidad a menudo cometen errores de formato al suponer que maquetar es mayormente una cuestión de estética ---``Si un documento luce bien artísticamente, está bien diseñado''---.  Pero como un documento se escribe para ser leído y no colgado en una galería de arte, su legibilidad es mucho más importante que su aspecto. Ejemplos:
\begin{itemize}
  \item El tamaño de los tipos y la numeración de las cabeceras debe escogerse para que la estructura de capítulos y secciones quede clara al lector.
  \item La longitud de línea debe ser suficientemente corta para no cansar a los ojos del lector, pero suficientemente larga para llenar la página apropiadamente.
\end{itemize}

Con sistemas \wi{WYSIWYG}, los autores a menudo generan documentos agradables estéticamente pero con muy poca o muy inconsistente estructura. \LaTeX{} impide tales errores de formato forzando al autor a declarar la estructura \emph{lógica} del documento. \LaTeX{} escoge entonces la composición más adecuada.

\subsection{Ventajas y desventajas}

Cuando gente del mundo \wi{WYSIWYG} se encuentra con usuarios de \LaTeX{}, a menudo discuten ``las \wi{ventajas de \LaTeX{}} sobre un procesador de textos normal'' o lo contrario.  Lo mejor que puede hacer cuando un debate tal comienza es mantenerse al margen, pues tales discusiones a menudo se salen de quicio.  Pero a veces uno no puede escapar\ldots

%\medskip
Pues he aquí algo de munición.  Las principales ventajas de \LaTeX{} sobre procesadores de texto normales son las siguientes:

\begin{itemize}
  \item Se dispone de composiciones diseñadas profesionalmente, lo que  hace que un documento parezca realmente ``impreso''.
  \item El soporte para la composición de fórmulas matemáticas es muy  adecuado.
  \item Los usuarios sólo tienen que aprender unas pocas órdenes fáciles  de entender, que especifican la estructura lógica del documento.  Casi nunca necesitan preocuparse del aspecto real del documento.
  \item Es fácil generar incluso estructuras complejas, como notas al  pie, referencias, índices o bibliografías.
  \item Existen paquetes libres (incluso gratuitos) que facilitan muchas tareas tipográficas especializadas, no soportadas directamente por el \LaTeX{} básico.  Por ejemplo, hay disponibles paquetes para incluir gráficos o para componer bibliografías según normas precisas.  Se describen muchos de estos paquetes en \companion.
  \item \LaTeX{} incita a los autores a escribir textos bien estructurados, porque así trabaja \LaTeX{} ---especificando la estructura---.
  \item \TeX, el motor de formateo de \LaTeXe, es libre y muy portable. Por tanto, puede ejecutarse en casi cualquier plataforma informática disponible.
%
% Add examples ...
%
\end{itemize}

%\medskip

%\noindent
\LaTeX{} tiene también algunas desventajas, y supongo que me es un poco difícil encontrar alguna notable, aunque estoy seguro de que otros le podrán hablar de cientos \texttt{;-)}


\begin{itemize}
  \item \LaTeX{} no funciona bien para quienes han vendido su alma a ciertas compañías\ldots
  \item Aunque pueden ajustarse algunos parámetros dentro de una cierta composición del documento, el diseño de una nueva composición completa es difícil y lleva mucho tiempo.\footnote{Un rumor dice que esto es uno de los elementos clave que serán tratados en el futuro sistema \LaTeX 3.}\index{LaTeX3@\LaTeX 3} 
  \item Es muy duro escribir documentos desestructurados y desorganizados.
  \item Puede que su aprendiz nunca llegue a entender, a pesar de ciertos primeros  pasos prometedores, a comprender el concepto de Marcado Lógico.
\end{itemize}
 
\section{\Filenomo{}s de entrada \LaTeX{}}

La entrada para \LaTeX{} es un \filenomo{} de texto puro.  Puede crearlo    
con cualquier editor de texto.  Contiene el texto del documento, así
como las órdenes que dirán a \LaTeX{} cómo componer el texto.

\subsection{Espacio}

\LaTeX{} trata los caracteres ``en blanco'', tales como el espacio en blanco o el tabulador, uniformemente como ``\wi{espacio}''. \emph{Varios caracteres consecutivos} \wi{en blanco} se tratan como \emph{un solo} ``espacio''.  Espacio en blanco al principio de una línea se ignora en general, y un salto de línea aislado se trata como ``espacio en blanco''.  \index{espacio en blanco!al principio de línea}

Una línea vacía entre dos líneas de texto define el fin de un párrafo. \emph{Varias} líneas vacías se tratan igual que \emph{una sola} línea vacía.  El texto de abajo es un ejemplo.  A la izquierda está es texto del \filenomo{} de entrada, y a la derecha está la salida formateada.

\begin{example}
No importa si usted deja
uno o varios     espacios
tras una palabra.

Una línea vacía comienza
un nuevo párrafo.
\end{example}
 
\subsection{Caracteres especiales}

Los siguientes símbolos sor \wi{caracteres reservados} que o tienen un significado especial bajo \LaTeX{} o no están disponibles en todas las tipografías.  Si los pone directamente en su texto, normalmente no se imprimirán, sino que obligarán a \LaTeX{} a hacer cosas que usted no pretendía.

\begin{code}
\verb.#  $  %  ^  &  _  {  }  ~  \ . %$
\end{code}

Como verá, se pueden usar estos caracteres en sus documentos añadiendo una antibarra (barra invertida) como prefijo:

\begin{example}
\# \$ \% \^{} \& \_ \{ \} \~{} 
\end{example}

Los demás símbolos y muchos más pueden imprimirse con órdenes especiales en fórmulas matemáticas o como acentos.  El carácter antibarra $\backslash$ \emph{no} puede introducirse añadiendo otra antibarra delante (\verb|\\|); esta secuencia se usa para saltar de línea.\footnote{Pruebe la orden \texttt{\$}\ci{backslash}\texttt{\$} en su lugar.  Produce una `$\backslash$'.}

\subsection{Órdenes \LaTeX{}}

Las \wi{órdenes} \LaTeX{} son sensibles a mayúsculas, y adoptan uno de los dos formatos siguientes:

\begin{itemize}
\item Comienzan con una \wi{antibarra} \verb|\| y luego tienen un
 nombre que consiste sólo en letras.  Los nombres de orden terminan
 con un espacio, un número o cualquier otra `no-letra'.
\item Consisten en una antibarra y exactamente una no-letra.
\end{itemize}

%
% \\* doesn't comply !
%

%
% Can \3 be a valid command ? (jacoboni)
%
\label{whitespace}

\LaTeX{} prescinde del espacio en blanco tras las órdenes.  Si quiere conseguir un \index{espacio en blanco!tras órdenes}espacio tras una orden, tiene que poner o \verb|{}| y un blanco o una orden especial de espaciado tras el nombre de la orden.  Las llaves \verb|{}| impiden a \LaTeX{} ``comerse''todo el espacio tras el nombre de la orden.

\begin{example}
He leído que Knuth divide a la
gente que trabaja con \TeX{} en
\TeX{}nicos y \TeX pertos.\\
Hoy es \today.
\end{example}

Algunas órdenes requieren un \wi{parámetro}, que tiene que ponerse entre \wi{llaves} \verb|{ }| tras el nombre de la orden.  Algunas órdenes soportan \wi{parámetros opcionales}, que se añaden tras el nombre de la orden entre \wi{corchetes}~\verb|[ ]|.  Los siguientes ejemplos usan algunas órdenes \LaTeX{}.  No se preocupe por ellos; se explicarán más adelante.

\begin{example}
¡Puede \textsl{fiarse} de mí!
\end{example}

\begin{example}
Por favor, comienza una nueva
línea ¡justo aquí!\newline
¡Gracias!
\end{example}

\subsection{Comentarios}
\index{comentarios}

Cuando \LaTeX{} encuentra un carácter \verb|%| al procesar un \filenomo{} de entrada, prescinde del resto de la línea actual, el salto de línea y todo el espacio en blanco al comienzo de la línea siguiente.

Esto puede usarse para escribir notas en el \filenomo{} de entrada, que no se mostrarán en la versión impresa.

\begin{example}
Este es un % estúpido
% Mejor: instructivo <----
ejemplo: Supercal%
              ifragilíst%
    icoespialidoso
\end{example}

El carácter \texttt{\%} también puede usarse para dividir líneas largas en la entrada donde no se permiten espacios ni saltos de línea.

Para comentarios más largos puede usar el entorno \ei{comment} proporcionado por los paquetes \pai{comment} o \pai{verbatim}.  Esto significa que tiene que añadir la línea \verb|\usepackage{verbatim}| o \verb|\usepackage{comment}| al preámbulo de su documento, como se explica abajo, antes de que pueda usar esta orden.

\begin{example}
Este es otro
\begin{comment}
bastante estúpido,
pero útil
\end{comment}
ejemplo para empotrar
comentarios en su texto.
\end{example}

Tenga en cuenta que eso no funciona dentro de entornos complejos, como
por ejemplo los matemáticos.

\section{Estructura del \filenomo{} de entrada}

Cuando \LaTeXe{} procesa un \filenomo{} de entrada, espera que siga una cierta \wi{estructura}.  Así, todo \filenomo{} de entrada ha de comenzar con la orden
\begin{code}
\verb|\documentclass{...}|
\end{code}
Esto indica qué tipo de documento pretende usted escribir.  Después, puede incluir órdenes que influyen el estilo de todo el documento, o puede cargar \wi{paquete}s que añaden nuevas prestaciones al sistema \LaTeX{}.  Para cargar un paquete use la orden
\begin{code}
\verb|\usepackage{...}|
\end{code}

Cuando todo el trabajo de preparación está hecho, comience a escribir el cuerpo del texto con la orden

\begin{code}
\verb|\begin{document}|
\end{code}

El área entre \texttt{\bs documentclass} y \texttt{\bs begin$\mathtt{\{}$document$\mathtt{\}}$} se llama \emph{\wi{preámbulo}}.

Ahora escriba el texto mezclado con órdenes \LaTeX{} útiles.  Al final del documento añada la orden
\begin{code}
\verb|\end{document}|
\end{code}
que dice a \LaTeX{} que termine el trabajo.  Cualquier cosa que siga a esta orden será ignorada por \LaTeX.

La Figura~\ref{mini} muestra el contenido de un \filenomo{} \LaTeXe{} mínimo.  Un \wi{\filenomo{} de entrada} algo más complejo aparece en la Figura~\ref{document}.

\begin{figure}[!bp]
\begin{lined}{7cm}
\begin{verbatim}
\documentclass{article}
\usepackage[spanish]{babel}
\usepackage[latin1]{inputenc}
\begin{document}
Gracián:  Lo bueno, si breve...
\end{document}
\end{verbatim}
\end{lined}
\caption{Un \filenomo{} \LaTeX{} mínimo.} \label{mini}
\end{figure}
 
\begin{figure}[!bp]
\begin{lined}{10cm}
\begin{verbatim}
\documentclass[a4paper,11pt]{article}
% define el título
\author{H.~Partl}
\title{Minimalismo}
\begin{document}
% genera el título
\maketitle 
% inserta el índice general
\tableofcontents
\section{Algunas palabras interesantes}
Y bien, aquí comienza mi articulillo.
\section{Adiós, Mundo}
...y aquí termina.
\end{document}
\end{verbatim}
\end{lined}
\caption[Ejemplo de un artículo de revista.]{Ejemplo de un artículo de revista.  Todas las órdenes que ve en este ejemplo se explicarán más tarde.}
\label{document}
\end{figure}

\section{Una típica sesión de consola o línea de órdenes}

Como se insinuaba antes (ver~ \ref{sec:basics}, p.~\pageref{sec:basics}) \LaTeX{} por sí mismo viene sin GUI (interfaz gráfica de usuario) ni botones para pulsar.  Es un programa de procesamiento por lotes que ``mastica'', ``traga'' y ``digiere'' su \filenomo{} de entrada para ``excretar'' su(s) \filenomo{}(s) de salida. Algunas instalaciones de \LaTeX{} ofrecen una interfaz gráfica donde usted puede escribir y compilar su \filenomo{} de entrada (\TeX{}nicCenter, \TeX{}maker, Kile).  En otros sistemas puede requerirse la escritura de ciertas órdenes, de modo que he aquí cómo lograr que \LaTeX{} compile su \filenomo{} de entrada en un sistema basado en texto.  Téngalo en cuenta: esta descripción supone que su \computernomo{} ya dispone de una instalación de \LaTeX{} funcional.

\begin{enumerate}
\item Edite/Cree su \filenomo{} de entrada \LaTeX{}.  Este \filenomo{} debe ser texto puro.  Puede crearlo con cualquier editor de texto: vi, emacs, Nano, Gedit, Kate, etc.  También puede usar un procesador de texto (Open/Libre\-Office Writer, Kword, Abiword), pero asegúrese de que guarda el \filenomo{} con formato \emph{Texto plano}.  Al escoger un nombre para el \filenomo{}, póngale como extensión \eei{.tex}.

\item Ejecute \LaTeX{} en su \filenomo{} de entrada.  Si tiene éxito aparecerá un \filenomo{} \texttt{.dvi}.  Puede que necesite ejecutar \LaTeX{} varias veces para que los índices y todas las referencias internas queden correctamente definidas.  Si su \filenomo{} de entrada tiene un error \LaTeX{} se lo dirá y parará el procesamiento de su \filenomo{} de entrada.  Escriba \texttt{ctrl-D} para volver a la línea de órdenes.
\begin{lscommand}
\verb+latex mi-documento.tex+
\end{lscommand}

\item Ahora puede visualizar el \filenomo{} DVI.  Hay varias maneras de hacerlo. Puede mostrar el \filenomo{} en pantalla con
\begin{lscommand}
\verb+xdvi mi-documento.dvi &+
\end{lscommand}
Esto funciona en GNU o Unix con X11.  En ReactOS o Windows puede probar \texttt{yap} (yet another previewer).

También puede convertir el \filenomo{} dvi a \PSi{} para imprimirlo o visualizarlo con Ghostscript.
\begin{lscommand}
\verb+dvips -Pcmz mi-documento.dvi -o mi-documento.ps+
\end{lscommand}

Su sistema \LaTeX{} puede incluir las herramientas \texttt{dvipdf} o \texttt{dvipdfm}, que le permiten convertir el \filenomo{} \texttt{.dvi}
directamente en pdf.
\begin{lscommand}
\verb+dvipdf mi-documento.dvi+
\end{lscommand}

Finalmente, PDF\LaTeX{} le permite compilar el \filenomo{} directamente en pdf.
\begin{lscommand}
\verb+pdflatex mi-documento+
\end{lscommand}
\end{enumerate}
 
\section{El aspecto del documento}
 
\subsection {Clases de documento}\label{sec:documentclass}

La primera información que \LaTeX{} necesita saber cuando procesa un \filenomo{} de entrada es el tipo de documento que el autor quiere crear. Esto se indica con la orden \ci{documentclass}.
\begin{lscommand}
\ci{documentclass}\verb|[|\emph{opciones}\verb|]{|\emph{clase}\verb|}|
\end{lscommand}
%
Aquí \emph{clase} indica el tipo de documento por crear.  El Cuadro~\ref{documentclasses} lista las clases de documentos explicadas en esta introducción.  La distribución de \LaTeXe{} proporciona clases adicionales para otros documentos, incluyendo cartas y diapositivas (presentaciones).  El parámetro \emph{\wi{opciones}} personaliza el comportamiento de la clase.  Las opciones tienen que separarse por comas.  Las opciones más comunes para las clases de documento habituales se listan en el Cuadro~\ref{options}.

\begin{table}[!bp]
\caption{Clases de documento.} \label{documentclasses}
\begin{lined}{\textwidth}
\begin{description} 
  \item [\normalfont\texttt{article}] para artículos en revistas científicas, informes breves, documentación de programas, invitaciones, \ldots \index{article clase}
  \item [\normalfont\texttt{proc}] para actas, basado en la clase  \emph{article}.  \index{proc clase}
  \item [\normalfont\texttt{minimal}] es lo más pequeña posible. Solamente establece un tamaño de página y una \fontnomo{} (tipo de letra).  Se usa principalmente para depurar errores. \index{minimal clase}
  \item [\normalfont\texttt{report}] para informes más largos que  contienen varios capítulos, pequeños libros, tesis doctorales,  \ldots \index{report clase}
  \item [\normalfont\texttt{book}] para libros reales \index{book clase}
  \item [\normalfont\texttt{slides}] para diapositivas.  La clase usa letras grandes sin serifas.  También puede en su lugar usar las clases Foil\TeX{}, Prosper o Beamer. \index{slides clase}\index{foiltex}\index{prosper}\index{beamer}
\end{description}
\end{lined}
\end{table}

\begin{table}[!bp]
\caption{Opciones de clases de documento.} \label{options}
\begin{lined}{\textwidth}
\begin{flushleft}
\begin{description}
  \item[\normalfont\texttt{10pt}, \texttt{11pt}, \texttt{12pt}] \quad Establece el tamaño de la principal \fontnomo{} del documento.  Si no se especifica ninguna opción, se aplica \texttt{10pt}.
    \index{tamaño de \fontnomo{} del documentd}\index{tamaño de \fontnomo{} básico}
  \item[\normalfont\texttt{a4paper}, \texttt{letterpaper}, \ldots] \quad Define el tamaño del papel.  El tamaño por omisión es \texttt{letterpaper}.  Además de esas dos, pueden indicarse \texttt{a5paper}, \texttt{b5paper}, \texttt{executivepaper}, y \texttt{legalpaper}.  \index{legal papel} \index{tamaño del papel} \index{A4 papel}\index{letter papel} \index{A5 papel}\index{B5 papel}\index{executive papel}
  \item[\normalfont\texttt{fleqn}] \quad Dispone las fórmulas destacadas  hacia la izquierda en vez de centradas.
  \item[\normalfont\texttt{leqno}] \quad Coloca los números de las  fórmulas a la izquierda en vez de a la derecha.
  \item[\normalfont\texttt{titlepage}, \texttt{notitlepage}] \quad Indica si tras el tras el \wi{título del documento} debe empezarse una página nueva o no.  La clase \texttt{article} no comienza página nueva por omisión, mientras que \texttt{report} y \texttt{book} sí la tienen. \index{título}
  \item[\normalfont\texttt{onecolumn}, \texttt{twocolumn}] \quad Dice a \LaTeX{} que componga el documento en \wi{una columna} o \wi{dos columnas} respectivamente.
  \item[\normalfont\texttt{twoside, oneside}] \quad Indica si quiere generar el documento a dos caras o a una, respectivamente.  Las clases \texttt{article} y \texttt{report} son a \wi{una cara} y la clase \texttt{book} es a \wi{dos caras} por omisión.  Tenga en cuenta que esta opción concierne solamente al aspecto del documento.  La opción \texttt{twoside} \emph{no} dice a su impresora que debería de hecho imprimir a dos caras.
  \item[\normalfont\texttt{landscape}] \quad Cambia la composición del  documento para imprimirlo en modo apaisado.
  \item[\normalfont\texttt{openright, openany}] \quad Hace que lo capítulos comiencen o sólo en páginas de la derecha, o en la siguiente página disponible.  Esto no funciona con la clase \texttt{article}, pues no entiende de capítulos.  La clase \texttt{report} por omisión comienza capítulos en la página siguiente disponible y la clase \texttt{book} los comienza en páginas de la derecha.
\end{description}
\end{flushleft}
\end{lined}
\end{table}

Ejemplo:  Un \filenomo{} de entrada para un documento \LaTeX{} podría empezar con la línea
\begin{code}
\ci{documentclass}\verb|[11pt,twoside,a4paper]{article}|
\end{code}
que manda a \LaTeX{} componer el documento como un \emph{artículo} con un tamaño de \fontnomo{} básica de \emph{once puntos}, y producir un documento adecuado para imprimir a \emph{doble cara} en \emph{papel A4}.
\pagebreak[2]

\subsection{Paquetes}
\index{package} 

Mientras escribe su documento, probablemente halle que hay algunas áreas donde el \LaTeX{} básico no puede resolver su problema.  Si quiere incluir \wi{gráficos}, \wi{texto en color} o código fuente de un \filenomo{} en su documento, necesita mejorar las capacidades de \LaTeX.  Tales mejoras se introducen con \emph{paquetes}.  Los paquetes se activan con la orden
\begin{lscommand}
\ci{usepackage}\verb|[|\emph{opciones}\verb|]{|\emph{paquete}\verb|}|
\end{lscommand}
donde \emph{paquete} es el nombre del paquete y \emph{opciones} es una lista de palabras clave que activan funciones especiales del paquete.  Algunos paquetes vienen con la distribucón básica de \LaTeXe{} (vea Cuadro~\ref{packages}).  Otros se proporcionan por separado.  Puede encontrar más información sobre los paquetes instalados en su \computernomo{} en la \guide.  La principal fuente de información sobre paquetes de \LaTeX{} es \companion.  Contiene descripciones de cientos de paquetes, junto con información sobre cómo escribir sus propias extensiones de \LaTeXe.

Las distribuciones modernas de \TeX{} vienen con un gran número de paquetes preinstalados.  Si está trabajando en un sistema GNU o Unix, use la orden \texttt{texdoc} para acceder a información sobre paquetes.

\begin{table}[btp]
\caption{Algunos paquetes distribuidos con \LaTeX.} \label{packages}
\begin{lined}{\textwidth}
\begin{description}
  \item[\normalfont\pai{doc}] Permite la documentación de programas \LaTeX{}.\\ Descrito en \texttt{doc.dtx}\footnote{Este \filenomo{} debería estar instalado en su sistema, y usted debería ser capaz de crear el correspondiente \texttt{dvi} escribiendo \texttt{latex doc.dtx} en cualquier directorio en que tenga permiso de escritura. Lo mismo aplica para todos los demás \filenomo{}s mencionados en este cuadro.}  y en \companion.
  \item[\normalfont\pai{exscale}] Proporciona versiones escaladas de la \fontnomo{} de la extensión matemática.\\ Descrito en \texttt{ltexscale.dtx}.
  \item[\normalfont\pai{fontenc}] Indica qué \wi{codificación de \fontnomo{}} debería usar \LaTeX{}.\\ Descrito en \texttt{ltoutenc.dtx}.
  \item[\normalfont\pai{ifthen}] Proporciona órdenes de la forma `si\ldots entonces\ldots si no\ldots'.\\ Descrito en \texttt{ifthen.dtx} y \companion.
  \item[\normalfont\pai{latexsym}] Para acceder a la \fontnomo{} de símbolos de \LaTeX{}, debería usar el paquete \texttt{latexsym}. Descrito en \texttt{latexsym.dtx} y en \companion. 
  \item[\normalfont\pai{makeidx}] Proporciona órdenes para producir  índices.  Descrito en la sección~\ref{sec:indexing} y en \companion.
  \item[\normalfont\pai{syntonly}] Procesa un documento sin componerlo.  Útil para localizar errores.  
  \item[\normalfont\pai{inputenc}] Permite indicar una codificación para la entrada como ASCII, ISO Latin-1, ISO Latin-2, páginas de código 437/850 IBM,  Apple Macintosh, Next, UTF-8 o una definida por el usuario. Descrito en \texttt{inputenc.dtx}.
\end{description}
\end{lined}
\end{table}

\subsection{Estilos de página}
 
\LaTeX{} soporta tres combinaciones predefinidas de \wi{cabeceras} y \wi{pies de página}, llamadas \wi{estilos de página}.  El parámetro \emph{estilo} de la orden \index{page style!plain@\texttt{plain}}\index{plain@\texttt{plain}} \index{page style!headings@\texttt{headings}}\index{headings@texttt{headings}} \index{page style!empty@\texttt{empty}}\index{empty@\texttt{empty}}
\begin{lscommand}
\ci{pagestyle}\verb|{|\emph{estilo}\verb|}|
\end{lscommand}
define cuál emplearse. 
El cuadro~\ref{pagestyle} lista los estilos de página predefinidos.

\begin{table}[!hbp]
\caption{Los estilos de página predifinidos de \LaTeX.} \label{pagestyle}
\begin{lined}{\textwidth}
\begin{description}
  \item[\normalfont\texttt{plain}] imprime los números de página en la  parte de abajo, en el centro del pie.  Es el estilo por omisión.
  \item[\normalfont\texttt{headings}] imprime el nombre del capítulo  actual y el número de página en la cabecera de cada página, mientras  que el pie queda vacío.  (Es el estilo usado en este documento)
  \item[\normalfont\texttt{empty}] deja vacíos tanto la cabecera como el  pie de página.
\end{description}
\end{lined}
\end{table}

Es posible cambiar el estilo de la página actual con la orden
\begin{lscommand}
\ci{thispagestyle}\verb|{|\emph{estilo}\verb|}|
\end{lscommand}
Se puede encontrar una descripción de cómo crear sus propias cabeceras y pies en \companion{} y en la sección~\ref{sec:fancy} en la página~\pageref{sec:fancy}.
%
% Pointer to the Fancy headings Package description !
%

\section{\Filenomo{}s que puede encontrarse}

Cuando trabaje con \LaTeX{} se encontrará pronto con un batiburrillo de \filenomo{}s con \wi{extensiones} variadas.  La lista siguiente explica los diversos \wi{tipos de \filenomo{}} que puede encontrar cuando trabaje con \TeX{}.  Tenga en cuenta que esta tabla no pretende ser una lista completa de extensiones, pero si encuentra una que piense que es importante, por favor escríbame indicándolo.

\begin{description}  
  \item[\eei{.tex}] \Filenomo{} de entrada \LaTeX{} (o \TeX{}).  Puede  compilarse con \texttt{latex} (o \texttt{tex}).
  \item[\eei{.sty}] \LaTeX{} Paquete de macros.  Es un \filenomo{} que puede  cargar en su documento \LaTeX{} usando la orden \ci{usepackage}.
  \item[\eei{.dtx}] \TeX{} documentado.  Es el formato principal para distribuir \filenomo{}s de estilo \LaTeX{}.  Si procesa un \filenomo{} .dtx obtiene código macro documentado del paquete \LaTeX{} contenido en el \filenomo{} .dtx.
  \item[\eei{.ins}] El instalador para los \filenomo{}s contenidos en el \filenomo{} .dtx correspondiente.  Si descarga un paquete \LaTeX{} de la red, normalmente obtendrá un \filenomo{} .dtx y uno .ins.  Ejecute \LaTeX{} sobre el \filenomo{} .ins para desempacar el \filenomo{} .dtx.
  \item[\eei{.cls}] Los \filenomo{}s de clase definen el aspecto de su  documento.  Se seleccionan mediante la orden \ci{documentclass}.
  \item[\eei{.fd}] \Filenomo{} de descripción de una \fontnomo{} que define  nuevas \fontsnomo{} para \LaTeX{}.
\end{description}

Los siguientes \filenomo{}s se generan cuando ejecuta \LaTeX{} sobre su \filenomo{} de entrada:

\begin{description}
  \item[\eei{.dvi}] Device Independent File (\filenomo{} independiente de dispositivo).  Es el principal resultado de una compilación de \LaTeX{}.  Puede visualizar su contenido con un programa visor DVI o puede imprimirlo mediante \texttt{dvips} o una aplicación similar.
  \item[\eei{.log}] Recoge un registro detallado de qué pasó durante la  última compilación.
  \item[\eei{.toc}] Almacena todas las cabeceras de sección.  Es leído  en la siguiente compilación para producir el índice general.
  \item[\eei{.lof}] Es como .toc pero para la lista de figuras.
  \item[\eei{.lot}] Lo mismo, para la lista de cuadros.
  \item[\eei{.aux}] Otro \filenomo{} que conserva información de una compilación a la siguiente.  Entre otras cosas, el \filenomo{} .aux se usa para las referencias cruzadas.
  \item[\eei{.idx}] Si su documento contiene un índice alfabético, \LaTeX{} almacena todas las palabras del índice en este \filenomo{}. Procese este \filenomo{} con \texttt{makeindex}.  Acuda a la sección \ref{sec:indexing} en la página \pageref{sec:indexing} para más información sobre indexado.
  \item[\eei{.ind}] El \filenomo{} .idx procesado, listo para ser incluido  en su documento en el próximo ciclo de compilaciones.
  \item[\eei{.ilg}] Registro con lo que hizo \texttt{makeindex}.
\end{description}

% Package Info pointer

% Add Info on page-numbering, ...
% \pagenumbering

\section{Proyectos grandes}

Cuando trabaje en proyectos grandes, puede servirle dividir el \filenomo{} de entrada en varias partes que puede reunir al compilarlo.  \LaTeX{} tiene dos órdenes que lo ayudan a hacerlo.

\begin{lscommand}
\ci{include}\verb|{|\emph{nombre-de-\filenomo{}}\verb|}|
\end{lscommand}
%
Puede usar esta orden en el cuerpo del documento para insertar el contenido de otro \filenomo{} llamado \emph{nombre-de-\filenomo{}.tex}.  Tenga en cuenta que \LaTeX{} comenzará una nueva página antes de procesar el material proveniente de \emph{nombre-de-\filenomo{}.tex}.

La segunda orden puede usarse en el preámbulo.  Le permite indicar a \LaTeX{} que solamente incluya algunos de los \filenomo{}s señalados mediante \verb|\include|.
\begin{lscommand}
\ci{includeonly}\verb|{|\emph{nombre-\filenomo{}-1}\verb|,|\emph{nombre-\filenomo{}-2}%
\verb|,|...\verb|}|
\end{lscommand}
Tras ejecutar esta orden en el preámbulo del documento, sólo se ejecutarán las órdenes \ci{include} para los \filenomo{}s listados en el argumento de la orden \ci{includeonly}.  Fíjese en que no ha de haber ningún espacio entre los nombres de \filenomo{}s y las comas.

La orden \ci{include} comienza componiendo el texto incluido en una nueva página.  Esto ayuda cuando usa \ci{includeonly}, porque los saltos de página no se moverán, incluso cuando se omitan algunos \filenomo{}s.  A veces esto no es deseable.  En tal caso, puede usar la orden
\begin{lscommand}
\ci{input}\verb|{|\emph{nombre-de-\filenomo{}}\verb|}|
\end{lscommand}
que simplemente incluye el \filenomo{} especificado, sin efectos especiales y sin insertar espacio adicional.

Para que \LaTeX{} inspeccione rápidamente su documento puede usar el paquete \pai{syntonly}.  Hace que \LaTeX{} recorra su documento sólo comprobando la corrección de la sintaxis y el uso de órdenes, pero no produce ninguna salida (DVI).  Puesto que \LaTeX{} se ejecuta más rápido de este modo puede hacerle ahorrar mucho tiempo valioso.  El uso es muy sencillo:

\begin{verbatim}
\usepackage{syntonly}
\syntaxonly
\end{verbatim}
Cuando quiera producir páginas, basta con comentar la segunda línea (mediante la adición de un signo de porcentaje al principio).

%