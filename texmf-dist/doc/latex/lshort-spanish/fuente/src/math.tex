%%%%%%%%%%%%%%%%%%%%%%%%%%%%%%%%%%%%%%%%%%%%%%%%%%%%%%%%%%%%%%%%
% Contents: Math typesetting with LaTeX
% $Id: F-BD65826A9DB5085DAF6992A66E92B128.tex,v 1.1 2008-03-06 19:21:17 carleos Exp $
%%%%%%%%%%%%%%%%%%%%%%%%%%%%%%%%%%%%%%%%%%%%%%%%%%%%%%%%%%%%%%%%%
 
\chapter{Composición de fórmulas matemáticas}

\begin{intro}
  ¡Ahora está listo! En este capítulo, abordaremos la mayor aptitud de \TeX{}: la composición matemática.  Pero cuidado, este capítulo solo trata la superficie.  Aunque lo que se explica aquí basta para mucha gente, no desespere si no encuentra aquí la solución a sus necesidades de composición matemática.  Es muy probable que su problema haya sido abordado en \AmS-\LaTeX{}\footnote{La \emph{American Mathematical Society} (Sociedad Matemática Estadounidense) ha producido una potente extensión de \LaTeX{}.  Muchos de los ejemplos de este capítulo hacen uso de dicha extensión.  Todas las distribuciones recientes de \TeX{} la proporcionan.  Si la suya no la tiene, visite \CTANref|macros/latex/required/amslatex|.}
\end{intro}
  
\section{Generalidades}

\LaTeX{} tiene un modo especial para componer \wi{matemáticas}. Hay dos posibildades: escribir las matemáticas dentro de un párrafo, en el mismo renglón que el resto del texto, o partir el párrafo para componer las matemáticas aparte, destacadas.  El texto matemático \emph{dentro} del párrafo se introduce entre \ci{(} y \ci{)},
\index{$@\texttt{\$}}, %$ 
entre \texttt{\$} y \texttt{\$}, o entre %}
\verb|\begin{|\ei{math}\verb|}| y \verb|\end{math}|.\index{formulae}
\begin{example}
Sume $a$ al cuadrado y $b$ al
cuadrado para obtener $c$ al
cuadrado.  Más formalmente:
$c^{2}=a^{2}+b^{2}$
\end{example}

\begin{example}
\TeX{} se pronuncia como
\(\tau\epsilon\chi\).\\[6pt]
100~m$^{3}$ de agua\\[6pt]
De todo 
\begin{math}\heartsuit\end{math}
\end{example}

Si quiere que sus ecuaciones o fórmulas matemáticas más grandes se sitúen destacadas aparte del resto del párrafo, es preferible \emph{aislarlas}.  Para ello, puede encerrarlas entre \ci{[} y \ci{]}, entre \verb|\begin{|\ei{displaymath}\verb|}| y \verb|\end{displaymath}|, o entre \verb|\begin{|\ei{equation}\verb|}| y \verb|\end{equation}|.
\begin{example}
Sume $a$ al cuadrado y $b$ al
cuadrado para obtener $c$ al
cuadrado.  Más formalmente:
\begin{displaymath}
c^{2}=a^{2}+b^{2}
\end{displaymath}
o puede teclear menos con:
\[c^2=a^2+b^2\]
\end{example}

Si quiere que \LaTeX{} enumere sus ecuaciones, puede usar el entorno \ei{equation}.  Puede etiquetar mediante \ci{label} la ecuación con un número y referirse a éste desde otro lugar del texto usando \ci{ref} o la orden \ci{eqref} del paquete \pai{amsmath}:
\begin{example}
\begin{equation} \label{eq:eps}
\epsilon > 0
\end{equation}
De (\ref{eq:eps}), se deduce
\ldots{} De \eqref{eq:eps} 
se deduce lo mismo.
\end{example}

Observe las diferencias de estilo entre las ecuaciones en párrafo y las aisladas:

\begin{example}
$\lim_{n \to \infty} 
\sum_{k=1}^n \frac{1}{k^2} 
= \frac{\pi^2}{6}$
\end{example}

\begin{example}
\begin{displaymath}
\lim_{n \to \infty} 
\sum_{k=1}^n \frac{1}{k^2} 
= \frac{\pi^2}{6}
\end{displaymath}
\end{example}


Hay diferencias entre \emph{modo matemático} y \emph{modo texto}.  Por ejemplo, en \emph{modo matemático}: 

\begin{enumerate}
\item La mayoría de los espacios y saltos de línea no significan nada, pues todos los espacios se deducen lógicamente de las expresiones matemáticas, o tienen que ser indicados con órdenes especiales como \ci{,}, \ci{quad} o \ci{qquad}. 
\item No se permiten renglones vacíos. Sólo un párrafo por fórmula.
\item Cada letra se considera como nombre de una variable y como tal será compuesta.  Si quiere componer texto normal dentro de una fórmula (tipo redondo y espaciado normal) entonces tiene que introducir el texto usando las órdenes \verb|\textrm{...}| (véase también la sección \ref{sec:fontsz} en la página \pageref{sec:fontsz}).
\end{enumerate}

\begin{example}
\begin{equation}
\forall x \in \mathbf{R}:
\qquad x^{2} \geq 0
\end{equation}
\end{example}

\begin{example}
\begin{equation}
x^{2} \geq 0\qquad
\textrm{para todo }x\in\mathbf{R}
\end{equation}
\end{example}
 
%
% Add AMSSYB Package ... Blackboard bold .... R for realnumbers
%
Los matemáticos pueden ser muy quisquillosos sobre qué símbolos usar: sería tradicional usar aquí la `\wi{negrita de pizarra}', \index{negrita} que se obtiene usando \ci{mathbb} del paquete \pai{amsfonts} o \pai{amssymb}.
\ifx\mathbb\undefined\else El último ejemplo se convierte en
\begin{example}
\begin{displaymath}
x^{2} \geq 0\qquad
\textrm{para todo }x\in\mathbb{R}
\end{displaymath}
\end{example}
\fi

\section{Agrupación en modo matemático}

La mayoría de las órdenes en modo matemático actúan sólo sobre el siguiente carácter, así que si quiere que una orden afecte a varios caracteres, debe agruparlos juntos entre llaves: \verb|{...}|. 
\begin{example}
\begin{equation}
a^x+y \neq a^{x+y}
\end{equation}
\end{example}

\section{Construcción de bloques de una fórmula matemática}

Esta sección describe las órdenes más importantes usadas en composición matemática.  Eche un vistazo a la sección~\ref{symbols} en la página~\pageref{symbols} donde se muestra una lista detallada de órdenes para componer símbolos matemáticos.

Las \textbf{\wi{letras griegas} minúsculas} se introducen con \verb|\alpha|, \verb|\beta|, \verb|\gamma|, \ldots, las mayúsculas se introducen con \verb|\Gamma|, \verb|\Delta|, \ldots\footnote{No hay definida una alfa mayúscula en \LaTeXe{} porque parece igual que una A latina normal.  Cuando se termine el nuevo código matemático, las cosas cambiarán.}
\begin{example}
$\lambda,\xi,\pi,\mu,\Phi,\Omega$
\end{example}

Los \textbf{exponentes y subíndices} pueden indicarse con\index{exponent}\index{subscript} los caracteres \verb|^|\index{^@\verb"|^"|}~y~\verb|_|\index{_@\verb"|_"|}.
\begin{example}
$a_{1}$ \qquad $x^{2}$ \qquad
$e^{-\alpha t}$ \qquad
$a^{3}_{ij}$\\
$e^{x^2} \neq {e^x}^2$
\end{example}

La \textbf{\wi{raíz cuadrada}} se introduce como \ci{sqrt}; la raíz $n^\mathrm{\acute esima}$ se genera con \verb|\sqrt[|$n$\verb|]|.  El tamaño del signo de la raíz lo determina automáticamente \LaTeX.  Si sólo necesita el signo (habitual en la tradición anglosajona, pero no en la tipografía española), use \verb|\surd|.
\begin{example}
$\sqrt{x}$ \qquad 
$\sqrt{ x^{2}+\sqrt{y} }$ 
\qquad $\sqrt[3]{2}$\\[3pt]
$\surd[x^2 + y^2]$
\end{example}

%\index{puntos!suspensivos}
%\index{puntos!verticales}
%\index{puntos!horizontales}
%Si bien el signo de punto para indicar la operación de multiplicación normalmente se omite, se escribe a veces para ayudar al ojo en la agrupacion de una formula. Utilice \cdot para componer un solo punto centrado. \cdots son tres puntos centrados mientras \ldots establece los puntos bajos o suspensivos (en la línea de base). Además de eso, hay \vdots para puntos verticales y \ddots para puntos diagonales.

\begin{example}
$\Psi = v_1 \cdot v_2
 \cdot \ldots \qquad 
 n! = 1 \cdot 2 
 \cdots (n-1) \cdot n$
\end{example}

Las órdenes \ci{overline} y \ci{underline} crean \textbf{líneas horizontales} justo encima o debajo de una expresión.
\index{horizontal!línea}
\begin{example}
$\overline{m+n}$
\end{example}

Las órdenes \ci{overbrace} y \ci{underbrace} crean \textbf{llaves horizontales} largas sobre o bajo una expresión.
\index{horizontal!brace}
\begin{example}
$\underbrace{a+b+\cdots+z}_{26}$
\end{example}

\index{matemático!acento} Para añadir acentos matemáticos como flechas pequeñas o \wi{tilde}s a las variables, puede usar las órdenes dadas en el Cuadro~\ref{mathacc} de la página \pageref{mathacc}.  Se consiguen circunflejos anchos y tildes que cubren varios caracteres mediante \ci{widetilde} y \ci{widehat}.  El símbolo \verb|'|\index{'@\verb"|'"|} produce una \wi{prima}.
% a dash is --
\begin{example}
\begin{displaymath}
y=x^{2}\qquad y'=2x\qquad y''=2
\end{displaymath}
\end{example}

Los \textbf{vectores}\index{vectors} suelen indicarse añadiendo \wi{flecha}s pequeñas encima de una variable.  Esto se hace con la orden \ci{vec}.  Las dos órdenes \ci{overrightarrow} y \ci{overleftarrow} son útiles para denotar un vector desde $A$ hasta $B$.
\begin{example}
\begin{displaymath}
\vec a\quad\overrightarrow{AB}
\end{displaymath}
\end{example}

No se suele escribir un punto explícito para indicar una multiplicación; sin embargo, a veces sí se escribe para ayudar a los ojos del lector a agrupar los elementos de una fórmula.  Puede usar \ci{cdot} en estos casos:
\begin{example}
\begin{displaymath}
v = {\sigma}_1 \cdot {\sigma}_2
    {\tau}_1 \cdot {\tau}_2
\end{displaymath}
\end{example}

Los nombres de funciones como log suelen componerse en una \fontnomo{} redonda, y no en cursiva como se hace con las variables, así que \LaTeX{} proporciona las siguientes órdenes para componer las nombres de funciones más importantes, tanto para documentos en inglés\ldots \index{mathematical!functions}

\begin{tabular}{llllll}
\ci{arccos} &  \ci{cos}  &  \ci{csc} &  \ci{exp} &  \ci{ker}    & \ci{limsup} \\
\ci{arcsin} &  \ci{cosh} &  \ci{deg} &  \ci{gcd} &  \ci{lg}     & \ci{ln}     \\
\ci{arctan} &  \ci{cot}  &  \ci{det} &  \ci{hom} &  \ci{lim}    & \ci{log}    \\
\ci{arg}    &  \ci{coth} &  \ci{dim} &  \ci{inf} &  \ci{liminf} & \ci{max}    \\
\ci{sinh}   & \ci{sup}   &  \ci{tan}  & \ci{tanh}&  \ci{min}    & \ci{Pr}     \\
\ci{sec}    & \ci{sin} \\
\end{tabular}

\ldots como para documentos en español:

\begin{tabular}{llllll}
\ci{cosec}  &  \ci{arcsen} &  \ci{deg}  &  \ci{arctg}  &  \ci{cotg} & \ci{sen} \\
\ci{arg}    &  \ci{inf}    &  \ci{senh} &  \ci{tg}     &  \ci{tgh}  \\
\end{tabular}

\begin{example}
\[\lim_{x \rightarrow 0}
\frac{\sen x}{x}=1\]
\end{example}

Para la función \wi{módulo}, hay dos órdenes: \ci{bmod} para el operador binario ``$a \bmod b$'' y \ci{pmod} para expresiones tales como ``$x\equiv a \pmod{b}$.''
\begin{example}
$a\bmod b$\\
$x\equiv a \pmod{b}$
\end{example}

Una \textbf{\wi{fracción}} vertical se compone con la orden \ci{frac}\verb|{...}{...}|.  A menudo es preferible la forma horizontal $1/2$, porque queda mejor para cantidades pequeñas de ``material fraccional''.
\begin{example}
$1\frac{1}{2}$~horas
\begin{displaymath}
\frac{ x^{2} }{ k+1 }\qquad
x^{ \frac{2}{k+1} }\qquad
x^{ 1/2 }
\end{displaymath}
\end{example}

Para componer coeficientes binomiales o estructuras similares, puede usar la orden \ci{binom} del paquete \pai{amsmath}.

\begin{example}
\begin{displaymath}
\binom{n}{k}\qquad\mathrm{C}_n^k
\end{displaymath}
\end{example}

Para relaciones binarias puede ser útil apilar símbolos uno sobre otro.  \ci{stackrel} pone el símbolo dado en el primer argumento con tamaño superíndice sobre el segundo, que se coloca en su posición habitual.
\begin{example}
\begin{displaymath}
\int f_N(x) \stackrel{!}{=} 1
\end{displaymath}
\end{example}

El operador \textbf{\wi{integral}} se genera con \ci{int}, el \textbf{\wi{sumatorio}} con \ci{sum} y el \textbf{\wi{productorio}} con \ci{prod}.  Los límites superior e inferior se indican con~\verb|^| y~\verb|_| como los superíndices y subíndices.\index{superíndice} \footnote{\AmS-\LaTeX{} además tiene super-/subíndices multi-renglón.}
\begin{example}
\begin{displaymath}
\sum_{i=1}^{n} \qquad
\int_{0}^{\frac{\pi}{2}} \qquad
\prod_\epsilon
\end{displaymath}
\end{example}

Para controlar más aún la colocación de índices en expresiones complejas, \pai{amsmath} proporciona dos herramientas adicionales: la orden \ci{substack} y el entorno \ei{subarray}:
\begin{example}
\begin{displaymath}
\sum_{\substack{0<i<n \\ 1<j<m}}
   P(i,j) =
\sum_{\begin{subarray}{l}
         i\in I\\
         1<j<m
      \end{subarray}}     Q(i,j)
\end{displaymath}
\end{example}

\medskip

\TeX{} proporciona todo tipo de símbolos como \textbf{\wi{llaves}} y otros \wi{delimitadores} (p.ej.~$[\;\langle\;\|\;\updownarrow$). Paréntesis y corchetes pueden introducirse con las teclas correspondientes, y llaves con \verb|\{|, pero el resto de los delimitadores se generan con órdenes especiales (p.ej.~\verb|\updownarrow|).  Para una lista de todos los delimitadores disponibles, vea el Cuadro~\ref{tab:delimiters} en la página \pageref{tab:delimiters}.
\begin{example}
\begin{displaymath}
{a,b,c}\neq\{a,b,c\}
\end{displaymath}
\end{example}

Si pone la orden \ci{left} ante un delimitador de apertura, y \ci{right} ante un delimitardor de cierre, \TeX{} determinará automáticamente el tamaño correcto del delimitador.  Tenga en cuente que ha de cerrar cada \ci{left} con el correspondiente \ci{right}, y que el tamaño se determina correctamente sólo si ambos se componen en la misma línea.  Si no quiere que aparezca nada a la derecha, use `\ci{right.}'
\begin{example}
\begin{displaymath}
1 + \left( \frac{1}{ 1-x^{2} }
    \right) ^3
\end{displaymath}
\end{example}

En algunos casos en necesario indicar el tamaño correcto de un delimitador matemático\index{matemático!delimitador} a mano, lo que puede hacerse con las órdenes \ci{big}, \ci{Big}, \ci{bigg} y \ci{Bigg} como prefijos de la mayoría de las órdenes de delimitador.\footnote{Estas órdenes no funcionan bien si se usa una orden de cambio de tamaño, o si se indican las opciones \texttt{11pt} o \texttt{12pt}.  Use los paquetes \pai{exscale} o \pai{amsmath} para corregir este comportamiento.}
\begin{example}
$\Big( (x+1) (x-1) \Big) ^{2}$\\
$\big(\Big(\bigg(\Bigg($\quad
$\big\}\Big\}\bigg\}\Bigg\}$
\quad
$\big\|\Big\|\bigg\|\Bigg\|$
\end{example}

 Hay varias órdenes para introducir \textbf{\wi{tres puntos}} en una fórmula. \ci{ldots} compone los puntos en la línea de base y \ci{cdots} los coloca centrados.  Además, están las órdenes \ci{vdots} para puntos verticales y \ci{ddots} para \wi{puntos diagonales}.\index{puntos verticales}\index{puntos horizontales} Puede entrontrar otro ejemplo en la sección~\ref{sec:vert}.
 \begin{example}
 \begin{displaymath}
 x_{1},\ldots,x_{n} \qquad
 x_{1}+\cdots+x_{n}
 \end{displaymath}
 \end{example}
 
\section{Espaciado en matemáticas}
\index{espaciado matemático} 

Si los espacios en las fórmulas elegidos por \TeX{} no son satisfactorios, pueden ajustarse insertando órdenes de espaciado especiales.  Hay varias órdenes para espacios pequeños: \ci{,} para $\frac{3}{18}\:\textrm{de cuadratín}$ (\demowidth{0.166em}), \ci{:} para $\frac{4}{18}\: \textrm{de cuadratín}$ (\demowidth{0.222em}) y \ci{;} para $\frac{5}{18}\: \textrm{de cuadratín}$ (\demowidth{0.277em}).  Es carácter espacio escapado \verb*.\ . genera un espacio de tamaño medio y \ci{quad} (\demowidth{1em}) y \ci{qquad} (\demowidth{2em}) producen espacios anchos.  El tamaño de un cuadratín \ci{quad} corresponde a la anchura del carácter `M' de la \fontnomo{} actual.  La orden \verb|\!|\cih{"!} produce un espacio negativo de $-\frac{3}{18}\:\textrm{de cuadratín}$
(\demowidth{0.166em}).
\begin{example}
\newcommand{\ud}{\mathrm{d}}
\begin{displaymath}
\int\!\!\!\int_{D} g(x,y)
  \, \ud x\, \ud y 
\end{displaymath}
en lugar de
\begin{displaymath}
\int\int_{D} g(x,y)\ud x \ud y
\end{displaymath}
\end{example}
Fíjese en que la ``d'' del diferencial se compone recta por convención.

\AmS-\LaTeX{} proporciona otra manera de afinar el espaciado entre múltiples signos integrales, mediante las órdenes \ci{iint}, \ci{iiint}, \ci{iiiint} y \ci{idotsint}.  Con el paquete \pai{amsmath} cargado, el ejemplo de arriba puede componerse así:
\begin{example}
\newcommand{\ud}{\mathrm{d}}
\begin{displaymath}
\iint_{D} \, \ud x \, \ud y
\end{displaymath}
\end{example}

Vea el documento electrónico testmath.tex (distribuido con \AmS-\LaTeX) o el capítulo 8 de \companion{} para más detalles.

\section{Material alineado verticalmente}
\label{sec:vert}

Para componer \textbf{matrices}, use el entorno \ei{array}.  Funciona más o menos como el entorno \texttt{tabular}.  La orden \verb|\\| se usa para cambiar de fila.
\begin{example}
\begin{displaymath}
\mathbf{X} =
\left( \begin{array}{ccc}
x_{11} & x_{12} & \ldots \\
x_{21} & x_{22} & \ldots \\
\vdots & \vdots & \ddots
\end{array} \right)
\end{displaymath}
\end{example}

El entorno \ei{array} también puede usarse para componer expresiones que tienen un delimitador grande usando  ``\verb|.|'' como un delimitador derecho (\ci{right}):
\begin{example}
\begin{displaymath}
y = \left\{ \begin{array}{ll}
 a & \textrm{si $d>c$}\\
 b+x & \textrm{por la mañana}\\
 l & \textrm{el resto del día}
  \end{array} \right.
\end{displaymath}
\end{example}

Al igual que con el entorno \verb|tabular|, puede también dibujar líneas en el entorno  \ei{array}, p.ej. separando los elementos de una matriz:
\begin{example}
\begin{displaymath}
\left(\begin{array}{c|c}
 1 & 2 \\
\hline
3 & 4
\end{array}\right)
\end{displaymath}
\end{example}


Para fórmulas que ocupan varios renglones o para \wi{sistemas de ecuaciones}, puede usar los entornos \ei{eqnarray} y \verb|eqnarray*| en lugar de \texttt{equation}. En \texttt{eqnarray} cada renglón lleva un número de ecuación;  en \verb|eqnarray*| no se numera ninguno.

Los entornos \texttt{eqnarray} y \verb|eqnarray*| funcionan como una tabla de tres columnas de la forma \verb|{rcl}|, donde la columna del medio puede usarse para el signo \emph{igual}, el signo \emph{distinto} o cualquier otro signo que quiera poner.  La orden \verb|\\| cambia de renglón.
\begin{example}
\begin{eqnarray}
f(x) & = & \cos x     \\
f'(x) & = & -\sin x   \\
\int_{0}^{x} f(y)dy &
 = & \sin x
\end{eqnarray}
\end{example}
Tenga en cuenta que el espacio en ambos lados del signo \emph{igual} es bastante grande.  Puede reducirse poniendo \verb|\setlength\arraycolsep{2pt}|, como en el siguiente ejemplo.

\index{ecuaciones largas} Las \textbf{ecuaciones largas} no se dividen automáticamente en trozos adecuados.  El autor ha de indicar dónde partirlas y cuánto sangrar los trozos.  Los siguientes dos métodos son los más habituales para conseguirlo. \begin{example}
{\setlength\arraycolsep{2pt}
\begin{eqnarray}
\sin x & = & x -\frac{x^{3}}{3!}
     +\frac{x^{5}}{5!}-{}
                    \nonumber\\
&& {}-\frac{x^{7}}{7!}+{}\cdots
\end{eqnarray}}
\end{example}
\begin{example}
\begin{eqnarray}
\lefteqn{ \cos x = 1
     -\frac{x^{2}}{2!} +{} }
                    \nonumber\\
 & & {}+\frac{x^{4}}{4!}
     -\frac{x^{6}}{6!}+{}\cdots
\end{eqnarray}
\end{example}

%\enlargethispage{\baselineskip}
La orden \ci{nonumber} dice a \LaTeX{} que no genere un número para la correspondiente ecuación.

Puede resultar difícil conseguir ecuaciones alineadas en vertical de forma satisfactoria con estos métodos; el paquete \pai{amsmath} proporciona un conjunto de alternativas más potentes. (Véanse los entornos \verb|align|, \verb|flalign|, \verb|gather|, \verb|multline| y \verb|split|.)

\section{Fantasmas}

No podemos ver a los fantasmas, pero ocupan algo de espacio (al menos en la mente de mucha gente). \LaTeX{} no es diferente.  Podemos aprovechar esto para conseguir ciertos efectos interesantes con el espaciado.

Al alinear verticalmente texto usando \verb|^| y \verb|_| \LaTeX{} a veces se pasa un poco de listo.  Mediante la orden \ci{phantom} puede reservar espacio para caracteres que no se muestran en la salida final.  La forma más fácil de entenderlo es fijarse en los siguientes ejemplos.
\begin{example}
\begin{displaymath}
{}^{12}_{\phantom{1}6}\textrm{C}
\qquad \textrm{frente a} \qquad
{}^{12}_{6}\textrm{C}
\end{displaymath}
\end{example}
\begin{example}
\begin{displaymath} 
\Gamma_{ij}^{\phantom{ij}k}
\qquad \textrm{frente a} \qquad
\Gamma_{ij}^{k}
\end{displaymath}  
\end{example}

\section{Tamaño de \fontnomo{} en matemáticas}\label{sec:fontsz}

\index{tamaño de \fontnomo{} en matemático} En modo matemático, \TeX{} elige el tamaño de \fontnomo{} según el contexto.  Superíndices, por ejemplo, se componen con una \fontnomo{} más pequeña.  Si quiere componer parte de una ecuación con letra recta, no use la orden \verb|\textrm|, porque el mecanismo de cambio de tamaño de \fontnomo{} no funcionará, pues \verb|\textrm| se escapa temporalmente a modo texto.  Use \verb|\mathrm| en su lugar para mantener activo el mecanismo de cambio.  Pero esté atento, \ci{mathrm} sólo funcionará bien sobre argumentos cortos.  Los espacios no estarán activos y los caracteres acentuados no funcionarán.\footnote{El paquete  \AmS-\LaTeX{} (\pai{amsmath}) permite que la orden \ci{textrm} funcione con el cambio de tamaño.}
\begin{example}
\begin{equation}
2^{\textrm{nd}} \quad 
2^{\mathrm{nd}}
\end{equation}
\end{example}

A veces tendrá que indicar a  \LaTeX{} el tamaño de \fontnomo{} correcto.  En modo matemático, éste se establece con las siguientes cuatro órdenes:
\begin{flushleft}
\ci{displaystyle}~($\displaystyle 123$),
 \ci{textstyle}~($\textstyle 123$), 
\ci{scriptstyle}~($\scriptstyle 123$) and
\ci{scriptscriptstyle}~($\scriptscriptstyle 123$).
\end{flushleft}

El cambio de estilo afecta también al modo en que se muestran los límites.
\begin{example}
\begin{displaymath}
\frac{\displaystyle 
\sum_{i=1}^n(x_i-\overline x)
  (y_i-\overline y)} 
  {\displaystyle\biggl[
\sum_{i=1}^n(x_i-\overline x)^2
\sum_{i=1}^n(y_i-\overline y)^2
\biggr]^{1/2}}
\end{displaymath}    
\end{example}
% This is not a math accent, and no maths book would be set this way.
% mathop gets the spacing right.

Este es un ejemplo con corchetes más grandes que los que proporciona \verb|\left[ \right]|.  Las órdenes \ci{biggl} y \ci{biggr} se usan para paréntesis izquierdos y derechos respectivamente.


\section{Lemas, teoremas, corolarios, \ldots}

Al escribir documentos matemáticos, probablemente necesite una manera de componer ``Lemas'', ``Definiciones'', ``Axiomas'' y estructuras similares. Esto se hace con la orden \verb!newtheorem!. 
\begin{lscommand}
\ci{newtheorem}\verb|{|\emph{nombre}\verb|}[|\emph{contador}\verb|]{|%
         \emph{texto}\verb|}[|\emph{sección}\verb|]|
\end{lscommand}
El argumento \emph{nombre} es una palabra corta usada para identificar el tipo de ``teorema''.  Con el argumento \emph{texto} se define el nombre real del ``teorema'', que aparecerá en el documento final.

Los argumentos entre corchetes son opcionales.  Se usan ambos para indicar  la numeración usada en el ``teorema''.  Use el argumento \emph{contador} para indicar el \emph{nombre} de un ``teorema'' declarado con anterioridad.  El nuevo ``teorema'' se numerará en la misma secuencia.  El argumento \emph{sección} le permite indicar una unidad de sección de la cual el ``teorema''tomará sus números.

Tras ejecutar la orden \ci{newtheorem} en el preámbulo de su documento, puede usar la siguiente orden dentro del documento.
\begin{code}
\verb|\begin{|\emph{nombre}\verb|}[|\emph{texto}\verb|]|\\
Este es mi interesante teorema\\
\verb|\end{|\emph{nombre}\verb|}|     
\end{code}

El paquete  \pai{amsthm} proporciona la orden \ci{newtheoremstyle}\verb|{|\emph{estilo}\verb|}| que le permite definir sobre qué va el teorema escogiendo entre tres estilos predefinidos: \texttt{definition} (título en negrita, cuerpo en recta), \texttt{plain} (título en negrita, cuerpo en cursiva) o \texttt{remark} (título en cursiva, cuerpo en recta).

Esto debería bastar como teoría.  Los siguientes ejemplos deberían despejar las dudas restantes, y dejar claro que el entorno \verb|\newtheorem| es demasiado complejo de entender.

% actually define things
\theoremstyle{definition} \newtheorem{ley}{Ley}
\theoremstyle{plain}      \newtheorem{jurado}[ley]{Jurado}
\theoremstyle{remark}     \newtheorem*{marg}{Margarita}

Primero defina los teoremas:

\begin{verbatim}
\theoremstyle{definition} \newtheorem{ley}{Ley}
\theoremstyle{plain}      \newtheorem{jurado}[ley]{Jurado}
\theoremstyle{remark}     \newtheorem*{marg}{Margarita}
\end{verbatim}

\begin{example}
\begin{ley} \label{ley:caja}
No esconder en la caja negra
\end{ley}
\begin{jurado}[Los Doce]
¡Podría ser usted!  Cuidado y
vea la ley~\ref{ley:caja}
\end{jurado}
\begin{marg}No, No, No\end{marg}
\end{example}

El teorema ``Jurado'' usa el mismo contador que el teorema ``Ley'', así que le corresponde un número en secuencia con las otras ``Leyes''.  El argumento entre corchetes se usa para indicar un título o algo similar para el teorema.
\begin{example}
\flushleft
\newtheorem{mur}{Murphy}[section]
\begin{mur}
Si hay dos o más formas de
hacer algo, y una de ellas
puede resultar catastrófica,
entonces alguien la escogerá.
\end{mur}
\end{example}

El teorema ``Murphy'' recibe un número que está ligado al número de la sección actual.  Podría usar otra unidad, como por ejemplo \verb+chapter+ o \verb+subsection+.

El paquete \pai{amsthm} también proporciona \ei{proof} para demostraciones.

\begin{example}
\begin{proof}
 Trivial, use
\[E=mc^2\]
\end{proof}
\end{example}

Con la orden \ci{qedhere} puede mover el `símbolo de fin de demostración' para las situaciones en que terminaría solo en un renglón.

\begin{example}
\begin{proof}
 Trivial, use
\[E=mc^2 \qedhere\]
\end{proof}
\end{example}

\section{Símbolos en negrita}
\index{símbolos en negrita}

Es bastante difícil conseguir símbolos en negrita en  \LaTeX{}; probablemente es a propósito, pues los compositores aficionados tienden a abusar de ellos.  La orden de cambio de \fontnomo{} \verb|\mathbf| da letras en negrita, pero éstas son rectas mientras que los símbolos matemáticos son normalmente en cursiva.  Hay una orden \ci{boldmath}, pero \emph{sólo puede usarse fuera del modo matemático}.  Funciona también para símbolos.
\begin{example}
\begin{displaymath}
\mu, M \qquad \mathbf{M} \qquad
\mbox{\boldmath $\mu, M$}
\end{displaymath}
\end{example}

%\noindent
Fíjese en que la coma también es negrita, lo que puede no ser lo que se pretende.

El paquete \pai{amsbsy} (incluido por \pai{amsmath}) y también el \pai{bm} facilitan la labor al proporcionar la orden \ci{boldsymbol}.
\ifx\boldsymbol\undefined\else
\begin{example}
\begin{displaymath}
\mu, M \qquad
\boldsymbol{\mu}, \boldsymbol{M}
\end{displaymath}
\end{example}
\fi

%

% Local Variables:
% TeX-master: "lshort2e"
% mode: latex
% mode: flyspell
% End: