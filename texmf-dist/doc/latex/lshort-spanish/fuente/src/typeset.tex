%%%%%%%%%%%%%%%%%%%%%%%%%%%%%%%%%%%%%%%%%%%%%%%%%%%%%%%%%%%%%%%%%
% Contents: Typesetting Part of LaTeX2e Introduction
%%%%%%%%%%%%%%%%%%%%%%%%%%%%%%%%%%%%%%%%%%%%%%%%%%%%%%%%%%%%%%%%%
\chapter{Composición de texto}

\begin{intro}
    Tras leer el capítulo previo, debería conocer lo básico para entender de qué está hecho un documento \LaTeXe{}.  En este capítulo se explica el resto de la estructura que se necesita saber para producir un documento útil. 
\end{intro}

\section{La estructura del texto y el idioma}
\secby{Hanspeter Schmid}{hanspi@schmid-werren.ch}

El quid de escribir un texto (salvo cierta literatura moderna) es comunicar ideas, información o conocimiento al lector.  El lector entenderá mejor el texto si dichas ideas están bien estructuradas, y verá y sentirá dicha estructura mucho mejor si la forma tipográfica refleja la estructura lógica y semántica del contenido.

\LaTeX{} se diferencia de otros sistemas de composición en que sólo tiene que decirle tal estructura.  La forma tipográfica del texto se deriva según las ``reglas'' dadas en el \filenomo{} de clase del documento y en los varios \filenomo{}s de estilo usados.

La unidad de texto más importante en \LaTeX{} (y en tipografía) es el \wi{párrafo}.  Lo llamamos ``unidad de texto'' porque un párrafo es la forma tipográfica que debería reflejar un pensamiento o una idea básica completos.  Así, si comienza un nuevo pensamiento, debería empezar un nuevo párrafo; y si no, deberían usarse sólo saltos de línea.  Si duda sobre insertar saltos de párrafo, recuerde que su texto es un vehículo de ideas y pensamientos.  Si tiene un salto de párrafo, pero el anterior pensamiento continúa, debería eliminar el salto.  Si aparece una línea de pensamiento totalmente nueva en el mismo párrafo, entonces debería insertar un salto.

Casi todo el mundo subestima completamente la importancia de saltos de párrafo bien situados.  Mucha gente no sabe siquiera cuál es el significado de un salto de párrafo o, especialmente en \LaTeX, introduce saltos de párrafo sin saberlo.  Este último error es especialmente fácil de cometer si se usan ecuaciones en el texto. Mire los siguientes ejemplos, y piense por qué a veces se usan líneas vacías (saltos de párrafo) antes y después de la ecuación, y a veces no.  (Si no entiende bien todavía todas las órdenes para entender estos ejemplos, lea este capítulo y el siguiente y luego lea esta sección otra vez.)

\begin{code}
\begin{verbatim}
% Ejemplo 1
...cuando Einstein presentó su fórmula 
\begin{equation} 
  e = m \cdot c^2 \; , 
\end{equation} 
que es al mismo tiempo la fórmula física 
más famosa y la menos entendida.


% Ejemplo 2
...de lo cual se sigue la ley de corrientes de Kirchhoff:
\begin{equation} 
  \sum_{k=1}^{n} I_k = 0 \; .
\end{equation} 

La ley de tensiones de Kirchhoff puede derivarse...


% Ejemplo 3
...lo que tiene varias ventajas.

\begin{equation} 
  I_D = I_F - I_R
\end{equation} 
es el núcleo de un modelo de transistor muy eficiente. ...
\end{verbatim}
\end{code} 

La siguiente unidad de texto más pequeña es la oración.  En textos ingleses, hay un espacio mayor tras un punto que termina una oración que tras uno que termina una abreviatura.  \LaTeX{} supone por omisión que un punto termina una oración; si se equivoca, debe indicarle qué es lo que desea.  Esto se explicará más tarde en este capítulo.  Afortunadamente, en español no afecta tanto esta distinción.

La estructuración de un texto se extiende incluso a partes de las oraciones.  La mayoría de los idiomas tienen reglas de puntuación muy complicadas, pero en muchos idiomas (incluido el español) acertará casi siempre con las comas si recuerda lo que representan: una pausa breve en el flujo del lenguaje.  Si no está seguro de dónde poner una coma, lea la oración en alto y tómese un breve respiro en cada coma. Si le suena mal en algún lugar, borre esa coma; si siente que le urge respirar (o hacer una breve pausa) en otro lugar, inserte una coma.

Finalmente, los párrafos de un texto deberían estar estructurados también a un nivel más alto, distribuyéndose en capítulos, secciones, subsecciones, y así sucesivamente.  Sin embargo, el efecto tipográfico de escribir p.ej.{} \verb|\section{La| \texttt{estructura del texto y el idioma}\verb|}| es tan obvio que es casi evidente cómo deben usarse estas estructuras de alto nivel.

\section{Saltos de línea y de página}
 
\subsection{Justificación de párrafos}

Los libros se suelen componer con líneas de igual longitud.  \LaTeX{} inserta los \wi{saltos de línea} y los espacios necesarios entre palabras optimizando el contenido de todo un párrafo.  Si es preciso, también divide palabras con guiones si no caben bien en una línea. Cómo se componen los párrafos depende de la clase del documento. Normalmente la primera línea de un párrafo lleva sangría, y no hay espacio adicional entre dos párrrafos.  Tiene más información al respecto en la sección~\ref{parsp}.

En casos concretos puede ser necesario ordenar a \LaTeX{} que salte de línea: 
\begin{lscommand}
\ci{\bs} ó \ci{newline} 
\end{lscommand}
comienza una nueva línea sin comenzar un nuevo párrafo.

\begin{lscommand}
\ci{\bs*}
\end{lscommand}
además prohíbe un salto de página tras el salto forzado de línea.

\begin{lscommand}
\ci{newpage}
\end{lscommand}
comienza una nueva página. 

\begin{lscommand}
\ci{linebreak}\verb|[|\emph{n}\verb|]|,
\ci{nolinebreak}\verb|[|\emph{n}\verb|]|, 
\ci{pagebreak}\verb|[|\emph{n}\verb|]|,
\ci{nopagebreak}\verb|[|\emph{n}\verb|]|
\end{lscommand}

producen un salto de línea, impiden un salto de línea, producen un salto de página, o impiden un salto de página, respectivamene.  Permiten al autor ajustar sus efectos mediante el argumento opcional \emph{n}, al que puede asignarse un número entre cero y cuatro.  Poniendo \emph{n} a un valor menor que 4, deja a \LaTeX{} la opción de no hacer caso de su orden si el resultado tiene mal aspecto.  No confunda estas órdenes ``-break''con las órdenes ``new-''.  Incluso si pone una orden ``-break'', \LaTeX{} aún intenta dejar bien el borde derecho de la página y la longitud total de la página, como se describe en la sección siguiente. Si realmente quiere iniciar una nueva línea, use la orden ``newline''.

\LaTeX{} siempre intenta producir los mejores saltos de página posibles.  Si no puede encontrar una manera de dividir las líneas que cumpla con sus expectativas, permite que una línea se salga por la derecha del párrafo.  \LaTeX{} se queja entonces (``\wi{overfull hbox}'') mientras procesa el \filenomo{} de entrada.  Esto sucede muy a menudo cuando \LaTeX{} no puede encontrar un lugar adecuado para dividir una palabra.\footnote{Aunque \LaTeX{} le avisa cuando ocurre (Overfull hbox) y muestra la línea problemática, tales líneas no siempre son fáciles de encontrar.  Si usa la opción \texttt{draft} en la orden \ci{documentclass}, tales líneas se marcarán con una línea negra gruesa en el margen derecho.}  Puede mandar a \LaTeX{} que baje sus expectativas un poco mediante la orden \ci{sloppy}.  Impide las líneas extra-largas incrementando el espaciado permitido entre palabras ---aunque la salida final no sea óptima---.  En tal caso se advierte al usuario (``\wi{underfull hbox}'').  En la mayoría de los casos el resultado no tiene muy buen aspecto.  La orden \ci{fussy}, por el contrario, indica a \LaTeX{} que debe ser más exigente en sus elecciones.

\subsection{Silabación} \label{hyph}

\LaTeX{} divide las palabras según sus sílabas al final del renglón si lo considera necesario.  Si el algoritmo de división no encuentra los puntos de silabación correctos, puede remediar la situación usando las siguientes órdenes para decirle a \TeX{} las excepciones.
La orden
\begin{lscommand}
\ci{hyphenation}\verb|{|\emph{lista de palabras}\verb|}|
\end{lscommand}
causa que las palabras listadas en el argumento se dividan
sólo en los puntos marcados con ``\verb|-|''.  El argumento de la orden debería contener sólo palabras de letras normales o, mejor dicho, signos que \LaTeX{} considera letras normales.  Las sugerencias de silabación se almacenan para el idioma activo mientras se da la orden.  Esto quiere decir que si da la orden en el preámbulo del documento entonces influirá la silabación del inglés.  Si sitúa la orden tras \verb|\begin{document}| y está usando algún paquete para otro idioma como \pai{babel}, entonces las sugerencias de silabación estarán activas para el idioma activo de \pai{babel}.

El ejemplo de abajo permitirá que ``guiones'' se divida, y también ``Guiones''; e impedirá que ``FORTRAN'', ``Fortran'' y ``fortran''se dividan en ningún caso.  Sólo se permiten caracteres ASCII (no las vocales acentuadas ni la eñe) en el argumento.

Ejemplo:
\begin{code}
\verb|\hyphenation{FORTRAN Gui-o-nes}|
\end{code}

La orden \ci{-} inserta un guión discrecional en una palabra, que se convierte en el único punto donde se permite la división en dicha palabra.  Esta orden es útil sobre todo para palabras que contienen caracteres especiales (p.ej.{} vocales acentuadas), porque \LaTeX{} no divide automáticamente tales palabras.
%\footnote{Unless you are using the new
%\wi{DC fonts}.}.

\begin{example}
Me parece que es: su\-per\-ca\-%
li\-fra\-gi\-lís\-ti\-co\-es\-%
pia\-li\-do\-so
\end{example}

Para mantener varias palabras juntas en el mismo renglón use la orden
\begin{lscommand}
\ci{mbox}\verb|{|\emph{texto}\verb|}|
\end{lscommand}
que causa que su argumento quede junto en todas las circunstancias.

\begin{example}
Mi número telefónico pasará 
a ser \mbox{677 843 860} a 
partir de mañana.

El parámetro indicado como
\mbox{\emph{nombre\filenomo{}}}
contiene el nombre del \filenomo{}.
\end{example}

\ci{fbox} es similar a \ci{mbox}, pero además dibujará un rectángulo
visible alrededor del argumento.


\section{Cadenas a medida}

En algunos ejemplos de las páginas anteriores, ha visto algunas órdenes simples de \LaTeX{} para componer cadenas de texto especiales:

\vspace{2ex}

\noindent
\begin{tabular}{@{}lll@{}}
Orden&Ejemplo&Descripción\\
\hline
\ci{today} & \today   & Fecha de hoy\\
\ci{TeX} & \TeX       & Su compositor favorito\\
\ci{LaTeX} & \LaTeX   & El nombre del hombre\\
\ci{LaTeXe} & \LaTeXe & La encarnación actual\\
\end{tabular}

\section{Símbolos y caracteres especiales}
 
\subsection{Comillas}

\emph{No} use \verb|"| para las \wi{comillas} \index{""@\texttt{""}} como haría con una máquina de escribir.  En tipografía hay comillas especiales de apertura y cierre.  En \LaTeX{}, use dos~\textasciigrave~(acentos graves) para abrir comillas y dos~\textquotesingle~(apóstrofos) para cerrar comillas \emph{inglesas}.  Para comillas \emph{inglesas} simples basta con poner una de cada una.
\begin{example}
``Por favor, pulse la tecla `x'.''
\end{example}
Tenga en cuenta que el apóstrofo aparece en el código fuente anterior como un acento agudo (simétrico al grave).

En la tipografía española, las comillas tradicionales son \guillemotleft{} y \guillemotright{}.  La versión española debería ser así:
\begin{example}
\guillemotleft
Por favor, pulse la tecla ``x''.%
\guillemotright
\end{example}

\subsection{Guiones y rayas}

\LaTeX{} conoce cuatro tipos de \wi{guión} o \wi{raya}, uno de los cuales es el signo matemático ``menos''.  Observe cómo obtenerlos: \index{-} \index{--} \index{---} \index{-@$-$} \index{matemático!menos}

\begin{example}
austro-húngaro, P-valor\\
páginas 13--67\\
sí ---dijo él--- \\
$0$, $1$ y $-1$
\end{example}
Los nombres de estos símbolos son: 
`-' \wi{guión}, `--' \wi{raya corta}, `---' \wi{raya} y `$-$' \wi{signo menos}.  (En tipografía tradicional española, la rayacorta no existe; en su lugar se utiliza siempre el guión.)

\subsection[Tilde]{Tilde ($\sim$)}
\index{www}\index{URL}\index{tilde} 

Se trata de un carácter que aparece a menudo en código informático y direcciones de red.  Para generarlo en \LaTeX{} puede usar \verb|\~| pero el resultado: \~{} no es realmente lo que busca.  Intente esto otro:

\begin{example}
http://www.rich.edu/\~{}rockefeller \\
http://www.clever.edu/$\sim$tesla
\end{example}  
 
\subsection{Slash (/)}
\index{Slash}
Para introducir una barra entre dos palabras, se puede simplemente escribir, por ejemplo, \texttt{read/write}, pero esto hace que \LaTeX{} trate a las dos palabras como una sola, e inhibe la división silábica 
en estas dos palabras, de modo que puede haber errores de cajas horizontales rebasadas (`\emph{overfull}').  
Para evitar esto, use \ci{slash}.  
Escriba, por ejemplo, `\verb|read\slash write|', lo que permite la partición silábica, pero puede aún usar el caracter de la barra normal `\texttt{/}' para cocientes o unidades, por ejemplo: \texttt{5 MB/s}.

\subsection{Símbolo de grado \texorpdfstring{($\circ$)}{}}

El siguiente ejemplo muestra cómo imprimir un símbolo de \wi{grado} en \LaTeX{}:

\begin{example}
Estamos a 
$-30\,^{\circ}\mathrm{C}$.
Pronto superconduciremos.
\end{example}

El paquete \pai{textcomp} dispone de la orden \ci{textcelsius} para producir el mismo símbolo sin tener que usar superíndices (\verb|^|).

\subsection{El símbolo monetario del euro \texorpdfstring{(\officialeuro)}{}}

Si escribe sobre dinero, casi seguro que necesite el símbolo del euro.  Muchas \fontsnomo{} actuales contienen el símbolo del euro. Tras cargar el paquete \pai{textcomp} en el preámbulo de su documento
\begin{lscommand}
\ci{usepackage}\verb|{textcomp}| 
\end{lscommand}
puede usar la orden
\begin{lscommand}
\ci{texteuro}
\end{lscommand}
para acceder a él.

Si su \fontnomo{} no proporciona su propio símbolo del euro o si no le gusta el símbolo de la \fontnomo{}, tiene más opciones:

Primero, el paquete \pai{eurosym}.  Proporciona el símbolo oficial del euro:
\begin{lscommand}
\ci{usepackage}\verb|[official]{eurosym}|
\end{lscommand}
Si prefiere un símbolo del euro que se ajuste a su \fontnomo{}, use la
opción \texttt{gen} el lugar de la opción \texttt{official}.

%If the Adobe Eurofonts are installed on your system (they are available for
%free from \url{ftp://ftp.adobe.com/pub/adobe/type/win/all}) you can use
%either the package \pai{europs} and the command \ci{EUR} (for a Euro symbol
%that matches the current font).
% does not work
% or the package
% \pai{eurosans} and the command \ci{euro} (for the ``official Euro'').

El paquete \pai{marvosym} también proporciona muchos y variados símbolos, incluido el del euro, con el nombre \ci{EUR} (y otras versiones como \ci{EURtm}).

\begin{table}[!htbp]
\caption{Una recopilación de euros} \label{eurosymb}
\begin{lined}{10cm}
\begin{tabular}{llccc}
LM+textcomp  &\verb+\texteuro+ & \huge\texteuro &\huge\sffamily\texteuro
                                                &\huge\ttfamily\texteuro\\
eurosym      &\verb+\euro+ & \huge\officialeuro &\huge\sffamily\officialeuro
                                                &\huge\ttfamily\officialeuro\\
$[$gen$]$eurosym &\verb+\euro+ & \huge\geneuro  &\huge\sffamily\geneuro
                                                &\huge\ttfamily\geneuro\\
%europs       &\verb+\EUR + & \huge\EURtm        &\huge\EURhv
%                                                &\huge\EURcr\\
%eurosans     &\verb+\euro+ & \huge\EUROSANS  &\huge\sffamily\EUROSANS
%                                             & \huge\ttfamily\EUROSANS \\
marvosym     &\verb+\EUR+  & \huge\mvchr{101}  &\huge\mvchr{99}
                                               &\huge\mvchr{100}
\end{tabular}
\medskip
\end{lined}
\end{table}

\subsection{Puntos suspensivos (\texorpdfstring{\ldots}{...})}

En una máquina de escribir, una \wi{coma} o un \wi{punto} ocupa el mismo espacio que cualquier otra letra.  En tipografía, estos caracteres ocupan muy poco espacio y casi se pegan a la letra anterior.  En tipografía española esto no es un problema, porque los `\wi{puntos suspensivos}' van casi juntos.  En tipografía inglesa no, así que en lugar de escribir tres puntos use la orden

\begin{lscommand}
\ci{ldots}
\end{lscommand}
\index{...@\ldots}

\begin{example}
Not like this ... but like this:\\
New York, Tokyo, Budapest, \ldots
\end{example}

En español sería:

\begin{example}
Puntos en medio\... y al final:\\
Nueva York, Tokio, Budapest...
\end{example}
 
\subsection{Ligaduras}

Algunas combinaciones de letras se componen no sólo poniendo una letra tras otra, sino usando símbolos especiales.
\begin{code}
{\large ff fi fl ffi\ldots}\quad
en lugar de\quad {\large f{}f f{}i f{}l f{}f{}i \ldots}
\end{code}
Las llamadas \wi{ligadura}s pueden evitarse insertando \ci{mbox}\verb|{}| entre las dos letras en cuestión.  Esto puede ser necesario para palabras compuestas de dos palabras (raro en castellano, pero común en otros idiomas).

\begin{example}
\Large No ``\^ceffarbisto''\\
sino ``\^cef\mbox{}farbisto''.
\end{example}
 
\subsection{Acentos y caracteres especiales}
 
\LaTeX{} soporta el uso de \wi{acento}s y \wi{caracteres especiales} para muchos idiomas.  El cuadro~\ref{accents} muestra todo tipo de acentos aplicados a la letra o. Por supuesto también funcionan con otras letras (vocales o consonantes).

Para situar un acento sobre una i o una j, hay que quitar sus puntos. Esto se consigue escribiendo \verb|\i| y \verb|\j|.

\begin{example}
H\^otel, na\"\i ve, \'el\`eve,\\ 
sm\o rrebr\o d, !`Se\~norita!,\\
Sch\"onbrunner, Schlo\ss{},
Stra\ss e,\\
\^Ce\^ha \^sa\u umman\^ga\^\j o
\end{example}

\begin{table}[!hbp]
\caption{Acentos y caracteres especiales.} \label{accents}
    \begin{lined}{10cm}
        \begin{tabular}{*4{cl}}
        \A{\`o} & \A{\'o} & \A{\^o} & \A{\~o} \\
        \A{\=o} & \A{\.o} & \A{\"o} & \B{\c}{c}\\[6pt]
        \B{\u}{o} & \B{\v}{o} & \B{\H}{o} & \B{\c}{o} \\
        \B{\d}{o} & \B{\b}{o} & \B{\t}{oo} \\[6pt]
        \A{\oe}  &  \A{\OE} & \A{\ae} & \A{\AE} \\
        \A{\aa} &  \A{\AA} \\[6pt]
        \A{\o}  & \A{\O} & \A{\l} & \A{\L} \\
        \A{\i}  & \A{\j} & !` & \verb|!`| & ?` & \verb|?`| 
        \end{tabular}
    \index{i y j sin punto (\i{} y \j)}\index{escandinavas letras}
    \index{ae@\ae}\index{umlaut}\index{grave}\index{agudo}
    \index{oe@\oe}\index{aa@\aa}

    \bigskip
    \end{lined}
\end{table}

\section{Soporte para otros idiomas}
\index{international} 

Cuando escriba documentos en \wi{idioma}s distintos del español, hay tres áreas en que \LaTeX{} tiene que configurarse adecuadamente:

\begin{enumerate}
    \item Todas las cadenas de texto generadas automáticamente\footnote{Índice general, Apéndice,~...} tienen que adaptarse al nuevo idioma.  Para muchos idiomas, estos cambios pueden llevarse a cabo mediante el paquete \pai{babel} de Johannes Braams.
    \item \LaTeX{} necesita saber las reglan de silabación para el nuevo idioma.  Crear reglas de silabación para  \LaTeX{} es algo más difícil.  Significa reconstruir el \filenomo{} de formato con patrones de silabación diferentes.  Su \guide{} debería darle más información sobre esto.
    \item Reglas tipográficas específicas del idioma.  En francés, por ejemplo, hay un espacio obligatorio antes de cada carácter de dos puntos (:).
\end{enumerate}

Si su sistema ya está configurado adecuadamente, puede activar el paquete  \pai{babel} añadiendo la orden
\begin{lscommand}
\ci{usepackage}\verb|[|\emph{idioma}\verb|]{babel}|
\end{lscommand}
tras la orden \verb|\documentclass|.  Puede listar los \emph{idioma}s construidos en su sistema \LaTeX{} cada vez que se ejecuta el compilador.  Babel activará automáticamente las reglas de silabación para el idioma que escoja.  Si su formato \LaTeX{} no soporta la silabación del idioma escogido, babel funcionará todavía pero desactivará la silabación, lo que tiene un efecto bastante negativo en la apariencia del documento compuesto.

\textsf{Babel} también define nuevas órdenes para algunos idiomas, que simplifican la escritura de caracteres especiales.  El idioma \wi{alemán}, por ejemplo, contiene muchas diéresis (\"a\"o\"u).  Con \textsf{babel}, puede escribir \"o tecleando \verb|"o| en lugar de~\verb|\"o|.

Si carga babel con múltiples idiomas
\begin{lscommand}
\ci{usepackage}\verb|[|\emph{idiomaA}\verb|,|\emph{idiomaB}\verb|]{babel}| 
\end{lscommand}
entonces el último idioma en la lista de opciones será el activo (es decir, idiomaB); puede usar la orden
\begin{lscommand}
\ci{selectlanguage}\verb|{|\emph{idiomaA}\verb|}|
\end{lscommand}
para cambiar el idioma activo.

%Input Encoding
\newcommand{\ieih}[1]{%
\index{encodings!input!#1@\texttt{#1}}%
\index{input encodings!#1@\texttt{#1}}%
\index{#1@\texttt{#1}}}
\newcommand{\iei}[1]{%
\ieih{#1}\texttt{#1}}
%Font Encoding
\newcommand{\feih}[1]{%
\index{encodings!font!#1@\texttt{#1}}%
\index{font encodings!#1@\texttt{#1}}%
\index{#1@\texttt{#1}}}
\newcommand{\fei}[1]{%
\feih{#1}\texttt{#1}}

La mayoría de los sistemas de \computernomo{} modernos le permiten escribir letras de diferentes alfabetos directamente desde el teclado.  Para manejar varias codificaciones de entrada usadas por diferentes grupos de idiomas en diferentes plataformas  \LaTeX{} emplea el paquete
\pai{inputenc}:
\begin{lscommand}
\ci{usepackage}\verb|[|\emph{codificación}\verb|]{inputenc}|
\end{lscommand}

Cuando use este paquete, debería considerar que otras personas podrían no poder editar sus \filenomo{}s de entrada en sus \computernomo{}es, porque usan una codificación diferente.  Por ejemplo, la a con diéresis \"a en OS/2 tiene el código 132, en sistemas GNU o Unix que usen ISO-LATIN~1 tiene el código 228, mientras que en la codificación cirílica cp1251 para ReactOS o Windows esta letra no existe; así que use este paquete con cuidado.  Las siguientes codificaciones pueden resultarle útiles, dependiendo del sistema en que esté trabajando\footnote{Para saber más sobre codificaciones de entrada soportadas para idiomas con alfabetos latino o cirílico, lea la documentación de \texttt{inputenc.dtx} y \texttt{cyinpenc.dtx} respectivamente.  La sección~\ref{sec:Packages} explica cómo generar la documentación de los paquetes.}:

\begin{center}
\begin{tabular}{l | r | r }
Sistema & \multicolumn{2}{c}{encodings}\\
operativo  & western Latin      & Cyrillic\\
\hline
Mac       &  \iei{applemac} & \iei{macukr}  \\
GNU, Unix &  \iei{latin1}   & \iei{koi8-ru} \\ 
Windows   &  \iei{ansinew}  & \iei{cp1251}  \\
DOS, OS/2 &  \iei{cp850}    & \iei{cp866nav}
\end{tabular}                
\end{center}                 

Si tiene un documento multilingüe con codificaciones que entran en conflicto, considere el uso de \textsc{unicode} a través de la codificación \texttt{utf-8}.

%carleos comenta:
%el paquete \pai{ucs}.

\begin{lscommand}
%\ci{usepackage}\verb|{ucs}|\\ 
%\ci{usepackage}\verb|[|\iei{utf8x}\verb|]{inputenc}| 
%carleos comenta lo anterior y pone
\ci{usepackage}\verb|[|\iei{utf8}\verb|]{inputenc}| 
\end{lscommand}
le permitirá crear \filenomo{}s de entrada \LaTeX{} en \iei{utf-8}, una codificación multi-octeto en que cada carácter puede ocupar desde un octeto hasta cuatro.

La codificación de \fontsnomo{} es una cuestión diferente.  Define en qué posición dentro de una \fontnomo{} \TeX{} se almacena cada letra. Múltiples codificaciones de entrada podrían corresponderse con la misma codificación de \fontnomo{}, lo que reduce el número de \fontsnomo{} requeridas.  Las codificaciones de \fontnomo{} se manejan mediante el paquete \pai{fontenc}: \label{fontenc}
\begin{lscommand}
\ci{usepackage}\verb|[|\emph{codificación}\verb|]{fontenc}|
\index{codificación de \fontnomo{}}
\end{lscommand}
donde \emph{codificación} es la codificación de \fontnomo{}. Es posible cargar varias codificaciones simultáneamente.

La codificación de \fontnomo{} por omisión en \LaTeX{} es \label{OT1} \fei{OT1}, la codificación de la \fontnomo{} original de \TeX{}, Computer Modern.  Contiene sólo los 128 caracteres del conjunto ASCII de 7 bites.  Cuando se requieren caracteres acentuados,  \TeX{} los crea combinando un carácter normal con un acento.  Aunque el resultado parece perfecto, este enfoque impide que la silabación automática funcione en palabras que contienen caracteres acentuados.  Además, algunas letras latinas no pueden crearse combinando un carácter normal con un acento; sin mencionar los casos de alfabetos no latinos, como el griego o el cirílico.

Para evitar estos inconvenientes, se crearon varias \fontsnomo{} de 8 bites similares a CM. Las \fontsnomo{} \emph{Extended Cork} (EC) en la codificación \fei{T1} contienen letras y signos de puntuación para la mayoría de los idiomas europeos basados en el alfabeto latino.  Las \fontsnomo{} LH contienen letras necesarias para componer documentos en idiomas que usan el alfabeto cirílico.  Dado el gran número de caracteres cirílicos, se organizan en cuatro codificaciones de \fontnomo{} ---\fei{T2A}, \fei{T2B}, \fei{T2C} y~\fei{X2}.\footnote{La lista de idiomas soportados por cada codificación puede hallarse en \cite{cyrguide}.}  El grupo CB contiene \fontsnomo{} en la codificación \fei{LGR} para la composición de texto griego.

Usando estas \fontsnomo{} puede mejorar/posibilitar la silabación en documentos de otros idiomas.  Otra ventaja de usar las nuevas \fontsnomo{} similares a CM es que proporcionan \fontsnomo{} de las familias CM en todos los pesos, formas y tamaños ópticamente escalables.

\subsection{Soporte para el castellano}

\secby{José Luis Rivera}{jlrn77<>gmail.com}

Para posibilitar la silabación y cambiar todos los textos automáticos al \wi{castellano}, \index{castellano} \index{espa\~nol} use la orden: 
\begin{lscommand}
\verb|\usepackage[spanish]{babel}|
\end{lscommand}

Como hay muchos acentos en castellano, debería usar
\begin{lscommand}
\verb|\usepackage[latin1]{inputenc}|
\end{lscommand}
para poder meterlos con el teclado, y también
\begin{lscommand}
\verb|\usepackage[T1]{fontenc}|
\end{lscommand}
para que la silabación sea correcta.

Vea el cuadro~\ref{castellano} para un preámbulo adecuado para el castellano.  Note que usamos la codificación de entrada latin1, que puede no ser correcta para su sistema.

\begin{table}[btp]
\caption{Preámbulo para documentos en castellano.} \label{castellano}
    \begin{lined}{5cm}
        \begin{verbatim}
        \usepackage[spanish]{babel}
        \usepackage[latin1]{inputenc}
        \usepackage[T1]{fontenc}
        \end{verbatim}
    \bigskip
    \end{lined}
\end{table}

La opción \verb|spanish| añade algunos atajos (\emph{shorthands}) útiles para la tipografía española del texto o las matemáticas. Estos atajos se explican en el cuadro~\ref{atajoses}.

\newcommand\nm[1]{\unskip\,$^{#1}$}
\newcommand\nt[1]{\quad$^{#1}$\,\ignorespaces}
\newcommand\toprule[1]{\cline{1-#1}\\[-2ex]}
\newcommand\botrule[1]{\\[.6ex]\cline{1-#1}}

 \begin{table}[!t]
 \center\small
 \caption{Abreviaciones} \label{atajoses}
 \vspace{1.5ex}
 \begin{tabular}{l@{\hspace{3em}}l@{\hspace{3em}}l}
 \toprule2
 \verb|'a 'e 'i 'o 'u| & á é í ó ú\nm{a}\\
 \verb|'A 'E 'I 'O 'U| & Á É Í Ó Ú\nm{a}\\
 \verb|'n 'N|          & ñ Ñ\nm{b}\\
 \verb|"u "U|          & "u "U\\
 \verb|"i "I|          & "i "I\\
 \verb|"a "A "o "O|    & Ordinales: 1"a, 1"A, 1"o, 1"O\\
 \verb|"er "ER|        & Ordinales: 1"er, 1"ER\\
 \verb|"c "C|          & "c "C\\
 \verb|"rr "RR|        & rr, pero -r cuando se divide\\
 \verb|"y|             & El antiguo signo para <<y>>\\
 \verb|"-|             & Como \verb|\-|, pero permite más divisiones\\
 \verb|"=|             & Como \verb|-|, pero permite mas divisiones\nm{c}\\
 \verb|"~|             & Guión estilístico\nm{d}\\
 \verb|"+ "+- "+--|    & Como \verb|-|, \verb|--| y \verb|---|, pero sin división\\
 \verb|~- ~-- ~---|    & Lo mismo que el anterior.\\
 \verb|""|             & Permite mas divisiones antes y después\nm{e}\\
 \verb|"/|             & Una barra algo más baja\\
 \verb+"|+             & Divide un logotipo\nm{f}\\
 \verb|"< ">|          & "< ">\\
 \verb|"` "'|          & \verb|\begin{quoting}| \verb|\end{quoting}|\nm{g}\\
 \verb|<< >>|          & Lo mismo que el anterior.\\
 \verb|?` !`|          & ?` !`\nm{h}\\
 \verb|"? "!|          & "? "! alineados con la linea base\nm{i}
 \botrule2
 \end{tabular}
 
 \vspace{1.5ex}

 \begin{minipage}{11cm}
 \footnotesize
 \nt{a} Requieren la opción \verb|activeacute|.
 \nt{b} La forma \verb|~n| no está activada por omisión a partir de 
 la versión 5.
 \nt{c} \verb|"=| viene a ser lo mismo que \verb|""-""|.
 \nt{d} Esta abreviación tiene un uso distinto
 en otras lenguas de babel.
 \nt{e} Como en <<entrada/salida>>.
 \nt{f} Carece de uso en castellano.
 \nt{g} Reemplazos para $<<$ o $>>$ con la opción \verb|es-noquoting|. 
 \nt{h} No proporcionadas por este paquete, sino por cada tipo;
 figuran aquí como simple recordatorio.
 \nt{i} útiles en rótulos en mayúsculas.
 \end{minipage}
 \end{table}

%Hay muchas opciones para el castellano, como veremos a continuación.

En general, la opción \verb|spanish| provee numerosos ajustes, pero el estilo está diseñado para que sea muy configurable. Para ello, se proporciona una serie de opciones de paquete, que en caso de emplearse deben ir \textit{después} de \textsf{spanish}.  Por ejemplo:
\begin{verbatim}
\usepackage[french,spanish,es-noindentfirst]{babel} 
\end{verbatim}
carga los estilos para el francés y el español, esta última como lengua principal; además, evita que \textsf{spanish} sangre el primer párrafo tras un título.  Otras opciones se pueden ajustar por medio de macros, en particular aquellas que se puede desear cambiar en medio del documento (por ejemplo, el formato de la fecha). Las macros pueden incluirse en el archivo de configuración \verb|spanish.cfg| para hacer cambios globales en un sistema completo.

El estilo modifica por omisión el diseño del texto o del documento en partes sustantivas. A continuación se enumeran los ajustes hechos al formato y las opciones y macros que los controlan.

\begin{itemize}
    \item Todo el espacio es uniforme, con \verb|\frenchspacing|.
    \item Se añade un punto después del número de todas las secciones. Se inhibe con la opción \verb|es-nosectiondot|.
    \item Todos los párrafos incluyen un \verb|\indentfirst|. Se inhibe con la opción \verb|es-noindentfirst|
    \item Los entornos \verb|enumerate| e \verb|itemize| se adaptan a reglas castellanas.

    Las opciones \verb|es-noenumerate| y \verb|es-noitemize| inhiben estas modificaciones por separado, y la opción \verb|es-nolists| inhibe ambas.

    Las macros \verb|\spanishdashitems| y \verb|\spanishsignitems| cambian los valores de las listas itemizadas a series de guiones o una serie alternativa de símbolos.

    \item Los contadores \verb|\alph| y \verb|\Alph| incluyen \textit{\~n} después de \textit{n}.
    \item Los marcadores de notas no numéricos se vuelven series de asteriscos.

    La opción \verb|es-nolayout| inhabilita los cambios al formato del documento.  Estos cambios afectan estas enumeraciones y llamadas a notas a pie de página.

    La opción \verb|es-uppernames| hace versiones con mayúsculas para las traducciones de los encabezados (capítulo, bibliografía, etc.).

    La opción \verb|es-tabla| reemplaza ``cuadro'' con ``tabla''.

    La macro \verb|\spanish|\textit{caption}\verb|{|\emph{}\verb|}| cambia el valor de la palabra clave automáticamente. Por ejemplo, \verb|\spanishcontents{Contenido}|.
\end{itemize}

Hay otras modificaciones que afectan la composición del texto, los caracteres
activos y los atajos.

\begin{itemize}
    \item Las comillas tipográficas en la codificación \verb|OT1| se toman de la \fontnomo\ \verb|lasy| en lugar de las macros \verb|\ll| y \verb|\gg|.
    \item \emph{En modo matemático}, un punto seguido de un dígito escribe una coma decimal.

    La macro \verb|\decimalpoint| restaura el decimal a punto, y la macro \verb|\spanishdecimal{|\emph{caracter}\verb|}| asigna     un caracter cualquiera.

    \item Se define un entorno \verb|quoting| y dos abreviaturas \verb|<<| y \verb|>>| para formatear citas largas.

    La opción \verb|es-noquoting| inhabilita los atajos \verb|<<| y \verb|>>| para el entorno \verb|quoting|, pero se conservan los atajos \verb|"`| y \verb|"'|.

    La macro \verb|\deactivatequoting| desactiva los atajos \verb|<<| y \verb|>>| temporalmente para habilitar los signos \verb|<| and \verb|>| en comparaciones numéricas y algunas macros de AMS\TeX. 

    La macro \verb|\spanishdeactivate{|\emph{caracteres}\verb|}| inhabilita temporalmente los atajos definidos por los caracteres incluidos en su argumento. Son elegibles los caracteres \verb|.'"~<>|. % These shorthand characters may be globally deactivated for Spanish adding this command to \verb|\shorthandsspanish|.

    La opción \verb|es-tilden| restaura el atajo \verb|~| para escribir eñes. Sólo se provee para componer documentos viejos.

    La opción \verb|es-noshorthands| inhabilita todos los atajos activados por \verb|"|, \verb|'|, \verb|<|, \verb|>|, \verb|~| y \verb|.|

    \item Los ordinales castellanos se forman con la orden \verb|\sptext| como en \verb|1\sptext{er}|. El punto preceptuado está incluido automáticamente.
    \item Funciones matemáticas acentuadas (l\'\i m, m\'ax, m\'\i n, m\'od) y espaciadas (arc\,cos, etc.). 

    Las órdenes \verb|\unspacedoperators| y \verb|\unaccentedoperators| inhabilitan estas funciones.  

    La macro \verb|\spanishoperators{|\emph{operators}\verb|}| define los nombres de las funciones y operadores del castellano.  Por ejemplo, la orden 
        \begin{verbatim}
            \renewcommand{\spanishoperators}{arc\,ctg m\acute{i}n}
        \end{verbatim}
    crea macros para estas funciones. Dentro de esta orden la macro \verb|\,| añade espacios finos (en \verb|\arcctg| en este caso), y la macro \verb|\acute{|\emph{letter}\verb|}| añade un acento (como \verb|m\acute{i}n| define \verb|\min| (m\'\i{}n). No es necesario añadir la \verb|\dotlessi| explícitamente.

    \item Se provee una orden \verb|\dotlessi| para uso en modo matemático.
    \item Se añade un espacio fino al signo porcentual \verb|\%|. La macro \verb|\spanishplainpercent| lo inhibe localmente. 
    \item Se provee una orden \verb|\lsc| para producir versalitas minúsculas, para siglas o números romanos. 
    \item Se redefine la orden \verb|\roman| para escribir números romanos en versalitas en lugar de minúsculas.

    La opción \verb|es-preindex| llama automáticamente al paquete \verb|romanidx.sty| para reparar llamadas de makeindex formateadas en versalitas. La macro \verb|\spanishindexchars| define los caracteres que determinan las entradas de índice.  Por omisión se usa \verb=\spanishindexchars{|}{(}{)}=.

    La opción \verb|es-ucroman| convierte todos los numerales romanos en mayúsculas en lugar de versalitas, y la opción \verb|es-lcroman| los convierte otra vez en minúsculas, si la macro \verb|lsc| falla por algún motivo. La primera opción es preferible a la segunda, que es de hecho una falta ortográfica.

    Tres macros controlan las mismas modificaciones temporalmente: \verb|\spanishscroman|, \verb|\spanishucroman|, and \verb|\spanishlcroman|. 
\end{itemize}

Algunas macros prestan facilidades adicionales para el formato de algunos documentos.

\begin{itemize}
    \item Las macros \verb|\spanishdatedel| y \verb|\spanishdatede| controlan el formato del artículo en las fechas (\verb|del| or \verb|de|).
    \item La macro \verb|\spanishreverseddate| pone el formato de fecha en la forma ``Mes Día del Año''.
    \item La macro \verb|\Today| inicia los nombres de los meses en mayúscula.
    \item Los puntos suspensivos en medio de una oración se escribe \verb|\dots| 
 \end{itemize}
 
 Finalmente, hay opciones que abrevian varias opciones al mismo tiempo. Se abrevian en el cuadro~\ref{tab:SpanishCustomizationOptions}.

 \begin{table}
 \centering
 \begin{tabular}{cccc}\hline
  Opciones complejas & \verb|es-minimal| & \verb|es-sloppy| & \verb|es-noshorthands| \\\hline
  \verb|es-noindentfirst|  & o  & o  &   \\
  \verb|es-nosectiondot|   & o  & o  &   \\
  \verb|es-nolists|        & o  & o  &   \\
  \verb|es-noquoting|      & o  & o  & o \\
  \verb|es-notilde|        & o  & o  & o \\
  \verb|es-nodecimaldot|   & o  & o  & o \\
  \verb|es-nolayout|       &    & o  &   \\
  \verb|es-ucroman|        & o  &    &   \\
  \verb|es-lcroman|        &    & o  &   \\\hline
 \end{tabular}
  \caption{Opciones globales del castellano}
  \label{tab:SpanishCustomizationOptions}
 \end{table}

Finalmente, toda esta maquinaria permite construir opciones regionales del castellano. Las primeras de ellas son \verb|mexico| y \verb|mexico-com|. Ambas opciones redefinen las comillas del entorno \verb|quoting|, y la primera añade \verb|es-nodecimaldot|, como es costumbre en México y otros países de Centroamérica y el Caribe.

% Finally, the Spanish 5 series begins the implementation of national
% variations of Spanish typography, beginning with Mexico.  Thus the
% global options |mexico| and |mexico-com| are adapted to practices
% spread in Mexico, and perhaps Central America, the Caribbean, and
% some countries in South America.\footnote{The main difference is
% that |mexico| disables the |decimaldot| mechanism, while
% |mexico-com| keeps it enabled; both change the |quoting|
% environment, disabling the use of guillemets.}
%
% \begin{itemize}
%
% \item Extras are divided in groups controlled by the commands
% |\textspanish|, |\mathspanish|, |\shorthandsspanish| y
% |\layoutspanish|; their values may be cancelled typing
% |\renewcommand|\marg{command}|{}|, or changed at will (check the
% Spanish documentation or the code for details).
% 
% \item The commands
% |\lquoti|\marg{string} |\rquoti|\marg{string} 
% |\lquotii|\marg{string} |\rquotii|\marg{string} 
% |\lquotiii|\marg{string} |\rquotiii|\marg{string} 
% set the quoting signs in the |quoting| environment, 
% nested from outside in. They may be |\renew|ed at will. 
% Default values are shown in table~\ref{tab:spanish-quote-ref}.
% \begin{table}
% \center\small
% \vspace{1.5ex}
% \begin{tabular}{l@{\hspace{5em}}l}
% |\lquoti|   &|"<|\\
% |\rquoti|   &|">|\\
% |\lquotii|  &|``|\\
% |\rquotii|  &|''|\\
% |\lquotiii| &|`|\\
% |\rquotiii| &|'|
% \end{tabular}
% \caption{Default quoting signs set for the \texttt{quoting} environment.}
% \label{tab:spanish-quote-ref}
% \end{table}
%
% \item The command |\selectspanish*| is obsolete: if |spanish| is the
% main language, all its features are available right after loading
% |babel|.  The |es-delayed| option is provided to restore the
% previous behavior and macros for backwards compatibility.
% 
% \end{itemize}
%
\subsubsection{Ajuste a la tipografía española}

Es posible dar aspecto ``español'' a un texto compuesto en otro idioma ``importando'' el formato del texto definido en \verb|spanish|. Basta cargar \verb|spanish| como idioma principal, añadir a los extras del idioma seleccionado (esperanto en este caso) las características que se quieren importar, y seleccionar el nuevo idioma principal al principio del documento.

\begin{verbatim}
\usepackage[esperanto,spanish]{babel}
\makeatletter
\addto\extrasesperanto{\textspanish}
\declare@shorthand{esperanto}{^a}{\textormath{\es@sptext{a}}{\ensuremath{^a}}}
\declare@shorthand{esperanto}{^A}{\textormath{\es@sptext{A}}{\ensuremath{^A}}}
\makeatother
\AtBeginDocument{\selectlanguage{esperanto}}
\end{verbatim}

De esta forma es posible componer texto en esperanto (u otro idioma cualquiera) y darle aspecto de ``compuesto en España''.

%%\subsection{Soporte para el portugués}
%%
%%\secby{Demerson Andre Polli}{polli@linux.ime.usp.br}
%%Para posibilitar la silabación y cambiar todos los textos automáticos
%%al \wi{portugués},
%%\index{portugu\^es} use la orden:
%%\begin{lscommand}
%%\verb|\usepackage[portuguese]{babel}|
%%\end{lscommand}
%%Si prefiere brasileño, cambie el idioma por \texttt{\wi{brazilian}}.
%%
%%Como hay muchos acentos en portugués, debería usar
%%\begin{lscommand}
%%\verb|\usepackage[latin1]{inputenc}|
%%\end{lscommand}
%%para poder meterlos con el teclado, y también
%%\begin{lscommand}
%%\verb|\usepackage[T1]{fontenc}|
%%\end{lscommand}
%%para que la silabación sea correcta.
%%
%%Vea el cuadro~\ref{portuguese} para un preámbulo adecuado para el
%%portugués.  Note que usamos la codificación de entrada latin1, que puede
%%no ser correcta para su sistema.
%%
%%\begin{table}[btp]
%%\caption{Preámbulo para documentos en portugués.} \label{portuguese}
%%\begin{lined}{5cm}
%%\begin{verbatim}
%%\usepackage[portuguese]{babel}
%%\usepackage[latin1]{inputenc}
%%\usepackage[T1]{fontenc}
%%\end{verbatim}
%%\bigskip
%%\end{lined}
%%\end{table}
%%
%%\subsection{Soporte para el esperanto}
%%
%%\secby{Carlos Carleos}{carleos@uniovi.es}
%%Para posibilitar la silabación y cambiar todos los textos automáticos
%%al \wi{esperanto},
%%\index{esperanto} use la orden:
%%\begin{lscommand}
%%\verb|\usepackage[esperanto]{babel}|
%%\end{lscommand}
%%
%%Para usar cómodamente las letras ``con sombrero'', puede usar como
%%codificación de entrada UTF-8 de Unicode 
%%\begin{lscommand}
%%\verb|\usepackage[utf-8]{inputenc}|
%%\end{lscommand}
%%o bien, en un entorno ASCII, hacer uso de las secuencias \verb|^c|,
%%\verb|^C|, \verb|^g|, etc.  Además, \verb|^j| elimina el punto de la
%%\verb|j| y \verb|^h| impide que el circunflejo quede demasiado alto.
%%
%%La secuencia \verb+^|+ inserta \verb|\discretionary{-}{}{}|.
%%
%%En \verb|esperant.sty| se definen \verb|\Esper| y \verb|\esper| como
%%alternativas a \verb|\Alph| y \verb|\alph|.  También \verb|\hodiau|,
%%como \verb|\today| pero incluyendo el artículo ``la'', y
%%\verb|\hodiaun| como versión en acusativo.
%%
%
%%\subsection{Soporte para interlingua}
%%
%%\secby{Carlos Carleos}{carleos@uniovi.es}
%%Para posibilitar la silabación y cambiar todos los textos automáticos
%%a \wi{interlingua},
%%\index{interlingua} use la orden:
%%\begin{lscommand}
%%\verb|\usepackage[interlingua]{babel}|
%%\end{lscommand}
%%
%%
%% \subsection{Support for French}

%% \secby{Daniel Flipo}{daniel.flipo@univ-lille1.fr}
%% Some hints for those creating \wi{French} documents with \LaTeX{}:
%% you can load French language support with the following command:

%% \begin{lscommand}
%% \verb|\usepackage[frenchb]{babel}|
%% \end{lscommand}

%% Note that, for historical reasons, the name of \textsf{babel}'s option 
%% for French is either \emph{frenchb} or \emph{francais} but not \emph{french}.

%% This enables French hyphenation, if you have configured your
%% \LaTeX{} system accordingly. It also changes all automatic text into
%% French: \verb+\chapter+ prints Chapitre, \verb+\today+ prints the current
%% date in French and so on. A set of new commands also
%% becomes available, which allows you to write French input files more
%% easily. Check out table \ref{cmd-french} for inspiration. 

%% \begin{table}[!htbp]
%% \caption{Special commands for French.} \label{cmd-french}
%% \begin{lined}{9cm}
%% \selectlanguage{french}
%% \begin{tabular}{ll}
%% \verb+\og guillemets \fg{}+         \quad &\og guillemets \fg \\[1ex]
%% \verb+M\up{me}, D\up{r}+            \quad &M\up{me}, D\up{r}  \\[1ex]
%% \verb+1\ier{}, 1\iere{}, 1\ieres{}+ \quad &1\ier{}, 1\iere{}, 1\ieres{}\\[1ex]
%% \verb+2\ieme{} 4\iemes{}+           \quad &2\ieme{} 4\iemes{}\\[1ex]
%% \verb+\No 1, \no 2+                 \quad &\No 1, \no 2   \\[1ex]
%% \verb+20~\degres C, 45\degres+      \quad &20~\degres C, 45\degres \\[1ex]
%% \verb+\bsc{M. Durand}+              \quad &\bsc{M.~Durand} \\[1ex]
%% \verb+\nombre{1234,56789}+          \quad &\nombre{1234,56789}
%% \end{tabular}
%% \selectlanguage{english}
%% \bigskip
%% \end{lined}
%% \end{table}

%% You will also notice that the layout of lists changes when switching to the
%% French language. For more information on what the \texttt{frenchb}
%% option of \textsf{babel} does and how you can customize its behaviour, run
%% \LaTeX{} on file \texttt{frenchb.dtx} and read the produced file
%% \texttt{frenchb.dvi}.

%% \subsection{Support for German}

%% Some hints for those creating \wi{German}\index{Deutsch}
%% documents with \LaTeX{}: you can load German language support with the following
%% command:

%% \begin{lscommand}
%% \verb|\usepackage[german]{babel}|
%% \end{lscommand}

%% This enables German hyphenation, if you have configured your
%% \LaTeX{} system accordingly. It also changes all automatic text into
%% German. Eg. ``Chapter'' becomes ``Kapitel.'' A set of new commands also
%% becomes available, which allows you to write German input files more quickly
%% even when you don't use the inputenc package. Check out table
%% \ref{german} for inspiration. With inputenc, all this becomes moot, but your 
%% text also is locked in a particular encoding world.

%% \begin{table}[!htbp]
%% \caption{German Special Characters.} \label{german}
%% \begin{lined}{8cm}
%% \selectlanguage{german}
%% \begin{tabular}{*2{ll}}
%% \verb|"a| & "a \hspace*{1ex} & \verb|"s| & "s \\[1ex]
%% \verb|"`| & "` & \verb|"'| & "' \\[1ex]
%% \verb|"<| or \ci{flqq} & "<  & \verb|">| or \ci{frqq} & "> \\[1ex]
%% \ci{flq} & \flq & \ci{frq} & \frq \\[1ex]
%% \ci{dq} & " \\
%% \end{tabular}
%% \selectlanguage{english}
%% \bigskip
%% \end{lined}
%% \end{table}

%% In German books you often find French quotation marks (\flqq guil\-le\-mets\frqq).
%% German typesetters, however, use them differently. A quote in a German book
%% would look like \frqq this\flqq. In the German speaking part of Switzerland,
%% typesetters use \flqq guillemets\frqq~the same way the French do.

%% A major problem arises from the use of commands
%% like \verb+\flq+: If you use the OT1 font (which is the default font) the  
%% guillemets will look like the math symbol ``$\ll$'', which turns a typesetter's stomach.
%% T1 encoded fonts, on the other hand, do contain the required symbols. So if you are using this type
%% of quote, make sure you use the T1 encoding. (\verb|\usepackage[T1]{fontenc}|)

%% \subsection[Support for Korean]{Support for Korean\footnotemark}\label{support_korean}%
%% \footnotetext{%
%% Considering a number of issues  Korean \LaTeX{} users
%% have to cope with.
%% This section was written by Karnes KIM on behalf of the
%% Korean lshort translation team. It  was translated into English
%% by SHIN Jungshik and shortened by Tobi Oetiker.}

%% To use \LaTeX{} for typesetting  \wi{Korean}, 
%% we need to solve three problems: 

%% \begin{enumerate}
%% \item 
%% We must be able to 
%% edit \wi{Korean input files}.
%% Korean input files must be in plain text format, but because Korean
%% uses its own character set outside the
%% repertoire of US-ASCII, they will look rather strange with a normal ASCII editor.  The two most widely used encodings for
%% Korean text files are  EUC-KR and its upward compatible
%% extension used in Korean MS-Windows, CP949/Windows-949/UHC.
%% In these encodings each US-ASCII character represents its normal ASCII
%% character similar to other ASCII compatible encodings such as
%% ISO-8859-\textit{x}, EUC-JP, Big5, or Shift\_JIS. On the other hand, Hangul
%% syllables, Hanjas (Chinese characters as used in Korea), Hangul Jamos,
%% Hiraganas, Katakanas, Greek and Cyrillic characters and other
%% symbols and letters drawn from KS~X~1001 are represented by two
%% consecutive octets. The first has its MSB set.
%% Until the mid-1990's, it took a considerable amount of time and effort to
%% set up a Korean-capable environment under a non-localized (non-Korean)
%% operating system. 
%% You can skim through the now much-outdated \url{http://jshin.net/faq} to get 
%% a glimpse of what it was like to use Korean under non-Korean OS in mid-1990's.
%% These days all three major operating systems (Mac OS, Unix, Windows) come equipped
%% with pretty decent multilingual support and internationalization features
%% so that editing Korean text file is not so much of a problem anymore, even
%% on non-Korean operating systems.

%% \item \TeX{} and \LaTeX{} were originally written for
%% scripts with no more than 256 characters in their alphabet.
%% To make them work for languages with considerably 
%% more characters such as
%% Korean%,
%%  \footnote{Korean Hangul is an alphabetic script with 14 basic consonants
%%  and 10 basic vowels (Jamos). Unlike Latin or Cyrillic scripts, the
%%  individual characters have to be arranged in rectangular
%%  clusters about the same size as Chinese characters. Each cluster
%%  represents a syllable. An unlimited number of syllables can be
%%  formed out of this finite set of vowels and consonants. Modern Korean
%%  orthographic standards (both in South Korea and  North Korea), however,
%%  put some restriction on the formation of these clusters.
%%  Therefore only a finite number of  orthographically correct syllables exist.
%%  The Korean Character encoding defines individual code points for each of these syllables (KS~X~1001:1998 and KS~X~1002:1992). So Hangul, albeit alphabetic, is
%%  treated like the Chinese and Japanese writing systems with tens of thousands of
%%  ideographic/logographic characters.  ISO~10646/Unicode offers both ways of
%%  representing Hangul used for \emph{modern} Korean by encoding Conjoining
%%  Hangul Jamos (alphabets: \url{http://www.unicode.org/charts/PDF/U1100.pdf})
%%  in addition to encoding all the orthographically allowed Hangul syllables in
%%  \emph{modern} Korean (\url{http://www.unicode.org/charts/PDF/UAC00.pdf}).
%%  One of the most daunting challenges in Korean typesetting with
%%  \LaTeX{} and related typesetting system is supporting Middle Korean---and possibly future Korean---syllables that can be only represented
%%  by conjoining Jamos in Unicode. It is hoped that future \TeX{} engines like $\Omega$ and
%%  $\Lambda$ will eventually provide solutions to this
%%  so that some Korean linguists and historians
%%  will defect from MS Word that already has  a pretty good support 
%%  for Middle Korean.}
%% or Chinese, a subfont mechanism was developed.
%% It divides a single CJK font with  thousands or tens of thousands of
%% glyphs into a set of subfonts with 256 glyphs each. 
%% For Korean, there are three widely used packages;  \wi{H\LaTeX}
%% by UN~Koaunghi, \wi{h\LaTeX{}p} by CHA~Jaechoon and the \wi{CJK package}
%% by Werner~Lemberg.\footnote{% 
%% They can be obtained at \CTANref|language/korean/HLaTeX/|\\
%%    \CTANref|language/korean/CJK/| and
%%    \texttt{http://knot.kaist.ac.kr/htex/}}
%% H\LaTeX{} and h\LaTeX{}p are specific to Korean and provide
%% Korean localization on top of the font support.
%% They both can process Korean input text files encoded in EUC-KR. H\LaTeX{} can
%% even process input files encoded in CP949/Windows-949/UHC and UTF-8
%% when used along with $\Lambda$, $\Omega$.

%% The CJK package is not specific to Korean. It can
%% process input files in UTF-8 as well as in various CJK encodings
%% including EUC-KR and CP949/Windows-949/UHC, it can be used to typeset documents with
%% multilingual content (especially Chinese, Japanese and Korean).
%% The CJK package has no Korean localization such as the one offered by H\LaTeX{} and it
%% does not come with as many special Korean fonts as H\LaTeX.

%% \item The ultimate purpose of using typesetting programs like \TeX{}
%% and \LaTeX{} is to get documents typeset in an `aesthetically' satisfying way.
%% Arguably the most important element in typesetting is  a set of
%% well-designed fonts. The H\LaTeX{} distribution
%% includes \index{Korean font!UHC font}UHC \PSi{} fonts 
%% of 10
%% different families and
%% Munhwabu\footnote{Korean Ministry of Culture.}
%% fonts (TrueType) of 5 different families.
%% The CJK package works with a set of fonts used by earlier versions
%% of H\LaTeX{} and it can use Bitstream's cyberbit TrueType
%% font.
%% \end{enumerate}

%% To use the  H\LaTeX{} package for typesetting your Korean text, put the following
%% declaration into the preamble of your document:
%% \begin{lscommand}
%% \verb+\usepackage{hangul}+
%% \end{lscommand}

%% This command turns the Korean localization on. The headings
%% of chapters, sections, subsections, table of content and table of
%% figures are all translated into Korean and the formatting of the document
%% is changed to follow Korean conventions. 
%% The package also provides automatic ``particle selection.''
%% In Korean, there are pairs of post-fix particles 
%% grammatically equivalent but different in form. Which 
%% of any given pair is correct depends on 
%% whether the preceding syllable ends with a  vowel or a consonant.
%% (It is a bit more complex than this, but this should give you
%% a good picture.)
%% Native Korean speakers have no problem picking the right particle, but
%% it cannot be determined which particle to use for references and other automatic
%% text that will change while you edit the document.
%% It 
%% takes a painstaking effort to place appropriate particles manually
%% every time you add/remove references or simply shuffle  parts
%% of your document around.
%% H\LaTeX{} relieves its users from this boring and error-prone process.

%% In case you don't need Korean localization features
%% but just want 
%% to  typeset Korean text, you can put the following line in the 
%% preamble, instead.
%% \begin{lscommand}
%% \verb+\usepackage{hfont}+
%% \end{lscommand}

%% For more details on typesetting  Korean with H\LaTeX{}, refer to
%% the \emph{H\LaTeX{} Guide}.  Check out the web site of the Korean
%% \TeX{} User Group (KTUG) at  \url{http://www.ktug.or.kr/}.
%% There is also a Korean translation
%% of this manual available.

%% \subsection{Writing in Greek}
%% \secby{Nikolaos Pothitos}{pothitos@di.uoa.gr}
%% See table~\ref{preamble-greek} for the preamble you need to write in the
%% \wi{Greek} \index{Greek} language.  This preamble enables hyphenation and
%% changes all automatic text to Greek.\footnote{If you select the
%% \texttt{utf8x}
%% option for the package \texttt{inputenc}, you can type Greek and polytonic
%% Greek
%% unicode characters.}

%% \begin{table}[btp]
%% \caption{Preamble for Greek documents.} \label{preamble-greek}
%% \begin{lined}{7cm}
%% \begin{verbatim}
%% \usepackage[english,greek]{babel}
%% \usepackage[iso-8859-7]{inputenc}
%% \end{verbatim}
%% \bigskip
%% \end{lined}
%% \end{table}

%% A set of new commands also becomes available, which allows you to write
%% Greek input files more easily.  In order to temporarily switch to English
%% and vice versa, one can use the commands \verb|\textlatin{|\emph{english
%% text}\verb|}| and \verb|\textgreek{|\emph{greek text}\verb|}| that both take
%% one argument which is then typeset using the requested font encoding. 
%% Otherwise you can use the command \verb|\selectlanguage{...}| described in a
%% previous section.  Check out table~\ref{sym-greek} for some Greek
%% punctuation characters.  Use \verb|\euro| for the Euro symbol.

%% \begin{table}[!htbp]
%% \caption{Greek Special Characters.} \label{sym-greek}
%% \begin{lined}{4cm}
%% \selectlanguage{french}
%% \begin{tabular}{*2{ll}}
%% \verb|;| \hspace*{1ex}  &  $\cdot$ \hspace*{1ex}  &  \verb|?| \hspace*{1ex}&  ;   \\[1ex]
%% \verb|((|               &  \og                    &  \verb|))|&  \fg \\[1ex]
%% \verb|``|               &  `                      &  \verb|''| &  '   \\
%% \end{tabular}
%% \selectlanguage{english}
%% \bigskip
%% \end{lined}
%% \end{table}


%% \subsection{Support for Cyrillic}

%% \secby{Maksym Polyakov}{polyama@myrealbox.com}
%% Version~3.7h of \pai{babel} includes support for the
%% \fei{T2*}~encodings and for typesetting Bulgarian, Russian and
%% Ukrainian texts using Cyrillic letters.  

%% Support for Cyrillic is based on standard \LaTeX{} mechanisms plus
%% the \pai{fontenc} and \pai{inputenc} packages. But, if you are going to
%% use Cyrillics in math mode, you need to load \pai{mathtext} package
%% before \pai{fontenc}:\footnote{If you use \AmS-\LaTeX{} packages, 
%% load them before \pai{fontenc} and \pai{babel} as well.}
%% \begin{lscommand}
%% \verb+\usepackage{mathtext}+\\
%% \verb+\usepackage[+\fei{T1}\verb+,+\fei{T2A}\verb+]{fontenc}+\\
%% \verb+\usepackage[+\iei{koi8-ru}\verb+]{inputenc}+\\
%% \verb+\usepackage[english,bulgarian,russian,ukranian]{babel}+
%% \end{lscommand}

%% Generally, \pai{babel} will authomatically choose the default font encoding,
%% for the above three languages this is \fei{T2A}.  However, documents are not
%% restricted to a single font encoding. For multi-lingual documents using
%% Cyrillic and Latin-based languages it makes sense to include Latin font
%% encoding explicitly. \pai{babel} will take care of switching to the appropriate
%% font encoding when a different language is selected within the document.

%% In addition to enabling hyphenations, translating automatically
%% generated text strings, and activating some language specific 
%% typographic rules (like \ci{frenchspacing}), \pai{babel} provides some 
%% commands allowing typesetting according to the standards of 
%% Bulgarian, Russian, or Ukrainian languages. 


%% For all three languages, language specific punctuation is provided:
%% The Cyrillic dash for the text (it is little narrower than Latin dash and
%% surrounded by tiny spaces), a dash for direct speech, quotes, and
%% commands to facilitate hyphenation, see Table~\ref{Cyrillic}.

%% % Table borrowed from Ukrainian.dtx
%% \begin{table}[htb]
%%   \begin{center}
%%   \index{""-@\texttt{""}\texttt{-}} 
%%   \index{""---@\texttt{""}\texttt{-}\texttt{-}\texttt{-}} 
%%   \index{""=@\texttt{""}\texttt{=}} 
%%   \index{""`@\texttt{""}\texttt{`}} 
%%   \index{""'@\texttt{""}\texttt{'}} 
%%   \index{"">@\texttt{""}\texttt{>}} 
%%   \index{""<@\texttt{""}\texttt{<}} 
%%   \caption[Bulgarian, Russian, and Ukrainian]{The extra definitions made
%%            by Bulgarian, Russian, and Ukrainian options of \pai{babel}}\label{Cyrillic}
%%   \begin{tabular}{@{}p{.1\hsize}@{}p{.9\hsize}@{}}
%%    \hline
%%    \verb="|= & disable ligature at this position.               \\
%%    \verb|"-| & an explicit hyphen sign, allowing hyphenation
%%                in the rest of the word.                         \\
%%    \verb|"---| & Cyrillic emdash in plain text.                      \\
%%    \verb|"--~| & Cyrillic emdash in compound names (surnames).       \\
%%    \verb|"--*| & Cyrillic emdash for denoting direct speech.         \\
%%    \verb|""| & like \verb|"-|, but producing no hyphen sign
%%                (for compound words with hyphen, e.g.\verb|x-""y|
%%                or some other signs  as ``disable/enable'').     \\
%%    \verb|"~| & for a compound word mark without a breakpoint.        \\
%%    \verb|"=| & for a compound word mark with a breakpoint, allowing
%%           hyphenation in the composing words.                   \\
%%    \verb|",| & thinspace for initials with a breakpoint
%%            in following surname.                                \\
%%    \verb|"`| & for German left double quotes
%%                (looks like ,\kern-0.08em,).                     \\
%%    \verb|"'| & for German right double quotes (looks like ``).       \\%''
%%    \verb|"<| & for French left double quotes (looks like $<\!\!<$).  \\
%%    \verb|">| & for French right double quotes (looks like $>\!\!>$). \\
%%    \hline
%%   \end{tabular}
%%   \end{center}
%% \end{table}


%% The Russian and Ukrainian options of \pai{babel} define the commands \ci{Asbuk}
%% and \ci{asbuk}, which act like \ci{Alph} and \ci{alph}, but produce capital
%% and small letters of Russian or Ukrainian alphabets (whichever is the
%% active language of the document). The Bulgarian option of \pai{babel} 
%% provides the commands \ci{enumBul} and \ci{enumLat} (\ci{enumEng}), which
%% make \ci{Alph} and \ci{alph} produce letters of either
%% Bulgarian or Latin (English) alphabets. The default behaviour of
%% \ci{Alph}  and \ci{alph} for the Bulgarian language option is to
%% produce letters from the Bulgarian alphabet. 

%% %Finally, math alphabets are redefined and  as well as the commands for math
%% %operators according to Cyrillic typesetting traditions. 

%Si encuentras un %TODO: es porque hay algo que no me convence del todo.

% \subsection{The Unicode option}
\subsection{La opción Unicode}

\secby{Axel Kielhorn}{A.Kielhorn@web.de}
%TODO:No sé si debo cambiar el autor, en realidad soy sólo traductor.
%\secby{Enrique Lazcorreta}{enrique.lazcorreta@gmail.com}
% Unicode is the way to go if you want to include several languages in one
% document, especially when these languages are not using the latin script.
% There are two \hologo{TeX}-engines that are capable of processing Unicode
% input:
Unicode es el modo más adecuado para incluir varios idiomas en un único 
documento, sobre todo cuando estos idiomas no utilizan el alfabeto latino. 
Hay dos motores \hologo{TeX} que son capaces de procesar la entrada de texto 
Unicode:

%\begin{description}
%\item[\hologo{XeTeX}] was developed for MacOS X but is now available for all
% architectures. It was first included into TexLive 2007.
%\item[\hologo{LuaTeX}] is the successor of pdf\TeX.  It was first included
% into TexLive 2008.  
%\end{description}
\begin{description}
\item[\hologo{XeTeX}] fue desarrollado para MacOS X pero está disponible para 
   todas las arquitecturas. Se incluyó por primera vez en \TeX{}Live 2007.
\item[\hologo{LuaTeX}] es el sucesor de pdf\TeX. Se incluyó por primera vez en
   \TeX{}Live 2008. 
\end{description}

%The following describes \hologo{XeLaTeX} as distributed with TexLive 2010.
A continuación se describe cómo se distribuye \hologo{XeLaTeX} con \TeX{}Live 
2010.

%\subsubsection{Quickstart}
\subsubsection{Inicio rápido}

%To convert an existing \LaTeX\ file to \hologo{XeLaTeX} the following needs to be done:
Para convertir un \filenomo\ \LaTeX\ existente a \hologo{XeLaTeX} es necesario
hacer lo siguiente:

%\begin{enumerate}
%\item Save the file as UTF-8
%\item Remove
%\begin{lscommand}
%\verb|\usepackage{inputenc}|\\
%\verb|\usepackage{fontenc}|\\
%\verb|\usepackage{textcomp}|
%\end{lscommand}
%from the preamble.
%\item Change
%\begin{lscommand}
%\verb|\usepackage[|\emph{languageA}\verb|]{babel}|
%\end{lscommand}
%to
%\begin{lscommand}
%\verb|\usepackage{polyglossia}|\\
%\verb|\setdefaultlanguage[babelshorthands]{|\emph{languageA}\verb|}|
%\end{lscommand}
%\item Add
%\begin{lscommand}
%\verb|\usepackage[Ligatures=TeX]{fontspec}|
%\end{lscommand}
%to the preamble.
%\end{enumerate}
\begin{enumerate}
\item Guarde el \filenomo como UTF-8
\item Elimine
\begin{lscommand}
\verb|\usepackage{inputenc}|\\
\verb|\usepackage{fontenc}|\\
\verb|\usepackage{textcomp}|
\end{lscommand}
del preámbulo.
\item Cambie
\begin{lscommand}
\verb|\usepackage[|\emph{languageA}\verb|]{babel}|
\end{lscommand}
por
\begin{lscommand}
\verb|\usepackage{polyglossia}|\\
\verb|\setdefaultlanguage[babelshorthands]{|\emph{languageA}\verb|}|
\end{lscommand}
\item Añada
\begin{lscommand}
\verb|\usepackage[Ligatures=TeX]{fontspec}|
\end{lscommand}
al preámbulo.
\end{enumerate}

%The package \pai{polyglossia}\cite{polyglossia} is a replacement for
%\pai{babel}. It takes care of the hyphenation patterns and automatically
%generated text strings. The option \verb|babelshorthands| enables
%\pai{babel} compatible shorthands for german and catalan.
El paquete \pai{polyglossia}\cite{polyglossia} es un sustituto de \pai{babel}. 
Se encarga de los patrones de separación silábica y de la generación automática 
de cadenas de texto. La opción \verb|babelshorthands| habilita la 
compatibilidad de abreviaturas de \pai{babel} para alemán y catalán.

%The package \pai{fontspec}\cite{fontspec} handles font loading for
%\hologo{XeLaTeX} and \hologo{LuaTeX}. The default font is Latin Modern
%Roman. It is a little known fact that some \hologo{TeX} command are ligatures
%defined in the Computer Modern fonts. If you want to use them with a
%non-\hologo{TeX} font, you have to fake them. The option
%\texttt{Ligatures=TeX} defines the following ligatures:
El paquete \pai{fontspec}\cite{fontspec} se encarga de la carga de fuentes para
\hologo{XeLaTeX} y \hologo{LuaTeX}. La fuente predeterminada es Latin Modern
Roman. Es un hecho poco conocido que algunos comandos \hologo{TeX} son 
ligaduras definidas in las fuentes Computer Modern. Si desea utilizarlas con 
una fuente no-\hologo{TeX} debería simularlas. La opción \texttt{Ligatures=TeX} 
define las siguientes ligaduras:

\begin{tabular}{rr}
\verb|--| & --\\
\verb|---|  & ---\\
\verb|''| & ''\\
\verb|``| & ``\\
\verb|!`| & !`\\
\verb|?`| & ?`\\
\verb|,,| & ,,\\
\verb|<<| & <<\\
\verb|>>| & >>\\
\end{tabular}

%TODO:He sustituido $\Gamma\rho\epsilon\epsilon\kappa$ por $\Gamma\rho\iota\epsilon\gamma\omicron$ pero no estoy seguro del cambio (estudié algo de latín hace ya 31 años, soy de ciencias puras aunque no sea excusa)
%\subsubsection{It's all $\Gamma\rho\epsilon\epsilon\kappa$ to me}
%%\subsubsection{It's all \textgreek{Ellenik'a} to me} % this requires greek babel.

\subsubsection{Todo es $\Gamma\rho\iota\epsilon\gamma o$ para mí}

%TODO:Mejorar traducción de "For small values of simple", se refiere a simply pero... Quizá "Devaluando...", parece una frase hecha en inglés pero no he encontrado su equivalente en español
%So far there has been no advantage to using a Unicode \hologo{TeX} engine.
%This changes when we leave the Latin script and move to a more interesting
%language like Greek and Russian.  With a Unicode based system, you can
%simply\footnote{For small values of simple.} enter the characters in your
%editor and \hologo{TeX} will understand them.
Hasta ahora no se ha visto ninguna ventaja al usar un motor \hologo{TeX} 
Unicode. Esto cambia cuando abandonamos el alfabeto latino y nos movemos a un 
idioma más interesante como el griego o el ruso. Con un sistema basado en 
Unicode se puede simplemente\footnote{Depreciando el concepto de simple.} 
escribir los caracteres en el editor y \hologo{TeX} los entenderá.

%Writing in different languages is easy, just specify the languages in the preamble:
Escribir en diferentes idiomas es fácil, basta con especificar los idiomas en 
el preámbulo.

\begin{lscommand}
\verb|\setdefaultlanguage{spanish}|\\
\verb|\setotherlanguage[babelshorthands]{german}|
\end{lscommand}
%
%To write a paragraph in German, you can use the German environment:
Para escribir un párrafo en alemán puede usar el entorno alemán:

\begin{lscommand}
Texto en español.\\
\verb|\begin{german}|\\
Deutscher Text.\\
\verb|\end{german}|\\
Más texto en español.
\end{lscommand}

%If you just need a word in a foreign language you can use the
%\verb|\text|\emph{language} command:
Si sólo necesita utilizar una palabra en otro idioma puede usar el comando
\verb|\text|\emph{language}:

\begin{lscommand}
\verb|Texto en español. \textgerman{Gesundheit} es en realidad una palabra alemana.|
\end{lscommand}

%This may look unnecessary since the only advantage is a correct hyphenation,
%but when the second language is a little bit more exotic it will be worth
%the effort.
Esto puede parecer innecesario ya que la única ventaja es una división correcta 
de palabras, pero cuando el segundo idioma es algo más exótico merece la pena 
el esfuerzo.

%Sometimes the font used in the main document does not contain glyphs that
%are required in the second language\footnote{Latin Modern does not contain
%Cyrillic letters}. The solution is to define a font that will be used for
%that language. Whenever a new language is activated, \pai{polyglossia} will
%first check whether a font has been defined for that language.
A veces, la fuente usada en el documento principal no contiene glifos que son 
necesarios en el idioma secundario\footnote{Latin Modern no contiene caracteres 
cirílicos.}. La solución consiste en definir la fuente que se utilizará para 
este idioma. Cada vez que se activa un nuevo idioma, \pai{polyglossia} 
comenzará comprobando si se ha definido una fuente para este idioma.

\begin{lscommand}
\verb|\newfontfamily\russianfont[Script=Cyrillic,(...)]{(font)}|
\end{lscommand}

%Now you can write
Ahora usted puede escribir

\begin{lscommand} \verb|\textrussian{Pravda} es un periódico ruso.|
\end{lscommand}
%
%Since this document is written in Latin1-encoding, I cannot show the actual
%Cyrillic\index{Russian}\index{Cyrillic} letters.
Como este documento está escrito con \hologo{pdfLaTeX}, no puedo mostrar los 
caracteres cirílicos\index{Russian}\index{Cyrillic} reales.

%TODO:No sé si "monotónico o politónico" es correcto
%The package \pai{xgreek}\index{Greek}\cite{xgreek} offers support for
%writing either ancient or modern (monotonic or politonic) greek.
El paquete \pai{xgreek}\index{Greek}\cite{xgreek} ofrece soporte para la 
escritura de griego antiguo o moderno (monotónico o politónico).

%TODO:No sé si hay una expresión más correcta en nuestro idioma
%\subsubsection{Right to Left (RTL) languages.}
\subsubsection{Idiomas de escritura de Derecha a Izquierda (RTL).}

%Some languages are written left to right, others are written right to
%left(RTL). \pai{polyglossia} needs the \pai{bidi}\cite{bidi}
%package\footnote{\texttt{bidi} does not support \hologo{LuaTeX}.} in order%
%to support RTL languages. The \pai{bidi} package should be the last package
%you load, even after \pai{hyperref} which is usually the last package.
%(Since \pai{polyglossia} loads \pai{bidi} this means that \pai{polyglossia}
%should be the last package loaded.)
Algunos idiomas se escriben de izquierda a derecha, otros se escriben de 
derecha a izquierda (RTL). \pai{polyglossia} necesita el paquete 
\pai{bidi}\cite{bidi}\footnote{\texttt{bidi} no soporta \hologo{LuaTeX}.} para
dar soporte a idiomas RTL. El paquete \pai{bidi} ha de ser el último paquete 
cargado, incluso después de \pai{hyperref} que suele ser el último paquete.
(Como \pai{polyglossia} carga \pai{bidi}, \pai{polyglossia} ha de ser el último 
paquete cargado.)

%The package \pai{xepersian}\index{Persian}\cite{xepersian} offers support
%for the Persian language. It supplies Persian \LaTeX-commands that allows
%you to enter commands like \verb|\section| in Persian, which makes this
%really attractive to native speakers. \pai{xepersian} is the only package
%that supports kashida\index{kashida} with \hologo{XeLaTeX}. A package for
%Syriac which uses a similar algorithm is under development.
El paquete \pai{xepersian}\index{Persian}\cite{xepersian} ofrece soporte para
el persa. Proporciona comandos \LaTeX\ persas que le permiten introducir 
comandos como \verb|\section| en persa, que lo hace muy atractivo para los 
hablantes nativos. \pai{xepersian} es el único paquete que soporta 
kashida\index{kashida} con \hologo{XeLaTeX}. Actualmente se está desarrollando 
un paquete para el siríaco que utiliza un algoritmo similar.

%The IranNastaliq font provided by the SCICT\footnote{Supreme Council of
%Information and Communication Technology} is available at their website
%\url{http://www.scict.ir/Portal/Home/Default.aspx}. 
La fuente IranNastaliq proporcionada por el SCICT\footnote{Supreme Council of
Information and Communication Technology} está disponible en su sitio web 
\url{http://www.scict.ir/Portal/Home/Default.aspx}. 

%The \pai{arabxetex}\cite{arabxetex} package supports several languages with
%an Arabic script:
El paquete \pai{arabxetex}\cite{arabxetex} es compatible con varios idiomas de
escritura árabe:

%TODO:He quitado "(Arabic)" del primer ítem
%\begin{itemize}
%\item arab (Arabic)\index{Arabic}
%\item persian\index{Persian}
%\item urdu\index{Urdu}
%\item sindhi\index{Sindhi}
%\item pashto\index{Pashto}
%\item ottoman (turk)\index{Ottoman}\index{Turkish}
%\item kurdish\index{Kurdish}
%\item kashmiri\index{Kashmiri}
%\item malay (jawi)\index{Malay}\index{Jawi}
%\item uighur\index{Uighur}
%\end{itemize}
\begin{itemize}
\item árabe\index{Arabic}
\item persa\index{Persian}
\item urdu\index{Urdu}
\item sindhi\index{Sindhi}
\item pashto\index{Pashto}
\item otomano (turco)\index{Ottoman}\index{Turkish}
\item kurdo\index{Kurdish}
\item kashmiri\index{Kashmiri}
\item malayo (jawi)\index{Malay}\index{Jawi}
\item uighur\index{Uighur}
\end{itemize}

%It offers a font mapping that enables \hologo{XeLaTeX} to process input
%using the Arab\TeX\ ASCII transcription.
Proporciona una asignación de fuentes que habilita a \hologo{XeLaTeX} para 
procesar la entrada de caracteres usando la transcripción ASCII Arab\TeX.

%Fonts that support several Arabic laguages are offered by the
%IRMUG\footnote{Iranian Mac User Group} at
%\url{http://wiki.irmug.org/index.php/X_Series_2}.
IRMUG\footnote{Iranian Mac User Group} proporciona fuentes que soportan algunos
idiomas árabes en \url{http://wiki.irmug.org/index.php/X_Series_2}.

%There is no package available for Hebrew\index{Hebrew} because none is
%needed. The Hebrew support in \pai{polyglossia} should be sufficient. But
%you do need a suitable font with real Unicode Hebrew. SBL Hebrew is free for
%non-commercial use and available at
%\url{http://www.sbl-site.org/educational/biblicalfonts.aspx}. Another font
%available under the Open Font License is Ezra SIL, available at
%\url{http://www.sil.org/computing/catalog/show_software.asp?id=76}.
No hay paquetes disponibles para el hebreo\index{Hebrew} porque no son
necesarios. El soporte para hebreo de \pai{polyglossia} debería ser suficiente. 
Pero es necesaria una fuente adecuada con verdadero Unicode hebreo. SBL hebreo 
es gratuito para uso no-comercial y está disponible en 
\url{http://www.sbl-site.org/educational/biblicalfonts.aspx}. Otra fuente
disponible bajo licencia Open Font es Ezra SIL, disponible en 
\url{http://www.sil.org/computing/catalog/show_software.asp?id=76}.

%Remember to select the correct script:
Recuerde seleccionar la secuencia de comandos correcta:

\begin{lscommand}
\verb|\newfontfamily\hebrewfont[Script=Hebrew]{SBL Hebrew}| \\
\verb|\newfontfamily\hebrewfont[Script=Hebrew]{Ezra SIL}|
\end{lscommand}


%\subsubsection{Chinese, Japanese and Korean (CJK)}
%\index{Chinese}\index{Japanese}\index{Korean}
\subsubsection{Chino, japonés y coreano (CJK)}
\index{Chinese}\index{Japanese}\index{Korean}

%The package \pai{xeCJK}\cite{xecjk} takes care of font selection and
%punctuation of these languages.
El paquete \pai{xeCJK}\cite{xecjk} se encarga de la selección de fuentes y de 
la puntuación de estos idiomas.

\section{El espacio entre palabras}

Para conseguir un margen derecho recto en la salida, \LaTeX{} inserta cantidades variables de espacio entre las palabras.  En tipografía inglesa, se inserta algo más de espacio al final de la oración, pues así el texto es más legible.  \LaTeX{} supone que las oraciones terminan en puntos, signos de interrogación o signos de exclamación. Si un punto sigue una letra mayúscula, no se considera un final de oración, pues los puntos tras letras mayúsculas suelen indicar una abreviatura.

Cualquier excepción a esas premisas tiene que indicarla el autor.  Una antibarra ante un espacio genera un espacio que no será expandido. Una tilde~`\verb|~|' genera un espacio que no será expandido y además impide el salto de línea.  La orden \verb|\@| ante un punto indica que dicho punto termina una oración aunque siga a una letra mayúscula. \cih{"@} \index{~@ \verb.~.} \index{tilde@tilde ( \verb.~.)} \index{., espacio tras}

\begin{example}
El Sr.~Aranda se alegró\\
cf.~Fig.~5\\
Adoro el LISP\@. ¿Y usted?
\end{example}

Al escribir en español, no se añade el espacio adicional tras los puntos.  En inglés tal adición se puede desactivar con la orden 
\begin{lscommand}
\ci{frenchspacing}
\end{lscommand}
que manda a \LaTeX{} \emph{no} insertar más espacio tras un punto que tras un signo ordinario.  Es lo habitual en idiomas distintos del inglés, salvo en bibliografías.  En tal caso, la orden \verb|\@| no es necesaria.

\section{Títulos, capítulos y secciones}

Para ayudar al lector a orientarse en su libro, debería dividirlo en capítulos, secciones y subsecciones.  \LaTeX{} lo permite mediante órdenes especiales que toman el título de la sección como argumento. Es tarea suya el usarlos en el orden correcto.

Las siguientes órdenes de sección están disponibles para la clase \texttt{article}: \nopagebreak

\begin{lscommand}
\ci{section}\verb|{...}|\\
\ci{subsection}\verb|{...}|\\
\ci{subsubsection}\verb|{...}|\\
\ci{paragraph}\verb|{...}|\\
\ci{subparagraph}\verb|{...}|
\end{lscommand}

Si quiere dividir su documento en partes sin influir en la numeración de secciones o capítulos puede usar
\begin{lscommand}
\ci{part}\verb|{...}|
\end{lscommand}

Cuando trabaje con las clases \texttt{report} o \texttt{book}, estará disponible una orden de sección adicional
\begin{lscommand}
\ci{chapter}\verb|{...}|
\end{lscommand}

Como la clase \texttt{article} no entiende de capítulos, es muy fácil añadir artículos como capítulos a un libro.  El espacio entre secciones, la numeración y el tamaño de \fontnomo{} de los títulos quedará correctamente establecido por \LaTeX.

Dos órdenes de sección son algo especiales:
\begin{itemize}
    \item La orden \ci{part} no modifica la secuencia de numeración de los  capítulos.
    \item La orden \ci{appendix} no toma ningún argumento.  Solamente cambia la numeración de capítulos de números a letras.\footnote{Para el estilo artículo cambia la numeración de las secciones.}
\end{itemize}

\LaTeX{} crea un índice general tomando los encabezados de sección y los números de página del último ciclo de compilación del documento.  La orden
\begin{lscommand} 
\ci{tableofcontents}
\end{lscommand} 
sitúa el índice general en el lugar en que se ejecuta la orden.  Un documento nuevo debe compilarse (``\LaTeX arse'') dos veces para conseguir un \wi{índice general} correcto.  A veces puede requerirse una tercera compilación.  \LaTeX{} le dirá cuándo es necesario.

Todas las órdenes de sección listadas anteriormente tienen una versión ``estrella''.  Se trata de órdenes con el mismo nombre pero seguido de un asterisco \verb|*|.  Generan encabezados de sección que no aparecen en el índice general y que no se numeran.  La orden \verb|\section{Ayuda}|, por ejemplo, tendría una versión estrella así: \verb|\section*{Ayuda}|.

Normalmente los encabezados aparecen en el índice general exactamente como se introducen en el texto.  A veces no es posible, porque el encabezado es demasiado largo y no cabe en el índice general.  La entrada para el índice general puede indicarse como un argumento opcional antes del encabezado real.

\begin{code}
    \verb|\chapter[Título para el índice general]{Un largo|\\
    \verb|    y aburrido título que aparecerá en el texto}|
\end{code} 

El \wi{título} de todo el documento se genera con la orden
\begin{lscommand}
\ci{maketitle}
\end{lscommand}
El contenido del título tiene que definirse mediante las órdenes
\begin{lscommand}
\ci{title}\verb|{...}|, \ci{author}\verb|{...}| 
y opcionalmente \ci{date}\verb|{...}| 
\end{lscommand}
antes de llamar a \verb|\maketitle|.  En el argumento de \ci{author}, puede poner varios nombres separados por órdenes \ci{and}. 

Un ejemplo de algunas de las órdenes mencionadas arriba puede verse en la Figura~\ref{document} de la página~\pageref{document}.

Además de las órdenes de sección ya explicadas, \LaTeXe{} tiene tres órdenes adicionales para usar con la clase \verb|book|.  Son útiles para dividir la publicación.  Las órdenes alteran los encabezados de los capítulos y los números de página para que aparezcan como se ve en muchos libros (sobre todo ingleses):
\begin{description}
    \item[\ci{frontmatter}] debería ser la primerísima orden tras el comienzo del cuerpo del documento (\verb|\begin{document}|).  Cambia la numeración de páginas a números romanos y las secciones no estarán numeradas.  Es como si usara las órdenes de sección con asterisco (p.ep.{} \verb|\chapter*{Preface}|) pero las secciones aparecerán en el índice general.
    \item[\ci{mainmatter}] viene justo antes del primer capítulo del libro.  Activa los números de página arábigos y recomienza el contador de páginas.
    \item[\ci{appendix}] marca el comienzo de material adicional en su libro.  Tras esta orden los capítulos se numerarán con letras. 
    \item[\ci{backmatter}] debería insertarse antes de los últimos elementos del libro, como la bibliografía y el índice alfabético. No tiene efecto visual en las clases típicas. 
\end{description}


\section{Referencias cruzadas}

En libros, informes y artículos, hay a menudo \wi{referencias cruzadas} a figuras, cuadros y trozos especiales de texto. \LaTeX{} proporciona las siguientes órdenes para referenciar
\begin{lscommand}
\ci{label}\verb|{|\emph{marcador}\verb|}|, \ci{ref}\verb|{|\emph{marcador}\verb|}| 
y \ci{pageref}\verb|{|\emph{marcador}\verb|}|
\end{lscommand}
donde \emph{marcador} es un identificador escogido por el usuario. \LaTeX{} remplaza \verb|\ref| por el número de la sección, subsección, figura, tabla o teorema tras el que se sitúa la orden \verb|\label| correspondiente. \verb|\pageref| imprime el número de página de la página donde la orden \verb|\label| se sitúa.\footnote{Tenga en cuenta que estas órdenes no saben a qué cosa se refieren. \ci{label} solamente guarda el último número generado automáticamente.}  Como para los títulos de sección, se usan los números de la compilación previa.

\begin{example}
Una referencia a esta subsección
\label{sec:esta} aparece así:
``ver sección~\ref{sec:esta} en
la página~\pageref{sec:esta}.''
\end{example}
 
\section{Notas al pie}

Con la orden
\begin{lscommand}
\ci{footnote}\verb|{|\emph{texto al pie}\verb|}|
\end{lscommand}
se imprime una nota al pie de la página actual.  Deben ponerse las notas\footnote{``nota'' es una palabra polisémica.} tras la parabra u oración a la que se refieren.  Las notas que se refieran a una sentencia o parte de ella deben por tanto ponerse tras la coma o el punto.\footnote{Fíjese en que las notas distraen al lector del flujo general del documento.  Después de todo, todo el mundo lee las notas ---somos una especie cotilla---, así que ¿por qué no integrar todo lo que quieres decir en el cuerpo del documento?\footnotemark} \footnotetext{Una señal indicadora no se encuentra necesariamente en el sitio al que está señalando.}

\begin{example}
Las notas al pie\footnote{Esto
es una nota al pie.} se usan
mucho en \LaTeX.
\end{example}
 
\section{Palabras enfatizadas}

Si un texto se escribe a máquina las palabras importantes \texttt{se  enfatizan \underline{subrayándolas}.}
\begin{lscommand}
\ci{underline}\verb|{|\emph{texto}\verb|}|
\end{lscommand}
En los libros impresos, sin embargo, las palabras se enfatizan componiéndolas con una \fontnomo{} \emph{cursiva}.  \LaTeX{} proporciona la orden
\begin{lscommand}
\ci{emph}\verb|{|\emph{texto}\verb|}|
\end{lscommand}
para enfatizar texto.  Lo que hace realmente la orden con su argumento depende del contexto:

\begin{example}
\emph{Si usa énfasis en un
  fragmento de texto ya
  enfatizado, entonces
  \LaTeX{} usa la \fontnomo{}
  \emph{normal} para
  enfatizar.}
\end{example}

Fíjese bien en la diferencia entre mandar a \LaTeX{} que \emph{enfatice} algo y mandarle que use una \emph{\fontnomo{}} diferente:

\begin{example}
\textit{También puede
  \emph{enfatizar} texto
  aunque esté en cursiva,} 
\textsf{en \fontnomo{} 
  \emph{sin serifado},}
\texttt{o en estilo
  \emph{mecanográfico}.}
\end{example}

\section{Entornos} \label{env}

% To typeset special purpose text, \LaTeX{} defines many different
% \wi{environment}s for all sorts of formatting:
\begin{lscommand}
\ci{begin}\verb|{|\emph{entorno}\verb|}|\quad
   \emph{texto}\quad
\ci{end}\verb|{|\emph{entorno}\verb|}|
\end{lscommand}
Aquí \emph{entorno} es un nombre de entorno.  Los entornos pueden anidarse uno dentro de otro mientras se mantenga el orden correcto.
\begin{code}
\verb|\begin{aaa}...\begin{bbb}...\end{bbb}...\end{aaa}|
\end{code}

En las siguientes secciones se explican todos los entornos importantes.

\subsection{Listas (\texttt{itemize}, \texttt{enumerate} y \texttt{description})}

El entorno \ei{itemize} es adecuado para listas simples, el entorno \ei{enumerate} para listas enumeradas y el entorno \ei{description} para descripciones.  \cih{item}

\begin{example}
\flushleft
\begin{enumerate}
\item Puede mezclar los 
entornos de lista a su gusto:
\begin{itemize}
\item Pero podría empezar a 
parecer estúpido.
\item[-] Con un guión.
\end{itemize}
\item Así que recuerde:
\begin{description}
\item[Estupideces] no mejoran 
por ponerlas en una lista.
\item[Lucideces] sin embargo, 
pueden parecer hermosas en 
una lista.
\end{description}
\end{enumerate}
\end{example}
 
\subsection{Alineación (\texttt{flushleft}, \texttt{flushright} y \texttt{center})}

Los entornos \ei{flushleft} y \ei{flushright} generan párrafos \wi{alineado}s a la izquierda o a la derecha respectivamente. El entorno \ei{center} genera texto centrado.  Si no indica los saltos de línea mediante \ci{\bs}, \LaTeX{} los determinará automáticamente.

\begin{example}
\begin{flushleft}
Este texto está alineado a 
la izquierda.  \LaTeX{} no trata 
de justificar las líneas, así
que así quedan.
\end{flushleft}
\end{example}

\begin{example}
\begin{flushright}
Texto alineado\\a la derecha.
\LaTeX{} no trata de 
justificar las líneas.
\end{flushright}
\end{example}

\begin{example}
\begin{center}
En el centro\\de la Tierra
\end{center}
\end{example}

\subsection{Citas (\texttt{quote}, \texttt{quotation} y \texttt{verse})}

El entorno \ei{quote} es útil para citas, frases importantes y ejemplos.

\begin{example}
Una regla empírica tipográfica
para la longitud de renglón es:
\begin{quote}
En promedio, ningún renglón
debería tener más de 66 signos.
\end{quote}
Por ello las páginas de \LaTeX{} 
tienen márgenes tan anchos por 
omisión, y los periódicos usan
múltiples columnas.
\end{example}

Hay dos entornos similares: el \ei{quotation} y el \ei{verse}.  El entorno \texttt{quotation} es útil para citas largas que se extienden varios párrafos, porque sangra la primera línea de cada párrafo.  El entorno \texttt{verse} es útil para poemas donde son importantes los saltos de línea.  Los renglones se separan mediante \ci{\bs} al final de línea y las estrofas mediante un renglón vacío.


\begin{example}
He aquí un fragmento de
todo un monstruo: Quevedo.
\begin{flushleft}
\begin{verse}
Pasa veloz del mundo la 
figura,\\
y la muerte los pasos 
apresura;\\
la vida nunca para,\\
ni el Tiempo vuelve atrás la 
anciana cara.
\end{verse}
\end{flushleft}
\end{example}

\subsection{Resumen (\texttt{abstract})}

En publicaciones científicas es habitual empezar con un resumen que da al lector una idea rápida de lo que puede esperar. \LaTeX{} proporciona el entorno \ei{abstract} con este propósito. Normalmente \ei{abstract} se usa para documentos compuestos con la clase \texttt{article}.

\newenvironment{abstract}%
        {\begin{center}\begin{small}\begin{minipage}{0.8\textwidth}}%
        {\end{minipage}\end{small}\end{center}}
\begin{example}
\begin{abstract}
Esta frase está en el resumen,
es un 80\% del ancho total.
\end{abstract}
Esta frase está fuera del 
resumen, así que es más ancha.
\end{example}

\subsection{Citas literales (\texttt{verbatim})}

El texto encerrado entre \verb|\begin{|\ei{verbatim}\verb|}| y \verb|\end{verbatim}| se escribirá directamente, como escrito a máquina, con todos los saltos de línea y espacios, sin ejecutar ninguna orden \LaTeX{}.

Dentro de un párrafo, un comportamiento similar se puede obtener con
\begin{lscommand}
\ci{verb}\verb|+|\emph{texto}\verb|+|
\end{lscommand}
El signo \verb|+| puede sustituirse por cualquier otro, salvo por letras, \verb|*| por espacios; sirve meramente para delimitar.  Muchos ejemplos de \LaTeX{} en esta introducción se componen mediante esta orden.

\begin{example}
Con \verb|\u{u}| obtengo \u{u}.

\begin{verbatim}
(LOOP
  (PRINT "HOLA MUNDO\n"))
\end{verbatim}
\end{example}

\begin{example}
\begin{verbatim*}
la versión con asterisco
del      entorno verbatim
destaca los espacios (no
finales)  del  texto
\end{verbatim*}
\end{example}

La orden \ci{verb} puede usarse también con un asterisco:

\begin{example}
\verb*|tal  que así :-) |
\end{example}

El entorno \texttt{verbatim} y la orden \verb|\verb| pueden estar prohibidos dentro de los parámetros de algunas órdenes.

 
\subsection{Tablas (\texttt{tabular})}

\newcommand{\mfr}[1]{\framebox{\rule{0pt}{0.7em}\texttt{#1}}}

El entorno \ei{tabular} se usa para componer lindas \wi{tabla}s con líneas opcionales horizontales o verticales.  \LaTeX{} determina el ancho de las columnas automáticamente.

El argumento \emph{espec} de la orden
\begin{lscommand}
\verb|\begin{tabular}[|\emph{pos}\verb|]{|\emph{espec}\verb|}|
\end{lscommand} 
define el formato de la tabla.  Use un \mfr{l} para una columna de texto alineado por la izquierda, \mfr{r} para alineación por la derecha y \mfr{c} para texto centrado; \mfr{p\{\emph{anchura}\}} para una columna con texto justificado con saltos de renglón y \mfr{|} para una línea vertical.

Si el texto de una columna es demasiado ancha para la página, \LaTeX{} no lo partirá automáticamente.  Mediante \mfr{p\{\emph{anchura}\}} puede definir un tipo de columna especial que partirá el texto como en un párrafo normal.

El argumento \emph{pos} indica la posición vertical de la tabla relativa a la base del texto alrededor.  Use una de las letras \mfr{t}, \mfr{b} o \mfr{c} para indicar alineación por lo alto, por lo bajo o por el centro, respectivamente.
 
En un entorno \texttt{tabular}, \texttt{\&} salta a la columna siguiente, \ci{\bs} comienza un nuevo renglón y \ci{hline} inserta una línea horizontal.  Puede añadir líneas parciales usando \ci{cline}\texttt{\{}\emph{j}\texttt{-}\emph{i}\texttt{\}}, donde \emph{j} e \emph{i} son los números de las columnas sobre las que debería extenderse la línea.

\index{"|@ \verb."|.}

\begin{example}
\begin{tabular}{|r|l|}
\hline
7C0 & hexadecimal \\
3700 & octal \\ \cline{2-2}
11111000000 & binario \\
\hline \hline
1984 & decimal \\
1194 & docenal \\
\hline
\end{tabular}
\end{example}

\begin{example}
\begin{tabular}{|p{4.7cm}|}
\hline
Bienvenidos a mi párrafo.
Esperamos que se diviertan
con el espectáculo.\\
\hline 
\end{tabular}
\end{example}

El separador de columnas puede indicarse con el constructo \mfr{@\{...\}}.  Esta orden elimina el espacio entre columnas y lo remplaza con lo que se ponga entre las llaves.  Un uso común de esta orden se explica abajo en un problema de alineación de decimales. Otra aplicación posible es suprimir el espacio adicional de una tabla
mediante \mfr{@\{\}}.

\begin{example}
\begin{tabular}{@{} l @{}}
\hline 
sin espacio extra\\
\hline
\end{tabular}
\end{example}

\begin{example}
\begin{tabular}{l}
\hline
con espacio a izq. y dcha.\\
\hline
\end{tabular}
\end{example}

%
% This part by Mike Ressler
%

\index{decimal alignment} Puesto que no hay manera predefinida para alinear columnas de números por el  decimal,\footnote{Compruebe si tiene instalado en su sistema el paquete \pai{dcolumn}.} podemos ``chapucear'' y hacerlo mediante dos columnas: enteros alineados por la derecha y fracciones alineadas por la izquierda.  La orden \verb|@{'}| en el renglón \verb|\begin{tabular}| remplaza el espacio normal entre columnas por una comilla ``\,'\,'', lo que da el aspecto de una sola columna alineada por una coma decimal. No olvide remplazar el punto decimal en sus números por un separador de columnas (\verb|&|). La cabecera de la ``columna'' puede conseguirse con la orden \ci{multicolumn}.
 
\begin{example}
\begin{tabular}{c r @{'} l}
Expresión con pi    &
\multicolumn{2}{c}{Valor} \\
\hline
$\pi$               & 3&1416  \\
$\pi^{\pi}$         & 36&46   \\
$(\pi^{\pi})^{\pi}$ & 80662&7 \\
\end{tabular}
\end{example}

Aunque los signos recomendado y permitido por ISO para los decimales son una coma baja (,) o un punto bajo (.) respectivamente, este ejemplo usa el signo tradicional para el decimal en la tipografía española, que es una coma alta ('), y muestra que puede usarse un símbolo cualquiera para alinear con el marcador \verb=@{ }=.

\begin{example}
\begin{tabular}{|c|c|}
\hline
\multicolumn{2}{|c|}{Unu} \\
\hline
Du & Tri! \\
\hline
\end{tabular}
\end{example}

El material compuesto con el entorno tabular siempre permanece junto en una misma página.  Si quiere componer tablas largas, debe usar entornos \pai{longtable}.

\section{Elementos deslizantes}

Actualmente la mayoría de las publicaciones contienen muchas figuras y cuadros.  Estos elementos requieren un tratamiento especial, porque no pueden dividirse entre dos páginas.  Un método posible sería empezar una nueva página cada vez que una figura o un cuadro es demasiado grande para encajar en la página actual.  Este enfoque dejaría páginas parcialmente vacías, lo que da mal aspecto.

La solución a este problema es deslizar (\emph{dejar flotar}) cualquier figura o cuadro que no encaje en la página actual hacia una página posterior, y rellenar la página actual con texto del documento. \LaTeX{} ofrece dos entornos para elementos deslizantes: \index{deslizantes, elementos} uno para cuadros y otro para figuras. Para aprovecharlos bien es importante entender aproximadamente cómo maneja \LaTeX{} internamente los deslizantes.  En caso contrario, pueden volverse una fuente de frustaciones, si \LaTeX{} nunca los pone donde usted quiere que vayan.

%\bigskip
Echemos primero un vistazo a las órdenes que \LaTeX{} proporciona para deslizantes.

Cualquier cosa que vaya dentro de un entorno \ei{figure} o \ei{table} se tratará como deslizante.  Ambos entornos admiten un parámetro opcional llamado \emph{colocador}.
\begin{lscommand}
\verb|\begin{figure}[|\emph{colocador}\verb|]| ó
\verb|\begin{table}[|\emph{colocador}\verb|]|
\end{lscommand}
Este parámetro se usa para decir a \LaTeX{} dónde se puede deslizar el elemento.  Se contruye un \emph{colocador} mediante una cadena de \emph{permisos de deslizamiento}. Véase el cuadro~\ref{tab:permiss}.

\begin{table}[!bp]
\caption{Permisos de deslizamiento.}\label{tab:permiss}
\begin{minipage}{\textwidth}
\medskip
\begin{center}
\begin{tabular}{@{}cp{8cm}@{}}
Signo&Permiso para deslizar...\\
\hline
\rule{0pt}{1.05em}\texttt{h} & aquí (\emph{here}) en el mismo lugar
del texto donde aparece.  Útil sobre todo para elementos pequeños.\\[0.3ex]
\texttt{t} & arriba (\emph{top}) en la página.\\[0.3ex]
\texttt{b} & abajo (\emph{bottom}) en la página.\\[0.3ex]
\texttt{p} & en una \emph{página} especial sólo con deslizantes.\\[0.3ex]
\texttt{!} & sin considerar la mayoría de los parámetros
internos\footnote{Como el número máximo de deslizantes por página
  permitido.}, que podrían impedir su colocación.
\end{tabular}
\end{center}
%Fíjese en que \texttt{pt} y \texttt{em} son unidades \TeX{}.  Lea más
%al respecto en el cuadro \ref{units} de la página \pageref{units}.
\end{minipage}
\end{table}

P.ej.{} un cuadro podría empezar con el renglón siguiente:
\begin{code}
\verb|\begin{table}[!hbp]|
\end{code}
El \wi{colocador} \verb|[!hbp]| permite que \LaTeX{} coloque el cuadro justo aquí (\texttt{h}) o abajo (\texttt{b}) en alguna página o en una página especial con deslizantes (\texttt{p}), todo ello incluso si no queda tan bien (\texttt{!}). Si no se indica un colocador, las clases típicas suponen \verb|[tbp]|.

\LaTeX{} colocará todos los deslizantes que encuentre según el colocador indicado por el autor.  Si un deslizante no puede colocarse en la página actual, quedará pospuesto en la cola de \emph{figuras} o en la de \emph{cuadros}.\footnote{Son colas FIFO ---`first in first out'---: primero en entrar, primero en salir.} Cuando comienza una nueva página, \LaTeX{} comprueba antes si es posible rellenar un página especial de deslizantes, con deslizantes de la colas.  Si no es posible, se considera el primer deslizante de cada cola como si acabase de aparecer en el texto: \LaTeX{} intenta de nuevo colocarlo según su colocador (salvo por la `h', que ya no es posible).  Se sitúa cualquier deslizante nuevo que aparezca en el texto dentro de las colas apropiadas.  \LaTeX{} mantiene estrictamente el orden original de aparición para cada tipo de deslizante.  Por eso una figura que no puede colocarse empuja todas las demás figuras hacia el final del documento.  Por tanto:

\begin{quote}
Si \LaTeX{} no coloca los deslizantes como usted esperaba, suele ser por culpa de un solo deslizante atascado en una de las dos colas. 
\end{quote}

Aunque se puede dar a \LaTeX{} un colocador de una sola letra, causa problemas.  Si el deslizante no encaja en el lugar indicado se queda atorado, y bloquea los deslizantes siguientes.  En concreto, no debería nunca jamás usar la opción [h] ---es tan mala que en versiones recientes de \LaTeX{} se sustituye automáticamente por [ht]---.

\bigskip
Habiendo explicado lo difícil, quedan más cosas por mencionar sobre los entornos \ei{table} y \ei{figure}.  Con la orden

\begin{lscommand}
\ci{caption}\verb|{|\emph{texto del pie}\verb|}|
\end{lscommand}

puede definir un pie para el deslizante.  \LaTeX{} añadirá un número correlativo y la cadena ``Figura'' o ``Cuadro''.

Las dos órdenes

\begin{lscommand}
\ci{listoffigures} y \ci{listoftables} 
\end{lscommand}

funcionan análogamente a la orden \verb|\tableofcontents|, imprimiendo un índice de figuras o cuadros, respectivamente.  Tales índices muestran los pies completos, así que si tiende a usar pies largos debe tener una versión más corta del pie para los índices.  Se consigue poniendo la versión corta entre corchetes tras la
orden \verb|\caption|.
\begin{code}
\verb|\caption[Corto]{LLLLLLLaaaaaaarrrrrrrgggggggoooooo}| 
\end{code}

Con \verb|\label| y \verb|\ref|, puede crear una referencia al flotante dentro del texto.

El ejemplo siguiente dibuja un cuadrado y lo inserta en el documento. Podría usarlo si quisiera reservar espacio para imágenes que vaya a pegar en el documento ya impreso.

\begin{code}
\begin{verbatim}
La figura~\ref{blanco} es un ejemplo de Arte Pop.
\begin{figure}[!hbp]
\makebox[\textwidth]{\framebox[5cm]{\rule{0pt}{5cm}}}
\caption{Cinco por cinco centímetros.\label{blanco}}
\end{figure}
\end{verbatim}
\end{code}

En el ejemplo de arriba, \LaTeX{} tratará \emph{con insistencia}~(\texttt{!})\ de colocar la figura \emph{aquí}~(\texttt{h}).\footnote{suponiendo que la cola de figuras está vacía.} Si no es posible, trata de colocar la figura \emph{abajo}~(\texttt{b}).  Si no puede colocar la figura en la página actual, determina si es posible crear una página de deslizantes que contenga esta figura y quizás algunos cuadros de la cola de cuadros. Si no hay bastante material para una página especial de deslizantes, \LaTeX{} comienza una nueva página, y una vez más trata la figura como si acabara de aparecer en el texto.

En ciertas circunstancias podrá requerirse el uso de la orden

\begin{lscommand}
\ci{clearpage} o incluso de \ci{cleardoublepage} 
\end{lscommand}

Manda a \LaTeX{} colocar inmediatamente todos los deslizantes que quedan en las colas y después empezar una página nueva. \ci{cleardoublepage} incluso salta a una nueva página a la derecha.

Aprenderá a incluir dibujos \PSi{} en sus documentos \LaTeXe{} más tarde en esta introducción.

\section{Protección de órdenes frágiles}

El texto dado como argumento de órdenes como \ci{caption} o \ci{section} puede aparecer más de una vez en el documento (p.ej.{} en el índice además de en el cuerpo del documento).  Algunas órdenes no funcionarán cuando se usen en el argumento de órdenes como \ci{section}.  La compilación de su documento fracasará.  Tales órdenes se llaman \wi{órdenes frágiles} ---por ejemplo, \ci{footnote} o \ci{phantom}.  Estas órdenes frágiles necesitan protección.  Puede protegerlas precediéndolas con la orden \ci{protect}.

\ci{protect} sólo se refiere a la orden que le sigue, ni siquiera a sus argumentos.  En la mayoría de los casos un \ci{protect} superfluo no hará daño.

\begin{code}
\verb|\section{Soy muy considerado|\\
\verb|      \protect\footnote{y protejo mis notas al pie.}}|
\end{code}
