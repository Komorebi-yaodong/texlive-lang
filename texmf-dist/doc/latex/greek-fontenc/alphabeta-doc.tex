% alphabeta-doc: Documentation and tests for alphabeta.sty
% ********************************************************
%
% :Copyright: © 2010, 2015 Günter Milde
% :Licence:   This work may be distributed and/or modified under the
%             conditions of the `LaTeX Project Public License`_, either
%             version 1.3 of this license or any later version.
%
% This LaTeX document can be compiled with 8-bit TeX (latex or pdflatex),
% XeTeX (xelatex), or LuaTeX (lualatex).
% As it contains tests for the different limitations, there will be warnings
% in the log, which can be safely ignored.

\documentclass{article}

\usepackage[unicode,colorlinks=true,linkcolor=blue]{hyperref}
% \usepackage{bookmark}
\hypersetup{colorlinks=true,linkcolor=blue,urlcolor=blue,pdfencoding=auto}
% \usepackage{parskip}
\usepackage{amssymb, amsmath}
\frenchspacing

\ifdefined \UnicodeEncodingName % set by LaTeX for Unicode-aware engines
  \ifdefined\XeTeXrevision
    \newcommand{\engine}{XeTeX}
  \fi
  \ifdefined\luatexversion
    \newcommand{\engine}{LuaTeX}
  \fi
  % Setup for Unicode fonts (Xe-/LuaTeX)
  \usepackage[no-math,tuenc]{fontspec}
  \setmainfont{Linux Libertine O}
  \setsansfont{Linux Biolinum O}
  \setmonofont{Liberation Mono}[Scale=MatchLowercase]
  \usepackage[libertine,slantedGreek]{newtxmath}
  % \usepackage{unicode-math} % package conflict
  \newcommand{\fontset}{fontspec with Unicode fonts}
\else
  \newcommand{\engine}{8-bit TeX}
  \usepackage[utf8]{inputenc}
  \DeclareUnicodeCharacter{03DE}{\textKoppa} % in LGR mapped to \textkoppa
  \usepackage[LGR,T1]{fontenc}
  \usepackage{textcomp}
  \usepackage{lmodern}
  % \usepackage{libertine}
  % \usepackage{gfsdidot}
  % \usepackage{kerkis}
  % \usepackage{newtxtext,newtxmath}
  % \usepackage{substitutefont}
  % \substitutefont{LGR}{\rmdefault}{artemisia}
  \usepackage{isomath}
  \newcommand{\fontset}{fontenc with TeX fonts}
\fi

% load alphabeta after math setup and encoding setup!
% \usepackage{alphabeta}[2015/08/08]
\usepackage[normalize-symbols]{alphabeta}[2015/08/08]

% Fallback macros:

\newcommand*{\missing}{\ensuremath{\oslash}}
% varstigma only defined with 8-bit LGR fonts
\ProvideTextCommandDefault{\textvarstigma}{\missing}
% varkappa, only defined with newtxmath
\providecommand*{\varkappa}{\missing}
% varbeta and varTheta only defined with unicode-math
\providecommand*{\varbeta}{\missing}

% print current font encoding:
\makeatletter
\newcommand{\currentEncoding}{\f@encoding}
\makeatother

\begin{document}

\title{The \emph{alphabeta} package}
\author{Günter Milde}
\date{2020/10/30}
\maketitle

\begin{abstract}\noindent
The \emph{alphabeta} package makes the standard macros for Greek letters in
mathematical mode also available in text mode. This way, you can input Greek
letters ``by name'' everywhere in the document. The mode determines whether
the characters are taken from the text or math font.

With 8-bit TeX and
\emph{\href{http://www.ctan.org/pkg/greek-inputenc}{greek-inputenc}},
literal Unicode charactes can also be used in mathematical mode.
\footnote{
  This document was compiled using the \href{https://ctan.org/pkg/encguide}%
  {font encoding} \encodingdefault{}
  \ifdefined \UTFencname % defined by fontspec
     (Unicode fonts).
     For a version using 8-bit fonts, see
     \href{alphabeta-doc.pdf}{alphabeta-doc.pdf}.
  \else
     (8-bit fonts).
     For a version using Unicode fonts, see
     \href{alphabeta-tu.pdf}{alphabeta-tu.pdf}.
  \fi
}
\end{abstract}

% \tableofcontents

\section{Requirements and Conflicts}

The \emph{alphabeta} package depends on
\emph{\href{textalpha-doc.pdf}{textalpha}} (both are part of
\emph{\href{http://www.ctan.org/pkg/greek-fontenc}{greek-fontenc}}). It can
be used under 8-bit TeX as well as XeTeX/LuaTeX (this document is typeset
with \engine{} and \fontset). Depending on the converter and fonts, different
\hyperref[sec:limitations]{limitations} apply.

The package conflicts with
\href{http://www.ctan.org/pkg/unicode-math}{\emph{unicode-math}}.

It also fails, if the \texttt{utf8x} input encoding is selected (interface
to the \href{http://www.ctan.org/pkg/ucs}{\emph{ucs}} package using a
non-compatible definition of \verb+\DeclareUnicodeCharacter+).

\section{Usage}

Load this package in the preamble of your document (after font and math
setup) with
\begin{verbatim}
      \usepackage{alphabeta}
\end{verbatim}
%
Now you can write a single Greek symbol (like \Psi{} or \mu{}) or
a \lambda\omicron\gamma\omicron\varsigma{} in non-Greek text as well as
ISO-conforming formulas with upright symbols for constants
like $A = \text{\pi} r^2$ (instead of $A = \pi r^2$).%
\footnote{The
  \href{http://mirrors.ctan.org/macros/latex/contrib/isomath/isomath.html}%
  {\emph{isomath} documentation} describes more alternatives for upright
  Greek symbols in math mode.}

Just like Latin letters, the Greek counterparts are by default italic in
math mode%
\footnote{Capital Greek letters are upright in TeX unless a package selects
the ``ISO'' math-style. See the
\href{http://mirrors.ctan.org/macros/latex/contrib/isomath/isomath.html}%
{\emph{isomath} documentation} for a detailled discussion of math-styles.}
and upright in text:

\begin{quote}
  Text: L \Gamma{} l \gamma,
  mathematics: $ L \ \Gamma \ l \ \gamma $
\end{quote}
%
See the source of this document \url{alphabeta-doc.tex} for a setup and
usage example.


\subsection{options}

Package options are passed to the \href{textalpha-doc.pdf}{\emph{textalpha}}
package. Example call with options:

\begin{verbatim}
      \usepackage[normalize-symbols,keep-semicolon]{alphabeta}
\end{verbatim}

\texttt{normalize-symbols} merges ``letter'' and ``symbol`` variants of
some Greek letters to the ``letter'' character:
\footnote{The normalize-symbols option was added in version 0.13 (2015-08-03).
  Unicode input of the symbol variants requires at least version~1.6
  (2015-08-05) of
  \emph{\href{http://www.ctan.org/pkg/greek-inputenc}{greek-inputenc}}.}
Without this option, the symbol variant characters cannot be used in text,
because they are not supported by 8-bit Greek fonts (LGR encoding).
The \texttt{normalize-symbols} option is ignored, if you compile the
document with XeTeX or LuaTeX using Unicode fonts.
\textbf{Attention}: Be careful in cases where the distinction between the
symbol variants may be important (e.g. in a mathematical or scientific
context). Use XeTeX/LuaTeX with Unicode fonts or the respective characters
in mathematical mode (e.g. $\pi$ vs. $\varpi$).

The option \texttt{keep-semicolon} prevents conversion of the
semicolon to an \emph{ano teleia} in 8-bit TeX
(see \emph{\href{textalpha-doc.pdf}{textalpha-doc}}).

\subsection{symbol variants}

Mathematical notation uses variant shapes of some Greek letters as
additional symbols. The variations have no syntactic meaning in Greek text
and text fonts may use the variant shapes in place of the “regular” ones as
a stylistic choice.

Unicode defines separate code points for the symbol variants. TeX supports
some of the variant shape symbols in mathematical mode, but its concept of
“standard” vs. “variant” symbols differs from the distinction between
“GREEK LETTER ...” vs. “GREEK ... SYMBOL” in the Unicode standard.

The \emph{alphabeta} package defines generic macros for these variants that
are short forms of the set defined in \texttt{tuenc-greek.def}
(cf. \href{test-tuenc-greek.pdf}{test-tuenc-greek}):
\begin{quote}
  \verb|\<name>| selects the Unicode GREEK LETTER ... variant,

  \verb|\<name>symbol| selects the Unicode
     GREEK ... SYMBOL variant,

  \verb|\var<name>| selects the variant
    shape according to TeX' mathematical mode
\end{quote}
See Table \ref{tab:symbol-variant-macros} for the full list.

\section{Limitations \label{sec:limitations}}

With 8-bit TeX, the limitations described in the
\href{textalpha-doc.pdf}{textalpha documentation} apply. See also the tests
in section \hyperref[sec:8-bit-limitations]{8 bit limitations}.
These limitations are lifted, if the text language is switched to ``greek''
with Babel, the text part is wrapped in \verb+\ensuregreek+, or the
document is compiled with XeTeX/LuaTeX.

With XeTeX/LuaTeX and Unicode fonts, literal Unicode characters cannot be
used in formulas (the log file reports missing characters) This is a generic
TeX limitation which \emph{alphabeta} overcomes if used under 8-bit TeX.
Under XeTeX/LuaTeX it may be lifted using the
\href{http://www.ctan.org/pkg/unicode-math}{\emph{unicode-math}} package.
However, \emph{unicode-math} conflicts with \emph{alphabeta}.


\section{Tests and examples}

\subsection{Greek alphabet}

Greek letters via generic ``name'' macros without language/font-encoding
switch (active font encoding \encodingdefault):

\begin{quote}
  \Alpha{} \Beta{} \Gamma{} \Delta{} \Epsilon{} \Zeta{} \Eta{} \Theta{}
  \Iota{} \Kappa{} \Lambda{} \Mu{} \Nu{} \Xi{} \Omicron{} \Pi{} \Rho{}
  \Sigma{} \Tau{} \Upsilon{} \Phi{} \Chi{} \Psi{} \Omega{}
  \\
  \alpha{} \beta{} \gamma{} \delta{} \epsilon{} \zeta{} \eta{} \theta{}
  \iota{} \kappa{} \lambda{} \mu{} \nu{} \xi{} \omicron{} \pi{} \rho{}
  \sigma{} \varsigma{} \tau{} \upsilon{} \phi{} \chi{} \psi{} \omega{}
  \\
  \digamma{} \Digamma{} \stigma{} \varstigma{}%
    \footnote{There is no separate Unicode code point for a stigma variant
      symbol, \texttt{\textbackslash varstigma} is not defined with
      Xe/LuaTeX and similar to \texttt{\textbackslash stigma} in some fonts.}
  \koppa{} \Koppa{}%
    \footnote{In LGR, there is no separate code point for Koppa as
      typographical practice normally does not observe a
      contrast between uppercase and lowercase forms for numeric koppa.}
      % https://en.wikipedia.org/wiki/Koppa_(letter)
  \qoppa{} \Qoppa{} \Stigma{} \Sampi{} \sampi{}
\end{quote}
%
Greek letters via Unicode (active font encoding \encodingdefault):

\begin{quote}
  Α Β Γ Δ Ε Ζ Η Θ Ι Κ Λ Μ Ν Ξ Ο Π Ρ Σ Τ Υ Φ Χ Ψ Ω\\
  α β γ δ ε ζ η θ ι κ λ μ ν ξ ο π ρ σ ς τ υ φ χ ψ ω\\
  ϝ Ϝ ϛ ϟ Ϟ ϙ Ϙ Ϛ Ϡ ϡ
\end{quote}

\subsection{Diacritics}

Accent macros are set up for use with the generic macros by definition of
``TextComposite'' commands.

Diacritics (except the dialytika) should placed
before capital letters and dropped with MakeUppercase:

\begin{quote}
\ensuregreek{
\<{\alpha} \>{\epsilon} \"'{\iota} \>`{\eta}
\'<{\omicron} \~<{\upsilon} \~>{\omega}
\\
\<{\Alpha} \>{\Epsilon} \'{\Iota} \>`{\Eta}
\'<{\Omicron} \~<{\Upsilon} \~>{\Omega}
\\
\MakeUppercase{%
 \<{\alpha} \>{\epsilon} \"'{\iota} \>`\eta{}
 \'<{\omicron} \~<{\upsilon} \~>{\omega}
}}
\end{quote}


\subsection{normalize-symbols}

The \texttt{normalize-symbols} option merges ``letters'' and ``symbol``
variants of some Greek letters to the ``letter'' character. It is ignored,
if the document uses Unicode fonts and is compiled with XeTeX or LuaTeX.
(This document is compiled using \engine.)
\begin{quote}
  The source of this quote uses both variants for beta (β|ϐ),
  epsilon (ε|ϵ), phi (φ|ϕ), kappa (κ|ϰ), pi (π|ϖ), rho (ρ|ϱ), theta (θ|ϑ),
  and Theta (Θ|ϴ) in the LaTeX source.%
\end{quote}


\subsection{\ensuregreek{%
    Ἑλληνικά (\<\Epsilon\lambda\lambda\eta\nu\iota\kappa\'\alpha{})}
  in PDF strings}

With the alphabeta package, you get Greek letters in both, the document body
and PDF metadata generated by hyperref if the input uses Unicode literals or
macros. Wrapping in \verb+\ensuregreek+ ensures the right placement of the
accents and breathings (before, not above capital letters). With LICR input
(accent macros as well as symbol macros), non-standard diacritics are
missing in the PDF data, as hyperref's PU encoding currently does not
support polytonic Greek. (Here, the dasia is dropped at the start of the
word in parentheses in the PDF toc. The warning ``\texttt{Glyph not defined
in PU encoding, removing `\textbackslash<' on input line 145.}'' is written
to the log.)


\subsection{Greek in maths $\Gamma = \sin\alpha / \cos{\beta}$}

In the main document, Greek in math mode should work as usual:

\[\Gamma = \frac{\sin\alpha}{\cos{\beta}}.
\]
Greek letters and symbols in math input as macro:%
\footnote{There are no math macros for Greek letters wich exist with similar
shape in the Latin alphabet}
\begin{align*}
  &
  % \Alpha{} \Beta{}
  \Gamma{} \Delta{}
  % \Epsilon{} \Zeta{} \Eta{}
  \Theta{}
  % \Iota{} \Kappa{}
  \Lambda{}
  % \Mu{} \Nu{}
  \Xi{}
  % \Omicron{}
  \Pi{}
  % \Rho{}
  \Sigma{}
  % \Tau{}
  \Upsilon{} \Phi{}
  % \Chi{}
  \Psi{} \Omega{}
\\&
  \alpha{} \beta{} \gamma{} \delta{} \epsilon{} \zeta{} \eta{} \theta{}
  \iota{} \kappa{} \lambda{} \mu{} \nu{} \xi{}
  % \omicron{}
  \pi{} \rho{}
  \sigma{} \varsigma{} \tau{} \upsilon{} \phi{} \chi{} \psi{} \omega{}
\\&
  \vartheta \varphi \varpi \digamma{} \varrho \varepsilon
\end{align*}
%
PDF strings do not know math mode. The content of a formula or equation is
evaluated in text mode with non-valid commands discarded (and warnings
written to the log). This works resonably well for simple formulas (but not,
e.g., for super-/subscripts). With the \emph{alphabeta} package, it works
also for Greek letters.

\subsection{Greek Unicode characters in math (only under 8-bit TeX)}

With the \texttt{utf8} option of \emph{inputenc} and
\href{http://www.ctan.org/pkg/greek-inputenc}{\emph{greek-inputenc}},
literal Greek Unicode characters are supported also in
mathematical mode.

\ifdefined\DeclareUnicodeCharacter
  \[
       Γ = \frac{\sin α}{\cos β}.
  \]
  Greek letters and symbols in math input as Unicode literals:
  \begin{align*}
               & Γ ΔΘΛΞΠΣΥ ΦΨ Ω \\
                 & αβγδεζηθικλμνξπρσςτυφχψω \\
               & ϑϕϖϝϱϵ
  \end{align*}
\fi

This does not work with XeTeX/LuaTeX (unless in 8-bit emulation mode).

The ``normal'' vs. ``variant'' shape of letters is used so that the output
matches the Unicode reference glyph (cf. Table
\ref{tab:symbol-variant-macros}). This corresponds to the behaviour of
\href{http://www.ctan.org/pkg/unicode-math}{\emph{unicode-math}}.

\subsection{8-bit limitations \label{sec:8-bit-limitations}}

Certain limitations apply if Greek symbols are used in non-Greek context.

\begin{itemize}

\item Composition of diacritics (like \verb+\>\'+) fails:
      \<{\alpha} \>{\epsilon} \"'{\iota} \>`\eta{}
      \'<{\omicron} \~<{\upsilon} \~>{\omega}

      Simple diacritics and long names (like \verb+\accdasiaoxia+) work in
      any font encoding, however they do not select precomposed characters
      (the difference becomes obvious if you drag-and-drop text from the PDF
      version of this document):
      %
      \ensuregreek{\<'\alpha{} \accdasia\acctonos\alpha{} \accdasiaoxia\alpha{}
      (\currentEncoding)} vs. \accdasiaoxia\alpha{} (\currentEncoding)

\item MakeUppercase fails with composite diacritics in other font encodings.
      % \MakeUppercase{%
      %  \<\alpha{} \>\epsilon{} \'\iota{} \`\eta{} \~\upsilon{}
      % }

\item There is no kerning between Greek letters, if the font encoding does not
      support Greek: compare \ensuregreek{\Alpha\Upsilon\Alpha{}
      (\currentEncoding)} to \Alpha\Upsilon\Alpha{} (\currentEncoding).
\end{itemize}
%
The \verb+\ensuregreek+ macro ensures that the argument is typeset with a
font encoding supporting Greek. This keeps kerning (if the kerning pair is
inside the argument, \ensuregreek{\Alpha\"\Upsilon\Alpha}), and allows
combining of accent macros where pre-composed characters are selected
(\ensuregreek{\<'\alpha}).
Setting the corrct language for Greek quotes with the \emph{Babel} package
additionally ensures correct hyphenation .


\begin{table}[bp]
  \centering
  \begin{tabular}{lcc}
  \hline
  macro & text & math \\
  \hline
  \verb$\beta$          & \beta          & $\beta$       \\
  \verb$\varbeta$       & \varbeta       & $\varbeta$    \\
  \verb$\betasymbol$    & \betasymbol    & $\betasymbol$ \\
  \hline
  \verb$\epsilon$       & \epsilon       & $\epsilon$    \\
  \verb$\varepsilon$    & \varepsilon    & $\varepsilon$ \\
  \verb$\epsilonsymbol$ & \epsilonsymbol & $\epsilonsymbol$ \\
  \hline
  \verb$\phi$           & \phi           & $\phi$        \\
  \verb$\varphi$        & \varphi        & $\varphi$     \\
  \verb$\phisymbol$     & \phisymbol     & $\phisymbol$  \\
  \hline
  \verb$\kappa$         & \kappa         & $\kappa$      \\
  \verb$\varkappa$      & \varkappa      & $\varkappa$   \\
  \verb$\kappasymbol$   & \kappasymbol   & $\kappasymbol$ \\
  \hline
  \verb$\pi$            & \pi            & $\pi$         \\
  \verb$\varpi$         & \varpi         & $\varpi$      \\
  \verb$\pisymbol$      & \pisymbol      & $\pisymbol$   \\
  \hline
  \verb$\rho$           & \rho           & $\rho$        \\
  \verb$\varrho$        & \varrho        & $\varrho$     \\
  \verb$\rhosymbol$     & \rhosymbol     & $\rhosymbol$  \\
  \hline
  \verb$\sigma$         & \sigma         & $\sigma$      \\
  \verb$\varsigma$      & \varsigma      & $\varsigma$   \\
  \verb$\finalsigma$    & \finalsigma    & $\finalsigma$ \\
  \hline
  \verb$\theta$         & \theta         & $\theta$      \\
  \verb$\vartheta$      & \vartheta      & $\vartheta$   \\
  \verb$\thetasymbol$   & \thetasymbol   & $\thetasymbol$ \\
  \hline
  \verb$\Theta$         & \Theta         & $\Theta$      \\
  \verb$\varTheta$      & \missing       & $\varTheta$   \\
  \verb$\Thetasymbol$   & \Thetasymbol   & \missing      \\
  \hline
  \end{tabular}
  \caption{Macros for Greek symbol variants (\missing = symbol only
    available with additional packages).
    With 8-bit TeX and the \texttt{normalize-symbols} option, the output for
    both variants in text mode is the same (8-bit Greek text fonts contain
    only one symbol variant). \label{tab:symbol-variant-macros}}
\end{table}

\end{document}
