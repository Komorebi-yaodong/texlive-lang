% Test definitions for accents and composite accents in Greek
\documentclass[a4paper,polutonikogreek,british]{article}
\pagestyle{empty}
\usepackage[margin=2.9cm]{geometry}
% \usepackage{cmap} % fix search and cut-and-paste in Acrobat Reader

\usepackage%
{lmodern}
% {gfsartemisia}
% {gfsbaskerville}
% [default]{gfsbodoni}
% [default]{gfscomplutum}
% {gfsdidot}
% [default]{gfsneohellenic}
% {lmodern} \usepackage{gfsporson} \renewcommand*\rmdefault{porson}
% [default]{gfssolomos}
% {kerkis}
% {teubner}
% \renewcommand{\ttdefault}{txtt}

% Load encoding definitions (after font package)
\usepackage[LGR,T1]{fontenc}
\usepackage{textalpha}

\usepackage{listings}
\lstset{basicstyle=\ttfamily}

% Babel package:

\usepackage{babel}

% With XeTeX/LuaTeX, load fontspec after babel to use Unicode
% fonts for Latin script and LGR for Greek:
\ifdefined\luatexversion \usepackage{fontspec}\fi
\ifdefined\XeTeXrevision \usepackage{fontspec}\fi


% "Lipsiakos" italic font `cbleipzig`:
\newcommand*{\lishape}{\fontencoding{LGR}\fontfamily{cmr}%
		       \fontshape{li}\selectfont}
\DeclareTextFontCommand{\textli}{\lishape}


\begin{document}

% \selectlanguage{british}

\title{Greek diacritics with standard accent macros}
\author{G\"unter Milde}
\date{\filedate}
\maketitle

The font encoding definition file \texttt{lgrenc.def} defines LICR macros
for all non-ASCII characters in the LGR font encoding. Greek accent macros
have names starting with \verb|\acc| followed by the Greek accent name as
used in the Unicode standard (e.g. \verb|\acctonos|). The standard symbol
accents \verb|\' \` \~| behave according to Greek typography if used in the
LGR font encoding.

The \emph{textalpha} or \emph{alphabeta} packages define the symbol macros
\verb|\<| and \verb|\>| as alias for the breathings (Dasia and Psili).%
\footnote{The definition of the macros 
          \texttt{\textbackslash<} and \texttt{\textbackslash>} was moved
	  from the font definition file \texttt{greek-fontenc.def} to
	  \texttt{textalpha.sty} in order to avoid clashes with local
	  definitions of this macros in documents using the LGR font
	  encoding via \emph{fontenc} or \emph{babel}.}
With these packages, all Greek diacritics can be input as backslash followed
by the LGR transliteration.%
\footnote{This makes it easy to follow the advise in
	  \emph{teubner-doc}: ``typeset your paper with the regular
	  accent vowel ligatures and [{\ldots}] substitute them in the
	  final revision with the accented vowel macros only in those
	  instances where the lack of kerning is disturbing''.}

The example in babel/contrib/greek/usage.pdf:
%
\begin{quote} \selectlanguage{greek}
    T\'i f\'hic? \<Id\`wn \>enj\'ede pa\~id''
    \>eleuj\'eran t\`ac plhs\'ion N\'umfac stefano\~usan,
    S\'wstrate, \>er\~wn \'ap\~hljec e\>uj\'uc?
\end{quote}
can be input as
\begin{lstlisting}
    T\'i f\'hic? \<Id\`wn \>enj\'ede pa\~id''
    \>eleuj\'eran t\`ac plhs\'ion N\'umfac stefano\~usan,
    S\'wstrate, \>er\~wn \'ap\~hljec e\>uj\'uc?
\end{lstlisting}
%
Improvements over the ligature-based approach in LGR:
%
\begin{itemize}

\item Accents can be placed on any character:
  \textgreek{\"k \`l \'m \~<n \<o \>'p \>9 \`\>-}

\item Kerning is preserved
  \selectlanguage{greek}
  \begin{tabular}[t]{llll}
     & \textlatin{roman} & \textlatin{italic} & \textlatin{cbleipzig} \\
    \foreignlanguage{british}{accent macro:}  &
      a\>ut'os & \emph{a\>ut\'os} & \textli{a\>ut\'os} \\
    \foreignlanguage{british}{transliteration:} &
      a>ut'os & \emph{a>ut'os} & \textli{a>ut'os}\\
  \end{tabular}
  \selectlanguage{british}

  Like in any font encoding, kerning only works with pre-composed glyphs:\\
  \textgreek{A\"UA $\ne$ A\~UA}, AVA $\ne$ A\'VA.

\item Compatible with hyperref (see greekhyperref.pdf).

\item Following Greek typesetting convention, diacritics (except the
  dialytika) are placed to the left of capital letters and and dropped
  by \verb|\MakeUppercase|:

  \begin{quote} \selectlanguage{greek}
    \'antropos $\mapsto$ \MakeUppercase{\'antropos},
    \>'antropos $\mapsto$ \MakeUppercase{\>'antropos},\\
    Aqill\'eas $\mapsto$ \MakeUppercase{Aqill\'eas},
    \>Aqille\'us $\mapsto$ \MakeUppercase{\>Aqille\'us}.
  \end{quote}

  The dialytika is printed even in cases where it's not needed
  in lowercase: % the "hiatus" feature
  \begin{quote} \selectlanguage{greek}
    \'aulos $\mapsto$ \MakeUppercase{\'aulos},
    \'>aulos $\mapsto$ \MakeUppercase{\'\>aulos}, 
    % from http://diacritics.typo.cz/index.php?id=69  μάινα -> ΜΑΪΝΑ
    m\'aina $\mapsto$ \MakeUppercase{m\'aina},\\
    % from  http://de.wikipedia.org/wiki/Neugriechische_Orthographie#Das_Trema
    % κέικ, ἀυπνία/αϋπνία
    k\'eik, $\mapsto$ \MakeUppercase{k\'eik},
    \>aupn\'ia $\mapsto$ \MakeUppercase{\>aupn\'ia}.
  \end{quote}
\end{itemize}

\selectlanguage{british} Composite diacritics can be specified as
named macro, backslash + LGR transliteration, or combined accent macros, 
e.\,g. \textgreek{\~>a} can be written as
\begin{quote}
  \verb+\accpsiliperispomeni{a}+,
  \verb+\~>a+, \verb+\>~a+,
  \verb+\~\>{a}+, or \verb+\~\>a+.
\end{quote}
However, braces in composite accents
(\verb+\~{\>a}+, \verb+\~{>a}+, or \verb+\~{\>{a}}+)
lead to errors.

\verb+\MakeUppercase+ works with most input variants but fails with a tilde
in a document which does \textbf{not} define the \texttt{greek} or
\texttt{polutonikogreek} language with Babel (which fixes the uccode for
characters used in the LGR transliteration).
Combining ``symbol macros'' (\verb+\>\~+) or reversing the order
(\verb+\~>+) is safe.
% \foreignlanguage{greek}{\~>a \>~a  \>\~a \~\>a $\mapsto$
%          \MakeUppercase{\~>a \>~a \>\~a \~\>a}}.

Accent macros can start with \verb|\a| instead of \verb|\| when the
short form is redefined, e.\,g. inside a \emph{tabbing} environment.
This also works for the new-defined Dasia and Psili shortcuts
(\verb|\a<| and \verb|\a>|):
%
\begin{quote}
\selectlanguage{greek}
\begin{tabbing}
T'i f'hic? \= T\a'i f\a'hic? \\
<Id`wn \> \a>enj\a'ede pa\a~id
\end{tabbing}
\end{quote}

\end{document}
