% Usar XeLaTeX ou LuaLaTeX
% acento nao sobrevive no arquivo temporario de showexpl
% Use diretamente do arquivo externo, ou use XeLaTeX/LuaLaTeX
%
\makeatletter
\@ifclassloaded{book}{}{
  \documentclass[12pt,a4paper,twoside,openright]{book}
  %  \newif\iftwosidedocument
  %  \twosidedocumenttrue
}
\makeatother

\sloppy % prefer underfull than overfull

\usepackage{iftex} % detectar motor TeX usado

% testing TeX engine
\ifPDFTeX
  \PackageError{\jobname}{Precisa de XeLaTeX ou LuaLaTeX}{Use XeLaTeX ou LuaLaTeX}
\fi

\usepackage[english,brazil]{babel}

% Font selection
\usepackage{fontspec}
\defaultfontfeatures{Ligatures=TeX}

% iffont not works on LuaLaTeX. Need patch it.
% On XeLaTeX need patch too for external font
\usepackage{iffont}

%\makeatletter
%  \renewcommand{\@iffont@firstfont}{MS Mincho}
%\makeatother
  
%\ifLuaTeX 

% patch on iffont
  \ifx \IfFontExistsTF\undefined
    %\PackageError{\jobname}{fontspec antigo}{Atualize para versão mais nova ou use XeLaTeX}
    \PackageError{\jobname}{fontspec antigo}{Atualize para versão mais nova}
  \fi
  % \usepackage{etoolbox} % iffont ja carregou
  \let\iffontexists\IfFontExistsTF
  \makeatletter
    \renewcommand{\@iffont@checkfont}[1]{
      \IfFontExistsTF{#1}{
        \nottoggle{@iffont@fontfound}{%
          \renewcommand{\@iffont@firstfont}{#1}
          \toggletrue{@iffont@fontfound}
        }{}
      }{\stepcounter{@iffont@fontsnotfound} }
    }
  \makeatother
%\fi  
  
%\ifXeTeX % need file name for font outside tex tree
%  \settofirstfound{\jafont}{NotoSerifCJK-Regular.ttc, msmincho.ttf, takaomincho.ttf, ipaexm.ttf, ipam.ttf}
%  \settofirstfound{\jagothic}{NotoSansCJK-Bold.ttc, msgothic.ttf, takaogothic.ttf, ipaexg.ttf, ipag.ttf}
%\else % LuaLa\teX (will use font name for all)  
  \settofirstfound{\jafont}{Noto Serif CJK JP, MS Mincho, TakaoMincho, IPAexMincho, IPAMincho, Hiragino Mincho Pro}
  \settofirstfound{\jagothic}{Noto Sans CJK JP, MS Gothic, TakaoGothic, IPAexGothic, IPAGothic, Hiragino Maru Gothic Pro}
%\fi

% DEBUG XeLaTeX
%\renewcommand\jafont{NotoSerifCJK-Regular.ttc}
%\renewcommand\jagothic{NotoSansCJK-Bold.ttc}

%  % versao 2016 nao tem \IfFontExistsTF no fontspec
%  \ifx \IfFontExistsTF\undefined
%    \PackageError{\jobname}{fontspec antigo}{Atualize para versão mais nova}
%  \fi
%
%  \IfFontExistsTF{Noto Serif CJK JP}{\newcommand{\jafont}{Noto Serif CJK JP}}%
%  {\IfFontExistsTF{MS Mincho}{\newcommand{\jafont}{MS Mincho}}%
%  {\IfFontExistsTF{TakaoMincho}{\newcommand{\jafont}{TakaoMincho}}%
%  {\IfFontExistsTF{IPAexMincho}{\newcommand{\jafont}{IPAexMincho}}%
%  {\IfFontExistsTF{IPAMincho}{\newcommand{\jafont}{IPAMincho}}%
%  {\IfFontExistsTF{Hiragino Mincho Pro}{\newcommand{\jafont}{Hiragino Mincho Pro}}%
%  {\newcommand{\jafont}{MS Mincho}}}}}}}
%  
%  \IfFontExistsTF{Noto Sans CJK JP}{\newcommand{\jafont}{Noto Sans CJK JP}}%
%  {\IfFontExistsTF{MS Gothic}{\newcommand{\jafont}{MS Gothic}}%
%  {\IfFontExistsTF{TakaoGothic}{\newcommand{\jafont}{TakaoGothic}}%
%  {\IfFontExistsTF{IPAexGothic}{\newcommand{\jafont}{IPAexGothic}}%
%  {\IfFontExistsTF{IPAGothic}{\newcommand{\jafont}{IPAGothic}}%
%  {\IfFontExistsTF{Hiragino Maru Gothic Pro}{\newcommand{\jafont}{Hiragino Maru Gothic Pro}}%
%  {\newcommand{\jafont}{MS Gothic}}}}}}}  

% versao antiga (2016) do fontspec ainda nao tem \IfFontExistsTF
% precisa atualizar o fontspec

%\ifXeTeX
%  \ifx \IfFontExistsTF\undefined
%    \let\IfFontExistsTF\iffontexists
%  \fi
%\else\ifLuaTeX
  \ifx \IfFontExistsTF\undefined
    \PackageError{\jobname}{fontspec is old}{Install updated fontspec package}
  \fi
%\fi\fi

\IfFontExistsTF{\jafont}{}{\PackageError{\jobname}{Fonte japonês Mincho (\jafont) não encontrada.}{Instale um dos seguintes: Noto Serif CJK JP, MS Mincho, TakaoMincho, IPAexMincho, IPAMincho, Hiragino Mincho Pro}}
\IfFontExistsTF{\jagothic}{}{\PackageError{\jobname}{Fonte japonês Gothic não encontrada.}{Instale um dos seguintes: Noto Sans CJK JP, MS Gothic, TakaoGothic, IPAexGothic, IPAGothic, Hiragino Maru Gothic Pro}}

\newfontfamily{\jafamily}[BoldFont=\jagothic]{\jafont}
\newcommand{\textja}[1]{{\jafamily #1}}

% amsmath deve ser carregado antes do unicode-math,
% se precisar de alguns comandos de amsmath que nao esta
% em unicode-math como o \DeclareMathOperator

% \usepackage{amssymb} % para \circlearrowleft que nao tem no unicode-math
\usepackage{amsmath}

% \usepackage{xunicode} % textcomp equiv. Old and not works on lualatex (March, 2018). Now, fontspec features are sufficient for this
\usepackage{unicode-math} % amssymb,amsmath equiv.
\ifLuaTeX
  \usepackage{lualatex-math}
\fi

% Other packages
\usepackage{xcolor,tikz} % graphics

% margins
\usepackage[margin=2cm]{geometry}

\usepackage[hyphens]{url} % will break url across line on hyphens
%\makeatletter
%  \g@addto@macro{\UrlBreaks}{\UrlOrds}
%\makeatother
% \usepackage{xurl}

\usepackage{hyperref} % link
\hypersetup{
  unicode=true,
  breaklinks=true,
  bookmarks=true
}
\usepackage{bookmark} % to add bookmark manually

% \providecommand{\url}[1]{\texttt{#1}}

\usepackage{lipsum}

\usepackage[stretch=10]{microtype} % microtipografia: funciona no PDFLaTeX e LuaLaTeX (parcialmente em XeLaTeX)
% atrapalha a copiar texto do PDF

\usepackage{setspace}

\onehalfspacing

% \usepackage{fancyhdr}

\usepackage{bm} % fórmula matematica em negrito.

\usepackage{enumerate} % controle de enumeração
% \usepackage{hhline} % linhas duplas na tabela
\usepackage{dcolumn} % alinhar decimais
\usepackage{tabularx} % igualar colunas
\usepackage{longtable}

\usepackage{makeidx}
\makeindex

\usepackage[nottoc]{tocbibind}


\usepackage{amsthm}

\usepackage{multicol}

\usepackage{makeidx} % pacote para produzir
% indice remissvo (glossário)
\makeindex % ativando o indice remissivo (glossário)

 % To solve probblem of \MakeUppercase that not works for accent char
 % and change to uppercase, the formula, ref, cite, etc too.
\usepackage[overload]{textcase}
\usepackage{listings} % format source code
\usepackage{verbatim} % For verbatim
\usepackage{fancyvrb} % for VerbatimOut

\usepackage{keyval}

\usepackage{showexpl} % ele carrega listings

%%% pacotes usados nos exemplos correspondentes
%
%\usepackage{tikz} % pacote gráfico
\usetikzlibrary{babel} % para compatiblidade com o pacote babel, requerido por agumas bibliotecas como o cd.
\usetikzlibrary{calc} % calc eh para efetuar calculos matematicos ou expressoes em coordenadas
\usetikzlibrary{through} % circulo passando por ponto, por exemplo.
\usetikzlibrary{patterns} % preenchimentos
\usetikzlibrary{intersections} % interseccao entre caminhos
\usetikzlibrary{matrix} % matriz no tikz
\usetikzlibrary{cd} % diagrama comutativa

\usepackage{lettrine} % Para paragrafo no estilo biblia
\usepackage{caption} % configurar o titulo da figura e tabela
\usepackage{subcaption} % subfigura ou subtabela
\usepackage{wrapfig} % texto contornando a figura ou tabela
\usepackage{pdflscape} % rotaciona uma página
\usepackage{pict2e} % para incrementar ambiente picture
\usepackage[mathlines]{lineno} % enumerar linhas de documento
\usepackage{blindtext} % gera parágrafos de textos sem sentido, para testes
\usepackage{fancybox} % caixas enfeitadas
\usepackage{framed} % moldura no texto
%\usepackage{exsheets} % para lista de exercicios e provas
% saida do xsim é formatado manualmente sem usar este pacote
%\usepackage{xsim} % sucessor de exsheets (gera exercicios e respostas) 
\usepackage{tasks} % para lista de exercicios e provas
\usepackage{answers} % respostas
\usepackage{siunitx} % para unidade internacional.
\usepackage[shortlabels,inline]{enumitem} % para controle de listas e listas compactas
\usepackage{tabto} % para tabular itens da lista compacta.
% o pacote tasks ja carregou o ulem. Carregar de novo causa erro
% Agora tasks não carrega mais (carregar explicitamente)
\usepackage[normalem]{ulem} % sublinhar,tachar, etc no texto
\usepackage{cancel} % cancelar elemento
% para cortar seta longa (negação da seta longa)
\usepackage{centernot} % cortar símbolo no centro
% fancyhdr and titlesec defines headrule and footrule,
% Use option pagestyles on titlesec package and
% load fancyhdr after titlesec
\usepackage[pagestyles]{titlesec} % para exemplo de titlesec
%\let\headrule\undefined
%\let\footrule\undefined
\usepackage{fancyhdr}
% para criar contorno
\usepackage{contour}
% texto sombreado
\usepackage{shadowtext}

%% caixa colorida usada para exemplo de certificado.
% \usepacakge{tcolorbox}

% bug no titlesec 2.10.1 remove enumeracao da secao (corrigido em 2.10.2 de 21/03/2016).
% O Usuario de ubuntu 16.04 devem atualizar o titlesec
% https://tex.stackexchange.com/questions/299969/titlesec-loss-of-section-numbering-with-the-new-update-2016-03-15/300259
%
% Atualize o titlesec no ubuntu 16.04
% sudo wget http://mirrors.ctan.org/macros/latex/contrib/titlesec/titlesec.sty -O /usr/share/texlive/texmf-dist/tex/latex/titlesec/titlesec.sty

%========================================================
% rodap\'e deve estar em espa\c{c}amento simples,
% independente do espa\c{c}amento no corpo de documento
% Devera ser feito com linespread{} \@currsize
% pois comandos do setspace nao funciona como desejado
%========================================================
% foi copiado do book e alterado
\makeatletter % ativa uso de ``@'' no nome
\renewcommand\@makefntext[1]{%
    \parindent 1em%
    % inicio da altera\c{c}\~ao    
    \linespread{1} \@currsize \noindent
    % \hb@xt@1.8em{\hss\@makefnmark}#1}
    \hb@xt@0.45em{\hss\@makefnmark}#1}
% fim da altera\c{c}\~ao
\makeatother % desativa o uso de ``@'' no nome

\let\orilstlistingname\lstlistingname
\renewcommand{\lstlistingname}{Exemplo}
\renewcommand{\lstlistlistingname}{Lista de \lstlistingname s}
% setup  listings
\lstset{
%  language=[LaTeX]TeX,
  numbers=none,
  backgroundcolor=\color{yellow!30},
  breaklines=true,
  breakautoindent=true,
  % breakatwhitespace=true,    % quebra linha somente no espaço
  belowcaptionskip=1\baselineskip, % espaço após o título
  % aboveskip=2ex,
  %keepspaces=true, % to keep indentation
  columns=fullflexible, % para poder copiar codigo do PDF (fixed, flexible, fullflexible)
  frame=lines, %none, single, line % do codigo
}

% aplicando patch no codigo de showexpl para que o parametro abovecaptionskip funcione tambem no LTXexample

\makeatletter
\renewcommand\SX@KillAboveCaptionskip{%
  \ifx\lst@caption\@empty\else
    \lst@IfSubstring t\lst@captionpos
      {\vskip-\abovecaptionskip}{}%
  \fi
  \ifx\lst@belowcaption\undefined \else %adicionado
    \belowcaptionskip=\lst@belowcaption\relax % adicionado
  \fi %adicionado
}
\makeatother

% setup showexpl
\lstset{explpreset={
  pos=b,
  belowcaptionskip=0.5\baselineskip, % espaço após o título
%  language=[LaTeX]TeX,
  %basicstyle=\ttfamily\small,
  %identifierstyle=\color{black},
  %keywordstyle=\color{blue},
  %commentstyle=\color{red},
  %extendedchars=true,
  % extendedchars=false
  %showspaces=false,
  %showstringspaces=false,
%  numbers=none,
  %numberstyle=\tiny,
%  breaklines=true,
  %backgroundcolor=\color{lightgray},
%  breakautoindent=true,
  %captionpos=b,
  %xleftmargin=0pt,
  % captionpos=t.
  % aboveskip=2ex,
  % columns=fullflexible, 
%  columns=flexible, % para poder copiar codigo do PDF
  % para editor (sem ele, terq espaço indesejado)
  % from http://latex.org/forum/viewtopic.php?t=5636
  % hsep=2ex,  
%  frame=lines, %none, single, line % do codigo
  %rframe={single} % do resultado
  }
}

% para arquivo de BibTeX no listings
% https://tex.stackexchange.com/questions/85998/include-bibtex-contents-in-a-listings-block 
\lstdefinelanguage{BibTeX}
  {keywords={%
      @article,@book,@collectedbook,@conference,@electronic,@ieeetranbstctl,%
      @inbook,@incollectedbook,@incollection,@injournal,@inproceedings,%
      @manual,@mastersthesis,@misc,@patent,@periodical,@phdthesis,@preamble,%
      @proceedings,@standard,@string,@techreport,@unpublished%
      },
   comment=[l][\itshape]{@comment},
   sensitive=false,
  }

\lstdefinestyle{lstexample}
{
%  language=[LaTeX]TeX,
%  keywordstyle=\bfseries,
%  identifierstyle=\itshape,
  backgroundcolor=\color{white},
  numbers=none,
  breaklines=true,
  breakautoindent=true,
%  belowcaptionskip=1\baselineskip, % espaço após o título
%  aboveskip=2ex,
  %keepspaces=true, % to keep indentation
  columns=fullflexible, % para poder copiar codigo do PDF
  frame=lines, %none, single, line % do codigo
%  explpreset={
%    attachfile=false,true
%    codefile=temporaryfilename,
%    explpreset=? defualti values is?
%    graphic=alternativegraphicsfileforoutput,
%    hsep=,
%    justification=\raggedright,% \raggedleft, \raggedright, \centering
%    overhang=,
%    pos=, % e t, b, l, r, and i
%    preset={},
%    rangeaccept=false, % true/false
%    rframe=,
%    varwidth=,
%    hsep,
%    wide,
%    width=
%  }
}

% LTXexample's parameter
%    attachfile=false,trure
%    codefile=temporaryfilename,
%    explpreset=? defualti values is?
%    graphic=alternativegraphicsfileforoutput,
%    hsep=,
%    justification=\raggedright,% \raggedleft, \raggedright, \centering
%    overhang=,
%    pos=, % e t, b, l, r, and i
%    preset={},
%    rangeaccept=false, % true/false
%    rframe=,
%    varwidth=,
%    hsep,
%    wide,
%    width=
%  }

% showexpl's LTXexample environment like ltxlisting
% is implemented (it use showexpl's preset (thus, need showexpl package}

\makeatletter 

% Para usar no exemplo de babel
\newenvironment{abstract}%
{\par\begin{center}\textbf{\small\abstractname}\vspace{-0.5em}\end{center}%
\begin{quotation}}%
{\end{quotation}}

%\newenvironment{abstract}{%
%  \if@twocolumn
%    \section*{\abstractname}%
%  \else
%    \small
%    \begin{center}%
%      {\bfseries \abstractname\vspace{-.5em}\vspace{\z@}}%
%    \end{center}%
%    \quotation
%  \fi}
%{\if@twocolumn\else\endquotation\fi}

\let\orititle\title      
\let\oriauthor\author
\let\oridate\date
\newcommand{\makearticletitle}{%
  \begin{center}%
%    T\'{\i}tulo: 
    {\LARGE \@title \par}%
    \vspace*{1.5em}%
    {\large\center \@author }%
    \ifx\@date\@empty%
    \else%
      \par%
      \vspace*{1em}%
      {\large \@date}%
    \fi%
  \end{center}%
  \vspace*{1.5em}%
  \global\let\@author\@empty
  \global\let\@date\@empty
  \global\let\@title\@empty
  \global\let\title\relax
  \global\let\author\relax
  \global\let\date\relax
}

%\def\@maketitle{%
%  \newpage
%  \null
%  \vskip 2em%
%  \begin{center}%
%  \let \footnote \thanks
%    {\LARGE \@title \par}%
%    \vskip 1.5em%
%    {\large
%      \lineskip .5em%
%      \begin{tabular}[t]{c}%
%        \@author
%      \end{tabular}\par}%
%    \vskip 1em%
%    {\large \@date}%
%  \end{center}%
%  \par
%  \vskip 1.5em}
%\fi

\makeatother

%% ambientes de teoremas usado no exemplo
%% Definindo teormas e similares. Contador unico, vinculado a capítulos.
%\newtheorem{theorem}{Teorema}[chapter] % contador vinculado a capitulos
%\newtheorem{corollary}[theorem]{Corolário}
%\newtheorem{lemma}[theorem]{Lema}
%\newtheorem{proposition}[theorem]{Proposição}
%\newtheorem{axiom}[theorem]{Axioma}
%\theoremstyle{definition}
%\newtheorem{definition}[theorem]{Definição}
%\newtheorem{example}[theorem]{Exemplo}
%\newtheorem{exercise}{Exercício}[chapter] % contador próprio, vinculado a capitulo
%\theoremstyle{remark}
%\newtheorem{remark}[theorem]{Observação}

%%%%%%%%%%%%%%%%%%%%%%%%%%%%%%%%%%%%%%%%

%% return to ams like mathcal instead of unicode-math ones
%\let\mathcal\relax % remove the definition by unicode-math
%\DeclareMathAlphabet{\mathcal}{OMS}{cmsy}{m}{n}
%% return to ams like mathbb instead of unicode-math ones
%\let\mathbb\relax % remove the definition by unicode-math
%\DeclareMathAlphabet{\mathbb}{U}{msb}{m}{n}

\let\orimathcal\mathcal
\let\orimathbb\mathbb
\DeclareMathAlphabet{\amsmathcal}{OMS}{cmsy}{m}{n}
\DeclareMathAlphabet{\amsmathbb}{U}{msb}{m}{n}

%\global\let\plainlabel\label
%\global\let\plainref\ref

% Fake bold in way to \mathversion{bold} works
% on XeLaTeX/LuaLaTeX with unicode-math.
% It loose several math symbols.
% \setmathfont[version=bold,FakeBold=3.5]{Latin Modern Math}
% De
% https://tex.stackexchange.com/questions/64839/how-to-change-listing-caption

% save some values for lstlinting
\newcounter{lstorichapter}
\newcounter{lstorisection}
%\let\orilabel\label
%\let\oriref\ref
\newcommand{\emptyaddcontentsline}[3]{}
\let\oriaddcontentsline\addcontentsline

\newcounter{orilstlisting}

%\let\oriref\ref % without link
%\makeatletter
%\let\refstar\@refstar % without link
%\makeatother

% Example of untf8 verbatim write
%\newenvironment{writefile}[1]
% {\deactivateeightbit\VerbatimOut{#1.out}}
% {\endVerbatimOut}
%\makeatletter
%\newcommand{\deactivateeightbit}{%
%  \count@=127
%  \loop
%    \catcode\count@=12
%    \ifnum\count@<255
%    \advance\count@\@ne
%  \repeat
%}
%\makeatother

% from https://tex.stackexchange.com/questions/192941/showexpl-not-showing-caption-of-a-lstlisting-within to create vertical space
%\def\xstrut{\protect\rule[-2ex]{0pt}{2ex}}
%----------------------------------------------------------------

% showexpl's LTXexample preset
%\newcommand*\SX@@preset{%
%\renewcommand\documentclass[2][]{\SX@eat@version}%
%\renewcommand\usepackage[2][]{\SX@eat@version}%
%\renewenvironment{document}{}{}%
%\renewenvironment{figure}[1][]{\def\@captype{figure}}{}%
%\renewenvironment{table}[1][]{\def\@captype{table}}{}%
%\renewcommand\cite[1][]{}%
%\let\tableofcontens\relax \let\listoffigures\relax
%\let\listoftables\relax \let\printindex\relax
%\let\listfiles\relax \let\nofiles\relax
%\let\index\@gobble \let\label\@gobble
%\let\bibliography\@gobble
%\let\pagestyle\@gobble \let\thispagestyle\@gobble
%%%\let\immediate\relax \let\write\@gobbletwo
%%%\let\closeout\@gobble \let\@@input\@gobble
%\renewcommand\marginpar[2][]{}%
%\renewcommand\footnote[2][]{}%
%\let\@footnotetext\@gobble
%%%\abovedisplayskip=\z@
%%%\abovedisplayshortskip=\z@
%}
\makeatletter
\providecommand\SX@eat@version[1][]{}

\def\ltxlistingpresetfromshowexpl{%
\renewcommand\documentclass[2][]{\SX@eat@version}%
\renewcommand\usepackage[2][]{\SX@eat@version}%
\renewenvironment{document}{}{}%
\renewenvironment{figure}[1][]{\def\@captype{figure}}{}%
\renewenvironment{table}[1][]{\def\@captype{table}}{}%
\renewcommand\cite[1][]{}%
\let\tableofcontens\relax \let\listoffigures\relax
\let\listoftables\relax \let\printindex\relax
\let\listfiles\relax \let\nofiles\relax
\let\index\@gobble 
% \let\label\@gobble
\let\bibliography\@gobble
\let\pagestyle\@gobble \let\thispagestyle\@gobble
%%\let\immediate\relax \let\write\@gobbletwo
%%\let\closeout\@gobble \let\@@input\@gobble
\renewcommand\marginpar[2][]{}%
%\renewcommand\footnote[2][]{}%
%\let\@footnotetext\@gobble
%%\abovedisplayskip=\z@
%%\abovedisplayshortskip=\z@
} % showexpl default preset

\makeatother

% ltxlisting preset
\newcommand{\ltxlistingpreset}{
 \ltxlistingpresetfromshowexpl
 %
 \setcounter{lstorichapter}{\value{chapter}}%
 \setcounter{chapter}{0}%
 \setcounter{lstorisection}{\value{section}}%
 \setcounter{section}{0}%
 \let\oriaddcontentsline\addcontentsline%
 \let\addcontentsline\emptyaddcontentsline%
 \let\orimathcal\mathcal%
 \let\orimathbb\mathbb%
 \let\mathcal\amsmathcal%
 \let\mathbb\amsmathbb%
 \let\orimaketitle\maketitle
 \let\maketitle\makearticletitle
 \let\title\orititle      
 \let\author\oriauthor
 \let\date\oridate
 \let\orithempfootnote\thempfootnote
 \renewcommand{\thempfootnote}{\arabic{mpfootnote}}
 \setcounter{orilstlisting}{\value{lstlisting}}
}

\newcommand{\ltxlistingposset}{
 \setcounter{chapter}{\value{lstorichapter}}%
 \setcounter{section}{\value{lstorisection}}%
 \let\addcontentsline\oriaddcontentsline%
 \let\mathcal\orimathcal%
 \let\mathbb\orimathbb%
 \let\maketitle\orimaketitle
 \let\thempfootnote\orithempfootnote
 \setcounter{lstlisting}{\value{orilstlisting}}
}

\makeatletter
% from
% https://tex.stackexchange.com/questions/9035/how-to-pass-an-optional-argument-to-an-environment-with-verbatim-content
%
% \expandafter trick from
% https://tex.stackexchange.com/questions/196655/how-to-pass-figure-position-to-keyval-macro
\newenvironment{ltxlisting}{%
    \catcode`\^^M=\active
    \@ifnextchar[% se tem argumento opcional
        {\catcode`\^^M=5 \lxt@listing@start}
        {\catcode`\^^M=5 \lxt@listing@start[]}
}{\ltx@listing@end}

% Use of keyval from
% http://www.tex.ac.uk/FAQ-keyval.html
%
% Trick to ignore unkonown keys from
% https://tex.stackexchange.com/questions/314944/ignore-undefined-key-in-argument-its-defined-in-the-macro-before-its-used

\define@key{ltxlisting}{preset}{#1} % executte preset
% como eh dificil expandir segundo argumento antes do comando, define um comando auxiliar. Depois usar \expanafter para expandir argumento antes
\newcommand{\ltx@listing@setkeys}[1]{\setkeys{ltxlisting}{#1}}

%\def\lxt@listing@start[#1]{%
%  \begingroup
%    \ltxlistingpreset
%    % own preset
%    \let\KV@errx@ORI\KV@errx   % Save original error handling
%    \let\KV@errx\@gobble       % Ignore unknown keys
%      \setkeys{ltxlisting}{#1}
%    \let\KV@errx\KV@errx@ORI   % Restore original error handling      
%    %    
%    \def\ltx@listing@args{language={[LaTeX]TeX},#1}
%    \VerbatimEnvironment
%    % \deactivateeightbit
%    \begin{VerbatimOut}{\jobname.tmp}%
%}
%
%\def\ltx@listing@end{%
% \end{VerbatimOut}
% \begin{singlespacing}%
% \expandafter\lstinputlisting\expandafter[\ltx@listing@args]{\jobname.tmp}
% %\noindent
% \begin{framed} % frmae the output
% \noindent
% \begin{minipage}{\linewidth} % box the output
% \input{\jobname.tmp}
% \end{minipage}
% \end{framed}
% \end{singlespacing}%
% \ltxlistingposset
%\endgroup
%}

\def\lxt@listing@start[#1]{%
  %\begingroup
    \def\ltx@listing@args{language={[LaTeX]TeX},#1}
    \VerbatimEnvironment
    % \deactivateeightbit
    \begin{VerbatimOut}{\jobname.tmp}%
}

\def\ltx@listing@end{%
 \end{VerbatimOut}    %    
 \begin{singlespacing}
 \expandafter\lstinputlisting\expandafter[\ltx@listing@args]{\jobname.tmp}
 \end{singlespacing}
% \begingroup
% \begin{singlespacing}%
%   \ltxlistingpreset
%    % own preset
%   \let\KV@errx@ORI\KV@errx   % Save original error handling
%   \let\KV@errx\@gobble       % Ignore unknown keys
%     \expandafter\ltx@listing@setkeys\expandafter{\ltx@listing@args}
%   \let\KV@errx\KV@errx@ORI   % Restore original error handling      
% %\noindent
% \begin{framed} % frmae the output
% \noindent
% \begin{minipage}{\linewidth} % box the output
% \input{\jobname.tmp}
% \end{minipage}
% \end{framed}
% \end{singlespacing}%
% \ltxlistingposset
%\endgroup
%
\begin{ltxlistingout}[\ltx@listing@args]
\input{\jobname.tmp}
\end{ltxlistingout}
}

\newenvironment{ltxlistingout}[1][]
{ \begingroup
  \begin{singlespacing}%
  \ltxlistingpreset
  % own preset
  \let\KV@errx@ORI\KV@errx   % Save original error handling
  \let\KV@errx\@gobble       % Ignore unknown keys
    \expandafter\ltx@listing@setkeys\expandafter{#1}  
%    \expandafter\ltx@listing@setkeys\expandafter{\ltx@listing@args}
  \let\KV@errx\KV@errx@ORI   % Restore original error handling      
  %\noindent
  \begin{framed} % frame the output
  \noindent
  \begin{minipage}{\linewidth} % box the output
}{\end{minipage}
  \end{framed}
  \end{singlespacing}%
  \ltxlistingposset
  \endgroup
}

%%%%%%%%%%%%%%%%%%%%%%%%%%%%%%%%%%%%%%%%

%%%%%%%% ajuste de cabecalhos %%%%%%%%%%
% redefine a saida do capitulo e da secao
% \pagestyle{fancy} redefine eles,
%  entao deve colocar no final do ps@fancy 
% para ter efeito. Usando \addto do babel
\makeatletter
  \addto\ps@fancy{%
    \renewcommand{\chaptermark}[1]{%
      \markboth{\if@mainmatter\thechapter. \fi #1}%
        {\if@mainmatter\thechapter. \fi #1}}
    \renewcommand{\sectionmark}[1]{%
      \markright{\if@mainmatter\thesection. \fi #1}}
  }
\makeatother

% redefinindo o estilo da pagina "fancy"
\fancyhf{} % limpa o cabecalho
% redefine o cabcalho. \nouppercase foi usado para eliminar conversao para maiusculo do sumario/bibliografia
\fancyhead[LE,RO]{\bfseries\thepage}
\fancyhead[LO]{\bfseries\nouppercase\rightmark}
\fancyhead[RE]{\bfseries\nouppercase\leftmark}

% redefine o "plain" (usado na primeira pagina do capitulo).
\fancypagestyle{plain}{%
   \fancyhead{} % get rid of headers
   \fancyhead[LE,RO]{\bfseries\thepage} % colocar enumeracao
   \renewcommand{\headrulewidth}{0.5pt} % com linha horizontal
}

% linha horizontal entre cabecalho e corpo do documento
\renewcommand{\headrulewidth}{0.5pt}

% sem linha horizontal entre corpo de texo e rodape
\renewcommand{\footrulewidth}{0pt}

%%%%%%%% fim da redefinicao do cabecalho %%%%%%%%%%%%

% Providenciando logos
%%%%%%%%%%%%%%%%%%%%%%
\makeatletter

% From
% https://wiki.lyx.org/LyX/Logotype
%\providecommand{\LyX}{L\kern-.1667em\lower.25em\hbox{Y}\kern-.125emX\@}

\ifx\LyX\undefined
\def\@@LyX{L\kern-.1667em\lower.25em\hbox{Y}\kern-.125emX\@}%
%
\@ifpackageloaded{hyperref}
{\DeclareRobustCommand{\LyX}{%
  \texorpdfstring{\@@LyX}{LyX}}}%
{\DeclareRobustCommand{\LyX}{\@@LyX}}
\fi

% From
% https://tex.stackexchange.com/questions/18089/are-there-any-command-for-producing-the-bibtex-logo
%\ifx\BibTeX\undefined
%  \def\BibTeX{\rm B\kern-.05em{\sc i\kern-.025em b}\kern-.08em%
%     T\kern-.1667em\lower.7ex\hbox{E}\kern-.125emX}%
%\fi

\ifx\BibTeX\undefined
  \def\@@BibTeX{B\kern-.05em{\sc i\kern-.025em b}\kern-.08em\TeX}%
%
\@ifpackageloaded{hyperref}
{
\DeclareRobustCommand{\BibTeX}{%
  \texorpdfstring{\@@BibTeX}{BibTeX}}}%
{\DeclareRobustCommand{\BibTeX}{\@@BibTeX}}    
\fi

% rotatebox trick from hologo (note that reflexbox fail inside section title)
%\ifx\XeTeX\undefined
%\@ifpackageloaded{hyperref}
%{
%\DeclareRobustCommand{\XeTeX}{%
%  \texorpdfstring{X\kern-.1667em\lower.7ex\hbox{\rotateboxbox[origin=c]{180}{E}}\kern-.125em\TeX}{XeTeX}}}%
%{\DeclareRobustCommand{\XeTeX}{%
%  X\kern-.1667em\lower.7ex\hbox{\rotateboxbox[origin=c]{180}{E}}\kern-.125em\TeX}}
%\fi

\ifx\XeLaTeX\undefined
\def\@@XeLaTeX{X\kern-.1667em\lower.7ex\hbox{\rotatebox[origin=c]{180}{E}}\kern-.125em\LaTeX}
%  
\@ifpackageloaded{hyperref}
{
\DeclareRobustCommand{\XeLaTeX}{%
  \texorpdfstring{\@@XeLaTeX}{XeLaTeX}}}%
{\DeclareRobustCommand{\XeLaTeX}{\@@XeLaTeX}}
\fi

% \LaTeX and \LaTeXe from latex.ltx
%\DeclareRobustCommand{\LaTeX}{L\kern-.36em%
%        {\sbox\z@ T%
%         \vbox to\ht\z@{\hbox{\check@mathfonts
%                              \fontsize\sf@size\z@
%                              \math@fontsfalse\selectfont
%                              A}%
%                        \vss}%
%        }%
%        \kern-.15em%
%        \TeX}
%\DeclareRobustCommand{\LaTeXe}{\mbox{\m@th
%  \if b\expandafter\@car\f@series\@nil\boldmath\fi
%  \LaTeX\kern.15em2$_{\textstyle\varepsilon}$}}


% fim do logo
\makeatother

% do amssymb, mas nao tem no unicode-math.
% usado no exemplo de diagrama comutativa
% " eh mapeado no babel. use ^^^^
% \providecommand{\circlearrowleft}{\symbol{^^^^27f2}}

% \providecommand{\circlearrowleft}{\symbol{10226}}

%\providecommand{\circlearrowleft}{\char{^^^^27f2}}
% Nao tem o comando \UnicodeMathSymbol?
%\makeatletter
% \let\UnicodeMathSymbol\um@mathsymbol@noparse
% \UnicodeMathSymbol{^^^^027F2}{\circlearrowleft}{\mathrel}{left circle arrow}%
% \DeclareUnicodeCharacter{27F2}{\circlearrowleft} % do inputenc (utf8)
% \def\circlearrowleft{\mathrel{\char^^^^27f2}}
%\makeatother

\providecommand{\circlearrowleft}{\acwopencirclearrow} % from unicode-math

\providecommand{\circlearrowleft}{\acwgapcirclearrow} % from unicode-math
% unicode version of \acwgapcirclearrow not appear on version 2017 of latin Modern (Cambria Math of Microsoft has this symbol).
% Need to tested on newer version of Latin Modern

% Defining from from amssymb
%\ifx\circlearrowleft\undefined
%  % From AMS (amssymb)
%  \DeclareSymbolFont{AMSa}{U}{msa}{m}{n} % from amsfonts
%  \DeclareMathSymbol{\circlearrowright}{\mathrel}{AMSa}{"08}
%  \DeclareMathSymbol{\circlearrowleft}{\mathrel}{AMSa}{"09}
%\fi

% \square do amssymb não tem no unicode-math
% providenciando como \openbox do amsthm

%\ifx\square\undefined
%  \let\square\openbox
%\fi
\providecommand{\square}{\openbox}

% lmusrrache/rmustache não tem no Latin Modern Math
% unicode math redfine no comeco do documento.
% Portanto, deve redefinir no comeco do documento (não no preambulo)
% usando \AtBeginDocument
\DeclareSymbolFont{largesymbolsoriginal}{OMX}{cmex}{m}{n}  
\AtBeginDocument{
%  \let\lmoustache\lmoustacheoriginal 
%  \let\rmoustache\rmoustacheoriginal
\DeclareMathDelimiter{\lmoustache}{\mathopen}{largesymbolsoriginal}{"7A}{largesymbolsoriginal}{"40}
\DeclareMathDelimiter{\rmoustache}{\mathclose}{largesymbolsoriginal}{"7B}{largesymbolsoriginal}{"41}
}

%%%%%%%%%%%%%%%%%%%%%%%%%%%%%%%%%%%%%%%%%%%%%%%%%%%%%%%%%%%%%%%%%%%%
% providenciando simbolos reservados para index, no indice remissivo
%
%\newcommand{\abssymbforindex}{|} % use \vert
%\index{delimitador!modulo@\abssymbforindex}
%\newcommand{\normsymbforindex}{\textbackslash |} % use \Vert
%\index{delimitador!norma@\normsymbforindex}

%\newcommand{\parenopensymbforindex}{$($}
%\index{delimitador!parenteaberta@$($}
%\newcommand{\parenclosesymbforindex}{$)$}
%\index{delimitador!parentefechada@$)$}
%\newcommand{\braceopensymbforindex}{$[$}
%\index{delimitador!colcheteaberta@[}
%\index{delimitador!colchetefechada@]}
%\newcommand{\curlyopensymbforindex}{\textbackslash\{}
\newcommand{\curlyopensymbforindex}{\{}
%\index{delimitador!chaveaberta@\curlyopensymbforindex}
%\newcommand{\curlyclosesymbforindex}{\textbackslash\}}
\newcommand{\curlyclosesymbforindex}{\}}
%\index{delimitador!chavefechada@\curlyclosesymbforindex}
\newcommand{\arrobasymbforindex}{$@$}
\newcommand{\exclsymbforindex}{$!$}

%%%%%%%%%%%%%%%%%%%%%
% provide some files
%-------------------
% \usepackage{filecontents} % nao precisa do pacote, se for usado no preamble e nao precisar reescrever quando arquivo ja existe
\begin{filecontents}{latex-via-exemplos-tabela.txt}
% Parabola
% % ou # marca a linha de comentario
-2	4
-1.5	2.25
-1.0	1
-0.5	0.25
0	0
0.5	0.25
1	1
1.5	2.25
2	4
\end{filecontents}

%\begin{filecontents}{latex-via-exemplos-lista-nomes.csv}
%"Nome","Trabalho","Observacao"
%"Nome 1","Trabalho 1",
%"Nome 2","Trabalho 2",
%\end{filecontents}

% Informações
%%%%%%%%%%%%%
\title{\LaTeXe{} Via Exemplos}
\author{Sadao Massago\thanks{DFQM-UFSCar -- Campus de Sorocaba, SP
(\url{http://dfqm.sorocaba.ufscar.br/})} }
% Versão preliminar: janeiro a abril de 2018
% Primeira versão publica (0.5): abril de 2018.
% Primiera atualizacao (0.5.1): junho de 2018.
% Segunda atualização, adicionando o ABNT (0.5.2): junho de 2018.
% Terceira atualização (0.5.3): junho de 2018.
% Quarta atualização (0.5.4): agosto de 2019
% Quinta atualização (0.5.6): dezembro de 2019 
\date{Versão 0.5.7, 17 de julho de 2021}

%%%%%%%%%%%%%%%%%%%%%%%
% Docuemt start here
\pagestyle{empty}
\begin{document}

\maketitle

\thispagestyle{empty}

Copyright \textcopyright 2018--2021 por Sadao Massago. 
Todos os direitos reservados.
Este documento é software livre; podendo ser redistribuído e/ou modificado de acordo com os termos da Licença Pública da GNU como publicado pela Free Software Foundation; versão 2 da Licença, ou (se for sua opção) qualquer versão posterior. 
Veja \url{https://opensource.org/licenses/gpl-license}.

\

A versão atual deste documento está disponível em \\
\url{https://ctan.org/pkg/latex-via-exemplos}

%%{CTAN://info/latex-via-exemplos}

\clearpage\thispagestyle{empty}\cleardoublepage
\frontmatter

% adiciona bookmark no PDF
\phantomsection
\pdfbookmark[chapter]{Prefácio}{prefacio}
%\pdfbookmark[0]{Prefácio}
\chapter*{Prefácio}
O \LaTeXe{} é um sistema de processamento de documentos implementados sobre o \TeX.

O objetivo deste documento é apresentar o \LaTeXe{} de forma gradativa, usando os exemplos.
Assim, o documento contém muitos exemplos e suas saídas, o que aumentou consideravelmente as páginas.

Este documento está dividido em 19 capítulos mais os apêndices.

Do Capítulo~1 até Capítulo~14 utilizam somente os pacotes da \texttt{base} e de \texttt{required} na qual qualquer sistema \LaTeX{} devem conter.
Do capítulo 15 a 18 utilizam os pacotes que não são da \texttt{base} e de \texttt{required} para incrementar a funcionalidade.
Nos apêndices, serão tratados alguns recursos extras que podem ser interessantes, mas que julgar que não sejam de interesse de todos.

Capítulo~1 é uma breve introdução sobre o sistema.

Capítulo~2 trata do primeiro documento em \LaTeXe, incluindo introdução à escrita de textos.

Capítulo~3 introduz sobre fórmulas matemáticas.

Capítulo~4 trata sobre estrutura de textos.

Capítulo~5 é uma continuação do Capítulo~3, tratando sobre fórmulas matemáticas.

Capítulo~6 trata sobre definições de comandos e ambientes.

Capítulo~7 discute sobre divisão lógica de documentos, conhecido como ``seccionamentos''.

Capítulo~8 trata de definições e uso dos ambientes do tipo teoremas.

Capítulo~9 discute sobre a criação de figuras e tabelas flutuantes, tabelas longas e inclusão de imagem externa.

Capítulo~10 trata sobre ajuste das fontes e espaçamentos.

Capítulo~11 trata sobre referências bibliográficas e índice remissivos. \BibTeX{} também será introduzido neste capítulo.

Capítulo~12 trata sobre medidas e contadores.

Capítulo~13 e 14 exploram várias possibilidades dentro dos pacotes de \texttt{base} e \texttt{required} que não foram discutidos nos capítulos anteriores.

Capítulo~15 discute o uso de pacotes fora do \texttt{base} e \texttt{required} para ajuste de documentos.
A partir deste capítulo, serão apresentados pacotes fora do \texttt{base} e \texttt{required}. 
Isto significa que os pacotes indicados podem precisar de instalação a parte.

Capítulo~16 discute o pacote \texttt{xcolor} especial para cores, e \texttt{tikz} espacial para criar ilustrações.

Capítulo~17 ilustra a criação de \texttt{poster} e \texttt{slides} de apresentação.

Capítulo~18 apresenta o uso de \XeLaTeX{} e Lua\LaTeX, considerado a próxima geração de \LaTeX.

Capítulo~19 apresenta a formatação no padrão ABNT.

Apêndice~A é sobre símbolos básicos do \LaTeX.

Apêndice~B é sobre como escrever um pacote (arquivo de estilos).

Apêndice~C trata de editor \LyX{} que permite elaborar documentos de forma mais visual como os editores para escritórios, em vez de editar diretamente o código fonte em \LaTeX.

Apêndice~D trata da mala direta e criação do caderno de resumos, interessantes para organizadores de eventos.

Apêndice~E trata de pacotes úteis para elaborar provas e lista de exercícios.

Apêndice~F trata de pacotes interessantes para desenvolver projetos tal como escrever um livro ou similar.

Apêndice~G descreve alguns aplicativos livres interessantes que auxilia os usuários de \LaTeX.
\

% Sorocaba, abril de 2018.
% Sorocaba, junho de 2018.
% Sorocaba, agosto de 2019.
% Sorocaba, dezembro de 2019.
Sorocaba, julho de 2021.


\

Sadao Massago \verb+<sadao@ufscar.br>+

%\pagestyle{headings}


\clearpage\thispagestyle{empty}\cleardoublepage

\pagestyle{fancy}

\phantomsection
\pdfbookmark[chapter]{\contentsname}{toc}
\tableofcontents

\clearpage\thispagestyle{empty}\cleardoublepage
\phantomsection
\pdfbookmark[chapter]{Lista de \lstlistingname s}{lstlisting}
\lstlistoflistings

\clearpage\thispagestyle{empty}\cleardoublepage
\mainmatter
\chapter{Introdução}

\LaTeX{} é um sistema de diagramação de documentos profissional largamente utilizados, desenvolvido sobre o \TeX{}.
Para quem entende inglês, os livros impressos recomendados são \cite{book:lamport:1994} e \cite{book:goossens:2004}. 
Outros textos recomendados que podem ser lidos gratuitamente são \cite{book:oeticker:2018}, \cite{misc:indiantug:2000} e \cite{book:wikibook:latex}.
Note que a tradução em português brasileiro do \cite{book:oeticker:2018} costuma estar desatualizado em relação à versão original.
Então prefira a versão em inglês.

\section{Uma breve história}
O \TeX{} \cite{book:knuth:1986} e a fonte padrão \texttt{Computer Modern}, foram desenvolvidos pelo \emph{Donald Kunuth}\index{Donald Kunuth} em 1977 a 1985.
Dos sistemas implementados sobre o \TeX, o mas usado é o \LaTeX{} \cite{book:lamport:1986} desenvolvido pela equipe de \emph{Leslie Lamport}\index{Leslie Lamport}, concluido em 1985.
A versão mais utilizada do \LaTeX{} foi concluído em 1994 que é o \LaTeXe{} \cite{book:lamport:1994}.
Em 2004, o \TeX{} começou a suportar oficialmente o \texttt{utf-8}, facilitando a elaboração de documentos multi-idiomas.
Em 2007, o Xe\TeX{} que suporta o uso de fontes do sistema implementado inicialmente no \texttt{Mac OSX} foi portado para \texttt{linux} e \texttt{Windows}.
Em 2008, o recurso de sincronização do documento fonte com o PDF tornou fácil e PDF foi substituindo a saída DVI do \TeX{} na diagramação.
Em 2010 foi lançado a primeira versão estável do Lua\TeX{}, considerado como o sucessor do PDF\TeX{}. 
O Lua\TeX{}, além de poder usar as fontes do sistema como o Xe\TeX, também pode estender a funcionalidade com a linguagem script \texttt{Lua}.

\section{Como usar \LaTeX{}}
O \TeX{}\index{TeX} é um sistema de compilação de documentos e o \LaTeX{}\index{LaTeX} é um conjunto de macros (instruções) para automatizar e facilitar a diagramação de documentos.

O documento é preparado como arquivo texto num editor de texto, compilado pelo \LaTeX{}, visualizado e corrigido e compilado novamente, até obter o resultado desejado.
O arquivo fonte do documento costuma ser editado no editor próprio para \LaTeX{} tais como \texttt{TeXMaker}\index{TeXMaker} (\url{http://www.xm1math.net/texmaker/}) e \texttt{TeXStudio}\index{TeXStudio} (\url{https://www.texstudio.org/}), ambos disponível livremente em várias plataformas. 
Além de recursos para facilitar a escrever documentos, eles contam com botões de compilação e visualização, assim como sincronismo de PDF com o código fonte.
Para quem quer a funcionalidade mais próxima dos aplicativos de escritórios que permite visualizar como vai ficar enquanto escreve e quer elaborar o documento usando botões e menus em vez de digitar comandos, poderá optar pelo \LyX{}\index{LyX} (\url{https://www.lyx.org/}) que também é livre e suporta várias plataformas.
Para evitar erros de escrita, é recomendável que ative o corretor ortográfico com idioma desejado, o que depende de cada editor.
Para compilar, deverá instalar alguma distribuição de \TeX{}. 
Em geral, o mais recomendado é o \texttt{TeXLive}\index{TeXLive} para \texttt{Linux}, \texttt{MacTeX}\index{MacTeX} para \texttt{Mac OSX} e \texttt{MikTeX}\index{MikTeX} para \texttt{Windows}, todos são livres.

Evite usar o espaço no nome de arquivos para \LaTeX, pois o sincronismo do código fonte com o PDF (posicionar PDF na posição correspondente à linha de código fonte e vice-versa) pode tornar parcial.

\clearpage\thispagestyle{empty}\cleardoublepage
\chapter{Iniciando um Documento}
Neste capítulo, vamos tratar do básico de como começar um documento \LaTeX.

\section{Primeiro documento}
Um documento em \LaTeX{} inicia-se com o comando \verb+\documentclass+\index{documentclass} que especifica qual tipo de documentos será diagramado.
Em seguida, será especificado os pacotes (conjunto de instruções) adicionais a serem carregados e também será realizada algumas configurações. Esta parte do documento é chamado de \texttt{preamble}\index{preamble} (preâmbulo) do documento.
Depois inicia o conteúdo do documento com \verb+\begin{document}+, escreve o corpo do documento e finaliza o documento com \verb+\end{document}+.

O código do primeiro documento e sua saída em pdf (que está dentro da moldura) seria como do Exemplo~\ref{ex02:inicial}.

% \begin{spacing}{1}
%\begin{singlespacing}
%\begin{LTXexample}[pos=b,caption={ex01-01.tex},label=ex:01-01]
\begin{ltxlisting}[caption={ex02-inicial.tex},label=ex02:inicial]
\documentclass[12pt,a4paper]{article}
\usepackage[T1]{fontenc}
\usepackage[utf8]{inputenc}
\usepackage[brazil]{babel}

\begin{document}
Este é primeiro parágrafo.
Continuando o primeiro parágrafo.

Este é o segundo parágrafo.

Espaço extra      são eliminados.
\end{document}
\end{ltxlisting}
%\end{LTXexample}
%\end{singlespacing}

O comando do \LaTeX{} inicia com o caractere especial ``\verb+\+''\index{\textbackslash@\verb+\+}. 
O primeiro comando \verb+\documentclass+\index{documentclass@\verb+\documentclass+} tem como parâmetros delimitados pelos colchetes e outro com chaves.
O que é delimitado pelo chaves é o parâmetro obrigatório\index{parâmetro!obrigatório} (aqui, é o \texttt{article}). 
Os parâmetros colocados entre colchetes, separados pela vírgula são os opcionais\index{parâmetro!opcional}. 
Neste exemplo, são \texttt{12pt} e \texttt{a4paper}. 
Opcionais significa que pode ou não colocar tais opções.
O parâmetro obrigatório do comando \verb+\documentclass+ é tipo de documentos. 
Aqui foi escolhido o \texttt{article}\index{article} (artigo) que tem como objetivo, colocar maior quantidade de informações no espaço limitado. 
Outro tipo de documento bastante usados é o \texttt{book}\texttt{book} (livro).
Os parâmetros opcionais usados foram \texttt{12pt}\index{12pt} que é o tamanho de letra\index{tamanho!letra} em 12pt e \texttt{a4paper}\index{a4paper} que é o tamanho do papel\index{tamanho!papel} em padrão \texttt{A4}.

Depois segue com sequência de comandos \verb+\usepackage+\index{usepackage@\verb+\usepackage+}.
O \verb+\usepackage+ carrega o pacote\index{pacote} (conjunto de instruções) que configura o documento ou disponibiliza os comandos específicos.
O \LaTeX{} dispõe uma grande quantidade de pacotes, uma para casa situação.
No Exemplo~\ref{ex02:inicial}, foram carregados os pacotes \texttt{fontenc}, \texttt{inputenc} e \texttt{babel}, todos eles com um parâmetro.
O primeiro pacote \texttt{fontenc}\index{fontenc} é usado para especificar a codificação das fontes de letras. 
A opção \texttt{T1} indica que a fonte está em \texttt{T1} que dispõe de letras de 8-bits (acentuadas). 
Esta opção é útil para definir a regra de hifenização local do documento no \texttt{preamble} e efetuar busca de texto com letras acentuadas no PDF final. 
A seguir, o pacote \texttt{inputenc}\index{inputenc} especifica a codificação do documento fonte (o que está editando). 
Atualmente, o \texttt{utf8} é o recomendado, mas antigamente usava o \texttt{latin1} para português brasileiro. 
Note que \texttt{inputenc} foi colocado depois da \texttt{fontenc}.
Assim, a faixa de coberta dos caracteres é maior, não limitando somente a faixa de \texttt{T1}.

O terceiro e último pacote deste exemplo é o \texttt{babel} com a opção \texttt{brazil}. 
O pacote \texttt{babel}\index{babel} seleciona a regra de hifenização e nomes dos elementos (como figura, capítulo, etc) para idioma especificada. 
A opção \texttt{brazil} ou \texttt{brazilian} escolhe o português brasileiro (não confundir com a opção \texttt{portugese} que escolherá português de Portugal).

Depois encontra o \verb+\begin{document}+.
O comando \verb+\begin+\index{begin@\verb+\begin+} inicia um ambiente. 
Um ambiente\index{ambiente} é uma configuração que será aplicado nos trechos entre \verb+\begin{<ambiente>}+ e \verb+\end{<ambiente>}+\index{end@\verb+\end+}.
O \verb+\begin{document}+ e \verb+\end{document}+
determina o ambiente de documento na qual seus conteúdos serão colocados no arquivo PDF.

A quebra de linha não efetua quebra de linha na saída. 
Para que tenha um novo parágrafo\index{parágrafo} de fato, deverá pular uma linha.

Note que, se tiver mais de um espaço, o \LaTeX{} interpretará como um único espaço.

Para gerar o arquivo PDF, deverá processar o \LaTeX.
Salve o arquivo com extensão \texttt{tex} e clique no botão de \texttt{Compilar} no caso de \texttt{TeXMaker} e botão \texttt{LaTeX} no caso de \texttt{TexStudio} (na versão mais nova, é um triângulo verde). 

\section{Mensagem de erro e correção}

Quando ocorre erro de compilação, o \texttt{TeXMaker} mostrará em vermelho na parte de baixo e indicará a linha onde ocorreu o erro. 
Maioria dos editores para \LaTeX{} posicionará automaticamente na linha do primeiro erro quando compila. 
Para demais erros, ao clicar nas mensagens de erros em vermelho, posicionará automaticamente na linha correspondente a tais erros.
Para melhorar a precisão da localização de erros, quebre o parágrafo em várias linhas (lembre-se que quebra de linha não afeta o documento final).

Assim, deve corrigir os erros e compilar de novo, até sumir com todos os erros.

Quando compila o documento, o \texttt{TeXMaker} posicionará na página do PDF, correspondente aonde fica o cursor no editor de texto. Para localizar qual código gerou uma determinada parte de PDF, clique no botão direito sobre parte do PDF e escolha ``clique para ir para a linha''.

\section{Caracteres especiais}
Existem vários caracteres especiais reservados para os comandos e similares do \LaTeX. Por exemplo, ``\verb+\+'' é usado para iniciar um comando, chaves é usado para indicar os parâmetros, etc. 
Para inserir estes caracteres especiais\index{caracteres especiais} no documento, deverá usar os comandos especiais de \LaTeX.

%A seguir, alguns deles. 
No código fonte do Exemplo~\ref{ex02:caracteres}, foi colocado somente os trechos que ficam no corpo do documento.
Para que o arquivo compile, deverá colocar entre \verb+\begin{document}+ e \verb+\end{document}+ do arquivo válido como do Exemplo~\ref{ex02:inicial}. 

%\begin{singlespacing}
% \begin{LTXexample}[pos=b,caption={ex01-02.tex},label=ex:01-02]
\begin{ltxlisting}[caption={ex02-caracteres.tex},label=ex02:caracteres]
% esta é comentário
Alguns caracteres especiais:
\textbackslash, \$, \#, \%, \&, \_, \{, \}

Alguns acentos no modo TeX:
\'a, \`a, \^e, \'\i, \~a, \"u

``Abrindo e fechando aspas''

Logo do \LaTeX{} e do \TeX
\end{ltxlisting}
%\end{LTXexample}
%\end{singlespacing}

O Exemplo~\ref{ex02:caracteres}
% é uma parte do arquivo de exemplo \texttt{ex01-02.tex} que 
mostra alguns comandos para produzir caracteres especiais.

Note que as letras com a acentuação direta\index{acentuacao@{acentuação}!direta} pelo teclado é suportado, desde que \verb+\usepackage[utf8]{inputenc}+ ou equivalentes sejam colocados no \texttt{preamble} do documento. 
No exemplo acima, foi mostrado a acentuação do modo \TeX{}\index{acentuacao@{acentuação}!modo \TeX} que pode ser usados em alguns casos especiais, tais como dentro do arquivo de referências bibliográficas, ou arquivos de pacotes.
Note que \verb+\'+ acentua a letra seguinte, mas ``i'' tem pingo e ``í'' (com acento) não tem pingo. \verb+\i+ é o comando para produzir ``i'' sem pingo\index{i sem pingo}.
No derivado de e-TeX como o TeX autual, costuma automatizar a retirada de pingo quando acentua, eliminando a necessidade do uso de i sem pingo.

Note também que o acento agudo no modo \TeX{} é produzido por apóstrofos e não pelo acento agudo.
Apóstrofos\index{apóstrofos} é aberto por um acento agudo e fechado pelo apóstrofos.
Aspas\index{aspas} é aberto pelos dois acentos agudos e fechado pelos dois apóstrofos (fiquem atentos de que fechamento de aspas é dois apóstrofos e não é aspas).

No \LaTeX, quando encontra ``\%''\index{comentario@{comentário}}\index{\%}, o restante desta linha será considerado como comentário\index{comentário} e é ignorado completamente.

Comentário é importante para inserir observações sobre o código, ou desativar um trecho do código.
Em geral, quem quer remover um trecho do código de \LaTeX{} no documento, simplesmente comenta o trecho, pois se algum dia quiser ativar, é só remover o ``\%'' do comentário.

Para comentar/descomentar um trecho maior no \texttt{TeXMaker}, selecione o trecho e use o \texttt{Editar->Comentar} e \texttt{Editar->Descomentar}.

Quando editar um documento, existem palavras que devem aparecer grudados (não pode ficar primeira parte no final de linha e outra no começo da linha) como no caso de enumeração de páginas, exemplos, teoremas, etc. 
Neste caso, usa-se o til (\verb+~+\index{\verb+~+}) em vez do espaço\index{espaço!não quebrável}. 
Por exemplo, no caso de ``\verb+página~1+'', não acontece de ``\texttt{página}'' ficar no final de uma linha e ``\verb+1+'' ficar no começo da próxima linha. Outros casos é colocar entre artigos e pronomes, como em ``O~Teorema de Potágoras'' para evitar que artigo fique no final de uma linha e restante na próxima linha.

Por último, quando insere comandos e precisa ter espaço depois dele, coloque um par de chaves\index{espaço!depois do comando}. Por exemplo, ``\verb+\LaTeX produz+'' ficaria como ``\LaTeX produz'' (grudados) enquanto que ``\verb+\LaTeX{} produz+'' ficará como ``\LaTeX{} produz'' (com espaço correto). Também lembre-se que no comando de \LaTeX, maiúsculo e minusculo são distinguidas.
Portanto, ``\verb+Latex+'' ou ``\verb+\latex+'' resultarão em erros em vez de produzir logotipo de \LaTeX.

\clearpage\thispagestyle{empty}\cleardoublepage
\chapter{Introdução às Fórmulas Matemáticas}
Neste capítulo, veremos um pouco sobre as fórmulas matemáticas.

\section{Fórmula \texttt{textstyle} e \texttt{displaystyle}}

Uma fórmula matemática (ou modo matemático) \texttt{textstyle}\index{textstyle} (ou \texttt{inlinestyle}\index{inlinestyle}) é uma fórmula no meio do texto. 
Por exemplo, denotaremos uma sequência por $\{x_n\}$ e seu limite por $\lim_n\to\infty x_n$ é modo \texttt{textstyle}, pois fórmulas estão no meio do texto.
Agora as fórmulas matemáticas que ocupam linha separada de texto como em
\[
x=\frac{-b\pm\sqrt{b^2-4ac}}{2a}
\]
é denominado de fórmula matemática no modo \texttt{displaystyle}\index{displaystyle}.

A fórmula no modo \texttt{textstyle} devem ficar delimitados entre ``\$''\index{\$} ou ``\textbackslash(''\index{\textbackslash parenteaberta@{\textbackslash$($}} e  ``\textbackslash)''\index{\textbackslash parentefechada@{\textbackslash$)$}}, ou ainda poderá usar o ambiente \texttt{math}\index{math}.
O mais usado é delimitar com ``\$''.

A fórmula no modo \texttt{displaystyle} devem ficar delimitados entre ``\$\$''\index{\$\$} (dois dólares) ou ``\textbackslash[''\index{\textbackslash[} e  ``\textbackslash]''\index{\textbackslash]}, ou ainda poderá usar o ambiente \texttt{displaymath}\index{displaymath}. 
E recomendado que delimite com \textbackslash['' e  ``\textbackslash]'',
pois delimitar com ``\$\$'' (dois dólares) dificultará a depuração (achar erros).

%\begin{singlespacing}
%\begin{LTXexample}[pos=b,caption={ex02-01.tex},label=ex:02-01]
\begin{ltxlisting}[caption={ex03-formulas.tex},label=ex03:formulas]
Solução da equação $ax^2+bx+x=0$ é dado pela fórmula
\[
x=\frac{-b\pm\sqrt{b^2-4ac}}{2a}
\]

Elemento da matriz $A$ costuma ser denotado por $a_{ij}$
\end{ltxlisting}
%\end{LTXexample}
%\end{singlespacing}


No modo matemático, ``\verb+^+''\index{\^{}@\verb+^+} é usado para indicar a potência\index{modo matematico@{modo matemático}!potencia@{potência}} e ``\_''\index{\_@\verb+_+} indica o índice\index{modo matematico@{modo matemático}!indice@{índice}}.
Outros comandos utilizados foram, \verb+\frac+\index{frac@\verb+\frac+} que produz frações\index{modo matematico@{modo matemático}!fracao@{fração}}, onde primeiro parâmetro é numerador e segundo, o denominador. 
O comando \verb+\sqrt+\index{sqrt@\verb+\sqrt+} produz raiz\index{raiz} do argumento. 
O comando \verb+\pm+\index{modo matematico@{modo matemático}!operador!\verb+\pm+, $\pm$} é o símbolo $\pm$. 
Na matemática, muitos símbolos são usados e o \LaTeX{} dispõe de comandos para cada uma desses símbolos. 
Em geral, o editor para \LaTeX{} dispõe de painel de inserção dos símbolos matemáticos para ajudar na elaboração do documento. 
No caso de \texttt{TeXMaker}, tem um botão no painel lateral esquerdo que permitem ativar tais painéis, organizados em grupos.
As letras gregas\index{letra!grega} no \LaTeX{} são produzidas pelo comando com nome em inglês. 
Se nome começar em maiúsculo, será letra grega maiúscula.

Note que, quando os parâmetros de um comando for mais de um (mais de uma letra, por exemplo), deverá colocar entre chaves, como foram feitos para \verb+\frac+ e \verb+sqrt+.

Isto vale também para expoentes e índices, como foi feito no elemento da matriz (com índices de duas letras).

O comando \verb+\sqrt+\index{sqrt@\verb+\sqrt+} aceita o parâmetro opcional (delimitado pelos colchetes) para poder produzir raiz $n$-ésima. 
O \verb+$\sqrt[n]{x}$+ produz $\sqrt[n]{x}$. 
O Exemplo~\ref{ex03:simbolos} ilustra algumas letras gregas e símbolos. 
Note que uma tabela completa de símbolos do \LaTeX{} \cite{book:pakin:2017} está disponível gratuitamente.

%\begin{singlespacing}
% \begin{LTXexample}[pos=b,caption={ex02-01.tex, parte b},label=ex:02-01:b]
\begin{ltxlisting}[caption={ex03-simbolos.tex},label=ex03:simbolos]
Algumas letras gregas minúsculas: $\alpha, \beta, \gamma, \lambda, \pi$.

Algumas letras gregas Maiúsculas: $\Gamma, \Delta, \Lambda, \Pi, \Omega$.

Alguns símbolos: 
$\le, \ge, \neq, \in, \notin, \exists, \nexists, \to, \infty, \forall, \therefore$.
\end{ltxlisting}
%\end{LTXexample}
%\end{singlespacing}

\section{Modo \texttt{displaystyle} no meio do texto}
As fórmulas no meio do texto (\texttt{textstyle}) são produzidos de forma que economize a sua altura. 
Por isso, elementos que normalmente ficariam empilhados, ficariam como índice e expoente. 
Por exemplo, $\displaystyle \lim_{n\to\infty}x_n$ no modo \texttt{textstyle}  ficaria como $\lim_{n\to\infty}x_n$. 
Para que uma fórmula fique como \texttt{displaystyle} no meio do texto\index{modo matematico@{modo matemático}!displaystyle!meio do texto}, coloque \verb+\displaystyle+ no começo das fórmulas. O Exemplo~\ref{ex03:displaystyle} ilustra o caso.

%\begin{singlespacing}
\begin{ltxlisting}[pos=b,caption={ex03-displaystyle.tex},label=ex03:displaystyle]
Temos que $\lim_{n\to\infty}x_n=0$ %textstyle
e  $\displaystyle \lim_{n\to\infty}x_n=0$ %displaystyle

Outros casos $\sum_{i=0}^n i = 0+1+\cdots+n$
e $\displaystyle \sum_{i=0}^n i = 0+1+\cdots+n$.

Forçando a colocar encima/embaixo
\[ \int\limits_I f = \int^b_af(x)dx \]
Forçar a não colocar encima/embaixo
\[ \sum\nolimits_{k=0}^n x_k = 1 \]
\end{ltxlisting}
%\end{LTXexample}
%\end{singlespacing}

Para forçar a colocar elementos embaixo em vez de como índice quando usa o ``\verb+_+'', usa-se o comando \verb+\limits+, como no Exemplo~\ref{ex03:displaystyle}.
% Observação: No LuaLaTeX 0.8 (2018-02-17), \limits acima não funcionou bem. O valor de baixo fica um pouco mais para direita.
% Idem para \mathop{}. No XeLaTeX, fica igual a do PDFLaTeX.
% Na versao 2018, ja estava resolvido tal problema

Note que, para colocar no modo \texttt{textstyle} nas fórmulas \texttt{displaystyle}, existe o comando \verb+\textstyle+\index{modo matematico@{modo matemático}!textstyle} que é usado de forma similar a \verb+\displaystyle+. 
Nesta família de comandos, também existe o comando \verb+\scriptstyle+\index{modo matematico@{modo matemático}!scriptstyle} que tenta  reduzir o tamanho das fórmulas.

\section{Equação enumerada e referências cruzadas}

Uma equação enumerada é produzido pelo ambiente \texttt{equation}\index{equation}. 
Quando algum comando do \LaTeX{} enumera automaticamente (equações, seções, capítulos, figuras, tabelas, itens de istas enumeradas, teoremas, etc), sua enumeração pode ser guardada e usada no outro lugar, recurso conhecido como referência cruzada\index{referência cruzada}. 
Para tanto, coloca-se o comando \verb+\label{nome do rótulo}+\index{label@\verb+\label+} que produz um rótulo\index{rótulo} que armazena a enumeração e \verb+\ref{nome do rótulo}+\index{ref@\verb+\ref+} aonde quer colocar esta enumeração.
O Exeplo~\ref{ex03:ref} ilustra o uso da referência cruzada.

%\begin{singlespacing}
%% truque do https://www.latex4technics.com/?note=30gq
%% para referencia funcionar no LTXexample
%\let\orilabel\label
%\setcounter{lstorichapter}{\value{chapter}}
%\begin{LTXexample}[pos=b,caption={ex02-02.tex},label=ex:02-02,preset={\let\label\orilabel,\setcounter{chapter}{1}}]
\begin{ltxlisting}[caption={ex03-ref.tex},label=ex03:ref]
\begin{equation}\label{eq:pitagoras}
a^2=b^2+c^2
\end{equation}
Pela Equação~\ref{eq:pitagoras} da página~\pageref{eq:pitagoras}\ldots
\end{ltxlisting}
%\end{LTXexample}
%\end{singlespacing}
%\setcounter{chapter}{\value{lstorichapter}}

Como já foi mencionado, o ambiente \texttt{equation} produz enumeração automática na equação. 
Assim, poderá criar um rótulo para referenciar o número desta equação com o comando \verb+\label+. 
O nome do rótulo escolhido para esta equação foi \texttt{eq:pitagoras}. 
Tome cuidado para não usar espaços, letras acentuadas ou símbolos especiais, o que dificulta a depuração ou causa erros. 
No Exemplo~\ref{ex03:ref}, ``:'' foi usado em vez do espaço para separar palavras. 
Note que foi colocado o prefixo ``eq'' para rótulo da equação. 
Como pode ter teoremas, equações, figuras, etc que podem sugerir o mesmo nome, é costume usar algum prefixo para cada categoria, o que facilitará a lembrar o elemento que é associado ao rótulo.
Para referenciar (usar o número) da equação, foi usado o comando \verb+\ref+ e para referenciar a página, foi usado o comando \verb+\pageref+\index{pageref@\verb+\pageref+}. 
Note que os editores especializados para \LaTeX, costuma ter recursos de listar os rótulos existentes para referências, o que facilita a editoração e prevenir erros.
Também note o uso de ``\verb+~+''\index{\~{}@\verb+~+} em vez do espaço para usar referências, o que evita que nome ``Equação'' fique no final de uma linha e numeração no começo da próxima linha.

Quando usa a referência cruzada, as enumerações sempre estarão coerentes e se usar o pacote \texttt{hyperref}, terão link automáticos no documento PDF.

Uma observação importante é o fato do \LaTeX{} utilizar arquivo auxiliar para armazenar rótulos. Assim, quando usa as referências cruzadas, precisam compilar o documento duas vezes para ter a enumeração ou link correta. 

\clearpage\thispagestyle{empty}\cleardoublepage
\chapter{Estrutura de Texto}\label{chap:estrutura:texto}
Neste capítulo, vamos ver sobre a formatação de textos.

\section{Alinhamentos}
O alinhamento do texto padrão no \LaTeX{} é justificado\index{alinhamento!justificado}, isto é, alinhado tanto a direita como a esquerda.
Para alinhar somente a esquerda\index{alinhamento!esquerda} ou a direita\index{alinhamento!direita}, usa-se o ambiente \texttt{flushleft}\index{flushleft} e \texttt{flushright}\index{flushright}. 
Para centralizar\index{alinhamento!centralizado}, usa-se o ambiente \texttt{center}\index{center}.
O Exemplo~\ref{ex04:alinhamento} ilustra os alinhamentos de texto.

%\begin{singlespacing}
% \begin{LTXexample}[pos=b,caption={ex01-04.tex},label=ex:01-04]
\begin{ltxlisting}[caption={ex04-alinhamento.tex},label=ex04:alinhamento]
Parágrafo normal.
\begin{center}
Parágrafo centralizado.
\end{center}
\begin{flushright}
Parágrafo alinhado a direita.
\end{flushright}
\end{ltxlisting}
%\end{LTXexample}
%\end{singlespacing}

Quando quer alinhar dentro do ambiente (o que tem \verb+\begin+ e \verb+\end+, poderá colocar como comando \verb+\center+\index{center@\verb+\center+}, \verb+\flushleft+\index{flushleft@\verb+\flushleft+} e \verb+\flushright+\index{flushright@\verb+\flushright+}, respectivamente que o alinhamento se aplicará até obter o final do ambiente em questão.

Para ter trecho justificado em ambos os lados dentro do ambiente com alinhamento a esquerda, direita ou centralizada, costuma colocar dentro do ambiente \texttt{minipage} que veremos na Seção~\ref{sec:minipage}.

Não é o alinhamento, mas para incluir linhas em branco\index{linhas em branco}, deverá colocar ``\verb+\+'' intercalado com linhas em branco.
Para indicar uma nova linha\index{nova linha} (finalizar uma linha sem esperar atingir final de linha), coloca se ``\verb+\\+''\index{\textbackslash\textbackslash} ou \verb+\newline+\index{newline@\verb+\newline+}. 
Para inserir quebra de linhas\index{quebra de linha}, mas que justifique, deverá usar o \verb+\linebreak+\index{linebreak@\verb+\linebreak+}, o que é ilustrado no Exemplo~\ref{ex04:nova:linha}

%\begin{singlespacing}
\begin{ltxlisting}[caption={ex04-nova-linha.tex},label=ex04:nova:linha]
Esta é a primeira linha \\
e esta e a segunda linha

\

Esta é a primeira linha \linebreak
e esta é a segunda linha.
\end{ltxlisting}
%\end{singlespacing}

Note o uso de ``\verb+\+''\index{\textbackslash@\verb+\+} para inserir linha em branco.

\section{Notas de rodapé e ênfase de texto}

Nota de rodapé\index{nota de rodapé} é colocado pelo comando \verb+\footnote{<texto>}+\index{footnote} que colocará marca de rodapé no local e \verb+<texto>+ na parte inferior da página atual.

Ele é usado para colocar alguma observação, mas não quer que conste como conteúdo do documento. 
Por exemplo, explicação dos termos que aparece, podem ser colocados como rodapé. Veja o Exemplo~\ref{ex04:rodape}

% showexpl usa \parbox e não o minipage.
% Assim, nao há como colocar footnote embaixo dele.
% Usando lstlisting e minipage para ter mesmo effeito de showexpl
%\begin{singlespacing}
%\begin{lstlisting}[caption={ex06-08.tex},label=ex:06-08]
\begin{ltxlisting}[caption={ex04-rodape.tex},label=ex04:rodape]
No meio do texto, podemos colocar a nota de rodapé\footnote{nota que fica na parte inferior da página} para explicações adicionais tais como significado da palavra, ou fonte que foi usada.
\end{ltxlisting}
% \end{singlespacing}

%\noindent
%\begin{framed}
%\begin{singlespacing}
%\begin{minipage}{\linewidth}
%\renewcommand{\thempfootnote}{\arabic{mpfootnote}}
%No meio do texto, podemos colocar a nota de rodapé\footnote{nota que fica na parte inferior da página} para explicações adicionais tais como significado da palavra, ou fonte que foi usada.
%\end{minipage}
%\end{singlespacing}
%\end{framed}

Para enfatizar o texto\index{enfase@{ênfase} de texto} dentro do contexto como os termos a ser definidos, usa-se o comando \verb+\emph{texto}+\index{emph@\verb+\emph+} ou o ambiente \texttt{em}\index{em}. 
Este comando (ou ambiente) alterna entre fontes romano reto e itálico para que o trecho seja enfatizado.
%Outra ênfase comum as palavras é usar o negrito que faz destacar no meio do texto. O negrito é produzido pelo comando \verb+\textbf{texto}+ ou pelo ambiente \texttt{bfseries}.
Note que os ambientes \texttt{em} tem a versão de comandos \verb+\em+\index{em@\verb+\em+} que pode ser usados dentro do outro ambiente (como os comandos de alinhamento).
Caso não esteja dentro do ambiente, poderá delimitar simplesmente pelos chaves e colocar o comando no começo dele. Isto está ilustrado no Exemplo~\ref{ex04:enfase}.

%\begin{singlespacing}
% \begin{LTXexample}[pos=b,caption={ex03-01.tex},label=ex:03-01]
\begin{ltxlisting}[caption={ex04-enfase.tex},label=ex04:enfase]
\emph{Enfatizado} ou 
\begin{em}enfatizado\end{em}
ou 
\begin{center}
\em Centralizado e enfatizado.
\end{center}
ou
{\em Enfatizado.}
Texto normal.
\end{ltxlisting}
%\end{LTXexample}
%\end{singlespacing}

%Para as diversas necessidades, é possível controlar as fontes, o que será estudado no Capítulo~\ref{chap:fontes}.

\section{Listas}

O texto pode ter várias estruturas básicas, tais como listas, cotações, etc.

Uma das mais utilizadas é listas. 
%Veremos alguns deles.

%Vamos começar pelas listas.

Existem três tipos básicos de listas\index{lista} que são especificadas pelos ambientes \texttt{enumerate}\index{enumerate} (lista enumerada\index{lista!enumerada}), \texttt{itemize}\index{itemize} (lista de itens\index{lista!itemizada}) e \texttt{description}\index{dscription} (lista de descrições\index{lista!descrição}).
Dentro de cada ambiente, cada item\index{lista!item} é iniciado com o comando \verb+\item+\index{item@\verb+\item+}.
O Exemplo~\ref{ex04:listas} ilustra algumas destas listas.

%\begin{singlespacing}
% \begin{LTXexample}[pos=b,caption={ex06-01.tex},label=ex:06-01]
\begin{ltxlisting}[caption={ex04-listas.tex},label=ex04:listas]
\begin{enumerate}
  \item Este é o primeiro item da lista enumerada.
  \item Agora, segundo item da lista enumerada
\end{enumerate}
\begin{itemize}
  \item Este é o primeiro item da lista de itens.
  \item Agora, segundo item da lista de itens.
\end{itemize}
\begin{itemize}
  \item Item com marcador padrão.
  \item [$\ast$] Item com marcador personalizado.
\end{itemize}
\begin{description}
  \item [enumerada] Cada item recebe uma enumeração.
  \item [itenizada] Em vez de enumeração, recebe um marcador.
  \item [descrição] Descrição das palavras.
\end{description}
\end{ltxlisting}
%\end{singlespacing}

As listas podem conter outras listas como itens\index{sub listas}. 
Veja Exemplo~\ref{ex04:sublistas}.

%\begin{singlespacing}
%\begin{LTXexample}[pos=b,caption={ex06-01.tex, parte b},label=ex:06-01:b]
\begin{ltxlisting}[caption={ex04-sublistas.tex},label=ex04:sublistas]
\begin{enumerate}
  \item Este é o primeiro item da lista enumerada.
  \item Segundo item é lista de itens.
    \begin{itemize}
      \item Este é o primeiro item da lista de itens.
      \item Agora, segundo item da lista de itens.
    \end{itemize}
  \item Este é o terceiro item da lista enumerada.
\end{enumerate}
\end{ltxlisting}
%\end{LTXexample}
%\end{singlespacing}

Para trabalhar com listas enumeradas, é aconselhável carregar o pacote \texttt{enumerate}\index{enumerate} que permite controlar as enumerações\index{lista!enumerada!controle}.
Para isso, acrescente o comando \verb+\usepackage{enumerate}+ no \texttt{preamble} do documento.

No \texttt{enumerate}, passa-se um parâmetro opcional que seria o modelo de enumeração do primeiro item. 
Neste modelo, ``1'', ``i'', `` I'', ``a'' e ``A'' serão considerados contadores e eles são incrementados a cada item.
Se aparecer estas letras que não sejam contadores, delimite pelos chaves.
Veja o Exemplo~\ref{ex04:enumerate}.

%\begin{singlespacing}
% \begin{LTXexample}[pos=b,caption={ex06-01.tex, parte c},label=ex:06-01:c]
\begin{lstlisting}[caption={ex04-enumerate.tex},label=ex04:enumerate]
\documentclass[12pt,a4paper]{article}
\usepackage[T1]{fontenc}
\usepackage[utf8]{inputenc}
\usepackage{amsmath,amssymb}
\usepackage{enumerate}
\begin{document}
\begin{enumerate}[Propr{i}ed{a}de 1)]
  \item $x + (y + z) = (x + y) + z$ (associativa)
  \item $x + y = y + x$ (comutativa)
\end{enumerate}
\end{document}
\end{lstlisting}
%\end{singlespacing}

% Precisa acertar a margem esquerda (conflito dos pacotes, ou
% por estar dentro do ambiente?
%\begin{singlespacing}
\begin{ltxlistingout}
\begin{enumerate}[label={Propriedade \arabic*)},leftmargin=8em]%listparindent={\widthof{\labelitemi}}]
  \item $x + (y + z) = (x + y) + z$ (associativa)
  \item $x + y = y + x$ (comutativa)
\end{enumerate}
\end{ltxlistingout}
%\end{singlespacing}
 
Se precisar de ambientes extras de listas diferentes do padrão, podemos definir usando o ambiente \texttt{list}\index{list}, o que não vamos entrar em detalhes.

\section{Tabelas}

Uma tabela\index{tabela} no modo texto é produzido pelo ambiente \texttt{tabular}\index{tabular} e a tabela no modo matemático é produzido pelo ambiente \texttt{array}\index{array}, que apresentam o mesmo sintaxe e a mesma funcionalidade.

O argumento obrigatório destes ambientes é o ``alinhamento'' das colunas que devem ser especificados com ``l'' (left), ``c'' (center), ``r'' (right) ou ``p\{largura\}'' (texto justificado com largura fixa). 
Para traçar uma linha vertical entre colunas ou no bordo, usa-se o ``\verb+|+''\index{"@vertical@\vert}, junto as especificações de alinhamento.
Os elementos da tabela é indicado, separado pelo ``\&''\index{\&}, sendo que a mudança de linhas é feito pelo ``\verb+\\+''\index{\textbackslash\textbackslash} que é \verb+\newline+. 
Para traçar uma linha horizontal, usa-se o comando ``\verb+\hline+''\index{hline@\verb+\hline+}.

O Exemplo~\ref{ex04:tabular} é um exemplo de tabelas no modo texto.

%\begin{singlespacing}
%\begin{LTXexample}[pos=b,caption={ex06-02.tex},label=ex:06-02]
\begin{ltxlisting}[caption={ex04-tabular.tex},label=ex04:tabular]
  \begin{tabular}{||l|c|r||} % linhas verticais duplas na borda e simples entre colunas
   \hline % uma linha horizontal no comeco da tabela
       & 2 & 3 \\ \hline % uma linha horizontal apos esta linha
   abc & 3 & 4 \\
   3   & 4 & $\frac{1+\sqrt{2}}{5}$ \\ \hline % uma linha horizontal pra finalizar
  \end{tabular}
  
  Tabela com uma coluna de largura fixa.

  \begin{tabular}{|p{3.5cm}|l|} \hline
   cenouras  (500g)          & R\$0,50  \\ \hline
   cogumelos (vidro de 500g) & R\$5,00  \\ \hline
   batata    (1Kg)           & R\$1,20  \\ \hline \hline
   total                     & R\$7,20  \\ \hline
\end{tabular}

Tabela de largura fixa

% largura da tabela é 0.5 de \columnwidth (largura da linha)
\begin{tabular*}{0.5\columnwidth}{@{\extracolsep{\fill}}|l|c|r|}
   \hline
       & 2 & 3 \\ \hline
   abc & 3 & 4 \\
   3   & 4 & $\frac{1+\sqrt{2}}{5}$ \\ \hline
  \end{tabular*}
  
\end{ltxlisting}
%\end{LTXexample}
%\end{singlespacing}

Para mesclar células\index{tabela!mesclar células} (juntar mais de uma célula como sendo uma única célula), usa-se o comando \texttt{multicolumn}\index{multicolumn} que tem como primeiro argumento, o número de células a serem juntados, o segundo especifica o alinhamento da coluna e terceiro, o que vai colocar nesta célula. 
Note que as formatações de colunas do \texttt{multicolumn} deve ser especificado um por um, independente de estar especificado ou não no começo da tabela.

No Exemplo~\ref{ex04:multicolumn} com o ambiente \texttt{array} (para ambiente matemático), foi usado o \texttt{multicolumn} para juntar duas primeiras colunas da primeira linha, centrando os dados e traçando linha vertical antes e depois da célula.

%\begin{singlespacing}
%\begin{LTXexample}[pos=b,caption={ex06-02.tex, parte b},label=ex:06-02:b]
\begin{ltxlisting}[caption={ex04-multicolumn.tex},label=ex04:multicolumn]
\[
  \begin{array}{||l|c|r||} \hline
   \multicolumn{2}{||c|}{2} & 3 \\ \hline
   2^3 & 3 & 4 \\
   3 & 4 & \frac{1+\sqrt{2}}{5} \\ \hline  
  \end{array}
\]
\end{ltxlisting}
%\end{LTXexample}
%\end{singlespacing}

Para traçar linhas horizontais apenas em algumas células, usa-se o comando \texttt{\textbackslash cline\{i-j\}}\index{cline@\verb+\cline+} onde ``\texttt{i}'' e ``\texttt{j}'' são colunas iniciais e finais onde a linha é traçada. Veja o Exemplo~\ref{ex04:cline}.

%\begin{singlespacing}
% \begin{LTXexample}[pos=b,caption={ex06-02.tex, parte c},label=ex:06-02:c]
\begin{ltxlisting}[caption={ex04-cline.tex},label=ex04:cline]
\begin{tabular}{||l|lr||} \hline
\textbf{Produto}  & \multicolumn{2}{|c||}{\textbf{Preços}} \\\hline
   cenouras   & R\$1.00 & (por Kg) \\ \cline{2-3}  
              & R\$0.20 & (por unidade) \\ \hline
   cogumelos  & R\$4.00 & \multicolumn{1}{|r||}{(por vidro)} \\ \cline{1-1} \cline{3-3}
   pêssego    &   & \multicolumn{1}{|r||}{(por Kg)} \\ \hline
 \end{tabular}
\end{ltxlisting}
%\end{LTXexample}
%\end{singlespacing}

No Exemplo~\ref{ex04:cline}, o comando \texttt{multicolumn} foi usado somente para formatar a coluna (colocar linha vertical) e não para juntar células.

Se precisar traçar linhas duplas ou similares na tabela, podemos usar o pacote \texttt{hhline}\index{hhline} que não entraremos em detalhes.

Para alinhar os decimais, usamos o pacote \texttt{dcolumn}\index{dcolum}
que define um novo especificador de colunas na tabela, especificado por \texttt{D\{<decimal em TeX>\}\{<decimal em PDF>\}\{<casas decimais>\}}.
 
\texttt{<decimal em TeX>} é a especificação de pontos decimais usado no arquivo TeX (como escreve no TeX), 
\texttt{<decimal em PDF>} é o decimal utilizado para documento de saída PDF, e o \texttt{<casas decimais>} é o número de casas decimais a serem considerados (casas decimais excedentes serão truncados).
Caso a especificação do número de casas decimais for negativo, qualquer número de casas decimais é aceito sem ser truncados (alinhando nos pontos decimais).
 
Como especificar três parâmetros toda vez que precise alinhar os pontos decimais é trabalhoso, é sugerido que defina um especificador no \texttt{preamble}, usando o comando \texttt{newcolumntype}\index{newcolumntype@\verb+newcolumntype+} como abaixo:

\begin{singlespacing}
\begin{lstlisting}[language={[LaTeX]TeX}]
\usepackage{dcolumn}
\newcolumntype{d}[1]{D{.}{\cdot}{#1}}
\newcolumntype{.}{D{.}{.}{-1}}
\newcolumntype{,}{D{,}{,}{2}}
\end{lstlisting}
\end{singlespacing}

Ele define especificador de colunas ``d{<num>}'' que é interpretado como \texttt{D\{.\}\{\cdot\}\{<num>\}} e, ``.'' e ``,'' que serão interpretados como sendo \texttt{D\{.\}\{.\}\{-1\}} e \texttt{D\{,\}\{,\}\{2\}} respectivamente.

Note que podemos modificar os comandos anteriores para que ``.'' seja convertida para ``,'' ou vice versa.
Veja o Exemplo~\ref{ex04:dcolumn}.

%\begin{singlespacing}
%\begin{LTXexample}[pos=b,caption={ex06-02.tex, parte d},label=ex:06-02:d]
\begin{ltxlisting}[caption={ex04-dcolumn.tex},label=ex04:dcolumn]
\documentclass[12pt,a4paper]{article}
\usepackage[T1]{fontenc}
\usepackage[utf8]{inputenc}
\usepackage{amsmath,amssymb}
\usepackage{dcolumn}
\newcolumntype{d}[1]{D{.}{\cdot}{#1}}
\newcolumntype{.}{D{.}{.}{-1}}
\newcolumntype{,}{D{,}{,}{2}}
\begin{document}
\begin{tabular}{|d{-1}|d{2}|.|,|}
\hline
1.2   & 1.2   &1.2    &1,2    \\
1.23  & 1.23  &12.5   &300,2  \\
1121.2& 1121.2&861.20 &674,29 \\
184   & 184   &10     &69     \\
.4    & .4    &       &,4     \\
      &       &.4     & \\ \hline
      
\end{tabular}
\end{document}
\end{ltxlisting}
%\end{LTXexample}
%\end{singlespacing}

Para ter as colunas igualadas automaticamente, usamos o pacote \texttt{tabularx}\index{tabularx} que define o ambiente \texttt{tabularx} onde primeiro argumento é largura da tabela e segundo é especificação de colunas, mas apresenta um especificador de coluna especial ``X''. 
Todas as colunas especificadas por ``X'' terá mesma largura 
(largura destas colunas depende da largura da tabela).
Veja o Exemplo~\ref{ex04:tabularx}.

%\begin{singlespacing}
% \begin{LTXexample}[pos=b,caption={ex06-02.tex, parte e},label=ex:06-02:e]
\begin{ltxlisting}[caption={ex04-tabularx.tex},label=ex04:tabularx]
\documentclass[12pt,a4paper]{article}
\usepackage[T1]{fontenc}
\usepackage[utf8]{inputenc}
\usepackage{amsmath,amssymb}
\usepackage{tabularx}
\begin{document}
\begin{tabularx}{\linewidth}{|c|X|c|X|}
\hline
\multicolumn{2}{|c|}{\texttt{multicolumn} } &
\scshape Terceira & \scshape Quarta \\ \hline
 primeira & a largura desta coluna depende da largura da 
tabela. &
 terceira & Quarta coluna funciona da forma similar a segunda e terá a mesma largura da segunda \\ \hline
\end{tabularx}
\end{document}
\end{ltxlisting}
%\end{LTXexample}
%\end{singlespacing}

No Exemplo~\ref{ex04:tabularx}, foi usado o comando \verb+\linewidth+\index{linewidth@\verb+\linewidth+} que é a medida da largura da linha atual.
Este comando aparece frequentemente quando queremos ajustar largura de um elemento para largura da linha atual. 
Mais sobre medidas, veja a Seção~\ref{sec:medidas:unidade} e Seção~\ref{sec:medidas} do Capítulo~\ref{chap:medidas:contadores}. 

O comando \verb+\footnote{texto}+ que coloca o ``texto'' no rodapé do documento funciona também dentro do \texttt{tabularx}.

Note que \texttt{hhline} que produz linhas duplas e similares não funciona no \texttt{tabularx}.

\section{Ambiente de tabulação}

Ambiente de tabulação\index{tabulação}\index{tabulação} é bastante útil para escrever conteúdos alinhados em diversos pontos, tal como no caso de algoritmos.
O ambiente é especificado pelo \texttt{tabbing}\index{tabbing}.

Para marcar a posição de tabulação, usa-se o comando \verb+\=+.
Quando uma linha for usado somente para marcar posição, coloca-se o comando \verb+\kill+\index{kill@\verb+\kill+} no final dele, para que o mesmo não seja impresso.

Para formatar conteúdos tabulados, é comum que queira tabular várias linhas sucessivas. 
O comando \verb"\+"\index{+@\verb'\+'} translada a margem esquerda por 
uma tabulação e \verb"\-"\index{-@\verb+\-+} retira uma tabulação da margem esquerda.
Com isso, podemos diagramar textos tabulados como no caso do algoritmo do Exemplo~\ref{ex04:tabbing}.

%\begin{singlespacing}
%\begin{LTXexample}[pos=b,caption={ex06-03.tex},label=ex:06-03]
\begin{ltxlisting}[caption={ex04-tabbing.tex},label=ex04:tabbing]
\begin{tabbing}
mm\=mm\=mm\=mm\=mm\= \kill % apenas para marcar tabulação
\{ Solução real da equação $ax^2+bx+c=0$ pela fórmula de Baskara \} \\
  $\Delta = b^2-4ac$ \\
  se d < 0 então \+ \\ % aplicar uma tabulação em todas linhas de abaixo
     escreva ``não há solução real'' \\
     pare \- \\ % retirar uma tabulação de todas linhas de abaixo
  senão \+ \\
     $x_1 = \frac{-b-\sqrt{\Delta}}{2a}$ \\ 
     $x_2 = \frac{-b+\sqrt{\Delta}}{2a}$ \- \\ 
  fim se
\end{tabbing}
\end{ltxlisting}
%\end{LTXexample}
%\end{singlespacing}

Se por algum motivo, pretende usar a acentuação no modo \TeX{} dentro do ambiente \texttt{tabbing}, note que existe três acentos no modo \TeX{} que não podem ser usados dentro do \texttt{tabbing} que são \verb+\'+, \verb+\`+ e \verb"\=".
Estes comandos são reservados para controle de tabulação. 
Assim, estes três comandos devem ser substituídos pelos comandos \verb+\a'+\index{a'@\verb+\a'+}, \verb+\a`+\index{a`@\verb+\a`+} e \verb"\a="\index{a=@\verb+\a=+} respectivamente (eles só funcionam dentro do \texttt{tabbing}).

Existem vários outros comandos que podem ser explorados no ambiente \texttt{tabbing}, mas não vamos entrar em detalhes.
Uma observação é que no caso de algoritmos ou código fonte de programas, existem pacotes específicos que implementam ambientes bem mais fáceis e versáteis do que \texttt{tabbing}.

% citados, \texttt{quotation}, \texttt{verse} e \texttt{verbatim}
\section{Textos de citações, versos e \texttt{verbatim}}

O ambiente \texttt{quote}\index{quote} é usado para escrever citações\index{citações} (trecho de outra fonte), exemplos e frases importantes.
O Exemplo~\ref{ex04:quote} foi extraído do ``The Not (So) Short Introduction to \LaTeX2e'' \cite{book:oeticker:2018}.

%\begin{singlespacing}
%\begin{LTXexample}[pos=b,caption={ex06-04.tex},label=ex:06-04]
\begin{ltxlisting}[caption={ex04-quote.tex},label=ex04:quote]
``Uma regra da tipografia sobre
o comprimento de uma linha é:
\begin{quote}
Em média, nenhuma linha deve
ser maior que 66 caracteres.
\end{quote}
Este é o motivo pelo qual as
páginas do \LaTeX{} possuem as
bordas tão grandes e também o
motivo pelo qual os jornais usam
impressão em colunas''.
\end{ltxlisting}
%\end{LTXexample}
%\end{singlespacing}

No ambiente \texttt{quote} usado para colocar citações (trecho tirado da outra fonte), o texto inteiro ficará com margens maiores a esquerda e a direita, o que tornaria visível quando colocar texto maior.

Quando o texto da citação é formado por vários parágrafos, pode querer que os parágrafos sejam indentados (tabulados para a direita).
Neste caso, use o ambiente \texttt{quotation}\index{quotation} em vez de 
\texttt{quote}.
O Exemplo~\ref{ex04:quotation} também usa o trecho do \cite{book:oeticker:2018} que explica os ambientes similares a \texttt{quote}, ficou com parágrafo indentado:

%\begin{singlespacing}
%\begin{LTXexample}[pos=b,caption={ex06-04.tex, parte b},label=ex:06-04:b]
\begin{ltxlisting}[caption={ex04-quotation.tex},label=ex04:quotation]
Veja que a cópia do trecho do ``The Not (So) Short Introduction to \LaTeX2e'' que explica os ambientes similares a \texttt{quote}, ficou com parágrafo indentado:
\begin{quotation}
``Existem dois ambientes similares: os ambientes quotation e verse.
Como ambiente \texttt{quotation} faz a indentação dos parágrafos, 
ele é usado para citações longas que se estendem por vários parágrafos. 
O ambiente \texttt{verse} é usado em poemas onde as quebras de linhas são importantes. 
As linhas são separadas por \verb+\\+ e por uma linha em branco no fim de cada verso''.
\end{quotation}
\end{ltxlisting}
%\end{LTXexample}
%\end{singlespacing}

No caso de versos\index{verso}, a linha que for quebrado pela falta de espaços, mas que constitui a mesma linha de cima, deve estar tabulado para direita.
O ambiente \texttt{verse}\index{verse} encarrega deste serviço.

O Exemplo~\ref{ex04:verse} é um exemplo do ``Humpty Dumpty'' do \cite{book:oeticker:2018}, colocado na caixa com largura insuficiente.
Para criar caixas mais estreitas, foi usado o ambiente \texttt{minipage} que será tratado na Seção~\ref{sec:minipage} do Capítulo~\ref{chap:estrutura:texto}.


%\begin{singlespacing}
%\begin{LTXexample}[pos=b,caption={ex06-04.tex, parte c},label=ex:06-04:c]
\begin{ltxlisting}[caption={ex04-verse},label=ex04:verse]
\begin{flushleft}
\begin{minipage}{0.5\linewidth}
\begin{verse}
Humpty Dumpty sat on a wall:\\
Humpty Dumpty had a great fall.\\
All the King's horses and all
the King's men\\
Couldn't put Humpty together
again.
\end{verse}
\end{minipage}
\end{flushleft}
\end{ltxlisting}
%\end{LTXexample}
%\end{singlespacing}

Para inserir o código fonte de programas\index{código fonte!programma} ou similar, usa-se o ambiente \texttt{verbatim}\index{verbatim} que coloca o texto de jeito que está (cópia \texttt{verbatim}). Este ambiente possui também a versão ``*'' \texttt{verbatim*}\index{verbatim*}.
Sem o ``*'', coloca o espaço e com o ``*'', coloca o caractere ``\verb*+ +'' no lugar de espaço.
Veja a diferença entre o código colocado pelo \texttt{verbatim} e \texttt{verbatim*} do Exemplo~\ref{ex04:verbatim}.
 
%\begin{singlespacing}
%\begin{LTXexample}[pos=b,caption={ex06-04.tex, parte d},label=ex:06-04:d]
\begin{ltxlisting}[caption={ex04-verbatim.tex},label=ex04:verbatim]
\begin{verbatim}
\begin{quote}
texto citado, texto importante ou exemplos.
\end{quote}
\end{verbatim}

\begin{verbatim*}
\begin{quote}
texto citado, texto importante ou exemplos.
\end{quote}
\end{verbatim*}
\end{ltxlisting}
%\end{LTXexample}
%\end{singlespacing}

O ambiente \texttt{verbatim} é usado para colocar conteúdos que devem ser lidos letra por letra, como no cado do código fonte de programas ou trecho de documentos muito importantes.

Para nome das variáveis do programa, nome de arquivos, etc que são curtos, mas também precisam ser lidos letra por letra, existe a versão comando que é \verb+\verb+\index{verb@\verb+\verb+}.
No \verb+\verb+, o primeiro caractere é o delimitador do argumento que deve ser usado também para indicar o final do argumento. Este delimitador pode ser escolhido (só não pode ser ``*''), mas fique atento que o final do argumento deve usar mesmo caractere. 
A versão com ``*''\index{verb*@\verb+\verb*+} exibe o espaço como``\verb+ +''.
Veja o Exemplo~\ref{ex04:verb}.

%\begin{singlespacing}
% \begin{LTXexample}[pos=b,caption={ex06-04.tex, parte e},label=ex:06-04:e]
\begin{ltxlisting}[caption={ex04-verb.tex},label=ex04:verb]
O comando \verb+\%+ produz ``\%''.

Nome de arquivos com espaço tal como \verb*|meu arquivo.tex|
não é recomendado para \LaTeX, pois dificulta a sincronização de PDF com o código fonte.
Use ``\verb+-+'' no lugar de espaço, como em \verb*|meu-arquivo.tex|.
\end{ltxlisting}
%\end{LTXexample}
%\end{singlespacing}

Para textos longos no ambiente \texttt{verbatim} deve carregar o pacote \texttt{verbatim}\index{verbatim} no ``preamble''.
Este pacote efetua algumas melhorias no ambiente \texttt{verbatim} e também implementa o comando \verb|\ver­batiminput|\index{verbatiminput@\verb+\verbatiminput+} que insere o arquivo externo diretamente no ambiente \texttt{verbatim}.

Note que, para inserir trecho dos códigos fontes de programas, existem pacotes apropriados que é mais prático do que usar o ambiente \texttt{verbatim}.

% \texttt{minipage}
\section{Caixa \texttt{minipage}}\label{sec:minipage}
O ambiente \texttt{minipage}\index{minipage} cria uma ``caixa'' que não é exatamente uma estruturação de textos, mas é usado frequentemente quando não encontra ambiente pronto para estruturação desejada.

Por exemplo, podemos colocar um bloco de texto justificado no lado direito da página, como no Exemplo~\ref{ex04:minipage}.

%\begin{singlespacing}
%\begin{LTXexample}[pos=b,caption={ex06-04.tex, parte f},label=ex:06-04:f]
\begin{ltxlisting}[caption={ex04-minipage.tex},label=ex04:minipage]
\begin{flushright}
\begin{minipage}{0.5\linewidth}
Este texto ficará justificado, mas como fica dentro de \texttt{minipage},
podemos colocar no lado direito. 
Para tanto, basta colocar \texttt{minipage} dentro do \texttt{flushright}. 
Claro que podemos colocar equações, figuras e outros elementos sem problemas. 
Por exemplo, o Teorema de Pitágoras foi posto abaixo:
\[ a^2=b^2+c^2 \]
\end{minipage}
\end{flushright}
\end{ltxlisting}
%\end{LTXexample}
%\end{singlespacing}

O argumento obrigatório para o ambiente \texttt{minipage} é a largura da ``caixa''. 
No Exemplo~\ref{ex04:minipage}, foi usado \verb+0.5\linewidth+ que é metade da largura da linha.

Este ambiente é importante para colocar um parágrafo de texto dentro dos comandos que não aceitam os parágrafos. 
Por exemplo, \verb+\fbox{}+\index{fbox@\verb+\fbox+} coloca moldura no elemento, mas não aceita o parágrafo de texto. 
Então, como colocar moldura no texto com várias linhas, ou nas figuras?
É simples: colocar tudo no \texttt{minipage} e colocar dentro do \texttt{fbox}, como no Exemplo~\ref{ex04:minipage:fbox}.

%\let\orifootnote\footnote
%\begin{singlespacing}
%\begin{LTXexample}[pos=b,caption={ex06-04.tex, parte g},label=ex:06-04:g,preset=\let\footnote\orifootnote]
\begin{ltxlisting}[caption={ex04-minipage-fbox.tex},label=ex04:minipage:fbox,preset={\renewcommand{\thempfootnote}{\alph{mpfootnote}}}]
\fbox{
\begin{minipage}{0.75\linewidth} % 3/4 da largura de linha
Usando o \texttt{minipage}, podemos colocar moldura no texto com parágrafo como este.

O \texttt{fbox} não acusa erros por ter parágrafo, pois o que está dentro de \texttt{minipage} é apenas um ``objeto''. 

Este truque funciona também para outros comandos que impede de colocar parágrafos, mudar linhas, etc.

Note que as notas de rodapé\footnote{rodapé é colocado pelo comando \texttt{\textbackslash footnote\{texto\} } } 
colocado dentro do minipage ficará na parte de baixo do \texttt{minipage} e não da página. 
O padrão dentro do \texttt{minipage} é enumerar eles como letra e não com números.
\end{minipage}
} % fbox
\end{ltxlisting}
%\end{LTXexample}
%\end{singlespacing}

Para elementos com poucas linhas de código, a versão comando \verb+\parbox+\index{parbox@\verb+\parbox+} é mais prático de ser usado.
Mas \verb+\parbox+ é mais simplificado do que \texttt{minipage}, podendo perder alguns recursos. 
Por exemplo, alguns comandos e ambientes tais como \verb+\verb+ e \texttt{verbatim} não funcionam dentro do \verb+\parbox+.
Assim, prefira usar o \texttt{minipage} sobre \verb+\parbox+.
Veja o Exemplo~\ref{ex04:parbox}.

%\begin{singlespacing}
%\begin{LTXexample}[pos=b,caption={ex06-07.tex},label=ex:06-07]
\begin{ltxlisting}[caption={ex04-parbox.tex},label=ex04:parbox]
Texto normal
\fbox{\parbox{3cm}{texto em várias linhas. \\ Com moldura.}}
Texto continua.
\end{ltxlisting}
%\end{LTXexample}
%\end{singlespacing}

O primeiro parâmetro do \verb+\parbox+ é a largura, como no caso de \texttt{minipage}.
Tanto no \texttt{minipage} como o \texttt{parbox}, \verb+\linewidth+\index{linewidth@\verb+\linewidth+} dentro dele é a largura da caixa passada como parâmetro.

%linewidth: \the\linewidth
%
%\parbox{0.5\linewidth}{linewidth: \the\linewidth}
%
%\begin{minipage}{0.5\linewidth}
%linewidth: \the\linewidth
%\end{minipage}

\section{Colunas múltiplas de texto}
Dependendo do documento, usa-se mais de uma coluna. 
Isto é o caso de alguns artigos que podem usar duas colunas, ou \texttt{poster} que costuma usar três colunas ou mais. 
Mesmo no documento de uma coluna, o índice remissivo costuma ficar em duas colunas. 
Apesar da maioria das classes de documentos tem implementado a opção de documento em duas colunas (opção \texttt{twocolumn}\index{twocolumn}) e comandos para alternar entre uma e duas colunas (comandos \verb+\onecolumn+\index{onecolumn@\verb+\onecolumn+}
e \verb+\twocolumn+\index{twocolumn@\verb+\twocolumn+}), a forma mais prático é usar o pacote \texttt{multicol}\index{multicol} que implementa o ambiente \texttt{muticols}\index{multicols} que permite criar qualquer número de colunas.
O padrão é não traçar linhas separando colunas, mas isto pode ser alterado, como no Exemplo~\ref{ex04:multicols}.
Aqui foi usado o \verb+\setlength+ para alterar a medida da espessura de linha que separa as colunas.
Medidas serão tratadas em Seção~\ref{sec:medidas:unidade} e Seção~\ref{sec:medidas} do Capítulo \ref{chap:medidas:contadores}.
O Exemplo~\ref{ex04:multicols} ilustra o uso de múltiplas colunas.

%\begin{singlespacing}
%\begin{LTXexample}[pos=b,caption={ex10-02.tex},label=ex:10-02]
\begin{ltxlisting}[caption={ex04-multicols.tex},label=ex04:multicols]
\documentclass[a4paper,12pt]{article}
\usepackage[T1]{fontenc}
\usepackage[utf8]{inputenc}
\usepackage[brazil]{babel}
\usepackage{color}
\usepackage{color}
\usepackage{multicol} % permite usar multiplas linhas
% espessura da linha que separa as colunas (0pt para desabilitar)
\setlength{\columnseprule}{1pt} 
\begin{document}
Este parágrafo está como uma única coluna, padrão para maioria das classes de documentos.
\begin{multicols}{2} % inicia duas colunas
Este parágrafo está em duas colunas.
O ambiente \texttt{multicols} balanceia os conteúdos em colunas para que todas colunas tenham mesmo tamanho.
Se preferir que passe para próxima coluna somente quando a coluna ficar cheia, use a versão ``*'' que não balanceia seus conteúdos.

Se quer finalizar a coluna manualmente, poderá usar o \verb+\columnbreak+ que finaliza a coluna atual e passa para próxima coluna.
\end{multicols} % finaliza o mode de duas colunas
Agora está no modo de uma única coluna novamente.
\end{document}
\end{ltxlisting}
%\end{LTXexample}
%\end{singlespacing}

%A cor da linha de divisão também pode ser configurada, o que não discutiremos aqui.
%% cor da linha que separa as colunas
%\def\columnseprulecolor{\color{blue}}

%
%
%Tratar do uso do pacote \texttt{multicols}

\clearpage\thispagestyle{empty}\cleardoublepage
\chapter{Aprofundando nas Fórmulas Matemáticas}
Neste capítulo, vamos aprofundar mais nas fórmulas matemáticas.
Para suporte à matemática, os pacotes \texttt{amssymb}\index{amssymb} e \texttt{amsmath}\index{amsmath} são usados. 
Portanto, coloque o comando \verb+\usepackage{amssymb,amsmath}+ no \texttt{preamble} do documento.

\section{Usando algumas fontes matemáticas}
Na matemática, além dos símbolos especiais e letras gregas, também usam alfabetos romanos de formatos diferentes. 
Por exemplo, para o conjunto dos números, costuma usar ``negrito do quadro negro'' (por exemplo, letra de traço duplo do AMS). 
Tais fontes são selecionados pelos comandos apropriados. 
Aqui veremos alguns dos mais usados.

O conjunto dos números como real, racional, etc, usam a letra maiúscula em ``negrito do quadro negro''. Para tanto, poderemos usar o comando \verb+\mathbb+\index{modo matematico@{modo matemático}!alfabeto!\verb+\mathbb{ABCdef123}+, $\mathbb{ABCdef123}$}.

Para Conjunto de funções contínuas, costuma usar a letra ``C'' maiúscula na forma caligráfica. 
A fonte caligráfica é indicado pelo comando \verb+\mathcal+\index{modo matematico@{modo matemático}!alfabeto!\verb+\mathcal{ABC}+, $\mathcal{ABC}$}.

Para algumas áreas de matemática, ainda usam a versão enfeitada de caracteres oferecido pelo comando \verb+\mathfrak+\index{modo matematico@{modo matemático}!alfabeto!\verb+\mathfrak{ABC}+, $\mathfrak{ABC}$}.

Para nome das funções, devemos usar a letra romana reta. 
O \LaTeX{} dispõe de comandos prontos para maioria das funções comumente usadas, mas as vezes precisamos escrever o nome da função que não está pronto, como o \verb+sen+ que representa a função seno em português. 
Como \LaTeX{} implementa nome das funções em inglês, o nome disponível para \texttt{seno} é \texttt{sin} produzido pelo comando \verb+\sin+. 
Também notemos que no Brasil costuma usar \texttt{tg} para \texttt{tangente} e não o \texttt{tan}.

Para estes e outros casos, podemos especificar os nomes das funções que \LaTeX{} não dispõe, colocando dentro do comando \verb+\mathrm+\index{modo matematico@{modo matemático}!alfabeto!\verb+\mathrm{ABCdef123}+, $\mathrm{ABCdef123}$} que usa a fonte romana reta.
Veja o Exemplo~\ref{ex05:mat:fontes:basico}.

%\let\mathcal\amsmathcal
%\let\mathbb\amsmathbb
%\begin{singlespacing}
%\begin{LTXexample}[pos=b,caption={ex04-03.tex},label=ex:04-03]
\begin{ltxlisting}[caption={ex05-mat-fontes-basico.tex},label=ex05:mat:fontes:basico]
\documentclass[12pt,a4paper]{article}
\usepackage[T1]{fontenc}
\usepackage[utf8]{inputenc}
\usepackage[brazil]{babel}
\usepackage{amssymb,amsmath}
\begin{document}
Exemplo do uso de letra caligráfica e de ``negrito do quadro negro''.
\[ f \in \mathcal{C}(\mathbb{R},\mathbb{R}) \]

Exemplo da letra romana maiúscula enfeitada $\mathfrak{R}$.

Exemplo do nome das funções
\[ \forall \theta \in \mathbb{R}, \cos^2\theta + \mathrm{sen}^2\theta = 1
\]

Outro exemplo: $\mathrm{tg}\theta = \frac{\mathrm{sen}\theta}{\cos\theta}$.

\[
\mathop{\mathrm{arg\,min}}_x f(x) = \{x : f(x) = \min_{x'} f(x')\}
\]
\end{document}
\end{ltxlisting}
%\end{LTXexample}
%\end{singlespacing}
%\let\mathcal\orimathcal
%\let\mathbb\orimathbb

Note que no \texttt{argmin}, o $x$ será colocado embaixo e não como índice. Para que o que foi colocado como índice fique embaixo, usa-se o comando \verb+\mathop+\index{modo matematico@{modo matemático}!basico@{básico}!\verb+\mathop+}.
O comando \verb+\,+ usado no meio de \texttt{argmin} é um comando de espaçamento usado na fórmula que insere pequeno espaço. Mas sobre espaçamentos no modo amtemático, veja a Seção~\ref{sec:medidas:unidade} e Seção~\ref{sec:medidas} do Caítulo~\ref{chap:medidas:contadores}.
 
Em geral, costuma definir comandos para nome das funções no \texttt{preamble} do documento para facilitar a digitação. 
A definição de comandos e ambientes serão estudados na Seção~\ref{sec:comandos} do Capítulo~\ref{chap:novo:comando}.

Os comandos para nome das funções pré-definidas estão na Seção~\ref{sec:fonte:matematica} do Capítulo~\ref{chap:fontes}.
%A seguir, alguns nome das funções pré-definidos.
%\verb+\arccos+, \verb+\cos+, \verb+\csc+, \verb+\exp+, \verb+\ker+, \verb+\limsup+, \verb+\min+, \verb+\sinh+, \verb+\arcsin+, \verb+\cosh+, \verb+\deg+, \verb+\gcd+, \verb+\lg+, \verb+\ln+, \verb+\Pr+, \verb+\sup+, \verb+\arctan+, \verb+\cot+, \verb+\det+, \verb+\hom+, \verb+\lim+, \verb+\log+, \verb+\sec+, \verb+\tan+, \verb+\arg+, \verb+\coth+, \verb+\dim+, \verb+\inf+, \verb+\liminf+, \verb+\max+, \verb+\sin+, \verb+\tanh+.

\section{Texto, função por partes e matrizes}
Para inserir texto nas fórmulas, usa-se o comando \verb+\text+\index{modo matematico@{modo matemático}!costrucao@{construção}!\verb+\text+} do pacote \texttt{amsmath}. 
Evite de usar o \verb+\mbox+\index{mbox@\verb+\mbox+} para este propósito.

Para ilustrar, vamos usar o ambiente \texttt{cases}\index{modo matematico@{modo matemático}!ams!cases} usado para definir funções por partes, no Exemplo~\ref{ex05:mat:text}.

%\begin{singlespacing}
%\begin{LTXexample}[pos=b,caption={ex04-03.tex, parte b},label=ex:04-03:b]
\begin{ltxlisting}[caption={ex05-mat-text.tex},label=ex05:mat:text]
\[
|x| = \begin{cases}
-x, & \text{ se } x < 0 \\
x, & \text{ caso contrário}\\
\end{cases}
\]
\end{ltxlisting}
%\end{LTXexample}
%\end{singlespacing}

O caractere ``\&''\index{\&} indica o ponto de alinhamento e \verb+\\+\index{\textbackslash\textbackslash} indica a mudança de linha.

O pacote \texttt{amsmath} dispõe de ambiente para produzir matrizes\index{modo matematico@{modo matemático}!matriz}. 
O ambiente \texttt{matrix}\index{modo matematico@{modo matemático}!ams!matrix} produz matriz sem o delimitador, \texttt{pmatrix}\index{modo matematico@{modo matemático}!ams!pmatrix} é delimitado pelos parenteses\index{matriz!delimitador!parenteses}, \texttt{bmatrix}\index{modo matematico@{modo matemático}!ams!bmatrix} é delimitado pelos colchetes\index{modo matematico@{modo matemático}!ams!matriz delimitado por cochetes},\texttt{Bmatrix}\index{modo matematico@{modo matemático}!ams!Bmatrix} é delimitado pelas chaves\index{modo matematico@{modo matemático}!ams!matriz delimitado por chaves}, \texttt{vmatrix}\index{modo matematico@{modo matemático}!ams!vmatrix} é delimitado pelas retas verticais\index{modo matematico@{modo matemático}!ams!matriz valor absoluto} e \texttt{Vmatrix}\index{modo matematico@{modo matemático}!ams!Vmatrix}  é delimitado pelas retas verticais duplas\index{modo matematico@{modo matemático}!ams!matriz norma}.
Eles estão ilustradas no Exemplo~\ref{ex05:matriz}.
%\begin{singlespacing}
%\begin{LTXexample}[pos=b,caption={ex04-04.tex},label=ex:04-04]
\begin{ltxlisting}[caption={ex05-matriz.tex},label=ex05:matriz]
\[
  \begin{pmatrix}
   1 & 2 & 3 \\
   2 & 3 & 4 \\
   3 & 4 & 5  
  \end{pmatrix}, 
  \begin{bmatrix}
   1 & 2 & 3 \\
   2 & 3 & 4 \\
   3 & 4 & 5  
  \end{bmatrix}, 
  \begin{vmatrix}
   1 & 2 & 3 \\
   2 & 3 & 4 \\
   3 & 4 & 5  
  \end{vmatrix},
  \begin{Vmatrix}
   1 & 2 & 3 \\
   2 & 3 & 4 \\
   3 & 4 & 5  
  \end{Vmatrix}
\]
\end{ltxlisting}
%\end{LTXexample}
%\end{singlespacing}

\section[Delimitadores, chaves e integrais]{Delimitadores auto ajustáveis, chaves embaixo e integrais}

Existem delimitadores auto ajustáveis que aumenta conforme o seu conteúdo aumenta de altura.
Para usar ele, use \verb+\left+\index{modo matematico@{modo matemático}!delimitador!\verb+\left+} antes do delimitador esquerdo (tais como %
``(''\index{modo matematico@{modo matemático}!delimitador!$($}, %
 ``[''\index{modo matematico@{modo matemático}!delimitador!$[$}, %
 ``\verb\+\{+''\index{modo matematico@{modo matemático}!delimitador!chaveaberta@{\textbackslash\curlyopensymbforindex,$\curlyopensymbforindex$}},%
 ``|''\index{modo matematico@{modo matemático}!delimitador!$\backslash\vert$,$\vert$}, %
 ``\verb+\|+''\index{modo matematico@{modo matemático}!delimitador!$\backslash\Vert$,$\Vert$}, %
 ``\verb+\langle+''\index{modo matematico@{modo matemático}!delimitador!\verb+\langle+,$\langle$}, %
 ``\verb+\lfloor+''\index{modo matematico@{modo matemático}!delimitador!\verb+\lfloor+,$\lfloor$}, %
 ``\verb+\lceil+''\index{modo matematico@{modo matemático}!delimitador!\verb+\lceil+,$\lceil$})%
 e use \verb+\left+\index{modo matematico@{modo matemático}!delimitador!\verb+\right+} antes do delimitador direito (tais como %
``)''\index{modo matematico@{modo matemático}!delimitador!$)$}, %
 ``]''\index{modo matematico@{modo matemático}!delimitador!$]$}, %
 ``\verb\+\}+''\index{modo matematico@{modo matemático}!delimitador!chavefechada@{\textbackslash\curlyclosesymbforindex,$\curlyclosesymbforindex$}},%
 ``|''\index{modo matematico@{modo matemático}!delimitador!$\backslash\vert$,$\vert$}, %
 ``\verb+\|+''\index{modo matematico@{modo matemático}!delimitador!$\backslash\Vert$,$\Vert$}, %
 ``\verb+\rangle+''\index{modo matematico@{modo matemático}!delimitador!\verb+\rangle+,$\rangle$}, %
 ``\verb+\rfloor+''\index{modo matematico@{modo matemático}!delimitador!\verb+\rfloor+,$\rfloor$}, %
 ``\verb+\rceil+''\index{modo matematico@{modo matemático}!delimitador!\verb+\rceil+,$\rceil$}).
 Para colocar delimitador somente em um dos lados, ``.''\index{modo matematico@{modo matemático}!delimitador!\verb+\left.+}\index{modo matematico@{modo matemático}!delimitador!\verb+\right.+} é usado para indicar delimitador vazio (quando tem somente em um dos lados, outro lado seria ``.'').
Cuidado para não usar \verb+||+ para barra vertical dupla.
Veja o Exemplo~\ref{ex05:delimitador}.

%\begin{singlespacing}
%\begin{LTXexample}[pos=b,caption={ex02-01.tex, parte c},label=ex:02-01:c]
\begin{ltxlisting}[caption={ex05-delimitador.tex},label=ex05:delimitador]
\[ \left\|\frac{\frac{1}{x}}{\sqrt[3]{x^2+1}} \right\| \]
\[\left\{x \in [0,1) : \frac{x+1}{x-1} < 0 \right\} \]
\[ 2\left[1+\frac{\left(1+\frac{1}{5}\right)^2}{2} \right]+5 \] 
\[ \left. \frac{x^2}{x+1}\right|_{x=1} = \frac{1}{2} \]
\[ \left\langle f, g \right\rangle = \int_b^a f(x)g(x)dx \]
\end{ltxlisting}
%\end{LTXexample}
%\end{singlespacing}

Para casos em que delimitadores auto ajustáveis não funcionam como esperado, poderá indicar o tamanho manualmente, colocando a especificação do tamanho antes do delimitador. O especificador do delimitador grande na ordem crescente são: \verb+\big+\index{modo matematico@{modo matemático}!delimitador!\verb+\big+}, \verb+\Big+\index{modo matematico@{modo matemático}!delimitador!\verb+\Big+}, \verb+\bigg+\index{modo matematico@{modo matemático}!delimitador!\verb+\bigg+} e \verb+\Bigg+\index{modo matematico@{modo matemático}!delimitador!\verb+\Bigg+}.
Veja Exemplo~\ref{ex05:delimitador:grande}.

%\begin{singlespacing}
%\begin{LTXexample}[pos=b,caption={ex02-01.tex, parte d},label=ex:02-01:d]
\begin{ltxlisting}[caption={ex05-delimitador-grande.tex},label=ex05:delimitador:grande]
\[ \bigg( \Big(x + (y - z) \Big) + w \bigg) \]

\[ \lambda \bigg( f(x+0) + \cdots f(x_n) \]
\[ + \sum_{i=0}^n x_i \bigg) \]
\end{ltxlisting}
%\end{LTXexample}
%\end{singlespacing}

A ``chave'' sobre (ou abaixo) da fórmula é colocado pelo 
\verb|\overbrace|\index{modo matematico@{modo matemático}!acentuacao@{acentuação}!\verb+\overbrace{abc}^{n}+, $\overbrace{abc}^{n}$}%
 e \verb|\underbrace|\index{modo matematico@{modo matemático}!acentuacao@{acentuação}!\verb+\underbrace{abc}_{n}+, $\overbrace{abc}_{n}$}%
.
Veja Exemplo~\ref{ex05:bracos}.

%\begin{singlespacing}
%\begin{LTXexample}[pos=b,caption={ex02-01.tex, parte e},label=ex:02-01:e]
\begin{ltxlisting}[caption={ex05-bracos.tex},label=ex05:bracos]
\[ \overbrace{1+1+\cdots+1}^{n-\text{vezes }} = n \]
\[ \underbrace{k+k+\cdots+k}_{n-\text{vezes}} = nk \]
\end{ltxlisting}
%\end{LTXexample}
%\end{singlespacing}

Alguns casos de integrais múltiplas estão ilustrados no Exemplo~\ref{ex05:integrais}.%
\index{modo matematico@{modo matemático}!integral!\verb+\int+, $\int$}%
\index{modo matematico@{modo matemático}!integral!\verb+\iint+, $\iint$}%
\index{modo matematico@{modo matemático}!integral!\verb+\iiint+, $\iiint$}%
\index{modo matematico@{modo matemático}!integral!\verb+\idotsint+, $\idotsint$}%
\index{modo matematico@{modo matemático}!integral!\verb+\oint+, $\oint$}%

%\begin{singlespacing}
%\begin{LTXexample}[pos=b,caption={ex04-04.tex},label=ex:04-05]
\begin{ltxlisting}[caption={ex05-integrais.tex},label=ex05:integrais]
Integrais múltiplas
\[
\int_a^b f(x)\,dx\; \iint_R f(x.y)\,dxdy \; \iiint_B f(x,y,z)\,dxdydz \]
\[\idotsint f(x_1,\ldots,x_n)\,dx_1\cdots dx_n \]

Integral de linha $\oint f(x)\,dx$.
\end{ltxlisting}
%\end{LTXexample}
%\end{singlespacing}

\index{modo matematico@{modo matemático}!integral!\verb+\int+, $\int$}%
Note que \verb+cdots+\index{modo matematico@{modo matemático}!pontuacao@{pontuação}!\verb+\cdots+, $\cdots$} produz três pontos no meio, enquanto que \verb+\ldots+\index{modo matematico@{modo matemático}!pontuacao@{pontuação}!\verb+\ldots+, $\ldots$} produz três pontos embaixo.
Os comandos \verb+\,+ e \verb+\;+ inserem espaços extras na fórmula. 
Veja a Seção~\ref{sec:medidas} para detalhes.

pequenos espaços 
\section{Quebrando fórmulas em várias linhas}
Para quebrar uma equação sem enumeração em mais de uma linha, podemos usar o ambiente \verb+split+\index{modo matematico@{modo matemático}!costrucao@{construção}!split}.

Para criar alinhamento na equação quebrada, coloca-se o \& no ponto de alinhamento.
Para quebrar equação enumerada em mais de uma linha, use o ambiente  \verb|aligned|\index{modo matematico@{modo matemático}!costrucao@{construção}!aligned} dentro do ambiente \verb+equation+ em vez de \verb|split| e insere o \& nos pontos de alinhamento, como no Exemplo~\ref{ex05:split}.

%\begin{singlespacing}
%\begin{LTXexample}[pos=b,caption={ex04-04.tex, parte b},label=ex:04-04:b]
\begin{ltxlisting}[caption={ex05-split.tex},label=ex05:split]
\[
\begin{split}
2x+y=3 \\
x-y=1+a
\end{split}
\]

\[
\begin{split}
2x+y&=3 \\
x-y&=1+a
\end{split}
\]

\begin{equation}
\begin{aligned}
2x+y&=3 \\
x-y&=1+a
\end{aligned}
\end{equation}
\end{ltxlisting}
%\end{LTXexample}
%\end{singlespacing}

Podemos tentar usar o ambiente \texttt{cases} para definir sistema de equações como no Exemplo~\ref{ex05:cases:aligned},
mas isto não efetua o alinhamento (na posição de igualdade). 
Não podemos usar o  \&, pois o alinhamento seria efetuado de forma próprio para colocar condição da expressão e não a equação.
Assim, é aconselhável usar o \verb+split+ (quando não enumera) ou \verb+equattion+ com \verb+aligned+ (quando enumera) em vez de \texttt{cases}.

%\begin{singlespacing}
%\begin{LTXexample}[pos=b,caption={ex04-04.tex, parte c},label=ex:04-04:c]
\begin{ltxlisting}[caption={ex05-cases-aligned.tex},label=ex05:cases:aligned]
\begin{equation}
  \begin{cases}
      2x+y=1 \\
      x-y=1+a 
  \end{cases}
\end{equation}
\begin{equation}
  \left\{ \begin{aligned}
      2x+y&=1 \\
      x-y&=1+a 
  \end{aligned} \right.
\end{equation}
\end{ltxlisting}
%\end{LTXexample}
%\end{singlespacing}

Existem diversos ambientes estilo \texttt{equations}. 
Vamos ver alguns deles:

O ambiente \texttt{gather}\index{modo matematico@{modo matemático}!costrucao@{construção}!gather} produz várias equações, todas centralizadas.
Para inibir enumeração em algumas delas, coloque o \verb|\nonumber|\index{modo matematico@{modo matemático}!costrucao@{construção}!\verb+\nonumber+} na equação que deseja remover a enumeração (antes da quebra de linhas). Veja o Exemplo~\ref{ex05:gather}

%\begin{singlespacing}
%\begin{LTXexample}[pos=b,caption={ex04-04.tex, parte d},label=ex:04-04:d]
\begin{ltxlisting}[caption={ex05-gather.tex},label=ex05:gather]
\begin{gather}
x+y+z=1 \\
x-y+z=2 \\
x+y=0
\end{gather}
\begin{gather}
x+y+z=1 \\
x-y+z=2 \nonumber \\
x+y=0
\end{gather}
\end{ltxlisting}
%\end{LTXexample}
%\end{singlespacing}

Para equações em várias linhas, com pontos de alinhamento, 
usa-se o \texttt{align}\index{modo matematico@{modo matemático}!costrucao@{construção}!align} que pode inibir enumeração de algumas equações como em \texttt{gather}.

Quando escreve uma expressão grande quebrado em várias linhas,
as vezes é legal tabular cada linha para direita.
Este efeito pode ser obtido pelo \verb|multline|\index{modo matematico@{modo matemático}!costrucao@{construção}!multiline}, como do Exemplo~\ref{ex05:multiline}.

%\begin{singlespacing}
%\begin{LTXexample}[pos=b,caption={ex04-04.tex, parte e},label=ex:04-04:e]
\begin{ltxlisting}[caption={ex05-multiline.tex},label=ex05:multiline]
\begin{align}
x+y+z&=1 \\
x-y+z&=2 \nonumber \\
x+y&=0
\end{align}
\begin{multline}
Ax+Ay+Az=
  a_{11}x_1+\cdots +a_{1n}x_n+\cdots +\cdots +a_{n1}x_1+\cdots a_{nn}x_n    \\
  a_{11}y_1+\cdots +a_{1n}y_n+\cdots +\cdots +a_{n1}y_1+\cdots a_{nn}y_n    \\
  a_{11}z_1+\cdots +a_{1n}z_n+\cdots +\cdots +a_{n1}z_1+\cdots a_{nn}z_n    
\end{multline}
\end{ltxlisting}
%\end{LTXexample}
%\end{singlespacing}

Para mais ambientes deste estilo, consulte o manual de AMS\LaTeX. \cite[Cap. 8]{book:goossens:2004} explica bem os ambientes e comandos matemáticos do AMS (que está sendo discutido parcialmente neste capítulo).

Note que, todos os ambientes do estilo \texttt{equation} 
(\texttt{equation}, \texttt{gather}, \texttt{align}, \texttt{multline}, etc)
apresentam a versão com *\index{equation*}\index{gather*}\index{align*}\index{multiline*} na qual remove toda enumeração (versão * equivale 
a colocar \verb|\nonumber| em todas equações).
Veja Exemmplo~\ref{ex05:align:star}.

%\begin{singlespacing}
%\begin{LTXexample}[pos=b,caption={ex04-04.tex, parte f},label=ex:04-04:f]
\begin{ltxlisting}[caption={ex05-align-star.tex},label=ex05:align:star]
\begin{align*}
x+y+z&=1 \\
x-y+z&=2 \\
x+y&=0
\end{align*}
\end{ltxlisting}
%\end{LTXexample}
%\end{singlespacing}

Para equações de várias linhas, uma linha independente de texto pode ser inserido pelo comando \verb+\intertext+\index{modo matematico@{modo matemático}!costrucao@{construção}!\verb+\intertext+}.

Veja Exemmplo~\ref{ex05:intertext}.

%\begin{singlespacing}
%\begin{LTXexample}[pos=b,caption={ex04-04.tex, parte f},label=ex:04-04:f]
\begin{ltxlisting}[caption={ex05-intertext.tex},label=ex05:intertext]
Para $y=2x$, temos
\begin{align*}
f(x)&=2x+y+3x 
\intertext{juntando o $x$, temos}
&=5x+y \\
 & = 7x
\end{align*}
\end{ltxlisting}
%\end{LTXexample}
%\end{singlespacing}
Note que não precisa colocar \verb+\\+ antes e depois do \verb+\intertext+. 
 
\section{Nome sobre setas e delimitador empilhados}

Para colocar nome nas setas\index{setas!com nome}, usa-se o \verb+\stackrel+\index{modo matematico@{modo matemático}!costrucao@{construção}!\verb+\stackrel+}.

Para colocar limitante de soma, produto, etc em duas linhas\index{limitante!duas linhas}, usa-se o \verb+\substack+\index{modo matematico@{modo matemático}!costrucao@{construção}!\verb+\substack+}.

 
Para colocar embaixo, poderá usar o \verb+\underset+\index{modo matematico@{modo matemático}!costrucao@{construção}!\verb+\underset+}.

Simplesmente empilhar um sobre outro, usa-se o \verb+\atop+\index{modo matematico@{modo matemático}!costrucao@{construção}!\verb+\atop+}.

Veja Exemplo~\ref{ex05:stackrel}.

%\begin{singlespacing}
%\begin{LTXexample}[pos=b,caption={ex04-05.tex, parte b},label=ex:04-05:b]
\begin{ltxlisting}[caption={ex05-stackrel.tex},label=ex05:stackrel]
\[ X \stackrel{f}{\to} Y\]
\[ f : \underset{x}{X}  \underset{\mapsto}{\to} \underset{f(x)}{Y} \]
\[ L_i(x)=\prod_{\substack{j=0 \\ i\ne j}} \frac{x-x_j}{x_i-x_j} \]
\[ X \atop Y \]
\end{ltxlisting}
%\end{LTXexample}
%\end{singlespacing}

Binomial de Newton pode ser produzido por \verb+binom+\index{modo matematico@{modo matemático}!costrucao@{construção}!\verb+\binom+} como no Exemplo~\ref{ex05:binom}.

%\begin{singlespacing}
%\begin{LTXexample}[pos=b,caption={ex04-05.tex, parte c},label=ex:04-05:c]
\begin{ltxlisting}[caption={ex05-binom.tex, parte c},label=ex05:binom]
Regra de Pascal:
\[ \binom{n}{k} = \binom{n-1}{k} + \binom{n-1}{k-1} \]
\end{ltxlisting}
%\end{LTXexample}
%\end{singlespacing}

\section{Subequações}
 
Podemos usar subequações\index{subequação} no \LaTeX{} como no Exemplo~\ref{ex05:subequacao}.

%\setcounter{lstorichapter}{\value{chapter}}
%%\let\orilabel\label
%\begin{singlespacing}
%\begin{LTXexample}[pos=b,caption={ex04-05.tex, parde d},label=ex:04-05:d,preset={\let\label\orilabel,\setcounter{chapter}{1}}]
\begin{ltxlisting}[caption={ex05-subequacao.tex},label=ex05:subequacao]
\begin{subequations}
\begin{equation}\label{eq:regiao:a}
  x^2+y^2 \le 1
\end{equation}    
\begin{equation}\label{eq:regiao:b}
  y \ge 0
\end{equation}
\end{subequations}
As equações \ref{eq:regiao:a} e  \ref{eq:regiao:b} determinam a parte superior do disco.
\end{ltxlisting}
%\end{LTXexample}
%\end{singlespacing}
%\setcounter{chapter}{\value{lstorichapter}}

\section{Acentuação no modo matemático}

Na matemática, usa a acentuação\index{modo matematico@{modo matemático}!acentuacao@{acentuação}} para produzir nomes relacionados ao original  tais como $\hat f$ e $\bar f$ relacionados a $f$.
Tais acentuações difere do modo texto.
Alguns desses acentuações comumente encontrados na matemática são: acento circunflexo, til, ponto, seta e barra.
Além das acentuações, tem também o caso de colorar expoentes como em $f'$ e $f^\ast$ que são relacionados com $f$. 
O \LaTeX{} permite produzir estes e outros acentos e símbolos usados como expoentes especiais na matemática.

\begin{tabular}{cc}
comando & exemplo \\
\verb+\hat+ & $\hat{a}$ \\
\verb+\tilde+ & $\tilde{a}$ \\
\verb+\dot+ & $\dot{a}$ \\
\verb+\ddot+ & $\ddot{a}$ \\
\verb+\dddot+ & $\dddot{a}$ \\
\verb+\ddddot+ & $\ddddot{a}$ \\
\verb+\bar+ & $\bar{a}$ \\
\verb+\vec+ & $\vec{a}$ \\
\verb+\grave+ & $\grave{a}$ \\
\verb+\acute+ & $\acute{a}$ \\
\verb+\breve+ & $\breve{a}$ \\
\verb+\check+ & $\check{a}$ \\
\verb+\overset{\circ}{}+ & $\overset{\circ}{a}$ \\
\end{tabular}%
\index{modo matematico@{modo matemático}!acentuacao@{acentuação}!\verb+\hat{a}+, $\hat{a}$}%
\index{modo matematico@{modo matemático}!acentuacao@{acentuação}!\verb+\tilde{a}+, $\tilde{a}$}%
\index{modo matematico@{modo matemático}!acentuacao@{acentuação}!\verb+\dot{a}+, $\dot{a}$}%
\index{modo matematico@{modo matemático}!acentuacao@{acentuação}!\verb+\ddot{a}+, $\ddot{a}$}%
\index{modo matematico@{modo matemático}!acentuacao@{acentuação}!\verb+\dddot{a}+, $\dddot{a}$}%
\index{modo matematico@{modo matemático}!acentuacao@{acentuação}!\verb+\ddddot{a}+, $\ddddot{a}$}%
\index{modo matematico@{modo matemático}!acentuacao@{acentuação}!\verb+\bar{a}+, $\bar{a}$}%
\index{modo matematico@{modo matemático}!acentuacao@{acentuação}!\verb+\vec{a}+, $\vec{a}$}%
\index{modo matematico@{modo matemático}!acentuacao@{acentuação}!\verb+\grave{a}+, $\grave{a}$}%
\index{modo matematico@{modo matemático}!acentuacao@{acentuação}!\verb+\acute{a}+, $\acute{a}$}%
\index{modo matematico@{modo matemático}!acentuacao@{acentuação}!\verb+\breve{a}+, $\breve{a}$}%
\index{modo matematico@{modo matemático}!acentuacao@{acentuação}!\verb+\check{a}+, $\check{a}$}%
\index{modo matematico@{modo matemático}!acentuacao@{acentuação}!\verb+\overset{\circ}{a}+, $\overset{\circ}{a}$}%

Alguns deles, como barra, requer alongamento quando tiver mais de um símbolo (ou símbolo grande). 
Neste caso, deverá usar versão ajustáveis:

\begin{tabular}{cc}
comando & exemplo \\
\verb+\widehat+ & $\widehat{abc}$ \\
\verb+\widetilde+ & $\widetilde{abc}$ \\
\verb+\overline+ & $\overline{ab}$ \\
\verb+\overrightarrow+ & $\overrightarrow{AB}$ \\
\verb+\overleftarrow+ & $\overleftarrow{ab}$ \\
\verb+\overleftrightarrow+ & $\overleftrightarrow{AB}$ \\
\end{tabular}%
\index{modo matematico@{modo matemático}!acentuacao@{acentuação}!\verb+\widehat{abc}+, $\widehat{abc}$}%
\index{modo matematico@{modo matemático}!acentuacao@{acentuação}!\verb+\widetilde{abc}+, $\widetilde{abc}$}%
\index{modo matematico@{modo matemático}!acentuacao@{acentuação}!\verb+\overline{abc}+, $\overline{abc}$}%
\index{modo matematico@{modo matemático}!acentuacao@{acentuação}!\verb+\overrightarrow{abc}+, $\overrightarrow{abc}$}%
\index{modo matematico@{modo matemático}!acentuacao@{acentuação}!\verb+\overleftarrow{abc}+, $\overleftarrow{abc}$}%
\index{modo matematico@{modo matemático}!acentuacao@{acentuação}!\verb+\overleftrightarrow{abc}+, $\overleftrightarrow{abc}$}%

As vezes, precisamos colocar embaixo, como sublinhar.

\begin{tabular}{cc}
comando & exemplo \\
\verb+\underbar+ & $\underbar{abc}$ \\
\verb+\underline+ & $\underline{abc}$ \\
\verb+\underrightarrow+ & $\underrightarrow{AB}$ \\
\verb+\underleftarrow+ & $\underleftarrow{ab}$ \\
\verb+\underleftrightarrow+ & $\underleftrightarrow{AB}$ \\
\end{tabular}%
\index{modo matematico@{modo matemático}!acentuacao@{acentuação}!\verb+\underbar{abc}+, $\underbar{abc}$}%
\index{modo matematico@{modo matemático}!acentuacao@{acentuação}!\verb+\underline{abc}+, $\underline{abc}$}%
\index{modo matematico@{modo matemático}!acentuacao@{acentuação}!\verb+\underrightarrow{abc}+, $\underrightarrow{abc}$}%
\index{modo matematico@{modo matemático}!acentuacao@{acentuação}!\verb+\underleftarrow{abc}+, $\underleftarrow{abc}$}%
\index{modo matematico@{modo matemático}!acentuacao@{acentuação}!\verb+\underleftrightarrow{abc}+, $\underleftrightarrow{abc}$}%

Colocar a direita (ou como expoente)

\begin{tabular}{cc}
comando & exemplo \\
\verb+'+ & $f'$ \\
\verb+^\prime+ & $f^\prime$ \\
\verb+^\backprime+ & $f^\backprime$ \\
\verb+^\ast+ & $f^\ast$ \\
\verb+^\circ+ & $20^\circ$ \\
\verb+^llcorner+ & $x^\llcorner$ \\
\verb+^ulcorner+ & $x^\ulcorner$ \\
\verb+^\perp+ & $\vec{v}^\perp$ \\
\end{tabular}%
\index{modo matematico@{modo matemático}!acentuacao@{acentuação}!$f'$}%
\index{modo matematico@{modo matemático}!acentuacao@{acentuação}!\verb+f^\prime+, $f^\prime$}%
\index{modo matematico@{modo matemático}!acentuacao@{acentuação}!\verb+f^\backprime+, $f^\backprime$}%
\index{modo matematico@{modo matemático}!acentuacao@{acentuação}!\verb+f^\ast+, $f^\ast$}%
\index{modo matematico@{modo matemático}!acentuacao@{acentuação}!\verb+f^\circ+, $f^\circ$}%
\index{modo matematico@{modo matemático}!acentuacao@{acentuação}!\verb+f^\llcorner+, $f^\llcorner$}%
\index{modo matematico@{modo matemático}!acentuacao@{acentuação}!\verb+f^\ulcorner+, $f^\ulcorner$}%
\index{modo matematico@{modo matemático}!acentuacao@{acentuação}!\verb+f^\perp+, $f^\perp$}%

Não confundir apóstrofos (usado na derivada $f'$ de $f$, por exemplo), com o acento agudo.
Note que i e j sem o ponto é produzido por \verb+\imath+\index{modo matematico@{modo matemático}!acentuacao@{acentuação}!\verb+\imath+, $\imath$} e \verb+\jmath+\index{modo matematico@{modo matemático}!acentuacao@{acentuação}!\verb+\jmath+, $\jmath$}, respectivamente como em $\hat\imath$ e $\hat\jmath$.

Veja o Exemplo~\ref{ex05:acentos}.

%\begin{singlespacing}
%\begin{LTXexample}[pos=b,caption={ex06-06.tex},label=ex:06-06]
\begin{ltxlisting}[caption={ex05-acentos.tex},label=ex05:acentos]
\[ \hat{f}, \tilde{f}, \dot{x}, \dddot{x}, \bar{y},  \overset{\circ}{X}, \vec{v} \]
\[ \widehat{ABC}, \overrightarrow{AB}, \]
\[ f', f^\ast, 20^\circ, \vec{v}^\perp \]
\end{ltxlisting}
%\end{LTXexample}
%\end{singlespacing}

Na matemática, colocar $/$ sobre o símbolo significa a negação do símbolo.
Apesar de vários símbolos de negação comumente usados estão definidos, pode precisar produzir a negação de alguns símbolos específicos. 
Para produzir tal negação\index{modo matematico@{modo matemático}!negacao@{negação}}, usa-se o comando \verb+\not+\index{modo matematico@{modo matemático}!simbolo@{símbolo}!\verb+\not+}.

Por exemplo, \verb+\not\subset+ produz $\not\subset$.

Veja o Exemplo~\ref{ex05:negacao}.

%\begin{singlespacing}
%\begin{LTXexample}[pos=b,caption={ex06-06.tex},label=ex:06-06]
\begin{ltxlisting}[caption={ex05-negacao.tex},label=ex05:negacao]
Alguns comandos de negação já existentes.
\[\nexists X : X \in X \]
\[\forall X, X \notin X \]
\[ X \neq \{X\} \]
Produzindo com comando \verb+\not+.
\[ X \not\subset Y \implies X \cap Y \neq X\]
\[X : X \not\ni x \]
\[r \not\perp s \]
\end{ltxlisting}
%\end{LTXexample}
%\end{singlespacing}

\clearpage\thispagestyle{empty}\cleardoublepage
\chapter{Definindo Comandos e Ambientes}\label{chap:novo:comando}
Vamos ver o básico da definição de comandos e ambientes

\section{Definindo comandos}\label{sec:comandos}
Quando tem o comando longo a ser digitado, poderá definir um comando\index{definir comandos} que funciona como um atalho. 
Por exemplo, \verb+\mathbb{R}+ toda vez que queremos escrever o conjunto dos números reais é cansativo. 
Definir comando para \verb+\mathbb{R}+ permite também, trocar facilmente pelo outro comando de fontes no lugar de \verb+\mathbb+.
%tais como  \verb+\mathds{R}+ do pacote \texttt{dsfont}, se desejar no futuro.

Para definir um comando, usa-se o \verb+newcommand+\index{newcommand@\verb+\newcommand+}.
Por exemplo, \verb+\newcommand{\Rset}{\mathbb{R}}+ define o comando \verb+\Rset+ que será substituído pelo \verb+\mathbb{R}+ quando é compilado.
Se quer que \verb+\Rset+ funcione tanto dentro como fora das fórmulas, poderá usar o \verb+\ensuremath+\index{ensuremath@\verb+\ensuremath+} como em \verb+\newcommand{\Rset}{\ensuremath{\mathbb{R}}}+.
Note que aqui foi evitado de definir como sendo \verb+\real+, pois o comando \verb+\natural+ já está definido no \LaTeX.

O nome do comando não deve ter números. 
Por exemplo, \verb+\newcommand{\R2}{$\mathbb{R}^2$}+ resulta em erros, pois o nome do comando \verb+\R2+ contém o número.

Analogamente, \verb+\newcommand{\sen}{\mathrm{sen}}+ define o comando para função seno.
Em geral, para definir nome da função, usa-se o \verb+\DeclaremathOperator+\index{DeclareMathOperator@\verb+\DeclareMathOperator+} do pacote \texttt{amsmath} tal como \verb+\DeclaremathOperator{\sen}{sen}+ 
em vez do \verb+\newcommand+.
Ele coloca automaticamente o \verb+\mathrm+ no segundo argumento.

O comando \verb+\newcommand{\argmin}{\mathop{\mathrm{arg\,min}}}+ define o comando \verb+\argmin+ para função \texttt{arg min}, mas por ter \verb+\mathop+\index{mathop@\verb+\mathop+}, o limitante será colocado embaixo em vez de ser colocado como índice. Note o uso de \verb+\,+ para inserir pequeno espaço entre \texttt{arg} e \texttt{min}.
Para facilitar, o pacote \texttt{amsmath} define a versão com ``*'' de \verb+\DeclaremathOperator+\index{DeclareMathOperator*@\verb+\DeclareMathOperator*+} que coloca o nome da função dentro do \verb+\mathrm+ e \verb+\mathop+.

Quando já existe o comando, poderá redefinir usando o \verb+\renewcommand+\index{renewcommand@\verb+\renewcommand+}. 

Se não sabe se existe o comando, poderá usar o \verb+\providecommand+\index{providecommand@\verb+\providecommand+} que define o comando caso não existir.

Estas definições de comandos costumam ficar no \texttt{preamble} do documento. Veja o Exemplo~\ref{ex06:newcommand}.
 
%\let\mathcal\amsmathcal
%\let\mathbb\amsmathbb
%\begin{singlespacing}
%\begin{LTXexample}[pos=b,caption={ex04-01.tex},label=ex:04-01]
\begin{ltxlisting}[caption={ex06-newcommand.tex},label=ex06:newcommand]
\documentclass[12pt,a4paper]{article}
\usepackage[T1]{fontenc}
\usepackage[utf8]{inputenc}
\usepackage{amsmath,amssymb}
\newcommand{\Rset}{\ensuremath{\mathbb{R}}}
%\DeclareMathOperator{\sen}{sen} % modo amsmath
\newcommand{\sen}{\mathrm{sen}} % modo normal (sem amsmath)
%\DeclareMathOperator*{\argmin}{argmin} % modo amsmath
\newcommand{\argmin}{\mathop{\mathrm{arg\,min}}} % modo sem amsmath
\begin{document}
\[ \forall \theta \in \Rset, \cos^2\theta + \sen^2\theta = 1 \]
\[ \argmin_x f(x) = \left\{x: f(x) = \min_{x'} f(x') \right\} \]
\end{document}
\end{ltxlisting}
%\end{LTXexample}
%\end{singlespacing}
%\let\mathcal\orimathcal
%\let\mathbb\orimathbb

Os comandos podem ter até 9 parâmetros.
Os comandos com parâmetros\index{comandos!com parâmetros}, é criado pelo comando \verb+\newcommand{\nome-do-comando}[n]{definição}+ onde \texttt{n} é o número de parâmetros.
No corpo da definição do comando, \verb+#1+, \verb+#2+,\verb+#3+, \ldots especificam os parâmetros na posição  $1$, $2$, $3$, etc.

Quando usa os parâmetros, o primeiro deles podem ser opcional\index{comandos!parâmetro opcional}.
Neste caso, usa-se a sintaxe
\verb+\newcommand{\nome-do-comando}[n][v]{definição}+
onde ``\texttt{v}'' é o valor padrão quando o parâmetro for omitido.
O Exemplo~\ref{ex06:parametro:opcional} ilustra o caso.

%\begin{singlespacing}
%\begin{LTXexample}[pos=b,caption={ex04-01.tex, parte b},label=ex:04-01:b]
\begin{ltxlisting}[caption={ex06-parametro-opcional.tex},label=ex06:parametro:opcional]
\documentclass[12pt,a4paper]{article}
\usepackage[T1]{fontenc}
\usepackage[utf8]{inputenc}
\usepackage{amsmath,amssymb}
\newcommand{\conj}[1]{\overline{#1}} % com um parametro
\newcommand{\norma}[2][]{\left\|#2\right\|_{#1}} % com parametro opcional
\newcommand{\seq}[2][n]{\left\{#2_0,\ldots,\,#2_{#1}\right\}}
\begin{document}
\[ \conj{z}\cdot z = \norma{z}^2 \neq \norma[\infty]{z}^2 \]
A sequência $\seq{x}$ para $n=5$ é $\seq[5]{x}$.

${\conj zw}$
\end{document}
\end{ltxlisting}
%\end{LTXexample}
%\end{singlespacing}
 
O comando \verb+\let+\index{let@\verb+\let+} cria uma copia do comando existente.
Por exemplo, \verb+\let\comandocopia\comandooriginal+ cria um comando \verb+\comandocopia+ que é uma cópia do \verb+\comandooriginal+.
Assim, podemos redefinir o \verb+\comandooriginal+ e se precisar do original, é só chamar o \verb+\comandocopia+.
Também poderá restaurar o comando original usando o \verb+\let+.
Por exemplo,

\begin{lstlisting}[language={[LaTeX]TeX}]
 \let\tanoriginal\tan
 \renewcommand{\tan}{\mathrm{tg}}
\end{lstlisting}
 
redefine o comando \verb+\tan+, mas se precisar do \verb+\tan+ original, eh só chamar o \verb+\tanoriginal+.
Para restaurar o original, é só usar o \verb+\let+ novamente com em \verb+\let\tan\tanoriginal+.

\section{Criando ambientes}
Para trecho maior de dados, o ambiente\index{ambiente} (o que tem \verb+\begin+ e \verb+\end+) é mais adequado que os comandos.
Para criar ambientes, usa-se o comando \verb+\newenvironment+\index{newenvironment@\verb+\newenvironment+} na qual o primeiro parâmetro é 
nome do ambiente, segundo é o que vai fazer antes e terceiro é o que vai fazer depois.
Quando tem o parâmetro\index{ambiente!parâmetro} (que pode ter até nove), coloca o número de parâmetro como parâmetro opcional de \verb+\newenvironment+, entre primeiro e segundo parâmetro.
Assim como o comando, o ambiente também pode ter o primeiro parâmetro como opcional\index{ambiente!parâmetro opcional}.
Neste caso, passa o seu valor padrão após número de parâmetro como sendo parâmetro opcional ao \verb+\newenvironment+.

No Exemplo~\ref{ex06:ambiente}, o ambiente \texttt{dem} foi definido somente como a ilustração.
Em geral, usa o ambiente \texttt{proof}\index{proof} do pacote \texttt{amsthm} para demonstrações.

%\begin{singlespacing}
%\begin{LTXexample}[pos=b,caption={ex04-02.tex},label=ex:04-02]
\begin{ltxlisting}[caption={ex06-ambiente.tex},label=ex06:ambiente]
\documentclass[12pt,a4paper]{article}
\usepackage[T1]{fontenc}
\usepackage[utf8]{inputenc}
\usepackage{amsmath,amssymb}
\newenvironment{myenv}{\begin{center}\em} {\end{center}}
\newenvironment{mypar}[1]{\par\noindent \hrulefill\fbox{ #1 }\hrulefill \\ } {\par\noindent\hrulefill \\}
\newenvironment{dem}[1][Demonstração]{\textbf{#1:}\ }  {\hfill\rule{1ex}{1ex}\par}
\begin{document}
\begin{myenv}
Texto enfatizado e centralizado.
\end{myenv}
\begin{mypar}{Teste}
Parágrafo com título simples.
\end{mypar}
\begin{dem}
\ldots
\end{dem}
\begin{dem}[Prova]
\ldots
\end{dem}
\end{document}
\end{ltxlisting}
%\end{LTXexample}
%\end{singlespacing}

Como não podemos clocar linhas em branco no meio da definição de comandos ou ambientes para indicar o parágrafo, usa-se o comando \verb+\par+\index{par@\verb+\par+} para esta finalidade.
O comando \verb+\noindent+\index{noindent@\verb+\noindent+} desabilita temporariamente a indentação (tabulação para direita) da primeira linha do parágrafo.

Note que, também pode definir os comandos no modo \TeX{} em vez do modo \LaTeX.
Para isso, basta usar o \verb+\def+ que é importante quando um comando define o outro. 
O \verb+\def+ que não vamos discutir aqui, deve ser usado com cuidado, pois ele não verifica se o comando exite ou não, redefinindo caso existir.

\section{Quebrando o código em várias linhas}

Em geral, a definição de comando ou ambiente deve ficar em uma única linha, mas quando isto é longo, precisamos dividir em mais de uma linha\index{comando!múltiplas linhas}\index{ambiente!múltiplas linhas} para boa organização. 
Neste caso, existe trechos que não podem ser quebrados, mas isto requer um conhecimento extra.
Para contornar isto, note que o \LaTeX{} considera que a próxima linha é uma continuação da linha atual quando encontrar ``\%'' grudado na última letra ativa (que não seja comentário).
Assim, quando quebrar a definição de comandos ou ambiente em mais de uma linha, coloque o ``\%'' grudado na última letra. 
Fique atento de não colocar espaço entre a última letra e ``\%''.
Veja o Exemplo~\ref{ex06:comando:multilinhas}.

\begin{singlespacing}
%\begin{lstlisting}[caption={ex04-02.tex, parte b},label=ex:04-02:b]
\begin{lstlisting}[caption={ex06-comando-multlinhas.tex},label=ex06:comando:multilinhas]
\newcommand{\norma}[2][]{%
  \left\|#2\right\|_{#1}%
}
\newenvironment{mypar}[1]%
  {\par\noindent \hrulefill\fbox{ #1 }\hrulefill \\}% o que faz antes
  {\par\noindent\hrulefill \\}% o que faz depois
\end{ltxlisting}
%\end{lstlisting}
\end{singlespacing}

\clearpage\thispagestyle{empty}\cleardoublepage
\chapter{Divisão Lógica de Documentos}
Neste capítulo, vamos estudar a divisão lógica de documentos.

\section{Capítulos, seções e similares}
Nos livros e relatórios, os conteúdos são organizados em capítulos\index{capítulo}, indicado por \verb+\chapter+\index{chapter@\verb+\chapter+}.
Os capítulos podem ser divididos em seções\index{seção} indicado pelo comando \verb+\section+\index{section@\verb+\section+}, se assim desejar. 
Neste caso, evite ter capítulos com seções e outro sem seções, para ter uniformidades.

Se a seção ficar grandes, podem ser subdivididos em subseções\index{subseção} com o comando \verb+subsection+\index{subsection@\verb+\subsection+}, mas tome cuidado para que todas seções tenham subseções.

Existem ainda, \verb+\subsubsection+\index{subsubsection@\verb+\subsubsection+} (subsubseções), \verb+\paragraph+\index{paragraph@\verb+\paragraph+} (parágrafos\index{parágrafo}) e \verb+\subparagraph+\index{subparagrapg@\verb+\subparagraph+} (subparágrafos\index{subparágrafo}) que são menos usados.

No caso de artigos, o conteúdo costuma ser divido em seções e caso desejar, seções podem ser subdivididos em subseções (evitando que tenha seção com subseção e outro sem subseção).

O capítulo inicia-se com o comando \verb+\chapter{Título}+ e seções com o comando \verb+\section{Título}+.

Estes comandos aceitam a opção de especificar os ``títulos curtos''\index{título curto} que são usados no sumário e cabeçalho, o que podem ser passados como parâmetro opcional colocado antes do título.
Neste formato, usado quando título é longo, tem a forma \verb+\chapter[Título Curto]{Título}+ e \verb+\section[Título Curto]{Título}+

%\setcounter{lstorichapter}{\value{chapter}}
%\setcounter{lstorisection}{\value{section}}
%\setcounter{chapter}{0}
%\begin{singlespacing}
%\begin{LTXexample}[pos=b,caption={ex05-01.tex},label=ex:05-01,preset={\setcounter{chapter}{0},\let\addcontentsline\emptyaddcontentsline}]
\begin{ltxlisting}[caption={ex07-capitulo.tex},label=ex07:capitulo]
\chapter[Titulo Curto]{Título Longo do Capítulo}
Apresentação bem rápida do capítulo.
\section[Título curto]{Título longo da seção}
Texto da seção
\section{Título da outra seção}
Texto da outra seção
\ldots
\end{ltxlisting}
%\end{LTXexample}
%\end{singlespacing}
%\setcounter{chapter}{\value{lstorichapter}}
%\setcounter{section}{\value{lstorisection}}
%% \addtocounter{chapter}{-1}
%\let\addcontentsline\oriaddcontentsline

Quando há os capítulos e/ou seções, o sumário\index{sumário} podem ser produzidos automaticamente pelo comando \verb+\tableofcontents+\index{tableofcontents@\verb+\tableofcontents+}.
Para que o sumário apareça devidamente no documento final, precisa compilar o documento duas vezes, pois o \LaTeX{} usa o arquivo auxiliar para isso.

Para comandos do tipo capítulos (\verb+\chapter+, \verb+\section+, etc), existem as versões com ``*''\index{chapter*@\verb+\chapter*+}\index{section*@\verb+\section*+} que não serão enumeradas e não serão colocadas no sumário, apesar de ter a mesma formatação.
Por exemplo, \verb+\chapter*{Resumo}+ não será enumerado, nem vai no sumário.

Note que, no caso de artigos, o resumo tem formatação diferente do  restante das seções.
Assim, existe o ambiente especial \texttt{abstract}\index{abstract} para o resumo, o que não é mesmo que \verb+\section*+.

Note a diferença e similaridade do Exemplo~\ref{ex07:capitulo:star} que usa a versão ``*'', com o Exemplo~\ref{ex07:capitulo} que usa a versão normal.

%\begin{singlespacing}
%\begin{LTXexample}[pos=b,caption={ex05-01.tex, parte b},label=ex:05-01:b]
\begin{ltxlisting}[caption={ex07-capitulo-star.tex},label=ex07:capitulo:star]
\chapter*{Resumo}
Resumo aqui.
\end{ltxlisting}
%\end{LTXexample}
%\end{singlespacing}

\section{Capa, conteúdo frontal e principal}

O Exemplo~\ref{ex07:matter} ilustra a estrutura básica de um documentos tipo livro na qual não vamos colocar a saída aqui, por ocupar várias páginas.

\begin{singlespacing}
%\begin{lstlisting}[caption={ex05-01.tex, parte c},label=ex:05-01:c]
\begin{lstlisting}[caption={ex07-matter.tex},label=ex07:matter]
\documentclass[12pt,a4paper]{book}
\usepackage[T1]{fontenc}
\usepackage[utf8]{inputenc}
\usepackage{amsmath,amssymb}
\pagestyle{empty}
\begin{document}
% capa
\frontmatter
\chapter*{Resumo}\thispagestyle{empty}
Resumo aqui.
\pagestyle{headings}
\tableofcontents % Sumário
\chapter{Prefácio}
Apresentação do trabalho.
\mainmatter
\chapter{Título do Capítulo Aqui}
Apresentação bem rápida do capítulo.
\section{Títuo da seção aqui}
Texto da seção
\ldots
\appendix % se existir apêndice
\chapter{Título do Apêndice 1} % se existir 
Texto do apêndice 1
\ldots
\backmatter % opcional
% referencia biliografica
% indice remissivo, se existir
\end{document}
\end{lstlisting}
\end{singlespacing}

O comando \verb+\pagestyle{empty}+\index{patestyle@\verb+\pagestyle+!empty} antes do \verb+\begin{document}+ remove as enumerações das páginas para que parte inicial do livro fiquem sem a enumeração.

A capa simples nos livros e relatórios podem ser produzidos pelo comando \verb+\maketitle+\index{maketitle@\verb+\maketitle+}, desde que seja fornecido algumas informações tais como título, autor e data pelos comandos \verb+\title+\index{title@\verb+\title+}, \verb+\author+\index{author@\verb+\author+} e \verb+\date+\index{date@\verb+\date+}, colocados antes do \verb+\maketitle+.
Veja Exemplo~\ref{ex07:maketitle}.

\begin{singlespacing}
\begin{lstlisting}[caption={ex07-maketitle.tex},label=ex07:maketitle]
\documentclass[12pt,a4paper]{book}
\usepackage[T1]{fontenc}
\usepackage[utf8]{inputenc}
\usepackage{amsmath,amssymb}
\title{Exemplo}
\author{Sadao Massago}
\date{Fevereiro de 2018}
\begin{document}
\maketitle
\frontmatter
\ldots
\end{document}
\end{lstlisting}
\end{singlespacing}

Note que, \verb+\date+, quando não especificado, será assumido como sendo \verb+\today+\index{today@\verb+\today+} que é data de compilação do documento. Se não quer que apareça data, deve colocar \verb+\date{}+.

Para criar títulos personalizados nos livros e relatórios, use o ambiente \texttt{titlepage}\index{titlepage}.
Note que nos livros e relatórios, a capa não será contada como páginas, mas todas as outras páginas, mesmo totalmente em branco, serão contadas.

O comando \verb+\frontmatter+\index{frontmatter@\verb+\frontmatter+} disponível somente para livros, indica que será matéria frontal\index{matéria!frontal}, tendo as páginas enumeradas em romano minúsculo e sem a contagem de capítulos (mesmo para versão sem ``*'').

O comando \verb+\thispagestyle{empty}+\index{thispagestyle@\verb+\thispagestyle+!empty} no \verb+\chapter*{Resumo}+ remove a enumeração da página atual (primeira página do ``Resumo''). 
Note que capítulos usa o estilo de página \texttt{plain}\index{estilo da página!plain} que enumera embaixo da página, mesmo que o estilo de página esteja em \texttt{empty}. 
Assim, colocar \verb+\thispagestyle{empty}+ é necessário para remover paginação desta primeira página do ``capítulo''.

Antes do sumário\index{sumário} colocado pelo comando \verb+\tableofcontents+\index{tableofcontents@\verb+\tableofcontents+}, foi colocado \verb+\pagestyle{headings}+\index{pagestyle@\verb+\pagestyle+!headings} para que volte a colocar páginas e cabeçalho superior. 
Note que, no livro, o que vem antes do sumário não devem ser paginados, mas o que vem depois do sumário, costuma ser paginados. 
Por ser \texttt{frontmatter}, a paginação será em romano minusculo.

Depois do prefácio, tem o comando \verb+\mainmatter+\index{mainmatter@\verb+\mainmatter+} que também é disponível somente para livros. 
Este comando reinicia a paginação (começa a contar de 1 novamente) e usa a enumeração em arábico. 
Também faz começar a contar capítulos.

O comando \verb+\appendix+\index{appendix@\verb+\appendix+} indica que, o que segue são apêndices\index{apêndice} que complementa o documento. 
Ele é disponível para livros, relatórios e artigos.
No caso de livros e relatórios, o comando \verb+\chapter+ começará a produzir título para apêndices em vez de capítulos (escreverá Apêndice A, Apêndice B, etc em vez de Capítulo 1, Capítulo 2, etc). 
No caso de artigos, o comando \verb+\section+ que produzirá os apêndices.
 
O comando \verb+\backmatter+\index{backmatter@\verb+\backmatter+} também é disponível somente para livros. 
O \verb+\backmatter+ faz com que \verb+\chapter+ funcione igual a \verb+\chapter*+ (versão ``*''). 
Como não é costume colocar \verb+\chapter+ no \texttt{backmatter}, o comando pode ser omitido.

Em geral, a parte final do documento é a referência bibliográfica. 
Se existir o índice remissivo, deverá ser colocado depois da referência bibliográfica.

Note que o relatório é um documento e não é um livro. 
Assim, todas páginas exceto a capa, devem estar enumeradas em arábico e em sequencias. 
Assim, não há divisão de \texttt{frontmatter} e \texttt{mainmatter}, não existindo comandos correspondentes.

A configuração do documento no artigo é ilustrado no Exemplo~\ref{ex07:article}. % que também omitiremos a sua saída final.

%\setcounter{lstorichapter}{\value{chapter}}
%\setcounter{lstorisection}{\value{section}}
%% \let\orilabel\label
\let\orithesection\thesection
\renewcommand{\thesection}{\arabic{section}}
\newcommand{\articleappendix}{\setcounter{section}{0}\def\thesection{\Alph{section}}}
%\begin{singlespacing}
%\begin{LTXexample}[pos=b,caption={ex05-01.tex, parte e},label=ex:05-01:e,preset={\setcounter{chapter}{1},\setcounter{section}{0},\let\addcontentsline\emptyaddcontentsline,\let\title\orititle,\let\author\oriauthor,\let\date\oridate,\let\maketitle\makearticletitle,\let\appendix\articleappendix}]
\begin{ltxlisting}[caption={ex07-article.tex},label=ex07:article,preset={\let\maketitle\makearticletitle\let\appendix\articleappendix}]
\documentclass[12pt,a4paper]{article}
\usepackage[T1]{fontenc}
\usepackage[utf8]{inputenc}
\usepackage[english,brazil]{babel}
\usepackage{amsmath,amssymb}
\title{Exemplo}
\author{Sadao Massago}
\date{Fevereiro de 2018}
\begin{document}
\maketitle
\begin{abstract}
Resumo aqui.
\end{abstract}
% \tableofcontents % so se for artigo longo
\section{Título da seção 1}
Texto da seção 1
\section{Título da seção 2}
Texto da seção 2
\ldots
\appendix % se tiver apêndice
\section{Título do Apêndice 1}
Texto do apêndice 1
\ldots
% referência biliográfica
\end{document}
\end{ltxlisting}
%\end{LTXexample}
%\end{singlespacing}
%%\let\ref\oriref
%\setcounter{chapter}{\value{lstorichapter}}
%\setcounter{section}{\value{lstorisection}}
%% \addtocounter{chapter}{-1}
\let\thesection\orithesection
%\let\addcontentsline\oriaddcontentsline


%\begin{singlespacing}
%\begin{lstlisting}[caption={ex05-01.tex, parte e},label=ex:05-01:e]
%\documentclass[12pt,a4paper]{article}
%\usepackage[T1]{fontenc}
%\usepackage[utf8]{inputenc}
%\usepackage{amsmath,amssymb}
%\title{Exemplo}
%\author{Sadao Massago}
%\date{Fevereiro, 2018}
%\begin{document}
%\maketitle
%\begin{abstract}
%Resumo aqui.
%\end{abstract}
%% \tableofcontents % se for artigo longo
%\section{Título da Seção 1}
%Texto da seção 1
%\section{Título da Seção 2}
%Texto da seção 2
%\ldots
%\appendix % se tiver apêndice
%\section{Título do apêndice 1}
%Texto do apêndice 1
%\ldots
%% referência biliográfica
%\end{document}
%\end{lstlisting}
%\end{singlespacing}

O \verb+\maketitle+\index{maketitle@\verb+\maketitle+} produz título do artigo com informações usando título, autor e data fornecidos como nas capas dos livros e relatórios.

Note que não costumamos usar o índice remissivo para artigos, apesar de ter comandos disponíveis para isso.

\section{Limpando o verso das páginas}
O livro costuma ser diagramado para impressão em dois lados (opção da classe de documento \texttt{twoside}\index{twoside}), embora poderá optar em \texttt{oneside}\index{oneside} que é um lado, se quiser.
Também podemos usar \texttt{openright}\index{openright} que começa o capítulo somente no lado direito (\texttt{openany}\index{openany} começa o capítulo tanto no lado esquerdo como no lado direito).

Nestes casos, o verso do sumário, resumo, etc, assim como versos de finais dos capítulos, se existir, ficarão em branco em vez de ter enumeração de páginas ou cabeçalhos.
Para que isso aconteça, é só inserir o código
\begin{verbatim}
\clearpage\thispagestyle{empty}\cleardoublepage
\end{verbatim}
antes de cada capítulo ou comandos que produzem os capítulos ou similares, tais como sumário, referências bibliográficas e índice remissivo. 
Também é necessário antes do \verb+\mainmatter+. 
Veja o Exemplo~\ref{ex07:clearpage} na qual a saída foi omitida.

\begin{singlespacing}
\begin{lstlisting}[caption={ex07-clearpage.tex},label=ex07:clearpage]
\documentclass[12pt,a4paper,openright]{book}
\usepackage[T1]{fontenc}
\usepackage[utf8]{inputenc}
\usepackage{amsmath,amssymb}
\pagestyle{empty}
\begin{document}
% capa
\frontmatter
\chapter*{Resumo}\thispagestyle{empty}
Resumo aqui.
\cleardoublepage
\pagestyle{headings}
\tableofcontents % Sumário
\clearpage\thispagestyle{empty}\cleardoublepage
\chapter{Prefácio}
Apresentação do trabalho.
\clearpage\thispagestyle{empty}\cleardoublepage
\mainmatter
\chapter{Título do Capítulo 1}
Texto do primeiro capítulo.
\clearpage\thispagestyle{empty}\cleardoublepage
\chapter{Título do Capítulo 2}
Texto do segundo capítulo.
\ldots
\clearpage\thispagestyle{empty}\cleardoublepage
\appendix % se existir apêndice
\chapter{Título do Apêndice 1} % se existir 
Texto do apêndice 1
\ldots
\backmatter % opcional: sem efeito visual
\clearpage\thispagestyle{empty}\cleardoublepage
% referencia biliografica
\clearpage\thispagestyle{empty}\cleardoublepage
% indice remissivo, se existir
\end{document}
\end{lstlisting}
\end{singlespacing}

O comando \verb+\clearpage+\index{clearpage@\verb+\clearpage+} finaliza a página. 
Por exemplo, se tiver algo pendente como figuras a serem colocadas, será feito.
Com \verb+\thispagestyle{empty}+\index{thispagestyle@\verb+\thispagestyle+!empty}, desabilita o estilo da página atual e com \verb+\cleardoublepage+\index{cleardoublepage@\verb+\cleardoublepage+}, inserir página em branco, se necessário. 

\section{Efetuando pequenos ajustes}
O primeiro parágrafo da seção ou capítulo não será indentado (empurrado para direita).
A indentação é feito para distinguir um parágrafo do outro, mas como não tem parágrafo anterior, não há necessidades de indentação.
No entanto, algumas pessoas podem querer indentar o primeiro parágrafo.
Se for o caso, basta usar o pacote \texttt{indentfirst}\index{indentfirst} no \texttt{preamble} do documento.

Por outro lado, se quer remover indentação de algum parágrafo como o que segue uma fórmula matemática no modo \texttt{displaystyle}, basta usar o comando \verb+\noindent+\index{noindent@\verb+\noindent+}. 
%Também pode inserir a indentação, usando o comando \verb+\indent+, exceto nos lugares proibidos como o primeiro parágrafo.

Também pode querer que uma página fique um pouco maior para acomodar o conteúdo atual.
Neste caso, use o comando \verb+\enlargethispage+\index{enlargethispage@\verb+\enlargethispage+}. 
Por exemplo, \verb+\enlargethispage{\baselineskip}+ aumenta a página atual por uma linha.
Este tipo de ajustes requer cuidados para não comprometer a qualidade da diagramação.

Se quer que algum comando seja executado após o término da página atual, use o pacote \texttt{afterpage}\index{afterpage} que implementa o comando de mesmo nome.
Por exemplo, \verb+\afterpage{\clearpage}+ colocará todas figuras e tabelas pendentes na próxima página.

\clearpage\thispagestyle{empty}\cleardoublepage
\chapter{Teoremas e Similares}
Os ambientes para teoremas e similares podem ser criados de forma apropriada, facilmente no \LaTeX.

\section{Criando ambiente para teoremas}

Para definir o ambiente para escrever teoremas\index{teoremas} e similares, 
costuma usar o pacote \texttt{amsthm}\index{amsthm} que melhora a usabilidade do \verb+\newtheorem+\index{newtheorem@\verb+\newtheorem+} usados para este propósito.

O comando \verb+\newtheorem{theorem}{Teorema}[chapter]+ cria um ambiente chamado \texttt{theorem} que usa como título ``Teorema'' com contador vinculado a capítulo (\texttt{chapter}).
Se for artigo, troque o \texttt{chapter} pelo \texttt{section} como em
\verb+\newtheorem{theorem}{Teorema}[section]+.

Para definir outros ambiente similares aos teoremas, tais como axioma, lema, corolário e proposição, poderá usar o comando tal como \verb+\newtheorem{proposition}[theorem]{Proposição}+ onde \texttt{theorem} no parâmetro opcional na segunda posição indica que será contado junto com o teorema. 
Asim ficará Teorema~1.1, Proposição~1.2, etc. 
O modelo para proposição serve para demais ambientes similares aos teoremas.

Para ambientes como definição, exemplos e exercícios, devemos usar o estilo definição, obtido pelo comando \verb+\theoremstyle{definition}+\index{theoremstyle@\verb+\theoremstyle+!definition}.

A seguir, usa-se o \verb+\newtheorem+ novamente como \verb+\newtheorem{definition}[theorem]{Definição}+ e \verb+\newtheorem{example}[theorem]{Exemplo}+.

Para observações e notas, deve mudar para estilo \texttt{remark} com o comando \verb+\theoremstyle{remark}+\index{theoremstyle@\verb+\theoremstyle+!remark}.

Note que \texttt{amsthm} providencia o ambiente para demonstrações, denominado de \texttt{proof}\index{proof}.
Veja o Exemplo~\ref{ex08:teorema}.
%\setcounter{lstorichapter}{\value{chapter}}
%\setcounter{lstorisection}{\value{section}}
%\begin{singlespacing}
%\begin{LTXexample}[pos=b,caption={ex06-05.tex},label=ex:06-05,preset={\setcounter{chapter}{0},\let\addcontentsline\emptyaddcontentsline}]

%\makeatletter
% % exercise env from xsim
% \let\originalexercise\exercise
% \let\originalendexercise\endexercise
% \let\c@exerciseoriginal\c@exercise
% % undefine env
% \let\exercise\@undefined
% \let\endexercise\@undefined
% \let\c@exercise\@undefined
%\makeatother
\begin{ltxlisting}[caption={ex08-teorema.tex},label=ex08:teorema]
\documentclass[12pt,a4paper]{book}
\usepackage[T1]{fontenc}
\usepackage[utf8]{inputenc}
\usepackage[brazil]{babel}
\usepackage{amssymb,amsmath}
\usepackage{amsthm} % para configurar o teorema
% Definindo teormas e similares. Contador unico, vinculado a capítulos.
\newtheorem{theorem}{Teorema}[chapter] % contador vinculado a capitulos
\newtheorem{corollary}[theorem]{Corolário}
\newtheorem{lemma}[theorem]{Lema}
\newtheorem{proposition}[theorem]{Proposição}
\newtheorem{axiom}[theorem]{Axioma}
\theoremstyle{definition}
\newtheorem{definition}[theorem]{Definição}
\newtheorem{example}[theorem]{Exemplo}
\newtheorem{exercise}{Exercício}[chapter] % contador próprio, vinculado a capitulo
\theoremstyle{remark}
\newtheorem{remark}[theorem]{Observação}
\begin{document}
\chapter{Triângulo Equilátero}
\begin{definition}
  Um triângulo é dito \emph{triângulo equilátero} quando todos os lados forem congruentes.
\end{definition}
\begin{theorem}
  Todo triângulo equilátero se, e somente se, todos ângulos forem congruentes.
\end{theorem}
\begin{proof}
\ldots
\end{proof}
\begin{remark}
Um triângulo cuja todos ângulos são congruentes é dito \emph{equiângulo}.
\end{remark}
\end{document}
\end{ltxlisting}
%\end{LTXexample}
%\end{singlespacing}
\theoremstyle{plain}
%\setcounter{chapter}{\value{lstorichapter}}
%\setcounter{section}{\value{lstorisection}}
%\let\addcontentsline\oriaddcontentsline
%\makeatletter
% % restore exercise env of xsim
% \let\exercise\originalexercise
% \let\endexercise\originalendexercise
% \let\c@exercise\c@exerciseoriginal
%\makeatother

\section{Parâmetros opcionais}

Note que os ambientes tipo teoremas permite colocar o título como parâmetro opcional do ambiente.
Isto também vale para o ambiente \texttt{proof} quando existem textos entre o enunciado do teorema e a sua prova.
O comando \verb+\proofname+\index{proofname@\verb+\proofname+} armazena o ``nome'' do ambiente \texttt{proof} que o valor atual é ``\texttt{\proofname}''.
Quando a demonstração é finalizada pela equação no modo \texttt{displaystyle}, a marca do final de demonstração fica na linha de baixo.
Para corrigir isso, existe o comando \verb+\qedhere+\index{qedhere@\verb+\qedhere+} que indica onde a marca de final de demonstração deve ser colocada.
Veja o Exemplo~\ref{ex08:teorema:parametro}.

% Para mudar a marca do final de demonstração, redefina o comando \qedsymbol
%\renewcommand{\qedsymbol}{cqd.}


% showexpl juntamente com teoremas e referencias, causa erro na segunda compilacao.
% Nao sei porque, pois na equacao com referencias,
% ou teoremas sem referencias funcionam bem.
% Usando lstlisting e minipage para ter mesmo effeito de showexpl

%\setcounter{lstorichapter}{\value{chapter}}
%%\setcounter{lstorisection}{\value{section}}
%\begin{singlespacing}
%\begin{LTXexample}[pos=b,caption={ex06-05.tex, parte b},label=ex:06-05:b,preset={\let\label\orilabel,\setcounter{chapter}{0}}]
%\begin{definition}
%Um triângulo é dito \emph{triângulo retângulo} quando tem um ângulo reto. O lado oposto ao ângulo reto é denominado de \emph{hipotenusa} e outros dois lados são denominados de \emph{catetos}.
%\end{definition}
%\begin{theorem}[Pitágoras]\label{thm:pitagoras}
%  Sejam $\Delta ABC$, um triângulo retângulo onde $a$ é hipotenusa. Então
%  \[ a^2=b^2+c^2 \] 
%\end{theorem}
%\begin{remark}
%A reciproca do Teorema~\ref{thm:pitagoras} também é verdadeira.
%\end{remark}
%\begin{proof}[\proofname{} do Teorema~\ref{thm:pitagoras}]
%\ldots
%Assim, temos que
%\[ a^2 + b^2 = c^2 \qedhere \]
%\end{proof}
%\end{LTXexample}
%\end{singlespacing}
%\setcounter{chapter}{\value{lstorichapter}}
%%\setcounter{section}{\value{lstorisection}}

%\setcounter{lstorichapter}{\value{chapter}}
%\setcounter{lstorisection}{\value{section}}
%\let\orilabel\label
%\begin{singlespacing}
%% \begin{lstlisting}[caption={ex06-05.tex, parte b},label=ex:06-05:b]
%\begin{LTXexample}[pos=b,caption={ex06-05.tex, parte b},label=ex:06-05:b,preset={\let\label\orilabel,\setcounter{chapter}{1}}]
\begin{ltxlisting}[caption={ex08-teorema-parametro.tex},label=ex08:teorema:parametro,preset={\setcounter{chapter}{1}}]
\begin{definition}
Um triângulo é dito \emph{triângulo retângulo} quando tem um ângulo reto. O lado oposto ao ângulo reto é denominado de \emph{hipotenusa} e outros dois lados são denominados de \emph{catetos}.
\end{definition}
\begin{theorem}[Pitágoras]\label{thm:pitagoras}
  Sejam $\Delta ABC$, um triângulo retângulo onde $a$ é hipotenusa. Então
  \[ a^2=b^2+c^2 \] 
\end{theorem}
\begin{remark}
A reciproca do Teorema~\ref{thm:pitagoras} também é verdadeira.
\end{remark}
\begin{proof}[\proofname{} do Teorema~\ref{thm:pitagoras}]
\ldots
Assim, temos que
\[ a^2 + b^2 = c^2 \qedhere \]
\end{proof}
\end{ltxlisting}
%\end{LTXexample}
%%\end{lstlisting}
%\end{singlespacing}
%\setcounter{chapter}{\value{lstorichapter}}
%\setcounter{section}{\value{lstorisection}}

%\noindent
%\setcounter{lstorichapter}{\value{chapter}}
%\setcounter{lstorisection}{\value{section}}
%\fbox{\begin{minipage}{\linewidth}
%\begin{singlespacing}
%\setcounter{chapter}{1}
%\begin{definition}
%Um triângulo é dito \emph{triângulo retângulo} quando tem um ângulo reto. O lado oposto ao ângulo reto é denominado de \emph{hipotenusa} e outros dois lados são denominados de \emph{catetos}.
%\end{definition}
%\begin{theorem}[Pitágoras]\label{thm:pitagoras}
%  Sejam $\Delta ABC$, um triângulo retângulo onde $a$ é hipotenusa. Então
%  \[ a^2=b^2+c^2 \] 
%\end{theorem}
%\begin{remark}
%A reciproca do Teorema~\ref{thm:pitagoras} também é verdadeira.
%\end{remark}
%\begin{proof}[\proofname{} do Teorema~\ref{thm:pitagoras}]
%\ldots
%Assim, temos que
%\[ a^2 + b^2 = c^2 \qedhere \]
%\end{proof}
%\end{singlespacing}
%\end{minipage}} 
%\setcounter{chapter}{\value{lstorichapter}}
%\setcounter{section}{\value{lstorisection}}
 
\clearpage\thispagestyle{empty}\cleardoublepage
\chapter[Figuras, Tabelas e Imagens Externas]{Figuras e Tabelas Flutuantes, Tabelas Longas e Imagem Externa}
%Tratar também de caixas gráficas.
Neste capítulo, trataremos de ilustrações e tabelas.

\section{Figuras flutuantes}

Elemento flutuante\index{elemento!flutuante} é aquele que não precisa ser colocado obrigatoriamente na posição ``digitada'', mas que será colocado na melhor posição possível em termos de apresentabilidade.

A figura flutuante\index{figura!flutuante} é especificada pelo ambiente \texttt{figure}\index{figure} e a tabela flutuante é especificada pelo ambiente \texttt{table}\index{table}, respectivamente.
Veja o Exemplo~\ref{ex09:figura}.

%%\setcounter{lstorichapter}{\value{chapter}}
%% \setcounter{lstorisection}{\value{section}}
%\begin{singlespacing}
%%\begin{LTXexample}[pos=b,caption={ex08-01.tex},label=ex:08-01,preset={\setcounter{chapter}{0},\let\addcontentsline\emptyaddcontentsline}]
%\begin{LTXexample}[pos=b,caption={ex08-01.tex},label=ex:08-01]
\begin{ltxlisting}[caption={ex09-figura.tex},label=ex09:figura]
\begin{figure}[hbp]
\center

    \

    Aqui se coloca a primeira figura

    \

 \caption[Primeira figura]{Primeira figura como elemento flutuante}
 \label{fig:simp}
\end{figure}
\end{ltxlisting}
%\end{LTXexample}
%\end{singlespacing}
%%\setcounter{chapter}{\value{lstorichapter}}
%%\setcounter{section}{\value{lstorisection}}
%%\let\addcontentsline\oriaddcontentsline

O Exemplo~\ref{ex09:figura} ilustra o uso do ambiente flutuante \texttt{figure}. 
O parâmetro opcional é uma sequência de caracteres, especificando a ordem que tentará colocar a figura.

\begin{description}
  \item[h]\index{float!h} Onde foi digitado
  \item[b]\index{float!b} na parte inferior da página
  \item[t]\index{float!t} na parte superior da página
  \item[p]\index{float!p} página separada.
  \item[!]\index{float!exclamation@\exclsymbforindex} ignorar a restrição de espaçamento.
\end{description}

  No ambiente flutuante, o \LaTeX{} tentará as posições seguindo a lista de especificação dos parâmetros e inserirá na primeira posição que satisfizer a exigência.
  Caso nenhuma posição for conveniente, criará uma página separada especialmente para ele.

 Durante a tentativa de inserção, existem restrições estéticas tal como considerar inadequado quando sobra muito pouco espaço para o texto. 
Restrições como estes podem ser ignorados quando utilizar a opção ``!''

O comando \verb+\caption+\index{caption@\verb+\caption+} produz enumeração e título da figura, podendo estar no começo ou no final do ambiente \texttt{figure}. 
Ele aceita o nome curto como argumento opcional para ser usado na lista de figuras, caso título da figura for longo.

Note que, o rótulo colocado pelo \verb+\label+\index{label@\verb+\label+} deverá ficar dentro ou depois de \verb+\caption+ que é responsável pela enumeração das figuras.

Para colocar moldura, poderá usar o \verb+\fbox+\index{fbox@\verb+\fbox+}, mas ele não pode ter parágrafo como argumento. 
Neste caso, podemos usar a combinação com \texttt{minipage} como no Exemplo~\ref{ex09:figura:fbox}.

%%\setcounter{lstorichapter}{\value{chapter}}
%% \setcounter{lstorisection}{\value{section}}
%\begin{singlespacing}
%\begin{LTXexample}[pos=b,caption={ex08-01.tex, parte b},label=ex:08-01:b]
\begin{ltxlisting}[caption={ex09-figura-fbox.tex},label=ex09:figura:fbox]
\begin{figure}[hbp!]
  \center
 \fbox{\begin{minipage}{0.5\linewidth}
  \center
  Aqui insere a segunda figura
  \caption{Segunda figura} \label{fig:frame}
 \end{minipage}
 } % fbox
\end{figure}
\end{ltxlisting}
%\end{LTXexample}
%\end{singlespacing}
%%\setcounter{chapter}{\value{lstorichapter}}
%%\setcounter{section}{\value{lstorisection}}

%Caso em que precise inserir duas ou mais imagens numa mesma figura,  use o pacote \texttt{subfigure}.
 
\section{Tabelas flutuantes}

A tabela flutuante\index{tabela!flutuante} é criado da forma similar, usando o ambiente \texttt{table}\index{table}. 
Os argumentos opcionais para controlar a posição de inserção é mesmo da figura. 
Veja o Exemplo~\ref{ex09:tabela:flutuante} que ilustra a tabela flutuante, contendo texto como elemento.

%%\setcounter{lstorichapter}{\value{chapter}}
%% \setcounter{lstorisection}{\value{section}}
%\begin{singlespacing}
%\begin{LTXexample}[pos=b,caption={ex08-01.tex, parte c},label=ex:08-01:c]
\begin{ltxlisting}[caption={ex09-tabela-flutuante.tex},label=ex09:tabela:flutuante]
\begin{table}[hbp]
\center

    \

    Aqui se coloca a primeira tabela

    \

 \caption{Primeira tabela}
 \label{tab:simp}
\end{table}
\end{ltxlisting}
%\end{LTXexample}
%\end{singlespacing}
%%\setcounter{chapter}{\value{lstorichapter}}
%%\setcounter{section}{\value{lstorisection}}

Normalmente, o conteúdo da tabela é criado pelo ambiente apropriado como \texttt{tabular} ou similar.
Alguns casos, podem ser criados como arquivo separado e inserido como no caso de figuras.

Note que, podemos criar moldura, incluindo o título da tabela através de \texttt{fbox} combinado com o \texttt{minipage}, como feito na figura.

No caso do Exemplo~\ref{ex09:tabela:tabular} ilustra o uso de \texttt{tabular} dentro do ambiente \texttt{table} para criar tabela flutuante.

%% \setcounter{lstorichapter}{\value{chapter}}
%% \setcounter{lstorisection}{\value{section}}
%\begin{singlespacing}
%\begin{LTXexample}[pos=b,caption={ex08-01.tex, parte d},label=ex:08-01:d]
\begin{ltxlisting}[caption={ex09-tabela-tabular.tex},label=ex09:tabela:tabular]
\begin{table}[hbp]
 \center
 \begin{tabular}{|p{3.5cm}|l|} \hline
   cenouras  (500g)          & R\$0,50  \\ \hline
   cogumelos (vidro de 500g) & R\$5,00  \\ \hline
   batata    (1Kg)           & R\$1,20  \\ \hline \hline
   total                     & R\$7,20  \\ \hline
 \end{tabular}
 \caption{Usando tabular} \label{tab:tab}
\end{table}
\end{ltxlisting}
%\end{LTXexample}
%\end{singlespacing}
%%\setcounter{chapter}{\value{lstorichapter}}
%%\setcounter{section}{\value{lstorisection}}

Note que, o comando \verb+\supressfloats+\index{supressfloats@\verb+\supressfloats+} podem ser usados para impedir que mais \texttt{floats} sejam inseridas na página específica, mas os ``floats'' com opção ``!'' não respeitam este comando.

Para que ``floats'' não processados (que ainda não foram colocados) sejam postos antes de mudar a página, basta usar
\verb+\clearpage+ ou \verb+\cleardoublepage+ usado para limpar configurações de páginas (chamado pelo \verb+\chapter+, por exemplo).

Note que, lista de figuras\index{lista de figuras} e de tabelas\index{lista de tabelas} são geradas pelos comandos \verb+\listoffigures+\index{listoffigures@\verb+\listoffigures+} e \verb+\listoftables+\index{listoftables@\verb+\listoftables+} que costumam ser colocados depois do \verb+\tableofcontents+\index{tableofcontents@\verb+\tableofcontents+} (sumário\index{sumário}).
 
\section{Tabelas longas}
 
O ambiente tabular é ideal para tabelas pequenas. 
Mas as vezes precisamos de tabelas grandes que podem ocupar mais de uma página. 
Para isso, existe o ambiente \texttt{longtable}\index{longtable} implementado no pacote do mesmo nome.
A tabela longa\index{tabela longa} é contado junto com as tabelas flutuantes e não devem ser colocados dentro do ambiente \texttt{table}. 
Ele também possui o comando \verb+\caption*+\index{caption*@\verb+\caption*+} (a versão ``*'' do \verb+\caption+) que coloca o título, mas não acrescenta na lista de tabelas.
Os parâmetros básicos são mesmo do ambiente \texttt{tabular}.

Veja o Exemplo~\ref{ex09:longtable}. Para que este exemplo funcione, deverá ter \verb+\usepackage{longtable}+ no \texttt{preamble} do documento.

%% \setcounter{lstorichapter}{\value{chapter}}
%% \setcounter{lstorisection}{\value{section}}
%\begin{singlespacing}
%\begin{LTXexample}[pos=b,caption={ex08-02.tex},label=ex:08-02]
\begin{ltxlisting}[caption={ex09-longtable.tex},label=ex09:longtable]
 \begin{longtable}{|c|r|}
 \caption{Tabela Longa} \label{ltab:teste}
 \\ % é necessário pular linha após definições preliminares: caption, label, etc.
 \hline
 \textbf{centrada} & \textbf{para direita} \\ \hline \hline
 coluna 1 & coluna 2 \\ \hline
 coluna 1 & coluna 2 \\ \hline
 $\vdots$ & $\vdots$ \\ \hline
 coluna 1 & coluna 2 \\ \hline
 \end{longtable}
\end{ltxlisting}
%\end{LTXexample}
%\end{singlespacing}
%%\setcounter{chapter}{\value{lstorichapter}}
%%\setcounter{section}{\value{lstorisection}}

Note que, ao iniciar o ambiente, coloca as configurações tais como títulos, o que fazer antes de mudar a página, etc.
Depois pula a linha com ``\verb+\\+'' e resto segue normalmente. 
Como \texttt{longtable} usa o arquivo auxiliar para armazenar sua largura, pode precisar compilar duas vezes para ter o resultado desejado (assim como acontece com referencias cruzadas).

Para efetuar as configurações tais como o que fazer quando muda a página, etc, veja o manual correspondente.

\section{Imagem externa}
As figuras podem ser elaboradas usando o próprio \LaTeX, mas muitas vezes criamos usando um programa externo.
As imagens externas\index{imagem externa} para serem inseridas no documento \LaTeX, deverá estar no formato \texttt{pdf}, \texttt{jpg/jpeg} ou \texttt{png}. 
O formato \texttt{pdf} costuma ser usado para ilustrações científicos por ser pequeno e é de alta qualidade (exceto quando convertido de algum formato \texttt{bitmap}).
Se o programa permite gerar \texttt{pdf}, prefira usar este formato. 
O formato \texttt{jpg/jpeg} é apropriado para fotos, mas não suporta transparências.
O formato \texttt{png} é similar a \texttt{jpg/jpeg}, mas suporta transparências.

O comando para incluir gráficos externos é \verb+\includegraphics+\index{includegraphics@\verb+\includegraphics+} implementado no pacote \texttt{graphicx}\index{graphicx} (não confundir com o pacote antigo \texttt{graphics} que tem menos recursos).
Para exemplo desta seção, assumimos que tenha \verb+\usepackage{graphicx}+ no \texttt{preamble} do documento.
Veja Exemplo~\ref{ex09:imagem}.

% \setcounter{lstorichapter}{\value{chapter}}
% \setcounter{lstorisection}{\value{section}}
%\begin{singlespacing}
%\begin{LTXexample}[pos=b,caption={ex09-01.tex},label=ex:09-01]
\begin{ltxlisting}[caption={ex09-imagem.tex},label=ex09:imagem]
\begin{figure}[hbp]
 \center
\includegraphics[width=0.5\linewidth]{latex-via-exemplos-fig}
\caption{Imagem PDF}
\end{figure}
\end{ltxlisting}
%\end{LTXexample}
%\end{singlespacing}
%\setcounter{chapter}{\value{lstorichapter}}
%\setcounter{section}{\value{lstorisection}}

Note que não foi colocado a extensão (\texttt{.pdf}) no nome do arquivo.
Em geral omitimos a extensão do arquivo para poder trocar de formatos, se desejar.

No parâmetro opcional, foi colocado \verb+width=0.5\linewidth+ para especificar que largura é metade da largura da linha atual.
Os parâmetros opcionais\index{includegraphics@\verb+\includegraphics+!parâmetro opcional} controlam o tamanho e rotação das imagens.

\begin{description}
  \item[width] largura da imagem
  \item[height] altura da imagem
  \item[totalheight] altura total (quando é rotacionado,\texttt{height} só mede da linha de base para cima).
  \item[scale] ampliação
  \item[angle] rotação em graus
  \item[origin] centro de rotação especificado pela combinação de ``l'' (left), ``r'' (right), ``t'' (top), ``b'' (botton) e ``c'' (center).
  \item[keepaspectratio] manter proporção quando \texttt{height} e \texttt{width} for especificado simultaneamente (usado sem o valor).
\end{description}

No caso de incluir o arquivo \texttt{PDF} com mais de uma página\index{imagem PDF!múltiplas páginas}, poderá usar a opção \texttt{page} para indicar a página a ser incluída.

O Exemplo~\ref{ex09:imagem:rotacao} ilustra a rotação por $30^\circ$ em torno do centro.
 
% \setcounter{lstorichapter}{\value{chapter}}
% \setcounter{lstorisection}{\value{section}}
%\begin{singlespacing}
%\begin{LTXexample}[pos=b,caption={ex09-01.tex, parte b},label=ex:09-01:b]
\begin{ltxlisting}[caption={ex09-imagem-rotacao.tex},label=ex09:imagem:rotacao]
\begin{figure}[hbp]
 \center
\includegraphics[width=0.5\linewidth,angle=30,origin=c]{latex-via-exemplos-fig}
\caption{Imagem PDF}
\end{figure}
\end{ltxlisting}
%\end{LTXexample}
%\end{singlespacing}
%\setcounter{chapter}{\value{lstorichapter}}
%\setcounter{section}{\value{lstorisection}}
 
As vezes queremos colocar figuras lado a lado para comparação ou para economia de espaços. Para tanto, podemos usar o \texttt{minipage}, como no Exemplo~\ref{ex09:imagem:lado}.

% \setcounter{lstorichapter}{\value{chapter}}
% \setcounter{lstorisection}{\value{section}}
%\begin{singlespacing}
%\begin{LTXexample}[pos=b,caption={ex09-01.tex, parte c},label=ex:09-01:c]
\begin{ltxlisting}[caption={ex09-imagem-ĺado.tex},label=ex09:imagem:lado]
\begin{figure}[hbp]
\center
\begin{minipage}{0.4\linewidth}
\center
\includegraphics[width=0.3\linewidth]{latex-via-exemplos-fig}
\caption{lado esquerdo} \label{fig:lado1}
\end{minipage}
\begin{minipage}{0.4\linewidth}
\center
\includegraphics[width=0.3\linewidth,angle=30,origin=c]{latex-via-exemplos-fig}
\caption{lado direito} \label{fig:lado2}
\end{minipage}
\end{figure}
\end{ltxlisting}
%\end{LTXexample}
%\end{singlespacing}
%\setcounter{chapter}{\value{lstorichapter}}
%\setcounter{section}{\value{lstorisection}}

\section{Desenhando sobre a imagem externa}
As vezes queremos desenhar sobre a imagem externa importada no documento, como no caso de acrescentar fórmulas ou mais alguns detalhes.
Para isso, podemos usar o ambiente gráfico \texttt{picture}\index{picture}.

No Exemplo~\ref{ex09:imagem:sobreposicao} ilustra a sobreposição para colocar fórmula sobre a imagem externa.

Note que a imagem foi inserida dentro do ambiente \texttt{picture}.

Observe como e onde foi usado o \verb|\unitlength|\index{unitlength@\verb+\unitlength+} para permitir o ajuste de escala (mudando o valor de \verb|\unitlength|) sem perder a posição de sobreposição já ajustada.

% \setcounter{lstorichapter}{\value{chapter}}
% \setcounter{lstorisection}{\value{section}}
%\begin{singlespacing}
%\begin{LTXexample}[pos=b,caption={ex09-04.tex},label=ex:09-04]
\begin{ltxlisting}[caption={ex09-imagem-sobreposicao.tex},label=ex09:imagem:sobreposicao]
\begin{figure}[htbp!]
\center
%
\unitlength=0.45\linewidth % unidade
\begin{picture}(1.0,1.0) % caixa reservada
\put(0,0){\includegraphics[width=\unitlength]{latex-via-exemplos-fig}}
\put(0.35,0.45){\LARGE $e^{\pi i}+1=0$}
\end{picture}
\caption{Técnica de sobreposicao\label{fig:sobreposicao}}
\end{figure}
\end{ltxlisting}
%\end{LTXexample}
%\end{singlespacing}
%\setcounter{chapter}{\value{lstorichapter}}
%\setcounter{section}{\value{lstorisection}}

O ambiente \texttt{picture} é um ambiente gráfico padrão do \LaTeX{} que permite criar ilustrações.
Note que, para criar ilustrações, existem vários pacotes gráficos apropriados, mas se quiser usar o ambiente \texttt{picture},
lembre-se de colocar o \verb+\usepackage{pict2e}+\index{pict2e} no \texttt{preamble} do documento para eliminar limitações de alguns comandos gráficos deste ambiente.

O Exemplo~\ref{ex09:picture} ilustra os comandos básicos do ambiente \texttt{picture} com o uso do pacote \texttt{pict2e}.

% \setcounter{lstorichapter}{\value{chapter}}
% \setcounter{lstorisection}{\value{section}}
%\begin{singlespacing}
%\begin{LTXexample}[pos=b,caption={ex09-04.tex, parte b},label=ex:09-04:b]
\begin{ltxlisting}[caption={ex09-picture.tex},label=ex09:picture]
\begin{figure}[hbtp!]
\center
\unitlength=1cm % medida da  unidade
% \linethickness{1mm} % tome cuidado que e comando, nao a medida
\begin{picture}(5,4)(0,0) 
% put coloca o objeto grafico na posição indicada
% line (segmento) e vector (vetor)
% Recebe vetor diretor e comprimento do segmento/vetor
\put(4,3.5){\line(0,1){0.5}} % segmento
\put(5,3){\vector(1,1){0.5}} % vetor
\put(2,2){\circle{1}} % circulo  com centro e raio. 
\put(1,1){\circle*{1.5}} % circle* é circulo solido
\put(3,2){\oval(3,2)} % oval
\put(3.5,2){\makebox(0,0)[cc]{$A=\pi r^2$}} % Caixa de texto (com formulas)
\put(3,0) { \polyline(0,0)(1,0)(1,1)} % linhas poligonais
\put(5,0) { \polygon(0,0)(1,0)(1,1)} % poligonos
\put(6.5,0) { \polygon*(0,0)(1,0)(1,1)} % poligonos preenchidos
\thicklines % linha mais grossa
\put(4,1){\qbezier(0.0,0.0)(1.3,0.0)(2.0,2.8)} % curva bezier quadratica
\thinlines %linha normal (mais fina)
\put(6,1){\cbezier(0.0,0.0)(1.0,0.0)(1.0,1.0)(2.0,3.0)} 
% curva bezier cubica
\end{picture}

\caption{Ilustração no formato \TeX{}.\label{fig:tex}}
\end{figure}
\end{ltxlisting}
%\end{LTXexample}
%\end{singlespacing}
%\setcounter{chapter}{\value{lstorichapter}}
%\setcounter{section}{\value{lstorisection}}

\section{Caixas gráficas}
O pacote \texttt{graphicx} dispões de vários comandos relacionados com a mudança de tamanho e rotações.
Eles são chamados de caixas gráficas.
Os argumentos não devem conter quebra de parágrafos. Assim, se precisar, deverá usar juntamente com o \texttt{minipage} (ou \verb+\parbox+). 

Para escalar, usa-se o \verb+\scalebox+\index{scalebox@\verb+\scalebox+}. Para indicar o tamanho fixo, usa-se o \verb+\resizebox+\index{resizebox@\verb+\resizebox+}.
Veja o Exemplo~\ref{ex09:scale}.

% \setcounter{lstorichapter}{\value{chapter}}
% \setcounter{lstorisection}{\value{section}}
%\begin{singlespacing}
%\begin{LTXexample}[pos=b,caption={ex09-02.tex},label=ex:09-02]
\begin{ltxlisting}[caption={ex09-scale.tex},label=ex09:scale]
Tamanho normal.

\scalebox{2.5}{Ampliado por 2.5 vezes}

\scalebox{2}[3]{Ampliado por 2x3 vezes} % 2 no horizontal e 3 em vertical (deforma)

\resizebox{10cm}{!}{Com 10 cm de comprimento}

\resizebox{!}{0.5cm}{Com 0.5 cm de altura}

\resizebox{5cm}{0.75cm}{Com 5cm$\times$0.7cm} % deforma
\end{ltxlisting}
%\end{LTXexample}
%\end{singlespacing}
%\setcounter{chapter}{\value{lstorichapter}}
%\setcounter{section}{\value{lstorisection}}

O ``!'' no \verb+\resizebox+ especifica que é calculado automaticamente com outra medida.

A rotação e feito pelo comando \verb+\rotatebox+\index{rotatebox@\verb+\rotatebox+}, onde parâmetro opcional \texttt{origin}\index{rotatebox@\verb+\rotatebox+!origin} permite configurar o centro de rotação, combinando ``l'' (left), ``r'' (right), ``t'' (top), ``b'' (botton) e ``c'' (center).
Veja o Exemplo~\ref{ex09:rotate}.
  
% \setcounter{lstorichapter}{\value{chapter}}
% \setcounter{lstorisection}{\value{section}}
%\begin{singlespacing}
%\begin{LTXexample}[pos=b,caption={ex09-02.tex, parte b},label=ex:09-02:b]
\begin{ltxlisting}[caption={ex09-rotate.tex},label=ex09:rotate]
\rotatebox{30}{Rotacionado} por $30^\circ$.

\rotatebox[origin=rb]{-30}{Rotacionado} por $-30^\circ$ em torno de lado direito inferior.

\rotatebox[origin=c]{45}{Rotacionado} por $45^\circ$ em torno de centro.

\reflectbox{Refletido} está refletido. Mesmo que o \scalebox{-1}[1]{Refletido}.

\raisebox{\depth}{\scalebox{1}[-1]{Refletido}} está refletido verticalmente.

\rotatebox[origin=c]{30}{\scalebox{1.5}{Ampliada e rotacionada}}
\end{ltxlisting}
%\end{LTXexample}
%\end{singlespacing}
%\setcounter{chapter}{\value{lstorichapter}}
%\setcounter{section}{\value{lstorisection}}

O \verb+\reflectbox+\index{reflectbox@\verb+\reflectbox+} reflete horizontalmente.
Para refletir verticalmente, use o \verb+\scalebox+, mas precisa ajustar a sua altura. 
%Aqui foi usado o \verb+\raisebox+\index{raisebox@\verb+\raisebox+}, o que não vamos entrar em detalhes.

Uma caixa é um bloco e além de ser manipulados no tamanho e posição, também pode ser usado mais de uma vez.
Por exemplo, se quer colocar um certo elemento em mais de um lugar, ou colocar mais tarde, poderá criar uma caixa e armazenar dentro dele.
Neste texto, não vamos entrar em detalhes sobre caixas, mas várias referências discutem sobre isso.

\clearpage\thispagestyle{empty}\cleardoublepage
\chapter{Ajuste das Fontes}\label{chap:fontes}
Neste capítulo, veremos o controle das fontes no \LaTeX.

\section{Seleção da família de fontes}
Num documento, usa-se vários tipos de fontes. 
Por exemplo, o corpo do documento nos livros e artigos costuma ser em romano\index{fonte!romano} (o que tem enfeite nas pontas, denominado de serifa\index{serifa}).
Os títulos dos artigos e livros podem ficar com fontes \texttt{sans serif}\index{sans serif} (sem serifa\index{fonte!sem serifa}). 
O corpo de slide de apresentação ou poster costuma usar \texttt{sans serif} também.
O código de programa, nome do arquivo, etc que deve ser lido no pé da letra costuma estar como \texttt{typewriter}\index{typewriter} (mono espaçado\index{fonte!nomo espaçado}).

Assim, nos documentos científicos, costumam usar três famílias de fontes básicos que são romano (com serifa\index{com serifa}), \texttt{sans serif} (sem serifa) e \texttt{typewriter} (mono espaçado) na qual devem estar em harmonia. 
O \LaTeX{} usa a fonte Computer Modern\index{fonte!Computer Modern} por padrão na qual tem essas três fontes estão coerentes entre si.
Note que podem trocar para outras fontes, se desejar, mas estas três fontes devem estar em harmonia.

A família das fontes são selecionadas pelo comando \verb+\textrm+\index{textrm@\verb+\textrm+} ou ambiente \texttt{rmfamily}\index{rmfamily} para romano, o que é padrão para artigos e livros, pelo comando \verb+\textsf+\index{textsf@\verb+\textsf+} ou ambiente \texttt{sffamily}\index{sffamily} para \texttt{sans serif}, e pelo comando \verb+\texttt+\index{texttt@\verb+\texttt+} ou ambiente \texttt{ttfamily}\index{ttfamily} para \texttt{typewriter}.

Quando aplica a mudança da fonte dentro de um ambiente, o ambiente de ajuste de fontes pode ser usado como comandos. Se não estiver dentro do ambiente e quer usar a versão comando do ambiente, poderá delimitar a sua atuação, delimitando pelos chaves.
Veja Exemplo~\ref{ex10:family}.

%\begin{singlespacing}
%\begin{LTXexample}[pos=b,caption={ex03-01.tex, parte b},label=ex:03-01:b]
\begin{ltxlisting}[caption={ex10-family.tex},label=ex10:family]
\textsf{Fontes sem serifa}

\begin{sffamily}
Texto na fonte sem serifa.
\end{sffamily}

\textsf{Fontes mono espaçado}

\begin{ttfamily}
Texto na fonte mono espaçado.
\end{ttfamily}

\begin{center}
\sffamily
texto centralizado e em sans serif.
\end{center}

{\ttfamily Texto em nomo espaçado}

Texto normal.
\end{ltxlisting}
%\end{LTXexample}
%\end{singlespacing}


\section{Seleção de formas e peso das fontes}

No \LaTeX, usa-se o comando para trocar forma de fontes, tais como negrito, itálico, etc. 
Usar comando adequado no \LaTeX{} é importante para automatizar o processo, não somente para formatação de textos.
O texto enfatizado\index{fonte!enfatizado} é produzido pelo comando \verb+\emph+\index{emph@\verb+\emph+} ou pelo ambiente \texttt{em}\index{em}.
O comando \verb+\emph+ altera entre fonte reto e itálico para destacar palavras tal como termo que está sendo definido.

O texto em negrito\index{fonte!negrito}, por sua vez, é produzido pelo comando \verb+textbf+\index{textbf@\verb+\textbf+} ou pelo ambiente \texttt{bfseries}\index{bfseries} para dar ênfase maior que destaca no meio do texto.
Fonte não negrito\index{fonte!não negrito} que é padrão pode ser especificado pelo comando \verb+\textmd+\index{textmd@\verb+\textmd+} ou ambiente \texttt{mdseries}\index{mdseries}. 

Ainda existem outras formas tal como o comando \verb+\textsc+\index{textsc@\verb+\textsc+} e ambiente \texttt{scshape}\index{scshape} para \texttt{small caps}\index{fonte!samll caps}, também conhecido como versaletes\index{versaletes}, comandos \verb+\textit+\index{textit@\verb+\textit+} e ambiente \texttt{itshape}\index{itshape} para itálicos\index{fonte!itálico}. Também existe o comando \verb+\textsl+\index{textsl@\verb+\textsl+} e ambiente \texttt{slshape}\index{slshape} que requer cuidado no uso.

O formato normal\index{fonte!formato normal} padrão é \verb+\textup+\index{textup@\verb+\textup+} ou ambiente \texttt{upshape}\index{upshape}.
Veja Exemplo~\ref{ex10:series}.

%\begin{singlespacing}
%\begin{LTXexample}[pos=b,caption={ex03-01.tex, parte c},label=ex:03-01:c]
\begin{ltxlisting}[caption={ex10-series.tex},label=ex10:series]
\textbf{Texto em negrito}

\begin{bfseries}
Negrito como ambiente
\end{bfseries}

\begin{em}
Parágrafo enfatizado. \emph{Texto enfatizado dentro dele}
\end{em}

\textsc{Small Caps}

\begin{center}
\bfseries
Negrito e centralizado.
\textmd{Normal}
\end{center}

{\scshape Small Caps dentro dos chaves}

Texto normal
\end{ltxlisting}
%\end{LTXexample}
%\end{singlespacing}

Note que os comandos antigos\index{fonte!comandos antigos} \verb+\rm+, \verb+\tt+, \verb+\bf+, \verb+\sf+, \verb+\it+ e \verb+\sc+ para ajuste de fontes não devem ser usados, pois não podem ser combinados (por exemplo, não é possível produzir itálico negrito), além de ter de problemas de ajustes de espaçamentos (como a necessidade de correção de itálico).

Para restaurar a fonte padrão\index{fonte!restaurar padrão}, basta usar o ambiente \texttt{normalfont}\index{normalfont} (que tem versão comando dele também).

\section{Tamanho das fontes}

Tamanho das fontes\index{fonte!tamanho} são especificados também pelos comandos e o \LaTeX{} efetua ajuste automático em relação ao tamanho padrão.

O tamanho de menor para maior são definidos pelo ambientes

\texttt{tiny}: {\tiny tiny\index{tiny}}

\texttt{scriptsize}: {\scriptsize scriptsize\index{scriptsize}}

\texttt{footnotesize}: {\footnotesize footnotesize}

\texttt{normalsize}: {\normalsize normalsize\index{normalsize}}

\texttt{large}: {\large large\index{large}}

\texttt{Large}: {\Large Large\index{Large}}

\texttt{LARGE}: {\LARGE LARGE\index{LARGE}}

\texttt{huge}: {\huge huge\index{huge}}

\texttt{Huge}: {\Huge Huge\index{Huge}}

Lembrando que existem comandos correspondentes a cada um dos ambientes, similar a outros ambientes de fontes.
Veja Exemplo~\ref{ex10:size}.

%\begin{singlespacing}
%\begin{LTXexample}[pos=b,caption={ex03-01.tex, parte d},label=ex:03-01:d]
\begin{ltxlisting}[caption={ex10-size.tex},label=ex10:size]
\begin{Large}
Letra grande (2 escalas acima)
\end{Large}

\begin{em}
\small Parágrafo enfatizado com letra pequena (uma escala abaixo)
\end{em}

{\footnotesize Fonte 2 escalas abaixo}

Texto normal
\end{ltxlisting}
%\end{LTXexample}
%\end{singlespacing}

A fonte com todos atributos padrão é especificado no \LaTeX{} pelo comando \verb+\textnormal+\index{textnormal@\verb+\textnormal+} ou pelo ambiente \texttt{normalfont}\index{normalfont}.

\section{Ajuste de fontes no modo matemático}\label{sec:fonte:matematica}
No \LaTeX, as especificações das fontes no texto e na fórmula matemática usam os comandos diferentes.
A especificação da fonte mo modo matemático corresponde ao 
\verb|\text??{}| são \verb|\math??{}|:
\verb|\mathrm{}|\index{mathrm@\verb+\mathrm+}, \verb|\mathsf{}|\index{mathsf@\verb+\mathsf+}, \verb|\mathtt{}|\index{mathtt@\verb+\mathtt+}, \verb|\mathbf{}|\index{mathbf@\verb+\mathbf+}, \verb|\mathit{}|\index{mathit@\verb+\mathit+}, \verb|\mathnormal{}|\index{mathnormal@\verb+\mathnormal+}, \verb|\mathcal{}|\index{mathcal@\verb+\mathcal+}.

O comando \verb|\mathcal{}|\index{mathcal@\verb+\mathcal+} que não tem correspondente no modo texto, usa a fonte caligráfica, mas somente em letras maiúsculas.

Os comandos acima não funcionam para símbolos e eles não podem ser combinados.
Por exemplo, \verb|\mathbf{\matrm{A}}| é mesmo que \verb|\mathrm{A}|. 

Assim, \verb+\mathbf+ não é apropriado para escrever fórmulas em negrito. Portanto, use \verb+\bm+\index{bm@\verb+\bm+} do pacote \texttt{bm}\index{bm} para converter parte da fórmula em negrito.
Note que, quando usa em conjunto com o \texttt{amsmath}, pode precisar colocar chaves duplas para prevenir erros em \verb+\bm+ (veja Exemplo~\ref{ex10:fontes:mat}).

O pacote \texttt{amssymb} dispõe de mais dois fontes bastante usados na matemática que são \verb+\mathfrak+\index{mathfrak@\verb+\mathfrak+} e \verb+\mathbb+\index{mathbb@\verb+\mathbb+}.
Em geral, usa-se os pacotes \texttt{amsmath} e \texttt{amssymb} quando produz textos matemáticos.
Então acrescente \verb+\usepackage{amssymb,amsmath}+
no \texttt{preamble} do documento como no Exemplo~\ref{ex10:fontes:mat}.

%\begin{singlespacing}
%\begin{LTXexample}[pos=b,caption={ex03-02.tex, parte b},label=ex:03-02:b]
\begin{ltxlisting}[caption={ex10-fontes-mat.tex},label=ex10:fontes:mat]
\documentclass[12pt,a4paper]{article}
\usepackage[T1]{fontenc}
\usepackage[utf8]{inputenc}
\usepackage[brazil]{babel}
\usepackage{amssymb,amsmath}
\usepackage{bm}
\begin{document}
\[ \forall x \in \mathbb{R},\mathrm{sen}^2(x)+\cos^2(x)=1 \]
{\bfseries Para $\bm{{i=0}}$}: Por definição, $0!=1$.
\end{document}
\end{ltxlisting}
%\end{LTXexample}
%\end{singlespacing}

Note que nomes (ou abreviações) das funções devem estar em fonte romano reto. 
Assim, $\mathrm{sen}$ foi produzido, usando \texttt{mathrm}.

Para não precisar especificar \verb|\mathrm{}|, existem comandos para nomes\index{comandos!nome das funções} de maioria das funções matemáticas mais conhecidas.

Nome das funções pré-definidas:

$\arccos$\index{arccos@\verb+\arccos+}, $\arcsin$\index{arcsin@\verb+\arcsin+}, $\arctan$\index{arctan@\verb+\arctan+}, $\arg$\index{arg@\verb+\arg+}, $\bmod$\index{bmod@\verb+\bmod+}, $\cos$\index{cos@\verb+\cos+}, $\cosh$\index{cosh@\verb+\cosh+}, $\cot$\index{cot@\verb+\cot+}, $\coth$\index{coth@\verb+\coth+}, $\csc$\index{csc@\verb+\csc+}, $\deg$\index{deg@\verb+\deg+}, $\det$\index{det@\verb+\det+}, $\dim$\index{dim@\verb+\dim+}, $\exp$\index{exp@\verb+\exp+}, $\gcd$\index{gcd@\verb+\gcd+}, $\hom$\index{hom@\verb+\hom+}, $\inf$\index{inf@\verb+\inf+}, $\ker$\index{ker@\verb+\ker+}, $\lg$\index{lg@\verb+\lg+}, $\lim$\index{lim@\verb+\lim+}, $\liminf$\index{liminf@\verb+\liminf+}, $\limsup$\index{limsup@\verb+\limsup+}, $\ln$\index{ln@\verb+\ln+}, $\log$\index{log@\verb+\log+}, $\max$\index{max@\verb+\max+}, $\min$\index{min@\verb+\min+}, $\sec$\index{sec@\verb+\sec+}, $\sin$\index{sin@\verb+\sin+}, $\sinh$\index{sinh@\verb+\sinh+}, $\sup$\index{sup@\verb+\sup+}, $\tan$\index{tan@\verb+\tan+}, $\tanh$\index{tanh@\verb+\tanh+}, $\Pr$\index{Pr@\verb+\Pr+}.

Note que o \emph{seno} em inglês é \emph{sine}, abreviado para \texttt{sin} que não coincide com o termo usado no Brasil que é \texttt{sen}.
Neste caso, costuma definir o comando para não ficar digitando \texttt{mathrm} toda hora.
De forma análoga, quando o comando produz nome não usual, como no caso de tangente que é $\tan$ no \LaTeX, mas costuma usar $\mathrm{tg}$ no Brasil, podem ser redefinidos.
A definição e redefinição de comandos estão na Seção~\ref{sec:comandos} do Capitulo~\ref{chap:novo:comando}. 

\clearpage\thispagestyle{empty}\cleardoublepage
\chapter{Referências Bibliográficas e Índice Remissivo}
Veremos como produzir índice remissivo e referências bibliográficas no \LaTeX.

\section{Referências bibliográficas}
Uma referências bibliográficas\index{referência bibliográfica} é uma lista de referências externas usadas no trabalho, comumente apresentado no final do documento.
Esta lista é definido pelo ambiente \texttt{thebibliography}\index{thebibliography} cuja argumento é o elemento com maior largura, para ajustar alinhamento de seus itens. 
Cada item da referência é especificado pelo comando \verb+\bibitem+\index{bibitem@\verb+\bibitem+} cujo argumento obrigatório é uma chave e argumento opcional é o rótulo a ser impresso no item (se for omitido, será usado números). 
Depois segue os dados da referência bibliográfica como devem ser impressas.
Em geral, esta lista será colocado no final do documento (se existir o índice remissivo, será antes do índice remissivo). 
No Exemplo~\ref{ex11:bib}, o argumento \texttt{OPHS15} do \texttt{thebibliography} será usado para medir o espaço deixado à esquerda dos itens.
Além disso, cada item tem o rótulo (argumento opcional) a ser impresso como nome. 
Depois vem a chave que será usado no documento para citar o item especifico.

%\begin{singlespacing}
%\begin{LTXexample}[pos=b,caption={ex07-01.tex},label=ex:07-01,preset={\let\addcontentsline\emptyaddcontentsline}]
\begin{lstlisting}[caption={ex11-bib.tex},label=ex11:bib]
\begin{thebibliography}{OPHS15}
\bibitem[GMS04]{Goossens:2004}
 Michel Goossens and Frank Mittelbach
 \emph{The {\LaTeX} companion (second edition)}, Adilson--Wesley, 2004.
\bibitem[Tea00]{IndianTUG:2000}
 Tutorial Team, \emph{Online tutorials on {\LaTeX}},
 Indian {\TeX} User Group, 2000. 
\bibitem[LL94]{Lamport:1994} 
 Leslie Lamport,
 \emph{\LaTeX: A Document Preparation System (2nd Edition)},  Addison-Wesley Professional, 1994. 
 \bibitem[OPHS15]{Oetiker:2015}
 Tobias Oetiker et. al.,
 \emph{The Not So Short Introduction to \emph{\LaTeXe}}, URL: https://ctan.org/pkg/lshort-english,
2015.
\end{thebibliography}
\end{lstlisting}
%\end{LTXexample}
%\end{singlespacing}

%\begin{singlespacing}
%\begin{LTXexample}[pos=b,caption={ex07-01.tex},label=ex:07-01,preset={\let\addcontentsline\emptyaddcontentsline}]
\begin{ltxlistingout}
\begin{thebibliography}{OPHS15}
\bibitem[GMS04]{Goossens:2004:temp}
 Michel Goossens and Frank Mittelbach
 \emph{The {\LaTeX} companion (second edition)}, Adilson--Wesley, 2004.
\bibitem[Tea00]{IndianTUG:2000:temp}
 Tutorial Team, \emph{Online tutorials on {\LaTeX}},
 Indian {\TeX} User Group, 2000. 
\bibitem[LL94]{Lamport:1994:temp} 
 Leslie Lamport,
 \emph{\LaTeX: A Document Preparation System (2nd Edition)},  Addison-Wesley Professional, 1994. 
 \bibitem[OPHS15]{Oetiker:2015:temp}
 Tobias Oetiker et. al.,
 \emph{The Not So Short Introduction to \emph{\LaTeXe}}, URL: https://ctan.org/pkg/lshort-english,
2015.
\end{thebibliography}
\end{ltxlistingout}
%\end{LTXexample}
%\end{singlespacing}

No texto, uma citação da referência bibliográfica é inserida pelo comando \verb+cite{chave}+\index{cite@\verb+\cite+} onde \texttt{chave} é a chave colocado no \verb+\bibitem+.

Se for citar mais de um item, coloque as chaves separadas pela virgula. Também poderá colocar informações adicionais como parâmetro opcional do \verb+\cite+.
Veja Exemplo~\ref{ex11:cite}.
 
\begin{singlespacing}
%\begin{lstlisting}[pos=b,caption={ex07-01.tex, parte b},label=ex:07-01:b]
\begin{lstlisting}[caption={ex11-cite.tex},label=ex11:cite]
Exemplo de referência bibliográfica (veja~\cite{Lamport:1994} e \cite{Oetiker:2015}).

Para exemplos usando o BibTeX, poderá consultar
\cite{IndianTUG:2000, Goossens:2004}.

Para recursos avançados de BibTeX, veja o 
\cite[Cap.~13]{Goossens:2004}.
\end{lstlisting}
\end{singlespacing}

%\noindent
%\fbox{\begin{minipage}{\linewidth}
%\begin{singlespacing}
\begin{ltxlistingout}
%Exemplo de referência bibliográfica (veja~\cite{Lamport:1994} e \cite{Oetiker:2015}).
Exemplo de referência bibliográfica (veja~[LL94] e [OPHS15]).

%Para exemplos usando o BibTeX, poderá consultar \cite{IndianTUG:00, Goossens:2004}.
Para exemplos usando o BibTeX, poderá consultar [Tea00, GMS04].

%Para recursos avançados de BibTeX, veja o \cite[Cap.~13]{Goossens:2004}.
Para recursos avançados de BibTeX, veja o 
[GMS04, Cap.~13].
\end{ltxlistingout}
%\end{singlespacing}
%\end{minipage}}

\section{Usando o \BibTeX}

Checar qual das referências foram usadas e formatar uniformemente de acordo com as exigências do editor é uma tarefa difícil.
Para automatizar este serviço, podemos usar o \BibTeX{}\index{BibTeX}  que é uma ferramenta especialmente desenvolvida para manipular referências bibliográficas.
Os editores costumam deixar um arquivo de estilo próprio para \BibTeX{} e usando ele, o \BibTeX{} formatará automaticamente as referências bibliográficas de acordo.
Outra coisa importante que o \BibTeX{} faz é checar quais dos itens foram usados e imprimir somente as referências citadas no texto.

Para usar o \BibTeX, prepara o arquivo de referência bibliográfica\index{arquivo!referência bibliográfica} separado, com extensão \texttt{.bib}, que contém informações das referências.
Em geral, costumamos usar os aplicativos gráficos tal como aplicativo gratuito e multi plataforma \texttt{jabref}\index{jabref} (\url{http://www.jabref.org/}) para editar o arquivo de \BibTeX, mas também pode ser editado manualmente.

Aqui, vamos ver como preparar manualmente o arquivo \texttt{.bib}.
Veja o Exemplo~\ref{ex11:bibtex:bib} para ver como deve ficar o arquivo \texttt{.bib}.
\begin{singlespacing}
%\begin{lstlisting}[language=BibTeX,pos=b,caption={ex07-02.bib},label=ex:07-02:bib]
\begin{lstlisting}[language=BibTeX,caption={ex11-bibtex.bib},label=ex11:bibtex:bib]
@string{AW="Adilson--Wesley"}
@string{TUG="{\TeX{}} User Group"}

@book{Lamport:1994,
author={Leslie Lamport},
title="{\LaTeX{}}, A Document Preparation System (2nd Edition)",
publisher=AW,
address ={Reading, MA},
year=1994
}

@book{Goossens:2004,
author={Michel Goossens and Frank Mittelbach},
title="The {\LaTeX} companion (second edition)",
publisher=AW,
address ={Reading, MA},
year=2004
}

@book{Oeticker:2015,
author={Tobians Oeticker and Hubert Partl and Irene Hyna and Elisabeth Schlegl},
title="The Not So Short Introduction to {\LaTeXe{}}",
publisher=CTAN,
note={https://ctan.org/pkg/lshort-english},
year=2015
}

@book{IndianTUG:2000,
author={{Tutorial Team}},
title="Online Tutorials on {\LaTeX{}}",
publisher="Indian " # TUG,
note = {http://www.tug.org/tutorials/tugindia/},
year=2000
}

@article{Mertz:2009,
author={Andrew Mertz and William Slough},
title={A TikZ tutorial: Generating graphics in the
spirit of {\TeX}},
journal=TUGboat,
note={http://www.tug.org/TUGboat/tb30-2/tb95mertz.pdf}
volume=30,
number=2,
year=2009
}
\end{lstlisting}
\end{singlespacing}

O comando de \BibTeX{} inicia com ``@''\index{BibTeX!arroba@\arrobasymbforindex} seguido de nome e seus  dados entre chaves.
Os dados são listas separados pela virgula na forma forma ``chave=dado'', onde dado pode ser delimitado por aspas ou chaves.

\texttt{@string}\index{BibTeX!arrobastring@{\arrobasymbforindex string}} permite definir abreviaturas.
No começo do arquivo, foi definido as abreviaturas \texttt{AW} para ``Adilson--Wesley'' e \texttt{TUG} para ``\{\TeX\{\}\} User Group'' para facilitar a digitação de dados.
Note que, comandos de \TeX{} devem ficar delimitados pelas chaves no arquivo de \BibTeX.

Depois encontra o \texttt{@book}\index{BibTeX!arrobabook@{\arrobasymbforindex book}} que especifica que é dado de um livro.
Entre chaves, estão os dados do livro tais como título, autor, etc, onde cada campo é separado da outra pela virgula e dados do campo são delimitados pelos chaves ou aspas.
Note que o primeiro elemento de \texttt{@book} é uma chave a ser usado dentro do documento pelo comando \verb+\cite+ e é único que não é da forma \texttt{campo = dado do campo}
Tanto aspas como chaves podem ser usados para delimitar dados do campo, mas se dado for uma única palavra como o ano, não precisa do delimitador.

Relembrando que os comandos de \TeX{} devem ficar delimitados por chaves, o que indica ao \BibTeX{} para não alterar esta parte (como converter entre maiúsculo e minúsculo).
Note o uso da abreviatura \texttt{AW} em alguns pontos do documento (sem colocar delimitadores).
Esta abreviatura foi definido no começo do arquivo e será substituído automaticamente pelo seu valor ``Adilson--Wesley'' pelo \BibTeX.

No \texttt{publisher="Indian " \# TUG}, ``\#''\index{BibTeX!\#} faz a concatenação de dois strings \verb*+Indian + e o string\texttt{TUG} que é \verb+{\TeX{}} User Group+.

Quando tem vários autores, separe os nomes com ``and'' que o \BibTeX{} vai entender.
Quando nome ou sobrenome é uma palavra composta, coloque \verb+{ }+ em vez de espaço, como em \verb+author="Paulo da Silva Gon{\c c}alves{ }Junior"+. Outra forma é usar o formato ``sobrenome, nome'' como em \verb+author="Gon{\c c}alves Junior, Paulo da Silva"+.
No exemplo, ``Tutorial Team'' no campo do autor foi escrito desta forma.

Parte do campo que não pode ser retocado (alterar entre maiúsculo e minúsculo, abreviar, etc) como no caso de comando de \LaTeX, coloque entre chaves extras. No caso do sobrenome dos autores, é recomendável não colocar entre chaves, pois a conversão automática entre maiúsculo e minúsculo será desativada.

% Quando é um nome estrangeiro na qual a ordenação não ocorre de forma devida, poderá usar o campo `` key'' para indicar a palavra usada para ordenação.

Tendo o arquivo \texttt{.bib} pronto, efetuamos as citações como no caso anterior.

Agora, no final do arquivo, onde colocamos o ambiente \texttt{thebibliography}, substituímos por

\begin{singlespacing}
\begin{lstlisting}
\bibliographystyle{[estilo bib]}
\bibliography{[arquivo bib]}
\end{lstlisting}
\end{singlespacing}

onde \texttt{[estilo bib]} é o estilo usado para formatar as referências bibliográficas e \texttt{[arquivo bib]} é o nome do arquivo de \BibTeX{} (extensão \verb+bib+) sem extensão. Quando tiver mais de um arquivo, separe os nomes pela vírgula.

Agora execute o \LaTeX, \BibTeX{} e \LaTeX{} novamente. 
Em geral, os editores especializados para \LaTeX{} tem os botões ou menus para rodar o \BibTeX, além dos botões de compilar (executar o \LaTeX).

Ao executar \BibTeX, será gerado a saída das referências bibliográficas somente com itens citados. 
Por exemplo, para as citações do Exemplo~\ref{ex11:cite}, o artigo com chave \texttt{Mertz:2009} será ignorado por não estar citados.

Para mudar a formatação das referências, é só alterar o estilo em \verb+\bibliographystyle+\index{bibliographystyle@\verb+\bibliographystyle+} e executar o \BibTeX e \LaTeX{} novamente.
Os estilos básicos são: \texttt{plain}\index{BibTeX!plain} (rótulo numérico), \texttt{unsrt}\index{BibTeX!unsrt} (similar a \texttt{plain}, mas sem ordenar -- na ordem que foi citado), \texttt{alpha}\index{BibTeX!alpha} (rótulo pelo sobrenome e ano), \texttt{abbrv}\index{BibTeX!abbrv} (similar a \texttt{plain}, mas mais compacto), \texttt{amsplain}\index{BibTeX!amsplain} (estilo \texttt{plain} do AMS), \texttt{amsalpha}\index{BibTeX!amsalpha} (estilo \texttt{alpha} do AMS).

Além desses, existem vários outros estilos, dependendo da instalação de cada sistema \TeX{} em uso.

Se o editor dispõe de arquivos de estilo para \BibTeX{} (com extensão \texttt{bst}\index{arquivo!bst}), deixe este arquivo junto com arquivo \texttt{tex} e no \verb+\bibliographystyle+, passe o nome deste arquivo \texttt{.bst} sem a extensão. 
Se arquivo é \texttt{meuart.bst}, o estilo é \texttt{meuart}.

Uma observação importante é que o \BibTeX{} não suporta  adequadamente a acentuação direta ainda. 
Assim, deverá acentuar no modo \TeX. 
Lembre-se que a acentuação no modo \TeX{} precisa estar entre chaves para \BibTeX{} não tentar processar esta parte.   

Quando o editor exige um padrão diferente do disponível e não fornece o arquivo de estilo para \BibTeX, podemos contornar o problema da seguinte forma.
Primeiro termine o artigo ou livro com citações necessárias e usando um estilo mais próximo do exigido.
Depois rode o \LaTeX{} e \BibTeX. O \BibTeX{} gerará um arquivo com extensão \texttt{bbl}\index{arquivo!bbl}, com o mesmo nome do arquivo \texttt{tex}. Este arquivo \texttt{.bbl} contém o ambiente \texttt{thebibliography} com itens de bibliografia já formatada. Copie ele para documento original onde está o 
\begin{verbatim}
\bibliographystyle{[estilo bib]}
\bibliography{[arquivo bib]}
\end{verbatim}
e efetue alterações manualmente.

Não esqueça de comentar o
\begin{verbatim}
\bibliographystyle{[estilo bib]}
\bibliography{[arquivo bib]}
\end{verbatim}

Para especificar dados no arquivo de \BibTeX, além da categoria \texttt{@book} (livro), existem muitas especificações de documentos tais como \texttt{@article}\index{BibTeX!arrobaarticle@{\arrobasymbforindex article}} (artigo), \texttt{@manual}\index{BibTeX!arrobamanual@{\arrobasymbforindex manual}} (documentação técnica), \texttt{@phdthesis}\index{BibTeX!arrobaphdthesis@{\arrobasymbforindex phdthesis}} (tese de doutorado), etc, uma para cada categoria. 
Se não enquadrar em nenhuma das categorias disponíveis, use o \texttt{@misc}\index{BibTeX!arrobamisc@{\arrobasymbforindex misc}} (diversos).

A lista destas categorias e seus campos estão resumidos na tabela a seguir.

\noindent
\begin{longtable}{|p{0.2\linewidth}|l|p{0.2\linewidth}|p{0.35\linewidth}|}
%\\ 
\hline
\textbf{tipo} & \textbf{bibtex} & \textbf{necessário} & \textbf{opcional}\\ \hline
artigos & @article & auther, title, journal year & volume, number, pages, month, note\\  \hline
tese (doutorado) & @phdthesis & author, title, school, year & type, address, month, note\\  \hline
dissertação (mestrado) & @mastersthesis & author, title, school, year & type, address, month, note\\  \hline
anais de conferência & @proceedings & title, year & editor, volume, number, series, address, month, organization, publisher, note\\  \hline
artigo no anais de conferêcnia & @inproceedings & author, title, booktitle, year & editor, volume, number, series, pages, address, month, organization, publisher, note\\  \hline
mesmo que inproceedings & @conference & author, title, booktitle, year & editor, volume, number, series, pages, address, month, organization, publisher, note\\  \hline
livro & @book & author ou editor, title, publisher, year & volume, number, series, address, edition, month, note\\  \hline
quase livro (notas e simlares) & @booklet & title & author, howpublished, address, month, year, note\\  \hline
parte de um livro & @inbook & author ou editor, title, chapter ou pages, publisher, year & volume, number, series, type, address, edition, month, year\\  \hline
parte de uma coleção & @incollection & author, title, booktitle, publisher, year & editor, volume, number, series, type, chapter, pages, address, edition, month, note\\  \hline
manual & @manual & title & author, organization, address, edition, month, year, note\\  \hline
relatório técnico & @techreport & author, title, institution, year & type, number, address, month, note\\  \hline
não pblicado & @unpublished & author, title, note & month, year\\  \hline
outros & @misc & nenhuma & author, title, howpublished, month, year, note\\ \hline
\end{longtable}

O texto sem estar com especificação que inicia com ``@'' será ignorado pelo \BibTeX{} e funciona como comentário\index{BibTeX!comentário}.
Mas existe o especificador \texttt{@comment}\index{BibTeX!arrobacomment@{\arrobasymbforindex comment}} especial para inserir comentários\index{BibTeX!comentário} para fins de organização.

Como existem muitas categorias e cada categoria tem campos diferentes a ser preenchidos, editar o arquivo de \BibTeX{} manualmente não é muito simples.
Assim, costumamos usar os aplicativos próprios para isso. 
Como foi dito antes, um desses aplicativos é o \texttt{jabref}\index{jabref} (\url{http://www.jabref.org/}) que funciona em quase toda plataforma e é gratuito. 

\section{Índice remissivo}

Para criar o índice remissivo, usa-se o pacote \texttt{makeidx}\index{makeidx} e o comando \verb+\makeindex+\index{makeindex@\verb+\makeindex+} no \texttt{preamble} do documento para ativar o seu uso.

Para acrescentar itens no índice remissivo\index{indice@{indice} remissivo}, usa-se o comando \verb+\index{}+\index{index@\verb+\index+}, mas lembre-se de não colocar espaços entre palavra referida e o comando.
Para colocar índice remissivo com sub-entrada\index{indice@{índice} remissivo!sub-entrada}, usa-se o ``!'' para separar entrada e sub-entrada, mas evite espaços nos lados de ``!'' (qualquer espaço extra pode causar confusões na organização do índice remissivo).

As vezes, é necessário usar uma ``chave'' para classificar os itens\index{indice@{índice} remissivo!com chave}.

Isto ocorre quando usamos símbolos matemáticos, aspas, ou outros caracteres especiais, formatação de caracteres, etc.
Sem a ``chave'' de classificação, \texttt{makeindex} pode classificar usando o que está escrito, o que nem sempre corresponde a posição correta. 
Para resolver este problema, usa-se o \verb+\index{chave@item}+.
Por exemplo
\begin{singlespacing}
\begin{lstlisting}
Ps ítens em negrito\index{item em negrito@\textbf{ítem em negrito}},
itens com aspas\index{item com aspas@``ítem com aspas''} e símbolos \index{simbolos@{símbolos}}, etc podem ser colocados com ordenação correta.
\end{lstlisting}
\end{singlespacing}
inserem entradas em negrito e item com aspas classificado corretamente.
Lembre também que as letras acentuadas costumam vir na posição diferente das não acentuadas.
Neste caso também poderá usar a chave para colocar na posição correta.
Veja o Exemplo~\ref{ex11:index}.

\begin{singlespacing}
%\begin{lstlisting}[pos=b,caption={ex05-02.tex},label=ex:05-02]
\begin{lstlisting}[caption={ex11-index.tex},label=ex11:index]
Podemos usar índice remissivo\index{índice remissivo} com sub-entrada\index{índice remissivo!sub-entrada}.
Também podemos usar ``chave'' para classificar os itens, como itens em negrito\index{item em negrito@\textbf{item em negrito}}, itens com aspas\index{item com aspas@``item com aspas''} e símbolos como $\alpha$\index{alfa@$\alpha$}.
Também podemos usar item e enumeração em negrito\index{item e enumeração em negrito@\textbf{item e enumeração em negrito}|textbf}
\printindex % colocar indice remissivo (glossário) aqui.
\end{lstlisting}
\end{singlespacing}

No corpo do documento, terá nenhuma diferença visual em ter o \verb+\index{}+ e logo, a sua saída foi omitida.
  
Podemos especificar sub-entrada com chaves como em

\begin{singlespacing}
\begin{lstlisting}
itens em negrito com sub-entrada\index{item@\textbf{item}!\textbf{em negrito}!\textbf{com sub-entrada}}. 
\end{lstlisting}
\end{singlespacing}

Observe a ordem que usa o ``@'' e ``!'' na especificação de itens com chave e subentrada.

Para indicar a ocorrência num trecho\index{indice remmissivo!trecho}, usa-se o \verb+\index{item|(+ para iniciar e \verb+\index{item|)+ para finalizar.

Quando um item referê-se ao outro, usa-se 
o \verb+\index{ítem|see{referência}}+ com em
\begin{lstlisting}
grossário\index{grossário|see{índice remissivo}}.
\end{lstlisting}

Para mudar a fonte usada na enumeração da página, coloque a especificação após ``|'' ou ``|('' dependendo de ser página ou trecho como em
\begin{lstlisting}
Índice com paginação em negrito\index{item e enumeração em negrito@\textbf{ítem e enumeração em negrito}|textbf}.
\end{lstlisting}

Para colocar o índice remissivo gerado, 
usa-se o comando \verb+\printindex+\index{printindex@\verb+\printindex+}.

Finalmente, para produzir o índice remissivo, deverá executar \LaTeX, \verb+makeindex+\index{makeindex} e \LaTeX{} novamente. 
Os editores especializados para \LaTeX{} costuma vir com botão ou menu para chamar o \texttt{makeindex}.

Note que, por padrão, o processador \texttt{makeindex} aceita somente até 3 níveis de sub-entrada.
Caso queira mais níveis, poderá optar por usar \texttt{xindy}\index{xindy} em vez de \texttt{makeindex} com configuração adequada.
O \texttt{xindy} suporta indexação internacional, além de diversas configurações, entre eles, aumentar o nível de sub-entradas via arquivo de configuração.
No entanto, dependendo do editor, pode não vir com botão para executar o \texttt{xindy}.

O índice remissivo do Exemplo~\ref{ex11:index} ficaria como segue

%\begin{singlespacing}
%\fbox{\begin{minipage}{\linewidth}
\begin{ltxlistingout}
\chapter*{Índice Remissivo}
\noindent índice remissivo, 1

\hspace{2em} sub-entrada, 1
   
  \medskip

\noindent $\alpha$, 1

  \medskip

\noindent ``item com aspas'', 1

\noindent \textbf{item em negrito}, 1

\noindent \textbf{item e enumeração em negrito}, \textbf{1}
\end{ltxlistingout}
%\end{minipage}}
%\end{singlespacing}

Note que @, !, | e \verb+"+ tem significados na entrada de índice remissivo. 
Para colocar estes caracteres na entrada de índice remissivo, coloque \verb+"+ antes dele.
Por exemplo, \verb+\index{"@author}+ insere ``@author'' na entrada de índice remissivo.

\clearpage\thispagestyle{empty}\cleardoublepage
\chapter{Medidas e Contadores}\label{chap:medidas:contadores}
Neste capítulo, veremos medidas e contadores.

\section{Unidade de medidas e espaçamentos}\label{sec:medidas:unidade}

As vezes o espaçamento ajustado automaticamente não está bom e queremos fazer pequenos ajustes.

Para aumentar espaçamento entre parágrafos em um determinado ponto, como entre texto e equações altas, poderá usar o \verb+\smallskip+\index{amallskip@\verb+\smallskip+}, \verb+\medskip+\index{medskip@\verb+\medskip+} e \verb+\bigskip+\index{bigskip@\verb+\bigskip+} para ter espaçamentos maior que o padrão.
Note que, se quer ter espaçamento maior entre texto e equações em todo o documento, deverá efetuar ajustes de parâmetros \verb+\abovedisplayskip+\index{abovedisplayskip@\verb+\abovedisplayskip+} e \verb+\belowdisplayskip+\index{belowdisplayskip@\verb+\belowdisplayskip+}.

Para espaço horizontal e vertical, usa-se o \verb+\hspace{medida}+\index{hspace@\verb+\haspace+} e \verb+\vspace{medida}+\index{vspace@\verb+\vspace+} respectivamente.
Caso estiver inserindo o espaço vertical no começo das páginas, ou espaço horizontal no começo das linhas, use a versão com ``*'' \verb+\hspace*{medida}+\index{hspace*@\verb+\hspace*+} e \verb+\vspace*{medida}+\index{vspace*@\verb+\vspace*+}.

Medidas comumente usados são:

\begin{description}
\item[pt] (point - unidade gráfica) = $\frac{1}{72.27}$in (polegada) ou $0.351$mm
\item[mm] (milimetro) = $2.845$pt
\item[pc] (pica) = $12$pt ou $4.218$mm
\item[cm] (centímetro) = $2.371$pc
\item[in] (polegada) = $25.4$mm ou $72.27$pt ou $6.022$pc
\item[ex] altura da letra ``x'' minusculo da fonte corrente 
\item[em] largura da letra ``M'' maiúsculo da fonte corrente
\item[mu] (math unit) = $\frac{1}{18}$em
\item[\texttt{\textbackslash stretch\{peso\}}] espaço esticável com peso especificado.
\end{description}

Existem ainda várias ouras medidas menos usadas que foram omitidos aqui.

O \verb+\stretch{peso}+\index{stretch@\verb+\stretch+} produz medida que preenche o espaço.
Existem alguns comandos deste tipo que são úteis: \verb+\hfill+\index{hfill@\verb+\hfill+} (equivale a \verb+\hspace{stretch{1}}+),
\verb+\vfill+\index{vfill@\verb+\vfill+} (equivale a \verb+\vspace{stretch{1}}+),
\verb+\hrulefill+\index{hrulefill@\verb+\hrulefill+} (similar a \verb+\hfill+, mas preenche com linha),
\verb+\dotfill+\index{dotfill@\verb+\dotfill+} (similar a \verb+\hfill+, mas preenche com pontos).

Ainda existem comandos para inserir pequenos espaços, que são 
\verb|\quad|\index{quad@\verb+\quad+} (insere espaço de $1$em = $18$mu),
\verb|\qquad|\index{qquad@\verb+\qquad+} (insere espaço de $2$\texttt{quad}=$2$em),
\verb|\enspace|\index{enspace@\verb+\enspace+} ou \verb|\enskip|\index{enskip@\verb+\enskip+} (insere espaço de $\frac{1}{2}$\texttt{quad}).
Estes comandos costumam ser usados para inserir pequenos espaçamentos para melhora a aparência das fórmulas matemáticas.
No modo matemático, ainda existem comandos de inserção de espaços menores que são
\begin{description}
\item[\textbackslash,]\index{,@\verb+\,+} ($\frac{3}{18}$\texttt{quad} = $3$mu),
\item[\textbackslash:]\index{:@\verb+\:+} ($\frac{4}{18}$\texttt{quad} = $4$mu),
\item[\textbackslash;]\index{;@\verb+\;+} ($\frac{5}{18}$\texttt{quad} = $5$mu), 
\item[\textbackslash!]\index{"!@{\textbackslash\exclsymbforindex}} ($\frac{-3}{18}$\texttt{quad} = $-3$mu).
\end{description}

Veja Exemplo~\ref{ex12:espacos}.

%\begin{singlespacing}
%\begin{LTXexample}[pos=b,caption={ex03-03.tex},label=ex:03-03]
\begin{ltxlisting}[caption={ex12-espacos.tex},label=ex12:espacos]
palavra a esquerda \hfill palavra a direita.

palavra a esquerda \hrulefill palavra a direita.

palavra a esquerda \dotfill palavra a direita.

primeiro\hfill segundo\hrulefill\hrulefill terceiro\dotfill\dotfill\dotfill quarto

\bigskip

$x \in \mathrm{I\!R}, \enspace x>0$.
\end{ltxlisting}
%\end{LTXexample}
%\end{singlespacing}

Nota: O \verb+\mathbf{R}+ construído acima é apenas como ilustração. 
Deverá usar a fonte apropriada tal como \verb+\mathbb+ do AMS.

Outra coisa que as vezes usamos para alinhamento dos elementos de fórmulas é o comando \verb+\phantom+\index{phantom@\verb+\phantom+} que, em vez de produzir elementos da fórmula, reserva o espaço usado por ele.
Por exemplo, $\Gamma_{ij}^{\phantom{ij}k}$ e $\Gamma_{ij}^k$. produzido respectivamente por \verb+$\Gamma_{ij}^{\phantom{ij}k}$+ e \verb+$\Gamma_{ij}^k$+.
 

\section[Medidas]{Medidas prédefinidas ou definidos pelo usuário}\label{sec:medidas}
Existem algumas medidas pré-definidas como comandos, relacionadas à configuração das páginas e similares que costuma ser usados com certa frequência. 
Aqui, vamos citar algumas delas.
%Para referenciar, coloque ``\textbackslash'' antes dele, para ficar como comandos
\begin{description}
\item[\textbackslash textwidth]\index{textwidth@\verb+\textwidht+} Largura de texto atual.
\item[\textbackslash linewidth]\index{linewidth@\verb+\linewidth+} Largura da linha atual. Por exemplo, dentro da lista, será menor que o \verb+\textwidth+. 
\item[\textbackslash columnwidth]\index{columnwidth@\verb+\columnwidth+} Largura da coluna. Se for em uma coluna, coincide com \verb+\textwidth+.
\item[\textbackslash columnsep]\index{columnsep@\verb+\columnsep+} Distância entre colunas no modo multi colunas.
\item[\textbackslash columnseprule]\index{columnseprle@\verb+\columnseprule+} Largura da linha que separa colunas no modo multi colunas (\verb+0pt+ para desabilitar).
\item[\textbackslash textheight]\index{textheight@\verb+\textheight+} Altura do texto atual.
\item[\textbackslash parindent]\index{parindent@\verb+\parindent+} Indentação (quanto deixa no lado esquerdo) do parágrafo.
\item[\textbackslash parskip]\index{parskip@\verb+\parskip+} Quanto deixa de espaço antes do parágrafo.
\item[\textbackslash paperwidth]\index{paperwidth@\verb+\paperwidth+} Largura do papel.
\item[\textbackslash paperheight]\index{paperheight@\verb+\paperheight+} Altura do papel.
\item[\textbackslash unitlength]\index{unitlength@\verb+\unitlength+} Medida de unidade usado no ambiente \texttt{picture}.
\end{description}

Dentro do \texttt{minipage}, \verb+\textwidth+, \verb+\linewidth+ e \verb+\columnwidth+ assumem a largura do \texttt{minipage}, mas no \verb+\parbox+, somente \verb+\linewidth+ assumirá a largura da caixa.

Para imprimir estas medidas, coloque o comando \verb+\the+\index{the@\verb+\the+} antes do comando de medidas.
Por exemplo, para imprimir o valor de \verb+\textwidth+, use \verb+\the\textwidth+.

Para definir uma nova medida, usa-se o comando \verb+\newlength{\minhamedida}+\index{newlength@\verb+\newlength+}. 
Este comando criará a medida \verb+\minhamedida+. 
Para definir ou altera o valor da medida, use \verb+\setlength{\minhamedida}{medida}+\index{setlength@\verb+\setlength+} onde \texttt{medida} é a medida da \verb+\minhamedida+.

O valor da medida pode ser configurado a partir de \texttt{box} (caixa) existente (como texto, fórmulas, etc).
Uma caixa (de elemento) tem altura (\texttt{height}), largura (\texttt{width}) e a distância da parte inferior até a linha base (\texttt{depth}). 

\verb+\settoheight{\minhamedida}{<conteúdo>}+\index{settoheight@\verb+\settoheigth+}, \verb+\settowidth{\minhamedida}{<conteúdo>}+\index{settowidth@\verb+\settowidth+} e \verb+\settodepth{\minhamedida}{<conteúdo>}+\index{settodepth@\verb+\settodepth+} configuram o valor da \verb+\minhamedida+ para essas medidas correspondentes ao seu parâmetro.

Note que, pela facilidade, as vezes usamos o modo \TeX{} para configurar as medidas como em
\verb+\minhamedida=3.0cm+
ou até mesmo, omitir ``\verb+=+'' e escrever como 
\verb+\minhamedida 3.0cm+

 Para facilitar os cálculos das medidas, normalmente usa-se o pacote \texttt{calc}.
 
 O pacote \texttt{calc}, além de permitir calcular medidas com facilidade, ainda acrescentam alguns novos comandos tais como
\verb+\settototalheight{\algumamedida}{<conteúdo>}+\index{settototalheight@\verb+\settototalheight+} que configura \verb+\algumamedida+ para \texttt{width+depth} e
\verb+\widthof{<conteúdo>}+\index{widthof@\verb+\widthof+}, \verb+\heightof{<conteúdo>}+\index{heightof@\verb+\heightof+}, \verb+\depthof{<conteúdo>}+\index{depthof@\verb+\depthof+} e \verb+\totalheightof{<conteúdo>}+\index{totalheightof@\verb+\totalheightof+} 
que retornam \texttt{width}, \texttt{height}, \texttt{depth}, e \texttt{width+depth}, respectivamente.

\section{Contadores}

\LaTeX{} usa diversos contadores para efetuar enumeração automática, tais como páginas, equações enumeradas, capítulos e seções, figuras, etc.

Estes contadores podem ser controladas, tanto na contagem como na sua aparência.

Em geral, o costume é nomear contador com mesmo nome do ambiente e o que precede \verb+\the+ será usado para imprimir o seu valor.
Por exemplo, a impressão de enumeração da página usa o \verb+\thepage+\index{thepage@\verb+\thepage+} e enumeração do capítulo usa \verb+\thechapter+\index{thechapter@\verb+\thechapter+} e assim por diante.

Os principais contadores pré definidos no \LaTeX{} são:
\texttt{part}, \texttt{chapter},
\texttt{section}, \texttt{subsection}, \texttt{subsubsection},
\texttt{paragraph}, \texttt{subparagraph},
\texttt{page}, \texttt{figure}, \texttt{table},
\texttt{footnote}, \texttt{mpfootnote} (rodapé dentro do minipage) e
\texttt{equation}.

Então, alterando o \verb"\the<contador>"\index{thecontador@\verb+\the<contador>+}, alterará como será impresso estes contadores.

Por exemplo, \verb+\renewcommand{\thepage}{\roman{page}}+ alterará a enumeração das páginas para i, ii, iii, etc.

O estilo de enumeração \verb+\roman+ acima, podem ser
\begin{description}
\item[\textbackslash\texttt{arabic}]\index{arabic@\verb+\arabic+} que é 1, 2, 3 ...
\item[\textbackslash\texttt{alph}]\index{alph@\verb+\alph+} que é a, b, c ... 
\item[\textbackslash\texttt{Alph}]\index{Alph@\verb+\Alph+} que é A, B, C ... 
\index[\textbackslash\texttt{roman}]\index{roman@\verb+\roman+} que é i, ii, iii ...
\index[\textbackslash\texttt{Roman}]\index{Roman@\verb+\Roman+} que é I, II, III ...
\index[\textbackslash\texttt{fnsymbol}]\index{fnsymbol@\verb+\fnsymbol+} que é sequencia de símbolos (pode ser usado no rodapé, se for pouco).
\end{description}
Veja Exemplo~\ref{ex12:contadores}.
%\begin{singlespacing}
%\begin{LTXexample}[pos=b,caption={ex10-04.tex},label=ex:10-04]
\begin{ltxlisting}[caption={ex12-contadores.tex},label=ex12:contadores,preset={\setcounter{section}{\value{lstorisection}}}]
A página atual é \thepage. 

Seção atual em romano minúsculo é \roman{section}.
\end{ltxlisting}
%\end{LTXexample}
%\end{singlespacing}

As vezes, precisamos ajustar valores iniciais dos contadores, tais como das páginas, listas enumeradas, etc.
Para isso, usamos os comandos \verb+\setcounter+\index{setcounter@\verb+\setcounter+}, \verb+\stepcounter+\index{stepcounter@\verb+\stepcounter+} e \verb+\addtocounter+\index{addtocounter@\verb+\addtocounter+}.
Para decrementar, use o valor negativo no parâmetro de \verb+\addtocounter+\index{addtocounter@\verb+\addtocounter+}.

No caso de listas enumeradas, os contadores são
\texttt{enumi}\index{enumi}, \texttt{enumii}\index{enumii} (sub lista),
\texttt{enumiii}\index{enumiii} (subsub lista), \texttt{enumiv}\index{enumiv} (subsubsub lista).

Para que estes contadores sejam impressos na fonte reta, mesmo no ambiente em itálico (enunciado do teorema, por exemplo), acrescente

\begin{singlespacing}
\begin{lstlisting}
\renewcommand{\labelenumi}{\textup{\theenumi.}}
\renewcommand{\labelenumii}{\textup{(\theenumii)}}
\renewcommand{\labelenumiii}{\textup{\theenumiii.}}
\renewcommand{\labelenumiv}{\textup{\theenumiv.}}
\end{lstlisting}
\end{singlespacing}

no \texttt{preamble} do documento.

Além disso, o contador novo pode ser criado por \verb+\newcounter+\index{newcounter@\verb+\newcounter+}, útil para ser usado no ambiente/comando novo a ser criado.

Aqui, vamos criar um contador para testá-los.
Veja o Exemplo~\ref{ex12:newcounter}.

%%\let\ref\plainref
%\let\orilabel\label
%\begin{singlespacing}
%\begin{LTXexample}[pos=b,caption={ex10-04.tex, parte b},label=ex:10-04:b,preset=\let\label\orilabel]
\begin{ltxlisting}[caption={ex12-newcounter.tex},label=ex12:newcounter]
\newcounter{teste} % cria e inicializa com o valor zero.
\setcounter{teste}{1} % novo valor

O contador é \theteste.

O contador em \texttt{Alph} é \Alph{teste}.

Adicionando 1 e usando.
\refstepcounter{teste}\label{count:teste}.
Novo valor é \theteste. 

Adicionando por 1 de novo \addtocounter{teste}{1}
Novo valor é \theteste. 
O valor atribuído no rótulo \texttt{count:teste} após o \texttt{refstepcounter} é \ref{count:teste}.

Agora, a lista enumerada começando de $3$
\begin{enumerate}\setcounter{enumi}{2}
\item Um item.
\item Outro item.
\end{enumerate}
\end{ltxlisting}
%\end{LTXexample}
%\end{singlespacing}
%%\let\ref\oriref

Para que o \verb+\label{}+ pegue o valor atual do contador, use o \verb+\refstepcounter+\index{refstepcounter@\verb+\refstepcounter+} que incrementa o contador por um e atualiza o valor para \verb+\label+.

Para pegar o valor de um contador (para ser usado como argumento para \verb+\setcounter+, por exemplo), use o comando \verb+\value+\index{value@\verb+\value+}.

Muitas vezes, um contador está vinculado no outro e quando outro for incrementado, ele será reinicializado. 
Além disso, contadores vinculados aos outros imprimem dois valores (contador do outro mais dele).

Este vínculo pode ser criado pelo comando
\verb+\numberwithin{<contador>}{<pai>}+\index{numberwithin@\verb+\numberwithin+}. 
Quando o \verb+<pai>+ for incrementado, o \verb+<contador>+ é reinicializado. 
Por exemplo, para que a enumeração da equação seja da forma \texttt{<no. do capítulo>.<no da equação>}, basta colocar \verb+\numberwithin{equation}{chapter}+ no \texttt{preamble} do documento.

Para eliminar um vínculo, usa-se o comando \verb+\@removefromreset+\index{arrobaremovefromreset@\arrobasymbforindex\verb+removefromreset+}. Por exemplo, pare remover o vínculo do contador da equação com o contador da seção, coloque o código

\begin{singlespacing}
\begin{lstlisting}
\makeatletter
\@removefromreset{equation}{section}
\makeatother
\end{lstlisting}
\end{singlespacing}
no \texttt{preamble} do documento.

Mais sobre manipulações dos contadores, veja o \cite[Capítulo 3, Seção 3]{book:wikibook:latex}.

%\url{https://en.wikibooks.org/wiki/LaTeX/Counters}.


%\chapter{Teste}
%
%\LTXinputExample{test-code}
%% Value for pos
%% t,b,l,r,o, and i
%%for top, bottom, left, right, outer, and inner.
%\begin{LTXexample}[pos=b,caption={Caption of LTXexample},label=ex:start]
%\documentclass{article}
%\usepackage[utf8]{inputenc}
%\usepackage{lipsum}
%
%\begin{document}
%% comentário
%maçã
%\lipsum[1-2]
%\end{document}
%\end{LTXexample}
%
%% truque do https://www.latex4technics.com/?note=30gq
%% para referencia fucnionar no LTXexample
% \let\orilabel\label
%\begin{LTXexample}[frame=none,preset=\let\label\orilabel]
%  \begin{enumerate}
%  \item No need to manually number items here!
%  \item\label{second} Number two added automatically
%  \item Third item
%  \end{enumerate}
%  And here is a reference to item number \ref{second}
%\end{LTXexample}

\clearpage\thispagestyle{empty}\cleardoublepage
\chapter{Mais Alguns Cuidados e Ajustes}
Neste capítulo, veremos mais alguns ajustes e incrementos no documentos.

\section{Comandos frágeis}
Existem comandos denominados frágeis por poder causar problemas quando é passado como parâmetros de alguns comandos ou ambientes. 
Por exemplo, o argumento do \verb+\chapter+ e \verb+\section+ costumam ser usado no sumário também. 
Se colocar comando que tem contadores como o \verb+\footnote+, causará problemas, pois tentará colocar rodapé também no sumário.

Assim, quando colocar comandos no argumento da função que usam o seu parâmetro em mais de um lugar, requer cuidados.
Além dos comandos de seccionamento (\texttt{chapter}, \texttt{section}, etc), \texttt{caption}, \texttt{thanks}, comandos que produz saída do cabeçalho como \verb+\markboth+, etc, também usam o seu argumento em mais de um lugar.

Os comandos que causam problemas quando é passado para argumento destes comandos são chamados de comandos frágeis\index{comandos frágil} e deve ser precedido de \verb+\protect+\index{protect@\verb+\protect+} para prevenir problemas.

Alguns comandos frágeis são: comandos com argumento opcional, ambientes, fórmula no modo \texttt{displaystyle}, fórmula no modo \texttt{textstyle} delimidado por \verb+\(+ e \verb+\)+ (no entanto, delimitado por ``\$'' não é frágil), \verb+\phantom+, ``\verb+\\+'', \verb+\item+ e \verb+\footnote+.

Veja Exemmplo~\ref{ex13:fragil}.

%\setcounter{lstorichapter}{\value{chapter}}
%\setcounter{lstorisection}{\value{section}}
%% \let\orilabel\label
%\begin{singlespacing}
%\begin{LTXexample}[pos=b,caption={ex06-08.tex, parte b},label=ex:06-08:b,preset={\setcounter{chapter}{1},\setcounter{section}{0},\let\addcontentsline\emptyaddcontentsline}]
\begin{ltxlisting}[caption={ex13-fragil.tex},label=ex13:fragil,preset={\setcounter{chapter}{1}}]
\section{Sublinhando texto\protect\footnote{Forma antiga de enfatizar o texto}} % footnote é frágil: usar \protect
\ldots
\section{Sobre a famosa fórmula $a^2=b^2+c^2$} % delimitado por "$" não é frágil
\ldots
\begin{figure}[hbp!]
\center
 Figura aqui.
 \caption{$\Gamma_{ij}^{\protect\phantom{ij}2}$} % \phantom é frágil
\end{figure}
\ldots
\end{ltxlisting}
%\end{LTXexample}
%\end{singlespacing}
%%\let\ref\oriref
%\setcounter{chapter}{\value{lstorichapter}}
%\setcounter{section}{\value{lstorisection}}
%% \addtocounter{chapter}{-1}
%\let\addcontentsline\oriaddcontentsline

\section{Babel e nomes}

O pacote de internacionalização \texttt{babel}\index{babel}
oferece regra de hifenização\index{hifenização} e nomes dos elementos tais como \texttt{capítulo}, \texttt{figura}, \texttt{tabela}, \texttt{sumário}, etc em vários idiomas\index{idioma}.
Para carregar mais de um idioma, coloque os idiomas separado pela vírgula no parâmetro do \texttt{babel}. 
O idioma do último será considerado padrão e outros são opcionais que podem ser ativados quando quiser.

Seleção de idiomas pode ser mudado por \verb+\selectlanguage{<idioma>}+\index{selectlanguage@\verb+\selectlanguage+}.
Para demarcar somente um trecho no outro idioma, usa-se o ambiente \texttt{otherlanguage}\index{otherlanguage}.
Para trecho bem curto, poderá usar  também o comando \verb+\foreignlanguage+\index{foreignlanguage@\verb+\foreignlanguage+}.
Veja Exemplo~\ref{ex13:babel}.

%\setcounter{lstorichapter}{\value{chapter}}
%\setcounter{lstorisection}{\value{section}}
%% \let\orilabel\label
%\let\orithesection\thesection
%\renewcommand{\thesection}{\arabic{section}}
%%\newcommand{\articleappendix}{\setcounter{section}{0}\def\thesection{\Alph{section}}}
%\begin{singlespacing}
%\begin{LTXexample}[pos=b,caption={ex10-03.tex},label=ex:10-03,preset={\setcounter{chapter}{1},\setcounter{section}{0},\let\addcontentsline\emptyaddcontentsline,\let\title\orititle,\let\author\oriauthor,\let\date\oridate,\let\maketitle\makearticletitle}]
\let\thesectionoriginal\thesection
\begin{ltxlisting}[caption={ex13-babel.tex},label=ex13:babel,preset={\def\thesection{\arabic{section}}}]
\documentclass[12pt,a4paper]{article}
\usepackage[T1]{fontenc}
\usepackage[utf8]{inputenc}
\usepackage{amsmath,amssymb}
\usepackage[english,brazil]{babel}
% Dados para títulos
\title{Exemplo}
\author{Sadao Massago} % caso de mais de um autor, separe com o comando \and
\date{Fevereiro, 2018}
\begin{document}
\maketitle
\begin{abstract}
Resumo aqui.
\end{abstract}
\begin{otherlanguage}{english}
\begin{abstract}
Abstract here.
\end{abstract}
\end{otherlanguage}
% \tableofcontents % so se for artigo longo
\section{Mudando o idioma de um trecho}
O trecho curto pode ser no outro idioma como em ``\foreignlanguage{english}{This is english}''.
Para trechos maiores, use o ambiente \texttt{otherlanguage} como em
\begin{quote}
\begin{otherlanguage}{english}
This environment switches all language-related definitions, like the language specific names for figures, tables etc. to the other language.
\end{otherlanguage}
\end{quote}
\ldots
% referência biliográfica
\end{document}
\end{ltxlisting}
\let\thesection\thesectionoriginal
%\end{LTXexample}
%\end{singlespacing}
%%\let\ref\oriref
%\setcounter{chapter}{\value{lstorichapter}}
%\setcounter{section}{\value{lstorisection}}
%% \addtocounter{chapter}{-1}
%\let\thesection\orithesection
%\let\addcontentsline\oriaddcontentsline

Para escrever o nome tais como ``Capítulo'', ``Figura'', ``Sumário'', etc, tem os comandos que produz estes nomes.
Redefinindo apropriadamente estes comandos, podemos alterar a sua saída.

Alguns nomes pré definidos são:
\verb+\abstractname+\index{abstractname@\verb+\abstractname+} (somente \texttt{article} e \texttt{report}), \verb+\appendixname+\index{appendixname@\verb+\appendixname+},
\verb+\bibname+\index{bibname@\verb+\bibname+} (somente \texttt{book} e \texttt{report}),
\verb+\chaptername+\index{chaptername@\verb+\chaptername+} (somente \texttt{book} e \texttt{report}),
\verb+\contentsname+\index{contentsname@\verb+\contentsname+},
\verb+\figurename+\index{figurename@\verb+\figurename+}, \verb+\indexname+\index{indexname@\verb+\indexname+},
\verb+\listfigurename+\index{listfigurename@\verb+\listfigurenname+},
\verb+\listtablename+\index{listtablename@\verb+\listtablename+},
\verb+\partname+\index{partname@\verb+\partname+},
\verb+\refname+\index{refname@\verb+\refname+} (somente \texttt{article}),
\verb+\tablename+\index{tablename@\verb+\tablename+}.

Note que o nome para referência bibliográfica no livro e relatório são \verb+\bibname+\index{bibname@\verb+\bibname+} enquanto que no artigo, é \verb+\refname+\index{refname@\verb+\refname+}.

Alguns pacotes definem o nome usado para seus pacotes. 
Por exemplo, \texttt{amsthm} que define o ambiente \texttt{proof}, define também \verb+\proofname+\index{proofname@\verb+\proofname+}.

Para alterar ou definir novos nomes quando usa o babel, requer usar o recurso do babel, pois nomes pré definidos no babel são redefinidos dinamicamente quando altera os idiomas.

Assim, deverá definir/redefinir os nomes dentro do comando \verb+\captions<idioma>+\index{caption<idioma>@\verb+\captionn<idioma>+} onde \texttt{<idioma>} é o nome do idioma. 
Mas, se redefinir o \texttt{captions}, perderá as definições anteriores feitas pelo \texttt{babel}. 
Logo, utiliza o comando \verb+\addto+\index{addto@\verb+\addto+} do \texttt{babel} que acrescenta porções de código no final do comando especificado.

Por exemplo,

\begin{singlespacing}
\begin{lstlisting}
\addto\captionsbrazil{%
  \renewcommand\refname{Referências Bibliográficas}%
}
\end{lstlisting}
\end{singlespacing}

no \texttt{preamble} altera o nome da referência bibliográfica no caso de artigos quando usa o idioma \texttt{brazil}.

Note que, no caso de português brasileiro, tanto pode usar o \texttt{brazil}\index{brazil} como o \texttt{brazilian}\index{brazilian}. 
Assim, precisará redefinir em ambos idiomas, se estiver implementando um pacote na qual não sabe qual opção o usuário final vai usar.
Neste caso, define um comando auxiliar e coloca em cada um dos idiomas.

\begin{singlespacing}
\begin{lstlisting}
\providecommand{\theoremname}{Theorem} % Providenciando novo nome com valor padrão
\addto\braziliannames{%
  \renewcommand\refname{Referências Bibliográficas}%
  \renewcommand{\theoremname}{Teorema}%  
}
\newcommand{\englishnames}{%
  \renewcommand{\theoremname}{Theorem}%
}

\addto\captionsbrazil{\braziliannames}
\addto\captionsbrazilian{\braziliannames}

\addto\captionsenglish{\englishnames}
\addto\captionsamerican{\englishnames}

\newtheorem{theorem}{\theoremname}[section]
\end{lstlisting}
\end{singlespacing}
%
no \texttt{preamble} após carregar o pacote \texttt{babel}
habilita suporte para português brasileiro e inglês.
Note que no \verb+\newtheorem+, está usando o comando \verb+\theoremname+ que será redefinido automaticamente quando o idioma muda.
Veja o exemplo~\ref{ex13:babel:caption}.

%\setcounter{lstorichapter}{\value{chapter}}
%\setcounter{lstorisection}{\value{section}}
%% \let\orilabel\label
%\let\orithesection\thesection
%\renewcommand{\thesection}{\arabic{section}}
%%\newcommand{\articleappendix}{\setcounter{section}{0}\def\thesection{\Alph{section}}}
\begin{singlespacing}
%\begin{lstlisting}[caption={ex10-03.tex, parte b},label=ex:10-03:b,preset={\setcounter{chapter}{1},\setcounter{section}{0},\let\addcontentsline\emptyaddcontentsline,\let\title\orititle,\let\author\oriauthor,\let\date\oridate,\let\maketitle\makearticletitle}]
\begin{lstlisting}[caption={ex13-babel-caption.tex},label=ex13:babel:caption]
\documentclass[12pt,a4paper]{article}
\usepackage[T1]{fontenc}
\usepackage[utf8]{inputenc}
\usepackage{amsmath,amssymb}
\usepackage{amsthm}
\usepackage[english,brazil]{babel}
\providecommand{\theoremname}{Theorem} % Providenciando novo nome com valor padrão
\addto\braziliannames{%
  \renewcommand\refname{Referências Bibliográficas}%
  \renewcommand{\theoremname}{Teorema}%  
}
\newcommand{\englishnames}{%
  \renewcommand{\theoremname}{Theorem}%
}

\addto\captionsbrazil{\braziliannames}
\addto\captionsbrazilian{\braziliannames}

\addto\captionsenglish{\englishnames}
\addto\captionsamerican{\englishnames}

\newtheorem{theorem}{\theoremname}[section]

\title{Exemplo}
\author{Sadao Massago}
\date{Fevereiro, 2018}
\begin{document}
\maketitle
\begin{abstract}
Resumo aqui.
\end{abstract}
% \tableofcontents % so se for artigo longo
\section{Teoremas multi lingue}
\begin{otherlanguage}{english}
\begin{theorem}
\ldots
\end{theorem}
\end{otherlanguage}
\begin{theorem}
\ldots
\end{theorem}
\begin{thebibliography}{99}
\bibitem[GMS04]{Goossens:2004}
 Michel Goossens and Frank Mittelbach
 \emph{The {\LaTeX} companion (second edition)}, Adilson--Wesley, 2004.
\end{thebibliography}
\end{document}
\end{lstlisting}
\end{singlespacing}
%%\let\ref\oriref
%\setcounter{chapter}{\value{lstorichapter}}
%\setcounter{section}{\value{lstorisection}}
%% \addtocounter{chapter}{-1}
%\let\thesection\orithesection
%\let\addcontentsline\oriaddcontentsline
%
%
%\let\title\orititle
%\let\author\oriauthor
%\let\date\oridate
%\title{Exemplo}
%\author{Sadao Massago}
%\date{Fevereiro, 2018}
%\let\maketitle\makearticletitle
%%
%\noindent
%\let\orithesection\thesection
%\setcounter{lstorisection}{\value{section}}
%\def\thesection{\arabic{section}}
%\setcounter{section}{0}
%\let\addcontentsline\emptyaddcontentsline
%%
%\fbox{\begin{minipage}{\linewidth}

\let\orithesection\thesection
\begin{ltxlistingout}[preset={\title{Exemplo}\author{Sadao Massago}\date{Fevereiro, 2018}}\renewcommand{\thesection}{\arabic{section}}]
\maketitle
\begin{abstract}
Resumo aqui.
\end{abstract}
% \tableofcontents % so se for artigo longo
\section{Teoremas multi lingue}

\textbf{Theorem \arabic{section}.1.} \ldots

\textbf{Teorema \arabic{section}.2.} \ldots

\section*{Referências Bibliográficas}
\begin{list}{--}{%
    %\setlength{\topsep}{2pt}%
    %\setlength{\itemsep}{0pt}%
    %\setlength{\parsep}{0pt}%
    %\setlength{\parskip}{0pt}%
    %\setlength{\labelwidth}{8pt}%
    \setlength{\leftmargin}{2em}%
    \setlength{\itemindent}{0.5em}%
    %\setlength{\labelsep}{2pt}%
  }%
  \item[[GMS04]] Michel Goossens and Frank Mittelbach
 \emph{The {\LaTeX} companion (second edition)}, Adilson--Wesley, 2004.  
\end{list}
\end{ltxlistingout}
\let\thesection\orithesection
%
%\end{minipage}}
%\setcounter{section}{\value{lstorisection}}
%\let\thesection\orithesection
%\let\addcontentsline\oriaddcontentsline

Note que, se usar a opção de idiomas como opção de documentos (no \verb+\documentclass+, o pacote \texttt{babel} reconhece este idioma e vai usar, mesmo que não tenha passado opções no \texttt{babel}.


\section[Espaçamento entre linhas]{Sobre espaçamentos entre linhas e ``estouro de linhas''}

O espaçamento entre linhas é definido pelo comando \verb+\linespread+\index{linespread@\verb+\linespread+} onde \verb+\linespread{1.3}+ é o espaçamento um e meio e \verb+\linespread{1.6}+ é o espaçamento duplo, o que atuará a partir do próximo parágrafo.

Note que o \LaTeX{} deixa ``estourar'' as linhas (\texttt{Overfull} \verb+\hbox+\index{Overfull hbox@Overfull \verb+\hbox+}) quando não consegue acomodar devidamente o conteúdo nas linhas.
Para que ele use o espaçamento grande entre elementos (\texttt{Underfull} \verb+\hbox+\index{Underfull hbox@Underfull \verb+\hbox+}) em vez de ``estourar linhas'', coloque \verb+\sloppy+\index{sloppy@\verb+\sloppy+} (para voltar, use \verb+\fussy+\index{fussy@\verb+\fussy+}).
Note que, independente de linha estar ``estourando'' ou ``espaçando demais'', deverá checar cada um deles para ver se não está prejudicado visualmente e se for o caso, resolver o problema.
 
\section{Sobre hifenização}

O pacote \texttt{babel} carrega a regra de hifenização para o idioma selecionada, mas as vezes encontramos as exceções na qual a regra do \texttt{babel} não funciona devidamente.
Neste caso, poderá indicar na palavra de texto, a posição que pode ser hifenizada por ''\verb+\-+''\index{-@\verb+\-+} como em \verb+re\-fe\-rên-\cia+.

Este comando é ignorado quando não há necessidade de hifenização. 
Quando precisar, será hifenizada somente nestes locais.

Em geral, quando encontra uma palavra hifenizada indevidamente, poderemos querer acrescentar regra de hifenização para tal palavra em vez de indicar localmente.
Para tanto, coloque o comando \verb+\hyphenation+\index{hyphenation@\verb+\hyphenation+} onde seu argumento é a lista de palavras separados pelo espaço, onde cada palavra contém ``-'' na posição de hifenização.
Note que, para colocar regra de hifenização das letras acentuadas pelo comando \verb+\hyphenation+, requer \verb+\usepackage[T1]{fontenc}+ ou similar para ativar fontes que suportam acentuações.

o seguinte trecho de código colocado depois do carregamento do pacote \texttt{babel} no \texttt{preamble}, acrescenta hifenizações das palavras correspondentes.

\begin{singlespacing}
\begin{lstlisting}
% regra de hifenização das palavras não acentuadas:
% não requer \usepackage[T1]{fontenc}
\hyphenation{li-vro tes-te cha-ve bi-blio-te-ca}
% regra de hifenização das palavras acentuadas:
% requer \usepackage[T1]{fontenc}
\hyphenation{co-men-tá-rio re-fe-rên-cia}
\end{lstlisting}
\end{singlespacing}

Para que ele não hifenize, é só colocar dentro de uma caixa, o que pode ser feito pelo comando \verb+\mbox+\index{mbox@\verb+\mbox+}.
O código do tipo \verb+\mbox{não hifenizar}+ não pode ser quebrado em linhas, mesmo na posição de espaços. 
Este recurso permite proibir hifenização das palavras localmente.

Para que não use a hifenização no documento inteiro, costuma aumentar o valor da penalidade de hifenização no \texttt{preamble} do documento, como em
\begin{singlespacing}
\begin{lstlisting}
\hyphenpenalty=10000
\exhyphenpenalty=10000
\end{lstlisting}
\end{singlespacing}


\section{Trocando fontes}

Fonte padrão do \LaTeX{} é \texttt{Computer Modern} desenhado especialmente para ele, mas existem outras fontes incluídos no \TeX{} que podem ser usados.

Em geral, selecionar fontes manualmente requer cuidados e conhecimento sobre tipografia para não combinar fontes incompatíveis. 
Portanto, recorremos aos pacotes desenvolvidos pelas especialistas no assunto.

Vamos ver como usar a fonte \texttt{Times}\index{Times} que é uma das fontes populares.
Para selecionar a fonte \texttt{times}, usa o pacote \texttt{mathptmx}\index{mathptmx}. 
Basta colocar \verb+\usepackage{mathptmx}+ no \texttt{preamble} para que fontes fiquem como \texttt{Times}.

Para o Palatino, use o pacote \texttt{mathpazo}\index{mathpazo} com o comando \verb+\usepackage{mathpazo}+.

Ainda existem várias outras fontes que vem em qualquer distribuição \TeX, assim como fontes adicionais do \TeX{} mais completo.

\section{Trocando marcador da lista itemizada}

A marca de itens\index{lista!marca de itens} da lista \texttt{itemize} também pode ser redefinidas.
Por exemplo, o código
\begin{singlespacing}
\begin{lstlisting}
\renewcommand{\labelitemi}{$\bullet$}
\renewcommand{\labelitemii}{$\cdot$}
\renewcommand{\labelitemiii}{$\diamond$}
\renewcommand{\labelitemiv}{$\ast$}
\end{lstlisting}
\end{singlespacing}
ajustado adequadamente no \texttt{preamble} faz isso.

\section{Cores no \LaTeX{}}

O pacote \texttt{color}\index{color} (carregado automaticamente pelo pacote \texttt{graphicx}) oferece recursos básicos para trabalhar com cores no \LaTeX. 
Aqui, vamos supor que o \verb+\usepackage{graphicx}+ já está no \texttt{preamble}.
Para poder referenciar cores pelos nomes, coloque \verb+\usepackage[usenames]{color}+\index{color!usenames} no \texttt{preamble} também.
 
Os comandos de cores usam o parâmetro opcional para indicar o modelo. 
Se for omitido, será assumido o que foi especificado na hora de carregar o pacote \texttt{color}.
Como estamos supondo que vai usar a opção \texttt{usenames} na opção do pacote \texttt{color}, se o modelo for omitido nos comandos de cores, será assumido como \texttt{named}\index{color!named} (por nomes). 
 
A cor é especificado pelo comando \verb+\color[modelo]{cor}+\index{color@\verb+\color+}, mas lembre-se de proteger pelas chaves para que a cor retorne ao padrão fora dele. 
Para a cor do fundo (das páginas), usa-se o comando \verb|\pagecolor[modelo]{cor}|\index{pagecolor@\verb+\pagecolor+}. 
  No caso do trecho de textos, podemos usar os comandos
\verb|\textcolor[modelo]{cor}{texto}|\index{textcolor@\verb+\textcolor+} (texto com cor especificado), \verb|\colorbox[modelo]{cor}{texto}|\index{colorbox@\verb+\colorbox+} (caixa de texto com cor do fundo especificado) e \verb|\fcolorbox[modelo]{fcor}{bcolor}{texto}|\index{fcolorbox@\verb+\fcolorbox+} (caixa de texto com cor \texttt{bcolor} e cor do contorno \texttt{fcolor}) são usados.
Veja o Exemplo~\ref{ex13:color}.
 
%\begin{singlespacing}
%\begin{LTXexample}[pos=b,caption={ex09-03.tex},label=ex:09-03]
\begin{ltxlisting}[caption={ex13-color.tex},label=ex13:color]
Cor normal. {\color{red} Somente neste trecho muda de cor.} Aqui é cor normal.

Combinando:
\textcolor{white}{\colorbox{black}{texto colorido na caixa colorida}}
\end{ltxlisting}
%\end{singlespacing}
O comando \verb+\normalcolor+ retorna a cor padrão (dentro do trecho com cores personalizados), o que pode ser útil em alguns casos.

Segue uma tabela de cores básicos. 

\begin{table}[htbp]
\caption{cores aceitos em todos \texttt{drivers} no pacote \texttt{color}}
 \ \\
 \begin{tabular}{|c|c|c|c|c|c|c|c|}
  \hline
  black & \colorbox{black}{\hspace*{1cm}} & white &  \colorbox{white}{\hspace*{1cm}} &
  red & \colorbox{red}{\hspace*{1cm}} & green & \colorbox{green}{\hspace*{1cm}}  \\ \hline
  blue & \colorbox{blue}{\hspace*{1cm}} & cyan &  \colorbox{cyan}{\hspace*{1cm}} &
  magenta & \colorbox{magenta}{\hspace*{1cm}} & yellow & \colorbox{yellow}{\hspace*{1cm}}  \\ \hline
 \end{tabular}
\end{table}

 Para mais opções de cores, poderá usar opções \texttt{dvipsnames,usenames} na opção do pacote \texttt{color}, mas memorizar muitas cores não é produtivo.
Assim, costuma usar o pacote mais avançado de cores como o \texttt{xcolor} que permite misturar cores.
  
\section{Uso de caixas}
Para criar comando um pouco complexo, é importante ter noção sobre as caixas.
No \LaTeX, os elementos são colocados nas caixas e serão distribuídas nas páginas.
Estas caixas não são quebráveis.
Por exemplo, \verb+\mbox+\index{mbox@\verb+\mbox+} cria uma caixa e coloca o seu argumento.
Como a caixa não pode ser quebrada, se colocar uma palavra dentro do \verb+\mbox+, ele não será hifenizada, apesar disso não ser a forma elegante de proibir a hifenização de uma palavra.
Antigamente, também usava o \verb+\mbox+ para inserir texto no meio das fórmulas, o que é feito atualmente pelo comando \verb+\text+ do pacote \texttt{amsmath}.

A versão completa do \verb+\mbox+ é o \verb+\makebox+\index{makebox@\verb+\makebox+} que aceitam parâmetros opcionais. 
Por exemplo, um \texttt{strut}\index{strut} (espaço reservado verticalmente) pode ser criado rapidamente com o \verb+\makebox+ com auxílio de \verb+\strut+\index{strut@\verb+\strut+} (\verb+\strut+ reserva altura de uma linha).

Outras caixas que é usado com certa frequência é o \verb+\fbox+\index{fbox@\verb+\fbox+} que coloca moldura e \verb+\parbox+\index{parbox@\verb+\parbox+} que é uma versão simples de \texttt{minipage}.

No caso de \verb+\fbox+, as medidas \verb+\fboxsep+\index{fboxsep@\verb+\fboxsep+} e 
\verb+\fboxrule+\index{fboxrule@\verb+\fboxrule+} controlam o espaço entre conteúdo e moldura, assim como a espessura de linha da moldura.

Uma versão completa com \verb+\fbox+ é o \verb+\framebox+\index{framebox@\verb+\framebox+} que pode controlar a largura e o posicionamento.
Usando em combinação com o comando \verb+\strut+ que insere espaço vertical, poderá criar um retângulo com \verb+\framebox+.

O \verb+\parbox+ permite colocar textos com parágrafos e similares, o que é proibido nas maioria das caixas. Assim, colocando os elementos dentro do \verb+\parbox+, poderá usar textos com parágrafos e similares dentro da caixa comum. Note que, alguns casos mais complexos, requer o sudo do  \texttt{minipage} em vez do \verb+\parbox+.

Veja o Exemplo~\ref{ex13:caixa:fbox}.

\begin{ltxlisting}[caption={ex13-caixa-fbox.tex}, label=ex13:caixa:fbox]
Exemplo de um \texttt{strut}\mbox{\strut}, continuação.

Texto com espaço\makebox[2em]{\strut}, continuação.

\fbox{Texto com moldura}

\setlength{\fboxsep}{10pt}
\setlength{\fboxrule}{5pt}
\fbox{Texto com moldura personalizada}

\setlength{\fboxsep}{0pt}
\setlength{\fboxrule}{1pt}
\framebox[5em]{\strut} % um retângulo

\fbox{% moldura no texto com mais de uma linha
\parbox{\textwidth}{%
Linha 1\\
Linha 2
}}
\end{ltxlisting}

Além de permitir manipulações (mudar de tamanho, posição, rotação, etc),
o uso de caixas permite utilizar o elemento várias vezes.
Por exemplo, poderá replicar várias vezes um elemento, ou efetuar medidas antes de posicionar elementos.
Para salvar uma caixa, usa-se o comando \verb+\sbox+\index{sbox@\verb+\sbox+} ou \verb+\savebox+.
Uma caixa salva pode ser usado usado pelo \verb+\usebox+.
Para salvar uma caixa, inicialmente cria uma variável para armazenar usando o \verb+\newsavebox+\index{newsavebox@\verb+\newsavebox+}.
Depois salva a caixa usando \verb+\savebox+\index{savebox@\verb+\savebox+}.
Uma caixa salva pode ser referenciado quantas vezes queira, pelo comando \verb+\usebox+\index{usebox@\verb+\usebox+}. Para alterar a altura onde caixa  será colocada, poderá usar o comando \verb+\raisebox+\index{raisebox@\verb+\raisebox+}.
Veja o Exemplo~\ref{ex13:caixas}.
\begin{ltxlisting}[caption={ex13-caixas.tex}, label=ex13:caixas]
\newsavebox{\mybox} % cria a variavel para caixa
\savebox{\mybox}{Teste} % armazena conteudo
\usebox{\mybox}
\usebox{\mybox}
\raisebox{-2ex}{\usebox{\mybox}}

Outra linha
\end{ltxlisting}
Diferente do parâmetro do comando, os atributos (cor, por exemplo) do elemento dentro da caixa salva não podem ser alterados.
Assim, quando precisar usar mesmo elemento mais de uma vez com atributos diferentes (como no caso do texto sombreado), precisará criar um comando em vez de salvar e reusar as caixas.

Note que uma barra pode ser criado pelo comando
\verb+\rule+\index{rule@\verb+\rule+}, embora ele não seja uma caixa.
Veja o Exemplo~\ref{ex13:rule}.

\begin{ltxlisting}[caption={ex13-rule.tex}, label=ex13:rule]
\rule{0.5\textwidth}{1pt}

Texto
\rule[0.1\textwidth]{1pt}{0.1\textheight}
continuação
\end{ltxlisting}

\clearpage\thispagestyle{empty}\cleardoublepage
\chapter{Algumas Dicas Para Criar Comandos e Ambientes}

\section[Comandos com ``@'' e ``*'']{Acessando os comandos com ``@'', criando comandos versão ``*'' e parâmetro do tipo \texttt{chave=valor}}

As vezes precisamos usar os comandos ou ambientes que contém ``@'' no seu nome. 
O caso típico é redefinir comandos existentes tais como \verb+\maketitle+, estilo de cabeçalho das páginas, etc.
Note que os comandos e ambientes que usam ``@'' no seu nome são comandos sensíveis e devem ser usados com cuidado.
Para acessar estes comandos e ambientes, coloque \verb+\makeatletter+\index{makeatletter@\verb+\makeatletter+} antes de usar e \makeatother+\index{makeatother@\verb+\makeatlther+} depois.

Por exemplo, ajustar o espaçamento de linhas como sendo simples no texto de rodapé, mesmo que esteja usando espaçamento um e meio ou dupla no corpo do documento. 
Isto pode ser feito, acrescentando a seguinte código no \texttt{preamble} do documento.

%========================================================
% rodap\'e deve estar em espa\c{c}amento simples,
% independente do espa\c{c}amento no corpo de documento
% Devera ser feito com linespread{} \@currsize
% pois comandos do setspace nao funciona
%========================================================
% foi copiado do book e alterado

\begin{singlespacing}
\begin{lstlisting} 
\makeatletter % ativa uso de ``@'' no nome
\renewcommand\@makefntext[1]{%
    \parindent 1em%
    % inicio da altera\c{c}\~ao    
    \linespread{1} \@currsize \noindent
    % \hb@xt@1.8em{\hss\@makefnmark}#1}
    \hb@xt@0.45em{\hss\@makefnmark}#1}
% fim da altera\c{c}\~ao
\makeatother % desativa o uso de ``@'' no nome
\end{lstlisting}
\end{singlespacing}

Outro exemplo e o caso de usar linhas pontilhadas no sumário.

\begin{singlespacing}
\begin{lstlisting}
\makeatletter
\renewcommand*\l@section{\@dottedtocline{1}{1.5em}{2.3em}}
\makeatother
\end{lstlisting}
\end{singlespacing}

no \texttt{preamble} do artigo habilita o uso de linhas pontilhadas no sumário\index{sumário!linhas pontilhadas}, para seções.
Para habilitar o pontilhado no capítulo do sumário é mais complexo.
Veja o código do \url{https://tex.stackexchange.com/questions/62438/how-to-add-leaders-to-table-of-contents-without-tocloft}.

\begin{singlespacing}
\begin{lstlisting}
\renewcommand*\l@chapter[2]{%
  \ifnum \c@tocdepth >\m@ne
    \addpenalty{-\@highpenalty}%
    \vskip 1.0em \@plus\p@
    \setlength\@tempdima{1.5em}%
    \begingroup
      \parindent \z@ \rightskip \@pnumwidth
      \parfillskip -\@pnumwidth
      \leavevmode \bfseries
      \advance\leftskip\@tempdima
      \hskip -\leftskip
      #1\nobreak
      \xleaders\hbox{$\m@th
        \mkern \@dotsep mu\hbox{.}\mkern \@dotsep
        mu$}\hfill%
      \nobreak\hb@xt@\@pnumwidth{\hss #2}\par
      \penalty\@highpenalty
    \endgroup
  \fi}
\end{lstlisting}
\end{singlespacing}
  
Note que, no \LaTeX, existem vários comandos e ambientes que tem versão normal e versão ``*'' na qual a versão ``*'' é uma variação da versão normal.
Para criar os comandos versão ``*'', usa-se o comando \verb+\@ifstar+\index{ifstar@\verb+\+\arrobasymbforindex\verb+ifstar+} que determina se está ou não usando a versão ``*'', mas seu uso requer cuidados.

O código
%
\begin{singlespacing}
\begin{lstlisting}
\makeatletter
\def\myemph{\@ifstar\@myemph\@@myemph} % selecionando os comandos
\newcommand{\@myemph}[1]{\underline{#1}} % versao ``*''
\newcommand{\@@myemph}[1]{\emph{#1}} % versao normal
\makeatother
\end{lstlisting}
\end{singlespacing}
%
define a versão normal e versão ``*'' do \verb+\myemph+.
Note que na primeira linha foi usado o \verb+\def+ em vez de \verb+\newcommand+, o que é necessário quando o comando tem parâmetros. 
Com o \verb+\@ifstar+, seleciona o comando de acordo com a existência do ``*''.
Note que o primeiro é versão com ``*'' e o segundo é a versão sem ``*''.

Depois define os comandos com ``*'' e comandos sem ``*'' que foi usado anteriormente. 
Note o uso de ``@'' no nome destes comandos para proteger do uso indevido deles.

Assim, podemos usar a versão normal que enfatiza e versão ``*'' que sublinha.
Veja o Exemplo~\ref{ex14:star}.

%\begin{singlespacing}
%\begin{LTXexample}[pos=b,caption={ex04-06.tex},label=ex:04-06]
\begin{ltxlisting}[caption={ex14-star.tex},label=ex14:star]
\documentclass[12pt,a4paper]{article}
\usepackage[T1]{fontenc}
\usepackage[utf8]{inputenc}
\usepackage[brazil]{babel}
\makeatletter
\def\myemph{\@ifstar\@myemph\@@myemph} % selecionando os comandos
\newcommand{\@myemph}[1]{\underline{#1}} % versao ``*''
\newcommand{\@@myemph}[1]{\emph{#1}} % versao normal
\makeatother
\begin{document}
Enfatizando palavras com \myemph{Teste} e \myemph*{Teste}.
\end{document}
\end{ltxlisting}
%\end{LTXexample}
%\end{singlespacing}

Note que no caso de ambientes, a versão com ``*'' pode ser definido diretamente, sem a necessidade do artifício dos comandos como em
\begin{singlespacing}
\begin{lstlisting}
\newenvironment{mypar}[1]% 1 argumento eh titulo
  {\par \center ---------------< #1 >--------------- \\ }% título e linha
  {\par \center ---------------< fim >--------------- \\}
  
\newenvironment{mypar*}[1]% 1 argumento que eh titulo
  {\par \center ---------------< #1 >--------------- \\ }% título e linha
  {\par \center ------------------------------------- \\}
\makeatother
\end{lstlisting}
\end{singlespacing}


Outra coisa que pode querer é criar comandos e ambientes que recebem os parâmetros na forma \texttt{chave=valor}\index{parâmetros!chave=valor}.

Para isso, usa-se o pacote \texttt{keyval}.
para implementar o parâmetro do tipo \texttt{chave=valor}. Inicialmente, precisa criar uma regra do que fazer quando tiver uma determinada chave no parâmetro, usando o comando \verb+\define@key+\index{definekey@\verb+\define+\arrobasymbforindex\verb+key+}.

O primeiro parâmetro é o grupo que o parâmetro pertence.
Em geral, coloca o nome do comando ou ambiente para evitar conflitos.
Segundo parâmetro é o nome da chave e terceiro é o que vai fazer. 
O \verb+#1+ será o valor da chave.

Em geral, se tiver o nome da chave sem o valor, dará erro.
Se quer aceitar a chave sem o valor, poderá passar o valor padrão como parâmetro opcional.

O parâmetro será evaluado pelo comando \verb+\setkeys+\index{setkeys@\verb+\setkeys+} onde primeiro parâmetro é o grupo que o parâmetro pertence.

Note que, se o parâmetro for definido pelo comando, precisará expandir antes, ou seja, retardar o \verb+\setkeys+. 
Para isso, usa-se o comando \verb+\expandafter+\index{expandafter@\verb+\expandafter+}, mas retardar o comando para expandir o segundo argumento primeiro não é simples.
Neste caso, define um comando auxiliar com um único argumento que chama o \verb+\setkeys+ com nome do grupo prefixado e aplicar \verb+\expandafter+ neste comando.
Veja o Exemplo~\ref{ex14:keyval}.

\begin{ltxlisting}[caption={ex14-keyval}, label=ex14:keyval]
\documentclass[12pt,a4paper]{article}
\usepackage[T1]{fontenc}
\usepackage[utf8]{inputenc}
\usepackage[brazil]{babel}
\makeatletter
% O que vai fazer se encontrar "foo" no parametro do grupo "my".
\define@key{my}{foo}{foo vale #1\par} % se receber foo sem o seu valor, dara erro
\define@key{my}{bar}[99]{bar vale #1\par} % com valor padrao "99" se valor nao for repassado.
\makeatother

% Definindo o comando auxiliar com um argumento
\def\mysetkeys#1{\setkeys{my}{#1}}
% Comando associado ao argumento
\def\mykeyvalue{foo=5}
\begin{document}
Testando o parâmetro chave=valor.

\setkeys{my}{foo=3,bar}

O argumento que esta ``armazenado'' no comando.

\mykeyvalue

% expandindo o argumento primeiro para poder evaluar
\expandafter\mysetkeys\expandafter{\mykeyvalue}
\end{document}
\end{ltxlisting}

Agora, veja o Exemplo~\ref{ex14:keyval:env} para exemplo de implementação do ambiente.
Nele, foi usado o \verb+\newif+\index{newif@\verb+\newif+} que cria um condicional.
O comando \verb+\newif\if<nome>+ cria o condicional \verb+\if<nome>+ que torna verdadeiro ou falso quando chamar \verb+if<nome>true+ ou \verb+if<nome>false+ respectivamente.

\begin{ltxlisting}[caption={ex14-keyval-env}, label=ex14:keyval:env]
\documentclass[12pt,a4paper]{article}
\usepackage[T1]{fontenc}
\usepackage[utf8]{inputenc}
\usepackage[brazil]{babel}
\makeatletter
\newif\ifmyparkeyvaltitle % testar se tem titulo
% O titulo será associado ao comando para poder ser usado posteriormente.
\define@key{myparkeyval}{title}{
\hrulefill\fbox{#1}\hrulefill\myparkeyvaltitletrue
\def\myparkeyvaltitle{#1}}

\newenvironment{myparkeyval}[1][]{
\myparkeyvaltitlefalse
\par\noindent
\setkeys{myparkeyval}{#1}
\ifmyparkeyvaltitle
\else
  \hrule
\fi
\par}{
\par\noindent
\ifmyparkeyvaltitle
  \hrulefill\fbox{fim do \myparkeyvaltitle}\hrulefill
\else
  \hrule
\fi
\par}
\makeatother

\begin{document}
\begin{myparkeyval}
Sem titulo.
\end{myparkeyval}

\begin{myparkeyval}[title=Titulo]
Com titulo.
\end{myparkeyval}
\end{document}
\end{ltxlisting}

\section{Ambiente com parâmetro na finalização e aplicação do comando no corpo}

No caso de ambiente precisar usar o argumento na finalização, precisará associar a algum comando.
Isto porque, o comando \verb+\newenvironment{ambiente}+ criará o par de comandos \verb+\ambiente+ (chamado pelo \verb+\begin{ambiente}+) e \verb+\endambiente+ (chamado pelo \verb+\end{ambiente}+) que funcionará em sincronismo. 
Os parâmetros do ambiente são passado para o comando de inicialização, mas não para a finalização.

Veja o Exemplo~\ref{ex14:parametro:finalizacao} que usa o segundo parâmetro na finalização.

\begin{ltxlisting}[caption={ex14-parametro-finilizacao.tex}, label=ex14:parametro:finalizacao]
\documentclass[12pt,a4paper]{article}
\usepackage[T1]{fontenc}
\usepackage[utf8]{inputenc}
\usepackage[brazil]{babel}
\newenvironment{mypartwo}[2]{
\par\noindent\hrulefill%
\ifx #1\empty
\else % se nao for vazio, coloca com fbox
  \fbox{#1}
\fi
\hrulefill\par
\def\mypartwoargtwo{#2}}{% agora eh finalizacao do ambiente
\par\noindent\hrulefill%
\ifx \mypartwoargtwo\empty
\else
  \fbox{\mypartwoargtwo}
\fi%
\hrulefill\par}
\begin{document}
\begin{mypartwo}{Inicio}{Fim}
Testando
\end{mypartwo}
\end{document}
\end{ltxlisting}


Note que no exemplo acima, foi usado o comando \verb+\ifx+\index{ifx@\verb+\ifx+} para verificar se o argumento correspondente é ou não vazio.
O comando \verb+\else+\index{else@\verb+\else+} é \texttt{senão} e \verb+\fi+\index{fi@\verb+\fi+} é \texttt{fim se}.


As vezes, queremos um ambiente que aplica um comando no corpo do ambiente.
No entanto, a chave aberta/fechada para indicar o parâmetro deve aparecer em par tanto na inicialização, com na finalização, não podendo delimitar o corpo do ambiente.

Para casos como este, podemos usar um comando em vez da chave para inicializar/finalizar o agrupamento. Para inicializar o agrupamento, usa-se o comando \verb+\bgroup+ e para finalizar, usa-se o comando \verb+\egroup+.

Como um exemplo, vamos supor que queremos um ambiente que sublinhe todo texto, aplicando o \verb+\emph+ em todo corpo do ambiente (já existe tal ambiente que é \verb+em+, mas aqui foi construído como um exemplo).
Exemplo~\ref{ex14:comando:corpo:ambiente}.

\begin{ltxlisting}[caption={ex14-comando-no-corpo-do-ambiente.tex}, label=ex14:comando:corpo:ambiente]
\documentclass[12pt,a4paper]{article}
\usepackage[T1]{fontenc}
\usepackage[utf8]{inputenc}
\usepackage[brazil]{babel}

\newenvironment{myulbox}{\emph\bgroup}{\egroup}

\begin{document}
\begin{myulbox}
Testando o ambiente.
\end{myulbox}
\end{document}
\end{ltxlisting}

O uso de \verb+\bgroup+/\verb+\egroup+ falham para vários comandos tais como \verb+\underline+ e \verb+\fbox+. Nestes casos, poderá usar o comando \verb+\collect@body+\index{collectbody@{\verb+\collect+\arrobasymbforindex\verb+body+}} do pacote \texttt{amsmath}.
Exemplo~\ref{ex14:comando:corpo:ambiente:amsmath} ilustra o uso do recurso de \texttt{amsmath} (Note que \verb+\collect@body+ vem antes do comando).

\begin{ltxlisting}[caption={ex14-comando-no-corpo-do-ambiente-amsmath.tex}, label=ex14:comando:corpo:ambiente:amsmath]
\documentclass[12pt,a4paper]{article}
\usepackage[T1]{fontenc}
\usepackage[utf8]{inputenc}
\usepackage[brazil]{babel}
\usepackage{amsmath}
\makeatletter
  \newenvironment{myfbox}{\collect@body\fbox}{}
\makeatother
\begin{document}
\begin{myfbox}
Testando o ambiente.
\end{myfbox}
\end{document}
\end{ltxlisting}

Para aplicar comandos com mais de um parâmetros, poderá recorrer ao pacote fora do \texttt{base}/\texttt{required} como o \texttt{environ}\index{environ}.

% Exemplo do uso:
% \usepackage{environ}
% \NewEnviron{myulbox}{\underline{\BODY}}

\section{Comandos definidos dentro do outro comando}

As vezes, queremos modificar um comando dentro do outro.
Por exemplo, comand \verb+\chapter+ e \verb+\section+ 
modificam os comandos para exibir informações no cabeçalho.
Outro caso é os comandos que definem as informações da capa 
como \verb+\title+ e \verb+\author+ na qual define comandos
internos para armazenar informações para uso posterior.
O Exemplo~\ref{ex14:comando:aninhado} ilustra o uso de definição do comando dentro do comando para armazenar o parâmetro no comando interno para o uso posterior. 

\begin{ltxlisting}[caption={ex14-comando-aninhado.tex}, label=ex14:comando:aninhado]
\documentclass[12pt,a4paper]{article}
\usepackage[T1]{fontenc}
\usepackage[utf8]{inputenc}
\usepackage[brazil]{babel}
\makeatletter
  \newcommand{\@my@title}{}
  \newcommand{\mytitle}[1]{\renewcommand{\@my@title}{#1}}
  \newcommand{\myprinttitle}{\@my@title}
\makeatother
\begin{document}
\mytitle{Teste}
O título armazenado é \myprinttitle
\end{document}
\end{ltxlisting}
Como cada chamada, o comando redefine o comando interno,
o comando interno deve existir antes.

Para redefinir um comando já existente, é interessante tirar uma cópia dele antes de redefinir, o que permitiria usar ou restaurar o comando original, caso necessário. Para copiar um comando, usa-se o \verb+\let+\index{\textbackslash let}.
O Exemplo~\ref{ex14:comando:let} ilustra o uso de \verb+\let+.

\let\thesectionoriginaldolivro\thesection
\renewcommand{\thesection}{\arabic{section} }
\begin{ltxlisting}[caption={ex14-comando-let.tex}, label=ex14:comando:let]
\documentclass[12pt,a4paper]{article}
\usepackage[T1]{fontenc}
\usepackage[utf8]{inputenc}
\usepackage[brazil]{babel}
\newcommand{\mythesection}{\roman{section}: }
\let\thesectionoriginal\thesection
\begin{document}
\let\thesection\mythesection
\section{Enumeração nova}
\let\thesection\thesectionoriginal
\section{Enumeração normal}
\end{document}
\end{ltxlisting}
\let\thesection\thesectionoriginaldolivro

Para desativar um comando, em geral associao ao comando \verb+\relax+ usando \verb+\let+.

Para definir comando com parâmetro dentro da definição de comandos, o parâmetro do comando aninhado (de dentro) deve ter dois ``\#'' em vez de uma para distinguir com o parâmetro do comando externo.
Veja o Exemplo~\ref{ex14:comando:aninhado:com:parametro}.

\begin{ltxlisting}[caption={ex14-comando-aninhado-com-parametro.tex}, label=ex14:comando:aninhado:com:parametro]
\documentclass[12pt,a4paper]{article}
\usepackage[T1]{fontenc}
\usepackage[utf8]{inputenc}
\usepackage[brazil]{babel}
\newcommand{\mytitledtext}[1]{##1}
\newcommand{\mytitlefortext}[1]{%
  \renewcommand{\mytitledtext}[1]{\textbf{#1:} ##1}%
}
\mytitlefortext{Nota}
\begin{document}
\mytitledtext{Teste de aninhamento}
\end{document}
\end{ltxlisting}

Muitas vezes, usamos o comando de definição do \TeX{} em vez do \LaTeX,
o que é mais rápido.
As vezes,  também usamos a versão \TeX{} para definir 
o comando definido dentro do \verb+newcommand+ para diferenciar.

Os comandos \TeX{} correspondentes ao \verb+\newcommand+
do \LaTeX{} são: \verb+\def+, \verb+edef+, \verb+gdef+ e \verb+xdef+.

Observe que \verb+\def+ redefine o comando 
caso já existir, o que requer cuidados para não alterar 
o comando importante já existente.
A diferença com o \verb+newcommand+ é a indicação dos números de parâmetros.

Para definir comando com parâmetros, acrescenta-se o \verb+#1#2+,\ldots, \verb+#<n>+, onde \verb+<n>+ é o número do parâmetro (no máximo $9$). 
O Exemplo~\ref{ex14:comando:def} ilustra o uso de definição do comando dentro do comando para armazenar o parâmetro no comando interno para o uso posterior. 
\begin{ltxlisting}[caption={ex14-comando-def.tex}, label=ex14:comando:def]
\documentclass[12pt,a4paper]{article}
\usepackage[T1]{fontenc}
\usepackage[utf8]{inputenc}
\usepackage[brazil]{babel}
\makeatletter
  \def\mytitle#1{\gdef\@my@title{#1}}
  \newcommand{\myprinttitle}{\@my@title}
  \mytitle{}
\makeatother
\begin{document}
\mytitle{Teste}
O título armazenado é \myprinttitle
\end{document}
\end{ltxlisting}

Note que foi usado \verb+\gdef+\index{\textbackslash gdef} que é atalho para \verb+\global\def+, em vez de \verb+\def+ para que a definição fique disponível globalmente, dai em diante. No caso de usar o \verb+\def+, a definição só vale para dentro do ambiente que contém (caso definir comando usando \verb+\def+ dentro de algum ambiente -- entre \verb+\begin+ e \verb+\end+ -- fora dele o comando não estará disponível. No caso de \verb+\gdef+, comando definido continua disponível fora do ambiente).

Existe o comando \verb+edef+\index{\textbackslash edef} que expande os argumentos na hora de definir.
Por exemplo, \verb+\def\pagina{thepage}+ define comando que imprime a página na qual o comando \verb+\pagina+ foi chamado, enquanto que \verb+\edef\pagina{thepage}+ define comando que imprime a página na qual o comando \verb+\pagina+ foi definido. Note que, \verb+xdef+ é um atalho para \verb+\global\edef+ que mantém a definição dos comandos globalmente.
No caso da definição de dentro ter parâmetro, a indicação de parâmetro também leva dois ``\#''.

\clearpage\thispagestyle{empty}\cleardoublepage
\chapter{Usando Pacotes Fora do \texttt{base} e \texttt{required}}

Até agora, só estudamos os pacotes disponíveis em qualquer distribuição \TeX, usando somente os pacotes do conjunto denominado de \texttt{base}\index{base} e \texttt{required}\index{required}.
No entanto, maioria das distribuições \LaTeX{} instala uma grande quantidade de pacotes adicionais para incrementar a sua funcionalidade.
Assim, neste capítulo, vamos aventurar em alguns destes pacotes adicionais para facilitar o nosso trabalho.

\section[Configuração das páginas]{Ajustando a configuração das páginas e similares}

Para ajustar as margens, tamanho de papeis, espaçamento entre cabeçalho e texto, etc, usamos o pacote \texttt{geometry}\index{geometry}.
As configurações podem ser feitas, passando no parâmetro opcional do pacote, a lista dos itens na forma \texttt{chave=valor}, separado pela virgula.
Também pode usar o comando \verb+\geometry+\index{geometry@\verb+\geometry+} após carregar o pacote.

Para ajustar margem superior (\texttt{tmargin}\index{geometry!tmargin}), margem inferior (\texttt{bmargin}\index{geometry!bmargin}), margem esquerda (\texttt{lmargin}\index{geometry!lmargin}, será de dentro se for frente/verso), margem direita (\texttt{rmargin}\index{geometry!rmargin}, será de fora se for frente/verso), basta colocar algo como
\begin{singlespacing}
\begin{lstlisting}
\usepackage[tmargin=2cm,bmargin=2cm,lmargin=2cm,rmargin=2cm]{geometry}
\end{lstlisting}
\end{singlespacing}
ou
\begin{singlespacing}
\begin{lstlisting}
\usepackage{geometry}
\geometry{tmargin=2cm,bmargin=2cm,lmargin=2cm,rmargin=2cm}
\end{lstlisting}
\end{singlespacing}
%
no \texttt{preamble} do documento.

Para ajuste de espaçamento entre linhas, usamos o pacote \texttt{setspace}\index{setspace} que tem  os comandos \verb+\singlespacing+\index{singlespacing@\verb+\singlespacing+} (espaçamento simples), \verb+\onehalfspacing+\index{onehalfspacing@\verb+\onehalfspacing+} (espaçamento um e meio) e \verb+\doublespacing+\index{doublespacing@\verb+\doublespacing+} (espaçamento duplo), além do ambiente \texttt{spacing}\index{spacing} que permite qualquer espaçamento dentro dele.

Para que o documento fique em espaçamento um e meio, coloque

\begin{singlespacing}
\begin{lstlisting}
\usepackage{setspace}
\onehalfspacing
\end{lstlisting}
\end{singlespacing}
%
no \texttt{preamble} do documento.

Usuário da classe \texttt{memoir}\index{memoir} devem colocar \verb+\DisemulatePackage{setspace}+\index{DisemulatePackagesetspace@\verb+\DisemulatePackage{setspace}+} antes de carregar o \texttt{setspace}.

Para desabilitar a hifenização em todo documento ou ativar hifenização na fonte mono espaçado (\verb+\ttfamily+), poderá usar o pacote \texttt{hyphenat}\index{hyphenat}.
\begin{singlespacing}
\begin{lstlisting}
\usepackage[none]{hypenat}
\end{lstlisting}
\end{singlespacing}
no \texttt{preamble} desabilita hifenização em todo documento.
Se não quer usar o pacote e quer desativar a hifenização, coloque
\begin{singlespacing}
\begin{lstlisting}
\hyphenpenalty=10000
\exhyphenpenalty=10000
\end{lstlisting}
\end{singlespacing}
no \texttt{preamble} do documento.

Para melhorar o ajuste de espaços, poderemos usar o pacote \texttt{microtype}\index{microtype} que usa o recurso de micro tipografia. 
Um desses ajustes é reduzir ou ampliar levemente o tamanho da fonte para melhorar o espaçamento.
Por exemplo, podemos colocar
\begin{singlespacing}
\begin{lstlisting}
\usepackage[stretch=10]{microtype}
\end{lstlisting}
\end{singlespacing}
no \texttt{preamble}.

Cabeçalho no estilo de página \texttt{headings}\index{headings} coloca o título do capítulo e seção no cabeçalho em maiúsculo. 
Para tanto, usa-se o comando \verb+\MakeUppercase+\index{MakeUppercase@\verb+\MakeUppercase+} que não funciona para letras acentuadas diretamente (funciona para acentuação no modo \TeX). 
Além disso, ele tenta converter inclusive as fórmulas, o que pode causar problemas.
O pacote \texttt{textcase}\index{textcase} permite sobrescrever o \verb+\MakeUppercase+ para evitar tais problemas, além de mais alguns recursos adicionais.
Coloque
\begin{singlespacing}
\begin{lstlisting}
\usepackage[overload]{textcase}
\end{lstlisting}
\end{singlespacing}
%
no \texttt{preamble}.

Normalmente, a referência bibliográfica e índice remissivo não costumam ficar no sumário.
Para que eles constem no sumário, use o pacote \texttt{tocbibind}\index{tocbibind}. 
Usando a opção adicional, pode impedir que o sumário, lista de figuras e de tabelas fiquem fora do sumário.
\begin{singlespacing}
\begin{lstlisting}
\usepackage[nottoc,notlof,notlot]{tocbibind}
\end{lstlisting}
\end{singlespacing}
%
no \texttt{preamble} efetua esta tarefa.

Note que no \LaTeX, o texto é justificado por padrão, mas não existe comando para justificar um trecho dentro do outro alinhamento. 
Uma saída é usar o \texttt{minipage}, mas \texttt{minipage} não permite quebrar entre páginas.
Para resolver este problema, existe o pacote \texttt{ragged2e}\index{ragged2e} que providencia o ambiente \texttt{justify}\index{justify} (e comando \verb+\justify+\index{justify@\verb+\justify+}) que pode ser usado como outros comandos de alinhamento de texto no parágrafo.

\section{Estilo europeu}
Em Alguns países da Europa, um parágrafo é separado pela outra com o espaçamento maior entre linhas em vez de indentação. 
Remover a indentação é feito, ajustando o valor de \verb+\parindent+\index{parindent@\verb+\parindent+} para zero, mas ajustar o espaçamento antes do parágrafo por \verb+\parskip+\index{parskip@\verb+\parskip+} e acertar alguns detalhes requer conhecimento extra.
Para facilitar, existe o pacote \texttt{parskip}\index{parskip} que, ao ser usado, ajustam de forma apropriada.
Se não quer usar o pacote, coloque
\begin{singlespacing}
\begin{lstlisting}
\setlength{\parindent}{0pt}
\setlength{\parskip}{1ex plus 0.5ex minus 0.2ex}
\end{lstlisting}
\end{singlespacing}
%
no \texttt{preamble}, mas isto não ajustará tudo que o pacote faz.

Para produzir o estilo literário francês\index{estilo literário!francês} na qual a primeira letra do parágrafo ocupa mais de uma linha, tem o pacote \texttt{lettrine}\index{lettrine}.
O uso comum nos livros de literatura é usar o \texttt{lettrine} no primeiro parágrafo do capítulo e manter o restante como normal.
Veja o Exemplo~\ref{ex15:lettrine} na qual é assumido que tem o \verb+\usepackage{lettrine}+ no \texttt{preamble} do documento.
%\setcounter{lstorichapter}{\value{chapter}}
%\setcounter{lstorisection}{\value{section}}
%\setcounter{chapter}{0} %\setcounter{section}{0}
%\let\addcontentsline\emptyaddcontentsline
%\begin{singlespacing}
%\begin{lstlisting}[pos=b,caption={ex11-03.tex},label=ex:11-03]
\begin{ltxlisting}[caption={ex15-lettrine.tex},label=ex15:lettrine]
\chapter{Um Capítulo}
\lettrine{E}{xiste} um pacote chamado \texttt{letrine} que oferece o estilo literário francês (estilo bíblico) na qual a primeira letra do parágrafo ocupa varias linhas. 
Alguns livros literários usa este estilo somente para o parágrafo do primeiro capítulo, mantendo outros parágrafos como sendo normal. O primeiro parâmetro é a letra (primeira letra) que ocupará várias linhas, e o segundo parâmetro é a continuação dele até finalizar a palavra, que será escrito em maiúsculo. O comando ainda aceita parâmetros opcionais para ajustar quantas linhas vai ocupar, etc.
\end{ltxlisting}
%\end{singlespacing}
%\setcounter{chapter}{\value{lstorichapter}}
%\setcounter{section}{\value{lstorisection}}
%\let\addcontentsline\oriaddcontentsline

%\setcounter{lstorichapter}{\value{chapter}}
%\setcounter{lstorisection}{\value{section}}
%\setcounter{chapter}{0} 
%%\setcounter{section}{0}
%%\let\addcontentsline\emptyaddcontensline
%%\newcommand{\emptyaddcontentsline}[3]{}
%% \let\oriaddcontentsline\addcontentsline
%\let\addcontentsline\emptyaddcontentsline
%\begin{singlespacing}
%%\fbox{
%\begin{framed}\noindent
%\begin{minipage}{\linewidth}
%\chapter{Um Capítulo}
%\lettrine{E}{xiste} um pacote chamado \texttt{letrine} que oferece o estilo literário francês (estilo bíblico) na qual a primeira letra do parágrafo ocupa varias linhas. 
%Alguns livros literários usam este estilo somente para o parágrafo do primeiro capítulo, mantendo outros parágrafos como sendo normal. O primeiro parâmetro é a letra (primeira letra) que ocupará várias linhas, e o segundo parâmetro é a continuação dele até finalizar a palavra, que será escrito em maiúsculo.
%O comando ainda aceita parâmetros opcionais para ajustar quantas linhas vai ocupar, etc.
%\end{minipage}
%\end{framed}
%%}
%\end{singlespacing}
%\setcounter{chapter}{\value{lstorichapter}}
%\setcounter{section}{\value{lstorisection}}
%\let\addcontentsline\oriaddcontentsline

\section[Cabeçalho e títulos]{Ajustando o cabeçalho, títulos de capítulos e de figuras}

Ajustar o cabeçalho diretamente não é simples.
Assim, costumamos usar o pacote \texttt{fancyhdr}\index{fancyhdr} que permite personalizar o cabeçalho das páginas.
Veja o Exemplo~\ref{ex15:fancyhdr} usado para produzir estilo de cabeçalho deste documento.

\begin{singlespacing}
\begin{lstlisting}[caption={ex15-fancyhdr.tex},label=ex15:fancyhdr]
\documentclass[12pt,a4paper]{article}
\usepackage[T1]{fontenc}
\usepackage[utf8]{inputenc}
\usepackage[brazil]{babel}

\usepackage{calc} % para calculo de medidas
\usepackage{fancyhdr} % personalizar o cabeçalho

% \fancyheadoffset[LE,RO]{\marginparsep+\marginparwidth}

% redefine a saida do capitulo e da secao
% \pagestyle{fancy} redefine eles,
%  entao deve colocar no final do ps@fancy 
% para ter efeito. Usando \addto do babel
\makeatletter
  \addto\ps@fancy{%
    \renewcommand{\chaptermark}[1]{%
      \markboth{\if@mainmatter\thechapter. \fi #1}%
        {\if@mainmatter\thechapter. \fi #1}}
    \renewcommand{\sectionmark}[1]{%
      \markright{\if@mainmatter\thesection. \fi #1}}
  }
\makeatother

% redefinindo o estilo da pagina "fancy"
\fancyhf{} % limpa o cabecalho
% redefine o cabcalho. \nouppercase foi usado para eliminar conversao para maiusculo do sumario/bibliografia
\fancyhead[LE,RO]{\bfseries\thepage}
\fancyhead[LO]{\bfseries\nouppercase\rightmark}
\fancyhead[RE]{\bfseries\nouppercase\leftmark}

% redefine o "plain" (usado na primeira pagina do capitulo).
\fancypagestyle{plain}{%
   \fancyhead{} % get rid of headers
   \fancyhead[LE,RO]{\bfseries\thepage} % colocar enumeracao
   \renewcommand{\headrulewidth}{0.5pt} % com linha horizontal
}

% linha horizontal entre cabecalho e corpo do documento
\renewcommand{\headrulewidth}{0.5pt}

% sem linha horizontal entre corpo de texo e rodape
\renewcommand{\footrulewidth}{0pt}

%%%%%%%% fim da redefinicao do cabecalho %%%%%%%%%%%%

% iniciar com estilo empty
\pagestyle{empty}
\begin{document}
...
\pagestyle{fancy} % inicia o cabeçalho personalizado
...
\end{document}
\end{lstlisting}
\end{singlespacing}

Note que o comando \verb+\fancypagestyle+\index{fancypagestyle@\verb+\fancypagestyle+} permite definir/redefinir estilo de páginas, exceto o \texttt{fancy}\index{fancy} que é configurado diretamente, o que requer cuidados. 
Se usar erroneamente o \verb+\fancypagestyle{fancy}+, o \LaTeX{} pode parar no meio da compilação quando encontrar o \verb+\pagestyle{fancy}+.


Para configurar o formato de título do capítulo e seções, usamos o pacote \texttt{titlesec}\index{titlesec}.
Note que, para ajuste somente de capítulos, existe o pacote \texttt{fncychap}\index{fncychap} que permite escolher um modelo entre alguns prontos através da opção do pacote (\texttt{Sonny}\index{fncychap!Sonny}, \texttt{Lenny}\index{fncychap!Lenny}, \texttt{Glenn}\index{fncychap!Glenn}, \texttt{Conny}\index{fncychap!Conny}, \texttt{Rejne}\index{fncychap!Rejne}, \texttt{Bjarne}\index{fncychap!Bjarne}, \texttt{Bjornstrup}\index{fncychap!Bjornstrup}) e ajustar as fontes com comandos tais como \verb+\ChNumVar+\index{fncychap!ChNumVar@\verb+\ChNumVar+} e \verb+\ChTitleVar+\index{fncychap!ChTitleVar@\verb+\ChTitleVar+}. Mas o pacote \texttt{fncychap} não consegue refazer o estilo como no  \texttt{titlesec}.

Assim, para configuração mais refinada de títulos de capítulos e seções, costumam usar o \texttt{titlesec}. 
Este pacote, além de configuração refinada de títulos, também pode ser usado para configurar o cabeçalho.
Note que, para configuração mais refinado de cabeçalhos, costuma usar o \texttt{fancyhdr}.
Veja o Exemplo~\ref{ex15:titlesec}.

%\begin{singlespacing}
%\begin{lstlisting}[caption={ex15-titlesec.tex},label=ex15:titlesec]
\begin{ltxlisting}[caption={ex15-titlesec.tex},label=ex15:titlesec]
\documentclass[12pt,a4paper]{article}
\usepackage[T1]{fontenc}
\usepackage[utf8]{inputenc}
\usepackage[brazil]{babel}

\usepackage{calc} % para cálculo de medidas

\usepackage[pagestyles]{titlesec} % para formatar título do capítulo e seção

% formatando o titulo do capítulo
\titleformat{\chapter}[display] % modo display
{\bfseries\Large} % fonte usado no título do capítulo
{\filleft\MakeUppercase{\chaptertitlename} \Huge\thechapter} % Como colocar o nome do capítulo
{4ex} % espaco entre nome do capítulo e titulo do capítulo
{\titlerule\vspace{2ex}\filright} % o que colocar antes do titulo do capitulo
[\vspace{2ex}\titlerule] % depois do titulo do capítulo

% \titlespacing*{\chapter}{0pt}{20pt}{16pt} % espacamento do titulo de capitulos
      
% Exemplo ajustado da documentação
% formatando o título da seção
\newcommand{\sectiontitlename}{Seção}

\titleformat{\section}[frame]% colocar moldura
{\normalfont}% fonte normal
{\filright\footnotesize
\enspace \MakeUppercase{\sectiontitlename}\enspace \thesection\enspace}% Como colocar o nome/enumeracao da seção
{8pt}% espaço antes do titulo da seção
{\Large\bfseries\filcenter}% antes do titulo da seção (ajustando fontes)

% \titlespacing{\section}{0pt}{*2}{*2}

%%%%%%%%%%%%%%%%%%%%%%%%%%%%%%%%%%%%%%%%%%%%%%%%%%
% definindo o estilo do cabeçalho usando titlesec
% para configuração mais sofisticada, use fancyhdr
    
\newpagestyle{main}{ % Novo estilo de pagina
  \headrule
  \sethead[\thechapter. \scshape\chaptertitle][][]{}{}{\thechapter. \slshape\chaptertitle}
  \setfoot[][\thepage][]{}{\thepage}{}
}
    
\pagestyle{empty}
\begin{document}
\pagestyle{main} % inicia o cabeçalho personalizado
\chapter{Um Capítulo}
Um capítulo novo.
\section{Usando \texttt{titlesec}}
\ldots       
\end{document}
\end{ltxlisting}
%\end{lstlisting}
%\end{singlespacing}

%{ % start group for titlesec
%\titleformat{\chapter}[display]{\bfseries\Large}
%{\filleft\MakeUppercase{\chaptertitlename} \Huge\thechapter}{4ex}{\titlerule\vspace{2ex}\filright}[\vspace{2ex}\titlerule]
%
%\newcommand{\sectiontitlename}{Seção}
%\titleformat{\section}[frame]{\normalfont}{\filright\footnotesize\enspace \MakeUppercase{\sectiontitlename}\enspace \thesection\enspace}{8pt}{\Large\bfseries\filcenter}
%
%\setcounter{lstorichapter}{\value{chapter}}
%\setcounter{lstorisection}{\value{section}}
%\setcounter{chapter}{0}\setcounter{section}{0}
%\let\addcontentsline\emptyaddcontentsline
%\fbox{\begin{minipage}{\linewidth}
%\chapter{Um Capítulo}
%Um capítulo novo.
%\section{Usando \texttt{titlesec}}
%\ldots       
%\end{minipage}}
%} % end group for titlesec
%
%\setcounter{chapter}{\value{lstorichapter}}
%\setcounter{section}{\value{lstorisection}}
%\let\addcontentsline\oriaddcontentsline

Note que a primeira página do capítulo será sempre do estilo \texttt{plain}. 
Então, se quer que a primeira página do capítulo seja diferente do \texttt{plain} padrão, deverá redefinir ele.

Para configurações mais complicadas de títulos de capítulos e seções, use a opção \texttt{explicit}\index{titlesec!explicit} na qual os textos de títulos só serão colocados se for referenciado diretamente. 

Para usar o \texttt{titlesec} e \texttt{fancyhdr} ao mesmo tempo, deverá carregar o \texttt{titlesec} com opção \texttt{pagestyles} e antes do \texttt{fancyhdr}.

%precisará ajustar o carretamento, pois eles definem os comandos \verb+headrule+ e \verb+\footrule+, causando erro no carregamento do segundo pacote.
%Se quer definir títulos de capítulos e seções com \texttt{titlesec} e cabeçalho da página com \texttt{fancyhdr}, carregue da seguinte forma.
%\begin{singlespacing}
%\begin{lstlisting}
%\usepackage[pagestyles]{titlesec}
%\let\headrule\undefined % remover o comando \headrule
%\let\footrule\undefined % remover o comando \footrule
%\usepackage{fancyhdr}
%\end{lstlisting}
%\end{singlespacing}
%Não troque as ordens destes dois pacotes, pois os comandos \verb+\headrule+ e \verb+\footrule+ destes pacotes são incompatíveis.

Para configurar a saída de títulos dos elementos flutuantes como figuras e tabelas que são produzidos pelo comando \verb+\caption+\index{caption@\verb+\caption+}, costuma usar o pacote \texttt{caption}\index{caption} que implementa o comando \verb+\captionsetup+\index{captionsetup@\verb+\captionsetup+} para configurar o formato de títulos das figuras e tabelas. Este pacote também implementa a versão ``*'' do \verb+\caption+\index{caption*@\verb+\caption*+} que não será contabilizado (só coloca o título).

No Exemplo~\ref{ex15:caption}, está configurado o nome da figura no ambiente \texttt{figure} como sendo \texttt{Imagem}. A fonte do nome será em negrito, título em itálico, formato do nome é modo simples e nome com título será separado pelo ponto.

%\setcounter{lstorichapter}{\value{chapter}}
%\setcounter{chapter}{0}
\begin{singlespacing}
%\begin{lstlisting}[caption={ex11-03.tex, parte c},label=ex:11-03:c]
\begin{lstlisting}[caption={ex15-caption.tex},label=ex15:caption]
\documentclass[12pt,a4paper]{article}
\usepackage[T1]{fontenc}
\usepackage[utf8]{inputenc}
\usepackage[brazil]{babel}

\usepackage{caption} % para formatar titulo do float (figura e tabela)

\captionsetup[figure]{name=Imagem,labelfont={bf}, textfont=it, labelformat=simple, labelsep=period}    
\pagestyle{empty}
\begin{document}

\begin{figure}[hbp!]
\center
Figura aqui.
\caption{Novo título}\label{fig:titulo:novo}
\end{figure}

\clearcaptionsetup{figure}

\begin{figure}[hbp!]
\center
Figura aqui.
\caption{Título normal}\label{fig:titulo:normal}
\end{figure}
\end{document}
\end{lstlisting}
\end{singlespacing}
%\setcounter{chapter}{\value{lstorichapter}}

%\fbox{\begin{minipage}{\linewidth}
\begin{ltxlistingout}
\begin{center}
Figura aqui.

{\textbf{Imagem 1.} \textit{Novo título}}
\end{center}
\begin{center}
Figura aqui.

{Figura 2: Título normal}
\end{center}
\end{ltxlistingout}
%\end{minipage}}

%\begin{figure}
%\caption{teste}
%\end{figure}


Para limpar as configurações, usa o comando \verb+\clearcaptionsetup{figure}+\index{clearcaptionsetup@\verb+\clearcaptionsetup+}, onde \texttt{figure} pode ser \texttt{table}, se for da tabela.

Para criar configurações complexas que é difícil de ser ajustado com comandos acima, poderá usar o comando \verb+\DeclareCaptionFormat+\index{DeclareCaptionFormat@\verb+\DeclareCaptionFormat+}.

Por exemplo, se quer que o nome da figura fique em maiúsculo negrito, separado por dois pontos e titulo em itálico, basta criar o estilo novo com 
\begin{singlespacing}
\begin{lstlisting}
\DeclareCaptionFormat{meuestilo}{{\MakeUppercase\bfseries #1}#2\textit{#3}\par}
\end{lstlisting}
\end{singlespacing}
e usar como 
\verb+\captionsetup[figure]{format=meuestilo, labelsep=colon}+.

%geometry, OK
%setspace, OK
%hyphenat, OK
%microtype, OK.
%textcase, OK
%tocbibind, OK
%fancyhdr, OK
%fncychap, OK
%%Options: Sonny, Lenny, Glenn, Conny, Rejne, Bjarne, Bjornstrup
%% \usepackage[Sonny]{fncychap}
%titlesec, OK
%caption, OK

%\begin{verbatim}
%\usepackage{caption}
%\captionsetup[figure]{name=New Table Name,labelfont=it,textfont={bf,it},
%labelformat=simple, labelsep=colon}
%
%Label Format	Result
%labelformat = empty	No label is shown - i.e. no number or letter
%labelformat = simple	Shows number or letter in caption
%labelformat = parens	Shows number or letter in caption in parentheses - i.e. (1), (A)
%
%The labelsep option can be set to:
%Label Separator
%labelsep = none
%labelsep = colon
%labelsep = period
%labelsep = space
%labelsep = quad
%labelsep = newline
%
%\clearcaptionsetup{figure}
%\DeclareCaptionFormat{plain}{#1#2#3\par}
%
%\DeclareCaptionFormat{plain}{{\MakeUppercase #1}#2#3\par}
%
%\end{verbatim}


\section{Links}

Quando usa o pacote \texttt{hyperref}\index{hyperref}, sumário, índice remissivo, as referências cruzadas e citações ganham link automaticamente. 
Além disso, poderá configurar informações gerais do documento PDF gerado, tais como titulo, autor, palavras-chave, etc relacionado com o documento PDF.

A configuração do PDF pode ser feito por algo como segue.
\begin{singlespacing}
\begin{lstlisting}
...
\usepackage{hyperref}

% configurando o PDF
\hypersetup{
  bookmarks=true, % show bookmarks bar?
  unicode=true, % non-Latin characters in Acrobat’s bookmarks
  breaklinks=true, %break long url across lines
%  pdftoolbar=true, % show Acrobat’s toolbar?
%  pdfmenubar=true, % show Acrobat’s menu?
%  pdffitwindow=false, % window fit to page when opened
%  pdfstartview={FitH}, % fits the width of the page to the window
  pdftitle={Pacote hyperref}, % title
  pdfauthor={Sadao Massago}, % author
  pdfsubject={Uso do pacote hyperref}, % subject of the document
%  pdfcreator={Creator}, % creator of the document
%  pdfproducer={Producer}, % producer of the document
%  pdfkeywords={keyword1, key2, key3}, % list of keywords
%  pdfnewwindow=true, % links in new PDF window
%  colorlinks=false, % false: boxed links; true: colored links
%  linkcolor=red, % color of internal links (change box color with linkbordercolor)
%  citecolor=green, % color of links to bibliography
%  filecolor=magenta, % color of file links
%  urlcolor=cyan % color of external links
}
...
\end{lstlisting}
\end{singlespacing}

Note que, apenas carregar o \texttt{hyperref} já é suficiente para maioria dos casos, mas em geral, se estiver usando pacotes que ajustam os comandos de referências cruzadas ou citações, o pacote \texttt{hyperref} deve vir depois deles para não perder os links.

Além do link automático, poderá inserir links para sites, etc, usando o comando \verb+\url+\index{url@\verb+\url+}.
\begin{singlespacing}
\begin{lstlisting}
Veja  \url{https://en.wikibooks.org/wiki/LaTeX/Hyperlinks} por exemplo.
\end{lstlisting}
\end{singlespacing}
Atualmente, o \texttt{url} é importante para ser colocado nas referências bibliográficas.
O estilo moderno para \BibTeX{} costuma usar o campo \texttt{url}\index{BibTeX!url} na qual o seu valor pode ser ser colocado no arquivo de \BibTeX{}, como em
\begin{singlespacing}
\begin{lstlisting}
...
url = {https://en.wikibooks.org/wiki/LaTeX/},
urldate = {2018-06-11},
...
\end{lstlisting}
\end{singlespacing}
onde \texttt{urldate}\index{BibTeX!urldate} é a data da última consulta.

Para estilos um pouco antigo que não tem o campo \texttt{urldate}, poderá colocar a data de último acesso no campo \texttt{note}\index{BibTeX!note} como em
\begin{singlespacing}
\begin{lstlisting}
...
url = {https://en.wikibooks.org/wiki/LaTeX/},
note = {(Visited on 2018-06-11)},
...
\end{lstlisting}
\end{singlespacing}

No caso de estar usando o estilo antigo na qual ignora o campo \texttt{url}, poderá colocar no campo \texttt{note}, usando o comando \verb+\url+.
%A data de acesso em geral será colocado junto a \texttt{note}, ou junto ao ano.
\begin{singlespacing}
\begin{lstlisting}
...
note = "URL: \url{https://en.wikibooks.org/wiki/LaTeX/} (Visited on 2018-06-11)",
...
\end{lstlisting}
\end{singlespacing}
mas cuidado para não colocar em \texttt{note} e \texttt{url} ao mesmo tempo, para evitar duplicações nos estilos que suportam \texttt{url}.

Para deseja um estilo clássico com suporte a \texttt{url}, o pacote \texttt{urlbst}\index{urlbst} dispõe de \texttt{alphaurl}\index{BibTeX!alphaurl}, \texttt{plainurl}\index{BibTeX!alphaurl}, \texttt{abbrvurl}\index{BibTeX!abbrvurl} e \texttt{unsrturl}\index{BibTeX!unsrturl} correspondentes a \texttt{alpha}, \texttt{plain}, \texttt{abbrv} e \texttt{unsrt} respectivamente. Note que, para estes estilos, a data de último acesso é no campo \texttt{lastchecked}\index{BibTeX!lastchecked} e não no \texttt{urldate}. Para que funcione tanto nestes estilos quanto no estilo mais moderno, poderá colocar ambos os campos.

Obviamente, o arquivo \texttt{tex} correspondente ao estilo bibliográfico com URL deve usar o pacote \texttt{hyperref} ou o \texttt{url}.

Quando quer acrescentar o link com texto onde clicar é diferente do endereço do link, poderá usar o \verb+\href+ como em

\begin{singlespacing}
\begin{lstlisting}
\href{https://en.wikibooks.org/wiki/LaTeX/}{wikibooks, \LaTeX{}}
\end{lstlisting}
\end{singlespacing}

Também poderá criar os links manualmente, dentro do documento.
Para isso, define a chegada do link com o comando \verb+\hypertarget+\index{hypertarget@\verb+\hypertarget+} e cria link para ele, com o comando \verb+\hyperlink+\index{hyperlink@\verb+\hyperlink+}.
Veja o \url{https://en.wikibooks.org/wiki/LaTeX/Hyperlinks} para detalhes.

Quando usa as fórmulas nos títulos, aparece ``lixo'' no indicador (\texttt{bookmarks}) do PDF.
Isto porque, no indicador, só o texto é aceito. 
Para evitar isso, existe o comando \verb+\texorpdfstring+\index{textorpdfstring@\verb+\textorpdfstring+} que coloca conteúdo de \LaTeX{} ou texto, dependendo de estar no documento ou no \texttt{bookmarks}.

Por exemplo,
\begin{singlespacing}
\begin{lstlisting}
\section{\texorpdfstring{$E=mc^2$}{E = mc ** 2}}
\end{lstlisting}
\end{singlespacing}

Usará $E=mc^2$ para o documento, mas \texttt{E = mc ** 2} para o indicador (\texttt{bookmarks}).

Alguns comandos pode não funcionar no \texttt{bookmarks} do PDF, como o caso de \verb+\MakeUppercase+.
O pacote \texttt{hyperref} dispõe de comando \verb+\pdfstringdefDisableCommands+\index{pdfstringdefDisableCommands@\verb+\pdfstringdefDisableCommands+}
para redefinir estes comandos dentro do \texttt{bookmarks}.
Por exemplo, para desativar o \verb+\MakeUppercase+ ou redefinir \verb+\\+ dentro do \texttt{bookmarks} do PDF, coloque
\begin{singlespacing}
\begin{lstlisting}
\pdfstringdefDisableCommands{%
  \let\MakeUppercase\relax%desativando
  \def\\{ }% em vez da nova linha, será espaço
}
\end{lstlisting}
\end{singlespacing}
no \texttt{preamble} do documento.

Quando adiciona um item manualmente no sumário através do comando \verb+\addcontentsline+\index{addcontentsline@\verb+\addcontentsline+}, o link do sumário pode não posicionar corretamente na página correta.
Para o funcionamento correto (posicionamento correto) do link no sumário requer que o \verb+\addcontentsline+ deve ser chamado logo em seguida da mudança de capítulo ou seção.
Caso não esteja, é só colocar \verb+\phantomsection+\index{phantomsection@\verb+\phantomsection+} oferecido pelo pacote \texttt{hyperref} antes do \verb+\addcontentsline+. 

Note que ``\verb+"|+'' na entrada do índice remissivo para inserir o símbolo ``\verb+|+'' não funciona em conjunto com o pacote \texttt{hyperref}.
A forma de contornar isso é definir um comando para ``\verb+|+'' e usar na entrada do índice remissivo.

%
%hyperref,
%
%\verb+\section{\texorpdfstring{$E=mc^2$}{E = mc ** 2}}+ para bookmark do PDF (hyperref)? Uso do comando \verb+\url{}+ no arquivo de BibTeX.

\section{Controle das listas e listas \texttt{inline}}

O pacote \texttt{enumerate} permite controlar as enumerações das listas.
Mas as vezes queremos efetuar controle mais refinado ou controlar o rótulo de listas não enumeradas também.
O pacote \texttt{enumitem}\index{enumitem} permite formatar a lista de forma similar ao pacote \texttt{enumerate}, mas com recursos adicionais.

Existe também os casos que queremos as listas que coloquem vários itens na mesma linha, denominado de listas \texttt{inline}\index{lista!inline}. 
Existem vários pacotes que permitem criar listas \texttt{inline}, mas esta tarefa também pode ser feita pelo pacote \texttt{enumitem}.

A opção \texttt{shortlabels}\index{enumitem!shortlabels} no pacote \texttt{enumitem} ativa o modo de compatibilidade com o pacote \texttt{enumerate}, permitindo formatar enumeração da lista \texttt{enumerate} via modelos.
A opção \texttt{inline}\index{enumitem!inline} criará a versão ``*'' dos ambientes de lista que são listas \texttt{inline} (não muda de linha entre os itens).
Para alinhar os itens das listas versão ``*'', poderá usar o pacote \texttt{tabto}\index{tabto}.

Veja o Exemplo~\ref{ex:d:enumitem} que usa os pacotes \texttt{enumitem} e \texttt{tabto}.

\begin{ltxlisting}[caption={ex-d-enumitem.tex},label=ex:d:enumitem]
\documentclass[12pt,a4paper]{article}
\usepackage[T1]{fontenc} % codificação da fonte em 8-bits
\usepackage[utf8]{inputenc} % acentuação direta
\usepackage[brazil]{babel} % em português brasileiro
\usepackage{amssymb} %para comando \square
\usepackage[inline,shortlabels]{enumitem}
\usepackage{tabto}
\usepackage{xcolor} % para mudar de cor
\begin{document}
% Em vez de indentar, cria espaço extra entre parágrafos
\parindent=0em % indentação nula
\parskip=\baselineskip % um linha entre parágrafos

% No parâmetro opcional, aceita a formatação de enumeração como do pacote enumerate (ativado pela opção shortlabels)
% Também poderá configurar as fontes
% As opções são separados pela vírgula
\begin{enumerate}[1),font=\color{blue}]
\item Item 1
\item Item 2
\item Item 3
\end{enumerate}

% Nas listas como o enumerate e itemize, poderá configurar o \label que rotula os itens. Neste exemplo, será colocado um circulo na enumeração.
% O "*" no label será substituido pelo enumi, enumii, etc dependendo do seu nível.
\begin{enumerate}[label={\large\protect\textcircled{\normalsize\arabic*}}]
\item Item 1
\item Item 2
\item Item 3
\end{enumerate}

%A versão ``*'' produz lista inline que não efetua mudança de linhas quando muda o item.
\begin{enumerate*}[a)]
\item Item 1
\item Item 2
\item Item 3
\item Item 4
\item Item 5
\item Item 6
\item Item 7
\item Item 8
\item Item 9
\item Item 10
\end{enumerate*}

% Para alinhas os ítens, use em conjunto com tabto.
% Quantas "colunas" vai ter na tabulação (que será usada na lista versão "*")
\NumTabs{5}
% Na lista versão *, poderá indicar o que colocar entre itens consecutivos com itemjoin
% No caso, o \tab será usado para alinhar itens em colunas.
\begin{enumerate*}[1.,itemjoin={\tab}]
\item Item 1
\item Item 2
\item Item 3
\item Item 4
\item Item 5
\item Item 6
\item Item 7
\item Item 8
\item Item 9
\item Item 10
\end{enumerate*}

%\NumTabs{4} % poderá mudar o número de colunas, se desejar
% Ítens longas podem ocupar dois ou mais "colunas" automaticamente
\begin{itemize*}[label=$\square$, itemjoin={\tab}]
\item Item 1
\item Item 2
\item Item 3
\item Item 4
\item Item 5
\item Item 6 e longa, com espaço maior
\item Item 7
\item Item 8
\end{itemize*}
\end{document}
\end{ltxlisting}


\section{Trocando as fontes}

\LaTeX{} dispõe de diversas fontes gratuitas para diagramar o documento.
Em geral, usamos o \texttt{Latin Modern}\index{Latin Modern} que é uma extensão do \texttt{Computer Modern}\index{Computer Modern} para derivados de latim que tem letras acentuadas.
Para isso, é só carregar o pacote \texttt{lmodern}\index{lmodern} com o comando
\begin{singlespacing}
\begin{lstlisting}
\usepackage{lmodern} % latin modern
\end{lstlisting}
\end{singlespacing}
%
no preamble.

Aqui, veremos algumas combinações que podem ser usadas, entre muitas outras.

Concrete (texto e fórmula)
\begin{singlespacing}
\begin{lstlisting}
\usepackage{lmodern} % para complementar
\usepackage{ccfonts} % Concrete (serifa grossa?)
\end{lstlisting}
\end{singlespacing}

Fonte \texttt{Times} pode ser escolhido com
\begin{singlespacing}
\begin{lstlisting}
\usepackage{newtxtext,newtxmath}
\end{lstlisting}
\end{singlespacing}

A fonte \texttt{Palatino} com
\begin{singlespacing}
\begin{lstlisting}
\usepackage{newpxtext,newpxmath}
\end{lstlisting}
\end{singlespacing}

A fonte \texttt{iwona} ideal para \texttt{poster} e \texttt{slides}.
\begin{singlespacing}
\begin{lstlisting}
\usepackage{lmodern} % para complementar
\usepackage[math]{iwona} % sans serif com transição grosso fino
\end{lstlisting}
\end{singlespacing}

Fonte \texttt{Schoolbook L}
\begin{singlespacing}
\begin{lstlisting}
\usepackage{lmodern} % para complementar
\usepackage{fouriernc} % schoobook L: Serifa grossa?
\end{lstlisting}
\end{singlespacing}

A fonte \texttt{stix} (estilo compatível com \texttt{Times New Roman})
\begin{singlespacing}
\begin{lstlisting}
\usepackage{newtxtext} % para complementar
\usepackage{stix} % compativel com New Times Roman
\end{lstlisting}
\end{singlespacing}

Note que, para complementar a fonte com a outra, como do \texttt{amssymb}, deve carregar o complemento antes do pacote de fonte desejado. 
Quando vários pacotes de fontes são carregados e tiver conflitos, o que vale é do pacote carregado por último.

Também existem pacotes de fontes que acrescentam comandos para fontes adicionais. 
Por exemplo, o pacote \texttt{dsfont}\index{dsfont} adiciona o comando \verb+\mathds+ para usar a fonte ``negrito de quadro negro'' alternativo dentro das fórmulas.
Da forma análoga, o pacote \texttt{calligra}\index{calligra} acrescenta o comando \verb+\textcalligra+\index{textcalligra@\verb+\textcalligra+} para textos na fonte caligráfica.

Para ver as fontes livres do \LaTeX{} e os pacotes correspondentes, veja o site \url{http://www.tug.dk/FontCatalogue/}.

\section{Texto somente com contorno e sombreado}
No caso de cartazes e folhetos, as vezes usamos texto somente com contorno para títulos na qual usa-se a letra grande.
Uma forma fácil de fazer isso é usar o pacote \texttt{contour}\index{contour}
que permite criar um contorno no elemento. 
Veja o Exemplo~\ref{ex15:contour}.

%\begin{singlespacing}
%\begin{LTXexample}[pos=b,caption={ex11-06.tex},label=ex:11-06]
\begin{ltxlisting}[caption={ex15-contour.tex},label=ex15:contour]
\documentclass[12pt,a4paper]{article}
\usepackage[utf8]{inputenc}
\usepackage[brazil]{babel}
\usepackage{contour}
%\contourlength{0.5pt} % controle da espessura do contorno
\begin{document}
{\Huge \contour{red}{\color{white}\scshape Atenção:}}
\end{document}
\end{ltxlisting}
%\end{LTXexample}
%\end{singlespacing}

Para sombrear o texto, costuma usar o pacote \texttt{shadowtext}.
Veja o Exemplo~\ref{ex15:shadow}.

%\begin{singlespacing}
%\begin{LTXexample}[pos=b,caption={ex11-06.tex},label=ex:11-06]
\begin{ltxlisting}[caption={ex15-shadow.tex},label=ex15:shadow]
\documentclass[12pt,a4paper]{article}
\usepackage[utf8]{inputenc}
\usepackage[brazil]{babel}
\usepackage{shadowtext}
\shadowcolor{blue!40!white}
%\shadowoffset{2pt} % deslocamento da sombra
\begin{document}
{\Huge \shadowtext{\sffamily Atenção}}
\end{document}
\end{ltxlisting}
%\end{LTXexample}
%\end{singlespacing}

\section{Controle das figuras e similares}

As vezes queremos que o elemento flutuante (figuras e tabelas) sejam inseridas exatamente no lugar onde foi colocado. 
Para isso, existe o pacote \texttt{float}\index{float} que fornece a opção extra ``H''\index{float!H} nas figuras e tabelas na qual impede de auto posicionar (manter no lugar). 
Para usar, é só carregar o pacote no \texttt{preamble} e usar o ``H'' como opção de posicionamento das figuras e tabelas que querem que fiquem no lugar.
Outro recurso do pacote \texttt{float} é alterar algumas configurações do elemento flutuante (como colocar moldura nas figuras), e criar um novo elemento flutuante.
Veja o Exemplo~\ref{ex15:float}

%\begin{singlespacing}
%\begin{LTXexample}[pos=b,caption={ex11-06.tex},label=ex:11-06]
\begin{ltxlisting}[caption={ex15-float.tex},label=ex15:float]
\begin{figure}[H] % No lugar
\centering
\begin{tikzpicture}
\draw (0,0) -- (1,0) -- (0.5, 0.5) -- cycle;
\end{tikzpicture}
\caption{Figura em onde foi inserido}
\end{figure}
\end{ltxlisting}
%\end{LTXexample}
%\end{singlespacing}

Para colocar subfiguras ou subtabelas, use o pacote \texttt{subcaption}\index{subcaption}.
Tendo \verb+\usepackage{subcaption}+ no \texttt{preamble}, podemos produzir subfiguras e subtabelas facilmente. 
Veja o Exemplo~\ref{ex15:subcaption}.

%\begin{singlespacing}
%\begin{LTXexample}[pos=b,caption={ex11-06.tex, parte b},label=ex:11-06:b]
\begin{ltxlisting}[caption={ex15-subcaption.tex},label=ex15:subcaption]
\begin{figure}[hbp!]
    \centering
    \begin{subfigure}[t]{0.45\linewidth}
        \centering
        Sub figura aqui.
        \caption{sub título 1}
    \end{subfigure}%
    ~
    \begin{subfigure}[t]{0.45\linewidth}
        \centering
        Segunda sub figura aqui.
        \caption{sub título 2}
    \end{subfigure}
    \caption{Uso de sub figura}
\end{figure}
\end{ltxlisting}
%\end{LTXexample}
%\end{singlespacing}

Se a classe/pacote conflitar com \texttt{subcaption}, poderá optar pelo pacote \texttt{subfig}\index{subfig}.

Para que o texto contorne as figuras ou tabelas, poderá usar o pacote \texttt{wrapfig}\index{wrapfig} ou \texttt{floatflt}\index{floatflt}. 
O Exemplo~\ref{ex15:wrapfig} ilustra o uso de \texttt{wrapfig}.
 
%\begin{singlespacing}
%\begin{LTXexample}[pos=b,caption={ex11-06.tex, parte c},label=ex:11-06:c]
\begin{ltxlisting}[caption={ex15-wrapfig.tex},label=ex15:wrapfig]
\begin{wrapfigure}{l}{0.3\linewidth}
\centering
\begin{tikzpicture}
\draw (0,0) rectangle (3,1);
\end{tikzpicture}
\caption{Figura com texto contornando}
\end{wrapfigure}
A figura deve ser colocado dentro do ambiente wrapfigure. No exemplo, ``l'' significa left (esquerda) que posiciona a figura a esquerda (e texto a direita). Se colocar o ``r'' (right), a figura ficará a direita. O próximo parâmetro é a largura reservada para figura que foi 30\% da largura da linha. 
Note que são contadas junto as outras figuras.
Se pretende colocar tabela com texto em torno dele, use o ambiente \texttt{wraptable}, também do pacote \texttt{wrapfig}.
\end{ltxlisting}
%\end{LTXexample}
%\end{singlespacing}

Quando tem figura ou tabela muito larga, precisará rotacionar a página.
Isto pode ser feito pelo pacote \texttt{pdflscape}\index{pdflscape}.

Para que uma página fique rotacionada, coloque dentro do ambiente \texttt{landscape}\index{landscape}.
Note que, para rotacionar tabelas longas que pode ocupar mais de uma página, deverá usar o pacote \texttt{rotating}\index{rotating}.
Veja o Exemplo~\ref{ex15:pdflscape}.

\begin{singlespacing}
%\begin{lstlisting}[caption={ex11-06.tex, parte d},label=ex:11-06:d]
\begin{lstlisting}[caption={ex15-pdflscape.tex},label=ex15:pdflscape]
\begin{landscape}
\center
\begin{figure}[H]
\center
\begin{tikzpicture}
\draw[fill=gray!20] (0,0) -- (\linewidth,0) -- (\linewidth,5) -- (0, 5) -- cycle;
\end{tikzpicture}
\caption{Figura larga rotacionada}
\end{figure}
\end{landscape}
\end{lstlisting}
\end{singlespacing}
%
%\noindent
%\fbox{\begin{minipage}{\linewidth}
\begin{ltxlistingout}
\center
\rotatebox{-90}{\begin{minipage}{0.5\linewidth}
\center
\begin{tikzpicture}
\draw[fill=gray!20] (0,0) -- (\linewidth,0) -- (\linewidth,5) -- (0, 5) -- cycle;
\end{tikzpicture}
{Figura: Figura larga rotacionada}
\end{minipage}}
\end{ltxlistingout}
%\end{minipage}}
 
% 
%float, OK
%wrapfig (ou floatflt), OK
%subfigure, OK.
%
%
%rotating (sideways,sidewaysfigure,sidewaystable) ou lscape/pdflscape? OK

\section{Escrevendo medidas internacionais}

Para escrever medidas, requer alguns cuidados tais como usar letra romana reta (por ser abreviatura da palavra), ter pequeno espaço entre valor e medida, entre outros.
Por exemplo, $\SI{10}{\kilo\gram}$ e não $10kg$, $\SI{5}{\litre}$ e não $5l$ e assim por diante.


Para facilitar a escrever unidades de medidas corretamente no documento, o \texttt{siunitx}\index{siunitx} providencia comandos para escrever na unidade internacional.

Para produzir o número rapidamente, tem o comando \verb+\num+\index{num@\verb+\num+}.
O comando \verb+\ang+\index{ang@\verb+\ang+} gera ângulos rapidamente.
Para exemplos a seguir, será assumido que foi carregado o pacote \texttt{siunitx} no \texttt{preamble}
Veja o Exemplo~\ref{ex15:num}.

%\begin{singlespacing}
%\begin{LTXexample}[pos=b,caption={ex11-09.tex},label=ex:11-09]
\begin{ltxlisting}[caption={ex15-num.tex},label=ex15:num]
Números

\num{12345.67890} \\ % espacando de 3 em 3 casas
\num{12345,67890} \\ % ``,'' também pode ser usado para decimal
\num{1+-2i}       \\ % ``+-'' vira $\pm$
\num{.3e45}       \\ % notacao cientifica
\num{1.654 x 2.34 x 3.430} % ``x'' vira $\times$

Ângulos

\ang{10}    \\ % grau
\ang{5.3}  \\ % grau com decimal
\ang{-1,5}   \\ % ``,"" também pode ser usado como decimal
\ang{1;2;3} \\ % grau, minuto e segundo
\ang{;;1}   \\ % so o segundo
\end{ltxlisting}
%\end{LTXexample}
%\end{singlespacing}

O comando \verb+\si+\index{si@\verb+\si+} produz unidade de medida de acordo com o seu parâmetro. 
A configuração de saída pode ser efetuado pelo parâmetro opcional.
O \verb+\si+, assim como \verb+\num+ e \verb+\ang+, funciona tanto no modo texto como no modo matemático.

Em geral, coloca-se pequeno espaço entre valor e medida.
O comando \verb+\SI+\index{SI@\verb+\SI+} automatiza isto, além de poder colocar símbolo antes do valor.

Ele tem a forma
\verb+\SI[opção]{valor}[simbolo anterior]{unidade}+
Veja o Exemplo~\ref{ex15:si}.

%\begin{singlespacing}
%\begin{LTXexample}[pos=b,caption={ex11-09.tex, parte b},label=ex:11-09:b]
\begin{ltxlisting}[caption={ex15-si.tex},label=ex15:si]
Entrada literal (Neste modo, não há opção de formatação).

\si{kg.m/s^2} \\
\si{g_{polymer}~mol_{cat}.s^{-1}}

Entrada pelo macro (permite configurar a formatação usando opção do comando ou do parâmetro do pacote).

\si{\kilo\gram\per\square\second} \\
\si{\gram_{polymer}\mol_{cat}\per\second}

Mais exemplos\\
\si{\kilo\gram\metre\per\square\second} \\
\si{\gram\per\cubic\centi\metre}        \\
\si{\square\volt\cubic\lumen\per\farad} \\
\si{\metre\squared\per\gray\cubic\lux}  \\
\si{\henry\second}

Medidas com valor

\SI{1.23}{J.mol^{-1}.K^{-1}}  \\ % modo textual
\SI{.23e7}{\candela}          \\ % como macros
\SI{1.99}[R\$]{\per\kilogram} \\ % com simbolo antes
\SI{1,345}{\coulomb\per\mole} % outro exemplo

Colocar medida após valor usando \verb+\si+ e 
colocar valor já com medida usando \verb+\SI+ 
pode apresentar diferença no espaçamento. \\
\si{10\celsius} \\ % valor concatenado com medida
\SI{10}{\celsius} % valor e medida usando macro
\end{ltxlisting}
%\end{LTXexample}
%\end{singlespacing}

Alguma das medidas padrão são: 
\verb+\ampere+\index{si!ampere@\verb+\ampere+}, \verb+\candela+\index{si!candela@\verb+\candela+}, \verb+\kelvin+\index{si!kelvin@\verb+\kelvin+}, \verb+\kilogram+\index{si!kilogram@\verb+\kilogram+}, \verb+metre+\index{si!metre@\verb+\metre+} ou \verb+meter+\index{si!meter@\verb+\meter+},
\verb+\mole+\index{si!mole@\verb+\mole+}, \verb+\second+\index{si!second@\verb+\second+}, \verb+\litre+\index{si!litre@\verb+\litre+}, \verb+\celsius+\index{si!celsius@\verb+\celsius+}, \verb+\percent+\index{si!percent@\verb+\percent+}, etc que são muitas. 
Para saber mais, consulte o manual do pacote.

Para conversão de medidas, 
\verb+\giga+\index{si!giga@\verb+\giga+}, \verb+\mega+\index{si!mega@\verb+\mega+}, \verb+\kilo+\index{si!kilo@\verb+\kilo+}, \verb+\hecto+\index{si!hecto@\verb+\hecto+}, \verb+\deca+\index{si!deca@\verb+\deca+}, \verb+\deci+\index{si!deci@\verb+\deci+}, \verb+\centi+\index{si!centi@\verb+\centi+}, \verb+\milli+\index{si!milli@\verb+\milli+}, \verb+\micro+\index{si!micro@\verb+\micro+}, \verb+\nano+\index{si!nano@\verb+\nano+}, etc. são disponíveis.

Exemplo: \verb+\SI{20}{\kilo\gram}+ e mesmo que \verb+\SI{20}{\kilogram}+.

Para operar, tem o \texttt{per, square}, etc.

\verb+\SI{30}{\kilo\meter\per\square\second}+
é mesmo que 
\verb+\SI{30}{\kilo\meter/\second^2}+
no modo padrão, mas o segundo não consegue controlar a formatação por usar forma literal em ``/'' e ``\verb+^2+''.

A configuração geral de formatação pode ser feito pela opção do pacote ou pelo comando \verb+\sisetup+\index{sisetup@\verb+\sisetup+}, mas também pode especificar localmente como no Exemplo~\ref{ex15:sisetup}.


%\begin{singlespacing}
%\begin{LTXexample}[pos=b,caption={ex11-09.tex, parte c},label=ex:11-09:c]
\begin{ltxlisting}[caption={ex15-sisetup.tex},label=ex15:sisetup]
\SI[mode=text]{30}{\kilo\meter\per\square\second}

\SI[per-mode=power]{30}{\kilo\meter\per\square\second}

\SI[per-mode=symbol]{30}{\kilo\meter\per\square\second}

\SI[per-mode=fraction]{30}{\kilo\meter\per\square\second}
\end{ltxlisting}
%\end{LTXexample}
%\end{singlespacing}

\section{Enfeites e código fontes}
Para enfeitar o documento, como colocar moldura nos elementos, usamos os pacotes adicionais, exceto molduras simples fornecidos pelo \verb+\fbox+\index{fbox@\verb+\fbox+}.

O pacote \texttt{fancybox}\index{fancybox} oferece caixas com molduras extras tais como \verb+\shadowbox+\index{shadowbox@\verb+\shadowbox+} (com sombra), \verb+doublebox+\index{doublebox@\verb+\doublebox+} (moldura dupla) e \verb+\ovalbox+\index{ovalbox@\verb+\ovalbox+} (moldura com quinas arredondadas).

Note que \verb+\shadowbox+ não permite controlar a cor da sombra. 
Se quer caixas mais sofisticadas, poderá usar o pacote \texttt{tcolorbox}\index{tcolorbox}.
No caso de querer construir uma caixa personalizada, também podemos usar o pacote gráfico tal como o \texttt{tikz}.
Veja o Exemplo~\ref{ex15:fancybox}.

%\begin{singlespacing}
%\begin{LTXexample}[pos=b,caption={ex11-08.tex},label=ex:11-08]
\begin{ltxlisting}[caption={ex15-fancybox.tex},label=ex15:fancybox]
\shadowbox{Caixa com sombra}

\doublebox{Caixa com moldura dupla}

\ovalbox{Caixa com quina arredondada}

\Ovalbox{Outra caixa com quina arredondada}
\end{ltxlisting}
%\end{LTXexample}
%\end{singlespacing}

Lembre-se que, se o conteúdo tiver parágrafo, deverá usar em conjunto com o \texttt{minipage}.
Também permite colocar conteúdo ou moldura na página com comandos tais como \verb+\fancyput+\index{fancyput@\verb+\fancyput+}, \verb+\thisfancyput+\index{thisfancyput@\verb+\thisfancyput+}, \verb+\fancypage+\index{fancypage@\verb+\fancypage+} e \verb+\thisfancypage+\index{thisfancypage@\verb+\thisfancypage+}.

Por exemplo, o código
\begin{singlespacing}
\begin{lstlisting}
\thisfancypage{%
\setlength{\fboxsep}{8pt}%
\setlength{\shadowsize}{8pt}%
\shadowbox}{}
\end{lstlisting}
\end{singlespacing}
acrescenta moldura com sombra na página atual.

Para colocar moldura nos conteúdos longos que podem ocupar mais de uma página, podemos usar o pacote \texttt{framed}\index{framed}.
Para colocar cor de fundo, deverá definir a cor do fundo \texttt{shadecolor}\index{shadowcolor}. 
Para isso, é recomendado usar o pacote \texttt{xcolor} visto na Seção~\ref{sec:xcolor} do Capítulo~\ref{chap:xcolor:tikz} (não é o pacote \texttt{color}) que tem facilidade de manipulação das cores.
Assim, para executar o código do Exemplo~\ref{ex15:framed}, deverá carregar tanto o pacote \texttt{framed} como o \texttt{xcolor}.

%\begin{singlespacing}
%\begin{LTXexample}[pos=b,caption={ex11-08.tex, parte b},label=ex:11-08:b]
\begin{ltxlisting}[caption={ex15-framed.tex},label=ex15:framed]
\begin{framed}
O texto com moldura pode ser produzido facilmente com o pacote \texttt{framed}.
\end{framed}

\colorlet{shadecolor}{black!15} % cor do fundo usando xcolor
\begin{shaded*}
O texto com fundo pode ser produzido facilmente com o pacote \texttt{framed}.
\end{shaded*}

\begin{leftbar}
O ambiente \texttt{leftbar} coloca traço no lado esquerdo do texto.
\end{leftbar}
\end{ltxlisting}
%\end{LTXexample}
%\end{singlespacing}

O comando \verb+\colorlet+\index{colorlet@\verb+\colorlet+} do pacote \texttt{xcolor} permite definir cor a partir das cores existentes.
\verb+black!15+ é 15\% de preto. Veja a Seção~\ref{sec:xcolor} do Capítulo~\ref{chap:xcolor:tikz} para mais detalhes sobre recursos de \texttt{xcolor}.

Note que, diferente das caixas padrões do \LaTeX{} ou do pacote \texttt{graphicx}, os ambientes do \texttt{framed} podem quebrar em linhas e em páginas.

Para indicar as correções a serem efetuadas, é útil ter linhas enumeradas.
Para enumerar as linhas do documento, usa-se o pacote \texttt{lineno}\index{lineno}.

Para que as fórmulas também sejam enumeradas, use
\begin{singlespacing}
\begin{lstlisting}
\usepackage[mathlines]{lineno}
\end{lstlisting}
\end{singlespacing}
%
no \texttt{preamble}.
Os comandos \verb+\linenumbers+\index{linenumbers@\verb+\linenumbers+} e \verb+\nolinenumbers+\index{nolinenumbers@\verb+\nolinenumbers+} são usados para ativar/desativar as enumerações.
Se pretende enumerar todo documento, coloque \verb+\linenumbers+ no \texttt{preamble}.
Para enumerar a cada ``<n>'' linhas,use \verb+\modulolinenumbers[<n>]+\index{modulolinenumbers@\verb+\modulolinenumbers+}.
Veja o Exemplo~\ref{ex15:lineno}.

%\begin{singlespacing}
%\setlength{\linenumbersep}{0pt} % default is 10pt
%\begin{internallinenumbers}
%\begin{lstlisting}[caption={ex12-06.tex},label=ex:12-06]
\begin{lstlisting}[caption={ex15-lineno.tex},label=ex15:lineno]
\linenumbers % ativa a enumeracao das linhas
\modulolinenumbers[3] % enumerar de 3 em 3

\lipsum[1] % preencher

\nolinenumbers % desativa a enumeração das linhas
\lipsum[1] % 1 paragrafos

% enumerando so um trecho.
\resetlinenumber % reset the line number
\modulolinenumbers[1] % enumerar todas
\begin{linenumbers}
   \lipsum[2] % 2 paragrafos
\end{linenumbers}
\end{lstlisting}
%\end{lstlisting}
%%\end{internallinenumbers}
%\end{singlespacing}
%
\begin{singlespacing}
\noindent
\fbox{\parbox{\linewidth-0.8em}{\hspace*{1em}
\begin{minipage}{\linewidth-1.8em}
%\setlength{\linenumbersep}{-5pt} % default is 10pt
\linenumbers % ativa a enumeracao das linhas
\modulolinenumbers[3] % enumerar de 3 em 3
\begin{internallinenumbers}
\lipsum[1] % preencher
\end{internallinenumbers}
\nolinenumbers % desativa a enumeração das linhas
\lipsum[1] % 1 paragrafos

% enumerando so um trecho.
\resetlinenumber % resset the line number
\modulolinenumbers[1] % enumerar todas
\begin{internallinenumbers}
%\begin{linenumbers}
        \lipsum[2] % 3 paragrafos
%\end{linenumbers}
\end{internallinenumbers}
\end{minipage}}}
%\setlength{\linenumbersep}{10pt} % set to default
\end{singlespacing}

No Exemplo~\ref{ex15:lineno} foi usado o comando \verb+\lipsum+\index{lipsum@\verb+\lipsum+} que gera textos aleatórios em grego para preencher as páginas para testes. 
Logo, precisa carregar o pacote \texttt{lipsum}\index{lipsum} no \texttt{preamble} para que o exemplo funcione.

Existem vários pacotes que permitem colocar conteúdos em todas as páginas, como no caso de marca d'água. 
Entre eles, vamos ver o caso do uso de \texttt{background}\index{background} e %, \texttt{draftwatermark}\index{draftwatermark} e
\texttt{eso-pic}\index{eso-pic}.
% \texttt{xwatermark}\index{xwatermark}.

O código
%
\begin{singlespacing}
\begin{lstlisting}
\documentclass{article}
\usepackage{background}
\usepackage{tikz} % necessário carregar para usar background
\usepackage{xcolor}
\usepackage{graphicx}
\usepackage{lipsum}
\backgroundsetup{contents={\sffamily\bfseries RASCUNHO}}
\begin{document}
\lipsum
\end{document}
\end{lstlisting}
\end{singlespacing}

Coloca a marca d'àgua ``RASCUNHO'' em todas páginas, usando o pacote \texttt{background}\index{background}.
Note que, se quer colocar a imagem externa, basta usar o \verb+\includegraphics+ em vez de texto.

%O mesmo efeito pode ser obtido pelo pacote \texttt{draftwatermark}\index{draftwatermark} como segue.
%
%\begin{singlespacing}
%\begin{lstlisting}
%\documentclass{article}
%\usepackage{draftwatermark}
%\usepackage{tikz}
%\usepackage{xcolor}
%\usepackage{graphicx}
%\usepackage{lipsum}
%\DraftwatermarkOptions{text={\sffamily\bfseries RASCUNHO}} % versão 2.x (nova)
%% \SetWatermarkText{\sffamily\bfseries\Huge RASCUNHO} % versão antiga
%\begin{document}
%\lipsum
%\end{document}
%\end{lstlisting}
%\end{singlespacing}
%
%O comando \verb+\DraftwatermarkOptions+ pode ser chamado no meio do documento, o que é útil para alterar a marca d'àgua ou ativar/desativar localmente.

% xwatermark não está tendo manutenção, não funcionado no sitema novo.
%No caso do \texttt{xwatermark}\index{xwatermark}, tem opção de colocar a marca d'àgua na frente, em vez do lado de trás. O código
%%
%\begin{singlespacing}
%\begin{lstlisting}
%\documentclass{article}
%\usepackage{xwatermark}
%\usepackage{xcolor}
%\usepackage{graphicx}
%% versão ``*'' coloca na na frente da figura
%\newwatermark*[allpages,color=red!50,angle=45,scale=3,xpos=0,ypos=0]{RASCUNHO}
%\begin{document}
%\lipsum[1-2]
%\begin{figure}[!ht]
%\centering
%\includegraphics[width=0.3\linewidth]{minha-figura}
%\end{figure}
%\lipsum[1-2]
%\end{document}
%\end{lstlisting}
%\end{singlespacing}
%
%Coloca RASCUNO em todas as  páginas. Usando a versão ``*'', poderá colocar na frente da figura ``minha-figura''.

Outro pacote é o \texttt{eso-pic}\index{eso-pic}.

Por exemplo, o código
%
\begin{singlespacing}
\begin{lstlisting}
\documentclass{article}
\usepackage{graphicx}
\usepackage{eso-pic}
\usepackage{xcolor}
\usepackage{lipsum}
% marca d'agua
% Versão sem * é na toda página. Versão * é somente na página corrente
\AddToShipoutPictureBG{%
  %\AtPageLowerLeft{% Este é padrão
  \unitlength=\paperwidth
    \put(0.3,0.5){\rotatebox{45}{%
      \scalebox{3}{\Huge\sffamily\bfseries\color{black!25}Rascunho}}%
    }
  %}%\AtPageLowerLeft
}%\AddToShipoutPictureBG
\begin{document}
\lipsum[1-4]
\end{document}
\end{lstlisting}
\end{singlespacing}
%
após carregar o pacote \texttt{eso-pic}, coloca a palavra ``Rascunho'' rotacionado por $45^\circ$ em todas as páginas como marca d'água.
O comando \verb+\AddToShipoutPictureBG+ criará um ambiente \texttt{picture}. Logo, podemos colocar os comandos aceitos no ambiente \texttt{picture} dentro dele.

Para limpar o fundo, usa-se o \verb+\ClearShipoutPictureBG+\index{ClearShipoutPictureBG@\verb+\ClearShipoutPictureBG+}.

Os comandos do \texttt{eso-pic} tem a versão ``BG'' para atrás da página como marca d'agua e versão ``FG'' para colocar na frente da página. Por exemplo, \verb+\AddToShipoutPictureFG+\index{AddToShipoutPictureFG@\verb+\AddToShipoutPictureFG+} coloca na frente em vez de trás da página.

Para colocar código fonte de um programa, costuma usar o pacote \texttt{listings}\index{listings} que implementa o ambiente \texttt{lstlisting}\index{lstlisting} na qual formata o código de acordo com a linguagem de programação escolhida.

Ele já vem com configurações para diversas idiomas,  tais como \texttt{Ada}, \texttt{Algol}, \texttt{Assember}, \texttt{awk}, \texttt{bash}, \texttt{Basic}, \texttt{C\#}, \texttt{C++}, \texttt{C}, \texttt{Cobol}, \texttt{Delphi}, \texttt{Fortran}, \texttt{Gnuplot}, \texttt{HTML}, \texttt{Java}, \texttt{Lisp}, \texttt{Logo}, \texttt{Lua}, \texttt{Make}, \texttt{Mathematica}, \texttt{Matlab}, \texttt{Metapost}, \texttt{Modula-2}, \texttt{Objective C}, \texttt{Octave}, \texttt{Pascal}, \texttt{Perl}, \texttt{PHP}, \texttt{Prolog}, \texttt{Python}, \texttt{R}, \texttt{Ruby}, \texttt{SAS}, \texttt{Scilab}, \texttt{sh}, \texttt{SQL}, \texttt{TeX}, \texttt{XML}, entre vários outros.
Se a linguagem pretendida não estiver na lista, poderá definir novos idiomas.
 
Para o seu uso, costuma efetuar configurações iniciais no \texttt{preamble} com o comando \verb+\lstset+\index{lstset@\verb+\lstset+} e no ambiente de \texttt{lstlisting}, efetuar mais alguns ajustes se necessário.
Veja o Exemplo~\ref{ex15:listings}

% Para listar o proprio lstlisting
\lstnewenvironment{lstlistinginside} {} {}

\begin{singlespacing}
\refstepcounter{lstlisting}\label{ex15:listings}
\addcontentsline{lol}{section}{\thelstlisting: ex15-listings.tex}
\begin{center}\lstlistingname~\thelstlisting: ex15-listings.tex\end{center}
% \vspace*{-1.5\baselineskip}
\vspace*{-0.5\baselineskip}
%\begin{lstlistinginside}[belowcaptionskip=1\baselineskip]
\begin{lstlistinginside}
\begin{lstlisting}[language=Pascal,caption={Alô pessoal}]
program teste;
{ Programa de teste }
begin
  writeln('Alo, pessoal!');
end.
\end{lstlisting}
\end{lstlistinginside}
\end{singlespacing}
%
\noindent
\begin{singlespacing}
\begin{framed}
\begin{center}Listing 1: Alô Pessoal\end{center}
\vspace{-1.6\baselineskip}
\noindent\hrulefill \\
\textbf{program} teste; \\ 
\emph{\{ Programa de teste \}} \\
\textbf{begin} \\
\hspace*{1em} \textbf{writeln}('Alo, pessoal!'); \\
\textbf{end}. \\
\vspace{-0.5\baselineskip}
\hrule %{\linewidth} \\
\end{framed}
\end{singlespacing}

%
%%\let\oricaption\caption
%\begin{singlespacing}
%\begin{center}
%\refstepcounter{lstlisting}
%\addcontentsline{lol}{section}{\thelstlisting: ex15-listings.tex}
%\lstlistingname~\thelstlisting: ex15-listings.tex\label{ex15:listings}
%\end{center}
%\vspace*{-1.2em}
%\noindent\hrulefill \\
%\vspace*{-0.2em}
%\colorbox{yellow!50}{teste}
%%\vspace*{-2.5em}
%\colorlet{shadecolor}{yellow!50}
%\begin{shaded}
%\begin{verbatim}
%\begin{lstlisting}[language=Pascal,caption={Alô pessoal}]
%program teste;
%{ Programa de teste }
%begin
%  writeln('Alo, pessoal!');
%end.
%\end{lstlisting}
%\end{verbatim}
%\end{shaded}
%\vspace{-1.2em}
%\noindent\hrulefill\end{singlespacing}
%
%\noindent
%
%\let\addcontentsline\emptyaddcontentsline
%\setcounter{orilstlisting}{\value{lstlisting}}
%\setcounter{lstlisting}{0}
%%\setcounter{lstorichapter}{\value{chapter}}
%\setcounter{lstorichapter}{\value{chapter}}
%\setcounter{chapter}{1}
%\let\currentlstlistingname\lstlistingname
%\let\lstlistingname\orilstlistingname
%% \fbox com lstlisting esta dando erro
%% usando framed
%\begin{framed}
%%\begin{minipage}{\linewidth-\parindent}%\hspace*{-\parindent}
%\begin{lstlisting}[language=Pascal,caption={Alô pessoal},style=lstexample]
%program teste;
%{ Programa de teste }
%begin
%  writeln('Alo, pessoal!');
%end.
%\end{lstlisting}
%%\end{minipage}%
%\end{framed}
%\setcounter{chapter}{\value{lstorichapter}}
%\setcounter{lstlisting}{\value{orilstlisting}}
%\let\lstlistingname\currentlstlistingname
%\let\addcontentsline\oriaddcontentsline

A formatação de saída depende da configuração do \texttt{lstlisting}. 
A saída do Exemplo~\ref{ex15:listings} foi obtida pela configuração
\begin{singlespacing}
\begin{lstlisting}
\lstset{
  numbers=none,
  breaklines=true,
  breakautoindent=true,
  columns=fullflexible, % para poder copiar código do PDF
  frame=lines, %none, single, line
}
\end{lstlisting}
\end{singlespacing}

Note o uso de \texttt{columns=fullflexible} para que o código fonte copiado do PDF não fique bagunçado, o que ajuda o leitor que cópia o código do documento para editor de texto.

Para o arquivo externo, podemos usar o comando \verb+\lstinputlisting+\index{lstinputlisting@\verb+\lstinputlisting+} que carrega o código diretamente do arquivo fonte.
%\lstinputlisting{test-code}

Note que \texttt{listings} não tem suporte nativo a \texttt{utf8}.
Portanto, quando usa os caracteres acentuadas, deverá tomar cuidados.
Para o \verb+\lstinputlisting+ funcione no \texttt{utf8}, basta carregar o pacote \texttt{listingsutf8}\index{listingsutf8}, mas para o código colocado diretamente no ambiente \texttt{lstlisting}, deverá criar tabela de caracteres unicode correspondente.
Para não ter tal trabalho, poderá executar no \XeLaTeX{} ou Lua\LaTeX{} em que tem total suporte a unicode.

Quando queremos colocar o trecho do código de \LaTeX{} juntamente com a sua saída, o pacote \texttt{showexpl}\index{showexpl} é prático. 
Ele usa o pacote \texttt{listings} para listar o código fonte.
A configuração inicial é feito com \verb+\lstset{explpreset={...}}+ no \texttt{preamble}. Os parâmetros em \texttt{explpreset} serão aplicados somente no ambiente \texttt{LTXexample}, enquanto que os parâmetros fora dele são aplicados tanto para \texttt{lstlisting} como \texttt{LTXexample}.
  
O ambiente \texttt{LTXexample}\index{LTXexample} aceita os parâmetros do \texttt{lstlisting} e também o \texttt{pos} e \texttt{preset} que especificam a posição onde colocar a saída, e a configuração inicial antes de executar o código.
Veja o Exemplo~\ref{ex15:showexpl}.

\begin{singlespacing}
%\begin{lstlisting}[caption={ex11-10, parte b}, label=ex:11-10:b]
\begin{lstlisting}[caption={ex15-showexpl.tex}, label=ex15:showexpl]
\begin{LTXexample}[pos=b,caption={\LaTeX}]
% comentario
Teorema de Pitágoras afirma que
\[ a^2 = b^2+c^2 \]
\end{ltxlisting}
\end{lstlisting}
\end{singlespacing}
%
%
%\let\mathbb\amsmathbb
%\let\addcontentsline\emptyaddcontentsline
\setcounter{orilstlisting}{\value{lstlisting}}
\setcounter{lstlisting}{0}
%%\setcounter{lstorichapter}{\value{chapter}}
%\setcounter{lstorichapter}{\value{chapter}}
%\setcounter{chapter}{1}
\let\currentlstlistingname\lstlistingname
\let\lstlistingname\orilstlistingname
% \fbox com lstlisting esta dando erro
% usando framed
\begin{ltxlistingout}
%\begin{minipage}{\linewidth-\parindent}%\hspace*{-\parindent}
\begin{LTXexample}[pos=r,caption={\LaTeX},style=lstexample,belowcaptionskip=-\z@]
% comentario
Teorema de Pitágoras afirma que
\[ a^2 = b^2+c^2 \]
\end{LTXexample}
%%\end{minipage}%
\end{ltxlistingout}
%\setcounter{chapter}{\value{lstorichapter}}
\setcounter{lstlisting}{\value{orilstlisting}}
\let\lstlistingname\currentlstlistingname
%\let\addcontentsline\oriaddcontentsline
%\let\mathbb\orimathbb

O valor de ``\texttt{pos}'' que determina onde ficará a saída do resultado do código em relação ao do código fonte, podem ser
``t'', ``b'', ``l'', ``r'' ,``o'' ou ``i'', que são 
abreviaturas de ``top'' (encima), ``bottom'' (embaixo), ``left'' (esquerda), ``right'' (direita), ``outer'' (lado de fora) e ``inner'' (lado de dentro).

Na saída do Exemplo~\ref{ex15:showexpl}, pode notar a falta de espaço vertical após o título, o que faz título esconder atrás do código e da sua saída. 
Teoricamente, o parâmetro opcional \texttt{belowcaptionskip} do ambiente \texttt{lstlisting} deveria ajustar tal espaçamento, mas devido a forma como \texttt{showexpl} foi implementado, isto não acontece por não chamar o \verb+\lst@Init+ encarregado disso.
Para corrigir, a atualização de valor pode ser inserido dentro do
\verb+\SX@KillAboveCaptionskip+ que é chamado antes de criar o titulo no \texttt{LTXexample}.
Para isso, coloque o código

\begin{singlespacing}
\begin{lstlisting}
\makeatletter
\renewcommand\SX@KillAboveCaptionskip{%
  \ifx\lst@caption\@empty\else
    \lst@IfSubstring t\lst@captionpos
      {\vskip-\abovecaptionskip}{}%
  \fi
  \ifx\lst@belowcaption\undefined \else %adicionado
    \belowcaptionskip=\lst@belowcaption\relax % adicionado
  \fi %adicionado
}
\makeatother
\end{lstlisting}
\end{singlespacing}
no \texttt{preamble} do documento.
Note que a configuração inicial do ambiente \texttt{LTXexample} não será alterado, mas agora podemos ajustar o espaçamento entre o título e o corpo com o parâmetro \texttt{belowcaptionskip} como em \texttt{lstlisting}.

%Uma forma de contornar isso e usar um \texttt{strut} (caixa de largura nula usado para reservar altura) no nome do título, reservando altura adicional. 
%O parâmetro \texttt{belowcaptionskip} será armazenado em \verb+\lst@belowcaption+.
%Então crie um comando tal como \verb+\lstcaptionstrut+
%pelo comando
%\begin{lstlisting}
%\newcommand{\lstcaptionstrut}{\rule[-\lst@belowcaption]{0pt}{0pt}}
%\end{lstlisting} 
%adicionado no preamble do documento.
%Depois, no final do parâmetro \texttt{caption}, adicione este comando.

Note que o \verb+\usepackage+ e similar estão desabilitados dentro do ambiente \texttt{LTXexample} e se o código precisar, deverá estar carregado no \texttt{preamble} do documento principal.
Agora outro exemplo, supondo que \texttt{amssymb} está carregado no documento principal (para usar \verb+\mathbb+).
Veja o Exemplo~\ref{ex15:showexpl:completo}.

\begin{singlespacing}
%\begin{lstlisting}[caption={ex11-10, parte c}, label=ex:11-10:c]
\begin{lstlisting}[caption={ex15-showexpl-completo}, label=ex15:showexpl:completo]
\begin{LTXexample}[pos=b,caption={\LaTeX{} completo},belowcaptionskip=\baselineskip]
\documentclass{article}
\usepackage[T1]{fontenc}
\usepackage[utf8]{inputenc}
\usepackage[brazil]{babel}
\usepackage{amssymb,amsmath}
\newcommand{\sen}{\mathrm{sen}}
\begin{document}
% comentario
\[ \forall \theta \in \mathbb{R}, \cos^2\theta + \sen^2\theta = 1 \]
\end{document}
\end{LTXexample}
\end{lstlisting}
\end{singlespacing}
%
%
%\let\mathbb\amsmathbb
%\let\addcontentsline\emptyaddcontentsline
\setcounter{orilstlisting}{\value{lstlisting}}
\setcounter{lstlisting}{1}
%\setcounter{lstorichapter}{\value{chapter}}
%\setcounter{lstorichapter}{\value{chapter}}
%\setcounter{chapter}{1}
\let\currentlstlistingname\lstlistingname
\let\lstlistingname\orilstlistingname
% \fbox com lstlisting esta dando erro
% usando framed
\begin{ltxlistingout}
%\begin{minipage}{\linewidth-\parindent}%\hspace*{-\parindent}
\begin{LTXexample}[pos=b,caption={\LaTeX{} completo},belowcaptionskip=\baselineskip,style=lstexample]
\documentclass{article}
\usepackage[T1]{fontenc}
\usepackage[utf8]{inputenc}
\usepackage[brazil]{babel}
\usepackage{amssymb,amsmath}
\newcommand{\sen}{\mathrm{sen}}
\begin{document}
% comentario
\[ \forall \theta \in \mathbb{R}, \cos^2\theta + \sen^2\theta = 1 \]
\end{document}
\end{LTXexample}
%\end{minipage}%
\end{ltxlistingout}
%\setcounter{chapter}{\value{lstorichapter}}
\setcounter{lstlisting}{\value{orilstlisting}}
\let\lstlistingname\currentlstlistingname
%\let\addcontentsline\oriaddcontentsline
%\let\mathbb\orimathbb


%
%fancybox, OK
%framed, OK
%lineno, OK
%eso-pic, OK
%
%listings, OK
%showexl, OK

\section{Mais sobre referências bibliográficas}

Na citação original do \LaTeX, usa-se o mesmo padrão tanto para caso de fazer parte do texto (citação textual) ou não fazer parte do texto (citação dentro de parenteses). 
Para fazer esta distinção e controlar melhor a forma de citação, foi desenvolvido o pacote chamado \texttt{natbib}\index{natbib}.

O pacote \texttt{natbib} possui comandos \verb+\citet+\index{natbib!\verb+\citet+}, usado para citações textuais e \verb+citep+\index{natbib!\verb+\citep+} para citações entre parenteses (ele já coloca parenteses automaticamente).

Além disso, o \texttt{natbib} permite configurar a formatação de citações.
Veja o Exemplo~\ref{ex15:natbib} com Exemplo~\ref{ex11:bibtex:bib}.

\begin{singlespacing}
%\begin{lstlisting}[pos=b,caption={ex12-08},label=ex:12-08]
\begin{lstlisting}[caption={ex15-natbib.tex},label=ex15:natbib]
\documentclass{article}
\usepackage[utf8]{inputenc}
\usepackage[brazil]{babel}
\usepackage{natbib}
\begin{document}
Para que a citação faça parte do texto (estilo textual), usa-se o comando \verb+\citet+ como em ``Por exemplo, \citet{IndianTUG:2000, Goossens:2004} explicam como usar o BibTeX''.

Para que ele não faça parte do texto, usa-se o \verb+\citep+ Como em ``\ldots recursos avançados de BibTeX \citep[Cap.~13]{Goossens:2004}''.
\bibliographystyle{apalike} % escolha um estilo
\bibliography{ex11-bibtex} % arquivo de referências
\end{document}
\end{lstlisting}
\end{singlespacing}

%\begin{framed}
%\begin{singlespacing}
\begin{ltxlistingout}
%\begin{minipage}{\linewidth-\parindent}%\hspace*{-\parindent}
Para que a citação faça parte do texto (estilo textual), usa-se o comando \verb+\citet+ como em ``Por exemplo, Tutorial Team (2000); Goossens and Mittelbach (2004) explicam como usar o BibTeX''.

Para que ele não faça parte do texto, usa-se o \verb+\citep+ Como em ``\ldots recursos avançados de BibTeX (Goossens and Mittelbach, 2004, Cap. 13)''.
%\end{minipage}%
\section*{Referências}
\noindent Goossens, M., Mittelbach, F. (2004).
{\em The {\LaTeX{}} Companion (second edition)}.\\
\hspace*{1em} Adilson--Wesley, Reading, MA.

\noindent Tutorial Team (2000).
{\em Online Tutorials on {\LaTeX{}}}.
Indian {\TeX{}} User Group. \\
\hspace*{1em} {\url{http://www.tug.org/tutorials/tugindia/}}.
\end{ltxlistingout}
%\end{singlespacing}
%\end{framed}

O \verb+\cite+ tradicional também estará disponível, funcionando como \verb+\citet+ no caso do estilo \texttt{autor-ano} e como \verb+\citep+ no caso do estilo numérico.

Usando a opção do pacote ou comandos apropriados, podermos configurar o estilo de citações.

Note que nos estilos antigos, os campos tais como ``\texttt{url}'' e ``\texttt{doi}''\index{bibtex!doi} não serão usados (logo, não aparecem nas referências).
Para que \texttt{URL} apareça na referência desses estilos antigos, colocavam no campo ``\texttt{note}''.

Para estilos mais modernos como do \texttt{natbib}, o \texttt{URL} deve ser colocado no campo ``\texttt{url}''\index{bibtex!url} e a data do último acesso em ``\texttt{urldate}''\index{bibtex!urldate}.
No estilo natigo em que \texttt{urldate} é ignorado, coloque o último acesso em ``\texttt{note}''.
% pode ser usado para outra finalidade tal como colocar a data de último acesso.

Estilo padrão do \texttt{natbib} são \texttt{plainnat}\index{natbib!plainnat}, \texttt{abbrvnat}\index{natbib!abbrvnat}, \texttt{unsrtnat}\index{natbib!unsrtnat}.
Também suporta vários \texttt{bst} de outros pacotes
do estilo \texttt{autor-ano} tais como \texttt{apalike}\index{natbib!apalike}, \texttt{newapa}\index{natbib!newapa}, (\texttt{chicago}\index{natbib!chicago}, \texttt{named}\index{natbib!named}, etc do \texttt{chicago}), 
(\texttt{agsm}\index{natbib!agsm}, \texttt{dcu}\index{natbib!dcu}, \texttt{kluwer}\index{natbib!kluwer}, etc do \texttt{harvard}), 
(\texttt{astron}\index{natbib!astron}, \texttt{apa}\index{natbib!apa}, \texttt{humanbio}\index{natbib!humanbio}, etc do \texttt{astron}),
(\texttt{authordate<n>}\index{natbib!authordate<n>} e \texttt{aaai-named}\index{natbib!aaai-named} do \texttt{authordate<n>}).

Note que \texttt{bst} não compatíveis com \texttt{autor-ano}, como o estilo padrão do \BibTeX{} será considerado como numérico.

Para criar seu próprio arquivo de estilo \texttt{bst}, comece gerando um modelo com \texttt{makebst}\index{makebst} do pacote \texttt{custom-bib}\index{custom-bib}.

%%%%%%%%%% biblatex

O Bib\LaTeX{}\index{biblatex} suporta o arquivo \texttt{bib} em \texttt{utf8}, permitindo que tenha citações dos livros em caracteres fora do \texttt{latin} e também suporta a localização da referência bibliográfica.

Outro ponto positivo para desenvolvedores é ter maior facilidade de elaborar arquivo de estilos.

O Bib\LaTeX{} não é compatível com maioria dos pacotes específicos para \BibTeX.
Também terá problemas com o pacote \texttt{titlesec} quando usar as opções \texttt{refsection}, \texttt{refsegment} ou \texttt{citereset} no Bib\LaTeX.

Note que, caso não tenha estilo de referência bibliográfica desejado, o truque de copiar o conteúdo do arquivo \texttt{bbl} (extensão \texttt{bbl}) e ajustar, não funciona no Bib\LaTeX, pois ele não usa o ambiente \texttt{thebibliography}. 
Isto vale mesmo que use \BibTeX{} como \texttt{backend}.
Como nem toda revista dispõe de estilos para Bib\LaTeX{} (mesmo tendo estilo para \BibTeX), fique atento nesta parte para decidir se vai usar o \BibTeX{} direto, ou usar o Bib\LaTeX.

Quando carrega o pacote \texttt{biblatex}, efetua algumas configurações.

\begin{singlespacing}
\begin{lstlisting}
\usepackage[
  backend=biber,
  bibencoding=utf8,
%  natbib=true, % natbib compatible mode (citep, citet, etc)
  style=authoryear,
]{biblatex}
\end{lstlisting}
\end{singlespacing}

Carrega o \texttt{biblatex} configurado para usar \texttt{biber}\index{biber} como processador de referências bibliográficas. 
Se usar o \texttt{bibtex} em vez do \texttt{biber}, perderá os recursos estendidos tais como suporte a \texttt{utf8} e localização da referência bibliográfica.
No código anterior, também foi indicado que o arquivo \texttt{bib} está em \texttt{utf8}, podendo conter caracteres unicode (fora do \texttt{latin}, por exemplo).
A opção \texttt{natbib} está comentado, mas se ativar, terá disponibilidade para usar comandos \verb+\citet+ e \verb+\citep+ similar a do \texttt{natbib}.
Finalmente, o estilo escolhido é \texttt{authoryear}.

Para indicar os arquivos \texttt{bib}, usa-se o comando \verb+\addbibresource{<arquivo bib>}+\index{addbibresource@\verb+\addbibresource+} que costuma estar no \texttt{preamble}, onde \texttt{<arquivo bib>} é nome do arquivo \texttt{bib} que deve incluir necessariamente a extensão (como o \verb+.bib+). 
Caso quer indicar vários arquivos \texttt{bib}, deverá colocar um \verb+\addbibresource+ para cada arquivo \texttt{bib}.

Após efetuar citações normalmente no documento, coloca-se o comando
\verb+\printbibliography+\index{printbibliography@\verb+\printbibliography+}
no lugar onde quer que apareça a referência bibliográfica.
Veja o Exemplo~\ref{ex15:biblatex} com Exemplo~\ref{ex11:bibtex:bib}..

\begin{singlespacing}
%\begin{lstlisting}[pos=b,caption={ex12-08},label=ex:12-08]
\begin{lstlisting}[caption={ex15-biblatex.tex},label=ex15:biblatex]
\documentclass{article}
\usepackage[utf8]{inputenc}
\usepackage[brazil]{babel}
\usepackage[
  backend=biber,
  bibencoding=utf8,
%  natbib=true, % natbib compatible mode (citep, citet, etc)
  style=authoryear, % estilo da citação no texto
% bibstyle=authortitle, % estilo da listagem pode ser diferente da citação no texto
]{biblatex}
\addbibresource{ex11-bibtex.bib} % não esquecer o .bib
\begin{document}
Para que a citação faça parte do texto (estilo textual), usa-se o comando \verb+\textcite+ como em ``Por exemplo, \textcite{IndianTUG:2000, Goossens:2004} explicam como usar o BibTeX''.

Para que ele não faça parte do texto, usa-se o \verb+\parencite+ Como em ``\ldots recursos avançados de BibTeX \parencite[Cap.~13]{Goossens:2004}''.
\printbibliography
\end{document}
\end{lstlisting}
\end{singlespacing}

%\begin{framed}
%\begin{singlespacing}
\begin{ltxlistingout}
%\begin{minipage}{\linewidth-\parindent}%\hspace*{-\parindent}
Para que a citação faça parte do texto (estilo textual), usa-se o comando \verb+\textcite+ como em ``Por exemplo, Tutorial Team (2000) e Goossens e Mittelbach (2004) explicam como usar o BibTeX''.

Para que ele não faça parte do texto, usa-se o \verb+\parencite+ Como em ``\ldots recursos avançados de BibTeX (Goossens and Mittelbach 2004, Cap. 13)''.
%\end{minipage}%
\section*{Referências}
\noindent Goossens, Michel e Frank Mittelbach (2004).
{\em The {\LaTeX{}} Companion (second edition)}.\\
\hspace*{1em} Reading, MA: Adilson--Wesley, 

\noindent Tutorial Team (2000).
{\em Online Tutorials on {\LaTeX{}}}.
{\url{http://www.tug.org/tutorials/tugindia/}}.
\hspace*{1em} Indian {\TeX{}} User Group. \\
\end{ltxlistingout}
%\end{singlespacing}
%\end{framed}

Quando usa o Bib\LaTeX, a citação textual (que faz parte do texto) será especificado pelo comando \verb+\textcite+\index{biblatex!textcite@\verb+\textcite+} que corresponde ao \verb+\citet+ do \texttt{natbib}.
O comando \verb+\parencite+\index{biblatex!parencite@\verb+\parencite+} produz citação que não faz parte do texto (corresponde ao \verb+\citep+ do \texttt{natbib}).
O comando \verb+\cite+ dependerá do estilo escolhido.

Se quer que a referência bibliográfica conste no sumário, coloque
\begin{singlespacing}
\begin{lstlisting}
\phantomsection % se estiver usando o hyperref
\printbibliography[heading=bibintoc]
\end{lstlisting}
\end{singlespacing}
onde \verb+\phantomsection+\index{phantomsection@\verb+\phantomsection+} é necessário somente quando usa o pacote \texttt{hyperref} para que o link do sumário para a referência bibliográfica aponte para a página correta.

O estilo padrão do Bib\LaTeX{} são \texttt{numeric}\index{biblatex!numeric}, \texttt{alphabetic}\index{biblatex!alphabetic}, \texttt{authoryear}\index{biblatex!authoryear}, \texttt{authortitle}\index{biblatex!authortitle}, \texttt{verbose}\index{biblatex!verbose}, \texttt{reading}\index{biblatex!reading} e \texttt{draft}\index{biblatex!draft}, mas existem muitos outros disponíveis em \url{http://mirror.ctan.org/macros/latex/exptl/biblatex-contrib}.

No estilo de Bib\LaTeX, foi adicionado o campo \texttt{urldate} para indicar a data de consulta do documento eletrônico, além do campo \texttt{url} para indicar a localização de documentos eletrônicos.

Para quem pretende criar o estilo novo, note que no Bib\LaTeX{} deve implementar tanto o estilo de referências (como será impresso na referências) com  o estilo de citações (como aparece dentro do texto).

%natbib,
%biblatex
\section{Ênfase modo antigo e cancelamento}
Além da ênfase em itálico, também poderá usar o sublinhado como era feito na maquina de escrever.

O pacote \texttt{ulem}\index{ulem} implementa várias formas de ``sublinhar'' o texto.
Para que os comandos \verb+emph+ e \verb+\em+, assim como o ambiente \texttt{em} não sejam trocados pelo comando de sublinhar texto, use a opção \texttt{normalem}.
Assim, coloque \verb+\usepackage[normalem]{ulem}+ no \texttt{preamble} para poder usar diversos efeitos de ``sublinhar'' o texto.

Veja o Exemplo~\ref{ex15:ulem}.

\begin{ltxlisting}[caption={ex15-ulem.tex},label=ex15:ulem]
Sublinhando: \uline{importante} \\
Sublinhando com linha dupla: \uuline{urgente} \\
Com onda: \uwave{barco} \\
Cancelando o texto: \sout{errado} \\
Removendo: \xout{removido} \\
Linha tracejada: \dashuline{tracejada} \\
Linha pontilhada: \dotuline{pontilhada}
\end{ltxlisting}

Para cancelar as equações, use o pacote \texttt{cancel}\index{cancel}.
Veja o Exemplo~\ref{ex15:cancel} para ver o uso.

\begin{ltxlisting}[caption={ex15-cancel.tex},label=ex15:cancel]
\renewcommand{\CancelColor}{\color{red}} %cor de cancelamento pode ser alterado
\[x+\cancel{2y} = z \]
\[x+\bcancel{2y} = z \]
\[x+\xcancel{2y} = z \]
\[x+\cancelto{0}{2y} = z \]
\end{ltxlisting}

\clearpage\thispagestyle{empty}\cleardoublepage
\chapter{Gráfico e Diagramas}\label{chap:xcolor:tikz}
Os recursos do ambiente \texttt{picture} do \LaTeX{} é bastante limitado, mesmo que use o pacote \texttt{pict2e}.
Assim, costumamos usar os pacotes fora do \texttt{base} e \texttt{required} como o \texttt{tikz}.
Neste capítulo, vamos estudar alguns pacotes relacionados aos gráficos, que não são da \texttt{base} e \texttt{required}.

\section{Misturando cores}\label{sec:xcolor}
O pacote de cores mais recomendados é o \texttt{xcolor}\index{xcolor} que estende a funcionalidade do pacote \texttt{color}.
Uma das extensões mais importantes do \texttt{xcolor} é a capacidade de especificar a quantidade de cores e permitir misturar cores.

Para misturar cores\index{misturar cores}, use o ``!'' seguido de percentual.
Por exemplo, \verb|\color{blue!30}|
 é 30\% de azul. %: \colorbox{blue!30}{\hspace*{1cm}}

No caso de \verb|\color{red!30!yellow}|
 é 30\% de vermelho e restante (70\%) de amarelo.
 % : \colorbox{red!30!yellow}{\hspace*{1cm}}

Em \verb|\color{blue!20!black!30!green}|
 é 20\% de azul, 50\% de preto do restante (do 80\% que é 40\%) e restante (40\%) de verde. %: \colorbox{blue!20!black!50!green}{\hspace*{1cm}}

Para definir nova cor a partir do existente, use o comando
\verb|\colorlet|\index{colorlet@\verb+\colorlet+}.

Por exemplo, % Ex.: \colorlet{mygreen}{green!80!yellow}
   \verb|\colorlet{mygreen}{green!80!yellow}| define \texttt{mygreen} como sendo a mistura de 80\% de verde e 10\% de amarelo. %, produzindo \colorbox{mygreen}{\hspace*{1cm}}
Veja o Exemplo~\ref{ex16:xcolor}.

%\begin{singlespacing}
%\begin{LTXexample}[pos=b,caption={ex11-07.tex},label=ex:11-07]
\begin{ltxlisting}[caption={ex16-xcolor.tex},label=ex16:xcolor]
\colorlet{mygreen}{green!80!yellow} % defindo cor a partir do existente

\colorbox{blue!30}{\hspace*{1cm}}

\colorbox{red!30!yellow}{\hspace*{1cm}}

\colorbox{blue!20!black!50!green}{\hspace*{1cm}}

\colorbox{mygreen}{\hspace*{1cm}}
\end{ltxlisting}
%\end{LTXexample}
%\end{singlespacing}

A seguir, cores básicos do pacote \texttt{xcolor}.


\begin{table}[htbp]
\caption{cores aceitos em todos \protect\texttt{drivers} no pacote xcolor}
 \ \\
 \begin{tabular}{|c|c|c|c|c|c|c|c|}
  \hline
   black & \colorbox{black}{\hspace*{1cm}} 
 & blue & \colorbox{blue}{\hspace*{1cm}}
 & brown & \colorbox{brown}{\hspace*{1cm}}
 & cyan &  \colorbox{cyan}{\hspace*{1cm}} \\ \hline
   darkgray &  \colorbox{darkgray}{\hspace*{1cm}}
 & gray &  \colorbox{gray}{\hspace*{1cm}}
 & green & \colorbox{green}{\hspace*{1cm}}
 & lightgray &  \colorbox{lightgray}{\hspace*{1cm}} \\ \hline
  lime & \colorbox{lime}{\hspace*{1cm}}
 & magenta & \colorbox{magenta}{\hspace*{1cm}}
 & olive & \colorbox{olive}{\hspace*{1cm}}
 & orange &  \colorbox{orange}{\hspace*{1cm}}  \\ \hline
   pink &  \colorbox{pink}{\hspace*{1cm}}
 & purple &  \colorbox{purple}{\hspace*{1cm}}
 & red & \colorbox{red}{\hspace*{1cm}}
 & teal & \colorbox{teal}{\hspace*{1cm}}  \\ \hline
   violet & \colorbox{violet}{\hspace*{1cm}}
 & white &  \colorbox{white}{\hspace*{1cm}}
 & yellow & \colorbox{yellow}{\hspace*{1cm}}  \\ \hline
 \end{tabular}
\end{table}

Dependendo da especificação do \texttt{driver}, terão muito mais cores, mas é mais prático trabalhar com cores básicos e suas misturas do que lidar com grande quantidade de nomes para cores.

\section{Criando ilustrações gráficas}

Para criar ilustrações gráficas, um dos mais indicados é o pacote \texttt{tikz}\index{tikz}\cite{book:tikz:tantau}.

O pacote \texttt{tikz} possui muitas extensões que podem ser carregados pelo comando \verb+\usetikzlibrary+\index{usetikzlibrary@\verb+\usetikzlibrary+}.

Nos exemplos a seguir, será assumido que tem o código do Exemplo~\ref{ex16:tikz:preamble} no \texttt{preamble}.

\begin{singlespacing}
%\begin{lstlisting}[caption={ex12-02.tex}, label=ex:12-02]
\begin{lstlisting}[caption={ex16-tikz-preamble.tex}, label=ex16:tikz:preamble]
\usepackage{tikz} % pacote gráfico
\usetikzlibrary{babel} % para compatiblidade com o pacote babel, requerido por agumas bibliotecas como o cd.

\usetikzlibrary{calc} % calc eh para efetuar calculos matematicos ou expressoes em coordenadas

\usetikzlibrary{through} % circulo passando por ponto, por exemplo.
 
\usetikzlibrary{patterns} % preenchimentos

\usetikzlibrary{intersections} % interseccao entre caminhos

\usetikzlibrary{matrix} % matriz no tikz

\usetikzlibrary{cd} % diagrama comutativa
\end{lstlisting}
\end{singlespacing}

Os comandos gráficos do \texttt{tikz} sempre termina com ponto e vírgula. 
Ao compilar o código sem ter terminado com ponto e virgula no trecho de \texttt{tikz}, pode travar o \LaTeX{}. 
No caso de estar usando o \LyX{} que não salva
antes de compilar, é importante que salve manualmente antes de compilar.
Note que o ``instant preview inset'' do \LyX{} não trava.
Veja o Exemplo~\ref{ex16:tikz:basico} para começar.
%\begin{singlespacing}
%\begin{LTXexample}[pos=b,caption={ex12-02.tex, parte b}, label=ex:12-02:b]
\begin{ltxlisting}[caption={ex16-tikz:basico.tex}, label=ex16:tikz:basico]
Poderá desenhar no modo \texttt{inline} como em \tikz \draw[fill] circle (2pt);
ou \tikz{\draw (0,0) -- (1.5,0);}   e também como figura independente

\begin{tikzpicture}
\draw[rounded corners=8pt] (0,0) -- (0,2) -- (2,2) -- (2,0) -- cycle; 
\end{tikzpicture}
\end{ltxlisting}
%\end{LTXexample}
%\end{singlespacing}

O comando \verb+\draw+\index{draw@\verb+\draw+} desenha o elemento. 
A forma como vai desenhar pode ser configurado com o parâmetro opcional. 
``\verb+--+'' indica que vai ligar os pontos a esquerda com da direita com uma linha.
Cores podem ser passado no parâmetro opcional do \verb+\draw+.
Veja o Exemplo~\ref{ex16:tikz:draw}.

%\begin{singlespacing}
%\begin{LTXexample}[pos=b,caption={ex12-02.tex, parte c}, label=ex:12-02:c]
\begin{ltxlisting}[caption={ex16-tikz-draw.tex}, label=ex16:tikz:draw]
\begin{center}
\begin{tikzpicture}
\draw (-1,-0.5) -- (1,0.5); % segmentos
\draw (0,1) -- (1,1) -- (1,2) -- (0,2); % Linhas poligonais
\draw (2,0) -- (3,0) -- (3,1) -- (2,1) -- cycle; % Linhas poligonais fechadas
\draw[rounded corners=8pt] (-1,-1) -- (1,-1) -- (1,-2) -- (-1,-2); % Quinas arredondadas
\draw[red] (2,-1) |- (3,-2);% Conectando com linha vertical-horizontal
\draw[blue] (2,-1) -| (3,-2); % Conectando com linha horizontal-vertical
\end{tikzpicture}
\end{center} 
\end{ltxlisting}
%\end{LTXexample}
%\end{singlespacing}

Quando tem mais de um parâmetro opcional, coloque separado pela vírgula.
Veja Exemplo~\ref{ex16:tikz:parametro}.

%\begin{singlespacing}
%\begin{LTXexample}[pos=b,caption={ex12-02.tex, parte d}, label=ex:12-02:d]
\begin{ltxlisting}[caption={ex16-tikz-parametro.tex}, label=ex16:tikz:parametro]
\begin{center}
\begin{tikzpicture}
% Epessuras: very thin, thin, thick, very thick
\draw[thick] (0,0) -- (1,1);
% Estilo: dashed, dotted
\draw[dotted] (0,0) -- (1,-1);
% Setas ->, <-, <->, etc
\draw[->] (0,0) -- (1,0);
% Combinadas
\draw[color=blue,thick,->] (0,0) -- (-1,0);
\end{tikzpicture}
\end{center}
\end{ltxlisting}
%\end{LTXexample}
%\end{singlespacing}

Podemos definir e usar coordenadas, assim como efetuar alguns cálculos com coordenadas.
Veja o Exemplo~\ref{ex16:tikz:coordenadas}.

%\begin{singlespacing}
%\begin{LTXexample}[pos=b,caption={ex12-02.tex, parte e}, label=ex:12-02:e]
\begin{ltxlisting}[caption={ex16-tikz-coordenadas.tex}, label=ex16:tikz:coordenadas]
\begin{center}
\begin{tikzpicture}
  \coordinate (A) at (0,0);
  \coordinate (B) at (1,0);
  \coordinate (C) at (1,1);

  \draw (A) -- (B) -- (C) -- cycle; 

  % interpolacao linear das coordenadas
  % $(A)!t!(C)$ sera interpretado como sendo (1-t)*(A)+t*(C)
  \draw[dotted] (A) -- ($(B)!0.5!(C)$);

  % Interpolação linear com rotação
  % rotacionado por angulo de 15 em torno de (B)
  \coordinate (D) at ($(B)!1!15:(C)$);

  \draw[dashed] (A) -- (D); 

  % projeção ortogonal
  % $(A)!(B)!(C)$ sera pé do perpendicular abaixado de (B) 
  % sobre a reta determinada pelos pontos (A) e (C)
  \coordinate (H) at ($(A)!(B)!(C)$);

  \draw (A) -- (B) -- (C) -- cycle; 

  \draw[blue] (B) -- (H);

  % Coordenada com rótulos (ja desenha o rotulo quando define)
  \coordinate[label=left:$X$] (X) at (3,0);
  \coordinate[label=right:$Y$] (Y) at (4,1);
  \draw (X) -- (Y);
  
  % Coordenada polar.
  % As coordenada polar é dado por (angulo:raio).
  % O ângulo é em graus.
  \coordinate (u) at (45:1); % (angulo:raio)
  \coordinate (v) at (90:2);

  % combinação linear das coordenadas
  % Note que devem ficar delimitados pelo $
  \draw[->] (A) -- ($1.5*(u)+0.5*(v)$); 
\end{tikzpicture}
\end{center} 
\end{ltxlisting}
%\end{LTXexample}
%\end{singlespacing}

Também podemos obter intersecções. 
Para pintar dentro, podemos usar o comando \verb+\fill+\index{fill@\verb+\fill+} (ou \verb+\draw+ com opção)
Veja o Exemplo~\ref{ex16:tikz:intersection}.

%\begin{singlespacing}
%\begin{LTXexample}[pos=b,caption={ex12-02.tex, parte f}, label=ex:12-02:f]
\begin{ltxlisting}[caption={ex16-tikz-intersection.tex}, label=ex16:tikz:intersection]
\begin{center}
\begin{tikzpicture}
  % desenhando e definindo coordenadas
  \draw (0,0) coordinate (A) -- (2,3) coordinate (B)
        (0,2) coordinate (C) -- (3,1) coordinate (D);
  \fill[red] (intersection of A--B and C--D) circle (2pt);
  
 % Outra forma de obter intersecção.
 \draw (4,0) coordinate (E) -- (6,3) coordinate (F)
        (4,2) coordinate (G) -- (7,1) coordinate (H);
  \fill[blue] (intersection cs:
       first line={(E) -- (F)},
       second line={(G) -- (H)}) circle (2pt);
\end{tikzpicture}
\end{center} 
\end{ltxlisting}
%\end{LTXexample}
%\end{singlespacing}

Círculos, elipses e arcos também podem ser desenhados.
Veja o Exemplo~\ref{ex16:tikz:circle}.

%\begin{singlespacing}
%\begin{LTXexample}[pos=b,caption={ex12-02.tex, parte g}, label=ex:12-02:g]
\begin{ltxlisting}[caption={ex16-tikz-circle.tex}, label=ex16:tikz:circle]
\begin{center}
\begin{tikzpicture}
  \draw[red] (-1.5,0) circle (0.5);   % circlo

  \coordinate [label=left:$A$] (A) at (0,0); 
  \coordinate [label=right:$B$] (B) at (1.25,0.25); 
  \draw (A) -- (B);
   % Circulo passando por (B), com centro em (A)
  \node [draw,circle through=(B),label=above:$c$] at (A) {};

  \coordinate (X) at (3,0);
  \def\localRadius{2}
  \def\localAngle{45}

  % fazendo circulo na ponta inicial e rotulando como $X$
  \draw (X) circle (2pt) node[anchor=south west] {$X$};

  % faz arco e os circulos na ponta final e rotula como $Y$
  \draw (X) arc (0:\localAngle:\localRadius) circle (2pt) node[anchor=south] {$Y$};

% Elipse: Atento pelo uso de "and" quando indica os raios da elipse
  \draw (5,0) ellipse (0.5 and 1.0);
\end{tikzpicture}
\end{center}
\end{ltxlisting}
%\end{LTXexample}
%\end{singlespacing}

Também podemos desenhar o gráfico das funções e curvas parametrizadas.
Veja o Exemplo~\ref{ex16:tikz:grafico}.

%\begin{singlespacing}
%\begin{LTXexample}[pos=b,caption={ex12-02.tex, parte h}, label=ex:12-02:h]
\begin{ltxlisting}[caption={ex16-tikz-grafico.tex}, label=ex16:tikz:grafico]
\begin{center}
\begin{tikzpicture}[domain=-3.2:3.2]
%\draw[dotted] (-3.2,-1.2) grid (3.2,1.2);
\draw[->] (-3.5,0) -- (6.5,0) node[right] {$x$}; 
\draw[->] (0,-5.2) -- (0,2.2) node[above] {$y$};

% O angulo padrão será em graus. Para usar em radianos, acrescente " r" no parametro da funcao trigonometrica
\draw[smooth,color=blue,domain=-pi-0.1:pi+0.1] plot (\x,{sin(\x r)}) node[above] {$f(x) =\mathrm{sen}(x)$}; 

% Curvas em coordenada polar. O angulo padrao sera em graus. A funcao deg() converte radiano para grau (é mesmo que colocar sufixo r)
\draw[smooth,color=red,domain=0:2*pi] plot ({deg(\x)}:\x) node[above] {$\rho(\theta) =\theta$}; 

% curvas parametricas.
% podera especificar o nome da variavel em vez de usar o padrao \x
\draw[smooth,color=green,domain=-1.8:1.8,variable=\t] plot ({cosh(\t)},{sinh(\t)}) node[above]{$\text{ramo de }x^2-y^2=1$};
\end{tikzpicture}
\end{center}
\end{ltxlisting} 
%\end{LTXexample}
%\end{singlespacing}

Também podemos desenhar gráficos da função tabelada. 
No Exemplo~\ref{ex16:tikz:funcao:tabelada}, está usando a função tabelada no arquivo de texto externo. 
Ele também faz o uso do laço \verb+\foreach+\index{foreach@\verb+\foreach+} para marcar valores sobre eixos.

%\begin{singlespacing}
%\begin{LTXexample}[pos=b,caption={ex12-02.tex, parte i}, label=ex:12-02:i]
\begin{ltxlisting}[caption={ex16-tikz-funcao-tabelada.tex}, label=ex16:tikz:funcao:tabelada]
\begin{center}
\begin{tikzpicture}
  % eixos
  \draw[->] (-2.5,0) -- (2.5,0) node[right] {$x$};
  \draw[->] (0,-0.5) -- (0,4.5) node[above] {$y$};

  % grid
  \draw[dotted] (-1,0) grid (1,1);
  \draw[dotted] (-2,0) -- (-2,4) -- (2,4) -- (2,0);
  
  % efetuando laco nos parametros para marcar escala
  \foreach \x in {-2,-1,1,2}   % sobre eixo x
    \node at (\x, 0) [below] {\x};
  \foreach \y in {1,4}   % sobre eixo y
    \node at (0,\y) [anchor=south west] {\y};

  % grafico tabelado no arquivo externo.
  % Note que tikz não reconhece notação científica
  % no arquivo de entrada.
  \draw[color=blue] plot[smooth] file {latex-via-exemplos-tabela.txt} node [right] {$y=x^2$};
\end{tikzpicture} 
\end{center}
\end{ltxlisting}
%\end{LTXexample}
%\end{singlespacing}

Também podemos pintar, preencher com padrão e sombrear.
Veja Exemplo~\ref{ex16:tikz:pintar}.

%\begin{singlespacing}
%\begin{LTXexample}[pos=b,caption={ex12-02.tex, parte j}, label=ex:12-02:j]
\begin{ltxlisting}[caption={ex16-tikz-pintar.tex}, label=ex16:tikz:pintar]
\begin{center}
\begin{tikzpicture}[scale=1.5] % podemos mudar a escala
% fill (pinta) e filldraw (pinta e desenha o contorno) podem ser usados para pintar. Outra forma é passar o parámetro fill para draw.

\filldraw[fill=red!25,draw=blue] (5,0) rectangle (6,1);

% pattern preenche com padrao de preenchimento.
% Parametro draw eh para desenhar o contorno
\pattern[pattern=north east lines, draw=blue] (3,0) circle (1);

% sombra eh feito pelo shade e sombra com contorno com shadedraw (igual a shade com parametro draw)
% sombra pode ser degrade e aceita o sintaxe do xcolor (cor!percentual)
% Na funcao, \x^2 nao funciona. Deve ser (\x)^2.
\shade[domain=-1:1,top color=blue!25,bottom color=red!25,draw=red] plot (\x,{(\x)^2});

% fazendo tudo com draw
% Iterando num intervalo
\foreach \x in {0,...,9} \draw[red,pattern=north east lines,pattern color=red!25] (\x,-1.5) circle (0.3);
% Iternado sobre listas
\foreach \x in {0,1,3,4} \draw[fill=blue!25,draw=blue] (\x,-2.5) circle (0.3);
% contador com iteracao em letras
\foreach \x [count = \xi] in {a,...,c,A,B,...,D}
{
  \draw[shade, left color=blue!25, right color=red!25,draw=blue] (\xi,-3) -- (\xi+0.5,-3) -- (\xi,-4) -- cycle;
  \draw (\xi+0.25,-4) node{\x};
}
\end{tikzpicture}
\end{center}
\end{ltxlisting}
%\end{LTXexample}
%\end{singlespacing}

Na matriz do \texttt{tikz}, podemos desenhar sobre ele.
Veja Exemplo~\ref{ex16:tikz:matriz}.

%\begin{singlespacing}
%\begin{LTXexample}[pos=b,caption={ex12-02.tex, parte k}, label=ex:12-02:k]
\begin{ltxlisting}[caption={ex16-tikz-matriz.tex}, label=ex16:tikz:matriz]
\begin{center}
\begin{tikzpicture} 
\matrix (A) [matrix of math nodes,left delimiter=[,right delimiter={]} ] 
{ 1 & 2 & -1 & 5 \\
0 & |[draw=red,circle]|3 & 2  & 1 \\
0 & -2 & \phantom{-}2  & -4 \\
}; 
% separando a matriz da parte aumentada
\draw[thick,dashed,blue] (A-1-3.north east) -- (A-3-3.south east);
% limitando a parte escalonada
\draw[thick,dotted,blue] (A-2-1.north west) -- (A-2-1.north east) -- (A-3-1.south east);
% onde vai colocar a seta para piv\^o
\coordinate (P) at ($(A-2-1)!-2!(A-2-2)$);
\coordinate (Q) at ($(A-2-1)!-1.3!(A-2-2)$);
% colocando a seta na linha de piv\^o
\draw[->] (P) -- (Q);
\node at ($(P)!-1.2!(Q)$) {piv\^o};
\end{tikzpicture}
\end{center}
\end{ltxlisting}
%\end{LTXexample}
%\end{singlespacing}

Caso precisar de recursos mais avançados de matriz, poderá usar o pacote \texttt{nicematrix}\index{nicematrix}, baseado em \texttt{tikz}.


O diagrama comutativo é uma espécie de tabela com comando \verb+\ar+\index{ar@\verb+\ar+} para desenhar setas. 
\verb+\ar+ é seta e o primeiro parâmetro opcional dele é para que lado vai a seta (pode ser combinado). ``r'' (direita), ``l'' (esquerda), ``d'' (para baixo), ``u'' (para cima). 
Também pode usar o ``\texttt{to=<destino>}'' para indicar a célula diretamente. 
O rótulo, caso exista, deve ser delimitado entre aspas.

%No Exemplo~\ref{ex16:tikz:diagrama}, precisará também do pacote \texttt{amssymb} para o comando \verb+\circlearrowleft+\index{circlearrowleft@\verb+\circlearrowleft+}, mesmo com \texttt{unicode-math} no \XeLaTeX/Lua\LaTeX{} (Se estiver usando \XeLaTeX/Lua\LaTeX, carregue antes do \texttt{unicode-math}).
%No \texttt{unicode-math}, este símbolo é \verb+\acwgapcirclearrow+\index{acwgapcirclearrow@\verb+\acwgapcirclearrow+}, mas na versão até 2017, ele não aparece. Isto porque, a fonte padrão ``Latin Modern Math'' ainda não tem este símbolo.


No Exemplo~\ref{ex16:tikz:diagrama}, foi usado o comando \verb+\circlearrowleft+\index{circlearrowleft@\verb+\circlearrowleft+} do pacote \texttt{amssymb}.
Se estiver usando o \texttt{unicode-math} no \XeLaTeX/Lua\LaTeX{}, poderá definir como sendo \verb+\acwopencirclearrow+\index{acwopencirclearrow@\verb+\acwopencirclearrow+}.
Para isso, coloque o comando
\verb+\providecommand{\circlearrowleft}{\acwopencirclearrow}+
no \texttt{preamble} do documento.

%\begin{singlespacing}
%\begin{LTXexample}[pos=b,caption={ex12-02.tex, parte l}, label=ex:12-02:l]
\begin{ltxlisting}[caption={ex16-tikz-diagrama.tex}, label=ex16:tikz:diagrama]
\begin{center}
\begin{tikzcd}
A \ar[r, "\phi"] \ar[d, red] \ar[to=2-2, phantom, "\circlearrowleft" description] & B \ar[d, "\psi" red] \\
C \ar[r, red, "\varphi" red, "\eta" {swap, blue}] & D
\end{tikzcd}
\end{center} 
\end{ltxlisting}
%\end{LTXexample}
%\end{singlespacing}

As bibliotecas para \texttt{tikz} é extensa, cobrindo vários tipos de desenhos e diagramas.
Se precisar de algo desse tipo, consulte o material para ver se já tem a biblioteca pronta.

Para desenhar gráficos de funções ou de dados em 2D ou em 3D, existe o pacote especializado \texttt{pgfplots}\index{pgfplots} que não vamos discutir aqui.

%xcolor, OK
%tikz, OK
%algumas bibliotecas do tikz, OK
%pgfplot? Nao veremos

\clearpage\thispagestyle{empty}\cleardoublepage
\chapter{Produzindo \texttt{Poster} e \texttt{Slides}}

\section{\texttt{Poster}}

Para quem quer produzir o \texttt{poster} diretamente no \LaTeX, a maneira mais prática é diagramar com um quarto (metade de largura e altura) do tamanho desejado, usando letra 12pt. 
Após finalizar, amplie o poster para produzir no tamanho real. 
Com tal procedimento, os caracteres serão ampliados para equivalentes a 24pt. 
Use a fonte \texttt{Sans Serif} para poster.
Caso de precisar usar a fonte romana, aumente o tamanho de caracteres em torno de 20\%.

Por exemplo, para poster final de 100x120cm, dimensione para 50x60cm, para dimensão final de A0, dimensione para A2, etc.

Para criar ``layout'' de divisão em colunas e similares, use o ambiente \texttt{multicols} (do pacote \texttt{multicol}) e \texttt{minipage}. 
O enfeite das ``caixas de texto'' podem ser feitos usando caixas como o \verb+\fbox+.

No Exemplo~\ref{ex17:poster}, foi usado o pacote \texttt{lipsum} para preencher o espaço com texto para ver como fica a aparência do poster. 

% \LTXinputExample{latex-via-exemplos-poster}
\begin{singlespacing}
% \lstinputlisting[caption={ex11-12.tex},label=ex:11-12]{latex-via-exemplos-poster}
\lstinputlisting[caption={ex17-poster.tex},label=ex17:poster]{latex-via-exemplos-poster}
\end{singlespacing}
%
\begin{framed}
\center
\includegraphics[width=0.7\linewidth]{latex-via-exemplos-poster}
\end{framed}

Se quer usar \LaTeX{} para produzir material publicitário como posters, jormais, revistas e livros ilustrados com a qualidade profissional, o pacote \texttt{flowfram}\index{flowfram} é o pacote adequado.
 Além da sua qualidade final, a facilidade e a versatilidade costumam ser superiores aos pacotes similares.
 
 Para criar \texttt{layout} complexos que inclui caixas não retangulares, poderá usar o programa livre e multi plataforma \texttt{flowframtk}\index{flowframtk} disponível no site
\url{http://www.dickimaw-books.com/latex/admin/html/flowfram.shtml}

Para quem estiver acostumado a usar o pacote \texttt{beamer} (pacote para produzir slides de apresentação), o pacote \texttt{beamerposter}\index{beamerposter} é interessante para produzir poster com cara de apresentação em \texttt{beamer}.
 
Em vez de usar o \LaTeX, poderá usar aplicativos gráficos para criar poster ou similares.
Para posters científicos, ideal que o aplicativo tenha suporte ao \LaTeX{} como no caso do editor gráfico \texttt{inkscape}\index{inkscape} e editor para publicações \texttt{scribus}\index{scribus}, ambos livres e multi plataforma.

Para aumentar o poster diagramado com fontes 12pt, como do Exemplo~\ref{ex17:poster}, podemos usar o aplicativo livre e multi plataforma \texttt{jPDFTweak}\index{jPDFTweak} que é um aplicativo para pós produção de documentos PDF, disponível em \url{http://jpdftweak.sourceforge.net/}.

Entre várias funções importantes deste aplicativo, vamos usar o redimensionamento para ampliar o poster.

No \texttt{jPDFTweak}:

\begin{enumerate}
\item indique o arquivo de entrada em ``input''.
\item indique o arquivo de saída em ``output''.
\item em [page size], cheque o ``[v] scale pages'' e selecione o tamanho da página como ``Page size'' = ``escala dobro do original''.
\item clique no [run]
\end{enumerate}

\textbf{Observação:} Para dobrar a escala no papel \texttt{ISO}, 
escolha dois números para baixo. 
Por exemplo, se o original estiver em \texttt{A2 landscape}, escolha \texttt{A0 landscape}. 
Caso a medida do papel estiver em centímetros, dobre o valor e em seguida, multiplique por $72/2.54$ para converter em \texttt{postscript point} que \texttt{jPDFTweak} usa na largura e na altura do papel. 
Por exemplo, se o original estiver com $\SI{45}{\cm}\times\SI{60}{\cm}$, 
vamos colocar $90\times 72/2.54=2551.18$ e $120\times 72/2.54=3401.57$
para largura e altura respectivamente.

\section{\texttt{Slides}}

Para criar apresentações, costuma usar a classe \texttt{beamer}\index{beamer}.

Algumas opções da classe \texttt{beamer} são tamanho da fonte (normalmente usa 12pt, embora suporta o tamanho maior),
\texttt{handout} (para impressão. Ignora o \texttt{overlay}), \texttt{notes} (incluir notas), \texttt{notesonly} (somente notas).

A classe \texttt{beamer} usa uma área reduzida com letra normal que será ampliado quando projetar na tela.
Assim, costuma usar a fonte de tamanho 12pt.

Nesta classe, cada tela de slide será delimitado pelo ambiente \texttt{frame}\index{frame}. 
Tem a versão comando do \texttt{frame}\index{frame@\verb+\frame+} também, para ser usado quando tem poucos comandos contidos nele, como no caso de título e sumário.

O tema (modelo) de slides é escolhido pelo comando
\verb+\usetheme+\index{usetheme@\verb+\usetheme+}.
Tema que vem como padrão são:
\texttt{AnnArbor, Antibes, Bergen, Berkeley, Berlin, Copenhagen,
Darmstadt, Dresden, Frankfurt, Goettingen, Hannover,
Ilmenau, JuanLesPins, Luebeck, Madrid, Malmoe,
Marburg, Montpellier, PaloAlto, Pittsburgh, Rochester,
Singapore, Szeged, Warsaw, boxes, default
e CambridgeUS}.

Para cada tema, ainda podemos escolher o tema de cores, ou combinação de cores a serem usados, escolhidos pelo comando \verb+\usecolortheme+\index{usecolortheme@\verb+\usecolortheme+}. 
Tema de cores padrão que vem são: 
\texttt{default, albatross, beaver, beetle, crane, dolphin, dove, fly, lily, orchid, rose, seagull, seahorse, whale e wolverine}.

Para ver como ficará a combinação do tema e tema de cores, poderá consultar o 
\url{https://hartwork.org/beamer-theme-matrix/}.

É pouco usado, mas ainda podemos alterar as combinações das fontes com o comando \verb+\usefonttheme+\index{usefonttheme@\verb+\usefonttheme+}. 
Tema de fontes padrão são: 
\texttt{default, serif, professionalfonts, structurebold,
 structureitalicserif, structuresmallcapsserif}.
Em geral, usa-se o comando \verb+\setbeamerfont+\index{setbeamerfont@\verb+\setbeamerfont+} para configurar fontes de seus elementos do que escolher um tema para fontes (combinação de fontes).

A classe \texttt{beamer} carrega o pacote \texttt{hyperref} por padrão.
Então podemos configurar alguns aspectos sobre informações do \texttt{PDF} com o \verb+\hypersetup+, mas isto não é obrigatório.

Para criar slide de títulos ou quando estiver usando tema que usa títulos, autores, etc, precisará informar o titulo, autor, instituição e data, respectivamente pelos comandos \verb+\title+, \verb+\author+, \verb+\institute+ e \verb+\date+.
Quando tiver mais de um autor, separe com \verb+\and+. 
O comando \verb+\inst+ dentro do \verb+\author+ e \verb+\institute+ faz a ligação de autor com o instituto correspondente.

Os elementos do slide pode ser alterado pelo comando \verb+\setbeamertemplate+\index{setbeamertemplate@\verb+\setbeamertemplate+}.
Para criar degrade no fundo do slide, costuma usar o recurso do pacote \texttt{tikz}.
Note que \texttt{xcolor} é carregado pelo \texttt{beamer} por padrão.
O comando \verb+\setbeamertemplate+ também pode ser usado para desativar os botões de navegação.

Os comandos \verb+\maketitle+ e \verb+\tableofcontents+ funcionam normalmente. 
Note que o sumário é construído a partir de \texttt{section} e não pelo \texttt{frame}.
Assim, para ter o sumário, deverá colocar \verb+\section+ entre os \texttt{frame}'s no local desejado.

Cada \texttt{slide} (tela) é criado pelo ambiente \texttt{frame}\index{frame} ou comando \verb+\frame+\index{frame@\verb+\frame+}.
O ambiente \texttt{frame} aceita o comando \verb+\frametitle+\index{frametitle@\verb+\frametitle+} para colocar títulos.

O comando \verb+\note+\index{note@\verb+\note+} serve para inserir notas adicionais que não são colocados no \texttt{slide}, mas podem ser gerados e impressos com a opção \texttt{notes} ou \texttt{notesonly} na opção da classe, para auxiliar na apresentação.

%Note que existe a versão ambiente de \texttt{note}\index{note} para anotações maiores, mas ele só funciona fora do \texttt{frame} (a versão comando funciona tanto dentro como fora do \texttt{frame}).
% Não está funcionando na versão 2016. Testar na versão mais recente.

As notas adicionadas fora do \texttt{frame} será assumido que do \texttt{frame} anterior.

O Exemplo~\ref{ex17:slides} ilustra um slide e uma das telas do slide gerado pelo código listado. 
Nele, foi usado o pacote \texttt{lipsum} para preencher alguns slides.

\begin{singlespacing}
%\lstinputlisting[caption={ex12-01.tex},label=ex:12-01]{latex-via-exemplos-slides}
\lstinputlisting[caption={ex17-slides.tex},label=ex17:slides]{latex-via-exemplos-slides}
\end{singlespacing}
%
\begin{framed}
\center
\includegraphics[page=7, width=0.7\linewidth]{latex-via-exemplos-slides}
\end{framed}

Note que, nos itens de listas, aceitam o parâmetro opcional de ``\texttt{overlay}''\index{overlay} delimitado por ``<'' e ``>''. 
Com o uso de \texttt{overlay}, o \texttt{slide} será apresentado por etapas. 
Em cada etapa do \texttt{overlay}, será mostrado uma parte do \texttt{slide}. 
O mais usado é ir mostrando cada vez mais elemento na medida que etapa avança (andar no \texttt{slide}).

A especificação do \texttt{overlay} são:
\begin{description}
\item[<n>] indica que será mostrado somente na etapa n.
\item[<m-n>] é mostrado somente nas etapas m até n.
\item[<n->] é mostrado nas etapas n em diate.
\item[<-n>] é mostrado até as etapas n.
\end{description}

Os comandos de formatação de textos do tipo \verb+\textsf{}+, \verb+\textbf{}+, \verb+\emph+, etc, também aceitam a opção de \texttt{overlay}.
Também foi acrescido o comando \verb+\alert+ neste conjunto para deixar o texto em vermelho.

Para criar \texttt{overlay} nas partes desejadas que não é necessariamente itens da lista ou formatação de texto,
existe o comando \verb+\only+\index{only@\verb+\only+} que mostra o conteúdo somente no \texttt{overlay} indicado.

Também tem o ambiente chamado de \texttt{block}\index{block} que produz ``bloco'' com ou sem títulos e aceita ``overlay''.

No \texttt{frame}, alguns parâmetros opcionais podem ser usados.
Alguns dos mais importantes são
\begin{description}
\item[fragile] Necessário quando usa o ambiente \texttt{verbatim} ou similar.
\item[allowframebreaks] Ativa a quebra automática de  \texttt{frame}'s quando o conteúdo não cabe no \texttt{frame} atual.
\item[plain] Limpa as configurações (cabeçalhos e rodapés) do \texttt{frame} para caber mais conteúdos. Útil para figuras e tabelas grandes.
\item[shrink] Reduz o tamanho do conteúdo para caber no \texttt{frame} atual, se necessário. Isto pode reduzir o tamanho da letra e por isso, deve ser usado com muita cautela. Use em conjunto com \texttt{plain} para ampliar a área de \texttt{slide}.
\end{description}

Note que \BibTeX{} não funciona no \texttt{beamer}.
Assim, deverá criar manualmente a referência bibliográfica com o ambiente \texttt{thebibliography}.

\clearpage\thispagestyle{empty}\cleardoublepage
\chapter{Usando  \XeLaTeX{} e Lua\LaTeX{}}
Aqui será tratado sobre \XeLaTeX{} e Lua\LaTeX{}, considerado as próximas gerações de \LaTeX.

\section{Lua\LaTeX{} e \XeLaTeX}

O Lua\LaTeX{}\index{LuaLaTeX@{Lua(La)\TeX}} foi designado para ser o sucessor do PDF\LaTeX{} e por isso, deve ser o \LaTeX{} padrão no futuro, mas pode ser um pouco mais lento do que o \XeLaTeX.
Até o momento (2018), o \XeLaTeX{}\index{XelaTeX@{Xe(La)\TeX}} é considerado mais estável do que Lua\LaTeX.
Lua(La)\TeX{} permite estender a funcionalidade do (La)\TeX{} usando a linguagem script \texttt{Lua} (\texttt{Lua} foi desenvolvido no PUC-Rio, aqui no Brasil).

Já existem pacotes especiais para Lua\LaTeX{} como o módulo de diagramação automática de grafos no \texttt{pgf/tikz}.
O suporte ao recurso de microtipografia (efetuar pequenos ajustes no tamanho das letras e espaçamento para que texto acomode melhor) pelo pacote \texttt{microtype} também funciona melhor com Lua\LaTeX{} do que em \XeLaTeX.

A maioria dos documentos de \LaTeX{} devem funcionar sem modificações no \XeLaTeX/Lua\LaTeX, mas para usar as fontes adicionais em Open Type, ou as fontes do sistema, devem usar a forma própria de selecionar as fontes. 
%No caso em que a fonte selecionada contiver símbolos matemáticos suficientes, costuma usar o \texttt{unicode-math}\index{unicode-math} em vez dos \texttt{amssymb} e \texttt{amsmath}.
%Caso precisar de alguns símbolos de \texttt{amssymb} ou comandos de \texttt{amsmath}, carregue antes do \texttt{unicode-math}.

%PDFLaTeX e XeLaTeX/LuaLaTeX usa esquema diferente para seleção de fontes e podem usar esquema diferente também para a seleção de idiomas.

A forma de ajustar a codificação de documento fonte no \LaTeX, \XeLaTeX{} e Lua\LaTeX{} diferem, mas como \XeLaTeX{} e Lua\LaTeX{}  usam \texttt{uft8} como padrão, não vamos preocupar com isso.
Só no \LaTeX{} que precisará especificar, usando o pacote \texttt{inputenc}.

Embora \texttt{babel} funcione no \XeLaTeX/Lua\LaTeX{} também, no Exemplo~\ref{ex18:lualatex} foi usado o \texttt{polyglossia}\index{polyglossia} feito especialmente para \XeLaTeX/Lua\LaTeX. 
%Ainda (2018) é cedo para saber se \texttt{babel} ou \texttt{polyglossia} vai ficar como pacote recomendável para \XeLaTeX/Lua\LaTeX.

Embora maioria dos pacotes de fontes do \LaTeX{} funcione no \XeLaTeX/Lua\LaTeX, no Exemplo~\ref{ex18:lualatex}, foi usado o método próprio deles que permite selecionar fontes adicionais, inclusive as fontes do sistema.

Se a fonte usada contiver símbolos matemáticos, use \texttt{unicode-math} em vez do \texttt{amssymb}/\texttt{amsmath}.
Se precisar dos símbolos ou comandos de \texttt{amssymb} e \texttt{amsmath}, carregue eles antes do \texttt{unicode-math}.

% O coamdo \DeclareMathOperator do amsmath nao esta no unicode-math.

Note que os pacotes e comandos do \LaTeX{} padrão, exceto o \texttt{inputenc} costumam funcionar no \XeLaTeX{} e Lua\LaTeX, mas não o contrário.
Assim, para que o documento possa ser compilado tanto em PDF\LaTeX{} como em \XeLaTeX/Lua\LaTeX, poderá usar o pacote \texttt{iftex}\index{iftex} para detectar a engenharia do \TeX{} em uso.


% O xunicode está obsoleto devido a funcionalidade incorpolado no fontsppec.
% No LuaLaTeX, ate pode dar erros caso usar este pacote.

%\setcounter{lstorichapter}{\value{chapter}}
%\setcounter{lstorisection}{\value{section}}
% \let\orilabel\label
\begin{singlespacing}
%\begin{LTXexample}[pos=b,caption={ex06-08.tex, parte b},label=ex:06-08:b,preset={\setcounter{chapter}{1},\setcounter{section}{0},\let\addcontentsline\emptyaddcontentsline}]
%\begin{lstlisting}[caption={ex12-04.tex},label=ex:12-04]
\begin{lstlisting}[caption={ex18-lualatex.tex},label=ex18:lualatex]
\documentclass[a4paper,12pt]{article}

\usepackage{iftex} % Para detectar engenharia de TeX

\ifPDFTeX % Se (PDF)LaTeX
  \usepackage[T1]{fontenc} % codificação da fonte em 8-bits
  \usepackage[utf8]{inputenc} % acentuação direta
  \usepackage[english,brazil]{babel} % em portugues brasileiro (ingles como secundario)
    
  \usepackage{lmodern} % latin Modern (Computer modern com extensao latin)
  
  % \usepackage{noto} % fonte Noto patrocinado pelo Goggle
  % \usepackage[italic]{mathastext} % A fonte Noto nao possui fontes matematicos ainda
  
  \usepackage{textcomp}
  \usepackage{amssymb,amsmath}
  
\else % Se XeLaTeX/LuaLaTeX
  \usepackage{polyglossia}
  \setdefaultlanguage{brazil}
  \setotherlanguage{english} % secundario
  
  % Estas fontes são padrões e não precisavam especificar. Foram colocados somente para ilustrar
  \setmainfont{Latin Modern Roman}
  \setsansfont{Latin Modern Sans}
  \setmonofont{Latin Modern Mono}
  % Se precisar do amssymb,amsmath, deverão carregar antes do unicode-math
  % \usepackage{amssymb,amsmath}
  
  \usepackage{unicode-math} % amsmath, amssymb equiv.
  \ifLuaTeX % Somente LuaLaTeX
    \usepackage{lualatex-math} % alguns fix do amsmath/mathtools equiv. para LuaLaTeX.
  \fi
  \setmathfont{Latin Modern Math} % Padrão e não precisava. Foi colocado somente para ilustração
  
  % Fonte Noto não tem fontes matemáticos ainda
  %\setmainfont[Ligatures=TeX]{NotoSerif}
  %\setsansfont{NotoSans}
  %\setmonofont{NotoMono}
  % Asana math (Palatino like)
  % \setmainfont{Asana-Math}
  % STIX (Times like font)
  %\setmainfont[Ligatures=TeX]{STIX}
  %\setmathfont{STIX Math}
  %% XITS (Times like font)
  %\setmainfont[Ligatures=TeX]{XITS}
  %\setmathfont{XITS Math}
\fi

% Mais alguns pacotes
\usepackage[margin=2.5cm]{geometry}
\usepackage{hyperref}
\hypersetup{
  bookmarks=true
}
\usepackage[stretch=10]{microtype} % microtipografia: funciona no PDFLaTeX e LuaLaTeX (parcialmente em XeLaTeX)

\begin{document}
O Lua\LaTeX{} foi designado para ser o sucessor do PDF\LaTeX{} e por isso, deve ser o \LaTeX{} padrão no futuro. Por outro lado, XeLaTeX e mais rápido e estável (2018).

\ldots

Se $F'(x)=f(x)$ for contínua, 
\[
\int_a^bf(x)dx = F(b) - F(a)
\]
\end{document}
\end{lstlisting}
\end{singlespacing}
%\let\ref\oriref
%\setcounter{chapter}{\value{lstorichapter}}
%\setcounter{section}{\value{lstorisection}}
%% \addtocounter{chapter}{-1}
%\let\addcontentsline\oriaddcontentsline

%\begin{singlespacing}
%\begin{framed}
\begin{ltxlistingout}
O Lua\LaTeX{} foi designado para ser o sucessor do PDF\LaTeX{} e por isso, deve ser o \LaTeX{} padrão no futuro. Por outro lado, \XeLaTeX{} e mais rápido e estável (2018).

\ldots

Se $F'(x)=f(x)$ for contínua, 
\[
\int_a^bf(x)dx = F(b) - F(a)
%\vspace*{-1.5\baselineskip}
\]
\end{ltxlistingout}
%\end{framed}
%\end{singlespacing}

O usuário de MS Windoes devem ficar atentos pelo fato de que a fonte ``Latin Modern Math'' usada no Exemplo~\ref{ex18:lualatex} não vem instalado por padrão no Mik\TeX{} Básico.
Como \XeLaTeX/Lua\LaTeX{} acessa as fontes diretamente sem usar os pacotes, costuma precisar a instalação manual do pacote \texttt{lm-math} usando o gerenciador do Mik\TeX.

Existem algumas fontes \texttt{Open Type} com suporte a matemática que podem ser usados no \XeLaTeX/Lua\LaTeX.
O site 

\noindent
\url{https://www.overleaf.com/help/193-what-otf-slash-ttf-fonts-are-supported-via-fontspec} lista algumas delas (com suporte aos símbolos matemáticos).

\begin{itemize}
\item \texttt{Cambria Math} (Microsoft, somente Windows).
\item \texttt{Latin Modern Math} (Bogusław Jackowski, Janusz M. Nowacki). Padrão do \XeLaTeX/Lua\LaTeX.
\item \texttt{TeX Gyre Pagella Math} (Bogusław Jackowski, Janusz M. Nowacki).
\item \texttt{TeX Gyre Termes Math}.
\item \texttt{Asana Math} (Apostolos Syropolous), Estilo \texttt{Palatino}.
\item \texttt{Neo Euler} (Khaled Hosny).
\item \texttt{STIX} (STI Pub).
\item \texttt{XITS Math} (Khaled Hosny).
\end{itemize}
% Outros: TeX Gyre Bonum Math, TeX Gyre Schola Math, 

Note que a fonte \verb+\mathcal+ e \verb+\mathbb+ do \texttt{unicode-math} é diferente do \texttt{amssymb}.
O \verb+\mathbb+ é mais próximo do \verb+\mathds+ do \texttt{dsfont} e \verb+\mathcal+ é mais parecido com do pacote \texttt{eucal} do que de \texttt{amssymb}.

Quem quer usar a versão do \texttt{amssymb}, poderá colocar
\begin{singlespacing}
\begin{lstlisting}
\let\mathcal\relax % remove the definition by unicode-math
\DeclareMathAlphabet{\mathcal}{OMS}{cmsy}{m}{n}
\let\mathbb\relax % remove the definition by unicode-math
\DeclareMathAlphabet{\mathbb}{U}{msb}{m}{n}
\end{lstlisting}
\end{singlespacing}
no \texttt{preamble}, após carregar o \texttt{unicode-math}.

Note que, nem todos os símbolos do \texttt{unicode-math} costumam estar presentes na fonte utilizada.
Quando a fonte utilizada não apresentar os símbolos correspondentes aos comandos de \texttt{unicode-math}, o símbolo simplesmente não aparece, sem emitir mensagens de erro, o que requer cuidados.
Neste caso, deverá complementar com outras fontes, ou procurar símbolos alternativos.
Por exemplo, carregar o pacote \texttt{amssymb} antes do \texttt{unicode-math} complementa com os símbolos de AMS.

\section[Fontes no \XeLaTeX/Lua\LaTeX]{Mais sobre básicos das fontes no \XeLaTeX/Lua\LaTeX}

No \XeLaTeX/Lua\LaTeX, poderá redigir em qualquer idioma desde que esteja utilizando as fontes que suportam tais idiomas.
Assim, quando redige o documento multi lingue, basta definir qual fontes será usado em qual trecho.
Para facilitar a especificação de fontes em cada trecho de textos, costuma definir família de fontes usando \verb+\newfontfamily+\index{newfontfamily@\verb+\newfontfamily+} do pacote \texttt{fontspec}\index{fontspec}. 
Ele define comandos para usar a fonte especificada com as configurações também especificadas, que funcionam como outros comandos de seleção de famílias de fontes pré definidos tais como \verb+\rmfamily+, \verb+\sffamily+, etc.

Por exemplo,
\begin{singlespacing}
\begin{lstlisting}
\newfontfamily{\notorm}{Noto Serif}
\newcommand{\textnotorm[1]{{\notorm #1}}
\newfontfamily{\notosf}{Noto Sans}
\newcommand{\textnotosf[1]{{\notosf #1}}
\end{lstlisting}
\end{singlespacing}
%
definem comandos \verb+\notorm+ e \verb+\notosf+ para usar fontes \texttt{Noto Serif} e \texttt{Noto Sans}.
O \verb+\newcommand+ foi usado para criar a versão \verb+\textxx+ que são \verb+\textnotorm+ e \verb+\textnotosf+ respectivamente.
Note o uso de chave dupla para que a configuração da fonte não ``vaze'' para fora.

Para usar fontes não definidos anteriormente, poderá usar o comando \verb+\fontspec+\index{fontspec@\verb+\fontspec+} para selecionar a fonte especificada.

Apesar de poder configurar a especificação da fonte com o parâmetro opcional dos comandos de seleção das fontes e definições de família de fontes,
isto também pode ser efetuado separadamente pelo comando \verb+\defaultfontfeatures+\index{defaultfontfeatures@\verb+\defaultfontfeatures+} para especificar recursos das fontes. 
O parâmetro opcional será usado para indicar o nome da fonte que está especificando.
Por exemplo,
\begin{singlespacing}
\begin{lstlisting}
\defaultfontfeatures[Noto Serif]{Ligatures=TeX}
\end{lstlisting} 
\end{singlespacing}
Especifica que no \texttt{Noto Serif}, será usado a ligadura no modo \TeX.
Além de indicar nome das fontes, poderá indicar pelo comando de família das fontes como em
\begin{singlespacing}
\begin{lstlisting}
\defaultfontfeatures[\rmfamily,\sffamily]{Ligatures=TeX}
\end{lstlisting} 
\end{singlespacing}
especifica que na fonte \texttt{romana} e \texttt{sans serif}, será usado a ligadura padrão do \TeX. 

Para limpar a especificação das fontes, basta usar como
\begin{singlespacing}
\begin{lstlisting}
\defaultfontfeatures[\rmfamily,\sffamily]{}
\end{lstlisting} 
\end{singlespacing}
%
O comando \verb+\defaultfontfeatures+\index{defaultfontfeatures*@\verb+\defaultfontfeatures*+} tem a versão ``+'' que acrescenta a especificação em vez de substituir.
O comando \verb+\addfontfeature+\index{addfontfeature@\verb+\addfontfeature+} pode ser usado para ajustar localmente a especificação. 
Coloque entre chaves para especificação adicional seja aplicado somente no trecho.

Quando lida com várias fontes no \XeLaTeX/Lua\LaTeX, as vezes é importante checar se a fonte existe.
\verb+\IfFontExistsTF+\index{IfFontExistsTF@\verb+\IfFontExistsTF+} do pacote \texttt{fontspec} checa se a fonte do primeiro parâmetro existe. 
Caso existir, executará o segundo parâmetro e caso não existir, executa o terceiro parâmetro.
O comando a seguir configura a fonte principal para \texttt{Noto Serif} caso ela existir.
\begin{singlespacing}
\begin{lstlisting}
\IfFontExistsTF{Noto Serif}{\setmainjfont{Noto Serif}}{%
  \PackageWarning{\jobname.tex}{Noto Serif not found. Using default fonts}
\end{lstlisting} 
\end{singlespacing}
% O comando \IfFontExistisTF nao esta disponivel no fontspec de 2016.

Ir testando várias fontes com este comando até achar uma disponível na lista é trabalhoso. 
Assim, para procurar fontes na lista, costuma usar o pacote \texttt{iffont}\index{iffont}. 
Ele implementa o comando \verb+\settofirstfound{\nomefonte}{<fontes>}+\index{settofirstfound@\verb+\settofirstfound+} onde \texttt{<fontes>} é lista de nome das fontes, separado pela vírgula.
Ele associa ao comando \verb+\nomefonte+, o nome da primeira fonte encontrada na lista, o que pode ser usado para selecionar ou definir família de fontes.
Caso nenhuma fonte existir, retorna o valor de \verb+\@iffont@firstfont+ que está como a fonte \texttt{Fira Sans}. % março de 2018
Em geral, coloca-se o nome de uma fonte existente no final da lista para evitar de cair no \texttt{Fira Sans} que nem sempre é desejável.

No entanto, \texttt{iffont} não funciona devidamente no XeLaTeX/LuaLaTeX recente.
% só funciona no \XeLaTeX{} (testados nas versões 2016 e março de 2018). 
Para que as funções do \texttt{iffont} funcione devidamente,
%no Lua\LaTeX, 
reescreva o comando \verb+\@iffont@checkfont+ usando o \verb+\IfFontExistisTF+ do pacote \texttt{fontspec} como em
\begin{singlespacing}
\begin{lstlisting}
...
\usepackage{iftex}
...
\usepackage{iffont}
% patch
% \usepackage{etoolbox} % iffont ja carregou
\let\iffontexists\IfFontExistsTF
\makeatletter
  \renewcommand{\@iffont@checkfont}[1]{
    \IfFontExistsTF{#1}{
      \nottoggle{@iffont@fontfound}{%
        \renewcommand{\@iffont@firstfont}{#1}
        \toggletrue{@iffont@fontfound}
      }{}
    }{\stepcounter{@iffont@fontsnotfound} }
  }
\makeatother
\end{lstlisting}
\end{singlespacing}

Com este ajuste, os comandos de \texttt{iffont} vão funcionar devidamente.%no Lua\LaTeX{} também.

Por exemplo, o código
\begin{singlespacing}
\begin{lstlisting}
\settofirstfound{\mainfont}{Noto Serif, Latin Modern Serif}
\setmainfont{\mainfont}
\end{lstlisting}
\end{singlespacing}
%
configura para usar o ``Noto Serif'' como fonte padrão caso existir. Caso ele não for encontrado, usa-se a fonte ``Latin Modern Serif'' que deve estar presente na instalação padrão dp \TeX.

Se o que quer é saber se fonte existe, o \texttt{iffont} também tem os comandos \verb+\iffontsexist+\index{iffontsexist@\verb+\iffontsexist+} e \verb+\iffontexists+\index{iffontexists@\verb+\iffontexists+} na qual executa o segundo parâmetro quando fonte existe e terceiro parâmetro quando a fonte não existe.
A diferença é que no \verb+\iffontsexist+, o parâmetro é lista de nome das fontes separado pela vírgula e no \verb+\iffontexists+, o parâmetro é nome de uma única fonte (igual a \verb+\IfFontExistsTF+ do \texttt{fontspec}).
Por exemplo,
\begin{singlespacing}
\begin{lstlisting}
\iffontexists{Noto Serif}{\setmainjfont{Noto Serif}}{%
  \PackageWarning{\jobname.tex}{Noto Serif not found. Using default font}
\end{lstlisting}
\end{singlespacing}
Configura para \texttt{Noto Serif} só quando ele for acessível. 

%Lembrando novamente que os comandos do pacote \texttt{iffont} não funcionam devidamente no Lua\LaTeX{} (testados nas versões 2016 e março de 2018).
%Se estiver usando o Lua\LaTeX, use \verb+\IfFontExistsTF+ em vez das funções do pacote \texttt{iffont}.

As fontes Open Type que podem ser usados pelo \XeLaTeX{} e Lua\LaTeX{} podem estar no diretório do documento, diretório de \TeX{} ou instalado no sistema operacional.
No Lua\LaTeX, estas fontes podem ser acessadas pelo nome das fontes, mas no \XeLaTeX, somente as fontes instaladas no sistema operacional podem ser acessados pelo nome. Fontes encontradas em outras localidades tais como junto ao documento \TeX{} ou no diretório de \TeX{} devem ser indicados pelo nome do arquivo, incluindo a sua extensão.
Assim, se os comandos do pacote \texttt{fontspec} ou de \texttt{iffont} não conseguirem encontrar as fontes pelo nome no \XeLaTeX, tente usar o nome do arquivo da fonte (sem esquecer da sua extensão).

No caso de querer usar as fontes da MicroSoft no MS Windows,% compilando com LuaLaTeX, 
coloque

\begin{singlespacing}
\begin{lstlisting}
    \setmainfont[Ligatures=TeX]{Cambria}
    \setsansfont{Calibri}
    \setmonofont{Consolas}
    \setmathfont{Cambria Math}
\end{lstlisting}
\end{singlespacing}

no preamble, lembrando que no caso de XeLaTeX, fontes que não está no sistema (como fontes que esão no diretório de \TeX) requer passar o nome do arquivo das fontes em vez dos nomes das fontes.

\section{Usando em conjunto com  Bib\LaTeX}
O documento em vários idiomas pode precisar também de referências bibliográficas em vários idiomas.
O Bib\LaTeX{} permite internacionalizar a referência bibliográfica (mesmo sem usar \XeLaTeX/Lua\LaTeX).

Para especificar o idioma da referência bibliográfica no arquivo \texttt{bib} para Bib\LaTeX, coloca-se o idioma no campo \texttt{langid}\index{biblatex!langid} de cada item.
O ``\verb+*_romanized+''\index{biblatex!romanized} é a escrita em alfabeto romano caso o campo  esteja em carácter que não seja romano. 
Ele será usado para a ordenação e similar.
O campo ``\verb+*_translated_<idioma>+''\index{biblatex!translated} é a tradução para o \verb+<idioma>+ se as referências forem formatados para \verb+<idioma>+.
Veja o Exemplo~\ref{ex18:biblatex:bib}.

\begin{singlespacing}
%{\jafamily
% \begin{lstlisting}[caption={ex12-05.bib},label=ex12-05:bib]
\begin{lstlisting}[caption={ex18-biblatex.bib},label=ex18:biblatex:bib,escapeinside={[*}{*]}]
@book{wikibooks:latex,
  LANGID = {english},
  author={wikibooks},
  title="{\LaTeX{}}",
  publisher="wikibooks",
  url = "\url{https://en.wikibooks.org/wiki/LaTeX}",
  date = {2018},
  urldate={2018-03-05}
}

@Book{Kawasaki,
  LANGID = {japanese},
  TITLE  = {[*\textja{バラと折り紙と数学と}*]},
  TITLE_romanised = {bara to origami to sugaku to},
  TITLE_translated_english = {Roses, Origami \& Math},
  AUTHOR           = {[*\textja{川崎敏和}*]},
  AUTHOR_romanised = {Kawasaki, Toshikazu},
  PUBLISHER                    = {[*\textja{森北出版株式会社}*], Japan},
  PUBLISHER_translated_english = {Morikita Syuppan Co. Ltd, Japan},
  year={1998},
  ISBN = {4-627-01671-9}  
}
\end{lstlisting}
%} % \jafamily
\end{singlespacing}

O comando de configuração para Bib\LaTeX{} é algo como
\begin{singlespacing}
\begin{lstlisting}
\usepackage[
  backend=biber,
  language=auto,
%  autolang=other, % for <otherlanguage> environment from babel and polygrossia
%  autolang=langname, % only for polygrossia <language> env.
  % bibencoding=utf8,
  style=authoryear,
]{biblatex}
\addbibresource{ex18-biblatex.bib} % arquivo bib
\end{lstlisting}
\end{singlespacing}

A opção \verb+language=auto+ indica que é para selecionar automaticamente o idioma (pelo campo \texttt{LANGID}).
Obviamente, onde quer que apareça a referência bibliográfica, coloca-se o 
\begin{singlespacing}
\begin{lstlisting}
\phantomsection % se estiver usando hyperref
\printbibliography[heading=bibintoc]
\end{lstlisting}
\end{singlespacing}

Note que, até agora, não usamos nada que não funcione no \LaTeX{} normal.
Mas como a referência bibliográfica contém caracteres em japonês, precisará definir o uso de japonês em algum trecho.
Isto será facilitado se estiver usando \XeLaTeX{} ou Lua\LaTeX{} em vez do PDF\LaTeX.
Assim, vamos supor que está usando o \XeLaTeX/Lua\LaTeX{} para prosseguir com a explicação.
Para definir comandos para texto em japonês, coloque o comando 
\begin{singlespacing}
\begin{lstlisting}
\usepackage[Ligatures=TeX]{fontspec}
\usepackage{iffont}
% patch
\let\iffontexists\IfFontExistsTF
\makeatletter
  \renewcommand{\@iffont@checkfont}[1]{
    \IfFontExistsTF{#1}{
      \nottoggle{@iffont@fontfound}{%
        \renewcommand{\@iffont@firstfont}{#1}
        \toggletrue{@iffont@fontfound}
      }{}
    }{\stepcounter{@iffont@fontsnotfound} }
  }
\makeatother

\settofirstfound{\mainjfont}{Noto Serif CJK JP, MS Mincho, TakaoMincho, IPAexMincho, Hiragino Mincho Pro}
\settofirstfound{\gothicjfont}{Noto Sans CJK JP, MS Gothic, TakaoGothic, IPAexGothic, Hiragino Maru Gothic Pro}
% If Gothic not found, use Mincho instead.
\IfFontExistsTF{\gothicjfont}{}{\let\gothicjfont\mainjfont}
% If Mincho not found, error.
\IfFontExistsTF{\mainjfont}{\newfontfamily{\jafamily}[BoldFont=\gothicjfont]{\mainjfont}}{%
\PackageError{\jobname}{Japanese Mincho font not found}{Install apropriate ones}
}%
\newcommand{\textja}[1]{{\jafamily #1}}
\end{lstlisting}
\end{singlespacing}
Lembre-se que, para o pacote \texttt{iffont} funcionar no Lua\LaTeX, precisará ajustar o comando \verb+\@iffont@checkfont+ após carregar o \texttt{iffont}.

O comando \verb+\jafamily+ será definido quando encontrar alguma fonte da lista.
Também será definido o comando \verb+\textja+ para trecho pequeno.
%Note que \verb+\settofirstfound+ do \texttt{iffont} não funciona no Lua\LaTeX{} (testado na versão 2016 e março de 2018).
%Então, se quer usar o Lua\LaTeX, deverá udar a sequencia de \verb+\IfFontExistsTF+ em vez de \verb+\settofirstfound+.

Agora, basta usar o comando para trocar de fontes padrão para a fonte japonês onde aparece o texto em japonês.
Veja o Exemplo~\ref{ex18:biblatex:tex}.
\begin{singlespacing}
%{\jafamily
%\begin{lstlisting}[caption={ex12-05.tex}, label=ex:12-05]
\begin{lstlisting}[caption={ex18-biblatex.tex}, label=ex18:biblatex:tex,escapeinside={[*}{*]}]
Figura contendo texto em japonês.
\begin{figure}[hbtp!]
\center
\begin{tikzpicture}
  \draw (0,0) circle(2) (0,1) node{\textja{[*\textja{集合}*]}};
  \draw (0,-0.5) circle(1) (0,-0.5) node{\textja{[*\textja{部分集合}*]}};
\end{tikzpicture}
\caption{Figura com texto em japonês}
\end{figure}

Sobre uso básico do \texttt{biblatex}, veja o \cite{wikibooks:latex}.
Agora, citando o livro em japonês \textja{\cite{Kawasaki}}.
\end{lstlisting}
%} % \jafamily
\end{singlespacing}
% 
%Terá saída
%\begin{singlespacing}
%\begin{framed}
\begin{ltxlistingout}
Figura contendo texto em japonês.
\begin{center}
\center
\begin{tikzpicture}
  \draw (0,0) circle(2) (0,1) node{\textja{集合}};
  \draw (0,-0.5) circle(1) (0,-0.5) node{\textja{部分集合}};
\end{tikzpicture}

{Figuraq 1: Figura com texto em japonês}
\end{center}

Sobre uso básico do \texttt{biblatex}, veja o \texttt{wikibooks} 2018.
Agora, citando o livro em japonês \textja{川崎敏和} 1998.
\end{ltxlistingout}
%\end{framed}
%\end{singlespacing}

Como a referência bibliográfica contém texto em japonês, colocamos o comando de seleção de fonte japonês.
\begin{singlespacing}
\begin{lstlisting}
\phantomsection % se estiver usando hyperref
{\jafamily \printbibliography[heading=bibintoc]}
\end{lstlisting}
\end{singlespacing}

e a saída será

%\let\addcontentsline\emptyaddcontentsline
%\begin{framed}
\begin{ltxlistingout}
{\jafamily
\begin{thebibliography}{99}
%\section*{Referências Bibliográficas}
\bibitem{wikibooks:latex}\texttt{wikibooks (2018)}. \emph{\LaTeX}. wikibooks. URL: \url{https://en.wikibooks.org/wiki/LaTeX} (acesso em 15/03/2018).
\bibitem{kawasaki}川崎敏和 (1998). バラと折り紙と数学と. 森北出版株式会社, Japan. ISBN: 4-627-01671-9.
\end{thebibliography}
} % \jafamily
\end{ltxlistingout}
%\end{framed}
%\let\addcontentsline\oriaddcontentsline
Note que, no exemplo acima, a fonte de todo trecho da referência bibliográfica foi trocada.
Assim, se tiver letras acentuadas na referências bibliográficas que não existem nas fontes japonesas escolhidas, terá problemas.
% ou que conflitam com letras japonesas, começará aparecer coisas indesejáveis.
Para contornar o problema, ou acentuar no modo \TeX, ou especificar o uso de japonês dentro do arquivo \texttt{bib} a cada trecho, em vez de aplicar no \verb+\printbibliography+.

No caso de escrever o documento em japonês ou similar, deverá carregar o pacote apropriado que configurará para ambiente do idioma correspondente, ajustando os parâmetros para particularidade daquele país.
Além disso, com o uso do tal pacote, a troca de fontes para japonês e vice versa será automática, sem a necessidade de estar especificando a família de fontes a ser usadas a cada trecho.

Como observação final sobre \XeLaTeX/Lua\LaTeX,
os comandos protegidos dos pacotes de \XeLaTeX/Lua\LaTeX{} costumam usar ``\verb+_+'' em vez de ``@'', ou ``\verb+_+'' em conjunto com ``@''.
Isto foi introduzido no \LaTeX{} pelo equipe de desenvolvimento \LaTeX{} 3 e aparecem nos pacotes mais recentes.
Assim, para usar estes comandos e ambientes no \texttt{preamble}, deverá colocar entre \verb+\ExplSyntaxOn+\index{ExplSyntaxOn@\verb+\ExplSyntaxOn+} e \verb+\ExplSyntaxOff+\index{ExplSyntaxOff@\verb+\ExplSyntaxOff+}.



%%%%%%%%%%%%%%%%%%%%%%%%%%%%%%%%%%%%%%%%%%%%%%%%%%%%%%%%%




%pacote iffont
%
%Algumas observações sobre unicode-math.
%
%Diferenca de amssymb/amsmath com unicode-math:
%mathbb, mathcal, \verb+\int\limits+

%\verb+\removenolimits+ e \verb+\addnolimits+
%nao permite usar versao com limits e nolimits ao mesmo tempo.

%\verb+\ExplSyntaxOn+
%e \verb+\ExplSyntaxOff+

%item \verb+\newfontfamily+ do fontspec

\clearpage\thispagestyle{empty}\cleardoublepage
\chapter{Diagramando na Norma ABNT}
Neste capítulo, vamos estudar uma classe e arquivos de suporte da referência bibliográfica para diagramar de acordo com a exigência do ABNT\index{ABNT} (Associação Brasileira das Normas Técnicas).

A classe para diagramar os documentos como trabalho de conclusão de cursos, monografias em geral, teses e dissertações em ABNT é a classe \texttt{abntex2}\index{abntex2} que foi implementado sobre a classe \texttt{memoir}, herdando diversas funcionalidades. Para saber estas funcionalidades adicionais, consulte o manual do \texttt{memoir}.

\section{Documentos em ABNT}
Para o documento em ABNT, usamos a classe \texttt{abntex2} como em
\begin{lstlisting}
\documentclass[12pt,a4paper,openright,twoside,english,brazil]{abntex2}
\end{lstlisting} 
onde \verb+12pt+ foi usado para ajustar o tamanho das letras, \verb+a4paper+ para papel \texttt{A4} e \verb+twoside+ e \verb+openright+ foi usado para criar impressão frente/verso seguindo a recomendação do ABNT (recomendação não é obrigatoriedade).  Se quer gerar a verão \texttt{online}, use a opção \verb+oneside+ no lugar.
Por último, \verb+english+ e \verb+brazil+ foi colocado para ser repassado no pacote \texttt{babel}, pois ABNT requer que tenha resumo na língua estrangeira (por exemplo, em inglês) além do resumo em português.

A classe \texttt{abntex2} implementa diversos comandos para produzir elementos de acordo com a exigência ou recomendações do ABNT. 
Alguns desses comandos possuem equivalência para comandos da classe \texttt{memoir} e estão mapeados para poder usar tanto os comandos do \texttt{memoir} como da própria classe \texttt{abntex2} (por exemplo, \verb+\autor+ e \verb+\author+, \verb+\titulo+ e \verb+\title+, \verb+\data+ e \verb+\date+, etc), enquanto que existem comandos que equivalente do \texttt{memoir} produz estilo diferente do exigido ou recomendado pelo ABNT (por exemplo, \verb+\imprimecapa+ e \verb+\maketitle+, \verb+\apendices+ e \verb+\appendix+).
Assim, requer cuidados.

Uma observação importante é usar sempre o \texttt{uft8} na classe ABNT, o que é necessário especificar com \verb+\usepackage[utf8]{inputenc}+ no caso de não estar usando o \XeLaTeX/Lua\LaTeX.

A classe \texttt{abntex2} faz a chamada de \verb+\frontmatter+ (ou \verb+\pretextual+) no início do documento. 
Assim, não é necessário colocar este comando no começo do documento.

O exemplo~\ref{ex19:abntex2} ilustra o uso desta classe, com alguns comentários importantes.
A sua saída será omitida aqui.

\begin{singlespacing}
\begin{lstlisting}[caption={ex19-abntex2.tex},label={ex19:abntex2}]
% \documentclass[12pt,a4paper,openright,twoside,english,brazil]{abntex2} % impressão no papel
\documentclass[12pt,a4paper,oneside,english,brazil]{abntex2} % online
\usepackage{iftex} % Para detectar o motor de TeX utilizado
% Fontes, de acordo com o motor do TeX
\ifPDFTeX
  \usepackage[T1]{fontenc} % codificação da fonte em 8-bits
  \usepackage{lmodern} % fontes latin modern
  \usepackage[utf8]{inputenc} % acentuação direta em UTF-8
\else
  \usepackage{fontspec} % pacote para configurar fontes
  \defaultfontfeatures{Ligatures=TeX}
  % seleção das fontes desejadas pode ser colocados aqui
\fi
% pacotes matemáticas de acordo com o motor TeX
\ifPDFTeX
  \usepackage{amsmath,amssymb} % para matematica
\else
  % para matematica
  \usepackage{unicode-math}
  \ifLuaTeX
    \usepackage{lualatex-math}
  \fi
\fi

\usepackage[overload]{textcase} % para poder ter comandos de LaTeX/fórmula  no titulo

\usepackage{hyperref} % para link automático no PDF
% Desabilitando ou redefinindo comandos que causa erros no bookmark do PDF
\pdfstringdefDisableCommands{%
  \let\MakeUppercase\relax% desabilitando
  \def\\{ }%quebra de linha será substituído pelo espaço
}
\usepackage{pdfpages} % para incluir documento PDF (ex.: folha de aprovação)
\usepackage{indentfirst} % se quer que o primeiro parágrafo seja indentada
%% Se for criar o índice remissivo (opcional)
%\usepackage{makeidx}
%\makeindex
% Dados para capa, folha de rosto, etc
\title{Exemplo de ABNTeX2} % ou \titulo{}
% caso de mais de um autor, separe com o comando \and
\author{Sadao Massago} % ou \autor{}
\date{Julho de 2021} % ou \data{}
\instituicao{DFQM-UFSCar} % Instituição
\local{Sorocaba} % Local.
% Texto do tipo "Monografia apresentada à ..., como requisito parcial para a obtenção do título de ..." impresso na folha de rosto e folha de aprovação.
% Consultar a sua instituição para saber o texto correto.
\preambulo{Texto do tipo ``Monografia apresentada ...'' que depende da instituição}
% Tipo de trabalho é usado na ficha catalográfica
% "Tese (doutorado)", "Dissertação (mestrado)", "Trabalho de conclusão de curso (graduação)", etc.
\tipotrabalho{Exemplo}
\orientador{Nome do orientador}
\coorientador{Nome dos coorientadores} % caso existam
% (Opcional): informações para PDF
% Somente para PDF
% Precisa ajustar o pdfkeywords (palavras-chave) de acordo com o trabalho.
\makeatletter
\hypersetup{pdftitle={\@title},
  pdfauthor={\@author},
  pdfsubject={\imprimirpreambulo},
  pdfkeywords={ABNT}{Trabalho acadêmico}{LaTeX}{ABNTeX2}
}
\makeatother
% INICIO DO DOCUMENTO
\begin{document}
\imprimircapa % Capa é obrigatória no ABNT (não use \mnaketitle)
% Observação: Lombada (parte de trás do livro) é opcional no ABNT e não tem comandos específicos para ele no abntex2.
% Folha de rosto é obrigatório no ABNT
% versao ``*'' não pula página, permitindo adicionar ficha catalográfica no verso
% Se não tiver a ficha catalográfica, use a versão sem ``*''
\imprimirfolhaderosto* % pule uma linha antes da ficha catalográfica.

% Ficha catalográfica, caso existir, deve ficar no verso da folha de rosto
\begin{fichacatalografica}
Ficha catalográfica provisória.
% Quando tiver versão definitiva, inserir aqui
% \includepdf{fichacatalograficafinal.pdf} 
\end{fichacatalografica}
% Errata, caso existir
\begin{errata}
Caso exista errata. Opcional no ABNT.
\end{errata}
% Folha de aprovação é obrigatório no ABNT
\begin{folhadeaprovacao}
% provisória: Veja o código no manual do abntex2.
% quando tiver a folha de aprovação devidamente assinada, incluir aqui
% \includepdf{folhadeaprovacaofinal.pdf} 
\end{folhadeaprovacao}
% Dedicatória
\begin{dedicatoria}
Dedicatória é um elemento opcional para ABNT.
\end{dedicatoria}
% Agradecimentos
\begin{agradecimentos}
Agradecimentos é opcional no ABNT, mas boa educação é agradecer aos que contribuíram para realização do trabalho.
\end{agradecimentos}
% Uma mensagem bonita
\begin{epigrafe}
Uma mensagem que é opcional no ABNT.
\end{epigrafe}
% Resumo
\begin{resumo}
Resumo em português, obrigatório no ABNT.

\noindent
\textbf{Palavras-chave}: ABNT, LaTeX, ABNTeX2.
\end{resumo}
% Resumo em linguagem estrangeira
\begin{resumo}[Abstract]
%[\MakeUppercase{Abstract}]
\begin{otherlanguage}{english}
Resumo em língua estrangeira (obrigatório no ABNT).

\noindent
\textbf{Keywords}: ABNT, LaTeX, ABNTeX2.
\end{otherlanguage}
\end{resumo}
%\clearpage\thispagestyle{empty}\cleardoublepage
% \phantomsection\pdfbookmark[chapter]{\listfigurename}{lof}
% \listoffigures % opcional no ABNT
%\clearpage\thispagestyle{empty}\cleardoublepage
%\phantomsection\pdfbookmark[chapter]{\listfigurename}{lot}
%\listoftables % opcional no ABNT
%\begin{siglas} % Lista de siglas é opcional no ABNT.
%\item[ABNT] Associação Brasileira de Normas Técnicas.
%\end{siglas}
%\begin{simbolos}% Lista de Símbolos é opcional no ABNT
%\item[$\pi$] Razão entre circunferência e raio.
%\end{simbolos}

% Sumário é obrigatório no ABNT
% A versão ``*'' remove a entrada "sumário" do bookmark do PDF e do sumário
\clearpage\thispagestyle{empty}\cleardoublepage
\tableofcontents*

% CONTEÚDO PRINCIPAL
\clearpage\thispagestyle{empty}\cleardoublepage
\mainmatter % ou \textual
\phantomsection%\pdfbookmark[chapter]{Prefácio}{prefacio}
\addcontentsline{toc}{chapter}{\MakeUppercase{Prefácio}}
\chapter*{Prefácio} % % prefácio (ou apresentação) é opcional no ABNT
Prefácio aqui. Se preferir, pode incorporar na introdução.
\clearpage\thispagestyle{empty}\cleardoublepage
\chapter{Introdução}
Introdução aqui.
\clearpage\thispagestyle{empty}\cleardoublepage
\chapter{Primeiro Capítulo}
Conteúdo do primeiro capítulo
\clearpage\thispagestyle{empty}\cleardoublepage
\chapter{Considerações Finais}
Conclusão do trabalho.
% PÓS TEXTUAL
\clearpage\thispagestyle{empty}\cleardoublepage
\backmatter % ou o comando \postextual 
%% Referencias bibliográficas é obrigatório no ABNT
%% Em geral, usa-se o BibLaTeX ou BibTeX
\begin{thebibliography}{99}
\bibitem{livro:primeiro} Primeira referência.
\end{thebibliography}
% \clearpage\thispagestyle{empty}\cleardoublepage
% Glossário é opcional no ABNT. 
% Apêndice é opcional no ABNT.
% Caso existir, inicio de apêndices será marcado com o comando \apendices
\clearpage\thispagestyle{empty}\cleardoublepage
\apendices % não usar o \appendix
\chapter{Primeiro Apêndice}
Conteúdo do primeiro apêndice.
\clearpage\thispagestyle{empty}\cleardoublepage
\chapter{Segundo Apêndice}
Conteúdo do segundo apêndice.
% Anexos são opcionais no ABNT
% Caso existir, inicio de anexos será marcado com o comando \anexos
\clearpage\thispagestyle{empty}\cleardoublepage
\anexos
\chapter{Primeiro Anexo}
Conteúdo do primeiro anexo
\chapter{Segundo Anexo}
Conteúdo do segundo anexo
% Índice remissivo (opcional)
% \clearpage\thispagestyle{empty}\cleardoublepage
% \printindex % Índice remissivo é opcional no ABNT
\end{document}
\end{lstlisting}
\end{singlespacing}

Note que a folha de aprovação provisória costuma estar presente no documento (é será substituída pela folha definitiva quando for aprovado).
Para criar uma folha de aprovação provisória adequada, sugerido no manual do \texttt{abntex2} \cite{book:abnt2:araujo} é
\begin{singlespacing}
\begin{lstlisting}
\begin{folhadeaprovacao}
% provisória
\begin{center}
{\ABNTEXchapterfont\large\imprimirautor}
\vspace*{\fill}\vspace*{\fill}
\begin{center}
\ABNTEXchapterfont\bfseries\Large\imprimirtitulo
\end{center}
\vspace*{\fill}
\hspace{.45\textwidth}
\begin{minipage}{.5\textwidth}
\imprimirpreambulo
\end{minipage}%
\vspace*{\fill}
\end{center}
Trabalho aprovado. \imprimirlocal, 24 de novembro de 2012:
\assinatura{\textbf{\imprimirorientador} \\ Orientador}
\assinatura{\textbf{Professor} \\ Convidado 1}
\assinatura{\textbf{Professor} \\ Convidado 2}
\assinatura{\textbf{Professor} \\ Convidado 3}
\assinatura{\textbf{Professor} \\ Convidado 4}
\begin{center}
\vspace*{0.5cm}
{\large\imprimirlocal}
\par
{\large\imprimirdata}
\vspace*{1cm}
\end{center}
% quando tiver a folha de aprovação devidamente assinada
% \includepdf{folhadeaprovacaofinal.pdf} 
\end{folhadeaprovacao}
\end{lstlisting}
\end{singlespacing}

Após ter a folha de aprovação devidamente assinadas, digitalize e converta no formato PDF.
Em seguida, inclua diretamente no documento como uma página (e não como um desenho) usando o comando \verb+\includepdf+\index{includepdf@\verb+\includepdf+} do pacote \texttt{pdfpages}\index{pdfpages}.

A classe \texttt{abntex2} aceita a opção \texttt{article} para diagramar artigos, mas na maioria dos casos, não é exigido que artigos sigam a norma ABNT, podendo ou não, requerer suas referências em ABNT.

\section{Usando o estilo ABNT no \BibTeX}

Quando as referências bibliográficas devem seguir a norma ABNT, use o pacote \texttt{abntex2cite}\index{abntex2cite} que pode ser usado separadamente da classe \texttt{abntex2}.
É importante lembrar que este pacote requer que o pacote \texttt{hyperref} seja carregado antes dele. 
Caso contrário, pode produzir erros, mas não costuma ser avisado de que \texttt{hyperref} precisa ser carregado antes.

A citação entre parentes deve ser feito com o comando \verb+\cite+, enquanto que a citação como parte de texto deve ser feito com o comando \verb+\citeonline+. As citações de citações (citações indiretas) são feitas pelos comandos \verb+\apud+ ou \verb+\apudonline+ para colocar entre parenteses ou fazer parte do texto, respectivamente.

Note que o pacote \texttt{abntex2cite} requer a codificação do documento em \texttt{utf8}.

Veja o Exemplo~\ref{ex19:abntex2cite}.

\begin{singlespacing}
\begin{lstlisting}[caption={ex19-abntex2cite.tex},label={ex19:abntex2cite}]
\documentclass[12pt,a4paper,english,brazil]{article}
\usepackage{iftex} % Para detectar o motor de TeX utilizado
\ifPDFTeX% Fontes, de acordo com o motor do TeX
  \usepackage[T1]{fontenc} % codificação da fonte em 8-bits
  \usepackage{lmodern} % fontes latin modern
  \usepackage[utf8]{inputenc} % acentuação direta em UTF-8
\else
  \usepackage{fontspec} % pacote para configurar fontes
  \defaultfontfeatures{Ligatures=TeX}
  % seleção de fontes, se desejar
  % ...
\fi
% pacotes matemáticas de acordo com o motor TeX
\ifPDFTeX
  \usepackage{amsmath,amssymb} % para matematica
\else
   % para matematica
  \usepackage{unicode-math}
  \ifLuaTeX
    \usepackage{lualatex-math}
  \fi
\fi
\usepackage{babel} % idiomas da opção da classe será usado aqui
\usepackage{hyperref}% abntex2cite requer hyoerref carregado antes dele
% Opção "alf" é para alfabetico. "num" para numérico.
\usepackage[alf]{abntex2cite} % Para citação no formato ABNT
\pdfstringdefDisableCommands{% desabilitando comandos proibidos no bookmarks
  \let\MakeUppercase\relax%
  \def\\{ }%
}
\begin{document}
Este modelo foi baseado no documento de \texttt{abntex2} \cite{book:abnt2:araujo}.

Para citações textuais, deverá usar o comando \verb+\citeonline+ como descrito em \citeonline{book:abnt2cite:araujo}.
Para citações de citações, usar o comando \verb+\apud+ ou \verb+\apudonline+.

Também poderá usar o \texttt{biblatex}. Veja o \citeonline{book:biblatex-abnt:marques} para detalhes.

% Especificacao de formatacao (\bibliographystyle) não será necessário, pois o pacote anbtex2cite já fez isso.
\bibliography{ex19-abntex2cite}
\end{document}
\end{lstlisting}
\end{singlespacing}

No arquivo \texttt{bib}, usar a acentuação no modo \TeX{}.
Para os documentos da internet, complete os campos \texttt{url} (endereço eletrônico) e \texttt{urlaccessdate} (data de última consulta).
Os campos \texttt{isbn} (identificador de livros) 
e \texttt{issn} (identificador de periódicos) também estão disponíveis.

Também observe que no ABNT, foram acrescentados categorias adicionais e por conta disso, o estilo \texttt{abnt} adicionou essas categorias que podem ser ignorados nos outros estilos de formatação. 
Veja \cite{book:abnt2cite:araujo} para detalhes.

O arquivo \texttt{ex19-abntex2cite.bib} é como segue

\begin{singlespacing}
\begin{lstlisting}[language={BibTeX}]
@book{book:abnt2:araujo,
author={Lauro C\'esar Araujo},
title={A classe abntex2: Documentos t\'ecnicos e cient{\'\i}ficos brasileiros compat{\'\i}veis com as normas {ABNT}},
url={https://ctan.org/pkg/abnt2/},
urlaccessdate={2018-06-11},
year=2016
}
@book{book:abnt2cite:araujo,
author={Lauro C\'esar Araujo},
title={O pacote abntex2cite: Estilos bibliogr\'aficos compat{\'\i}veis com a {ABNT} {NBR} 6023},
url={https://ctan.org/pkg/abnt2/},
urlaccessdate={2018-06-11},
year=2016
}
@book{book:biblatex-abnt:marques,
author={Daniel Ballester Marques},
title={biblatex-abnt 3.3},
url={https://ctan.org/pkg/biblatex-abnt/},
urlaccessdate={2018-06-11},
year=2018
}
\end{lstlisting}
\end{singlespacing}

A saída do Exemplo~\ref{ex19:abntex2cite} e algo como segue.

%\newlength{\labelseporiginal}
%\setlength{\labelseporiginal}{\labelsep}
\begin{ltxlistingout}
Este modelo foi baseado no documento de \texttt{abntex2} (ARAÚJO, 2018a).

Para citações textuais, deverá usar o comando \verb+\citeonline+ como descrito em Araújo (2016b).
Para citações de citações, usar o comando \verb+\apud+ ou \verb+\apudonline+.

Também poderá usar o \texttt{biblatex}. Veja o Marques (2018) para detalhes.

\section*{Referências}

\noindent
{ARAUJO, L.~C. \emph{A classe abntex2: Documentos t\'ecnicos e cient{\'\i}ficos
  brasileiros compat{\'\i}veis com as normas ABNT}. [s.n.], 2016.
  Dispon{\'\i}vel em: <\url{https://ctan.org/pkg/abnt2/}>. Acesso em:
  2018-06-11.}

\medskip
\noindent
{ARAUJO, L.~C. \emph{O pacote abntex2cite: Estilos bibliogr\'aficos
  compat{\'\i}veis com a ABNT NBR 6023}. [s.n.], 2016. Dispon{\'\i}vel em:
  <\url{https://ctan.org/pkg/abnt2/}>. Acesso em: 2018-06-11.}

\medskip
\noindent
{MARQUES, D.~B. \emph{biblatex-abnt 3.3}. [s.n.], 2018. Dispon{\'\i}vel em:
  <\url{https://ctan.org/pkg/biblatex-abnt/}>. Acesso em: 2018-06-11.}
\end{ltxlistingout}
%\setlength{\labelsep}{\labelseporiginal}

\section{Usando o estilo ABNT no Bib\LaTeX} 

Também poderá elaborar referências bibliográficas no estilo ABNT usando o Bib\LaTeX.

Para isso, coloque a opção \texttt{style=abnt}\index{biblatex!abnt} na opção de chamada do pacote \texttt{biblatex}.

A data de último acesso ao material disponível na internet será colocado no campo \texttt{urldate} do arquivo \texttt{.bib} no formato \texttt{ano-mes-dia} onde ano será com $4$ digitos. 
Note que \texttt{urlaccessdate} usado pelo \BibTeX{} também pode ser usado no Bib\LaTeX, mas \texttt{urldate} funcionam em vários estilos e não somente no estilo \texttt{abnt}.
Caso queira usar tanto o \BibTeX, assim como Bib\LaTeX, poderá especificar ambos os campos (e usar a acentuação no modo \TeX).

No estilo ABNT de \BibTeX{} do pacote \texttt{abntex2cite}, usa o \verb+\citeonline+ para citações textuais, mas no estilo ABNT de Bib\LaTeX, usa o comando \verb+\textcite+\index{textcite@\verb+\textcite+} já existentes no Bib\LaTeX.
Se no documento estiver usando o comando \verb+\citeonline+, basta criar um atalho no \texttt{preamble} do documento com 
\verb+\let\citeonline\textcite+.
Por outro lado, se quiser usar o \verb+\textcite+ com \texttt{abntex2cite}, prevendo a futura transição para BiB\LaTeX, 
basta colocar \verb+\let\textcite\citeonline+ no \texttt{preamble} do documento.

Novamente, use a codificação em \texttt{utf8} e note que o Bib\LaTeX{} aceita a acentuação direta no arquivo de \BibTeX. 
O uso de \texttt{hyperref} é opcional, mas recomendável.

Como referências bibliográficas no estilo ABNT do Bib\LaTeX{} está projetado para poder ser usado fora da classe \texttt{abntex2}, a formatação do título da referência bibliográfica não costuma aparecer de acordo com o \texttt{ABNT} mesmo na classe \texttt{abntex2} devido a algumas questões técnicas.
Isto pode ser acertado sem problemas com alguns códigos no \texttt{preâmbulo} como segue.

% Consertando o titulo que biblatex redefiniu
% Para biblatex: consertando para Referências ficar em maiusculo no sumário
% combinação de
% https://github.com/abntex/abntex2/issues/200
% e
% https://groups.google.com/forum/#!topic/abntex2/E9Bv9YhrjLk

\begin{singlespacing}
\begin{lstlisting}[language={TeX}]
% BibLaTeX com estilo abnt
\usepackage[style=abnt]{biblatex} 
\addbibresource{ex19-biblatex-abnt.bib}  % arquivo bib
% Consertando o titulo que biblatex redefiniu
\DefineBibliographyStrings{brazilian}{bibliography={Referências}}
% Consertando para Referências ficar em maiusculo no sumário
\defbibheading{bibliography}[\bibname]{%
    \chapter*{#1}
    \bibmark
    \ifnobibintoc\else
      \phantomsection
      \addcontentsline{toc}{chapter}{\uppercase{#1}}
    \fi
    \prebibhook
}
\end{lstlisting}
\end{singlespacing} 

O Exemplo~\ref{ex19:biblatex-abnt} ilustra como ficará o documento.

\begin{singlespacing}
\begin{lstlisting}[caption={ex19-biblatex-abnt.tex},label={ex19:biblatex-abnt}]
\documentclass[12pt,a4paper,english,brazil]{abntex2}
\usepackage{iftex}% Para detectar o motor de TeX utilizado
% Fontes, de acordo com o motor do TeX
\ifPDFTeX
  \usepackage[T1]{fontenc} % codificação da fonte em 8-bits
  \usepackage{lmodern} % fontes latin modern
  \usepackage[utf8]{inputenc} % acentuação direta em UTF-8
\else
  \usepackage{fontspec} % pacote para configurar fontes
  \defaultfontfeatures{Ligatures=TeX}
  % comandos de seleção das fontese, se desejar
  % ...
\fi
% pacotes matemáticas de acordo com o motor TeX
\ifPDFTeX
  \usepackage{amsmath,amssymb} % para matematica
\else
  \usepackage{unicode-math}
  \ifLuaTeX
    \usepackage{lualatex-math}
  \fi
\fi
% babel ou poligrossia é carregado automaticamente pelo abntex2
\usepackage{hyperref} % recomendável
\pdfstringdefDisableCommands{%desativar alguns comandos no bookmark
  \let\MakeUppercase\relax%
  \def\\{ }%
}
\usepackage[
  backend=biber, % aceita acentuação direta no arquivo bib (recomendável).
  style=abnt, % Sistema alfabético
  %style=abnt-numeric, % Sistema numérico
  %style=abnt-ibid, % Notas de referência
  % language=brazil, % padrão 
  % bibencoding=utf8, % padrão
]{biblatex}
% \citeonline como \textcite
\let\citeonline\textcite
\let\apudonline\textapud
\addbibresource{ex19-biblatex-abnt.bib}  % arquivo bib
% Consertando o titulo que biblatex redefiniu
\DefineBibliographyStrings{brazilian}{bibliography={Referências}}
% Para biblatex: consertando para Referências ficar em maiusculo no sumário
\defbibheading{bibliography}[\bibname]{%
    \chapter*{#1}
    \bibmark
    \ifnobibintoc\else
      \phantomsection
      \addcontentsline{toc}{chapter}{\uppercase{#1}}
    \fi
    \prebibhook
}
% Iniciando o docuemnto
\begin{document}
Este modelo foi baseado no documento de \texttt{abntex2} \cite{book:abnt2:araujo}.

Para citações textuais, deverá usar o comando \verb+\textcite+ como descrito em \textcite{book:biblatex-abnt:marques}.

% Para mais de uma citação por vez, use a versão plural \verb+\cites+ e \verb+\textcites+ (não é necessário).

Para citações de citações, usar o comando \verb+\apud+ ou \verb+\textapud+.

Existem vários outros comandos neste estilo. Veja o 
\textcite{book:biblatex-abnt:marques} para mais detalhes.

Também poderá usar o BibTeX em vez do BibLaTeX. Veja o \textcite{book:abnt2cite:araujo} para detalhes.
%
\phantomsection
% para que link do hyperref funcione corretamente para referencias bibliograficas no sumario e no bookmark. Parece não ser necessário na classe abntex2
\printbibliography%[heading=bibintoc]
% [heading=bibintoc] é para acrescentar a referências no sumário (caso tenha). Não é necessário para classe abntex2
\end{document}
\end{lstlisting}
\end{singlespacing}
  
O arquivo \texttt{ex19-biblatex-abnt.bib}  fica como segue.

\begin{singlespacing}
\begin{lstlisting}[language={BibTeX}]
@book{book:abnt2:araujo,
author={Lauro César Araujo},
title={A classe abntex2: Documentos técnicos e científicos brasileiros compatíveis com as normas {ABNT}},
url={https://ctan.org/pkg/abnt2/},
urldate={2018-06-12},
year=2016
}
@book{book:abnt2cite:araujo,
author={Lauro César Araujo},
title={O pacote abntex2cite: Estilos bibliográficos compatíveis com a {ABNT} {NBR} 6023},
url={https://ctan.org/pkg/abnt2/},
urldate={2018-06-12},
year=2016
}
@book{book:biblatex-abnt:marques,
author={Daniel Ballester Marques},
title={biblatex-abnt 3.3},
url={https://ctan.org/pkg/biblatex-abnt/},
urldate={2018-06-12},
year=2018
}
\end{lstlisting}
\end{singlespacing}

A saída do Exemplo~\ref{ex19:biblatex-abnt} e algo como segue.

%\newlength{\labelseporiginal}
%\setlength{\labelseporiginal}{\labelsep}
\begin{ltxlistingout}
Este modelo foi baseado no documento de \texttt{abntex2} (ARAÚJO, 2018a).

Para citações textuais, deverá usar o comando \verb+\textcite+ como descrito em Marques (2018).

% Para mais de uma citação por vez, use a versão plural \verb+\cites+ e \verb+\textcites+ (não é necessário).

Para citações de citações, usar o comando \verb+\apud+ ou \verb+\textapud+.

Existem vários outros comandos neste estilo. Veja o 
Marques (2018) para mais detalhes.

Também poderá usar o BibTeX em vez do BibLaTeX. Veja o Araújo (2016b) para detalhes.

%\setlength{\labelsep}{0pt}
\section*{Referências}

\noindent
{ARAUJO, L.~C. \emph{A classe abntex2: Documentos técnicos e científicos
  brasileiros compatíveis com as normas ABNT}. [s.n.], 2016.
  Disponível em: <\url{https://ctan.org/pkg/abnt2/}>. Acesso em:
  2018-06-12.}

\medskip
\noindent
{ARAUJO, L.~C. \emph{O pacote abntex2cite: Estilos bibliográficos
  compatíveis com a ABNT NBR 6023}. [s.n.], 2016. Disponível em:
  <\url{https://ctan.org/pkg/abnt2/}>. Acesso em: 2018-06-12.}

\medskip
\noindent
{MARQUES, D.~B. \emph{biblatex-abnt 3.3}. [s.n.], 2018. Disponível em:
  <\url{https://ctan.org/pkg/biblatex-abnt/}>. Acesso em: 2018-06-12.}
\end{ltxlistingout}
%\setlength{\labelsep}{\labelseporiginal}
  
  Para mais detalhes sobre comandos do estilo ABNT para Bib\LaTeX, veja \cite{book:biblatex-abnt:marques}.
  

%%%%%%%%%%%%%% APÊNDICE %%%%%%%%%%

\clearpage\thispagestyle{empty}\cleardoublepage
\appendix
\chapter{Símbolos Básicos de \LaTeX\protect\index{simbolos basicos@{símbolos básicos}}}
Aqui, veremos os símbolos básicos de \LaTeX{} que não requer o uso de pacotes adicionais. 

Em geral, os editores atuais costumam apresentar painel de inserção de símbolos, o que torna desnecessário ter lista de símbolos em mão.
Atualmente, muitos sistemas disponíveis na internet tais como alguns \texttt{blog}'s, \texttt{wiki}'s e sistema de ensino/aprendizagem \texttt{moodle} costumam ativar a fórmula \LaTeX{} através de \texttt{mimetex}, \texttt{mathjax} ou similar.
Alguns destes permitem somente os comandos e símbolos básicos sem os pacotes para deixar mais leve, enquanto que outros ativam os pacotes do \texttt{AMS} por padrão para incrementar a sua funcionalidade.

Aqui será apresentado somente os símbolos básicos que poderão ser usados em qualquer sistema na qual o suporte da fórmula \LaTeX{} esteja ativa. Para usar o sistema na qual os pacotes do \texttt{AMS} está ativa, é aconselhável procurar os símbolos e construções adicionais no outro material.

Existem pacotes específicos para cada área, adicionando símbolos e funcionalidades adicionais. Assim, é importante estudar pacotes relacionados à área desejada.
Por exemplo, na área de matemática, costuma usar os pacotes do AMS (\texttt{amsmath}, \texttt{amssymb}, \texttt{amsthm}).

Para procurar símbolos incomuns, existe uma lista completa de símbolos do \LaTeX{} \cite{book:pakin:2017} disponível gratuitamente.

\section{Caracteres especiais e acentuação no modo \TeX\protect\index{acentuacao@{acentuação}!modo TeX}}
Alguns caracteres são reservados no \LaTeX{} e requer o uso de comandos para inserir no documento. Além dos caracteres reservados, alguns caracteres extras (não \texttt{ASCII}) também podem ser produzidos.
Dependendo do caso, ainda precisa da acentuação pelo comando (acentuação no modo \TeX) como no caso de usa o \BibTeX{} na qual não consegue lidar com os caracteres acentuados.

O Exemplo~\ref{ex-a:especiais-acentuacao}%
\index{caracter especial!\textbackslash chaveaberta@\curlyopensymbforindex}%
\index{caracter especial!\textbackslash chavefechada@\curlyclosesymbforindex}%
%\index{\{}\index{\}}
\index{modo texto!simbolo@{símbolo}!\textbackslash chaveaberta@\curlyopensymbforindex}\index{modo matematico@{modo matemático}!basico@{básico}!\textbackslash chaveaberta@\curlyopensymbforindex}%
\index{modo texto!simbolo@{símbolo}!\textbackslash chavefechada@\curlyclosesymbforindex}\index{modo matematico@{modo matemático}!basico@{básico}!\textbackslash chavefechada@\curlyclosesymbforindex}%
%\index{\{}\index{\}}%
\index{modo texto!simbolo@{símbolo}!\verb+\$+, \$}\index{modo matematico@{modo matemático}!basico@{básico}!\verb+\$+, $\$$}%
\index{modo texto!simbolo@{símbolo}!\verb+\%+, \%}\index{modo matematico@{modo matemático}!basico@{básico}!\verb+\%+, $\%$}%
\index{modo texto!simbolo@{símbolo}!\verb+_+, \_{}}\index{modo matematico@{modo matemático}!basico@{básico}!\verb+_+, $\_{}$}%
\index{modo texto!simbolo@{símbolo}!\verb+\&+, \&}\index{modo matematico@{modo matemático}!basico@{básico}!\verb+\&+, $\&$}%
\index{modo texto!simbolo@{símbolo}!\verb+\#+, \#}\index{modo matematico@{modo matemático}!basico@{básico}!\verb+\#+, $\#$}%
\index{modo texto!simbolo@{símbolo}!\verb+\copyright+, \copyright}\index{modo matematico@{modo matemático}!basico@{básico}!\verb+\copyright+, $\copyright$}%
\index{modo texto!simbolo@{símbolo}!\verb+\dag+, \dag}\index{modo matematico@{modo matemático}!basico@{básico}!\verb+\dag+, $\dag$}%
\index{modo texto!simbolo@{símbolo}!\verb+\ddag+, \ddag}\index{modo matematico@{modo matemático}!basico@{básico}!\verb+\ddag+, $\ddag$}%
\index{modo texto!simbolo@{símbolo}!\verb+\dots+, \dots}\index{modo matematico@{modo matemático}!basico@{básico}!\verb+\dots+, $\dots$}%
\index{modo texto!simbolo@{símbolo}!\verb+\P+, \P}\index{modo matematico@{modo matemático}!basico@{básico}!\verb+\P+, $\P$}%
\index{modo texto!simbolo@{símbolo}!\verb+\pounds+, \pounds}\index{modo matematico@{modo matemático}!basico@{básico}!\verb+\pounds+, $\pounds$}%
\index{modo texto!simbolo@{símbolo}!\verb+\S+, \S}\index{modo matematico@{modo matemático}!basico@{básico}!\verb+\S+, $\S$}%
%
\index{modo texto!caracter especial!\verb+\aa+, \aa}\index{modo texto!caracter especial!\verb+\AA+, \AA}%
\index{modo texto!caracter especial!\verb+\ae+, \ae}\index{modo texto!caracter especial!\verb+\AE+, \AE}%
\index{modo texto!caracter especial!\verb+\ij+, \ij}\index{modo texto!caracter especial!\verb+\IJ+, \IJ}%
\index{modo texto!caracter especial!\verb+\l+, \l}\index{modo texto!caracter especial!\verb+\L+, \L}%
\index{modo texto!caracter especial!\verb+\o+, \o}\index{modo texto!caracter especial!\verb+\O+, \O}%
\index{modo texto!caracter especial!\verb+\oe+, \oe}\index{modo texto!caracter especial!\verb+\OE+, \OE}%
\index{modo texto!caracter especial!\verb+\ss+, \ss}\index{modo texto!caracter especial!\verb+\SS+, \SS}%
%
\index{modo texto!acentuacao@{acentuação}}%
\index{modo texto!letra circulada, \verb+\textcircled{a}+, \textcircled{a}}%
\index{modo texto!acentuacao@{acentuação}!i sem pingo,\verb+\i+,\i}%
\index{modo texto!acentuacao@{acentuação}!j sem pingo,\verb+\j+,\j}%
\index{modo texto!apostrofos@{apóstrofos}}%
\index{modo texto!aspas}%
\index{modo texto!acentuacao@{acentuação}!\verb+\\verb+\"{a}+, +, \"{a}}%
\index{modo texto!acentuacao@{acentuação}!\verb+\'{a}+, \'{a}}%
\index{modo texto!acentuacao@{acentuação}!\verb+\.{a}+, \.{a}}%
\index{modo texto!acentuacao@{acentuação}!\verb+\={a}+, \={a}}%
\index{modo texto!acentuacao@{acentuação}!\verb+\^{a}+, \^{a}}%
\index{modo texto!acentuacao@{acentuação}!\verb+\`{a}+, \`{a}}%
\index{modo texto!acentuacao@{acentuação}!\verb+\~{a}+, \~{a}}%
\index{modo texto!acentuacao@{acentuação}!\verb+\b{a}+, \b{a}}%
\index{modo texto!acentuacao@{acentuação}!\verb+\c{c}+, \c{c}}%
\index{modo texto!acentuacao@{acentuação}!\verb+\d{a}+, \d{a}}%
\index{modo texto!acentuacao@{acentuação}!\verb+\H{a}+, \H{a}}%
\index{modo texto!acentuacao@{acentuação}!\verb+\r{a}+, \r{a}}%
\index{modo texto!acentuacao@{acentuação}!\verb+\t{a}+, \t{a}}%
\index{modo texto!acentuacao@{acentuação}!\verb+\u{a}+, \u{a}}%
\index{modo texto!acentuacao@{acentuação}!\verb+\v{a}+, \v{a}}%
 lista estes comandos que devem funcionar para qualquer configuração.

%\begin{singlespacing}
%\begin{LTXexample}[pos=b,caption={ex02-01.tex},label=ex:02-01]
\begin{ltxlisting}[caption={ex-a-especiais-acentuacao.tex},label=ex-a:especiais-acentuacao]
Caracteres especiais (modo texto e matemático)

\{  \}  \$  \%  \_  \&  \#

\copyright \dag \ddag \dots \P \pounds \S
 
% \ também é caractere especial, mas usa o comando diferente para o modo texto (\textbackslash) e modo matemático (\backslash).

Caracteres não \texttt{ASCII} (modo texto)

\aa  \AA  \ae \AE

\ij \IJ

\l \L  \o \O

\oe  \OE  \ss  \SS

Acentuação no modo texto

\"{A}\"{a}
\'{A}\'{a}
\.{A}\.{a}
\={A}\={a}
\^{A}\^{a}
\`{A}\`{a}

\~{A}\~{a}
\b{A}\b{a}
\c{C}\c{c}
\d{A}\d{a}
\H{A}\H{a}
\r{A}\r{a}
\t{A}\t{a}
\u{A}\u{a}
\v{A}\v{a}

Circulando a letra

\textcircled{A}\textcircled{a}

i e j sem pingo

\i  \j

Apóstrofos e aspas

` (ou ^^ ) e '

`` (ou ^^ ^^ ) e ''

Pontos para abreviaturas, colocar espaço forçado \verb*+\ + depois dele
para distinguir do ponto final como em ``i.\ é.\ ''.

O ponto sem espaço depois dele é inserido com \verb+\@.+ como em ``p\@.v\@.i\@.''.

\end{ltxlisting}
%\end{LTXexample}
%\end{singlespacing}

Note que, para abrir apóstrofo\index{modo texto!apostrofo@{apóstrofo}} (ou aspas\index{modo texto!aspas}), usa-se um (ou dois) acentos agudos e para fechar, usa-se um (ou dois) apóstrofos. Na falta de acento crase no teclado, poderá substituir cada um dos acentos crase por dois acentos circunflexos seguidos de um espaço.

Os comandos \verb+\i+ e \verb+\j+ produz i e j sem pingo para ser acentuados. Por exemplo, ``sa\'{\i}da''. 
No LaTeX atual, devem produzir corretamente mesmo sem usar \verb+\i+ como em ``sa\'{i}da''.

\section{Símbolos no modo texto\protect\index{simbolos@{símbolos}!modo texto}}
Alguns comandos dos símbolos funcionam tanto no modo matemático, como no modo texto, mas maioria dos comandos funcionam somente no modo texto ou no modo matemático.

O Exemplo~\ref{ex-a:simbolos-texto}%
\index{modo texto!simbolo@{símbolo}!\textbackslash chaveaberta@\curlyopensymbforindex}\index{modo matematico@{modo matemático}!basico@{básico}!\textbackslash chaveaberta@{$\curlyopensymbforindex$}}%
\index{modo texto!simbolo@{símbolo}!\textbackslash chavefechada@\curlyclosesymbforindex}\index{modo matematico@{modo matemático}!basico@{básico}!\textbackslash chavefechada@{$\curlyclosesymbforindex$}}%
%\index{\{}\index{\}}%
\index{modo texto!simbolo@{símbolo}!\verb+\$+, \$}\index{modo matematico@{modo matemático}!basico@{básico}!\verb+\$+, $\$$}%
\index{modo texto!simbolo@{símbolo}!\verb+\%+, \%}\index{modo matematico@{modo matemático}!basico@{básico}!\verb+\%+, $\%$}%
\index{modo texto!simbolo@{símbolo}!\verb+_+, \_{}}\index{modo matematico@{modo matemático}!basico@{básico}!\verb+_+, $\_{}$}%
\index{modo texto!simbolo@{símbolo}!\verb+\&+, \&}\index{modo matematico@{modo matemático}!basico@{básico}!\verb+\&+, $\&$}%
\index{modo texto!simbolo@{símbolo}!\verb+\#+, \#}\index{modo matematico@{modo matemático}!basico@{básico}!\verb+\#+, $\#$}%
\index{modo texto!simbolo@{símbolo}!\verb+\copyright+, \copyright}\index{modo matematico@{modo matemático}!basico@{básico}!\verb+\copyright+, $\copyright$}%
\index{modo texto!simbolo@{símbolo}!\verb+\dag+, \dag}\index{modo matematico@{modo matemático}!basico@{básico}!\verb+\dag+, $\dag$}%
\index{modo texto!simbolo@{símbolo}!\verb+\ddag+, \ddag}\index{modo matematico@{modo matemático}!basico@{básico}!\verb+\ddag+, $\ddag$}%
\index{modo texto!simbolo@{símbolo}!\verb+\dots+, \dots}\index{modo matematico@{modo matemático}!basico@{básico}!\verb+\dots+, $\dots$}%
\index{modo texto!simbolo@{símbolo}!\verb+\P+, \P}\index{modo matematico@{modo matemático}!basico@{básico}!\verb+\P+, $\P$}%
\index{modo texto!simbolo@{símbolo}!\verb+\pounds+, \pounds}\index{modo matematico@{modo matemático}!basico@{básico}!\verb+\pounds+, $\pounds$}%
\index{modo texto!simbolo@{símbolo}!\verb+\S+, \S}\index{modo matematico@{modo matemático}!basico@{básico}!\verb+\S+, $\S$}%
%
\index{modo texto!simbolo@{símbolo}!\verb+\textasciicircum+, \textasciicircum}%
\index{modo texto!simbolo@{símbolo}!\verb+\textasciitilde+, \textasciitilde}%
\index{modo texto!simbolo@{símbolo}!\verb+\textasteriskcentered+, \textasteriskcentered}%
\index{modo texto!simbolo@{símbolo}!\verb+\textbackslash+, \textbackslash}%
\index{modo texto!simbolo@{símbolo}!\verb+\textbar+, \textbar}%
\index{modo texto!simbolo@{símbolo}!\verb+\textbardbl+, \textbardbl}%
\index{modo texto!simbolo@{símbolo}!\verb+\textbullet+, \textbullet}%
\index{modo texto!simbolo@{símbolo}!\verb+\textemdash+, \textemdash}%
\index{modo texto!simbolo@{símbolo}!\verb+\textendash+, \textendash}%
\index{modo texto!simbolo@{símbolo}!\verb+\textexclamdown+, \textexclamdown}%
\index{modo texto!simbolo@{símbolo}!\verb+\textquestiondown+, \textquestiondown}%
\index{modo texto!simbolo@{símbolo}!\verb+\textordfeminine+, \textordfeminine}%
\index{modo texto!simbolo@{símbolo}!\verb+\textordmasculine+, \textordmasculine}%
\index{modo texto!simbolo@{símbolo}!\verb+\textperiodcentered+, \textperiodcentered}%
%\index{\textquotedblleft}\index{\verb+\textquotedbright+, \textquotedblright}%
%\index{\textquoteleft}\index{\verb+\textquoteright+, \textquoteright}%
\index{modo texto!simbolo@{símbolo}!\verb+\textregistered+, \textregistered}%
\index{modo texto!simbolo@{símbolo}!\verb+\texttrademark+, \texttrademark}%
\index{modo texto!simbolo@{símbolo}!\verb+\textvisiblespace+, \textvisiblespace}%
 lista comandos que funcionam no modo texto (incluindo os que funcionam tanto no modo texto como no modo matemático).

%\begin{singlespacing}
%\begin{LTXexample}[pos=b,caption={ex02-01.tex},label=ex:02-01]
\begin{ltxlisting}[caption={ex-a-simbolos-texto.tex},label=ex-a:simbolos-texto]
Comandos que funcionam tanto no modo texto como no modo matemático.

\{  \}  \$  \%  \_  \&  \#

\copyright  \dag  \ddag  \dots

\P  \pounds  \S

Comando que funcionam somente no modo texto.

%\textasciicircum % mesmo que \^{}
%\textasciitilde % mesmo que \~{}
\textasteriskcentered
\textbackslash
\textbar
\textbardbl
%\textbraceleft % use \{
%\textbraceright % use \}
\textbullet
%\textcopyright % use \copyright
%\textdagger % use \dag
%\textdaggerdbl % use ddag
%\textdollar % use \$
%\textellipsis % use \dots

---
--
%\textemdash{} % mesmo que ---
%\textendash % mesmo que --

!`?`
%\textexclamdown % mesmo que !`
%\textquestiondown % mesmo que ?`

%\textgreater % use >
%\textless % use <

\textordfeminine
\textordmasculine
%\textparagraph % use \P
\textperiodcentered

%\textquotedblleft %use ``
%\textquotedblright % use ''
%\textquoteleft % use `
%\textquoteright % use '

\textregistered
%\textsection % use \S
%\textsterling % use \pounds
\texttrademark
%\textunderscore % use \_
\textvisiblespace
\end{ltxlisting}
%\end{LTXexample}
%\end{singlespacing}

Pontos de exclamação e interrogação invertidos são obtidos, colocando acento crase após eles. Na falta do acento crase no teclado, existem os comandos \verb+\textexclamdown+ e \verb+\textquestiondown+ para produzir eles.

Existem muitos outros símbolos que podem ser acessados no modo texto com o uso do pacote \texttt{textcomp}\index{textcomp} que faz parte da base do sistema \TeX.

\section{Símbolos matemáticos\protect\index{simbolos@{símbolos}!matematicos@{matemáticos}}}

Mo modo matemático, algumas construções básicas\index{modo matematico@{modo matemático}!construcao basica@{construção básica}} estão disponíveis, como ilustrado no Exemplo~\ref{ex-a:math-construcao}%
\index{modo matematico@{modo matemático}!construcao basica@{construção básica}!\verb+^+}%
\index{modo matematico@{modo matemático}!construcao basica@{construção básica}!expoente}%
\index{modo matematico@{modo matemático}!construcao basica@{construção básica}!\verb+_+}%
\index{modo matematico@{modo matemático}!construcao basica@{construção básica}!indice@{índice}}%
\index{modo matematico@{modo matemático}!construcao basica@{construção básica}!'@{$'$}}%
\index{derivada}\index{modo matematico@{modo matemático}!construcao basica@{construção básica}!frac@{\verb+\frac+}}%
\index{modo matematico@{modo matemático}!construcao basica@{construção básica}!fracao@{fração}}%
\index{modo matematico@{modo matemático}!construcao basica@{construção básica}!sqrt@{\verb+\sqrt+}}%
\index{modo matematico@{modo matemático}!construcao basica@{construção básica}!raiz}%
\index{modo matematico@{modo matemático}!construcao basica@{construção básica}!stackrel@{\verb+\stackrel+}}%
\index{modo matematico@{modo matemático}!construcao basica@{construção básica}!atop@{\verb+\atop+}}%
.
Para construções mais complexas, deve considerar o uso do pacote \texttt{amsmath} de \texttt{AMS} que faz parte da base do sistema \TeX.

%\begin{singlespacing}
%\begin{LTXexample}[pos=b,caption={ex02-01.tex},label=ex:02-01]
\begin{ltxlisting}[caption={ex-a-construcao.tex},label=ex-a:math-construcao]
Algumas construções básicas
$
x^{n} % expoente
x_{i+1} % índice
x_i^{n+1} % índice e expoente
f' % derivada de f
\frac{m}{n} % fração
\sqrt{2} % raiz quadrada
\sqrt[n]{x} % raiz n-ésima
\stackrel{f}{\to} % colocando nome encima
{X \atop Y} % empilhando
{\displaystyle\mathop{X}_i^j} % índice encima/embaixo
$
\end{ltxlisting}
%\end{LTXexample}
%\end{singlespacing}

O comando \verb+\mathop+\index{modo matematico@{modo matemático}!construcao basica@{construção básica}!mathop@{\verb+\mathop+}}
 permite colocar elementos embaixo/encima em vez de índice/expoente no modo \texttt{display}. 
Para usar temporariamente o modo \texttt{display} dentro do modo \texttt{inline}, basta usar o comando \verb+\display+\index{display@{\verb+\display+}}. 

Existem caracteres que podem ser usados diretamente na qual o \LaTeX{} substitui diretamente com o símbolo adequado. Estes símbolos, juntamente com os símbolos que funcionam também no modo texto (apresentado também no Exemplo~\ref{ex-a:simbolos-texto}) estão no Exemplo~\ref{ex-a:math-basic}%
\index{modo texto!simbolo@{símbolo}!\textbackslash chaveaberta@\curlyopensymbforindex}\index{modo matematico@{modo matemático}!basico@{básico}!\textbackslash chaveaberta@{$\curlyopensymbforindex$}}%
\index{modo texto!simbolo@{símbolo}!\textbackslash chavefechada@\curlyclosesymbforindex}\index{modo matematico@{modo matemático}!basico@{básico}!\textbackslash chavefechada@{$\curlyclosesymbforindex$}}%
%\index{\{}\index{\}}%
\index{modo texto!simbolo@{símbolo}!\verb+\$+, \$}\index{modo matematico@{modo matemático}!basico@{básico}!\verb+\$+, $\$$}%
\index{modo texto!simbolo@{símbolo}!\verb+\%+, \%}\index{modo matematico@{modo matemático}!basico@{básico}!\verb+\%+, $\%$}%
\index{modo texto!simbolo@{símbolo}!\verb+_+, \_{}}\index{modo matematico@{modo matemático}!basico@{básico}!\verb+_+, $\_{}$}%
\index{modo texto!simbolo@{símbolo}!\verb+\&+, \&}\index{modo matematico@{modo matemático}!basico@{básico}!\verb+\&+, $\&$}%
\index{modo texto!simbolo@{símbolo}!\verb+\#+, \#}\index{modo matematico@{modo matemático}!basico@{básico}!\verb+\#+, $\#$}%
\index{modo texto!simbolo@{símbolo}!\verb+\copyright+, \copyright}\index{modo matematico@{modo matemático}!basico@{básico}!\verb+\copyright+, $\copyright$}%
\index{modo texto!simbolo@{símbolo}!\verb+\dag+, \dag}\index{modo matematico@{modo matemático}!basico@{básico}!\verb+\dag+, $\dag$}%
\index{modo texto!simbolo@{símbolo}!\verb+\ddag+, \ddag}\index{modo matematico@{modo matemático}!basico@{básico}!\verb+\ddag+, $\ddag$}%
\index{modo texto!simbolo@{símbolo}!\verb+\dots+, \dots}\index{modo matematico@{modo matemático}!basico@{básico}!\verb+\dots+, $\dots$}%
\index{modo texto!simbolo@{símbolo}!\verb+\P+, \P}\index{modo matematico@{modo matemático}!basico@{básico}!\verb+\P+, $\P$}%
\index{modo texto!simbolo@{símbolo}!\verb+\pounds+, \pounds}\index{modo matematico@{modo matemático}!basico@{básico}!\verb+\pounds+, $\pounds$}%
\index{modo texto!simbolo@{símbolo}!\verb+\S+, \S}\index{modo matematico@{modo matemático}!basico@{básico}!\verb+\S+, $\S$}%
%
\index{modo matematico@{modo matemático}!basico@{básico}!$+$}%
\index{modo matematico@{modo matemático}!basico@{básico}!$-$}%
\index{modo matematico@{modo matemático}!basico@{básico}!$/$}%
\index{modo matematico@{modo matemático}!basico@{básico}!$<$}%
\index{modo matematico@{modo matemático}!basico@{básico}!$>$}%
\index{modo matematico@{modo matemático}!basico@{básico}!$=$}%
\index{modo matematico@{modo matemático}!basico@{básico}!$,$}%
\index{modo matematico@{modo matemático}!basico@{básico}!$.$}%
\index{modo matematico@{modo matemático}!basico@{básico}!$;$}%
\index{modo matematico@{modo matemático}!basico@{básico}!\exclsymbforindex}%
.

%\begin{singlespacing}
%\begin{LTXexample}[pos=b,caption={ex02-01.tex},label=ex:02-01]
\begin{ltxlisting}[caption={ex-a-simbolos-basicos.tex},label=ex-a:math-basic]
Comandos que funcionam tanto no modo texto como no modo matemático.

$\{  \}  \$  \%  \_  \&  \#$
~
$\copyright  \dag  \ddag  \dots$
~
$\P  \pounds  \S$

Símbolos que podem ser usados diretamente

$+  -  :  /  <  >  =  ,  .  ;  !$
$:$ é para relação (para pontuação, use $\colon$)
\end{ltxlisting}
%\end{LTXexample}
%\end{singlespacing}

As letras gregas\index{letras gregas} maiúsculas estão disponíveis somente uma parte dela, como pode ser observado no Exemplo~~\ref{ex-a:math-greek}%
\index{modo matematico@{modo matemático}!grega!\verb+\alpha+, $\alpha$}%
\index{modo matematico@{modo matemático}!grega!\verb+\beta+, $\beta$}%
\index{modo matematico@{modo matemático}!grega!\verb+\gamma+, $\gamma$}%
\index{modo matematico@{modo matemático}!grega!\verb+\delta+, $\delta$}%
\index{modo matematico@{modo matemático}!grega!\verb+\epsilon+, $\epsilon$}%
\index{modo matematico@{modo matemático}!grega!\verb+\varepsilon+, $\varepsilon$}%
\index{modo matematico@{modo matemático}!grega!\verb+\zeta+, $\zeta$}%
\index{modo matematico@{modo matemático}!grega!\verb+\eta+, $\eta$}%
\index{modo matematico@{modo matemático}!grega!\verb+\theta+, $\theta$}%
\index{modo matematico@{modo matemático}!grega!\verb+\vartheta+, $\vartheta$}%
\index{modo matematico@{modo matemático}!grega!\verb+\iota+, $\iota$}%
\index{modo matematico@{modo matemático}!grega!\verb+\kappa+, $\kappa$}%
\index{modo matematico@{modo matemático}!grega!\verb+\lambda+, $\lambda$}%
\index{modo matematico@{modo matemático}!grega!\verb+\mu+, $\mu$}%
\index{modo matematico@{modo matemático}!grega!\verb+\nu+, $\nu$}%
\index{modo matematico@{modo matemático}!grega!\verb+\xi+, $\xi$}%
\index{modo matematico@{modo matemático}!grega!$o$}%
\index{modo matematico@{modo matemático}!grega!\verb+\pi+, $\pi$}%
\index{modo matematico@{modo matemático}!grega!\verb+\varphi+, $\varpi$}%
\index{modo matematico@{modo matemático}!grega!\verb+\rho+, $\rho$}%
\index{modo matematico@{modo matemático}!grega!\verb+\varrho+, $\varrho$}%
\index{modo matematico@{modo matemático}!grega!\verb+\sigma+, $\sigma$}%
\index{modo matematico@{modo matemático}!grega!\verb+\varsigma+, $\varsigma$}%
\index{modo matematico@{modo matemático}!grega!\verb+\tau+, $\tau$}%
\index{modo matematico@{modo matemático}!grega!\verb+\upsilon+, $\upsilon$}%
\index{modo matematico@{modo matemático}!grega!\verb+\phi+, $\phi$}%
\index{modo matematico@{modo matemático}!grega!\verb+\varphi+, $\varphi$}%
\index{modo matematico@{modo matemático}!grega!\verb+\chi+, $\chi$}%
\index{modo matematico@{modo matemático}!grega!\verb+\psi+, $\psi$}%
\index{modo matematico@{modo matemático}!grega!\verb+\omega+, $\omega$}%
%
\index{modo matematico@{modo matemático}!grega maiuscula@{grega maiúscula}!\verb+\Gamma+, $\Gamma$}%
\index{modo matematico@{modo matemático}!grega maiuscula@{grega maiúscula}!\verb+\Delta+, $\Delta$}%
\index{modo matematico@{modo matemático}!grega maiuscula@{grega maiúscula}!\verb+\Theta+, $\Theta$}%
\index{modo matematico@{modo matemático}!grega maiuscula@{grega maiúscula}!\verb+\Lambda+, $\Lambda$}%
\index{modo matematico@{modo matemático}!grega maiuscula@{grega maiúscula}!\verb+\Xi+, $\Xi$}%
\index{modo matematico@{modo matemático}!grega maiuscula@{grega maiúscula}!\verb+\Pi+, $\Pi$}%
\index{modo matematico@{modo matemático}!grega maiuscula@{grega maiúscula}!\verb+\Sigma+, $\Sigma$}%
\index{modo matematico@{modo matemático}!grega maiuscula@{grega maiúscula}!\verb+\Upsilon+, $\Upsilon$}%
\index{modo matematico@{modo matemático}!grega maiuscula@{grega maiúscula}!\verb+\Phi+, $\Phi$}%
\index{modo matematico@{modo matemático}!grega maiuscula@{grega maiúscula}!\verb+\Psi+, $\Psi$}%
\index{modo matematico@{modo matemático}!grega maiuscula@{grega maiúscula}!\verb+\Omega+, $\Omega$}%
%
 que lista estes comandos.

%\begin{singlespacing}
%\begin{LTXexample}[pos=b,caption={ex02-01.tex},label=ex:02-01]
\begin{ltxlisting}[caption={ex-a-grega.tex},label=ex-a:math-greek]
Letra grega minúscula
$\alpha  \beta  \gamma  \delta  \epsilon  \varepsilon$
~
$\zeta  \eta  \theta  \vartheta  \iota  \kappa  \lambda$
~
$\mu  \nu  \xi  o  \pi  \varpi  \rho  \varrho  \sigma \varsigma$
~
$\tau  \upsilon  \phi  \varphi  \chi  \psi  \omega$

Note que \texttt{omikron} não tem comando correspondente por ser mesmo símbolo que o ``o'' minúsculo.

Letra grega maiúscula (muitos deles são mesmos da letra romana maiúscula, não tendo comandos correspondentes)

$\Gamma  \Delta  \Theta  \Lambda  \Xi  \Pi$
~
$\Sigma  \Upsilon  \Phi  \Psi  \Omega$
\end{ltxlisting}
%\end{LTXexample}
%\end{singlespacing}

Agora, veja os operadores binários\index{operador binário} e relações binária\index{relação binária} no Exemplo~\ref{ex-a:math-binary}.%
\index{modo matematico@{modo matemático}!binario@{binário}!\verb+\amalg+, $\amalg$}%
\index{modo matematico@{modo matemático}!binario@{binário}!\verb+\ast+, $\ast$}%
\index{modo matematico@{modo matemático}!binario@{binário}!\verb+\bigcirc+, $\bigcirc$}%
\index{modo matematico@{modo matemático}!binario@{binário}!\verb+\bigtriangledown+, $\bigtriangledown$}%
\index{modo matematico@{modo matemático}!binario@{binário}!\verb+\bigtriangleup+, $\bigtriangleup$}%
\index{modo matematico@{modo matemático}!binario@{binário}!\verb+\bullet+, $\bullet$}%
\index{modo matematico@{modo matemático}!binario@{binário}!\verb+\cap+, $\cap$}%
\index{modo matematico@{modo matemático}!binario@{binário}!\verb+\cdot+, $\cdot$}%
\index{modo matematico@{modo matemático}!binario@{binário}!\verb+\circ+, $\circ$}%
\index{modo matematico@{modo matemático}!binario@{binário}!\verb+\cup+, $\cup$}%
\index{modo matematico@{modo matemático}!binario@{binário}!\verb+\dagger+, $\dagger$}%
\index{modo matematico@{modo matemático}!binario@{binário}!\verb+\diamond+, $\diamond$}%
\index{modo matematico@{modo matemático}!binario@{binário}!\verb+\div+, $\div$}%
\index{modo matematico@{modo matemático}!binario@{binário}!\verb+\mp+, $\mp$}%
\index{modo matematico@{modo matemático}!binario@{binário}!\verb+\odot+, $\odot$}%
\index{modo matematico@{modo matemático}!binario@{binário}!\verb+\ominus+, $\ominus$}%
\index{modo matematico@{modo matemático}!binario@{binário}!\verb+\oplus+, $\oplus$}%
\index{modo matematico@{modo matemático}!binario@{binário}!\verb+\oslash+, $\oslash$}%
\index{modo matematico@{modo matemático}!binario@{binário}!\verb+\otimes+, $\otimes$}%
\index{modo matematico@{modo matemático}!binario@{binário}!\verb+\pm+, $\pm$}%
\index{modo matematico@{modo matemático}!binario@{binário}!\verb+\setminus+, $\setminus$}%
\index{modo matematico@{modo matemático}!binario@{binário}!\verb+\sqcap+, $\sqcap$}%
\index{modo matematico@{modo matemático}!binario@{binário}!\verb+\sqcup+, $\sqcup$}%
\index{modo matematico@{modo matemático}!binario@{binário}!\verb+\times+, $\times$}%
\index{modo matematico@{modo matemático}!binario@{binário}!\verb+\triangleleft+, $\triangleleft$}%
\index{modo matematico@{modo matemático}!binario@{binário}!\verb+\triangleright+, $\triangleright$}%
\index{modo matematico@{modo matemático}!binario@{binário}!\verb+\uplus+, $\uplus$}%
\index{modo matematico@{modo matemático}!binario@{binário}!\verb+\vee+, $\vee$}%
\index{modo matematico@{modo matemático}!binario@{binário}!\verb+\wedge+, $\wedge$}%
\index{modo matematico@{modo matemático}!binario@{binário}!\verb+\wr+, $\wr$}%
%
\index{modo matematico@{modo matemático}!binario ajustavel@{binário ajustável}!\verb+\bigcap+, $\bigcap$}%
\index{modo matematico@{modo matemático}!binario ajustavel@{binário ajustável}!\verb+\bigotimes+, $\bigotimes$}%
\index{modo matematico@{modo matemático}!binario ajustavel@{binário ajustável}!\verb+\bigwedge+, $\bigwedge$}%
\index{modo matematico@{modo matemático}!binario ajustavel@{binário ajustável}!\verb+\bigcup+, $\bigcup$}%
\index{modo matematico@{modo matemático}!binario ajustavel@{binário ajustável}!\verb+\bigsqcup+, $\bigsqcup$}%
\index{modo matematico@{modo matemático}!binario ajustavel@{binário ajustável}!\verb+\coprod+, $\coprod$}%
\index{modo matematico@{modo matemático}!binario ajustavel@{binário ajustável}!\verb+\bigodot+, $\bigodot$}%
\index{modo matematico@{modo matemático}!binario ajustavel@{binário ajustável}!\verb+\biguplus+, $\biguplus$}%
\index{modo matematico@{modo matemático}!binario ajustavel@{binário ajustável}!\verb+\int+, $\int$}%
\index{modo matematico@{modo matemático}!binario ajustavel@{binário ajustável}!\verb+\bigoplus+, $\bigoplus$}%
\index{modo matematico@{modo matemático}!binario ajustavel@{binário ajustável}!\verb+\bigvee+, $\bigvee$}%
\index{modo matematico@{modo matemático}!binario ajustavel@{binário ajustável}!\verb+\oint+, $\oint$}%
\index{modo matematico@{modo matemático}!binario ajustavel@{binário ajustável}!\verb+\sum+, $\sum$}%
\index{modo matematico@{modo matemático}!binario ajustavel@{binário ajustável}!\verb+\prod+, $\prod$}%
%
\index{modo matematico@{modo matemático}!relacao@{relação}!\verb+\approx+, $\approx$}%
\index{modo matematico@{modo matemático}!relacao@{relação}!\verb+\asymp+, $\asymp$}%
\index{modo matematico@{modo matemático}!relacao@{relação}!\verb+\bowtie+, $\bowtie$}%
\index{modo matematico@{modo matemático}!relacao@{relação}!\verb+\cong+, $\cong$}%
\index{modo matematico@{modo matemático}!relacao@{relação}!\verb+\doteq+, $\doteq$}%
\index{modo matematico@{modo matemático}!relacao@{relação}!\verb+\equiv+, $\equiv$}%
\index{modo matematico@{modo matemático}!relacao@{relação}!\verb+\frown+, $\frown$}%
\index{modo matematico@{modo matemático}!relacao@{relação}!\verb+\mid+, $\mid$}%
\index{modo matematico@{modo matemático}!relacao@{relação}!\verb+\models+, $\models$}%
\index{modo matematico@{modo matemático}!relacao@{relação}!\verb+\parallel+, $\parallel$}%
\index{modo matematico@{modo matemático}!relacao@{relação}!\verb+\perp+, $\perp$}%
\index{modo matematico@{modo matemático}!relacao@{relação}!\verb+\prec+, $\prec$}%
\index{modo matematico@{modo matemático}!relacao@{relação}!\verb+\preceq+, $\preceq$}%
\index{modo matematico@{modo matemático}!relacao@{relação}!\verb+\propto+, $\propto$}%
\index{modo matematico@{modo matemático}!relacao@{relação}!\verb+\sim+, $\sim$}%
\index{modo matematico@{modo matemático}!relacao@{relação}!\verb+\simeq+, $\simeq$}%
\index{modo matematico@{modo matemático}!relacao@{relação}!\verb+\smile+, $\smile$}%
\index{modo matematico@{modo matemático}!relacao@{relação}!\verb+\succ+, $\succ$}%
\index{modo matematico@{modo matemático}!relacao@{relação}!\verb+\succeq+, $\succeq$}%
%
 que requer comandos.
%\begin{singlespacing}
%\begin{LTXexample}[pos=b,caption={ex02-01.tex},label=ex:02-01]
\begin{ltxlisting}[caption={ex-a-simbolo-binario.tex},label=ex-a:math-binary]
Operadores binários

$\amalg  \ast  \bigcirc  \bigtriangledown  \bigtriangleup$
~
$\bullet  \cap  \cdot  \circ  \cup  \dagger  \ddagger  \diamond$
~
$\div  \mp  \odot  \ominus  \oplus  \oslash  \otimes  \pm$
~
$\setminus  \sqcap  \sqcup  \star  \times  \triangleleft  \triangleright$
~
$\uplus  \vee  \wedge  \wr$

Nota: Para indicar o grau, costuma usar o \verb+\circ+ como em $90^\circ$.

Operadores de tamanhos variáveis.

$\bigcap  \bigotimes  \bigwedge  \bigcup  \bigsqcup  \coprod$
~
$\bigodot  \biguplus  \int  \bigoplus  \bigvee  \oint$
~
$\sum  \prod$
Exemplo: $\sum_{i=1}^n\frac{1}{i}$ e  
$\prod_{i=1}^n i = i!$

Relação binária

$\approx  \asymp  \bowtie  \cong  \dashv  \doteq$
~
$\equiv  \frown  \mid  \models  \parallel  \perp$
~
$\prec  \preceq  \propto  \sim  \simeq  \smile  \succ  \succeq$
\end{ltxlisting}
%\end{LTXexample}
%\end{singlespacing}

A diferença entre \verb+\mid+ e \verb+|+ é que o primeiro é um operador binário (como em $a\mid b$ que é $a$ divide $b$), enquanto que o segundo não é (como em $|a|$ que é valor absoluto de $a$)

Setas\index{seta} estão ilustradas no Exemplo~\ref{ex-a:math-arrow}.%
\index{modo matematico@{modo matemático}!seta!\verb+\Downarrow+, $\Downarrow$}%
\index{modo matematico@{modo matemático}!seta!\verb+\downarrow+, $\downarrow$}%
\index{modo matematico@{modo matemático}!seta!\verb+\hookleftarrow+, $\hookleftarrow$}%
\index{modo matematico@{modo matemático}!seta!\verb+\hookrightarrow+, $\hookrightarrow$}%
\index{modo matematico@{modo matemático}!seta!\verb+\leftarrow+, $\leftarrow$}%
\index{modo matematico@{modo matemático}!seta!\verb+\Leftarrow+, $\Leftarrow$}%
\index{modo matematico@{modo matemático}!seta!\verb+\Leftrightarrow+, $\Leftrightarrow$}%
\index{modo matematico@{modo matemático}!seta!\verb+\leftrightarrow+, $\leftrightarrow$}%
\index{modo matematico@{modo matemático}!seta!\verb+\longleftarrow+, $\longleftarrow$}%
\index{modo matematico@{modo matemático}!seta!\verb+\Longleftarrow+, $\Longleftarrow$}%
\index{modo matematico@{modo matemático}!seta!\verb+\longleftrightarrow+, $\longleftrightarrow$}%
\index{modo matematico@{modo matemático}!seta!\verb+\Longleftrightarrow+, $\Longleftrightarrow$}%
\index{modo matematico@{modo matemático}!seta!\verb+\longmapsto+, $\longmapsto$}%
\index{modo matematico@{modo matemático}!seta!\verb+\Longrightarrow+, $\Longrightarrow$}%
\index{modo matematico@{modo matemático}!seta!\verb+\longrightarrow+, $\longrightarrow$}%
\index{modo matematico@{modo matemático}!seta!\verb+\mapsto+, $\mapsto$}%
\index{modo matematico@{modo matemático}!seta!\verb+\nearrow+, $\nearrow$}%
\index{modo matematico@{modo matemático}!seta!\verb+\nwarrow+, $\nwarrow$}%
\index{modo matematico@{modo matemático}!seta!\verb+\Rightarrow+, $\Rightarrow$}%
\index{modo matematico@{modo matemático}!seta!\verb+\rightarrow+, $\rightarrow$}%
\index{modo matematico@{modo matemático}!seta!\verb+\searrow+, $\searrow$}%
\index{modo matematico@{modo matemático}!seta!\verb+\swarrow+, $\swarrow$}%
\index{modo matematico@{modo matemático}!seta!\verb+\uparrow+, $\uparrow$}%
\index{modo matematico@{modo matemático}!seta!\verb+\Uparrow+, $\Uparrow$}%
\index{modo matematico@{modo matemático}!seta!\verb+\updownarrow+, $\updownarrow$}%
\index{modo matematico@{modo matemático}!seta!\verb+\Updownarrow+, $\Updownarrow$}%
%
\index{modo matematico@{modo matemático}!seta!\verb+\leftharpoondown+, $\leftharpoondown$}%
\index{modo matematico@{modo matemático}!seta!\verb+\leftharpoonup+, $\leftharpoonup$}%
\index{modo matematico@{modo matemático}!seta!\verb+\rightharpoondown+, $\rightharpoondown$}%
\index{modo matematico@{modo matemático}!seta!\verb+\rightharpoonup+, $\rightharpoonup$}%
\index{modo matematico@{modo matemático}!seta!\verb+\rightleftharpoons+, $\rightleftharpoons$}%
%
\index{modo matematico@{modo matemático}!se, e somente se!\verb+\iff+, $\iff$}%
\index{modo matematico@{modo matemático}!implica!\verb+\implies+, $\implies$}%
%
%\begin{singlespacing}
%\begin{LTXexample}[pos=b,caption={ex02-01.tex},label=ex:02-01]
\begin{ltxlisting}[caption={ex-a-setas.tex},label=ex-a:math-arrow]
Setas

$\Downarrow  \downarrow  \hookleftarrow  \hookrightarrow$
~
$\leftarrow  \Leftarrow  \Leftrightarrow  \leftrightarrow$
~
$\longleftarrow  \Longleftarrow  \longleftrightarrow  \Longleftrightarrow$
~
$\longmapsto  \Longrightarrow  \longrightarrow  \mapsto$
~
$\nearrow  \nwarrow  \Rightarrow  \rightarrow  \searrow  \swarrow$
~
$\uparrow  \Uparrow  \updownarrow  \Updownarrow$

A seta dupla \verb+\Longleftrightarrow+ costuma ser usado para ``se, e somente se'', mas com pequeno ajuste no espaçamento. Para simplicidade, existe o comando \verb+\iff+ que produz $\iff$.
No pacote \texttt{amsmath}, define também o \verb+\implies+ para ``implica'' que tem mesmo símbolo que \verb+\Longrightarrow+, mas com espaçamento já ajustado devidamente. 

Arpões

$\leftharpoondown  \leftharpoonup  \rightharpoondown  \rightharpoonup$
~
$\rightleftharpoons$
\end{ltxlisting}
%\end{LTXexample}
%\end{singlespacing}

Para produzir o símbolo de ``não implica'', use o \verb+\centernot+ do pacote \texttt{centernot}\index{modo matematico@{modo matemático}!centernot@{\verb+\centernot+}}%
 (que não é da \texttt{base} nem do \texttt{required}) em vez do \verb+\not+\index{modo matematico@{modo matemático}!not@{\verb+\not+}}, como em $\centernot\implies$.

\section{Nome das funções e delimitadores no modo matemático}\label{apendice:funcao:e:delimitador}

Os nomes das funções\index{função matemática!nome da} devem estar em \verb+\mathrm+ e tem vários comandos já definidos.
No caso de precisar definir um novo (como nome em português), use
algo como \verb+\newcommand{\sen}{\mathrm{sen}}+ no preâmbulo.
Se estiver usando o pacote AMS, poderá substituir por \verb+\DeclaremathOperator{\sen}{sen}+ no preâmbulo.
No caso de definir nomes do estilo limite na qual o índice fica embaixo no modo display, use em conjunto com \verb+\mathop+, como em \verb+\newcommand{\argmin}{\mathop{\mathrm{arg\,min}}}+.
Note o uso de \verb+\,+ para inserir espaço entre duas palavras. Com o pacote do AMS, poderá escrever como \verb+\DeclaremathOperator*{\argmin}{arg\,min}+ (usando a versão ``*'' do \verb+\DeclaremathOperator+).

Veja o Exemplo~\ref{ex-a:math-func-names} para nome das funções pré-definidos.%
\index{modo matematico@{modo matemático}!funcao@{função}!\verb+\arccos+, $\arccos$}%
\index{modo matematico@{modo matemático}!funcao@{função}!\verb+\arcsin+, $\arcsin$}%
\index{modo matematico@{modo matemático}!funcao@{função}!\verb+\arctan+, $\arctan$}%
\index{modo matematico@{modo matemático}!funcao@{função}!\verb+\arg+, $\arg$}%
\index{modo matematico@{modo matemático}!funcao@{função}!\verb+\bmod+, $\bmod$}%
\index{modo matematico@{modo matemático}!funcao@{função}!\verb+\cos+, $\cos$}%
\index{modo matematico@{modo matemático}!funcao@{função}!\verb+\cosh+, $\cosh$}%
\index{modo matematico@{modo matemático}!funcao@{função}!\verb+\cot+, $\cot$}%
\index{modo matematico@{modo matemático}!funcao@{função}!\verb+\coth+, $\coth$}%
\index{modo matematico@{modo matemático}!funcao@{função}!\verb+csc+, $\csc$}%
\index{modo matematico@{modo matemático}!funcao@{função}!\verb+\deg+, $\deg$}%
%
\index{modo matematico@{modo matemático}!funcao@{função}!\verb+\det+, $\det$}%
\index{modo matematico@{modo matemático}!funcao@{função}!\verb+\dim+, $\dim$}%
\index{modo matematico@{modo matemático}!funcao@{função}!\verb+\exp+, $\exp$}%
\index{modo matematico@{modo matemático}!funcao@{função}!\verb+\gcd+, $\gcd$}%
\index{modo matematico@{modo matemático}!funcao@{função}!\verb+\hom+, $\hom$}%
\index{modo matematico@{modo matemático}!funcao@{função}!\verb+\inf+, $\inf$}%
\index{modo matematico@{modo matemático}!funcao@{função}!\verb+\ker+, $\ker$}%
\index{modo matematico@{modo matemático}!funcao@{função}!\verb+\lg+, $\lg$}%
%
\index{modo matematico@{modo matemático}!funcao@{função}!\verb+\lim+, $\lim$}%
\index{modo matematico@{modo matemático}!funcao@{função}!\verb+\liminf+, $\liminf$}%
\index{modo matematico@{modo matemático}!funcao@{função}!\verb+\limsup+, $\limsup$}%
\index{modo matematico@{modo matemático}!funcao@{função}!\verb+\ln+, $\ln$}%
\index{modo matematico@{modo matemático}!funcao@{função}!\verb+\log+, $\log$}%
\index{modo matematico@{modo matemático}!funcao@{função}!\verb+\max+, $\max$}%
\index{modo matematico@{modo matemático}!funcao@{função}!\verb+\min+, $\min$}%
\index{modo matematico@{modo matemático}!funcao@{função}!\verb+\Pr+, $\Pr$}%
\index{modo matematico@{modo matemático}!funcao@{função}!\verb+\sec+, $\sec$}%
\index{modo matematico@{modo matemático}!funcao@{função}!\verb+\sin+, $\sin$}%
\index{modo matematico@{modo matemático}!funcao@{função}!\verb+\sinh+, $\sinh$}%
\index{modo matematico@{modo matemático}!funcao@{função}!\verb+\sup+, $\sup$}%
\index{modo matematico@{modo matemático}!funcao@{função}!\verb+\tan+, $\tan$}%
\index{modo matematico@{modo matemático}!funcao@{função}!\verb+\tanh+, $\tanh$}%
\index{modo matematico@{modo matemático}!funcao@{função}!\verb+\pmod+, $\pmod{n}$}%

\begin{singlespacing}
%\begin{LTXexample}[pos=b,caption={ex02-01.tex},label=ex:02-01]
\begin{lstlisting}[caption={ex-a-nome-funcao.tex},label=ex-a:math-func-names]
$\arccos  \arcsin  \arctan  \arg  \bmod  \cos  \cosh  \cot  \coth  \csc  \deg$
~
$\det  \dim  \exp  \gcd  \hom  \inf  \ker  \lg$
~
$\lim  \liminf  \limsup  \ln  \log  \max  \min  \Pr  \sec  \sin  \sinh  \sup  \tan  \tanh$

Para indicar a congruência módulo $n$, podemo suar o comando \verb+\pmod+ como em $a \equiv b \pmod{n}$.
\end{lstlisting}
%\end{LTXexample}
\end{singlespacing}
% Uso separarado é porque \bmod sem estar entre chaves dentro deste ambiente puxa o próximo e sobrepõe.
\begin{ltxlistingout}
$\arccos  \arcsin  \arctan  \arg  {\bmod}  \cos  \cosh  \cot  \coth  \csc  \deg$
~
$\det  \dim  \exp  \gcd  \hom  \inf  \ker  \lg$
~
$\lim  \liminf  \limsup  \ln  \log  \max  \min  \Pr  \sec  \sin  \sinh  \sup  \tan  \tanh$

Para indicar a congruência módulo $n$, podemo suar o comando \verb+\pmod+ como em $a \equiv b \pmod{n}$.
\end{ltxlistingout}

Para delimitadores\index{delimitador}, veja o Exemplo~\ref{ex-a:math-delimiters}.
Em geral, o tamanho dos delimitadores serão ajustados automaticamente quando usar em conjunto com \verb+\left+\index{modo matematico@{modo matemático}!delimitador!left@\verb+\left.+} e \verb+\right+\index{modo matematico@{modo matemático}!delimitador!right@\verb+\right.+}.

O comando \verb+\left+ indica que o delimitador é do lado esquerdo e \verb+\right+ indica que o delimitador é do lado direito. Exemplo: $\left(\frac {x^2}{2}\right|_0^2=\frac{2^2}{2} - \frac{0^2}{2}=2$.
Quando usa o delimitador em apenas um dos lados, use "." para indicar que não há delimitador no lado especificado, como em $\left\{x+y=1 \atop x-y=0\right.$.

Para colocar delimitador ajustável no meio, poderá usar \verb+\middle+ na variante do e-\TeX como no caso do \TeX{} atual. 
Por exemplo, $\left\langle\phi\middle|\psi\right\rangle$ insere delimitador ajustável \verb+|+ no meio.%
\index{modo matematico@{modo matemático}!delimitador!\verb+\downarrow+,$\downarrow$}%
\index{modo matematico@{modo matemático}!delimitador!\verb+\langle+,$\langle$}%
\index{modo matematico@{modo matemático}!delimitador!\verb+\lceil+,$\lceil$}%
\index{modo matematico@{modo matemático}!delimitador!\verb+\lfloor+,$\lfloor$}%
\index{modo matematico@{modo matemático}!delimitador!$($}%
\index{modo matematico@{modo matemático}!delimitador!\verb+\Downarrow+,$\Downarrow$}%
\index{modo matematico@{modo matemático}!delimitador!\verb+\rangle+,$\rangle$}%
\index{modo matematico@{modo matemático}!delimitador!\verb+\rceil+,$\rceil$}%
\index{modo matematico@{modo matemático}!delimitador!\verb+\rfloor+,$\rfloor$}%
\index{modo matematico@{modo matemático}!delimitador!$)$}%
\index{modo matematico@{modo matemático}!delimitador!\verb+\backslash+,$\backslash$}%
\index{modo matematico@{modo matemático}!delimitador!$[$}%
\index{modo matematico@{modo matemático}!delimitador!$\vert$}%
\index{modo matematico@{modo matemático}!delimitador!\verb+\uparrow+,$\uparrow$}%
\index{modo matematico@{modo matemático}!delimitador!\verb+\updownarrow+,$\updownarrow$}%
\index{modo matematico@{modo matemático}!delimitador!chaveaberta@{\textbackslash\curlyopensymbforindex,$\curlyopensymbforindex$}}%
\index{modo matematico@{modo matemático}!delimitador!$]$}%
\index{modo matematico@{modo matemático}!delimitador!$\backslash\Vert$,$\Vert$}%
\index{modo matematico@{modo matemático}!delimitador!\verb+\Uparrow+,$\Uparrow$}%
\index{modo matematico@{modo matemático}!delimitador!\verb+\Updownarrow+,$\Updownarrow$}%
\index{modo matematico@{modo matemático}!delimitador!chavefechada@{\textbackslash\curlyclosesymbforindex,$\curlyclosesymbforindex$}}%
%
\index{modo matematico@{modo matemático}!delimitador!lmoustache@{\verb+\lmoustache+,$\left\lmoustache\phantom{x}\right.$}}%
\index{modo matematico@{modo matemático}!delimitador!rmoustache@{\verb+\rmoustache+,$\left.\phantom{x}\right\rmoustache$}}%
\index{modo matematico@{modo matemático}!delimitador!lgroup@{\verb+\lgroup+,$\left\lgroup\right.$}}%
\index{modo matematico@{modo matemático}!delimitador!rgroup@{\verb+\rgroup+,$\left.\phantom{x}\right\rgroup$}}%

%\begin{singlespacing}
%\begin{LTXexample}[pos=b,caption={ex02-01.tex},label=ex:02-01]
\begin{ltxlisting}[caption={ex-a-delimitadores.tex},label=ex-a:math-delimiters]
Delimitador que pode ser usado independente ou como delimitador auto ajustável (com \verb+\left+ e \verb+\right+).

$\downarrow  \langle  \lceil  \lfloor  (  /$
~
$\Downarrow  \rangle  \rceil  \rfloor  )  \backslash$
~
$[  |  \uparrow  \updownarrow  \{  ]  \|  \Uparrow  \Updownarrow  \}$
 
Símbolos de tamanhos variáveis grandes que só podem ser usados em conjunto com \verb+\left+ e \verb+\right+.

$\left\lmoustache\right.  \left.\right\rmoustache$% não tem no Latin Moden Math
$\left\lgroup\right.  \left.\right\rgroup$

\end{ltxlisting}
%\end{LTXexample}
%\end{singlespacing}

Os comandos \verb+\vert+ e \verb+\Vert+ são mesmos que \verb+|+ e \verb+\|+, mas podem ser usados onde \verb+|+ causa problemas como na entrada de índices remissivos.


Quando tiver quebra de linha no meio da equação (que pode ser efetuado pelos ambientes do pacote \texttt{amsmath}), os delimitadores auto ajustáveis falham. Outro caso é ter um delimitador dentro do outro de mesmo tamanho. Neste caso, poderá ajustar o tamanho manualmente pelos comandos tipo ``big'' em vez de ``left/right''. O especificador do delimitador grande na ordem crescente são: \verb+\big+\index{modo matematico@{modo matemático}!delimitador!\verb+\big+}, \verb+\Big+\index{modo matematico@{modo matemático}!delimitador!\verb+\Big+}, \verb+\bigg+\index{modo matematico@{modo matemático}!delimitador!\verb+\bigg+} e \verb+\Bigg+\index{modo matematico@{modo matemático}!delimitador!\verb+\Bigg+}.
Veja o Exemplo~\ref{ex-a:math-big-delimiters}.

%\begin{singlespacing}
%\begin{LTXexample}[pos=b,caption={ex02-01.tex},label=ex:02-01]
\begin{ltxlisting}[caption={ex-a-big-delimitadores.tex},label=ex-a:math-big-delimiters]
Exemplo de ajuste manual do tamanho doas delimitadores.

\[
  \Bigg(x+\bigg(y+\Big(z+\big(w+v\big)\Big)\bigg)\Bigg)
\]
\end{ltxlisting}
%\end{LTXexample}
%\end{singlespacing}

%Símbolos de tamanhos variáveis grandes (requer\verb+\left+ e \verb+\right+) que não é suportado pelo \texttt{unicode-math} (logo, no \XeLaTeX{} e Lua\LaTeX).
%$
%\left\arrowvert\right.
%\left\Arrowvert\right.
%\left\bracevert\right.
%$

\section{Outros símbolos\protect\index{símbolos!diversos}}

Símbolos que costuma ser usados como caracteres estão listados no Exemplo~\ref{ex-a:math-letter-like}.%
\index{modo matematico@{modo matemático}!diverso!\verb+\bot+, $\bot$}%
\index{modo matematico@{modo matemático}!diverso!\verb+\ell+, $\ell$}%
\index{modo matematico@{modo matemático}!diverso!\verb+\exists+, $\exists$}%
\index{modo matematico@{modo matemático}!diverso!\verb+\forall+, $\forall$}%
\index{modo matematico@{modo matemático}!diverso!\verb+\Im+, $\Im$}%
\index{modo matematico@{modo matemático}!diverso!\verb+\imath+, $\imath$}%
\index{modo matematico@{modo matemático}!diverso!\verb+\in+, $\in$}%
\index{modo matematico@{modo matemático}!diverso!\verb+\jmath+, $\jmath$}%
\index{modo matematico@{modo matemático}!diverso!\verb+\ni+, $\ni$}%
\index{modo matematico@{modo matemático}!diverso!\verb+\partial+, $\partial$}%
\index{modo matematico@{modo matemático}!diverso!\verb+\Re+, $\Re$}%
\index{modo matematico@{modo matemático}!diverso!\verb+\top+, $\top$}%
\index{modo matematico@{modo matemático}!diverso!\verb+\wp+, $\wp$}%
\index{modo matematico@{modo matemático}!diverso!not@\verb+\not+}%

%\begin{singlespacing}
%\begin{LTXexample}[pos=b,caption={ex02-01.tex},label=ex:02-01]
\begin{ltxlisting}[caption={ex-a-tipo-letra.tex},label=ex-a:math-letter-like]
$\bot  \ell  \exists  \forall  \hbar  \Im  \imath  \in  \jmath$
~
$\ni  \partial  \Re  \top  \wp$

Para produzir símbolo de negação, poderá usar \verb+\not+ antes do símbolo, como em $\not\in$ ou $\not\exists$.
\end{ltxlisting}
%\end{LTXexample}
%\end{singlespacing}

O Exemplo~\ref{ex-a:math-misc} ilustra mais alguns símbolos adicionais.%
\index{modo matematico@{modo matemático}!diverso!\verb+\aleph+, $\aleph$}%
\index{modo matematico@{modo matemático}!diverso!\verb+\emptyset+, $\emptyset$}%
\index{modo matematico@{modo matemático}!diverso!\verb+\angle+, $\angle$}%
\index{modo matematico@{modo matemático}!diverso!\verb+\backslash+, $\backslash$}%
\index{modo matematico@{modo matemático}!diverso!\verb+\infty+, $\infty$}%
\index{modo matematico@{modo matemático}!diverso!\verb+\nabla+, $\nabla$}%
\index{modo matematico@{modo matemático}!diverso!\verb+neg\+, $\neg$}%
\index{modo matematico@{modo matemático}!diverso!\verb+\prime+, $\prime$}%
\index{modo matematico@{modo matemático}!diverso!\verb+\surd+, $\surd$}%
\index{modo matematico@{modo matemático}!diverso!\verb+\triangle+, $\triangle$}%
%
\index{modo matematico@{modo matemático}!diverso!\verb+\flat+, $\flat$}%
\index{modo matematico@{modo matemático}!diverso!\verb+\natural+, $\natural$}%
\index{modo matematico@{modo matemático}!diverso!\verb+\sharp+, $\sharp$}%

%\begin{singlespacing}
%\begin{LTXexample}[pos=b,caption={ex02-01.tex},label=ex:02-01]
\begin{ltxlisting}[caption={ex-a-simbolos-diversos.tex},label=ex-a:math-misc]
Símbolos diversos

$\aleph  \emptyset  \angle  \backslash  \infty  \nabla  \neg  \prime  \surd  \triangle$

Símbolos musicais

$\flat  \natural  \sharp$
\end{ltxlisting}
%\end{LTXexample}
%\end{singlespacing}

Alguns autores preferem usar o \verb+\varnothing+\index{\varnothing} do \texttt{amssymb} em vez do \verb+\emptset+ para representar o conjunto vazio.

\section{Acentuação no modo matemático\protect\index{modo matematico@{modo matemático}!acentuacao@{acentuação}}}
Nas fórmulas, costuma usar vários tipos de acentuações para produzir símbolos novos que é relacionado com outros já existentes (não acentuado). Veja o Exemplo~\ref{ex-a:math-punct}%
\index{modo matematico@{modo matemático}!acentuacao@{acentuação}!\verb+\acute{a}+, $\acute{a}$}%
\index{modo matematico@{modo matemático}!acentuacao@{acentuação}!\verb+\bar{a}+, $\bar{a}$}%
\index{modo matematico@{modo matemático}!acentuacao@{acentuação}!\verb+\breve{a}+, $\breve{a}$}%
\index{modo matematico@{modo matemático}!acentuacao@{acentuação}!\verb+\check{a}+, $\check{a}$}%
\index{modo matematico@{modo matemático}!acentuacao@{acentuação}!\verb+\ddot{a}+, $\ddot{a}$}%
\index{modo matematico@{modo matemático}!acentuacao@{acentuação}!\verb+\dot{a}+, $\dot{a}$}%
\index{modo matematico@{modo matemático}!acentuacao@{acentuação}!\verb+\grave{a}+, $\grave{a}$}%
\index{modo matematico@{modo matemático}!acentuacao@{acentuação}!\verb+\hat{a}+, $\hat{a}$}%
\index{modo matematico@{modo matemático}!acentuacao@{acentuação}!\verb+\mathring{a}+, $\mathring{a}$}%
\index{modo matematico@{modo matemático}!acentuacao@{acentuação}!\verb+\tilde{a}+, $\tilde{a}$}%
\index{modo matematico@{modo matemático}!acentuacao@{acentuação}!\verb+\vec{a}+, $\vec{a}$}%
%
\index{modo matematico@{modo matemático}!acentuacao@{acentuação}!\verb+\imath+, $\imath$}%
\index{modo matematico@{modo matemático}!acentuacao@{acentuação}!\verb+\jmath+, $\jmath$}%
%
\index{modo matematico@{modo matemático}!acentuacao@{acentuação}!\verb+\widetilde{abc}+, $\widetilde{abc}$}%
\index{modo matematico@{modo matemático}!acentuacao@{acentuação}!\verb+\widehat{abc}+, $\widehat{abc}$}%
\index{modo matematico@{modo matemático}!acentuacao@{acentuação}!\verb+\overleftarrow{abc}+, $\overleftarrow{abc}$}%
\index{modo matematico@{modo matemático}!acentuacao@{acentuação}!\verb+\overline{abc}+, $\overline{abc}$}%
\index{modo matematico@{modo matemático}!acentuacao@{acentuação}!\verb+\underline{abc}+, $\underline{abc}$}%
\index{modo matematico@{modo matemático}!acentuacao@{acentuação}!\verb+\overbrace{abc}^{n}+, $\overbrace{abc}^{n}$}%
\index{modo matematico@{modo matemático}!acentuacao@{acentuação}!\verb+\overrightarrow{abc}+, $\overrightarrow{abc}$}%
\index{modo matematico@{modo matemático}!acentuacao@{acentuação}!\verb+\underbrace{abc}_{n}+, $\underbrace{abc}_{n}$}%
\index{modo matematico@{modo matemático}!acentuacao@{acentuação}!\verb+\sqrt{abc}+, $\sqrt{abc}$}%
%
\index{modo matematico@{modo matemático}!acentuacao@{acentuação}!\verb+\sqrt[n]{abc}+, $\sqrt[n]{abc}$}%
%
\index{modo matematico@{modo matemático}!pontuacao@{pontuação}!\verb+\cdotp+, $\cdotp$}%
\index{modo matematico@{modo matemático}!pontuacao@{pontuação}!\verb+\cdots+, $\cdots$}%
\index{modo matematico@{modo matemático}!pontuacao@{pontuação}!\verb+\colon+, $\colon$}%
\index{modo matematico@{modo matemático}!pontuacao@{pontuação}!\verb+\ldotp+, $\ldotp$}%
\index{modo matematico@{modo matemático}!pontuacao@{pontuação}!\verb+\ddots+, $\ddots$}%
\index{modo matematico@{modo matemático}!pontuacao@{pontuação}!\verb+\ldots+, $\ldots$}%
\index{modo matematico@{modo matemático}!pontuacao@{pontuação}!\verb+\vdots+, $\vdots$}%
%
para acentuação no modo matemático.

%\begin{singlespacing}
%\begin{LTXexample}[pos=b,caption={ex02-01.tex},label=ex:02-01]
\begin{ltxlisting}[caption={ex-a-acentuacao.tex},label=ex-a:math-punct]
Acentuação no modo matemático

$\acute{a}  \bar{a}  \breve{a}  \check{a}  \ddot{a}  \dot{a}$
~
$\grave{a}  \hat{a}  \mathring{a}  \tilde{a}  \vec{a}$

Para i e j sem pontos, existem os comandos \verb+\imath+ e \verb+\jmath+ que podem ser usado como em $\hat{\imath}$ e $\hat{\jmath}$.

Acentos matemáticos extensíveis.

$\widetilde{abc}  \widehat{abc}  \overleftarrow{abc}$
~
$\overline{abc}  \underline{abc}  \overbrace{abc}^{n}$
~
$\underbrace{abc}_{n}  \overrightarrow{abc}  \sqrt{abc} \sqrt[n]{abc}$

Pontuação no modo matemático
$\cdotp  \cdots  \colon  \ldotp  \ddots  \ldots  \vdots$

Note que vários símbolos de pontuação são usados como relações no modo matemático.
Assim, quando precisar usar como pontuação, existem comandos para tal (a diferença entre relação e pontuação está no espaçamento. Por exemplo, $a:b$ é uma relação de proporção, enquanto que $a\colon b$ é uma pontuação.
\end{ltxlisting}
%\end{LTXexample}
%\end{singlespacing}

Note que, diferente do \verb+\bar+, o comando \verb+\overline+ adjacentes ficam grudados, como em $\overline{A}\overline{B}$. 
Para evitar isso, siga a instrução do \cite{book:pakin:2017} para definir um comando apropriado denominado de \verb+\closure+ que não grudam. 
Para isso, basta colocar o seguinte código no \texttt{preâmbulo}.

\begin{singlespacing}
\begin{lstlisting}
\newcommand{\closure}[2][3]{%
{}\mkern#1mu\overline{\mkern-#1mu#2}}
\end{lstlisting}
\end{singlespacing}

Alguns alfabetos matemáticos\index{alfabeto!matemático} disponíveis por padrão estão listado no Exemplo~\ref{ex-a:math-alphabet}.%
\index{modo matematico@{modo matemático}!alfabeto!\verb+\mathrm{ABCdef123}+, $\mathrm{ABCdef123}$}%
\index{modo matematico@{modo matemático}!alfabeto!\verb+\mathit{ABCdef123}+, $\mathit{ABCdef123}$}%
\index{modo matematico@{modo matemático}!alfabeto!\verb+\mathbf{ABCdef123}+, $\mathbf{ABCdef123}$}%
\index{modo matematico@{modo matemático}!alfabeto!\verb+\mathnormal{ABCdef123}+, $\mathnormal{ABCdef123}$}%
\index{modo matematico@{modo matemático}!alfabeto!\verb+\mathcal{ABC}+, $\mathcal{ABC}$}%
%
\index{modo matematico@{modo matemático}!caracter de extensao@{caracter de extensão}!\verb+\relbar+, $\relbar$}%
\index{modo matematico@{modo matemático}!caracter de extensao@{caracter de extensão}!\verb+\Relbar+, $\Relbar$}%

%\begin{singlespacing}
%\begin{LTXexample}[pos=b,caption={ex02-01.tex},label=ex:02-01]
\begin{ltxlisting}[caption={ex-a-alfabeto.tex},label=ex-a:math-alphabet]
Alfabetos matemáticos

$\mathrm{ABCdef123}  \mathit{ABCdef123}  \mathbf{ABCdef123}$
~
$\mathnormal{ABCdef123}  \mathcal{ABC}$
% \mathcal é somente para romana maiúscula

Caractere de extensão.
Diferença com $-=$ (relação) é somente o espaçamento.
Quais são os caso de uso?

$\relbar\Relbar$
\end{ltxlisting}
%\end{LTXexample}
%\end{singlespacing}
O alfabeto \verb+\mathbb+ requer o pacote do \texttt{AMS}. 
No sistema de \texttt{blog} e \texttt{moodle} ou similar sem a suporte ao pacote adicional, use o \verb+\mathbf+ em vez de \verb+\mathbb+.

\chapter{Desenvolvendo Pacotes}
Aqui veremos como escrever seu próprio pacote, também chamado de arquivos de estilos.

\section{Criando pacotes}
Pacote é conjunto de configurações de documentos e definições de comandos ou similares para ser aplicado no documento desejado. 
Pacote também é chamado de arquivo de estilos e tem a extensão \texttt{sty}\index{arquivos!sty}.

Um pacote começa com o cabeçalho.
O comando \verb+\NeedsTeXFormat{LaTeX2e}+\index{NeedsTeXFormat@\verb+\NeedsTeXFormat+} colocado no começo do pacote indica que ele precisa do \LaTeXe.
Em seguida, costuma colocar o comando
%\verb+\ProvidesFile{<nome do pacote>}[<mensagem>]+\index{ProvidesFile@\verb+\ProvidesFile+}.
\verb+\ProvidesPackage{<nome do pacote>}[<mensagem>]+\index{ProvidesPackage@\verb+\ProvidesPackage+}.
onde \texttt{<nome do arquivo>} é nome do arquivo, incluindo a extensão e o argumento opcional \texttt{<mensagem>} é a mensagem a ser emitido quando carrega o pacote.
As informações adicionais tais como versões, mensagens de advertência ou erros, etc pode ser emitido pelo comando \verb+\typeout{<mensagem>}+\index{typeout@\verb+\typeout+} a qualquer momento.

Quando o pacote é uma classe de documento baseado num documento já existente, poderá carregar a classe de documento pelo comando \verb+\LoadClass+ e efetuar ajustes necessários.

Note que não precisa colocar \verb+\makeatletter+\index{makeatletter@\verb+\makeatletter+} e \verb+\makeatother+\index{makeatother@\verb+\makeatother+} (nem deve) no arquivo de estilo para acessar comandos que usam ``@'', pois já está com o uso de ``@'' ativado por padrão.
No caso de \XeLaTeX/Lua\LaTeX{} que usam pacotes mais modernos, usa ``\verb+_+'' no nome dos comandos. 
Para usar estes comandos no \texttt{pramble}, use entre \verb+\ExplSyntaxOn+\index{ExplSyntaxOn@\verb+\ExplSyntaxOn+} e \verb+\ExplSyntaxOff+\index{ExplSyntaxOff@\verb+\ExplSyntaxOff+}. 
Também não é necessário no arquivo de estilos.

É recomendado que use a acentuação no modo \TeX, pois não saberemos a codificação que o usuário final vai escolher no seu documento.

Dentro do arquivo de estilos, o comando para carregar os pacotes é \verb+\RequirePackage+\index{RequirePackage@\verb+\RequirePackage+} em vez de \verb+\usepackage+.
Para testar se o pacote existe, cheque se tem o arquivo correspondente com o \verb+\IfFileExists+\index{IfFileExists@\verb+\IfFileExists+}. 
Não esqueça da extensão do arquivo no nome. 
Se arquivo do primeiro parâmetro existir, executará o segundo parâmetro.
Caso não existir, executará o terceiro parâmetro.

Também é bom checar se não está executando no modo de compatibilidade (modo antigo). 
Para isso, usa-se o comando \verb+\if@compatibility+\index{ifcompatibility@{\verb+\if+\arrobasymbforindex\verb+compatibility+}} que é da forma
\begin{singlespacing}
\begin{lstlisting}
\if@compatibility 
<comando 1>
\else
<comando 2>
\fi
\end{lstlisting}
\end{singlespacing}
A mensagem de erro é emitido por

\verb+\PackageError{<nome do pacote>}{<mensagem curta>}{<mensagem longa>}+\index{PackageError@\verb+\PackageError+}.

Se for só a advertência, usa-se o comando
\verb+\PackageWarning{<nome do pacote>}{<mensagem>}+\index{PackageWarning@\verb+\PackageWarning+}.

Para saber se foi carregado a classe de documento compatível, usa-se o comando \verb+\@ifclassloaded{}+\index{ifclassloaded@{\textbackslash\arrobasymbforindex\verb+ifclassloaded+}}
 que tem a forma \verb+\@ifclassloaded{<classe>}{<cmd1>}{<cmd2>}+.
Quando \texttt{<class>} for usado, executa o \texttt{<cmd1>} e caso não for a \texttt{<classe>}, executa o \texttt{<cmd2>}.
Com isso, podemos elaborar o cabeçalho do arquivo de estilo. 
O Exemplo~\ref{ex:b:estilo:sty} é uma listagem do exemplo de arquivos de estilos.
\begin{singlespacing}
%\begin{lstlisting}[caption={ex12-07.sty}, label=ex:12-07:sty]
\begin{lstlisting}[caption={ex-b-estilo.sty}, label=ex:b:estilo:sty]
% ex-b-estylo.sty

\NeedsTeXFormat{LaTeX2e}
\ProvidesPackage{ex-b-estilo}[2021/07/14 v0.6 ex-b-estilo style (require book, amsbook, or report as document class)]

\typeout{ex-b-estilo style 0.6 <July/2021>.}

%%%%%%%%%%%%%%%%%%%%%%%%%%%%%%%%%%%%%%%%%%%%%%%%%%
% Para evitar a tentativa de usar no modo de
% compatibilidade com LaTeX antigo (documentstyle)
%---------------------------------------
\if@compatibility % modo de compatibilidade com LaTeX antigo?
  \PackageError{ex-b-estilo}{Not support older compatible mode (documentstyle)}
     {Use documentclass instead of documentstyle}
  \endinput
\else
  % OK
\fi

%%%%%%%%%%%%%%%%%%%%%%%%%%%%%%%%%%%%%%%%%%%%%%%%
% requer book, amsbook, ou report como classe de
% documento (enumeracao do teorema eh vinculado
% ao chapter)
%---------------------------------------
\@ifclassloaded{book}% Using book?
  {}% YES (OK)
  {\@ifclassloaded{amsbook}% No, but are using AMS book?
    {}% YES (OK)
    {\@ifclassloaded{report}% No, but are using report?
      {}% YES (OK)
      { % No! Then can'not apply this package
        \PackageError{ex-b-estilo}{Require book, amsbook, or report as document class}
           {Use the required document class}
        %\endinput
      }
    }
   }

%%%%%%%%
% opcoes
%-------------------------------
% Aqui, opcoes ativa o if para efetuar configuracoes posteriormente

% \RequirePackage{ifthen} % da base/required

\newif\if@ex@b@estilo@usedsfont

% se usedsfont for usado como opcao
\DeclareOption{usedsfont}{\@ex@b@estilo@usedsfonttrue}{\@ex@b@estilo@usedsfontfalse}
  
%% Fallback (opcoes nao declaradas)
\DeclareOption*{
  \PackageWarning{ex-b-estilo}{Unknown option '\CurrentOption'}
}

% Apos declarar todas opcoes, colocar este comando
\ProcessOptions\relax



%%%%%%%%%%%%%%%%%%%%%%
% Pacotes necessarios

% Da base/required
\RequirePackage{amsthm, amssymb,amsmath}

% \RequirePackage[brazil]{babel}

% pacotes nao obrigatorios do LaTeX
% \RequirePackage{geometry} % carregar sempre

%\IfFileExists{dsfont.sty}% carregar, se existir
%{\RequirePackage{dsfont}}{}%

% testando se tem o pacote instalado e providenciando o
% comando, caso pacote nao exista.
\IfFileExists{hyperref.sty}%
{\RequirePackage{hyperref}}%
{\PackageWarning{ex-b-estilo}{hyperref.sty not found. Using draft mode.}%
 \providecommand\url[1]{\texttt{##1}}%
}
  
%%%%%%%%%%%%%%%%%%%%%%%%%%%%%%%%%%%%%%%%%%%%%%%%
% definicao dos macros para ambiente de teoremas
% teoremas, lemas, proposi\c{c}\~oes, etc

\theoremstyle{plain}

\newtheorem{theorem}{Teorema}[chapter]
\newtheorem{axiom}[theorem]{Axioma}
\newtheorem{corollary}[theorem]{Corol\'ario}
\newtheorem{lemma}[theorem]{lema}
\newtheorem{proposition}[theorem]{Proposi\c{c}\~ao}
\newtheorem{conjecture}[theorem]{Conjectura}

% defini\c c\~ao de defini\c c\~oes, exemplos, etc.
\theoremstyle{definition}
\newtheorem{definition}[theorem]{Defini\c{c}\~ao}

\theoremstyle{remark}
\newtheorem{remark}[theorem]{Obserca\c{c}\~ao}

\newtheorem{note}[theorem]{Nota}
\newtheorem{example}[theorem]{Exemplo}
%
\newtheorem{question}[theorem]{Pergunta}

% alterando as penalidades (para corte de linhas em...)
\hyphenpenalty=5000 % hifeniza\c{c}\~{a}o
\exhyphenpenalty=500 % palavras com hifem
\binoppenalty=3000 % operador binario (+, - , etc)
\relpenalty=2000 % operador relacional ( = \cong \ne, ...)
\clubpenalty=1000 % ???
\brokenpenalty=1000 % ???

\sloppy % prefere underfull do que overfull

% conjunto numerico (evitando o real, complexo, etc,

\newcommand{\Rset}{\mathbb{R}}
\newcommand{\Cset}{\mathbb{C}}
\newcommand{\Zset}{\mathbb{Z}}
\newcommand{\Iset}{\mathbb{I}}
\newcommand{\Qset}{\mathbb{Q}}
\newcommand{\Nset}{\mathbb{N}}

% Funcoes em portugues (nome da funcao deve ser em romano)
\DeclareMathOperator{\sen}{sen}
\DeclareMathOperator{\arcsen}{arcsen}
\DeclareMathOperator{\senh}{senh}
\DeclareMathOperator{\arcsenh}{arcsenh}

%%%%%%%%%%%%%%%%%%%%%%%%%%%%%%%%%%%%%%
% Reconfigurando de acordo com a opcao

% Usar dsfont
\if@ex@b@estilo@usedsfont% se usedsfont for usado como opcao
  \IfFileExists{dsfont.sty}%
  {%
    \RequirePackage{dsfont}
    \renewcommand{\Rset}{\mathds{R}}
    \renewcommand{\Cset}{\mathds{C}}
    \renewcommand{\Zset}{\mathds{Z}}
    \renewcommand{\Iset}{\mathds{I}}
    \renewcommand{\Qset}{\mathds{Q}}
    \renewcommand{\Nset}{\mathds{N}}  
  }{%
  \PackageError{ex-b-estilo}{dsfont.sty not found.}%
    {Install dsfont package.}%
  }
\fi% \if@ex@b@estilo@usedsfont

% fim: ex-b-estilo.sty
\end{lstlisting}
\end{singlespacing}

Quando o arquivo de estilo usa opções, coloca o que vai fazer quando uma determinada opção for especificada, com o comando
\verb+\DeclareOption{<opção>}{<comando>}+\index{DeclareOption@\verb+\DeclareOption+}.
Lembre-se que não é permitido carregar o pacote dentro da declaração de opções, mas poderemos testar se o pacote existe.
Para evitar tais desconfortos e deixar o código mais fácil de ser analisados, costuma colocar a declarações de opções no começo e nele, colocar ativação do ``if'' previamente criado. O corpo da opção ficará mais adiante.
Para criar ``if'' novo, usa-se o pacote \texttt{ifthen}\index{ifthen} e utiliza o \verb+\newif+\index{\textbackslash newif} seguido de \verb+\if<nome>+  onde \verb+<nome>+ é um nome desejado. O ``flag'' associado pode ser ativado ou desativado com o comando \verb+\<nome>true+ e \verb+\<nome>false+ respectivamente.
O código
\begin{verbatim}
\if<nome>
comando 1
\else
comando 2
\fi
\end{verbatim}

Executará \texttt{comando 1} se ``flag'' associado a \verb+<nome>+ estiver ligado e \texttt{comando 2} caso contrário. Quando não tiver o \texttt{comando 2}, \verb+\else+ pode ser omitido como em
\begin{verbatim}
\if<nome>
comando 1
\fi
\end{verbatim}
Note que existe um pacote \texttt{ifthen} (da base) para manipular condições de forma mais confortável do que criar ``if''.

Também podemos passar opções para classe que será carregado posteriormente, pelo comando \verb+\PassOptionsToClass+\index{PassOptionsToClass@\verb+\PassOptionsToClass+} ou para o pacote a ser carregado posteriormente, com o comando \verb+\PassOptionsToPackage+\index{PassOptionsToPackage@\verb+\PassOptionsToPackage+}.

No caso de querer que alguma opção seja aplicada como padrão, poderá usar o comando \verb+\ExecuteOptions+.

Quando tem a declaração de opções, deve colocar \verb+\ProcessOptions+\index{ProcessOptions@\verb+\ProcessOptions+} após terminar todos \verb+\DeclareOption+ e \verb+\ExecuteOptions+ para que opções fiquem prontos para o uso.
% Em geral, costuma colocar o \verb+\ProcessOptions+ no final do arquivo.

O Exemplo~\ref{ex:b:estilo:tex} é um exemplo do uso do pacote do Exemplo~\ref{ex:b:estilo:sty}.

\begin{singlespacing}
%\begin{lstlisting}[caption={ex12-07.tex}, label=ex:12-07]
\begin{lstlisting}[caption={ex-b-estilo.tex}, label=ex:b:estilo:tex]
\documentclass[a4paper,12pt]{book}
\usepackage[T1]{fontenc} % codificação da fonte em 8-bits
\usepackage[utf8]{inputenc} % acentuação direta
\usepackage[brazil]{babel} % em portugues brasileiro

\usepackage{ex-b-estilo} % arquivo de estilo definido pelo usuário.

% \usepackage[usedsfont]{ex-b-estilo} % com opcao (se usar este, conjuntos numéricos ficam como dsfont, se dsfont estiver instalado)
\begin{document}
\chapter{Teste}

\begin{definition}\label{def:triangulo:retangulo}
Um triângulo é dito retângulo se tiver um ângulo reto.
\end{definition}

\begin{theorem}[Pitágoras]\label{thm:pitagoras}
Dado um triângulo retângulo $ABC$ com ângulo reto em $A$, temos que 
\begin{equation}\label{eq:pitagoras}
a^2=b^2+c^2
\end{equation}
\end{theorem}

\begin{proof}
Demonstração aqui.
\end{proof}

Usando a definição do $\sen$ e $cos$ juntamente com o Teorema~\ref{thm:pitagoras}, temos que

\begin{proposition}
\[
\forall t \in \Rset, \sen^2 t + \cos^2 t = 1
\]
\end{proposition}
\end{document}
\end{lstlisting}
\end{singlespacing}

%\let\mathbb\amsmathbb
%\setcounter{lstorichapter}{\value{chapter}}
%\setcounter{lstorisection}{\value{section}}
%\setcounter{chapter}{0}
%\let\addcontentsline\emptyaddcontentsline
%\let\orithispagestyle\thispagestyle
%\makeatletter
%\let\thispagestyle\@gobble
%\makeatother
%\begin{singlespacing}
%\begin{framed}
\let\oriappendixname\appendixname
\let\orithechapter\thechapter
\renewcommand{\appendixname}{\chaptername}
\renewcommand{\thechapter}{\arabic{chapter}}
\begin{ltxlistingout}
\chapter{Teste}

\begin{definition}%\label{def:triangulo:retangulo}
Um triângulo é dito retângulo se tiver um ângulo reto.
\end{definition}

\begin{theorem}[Pitágoras]%\label{thm:pitagoras}
Dado um triângulo retângulo $ABC$ com ângulo reto em $A$, temos que 
\begin{equation}%\label{eq:pitagoras}
a^2=b^2+c^2
\end{equation}
\end{theorem}

\begin{proof}
Demonstração aqui.
\end{proof}

Usando a definição do $\mathrm{sen}$ e $\cos$ juntamente com o Teorema~1.2%\ref{thm:pitagoras}
, temos que

\begin{proposition}
\[
\forall t \in \mathbb{R}, \mathrm{sen}^2 t + \cos^2 t = 1
\]
\end{proposition}
\end{ltxlistingout}
\let\appendixname\oriappendixname
\let\thechapter\orithechapter
%\end{framed}
%\end{singlespacing}
%\setcounter{chapter}{\value{lstorichapter}}
%\setcounter{section}{\value{lstorisection}}
%\let\mathbb\orimathbb
%% \addtocounter{chapter}{-1}
%\let\thispagestyle\orithispagestyle
%\let\addcontentsline\oriaddcontentsline

\section{Criando classes}
Em geral, a classe é criado de mesma forma que um arquivo de estilo.
Em muitos casos, a classe criada consiste na reconfiguração da classe existente. Quando o ajuste pode ser efetuado sobre diversas classes base, é conselhável criar um arquivo de estilo. Quando o ajuste é aplicado numa classe específica, é aconselhável criar uma classe sobre tal classe base.

Os comandos
\verb+\ProvidesPackage{<nome do pacote>}[<mensagem>]+ será substituído por
\verb+\ProvidesClass{<nome da classe>}[<mensagem>]+\index{ProvidesClass@\verb+\ProvidesClass+}.

A classe base é lido pelo comando \verb+\LoadClass{}+ após declaração das opções, passando as opções não processadas para classe base. Isto pode ser feito com o código do tipo
\begin{verbatim}
% Demais opções
\DeclareOption*{%
  \PassOptionsToClass{\CurrentOption}{book}% repassar para book
}
\ProcessOptions\relax
% carrega a classe base
\LoadClass{book}
\end{verbatim}

O \verb+\PassOptionsToClass+\index{PassOptionsToClass@\verb+\PassOptionsToClass+} é comando para agendar os parâmetros para a classe a ser lido posteriormente. Note que existe o comando \verb+\PassOptionsToPackage+\index{PassOptionsToPackage@\verb+\PassOptionsToPackage+} para pacotes, com mesma finalidade.

O Exemplo~\ref{ex:b:classe:cls} é uma listagem do exemplo de arquivos de classe.
\begin{singlespacing}
%\begin{lstlisting}[caption={ex12-07.cls}, label=ex:12-07:cls]
\begin{lstlisting}[caption={ex-b-classe.cls}, label=ex:b:classe:cls]
\NeedsTeXFormat{LaTeX2e}
\NeedsTeXFormat{LaTeX2e}

\ProvidesClass{ex-b-classe}[2021/07/14 v0.6 ex-b-classe class file (implemented over book class)]

\typeout{ex-b-classe 0.5 <July/2021>.}

%%%%%%%%%%%%%%%%%%%%%%%%%%%%%%%%%%%%%%%%%%%%%%%%%%
% Para evitar a tentativa de usar no modo de
% compatibilidade com LaTeX antigo (documentstyle)
%---------------------------------------
\if@compatibility % modo de compatibilidade com LaTeX antigo?
  \ClassError{ex-b-classe}{Not support older compatible mode (documentstyle)}
     {Use documentclass instead of documentstyle}
  \endinput
\else
  % OK
\fi

%%%%%%%%
% opções
%-------------------------------
% Aqui, algumas opçoes ativa o if para efetuar configurações posteriormente

\RequirePackage{ifthen} % da base/required
\RequirePackage[english,brazilian]{babel} % da base/required (ultima é idioma principal)

\newboolean{ex@b@classe@usedsfont}

% se usedsfont for usado como opcao
\DeclareOption{usedsfont}{\setboolean{ex@b@classe@usedsfont}{true}}{\setboolean{ex@b@classe@usedsfont}{false}}

\DeclareOption{english}{ % idioma principal como english
\main@language{english}
% \selectlanguage{english}
}

\DeclareOption{brazilian}{ % idioma principal como brazilian
\main@language{brazilian}
% \selectlanguage{brazilian}
}

\DeclareOption{brazil}{ % idioma principal como brazilian
\main@language{brazilian}
% \selectlanguage{brazilian}
}

%% Fallback (opcoes nao declaradas)
\DeclareOption*{
  \PassOptionsToClass{\CurrentOption}{book}%
}
% Após declarar todas opções, colocar este comando
\ProcessOptions\relax

% lendo a classe base
\LoadClass{book}

%%%%%%%%%%%%%%%%%%%%%%%%%%%%%%%%%%%%%%
% Pacotes necessários (além de ifthen)

% Da base/required
\RequirePackage{amsthm, amssymb,amsmath}

% \RequirePackage[brazil]{babel}

% pacotes nao obrigatorios do LaTeX
% \RequirePackage{geometry} % carregar sempre

%\IfFileExists{dsfont.sty}% carregar, se existir
%{\RequirePackage{dsfont}}{}%

% testando se tem o pacote instalado e providenciando o
% comando, caso pacote nao exista.
\IfFileExists{hyperref.sty}%
{\RequirePackage{hyperref}}%
{\ClassWarning{ex-b-estilo}{hyperref.sty not found. Using draft mode.}%
 \providecommand\url[1]{\texttt{##1}}%
}

%%%%%%%%%%%%%%%%%%%%%%%%%%%%%%%%%%%%%%%%%
% Ambiente tipo teorema com Multi idiomas
% (ingles, portugues brasileiro)
% \addto do pacote babel adiciona comandos no final do macro existente.

\addto\extrasenglish{
  \engnames
}

\addto\extrasbrazilian{
  \brnames
}

\addto\extrasbrazil{
  \brnames
}
 
\newcommand\engnames{ % english names
  \def\axiomname{Axiom}
  \def\theoremname{Theorem}
  \def\corollaryname{Corollary}
  \def\lemmaname{Lemma}
  \def\propositionname{Proposition}
  \def\conjecturename{Conjecture}

  \def\definitionname{Definition}
  \def\notename{Note}
  \def\examplename{Example}
  \def\questionname{Question}
  \def\remarkname{Remark}
  \def\proofname{Proof}

  %  %%%% data abreviada: mes/dia/ano
  %  \def\todayabrev{%
  %    \ifnum \month < 10 0\fi%
  %    \number\month/
  %    \ifnum \day < 10 0\fi%
  %    \number\day/
  %    \number\year}

  \def\seename{see}
} % engnames


\newcommand\brnames{ % Brazilian names
  \def\axiomname{Axioma}
  \def\theoremname{Teorema}
  \def\corollaryname{Corol\'{a}rio}
  \def\lemmaname{Lema}
  \def\propositionname{Proposi\c{c}\~{a}o}
  \def\conjecturename{Conjectura}
  %
  \def\definitionname{Defini\c{c}\~{a}o}
  \def\notename{Nota}
  \def\examplename{Exemplo}
  \def\questionname{Exerc\'{\i}cio}
  \def\remarkname{Observa\c{c}\~{a}o}
  \def\proofname{Demonstra\c{c}\~{a}o}

  %%% data abreviada: dia/mes/ano
  %  \def\todayabrev{\ifnum \day < 10 0\fi%
  %    \number\day/\ifnum \month < 10 0\fi%
  %    \number\month/\number\year}

  \def\seename{veja}

  %  \typeout{brnames chamado}
} % brnames

%%%%%%%%%%%%%%%%%%%%%%%%%%%%%%%%%%%%%%%%%%%%%%%%
% definicao dos macros para ambiente de teoremas
% teoremas, lemas, proposicoes, etc

\theoremstyle{plain}

\newtheorem{theorem}{\theoremname}[chapter]
\newtheorem{axiom}[theorem]{\axiomname}
\newtheorem{corollary}[theorem]{\corollaryname}
\newtheorem{lemma}[theorem]{\lemmaname}
\newtheorem{proposition}[theorem]{\propositionname}
\newtheorem{conjecture}[theorem]{\conjecturename}

% definicao de definicoes, exemplos, etc.
\theoremstyle{definition}
\newtheorem{definition}[theorem]{\definitionname}

\theoremstyle{remark}
\newtheorem{remark}[theorem]{\remarkname}

\newtheorem{note}[theorem]{\notename}
\newtheorem{example}[theorem]{\examplename}
%
\newtheorem{question}[theorem]{\questionname}

% alterando as penalidades (para corte de linhas em...)
\hyphenpenalty=5000 % hifeniza\c{c}\~{a}o
\exhyphenpenalty=500 % palavras com hifem
\binoppenalty=3000 % operador binario (+, - , etc)
\relpenalty=2000 % operador relacional ( = \cong \ne, ...)
\clubpenalty=1000 % ???
\brokenpenalty=1000 % ???

% redefinição de sloppy
\def\sloppy{\tolerance=9999 \hfuzz=.5pt \vfuzz=.5pt}
\sloppy % prefere underfull do que overfull

% conjunto numerico
%------------------

\newcommand{\Rset}{\mathbb{R}}
\newcommand{\Cset}{\mathbb{C}}
\newcommand{\Zset}{\mathbb{Z}}
\newcommand{\Iset}{\mathbb{I}}
\newcommand{\Qset}{\mathbb{Q}}
\newcommand{\Nset}{\mathbb{N}}

% Funcoes em portugues (nome da funcao deve ser em romano)
\DeclareMathOperator{\sen}{sen}
\DeclareMathOperator{\arcsen}{arcsen}
\DeclareMathOperator{\senh}{senh}
\DeclareMathOperator{\arcsenh}{arcsenh}

%%%%%%%%%%%%%%%%%%%%%%%%%%%%%%%%%%%%%%
% Reconfigurando de acordo com a opção

% Usar dsfont
\ifthenelse{\boolean{ex@b@classe@usedsfont}}{% se usedsfont for usado como opcao
  \IfFileExists{dsfont.sty}%
  {%
    \RequirePackage{dsfont}
    \renewcommand{\Rset}{\mathds{R}}
    \renewcommand{\Cset}{\mathds{C}}
    \renewcommand{\Zset}{\mathds{Z}}
    \renewcommand{\Iset}{\mathds{I}}
    \renewcommand{\Qset}{\mathds{Q}}
    \renewcommand{\Nset}{\mathds{N}}  
  }{%
  \ClassError{ex-b-classe}{dsfont.sty not found.}%
    {Install dsfont package.}%
  }
}{%else
}% \ifthenelse{\boolean{ex@b@classe@usedsfont}}

% fim: ex-b-classe.cls
\end{lstlisting}
\end{singlespacing}
Note que alguns comandos de mensagem foi trocado da versão \verb+Package+ para a versão \verb+Class+ a fim de emitir mensagem correta.

O Exemplo~\ref{ex:b:classe:tex} é um exemplo do uso do pacote do Exemplo~\ref{ex:b:classe:cls}.

\begin{singlespacing}
\begin{lstlisting}[caption={ex-b-classe.tex}, label=ex:b:classe:tex]
% Usando a classe definida pelo usuário
\documentclass[a4paper,12pt]{ex-b-classe}
%\documentclass[a4paper,12pt,english]{ex-b-classe} %idioma inglês
% \documentclass[a4paper,12pt,usedsfont]{ex-b-classe} % com dsfont
\usepackage[T1]{fontenc} % codificação da fonte em 8-bits
\usepackage[utf8]{inputenc} % acentuação direta

\begin{document}
\chapter{Teste}

\begin{definition}\label{def:triangulo:retangulo}
Um triângulo é dito retângulo se tiver um ângulo reto.
\end{definition}

\begin{theorem}[Pitágoras]\label{thm:pitagoras}
Dado um triângulo retângulo $ABC$ com ângulo reto em $A$, temos que 
\begin{equation}\label{eq:pitagoras}
a^2=b^2+c^2
\end{equation}
\end{theorem}

\begin{proof}
Demonstração aqui.
\end{proof}

Usando a definição do $\sen$ e $cos$ juntamente com o Teorema~\ref{thm:pitagoras}, temos que

\begin{proposition}
\[
\forall t \in \Rset, \sen^2 t + \cos^2 t = 1
\]
\end{proposition}
\end{document}
\end{lstlisting}
\end{singlespacing}

A saída foi omitida por ser mesmo do Exemplo~\ref{ex:b:estilo:tex}.

\section{Preenchendo o documento para teste}
Para testar estilo de formatação, as vezes queremos preencher rapidamente o espaço.

Para isso existe o pacote \texttt{lipsum}\index{lipsum} que permite gerar parágrafos de texto em grego sem sentido.

O comando \verb+\lipsum+\index{lipsum@\verb+\lipsum+} gera em torno de uma página de texto.
O comando \verb+\lipsum[<n>]+ gera $n$-ésimo parágrafo de texto.
O comando \verb+\lipsum[<a>-<b>]+ gera o $a$-ésimo até $b$-ésimo parágrafo de texto.
Veja o Exemplo~\ref{ex:b:lipsum}.

%\begin{singlespacing}
%\begin{LTXexample}[pos=b,caption={ex11-11.tex},label=ex:11-11]
\begin{ltxlisting}[caption={ex-b-lipsum.tex},label=ex:b:lipsum]
% \lipsum % em torno de uma página
\lipsum[5] % quinto parágrafo
% \lipsum[2-3] % de parágrao 2 até parágrafo 3
\end{ltxlisting}
%\end{LTXexample}
%\end{singlespacing}

Note que \texttt{lipsum} gera somente parágrafos de textos.
Para testar fórmulas e listas também, poderá usar o pacote \texttt{blindtext}\index{blindtext} que tem o comando de mesmo nome que gera um texto.

O comando \verb+\blindtext[<n>]+\index{blindtext@\verb+\blindtext+} gera o texto $n$ vezes.
O comando \verb+\Blindtext+\index{Blindtext@\verb+\Blindtext+} gera um parágrafo.
O parâmetro opcional pode ser usado para repetir $n$ vezes.

Tem o comando \verb+\blindmathtrue+\index{blindmathtrue@\verb+\blindmathtrue+} que ativa as fórmulas no texto gerado, mas ele só funciona em inglês (idioma selecionado pelo babel). 
Assim, se não estiver em inglês, use o comando \verb+\blindmathpaper+\index{blindmathpaper@\verb+\blindmathpaper+} que gera parágrafos de textos e fórmulas.
Veja o Exemplo~\ref{ex:b:blindtext}.

%\begin{singlespacing}
%\begin{LTXexample}[pos=b,caption={ex11-11.tex, parte b},label=ex:11-11:b]
\begin{ltxlisting}[caption={ex-b-blindtext.tex},label=ex:b:blindtext]
% \blindmathtrue % funciona somente em ingles
% \blindtext % texto
% \Blindtext % Gera vários parágrafos
\Blindtext[1] % um parágrafo de textos
%\blinditemize % lista de itens
%\blindenumerate % lista de enumeracao
%\blinddescription % lista de descricao
%\blindmathpaper % texto e formulas.
%\blinddocument % documento com secao, subsecao, listas, etc. Nao inclui formulas
% \Blinddocument % documento longo, com texto com secao, subsecao, listas, etc. Este inclui formulas
\end{ltxlisting}
%\end{LTXexample}
%\end{singlespacing}

\clearpage\thispagestyle{empty}\cleardoublepage
\chapter{Usando o Editor \LyX}
Neste apêndice, será explicado o básico sobre o editor \LyX\index{LyX@\LyX}.
\LyX{} é um editor do tipo \texttt{WYSIWYG} (o que você vê é o que você obtém) e poderá editar o documento de forma similar ao aplicativo de escritório, mas processando o documento final com o \LaTeX.


\section{Visualização do documento final}

Para compilar/ver o documento, poderá usar o formatos comumente suportados
pelo \LaTeX{}.
O botão de ``visualizar'' (botão de olho, que fica
no lado esquerdo) é usado para visualizar o documento em PDF.

Se estiver usando o pacote que precisa ser compilado para \texttt{dvi} como o pacote gráfico \texttt{pstrick} e o \texttt{psfrag}, poderá escolher outro formato de visualização em ``visualizar outros formatos'' (botão que fica mais a direita, com olho e um pequeno triangulo).

O uso do pacote \texttt{hyperref} é recomendado para que o \LaTeX{} crie links automáticos, assim como o \texttt{bookmarks} (indicadores). 
Os usuários de Windows, deve considerar em evitar o uso do Adobe Reader na fase de elaboração, por ser pesado e apresentar dificuldade de interagir com os editores de \LaTeX. 
Uma boa opção é o freeware \texttt{SumatraPDF}\index{SumatraPDF}.

Lembre-se que o \LyX{} permite exportar/importar arquivos fonte em
\LaTeX, além de vários outros formatos na qual existem conversores
gratuitos do/para \LaTeX. 

\textbf{Dicas:} Para facilitar a navegação do documento em elaboração,
é bom ativar o ``Exibir->Painel de estrutura de tópicos''. 
Se usar o código \LaTeX{} inserido no documento com frequência, como o do
pacote \texttt{tikz}, seria bom ativar a pré-visualização na ``Ferramentas->Preferências'' e em ``Aparências \& Comportamento'', ítem ``Exibição'', na ``Pré-visualização Instantânea'', escolher ``sem matemática'', e colocar pré-visualização no bloco que contém a caixa de \LaTeX{} que julgar conveniente, com ``Inserir->Pré-visualização''.
Se quer que toda fórmula matemática tenha pré-visualização automática, poderá escolher ``ligado'' em vez de ``sem matemática''.
No entanto, as equações no modo ``Instant Preview'' ficará com a mesma cor do texto, tornando mais difícil de identificar rapidamente, o que não é recomendável.
Para ver como ficaria o código \LaTeX{} em edição, poderá ativar o ``View->View source''. 
Se o texto todo ficar sublinhado em vermelho, ou não sublinha a palavra errada, pode ser que o corretor ortográfico não está selecionado corretamente. 
Vá em ``Ferramentas->Preferências'' e em ``Configurações de Idiomas'', item ``Verificador Ortográfico'', escolha um verificador em ``verificador ortográfico''. 
Também certifique de que botão de ``verificar ortografia continuamente'' (Botão ABC sublinhado em vermelho) está ativa.

Depois de tudo pronto, ``file->export'' permite criar arquivo final
no formato desejado.

Note que a maioria dos editores de \LaTeX{} salva o documento antes de
compilar, mas o \LyX{} não faz isso. 
É necessário salvar manualmente sempre que convém.

\section{ABC do \LyX}

%O \LyX{} tem as ferramentas visuais e edição visual do documento,
%mas precisará conhecer a estrutura e o funcionamento do \LaTeX{}
%para usar corretamente. Também é aconselhável que tenha um pouco de
%conhecimento sobre como escrever fórmulas matemáticas no LaTeX. Para
%tanto, consulte o manual de LaTeX na medida do necessário.

O tutorial e guias do usuário do \LyX{} pode ser acessado pelo ``Help->Tutorial'' e ``Help->User's Guide'' respectivamente.

Também é importante notar que copiar/colar padrão funciona somente entre trechos de documentos do \LyX. 
Caso de colar trecho de código do outro aplicativo como navegador de internet ou editor de texto para o campo de comando \LaTeX{} (botão \TeX{} ou <ctrl>L), use o colar especial (editar->colar especial->texto simples) ou ctrl+shift+v, para não perder a quebra de linhas.

No lado esquerdo da barra de ferramentas, poderá escolher o tipo de texto (parágrafo). 
Alguns dos mais importantes são:

\begin{description}
\item[Standard] texto normal.

\item[Itemize] lista não enumerada.

\item[Enumerate] lista enumerada.

\item[Chapter] Título do capítulo

\item[Section] Título da seção.

\item[Subsection] Título da subseção.

\item[Lyx-code] O texto como foi escrito, inclusive quebra de linhas
(para colocar algoritmo, código fonte do programa, etc).
\end{description}

Uma das formas de colocar comentário no documento é através da nota inserido por ``insert->note'' que é adequado para notas mais longas por poder ``fechar'' a ``caixa de notas''.

As ``caixas'' como de notas, \TeX{}, Figure, etc tem um botão junto a ``caixa'' e clicando nele, poderá abrir ou fechar, para facilitar a editoração. % Experimente na ``caixa'' de nota acima (no \LyX{}).
O fato de caixa estar fechada ou aberta não influencia no processamento do documento final.

Como notas não serão impressas, também serve para desativar temporariamente o trecho do documento, selecionando e clicando em ``insert->note''.
Para desfazer a caixa de notas ou similares, use o ``disolve->inset'' acessível pelo botão direito sobre caixa de notas ou similares.

%Observemos que a versão com {*} como ``chapter{*} ``, ``section{*}``, etc não serão enumeradas e não vão para o sumário.

%Quando insere uma imagem externa, ele é colocado como referências
%(link). Isto quer dizer que a figura fica fora do documento (não dentro
%do arquivo como nos aplicativos Office). Isto quer dizer que as imagens
%não devem ser apagadas e quando copiar o documento \LyX{}, deverá
%copiar imagens juntas. No documento final em PDF, estas imagens serão
%convertidas e inseridas dentro do documento PDF.

Apesar de \LyX{} efetuar conversão automática de imagens, quando pretende exportar o código para \LaTeX, é bom deixar convertido para pdf/jpg/png (para PDF\LaTeX{} usado atualmente) ou eps (para \LaTeX{} antigo) a fim de evitar problema de compatibilidade de imagem com o código fonte em (PDF)\LaTeX{}.

\textbf{Acertando a configuração do documento}

No ``Documento'', escolha a classe de documento, margens e tamanho
das fontes.

Quando compila um documento, o \LyX{} costuma colocar uma opção de
usar o pacote \texttt{amsmath}, mas no caso de inserir o comando de \TeX{}
diretamente, pode não ser detectado a necessidade de \texttt{amsmath}.

Para resolver este caso, entre em ``Document->settings'' e no ``math Package'', tire o check do ``use amsmath package automatically'' e cheque em ``use ams package''.
O problema similar pode acontecer quanto inclui figuras, usar cores, etc dentro da caixa de \LaTeX, sem estar usando fora dela. 
Neste caso, carregue os pacotes necessários, colocando o comando \LaTeX{} \verb+\usepackage+ no \texttt{preamble} (Document->Settings, [LaTeX Preamble]).

Configuração do documento como estilo, margens, etc podem ser alterado
pelo ``Document->Settings'' em qualquer momento.

Quando colar trecho de um documento para outro, pode ficar com a especificação da linguagem diferente e começar a ser sublinhado. 
Neste caso, selecione este trecho e no ``text style'' (botão com letras ``ab''), escolha o ``reset'' na opção de [linguagem].

\textbf{Inserindo o comando de \LaTeX}

Para colocar pacotes adicionais, definições, configurações adicionas,
etc que ainda não está disponível no \LyX{} (ou que não tem paciência
de descobrir como fazer), deverá ser colocado no campo de ``preamble''
em ``Document->settings, [LaTeX Preamble]'', usando o comando
de \LaTeX. 
No preamble, deverá usar acentuação no modo \TeX{} para evitar problemas de portabilidade.

Quando precisar colocar um comando extra de \LaTeX{} no documento, clique no botão ``\TeX'' ou pressione <ctrl>L na qual insere uma caixa de comando \LaTeX, podendo colocar qualquer comando válido de \LaTeX. 
Para evitar problemas, coloque um espaço (ou par de chaves) no final do campo de comandos de \LaTeX, caso finalizar com o comando \LaTeX.

O comando \LaTeX{} pode ficar divididos. 
Por exemplo, uma caixa pode conter \verb+\begin{ambiente}+ e outra caixa pode conter \verb+\end{ambiente}+, para colocar trecho do código no ambiente desejado.

Note que o campo de fórmulas também aceitam diretamente os comandos
de \LaTeX{}.

\textbf{Para letras acentuadas fora do corpo do documento}

O campo exibido na janela separada como no caso do preamble (Document->Settings, [LaTeX preamble]), entrada de índice remissivo (insert->index entry), etc usam a exibição nativa do sistema.
Assim poderá apresentar problemas na exibição ou compilação quando muda de sistema operacional.
Para evitar problemas de portabilidade, use a acentuação pelo comandos
\TeX{} nestes casos.
Caixa de código \LaTeX{} (inserido com botão \TeX{}) não apresenta este tipo de problemas.


\section{Formatando textos}

Para alterar o tipo de fontes do texto, selecione o trecho e clique
no botão ``\emph{abc}''.
Nele tem a opção de escolher a família, forma e peso.


%\textbf{Family:} roman (serifado) é usado para o corpo do documento;
%sans serif (sem serifa) pode ser usado para títulos; typewriter (maquina
%de escrever) é usado para trecho que deve ser lido no pé da letra
%(código fonte do programa, nome de arquivos, etc).
%
%\textbf{Series:} medium (normal) e bold (negrito). Tecla de atalho
%para negrito é <control>b.
%
%\textbf{Shape:} upright (reto) usado para corpo do documento; italic
%(itálico) usado para destacar palavra a ser definido ou corpo do enunciado
%do teorema, etc; slanted (inclinado, mas não é itálico) que deve ser
%evitado o seu uso quando está usando o itálico; small caps (caixa
%alta) útil para títulos.
%
%\textbf{size:} Ele esta na ordem crescente, onde normal é o tamanho
%normal do documento.
%
%\textbf{misc:} emph (enfatizar) usado para enfatizar palavra dentro
%do texto, alterando entre fontes itálico e reto (botão de ``E''
%na barra de ferramenta - tecla de atalho é <ctrl>e); underbar (sublinhada);
%noun é maiúsculo.

O \texttt{emph} fica na opção \texttt{misc} e \texttt{language} permite escolher idioma neste trecho.
Note que <cntrol>b é atalho para negrito e <cntrl>e é atalho para enfatizar.

% 
%
%\textbf{language:} linguagem em que o texto está escrito. A hifenização,
%corretor ortográfico, etc é efetuado com a linguagem especificada. 

A opção ``reset'' de cada item restaura o item correspondente como padrão do
documento.

O botão ``A com seta redonda'' aplicará a última configuração. 
Para saber o que cada botão faz, posicione o mouse sobre o botão e deixe
parado para aparecer o texto informativo.

Para criar um novo parágrafo ou linhas:

<ENTER> cria novo parágrafo

<control>+<ENTER> quebra de linha (sem parágrafo)

<control>+<space> espaço forçado (\verb+~+ de \LaTeX).

Note que, para inserir linhas vazias, deve alterar (quebras de linhas
ou parágrafo) com o espaço forçado.

%\texttt{~}
%
%\texttt{~}
%
%\texttt{~}
%
%\texttt{~}
%
%Notas de rodapé\footnote{Notas de rodapé já ficará com espaço simples como exigido, por estar
%redefinido no arquivo de estilo. Também foi eleminado indentação para
%caber mais, apesar de não ser a exigência} pode ser inserido pelo ``insert->footnote''. 

\section{Matemática\label{sec:matematica}}

Fórmulas matemáticas dentro do texto denominado de \texttt{inlinestyle}
(textstyle) pode ser inserido, clicando no ícone de somatório $\sum$ na barra de ferramentas, ou com menu ``insert->math->inline formula'' ou
com a tecla <control>M. 

Para inserir símbolos e fórmulas, podemos usar o painel matemático
ativado quado o cursor está dentro da caixa de fórmulas.
Para criar fórmulas de maneira rápida, é recomendado memorizar alguns comandos básicos de \LaTeX{} que podem ser digitados (em vez de clicar no
mouse). 

%Alguns deles são
%
%\texttt{x\textasciicircum{}2<seta para direia>\_i} $x_{i}^{2}$
%
%\texttt{\textbackslash{}int\_a<seta para direita>\textasciicircum{}b}
%$\int_{a}^{b}$
%
%\texttt{\textbackslash{}frac a<seta para baixo>b} $\frac{a}{b}$
%
%\texttt{\textbackslash{}alpha} $\alpha$
%
%\texttt{\textbackslash{}beta} $\beta$
%
%\texttt{\textbackslash{}pi} $\pi$
%
%\texttt{\textbackslash{}sum\_i=1<seta para direita>\textasciicircum{}n}
%$\sum_{i=1}^{n}$
%
%\texttt{\textbackslash{}infty} $\infty$
%
%\texttt{\textbackslash{}partial} $\partial$
%
%\texttt{\textbackslash{}ge} $\ge$
%
%\texttt{\textbackslash{}le} $\le$
%
%\texttt{\textbackslash{}ne} $\ne$
%
%\texttt{\textbackslash{}cos x} $\cos x$
%
%\texttt{\textbackslash{}sin x} $\sin x$
%
%\texttt{\textbackslash{}ln x} $\ln x$
%
%\texttt{\textbackslash{}sup x} $\sup x$
%
%\texttt{\textbackslash{}lim x} $\lim x$
%
%\textbackslash{}sqrt x $\sqrt{x}$
%
%\textbackslash{}root n <seta para direita> x $\sqrt[n]{x}$. 
%

Note que o \LyX{} tem dificuldade em lidar com o parâmetro opcional
no modo matemático, dificultando o uso da forma \verb+\sqrt[n]{x}+.
Para contornar, foi definido o comando \verb+\root+ no \LyX{} que não é definido por padrão no \LaTeX{}.

Para descobrir nomes de comandos \LaTeX{} que estão no ``Painel Matemático'', posicione o mouse sobre os símbolos no ``Math Panel'' (mostrado quando o cursor está na fórmula) que exibirá o nome do comando.

Uma fórmula numa linha independente denominado de ``displaystyle''
pode ser criado com Control+Shift+M ou pelo menu ``insert->math->Display
formula''.

Para sair rapidamente de uma fórmula, pressione o ``ESC'' que posicionará o cursor logo após a fórmula em edição.

%A posição dos limites no ``inlinestyle'' não ficam embaixo/encima
%para economizar altura da fórmula, como em $\sum_{n=1}^{\infty}\frac{1}{n}=\infty$.
%Se quer que estas e outras exibições no ``inlinestyle'' fiquem igual
%ao do ``displaystyle'', poderá usar o comando do \LaTeX{} \textbackslash{}displaystyle.
%Digite \textbackslash{}displaystyle seguido de espaço. Na ``caixinha''
%criada, a fórmula digitado ficará no modo ``displaystyle'' como
%em ${\displaystyle \sum_{n=1}^{\infty}\frac{1}{n}=\infty}$ mesmo
%no meio do texto.

Para converter tipo de fórmulas já digitadas, clique o botão direito
sobre a fórmula e escolha o novo tipo de fórmula. 
Também pode posicionar o cursor na fórmula e use o ``edit->math->change formula type''.

A função por partes pode ser criado com ``cases'', inserido com
botão direito do mouse sobre a fórmula, ou pelo menu ``insert->math->cases''.

Nas fórmulas multi linhas, <control>+ENTER abre uma nova linha.

%$|x|=\begin{cases}
%x & ,x\ge0\\
%-x & ,x<0
%\end{cases}$

%Para inserir texto no meio da fórmula, use o comando \textbackslash{}text
%
%\begin{eqnarray*}
%f(x) & = & \begin{cases}
%x^{2}, & \text{ se }x>0\\
%-x^{2}, & \text{ se }x\le0
%\end{cases}
%\end{eqnarray*}
%
%Poderá inserir parentes ou similares usando o comando \textbackslash{}left
%e \textbackslash{}right em vez do painel.
%
%Por exemplo, ''\textbackslash{}left. x\textasciicircum{}2 \textbackslash{}right
%{]}`` produz $\left.x^{2}\right]$. As vezes, isto torna mais rápido
%do que usar o painel de inserção de delimitadores.

As fórmulas que ocupa uma linha independente tais como ``Display fórmula'' (fórmula no modo ``displaystyle''), ``AMS align environment'',
etc podem ser enumerados automaticamente.

Para ativar/desativar enumeração na equação com linha independente, clique no botão direito do mouse sobre a fórmula, ou use o menu ``edit->math->toggle numbering''.

Para referenciar equações ou elementos enumerados,
% como
%
%\begin{equation}
%a^{2}=b^{2}+c^{2}\label{eq:pitagoras}
%\end{equation}
use o botão de etiqueta para inserir rótulo e referência cruzada (que
fica no lado) para inserir referências.% \ref{eq:pitagoras}.

\section{Observação adicional}
Em geral, o estilo ABNTeX2 não costuma vir instalado no LyX.

Para instalar, siga as instruções do site oficial
\url{https://github.com/abntex/abntex2/wiki/LyX}.

\clearpage\thispagestyle{empty}\cleardoublepage
\chapter{Para Organizadores do Evento}

Aqui, veremos alguns pacotes úteis para organizadores de eventos. 
Um deles é o pacote para mala direta no \LaTeX, útil para emitir certificados e outro é para criar caderno de trabalhos apresentados.

\section{Mala direta no \LaTeX}
Certificados de congressos científicos pode conter fórmulas nos títulos, o que complica a mala direta nos aplicativos de escritórios.
Assim, costumamos efetuar a mala direta no \LaTeX.

A mala direta pode ser efetuada pelo \texttt{datatool}\index{datatool} que lê os dados de um arquivo \texttt{CSV}. 
Após carregar o \texttt{datatool}, comece configurando o separador de colunas e delimitador de campos.
\begin{singlespacing}
\begin{lstlisting}
\DTLsetseparator{,} % Separador de campos
\DTLsetdelimiter{"} % delimitador de celulas
\end{lstlisting}
\end{singlespacing}

Para acessar o arquivo \texttt{CSV}, usa-se o comando
\verb+\DTLloaddb{lista}{ex-c-lista-nomes.csv}+\index{DTLloaddb@\verb+\DTLloaddb+} para associar o arquivo \texttt{ex-c-lista-nomes.csv} no atalho \texttt{lista}. 
Assim, ao referenciar \texttt{lista}, estará referenciando o arquivo \texttt{ex-c-lista-nomes.csv}

A primeira linha do arquivo \texttt{ex-c-lista-nomes.csv} é assumido que é título, isto é, contém nome das colunas.
Caso não tenha linha de título, poderá providenciar como em

\verb+\DTLloaddb[noheader,keys={Nome,Trabalho}]{lista}{lista.csv}+. 

Neste caso, está indicando que não há linha de título no arquivo e está colocando nomes \texttt{Nome} e \texttt{Trabalho} na primeira e segunda coluna do arquivo de dados.

Agora podemos efetuar um laço de repetição que é algo como
\begin{singlespacing}
\begin{lstlisting}
\DTLforeach{lista}{%
\personname=Nome,\worktitle=Trabalho}{%
...
} % \DTLforeach
\end{lstlisting}
\end{singlespacing}

O primeiro parâmetro do comando \verb+\DTLforeach+\index{DTLforeach@\verb+\DTLforeach+} é o atalho para arquivo \texttt{CSV} que foi criado pelo comando \verb+\DTLloaddb+.
O segundo parâmetro é a lista de associação de comandos com valor do campo.
\verb+\personname=Nome+ e \verb+\worktitle=Trabalho+ associa o campo (dado da coluna) \texttt{Nome} para o comando \verb+\personname+ e o campo \texttt{Trabalho} para o comando \verb+\worktitle+.
O terceiro parâmetro são comandos a serem executados.
O arquivo de entrada e como do Exemplo~\ref{ex:d:certificado:csv}

\begin{singlespacing}
\begin{lstlisting}[caption={ex-d-lista-nomes.csv},label=ex:d:certificado:csv]
"Nome","Trabalho","Observacao"
"Nome 1","Trabalho 1",
"Nome 2","Trabalho 2",
"Nome 3","Trabalho 3",
...
\end{lstlisting}
%\begin{center}
%\lstlistingname~\refstepcounter{lstlisting}\thelstlisting: lista-nomes.csv\label{ex:c:certificado:csv}
%\end{center}
%\begin{framed}
%\begin{verbatim}
%"Nome","Trabalho","Observacao"
%"Nome 1","Trabalho 1",
%"Nome 2","Trabalho 2",
%...
%\end{verbatim}
%\end{framed}
\end{singlespacing}

Agora, o certificado pode ter enfeites como molduras, logotipos, etc.

Ele será diagramado para papel \texttt{A5}, para ser impresso no papel \texttt{A4}, ampliando-o. Fonte também será alterado para \texttt{bookman} que é uma serifa grossa, ideal para letras maiores como este.
Veja o Exemplo~\ref{ex:d:certificado}.

\begin{singlespacing}
\lstinputlisting[caption={ex-d-certificado.tex},label=ex:d:certificado]{latex-via-exemplos-certificado}
\end{singlespacing}
%
Um dos certificados gerados é como segue.
\begin{framed}
\center
\includegraphics[page=1, width=0.7\linewidth]{latex-via-exemplos-certificado}
\end{framed}

% Note que, o que deixou o código complexo foi o sombreamento.

\section{Caderno de resumos}
Quando recebe vários artigos e quer ``encadernar'', poderá usar a classe \texttt{combine}\index{combine} em vez de gerar PDF de cada um e grudar.
A classe \texttt{combine} agrupa os artigos automaticamente em um único documento, gerando um ``caderno'' de artigos.

Para usar o \texttt{combine}, todos documentos que serão incluídos nele devem estar usando a mesma classe que podem ser \texttt{memoir}, \texttt{book}, \texttt{report}, \texttt{letter} ou \texttt{article}. 
As classes não suportadas oficialmente como o \texttt{amsart} pode precisar de ajustes manuais para funcionar perfeitamente.

%Para \texttt{combine}, o papel está fixado. 
%Assim, deverá usar o pacote \texttt{geometry} para efetuar ajustes de papel, se necessário.

A classe \texttt{combine} requer mais recursos que o documento normal.
Assim, para aumentar a memória reservada para isso, coloque
\begin{singlespacing}
\begin{lstlisting}
\usepackage{etex} % to use etex extensions 
\reserveinserts{28} % using etex extensions
\end{lstlisting} 
\end{singlespacing}
no preamble, caso ocorrer o erro do tipo ``No room for ...''.
Se ainda ocorrer erro, aumente o valor de $28$ para $50$, por exemplo.

Note que \verb+\maketitle+ ou \texttt{titlepage} é necessário. 
Se não existir, causa erros.

Para gerar sumário de artigos incluídos, carregue o pacote \texttt{combinet}\index{combinet}.
Para usar recursos do \texttt{natbib}, carregue o pacote \texttt{combnat}\index{combinat} no lugar de \texttt{natbib}.
 
O modo padrão é ignorar os pacotes carregados pelos artigos. 
Assim, todos pacotes necessários devem estar carregados no arquivo mestre (arquivo do \texttt{combine} que vai incluir artigos).

Vamos supor que artigos a serem incluídos tem a forma como no Exemplo~\ref{ex:d:conferencia}.

\begin{singlespacing}
\begin{lstlisting}[caption={ex-d-conferencia1.tex},label=ex:d:conferencia]
documentclass[10pt,a4paper]{article}
% Este arquivo sera processado automaticamente por um programa.
% Por favor nao altere nada no preamble (ate \begin{document} ).
\usepackage[utf8]{inputenc}
\usepackage[english,brazil]{babel}
\usepackage{amsmath}
\usepackage{amssymb} % Ele carrega amsfonts tambem
\usepackage{enumerate}
% Por favor nao inclua outros "packages".
\begin{document}
\title{Título da Conferência 1}
\author{Autor da conferência 1}
\maketitle

Texto da conferência 1
% Incluir bibliografia e opcional. Se decidir inclui-la use o formato abaixo

%\begin{thebibliography}{99}
%\bibitem{key1} Lamport, L. \emph{LaTeX: A Document Preparation System}, Addison-Wesley, \textbf{1986}.
%\end{thebibliography}
\end{document}
\end{lstlisting}
\end{singlespacing}

Então o caderno, pode ser criado como no Exemplo~\ref{ex:d:caderno}.

\begin{singlespacing}
\begin{lstlisting}[caption={ex-d-caderno.tex},label=ex:d:caderno]
\documentclass[12pt,a5paper]{combine}
% combine supports: memoir, book, report, and article

%%%%%%%%%%%%%%%%%%%%%%%%%%%%%%%%%%%%%%%%%%%
% If the error ``No room for ..'' occur, increase the reserved insets
\usepackage{etex} % to use etex extensions 
\reserveinserts{28} % using etex extensions

% To create table of cotents automatically
\usepackage{combinet}

% Todos pacotes usados nos artigos a ser importados
\usepackage[utf8]{inputenc}

\usepackage{amsmath}
\usepackage{amssymb} % it's include amsfonts too
\usepackage{enumerate}
\usepackage[english,brazil]{babel}

% packages usados somente no corpo de anais do congresso 
\usepackage{graphicx}
\usepackage{geometry}
\geometry{lmargin=2.5cm,rmargin=2cm,tmargin=2cm,bmargin=2cm}

\sloppy % prefer underfull
% \fussy % prefer overfull

% No combine, chamada de \maketitle é obrigatorio.
% Assim, quando faz a capa manualmente, deverá desabilitar isto
\makeatletter\c@lmtitlempty\makeatother

\begin{document}
% capa
\begin{titlepage}
\begin{center}
  \thispagestyle{empty}
  \begin{flushleft}
    \noindent
    \unitlength=0.04\textwidth
    \begin{picture}(0,0)(2,33)
      % \includegraphics[width=1.16\textwidth]{fundo}
    \end{picture}
  \end{flushleft}
  {\Large T\'ITULO DO CONGRESSO} \\
  HOMENAGEM, ETC.
 
  \vfill
  {\huge Resumos de apresentações \\
  (Abstract of talks and posters)}
  \vfill
  DATA
  \vfill
  LOCAL
  \vfill
\end{center}
\end{titlepage}

% contra capa
\clearpage
\thispagestyle{empty}
\noindent
{\bfseries Comitê Científico (Scientific Comittee)} \\
Membro 1 (Instituição 1) \\
Membro 2 (Instituição 2)

\

\noindent
{\bfseries Comissão Organizadora (Organizing Comittee)} \\
Membro 1 (Instituição 1) \\
Membro 2 (Instituição 2)

\

 \begin{flushright}
   \begin{minipage}{0.9\textwidth}
   Realização (Hosted by): \\
   INSTITUIÇÕOES
   \end{minipage}
 \end{flushright}

\clearpage
\newpage

% Table of contents
% \pagestyle{combine}
\pagestyle{plain}

\tableofcontents
\clearpage
~ % um espaço para gerar página não vazia
\newpage % pulando uma página

% Importa artigos de conferências
% Usar o comando \import em vez de \input ou \include
\begin{papers}
\import{conferencia1}
\import{conferencia2}
...
\end{papers}

\end{document}
\end{lstlisting}
\end{singlespacing}

Note que foi diagramado no papel \texttt{A5} para poder efetuar \texttt{booklet}\index{booklet} (colocar duas páginas em uma, reordenar para grampear no meio) no papel \texttt{A4}. 
O processo de \texttt{booklet} costuma ser feito no aplicativo como o \texttt{jPDFTWeak} (\url{http://jpdftweak.sourceforge.net/}).

%certificados.

%caderno de resumos 

\clearpage\thispagestyle{empty}\cleardoublepage
\chapter{Para Professores}

Será tratado algumas questões úteis aos professores.

\section{Lista de exercícios e provas}

Nas provas ou listas de exercícios, os itens podem ser curtas.
Neste caso, é aconselhável que coloque mais de um item por linha, mas mantendo o alinhamento.

Existem vários pacotes para elaboração de lista de exercícios e provas.
Aqui será apresentado uma delas.
%O pacote \texttt{exsheets}\index{exsheets} implementa o ambiente \texttt{question}\index{question} para produzir questionário enumerado 
O pacote \texttt{xsim}\index{xsim} é sucessor de \texttt{exsheets}\index{exsheets}.
Este pacote implementa o ambiente \texttt{exercise}\index{exercise} para produzir exercícios enumerados (e o ambiente \texttt{solution} para escrever soluções)
e o pacote \texttt{tasks}\index{tasks} permite colocar vários itens enumeradas e alinhadas em uma única linha (como é feito pela combinação dos pacotes \texttt{enumitem} e \texttt{tabto}).

%Veja o Exemplo~\ref{ex:e:exsheets}.
%
%\begin{singlespacing}
%%\begin{LTXexample}[pos=b,caption={ex12-10.tex},label=ex:12-10]
%\begin{lstlisting}[caption={ex-e-exsheets.tex},label=ex:e:exsheets]
%\documentclass[12pt,a4paper]{article}
%\usepackage[T1]{fontenc}
%\usepackage[utf8]{inputenc}
%\usepackage[brazil]{babel}
%\usepackage{tasks} % cria lista curta
%\usepackage{exsheets} % cria questoes
%\SetupExSheets[question]{type=exam} % configurando
%
%\begin{document}
%% O ambiente question produz questões.
%
%\begin{question}
%  Qual item não é metal?
%   % itens da questão será produzido pelo ambiente tasks
%  \begin{tasks}(4) % 4 itens por linha
%    \task ferro
%    \task carbono
%    \task cobre 
%    \task mercúrio
%  \end{tasks}
%\end{question}
%
%% Alterando a enumeração de tasks
%\settasks{
%  counter-format=(tsk[r]), % entre parenteses, em romano minusculo
%  label-width=4ex % largura reservada para rótulo
%}
%
%\begin{question}
%  O que é mamífero?
%  \begin{tasks}(2) % 2 itens por linha
%    \task Animais que botam ovos.
%    \task Animais que amamentam.
%    \task Animais que voam.
%    \task Animais com 4 patas.
%  \end{tasks}
%\end{question}
%\end{document}
%\end{lstlisting}
%%\end{LTXexample}
%\end{singlespacing}
%%
%\begin{ltxlistingout}
%\begin{question}
%  Qual item não é metal?
%
%%  \begin{tasks}(4) % 4 itens por linha coloca ultimo item na linha seguinte
%% Isto acontece somente dentro do minipage e com o uso do geometry   
%  \bigskip  
%~~ a) ferro \hfill
%b) carbono \hfill
%c) cobre \hfill
%d) mercúrio \hfill ~
%\end{question}
%
%% Alterando a enumeração de tasks
%\settasks{
%  counter-format=(tsk[r]), % entre parenteses, em romano minusculo
%  label-width=4ex % largura reservada para rótulo
%}
%
%\begin{question}
%  O que é mamífero?
%  \begin{tasks}(2) % 2 itens por linha
%    \task Animais que botam ovos.
%    \task Animais que amamentam.
%    \task Animais que voam.
%    \task Animais com 4 patas.
%  \end{tasks}
%\end{question}
%\end{ltxlistingout}

Veja o Exemplo~\ref{ex:e:xsim}.

\begin{singlespacing}
%\begin{LTXexample}[pos=b,caption={ex12-10.tex},label=ex:12-10]
\begin{lstlisting}[caption={ex-e-exsheets.tex},label=ex:e:xsim]
\documentclass[12pt,a4paper]{article}
\usepackage[T1]{fontenc}
\usepackage[utf8]{inputenc}
\usepackage[brazil]{babel}
\usepackage{tasks} % cria lista curta
\usepackage{xsim} % cria exercicios (e respostas)
% traduzindo
\DeclareExerciseTranslation{brazil}{exercise}{Exercício}
\DeclareExerciseTranslation{brazil}{exercises}{Exercícios}
% necessário se for colocar soluções também
\DeclareExerciseTranslation{brazil}{solution}{Solução}
\DeclareExerciseTranslation{brazil}{solutions}{Soluções}

\begin{document}
% O ambiente exercise produz questões.

\begin{exercise}
  Qual item não é metal?
   % itens da questão será produzido pelo ambiente tasks
  \begin{tasks}(4) % 4 itens por linha
    \task ferro
    \task carbono
    \task cobre 
    \task mercúrio
  \end{tasks}
\end{exercise}

% Alterando a enumeração de tasks
\settasks{
  label=(\alph*), % entre parenteses, em romano minusculo
  label-width=4ex % largura reservada para rótulo
}

\begin{exercise}
  O que é mamífero?
  \begin{tasks}(2) % 2 itens por linha
    \task Animais que botam ovos.
    \task Animais que amamentam.
    \task Animais que voam.
    \task Animais com 4 patas.
  \end{tasks}
\end{exercise}
\end{document}

\end{document}
\end{lstlisting}
%\end{LTXexample}
\end{singlespacing}
%
\begin{ltxlistingout}
\subsection*{Exercício 1}
  Qual item não é metal?

%  \begin{tasks}(4) % 4 itens por linha coloca ultimo item na linha seguinte
% Isto acontece somente dentro do minipage e com o uso do geometry   
  \bigskip  
~~ a) ferro \hfill
b) carbono \hfill
c) cobre \hfill
d) mercúrio \hfill ~

% Alterando a enumeração de tasks
\settasks{
  label=(\alph*), % entre parenteses, em romano minusculo
  label-width=4ex % largura reservada para rótulo
}

\subsection*{Exercício 2}
  O que é mamífero?
  \begin{tasks}(2) % 2 itens por linha
    \task Animais que botam ovos.
    \task Animais que amamentam.
    \task Animais que voam.
    \task Animais com 4 patas.
  \end{tasks}
\end{ltxlistingout}

\section{Com e sem respostas}
As vezes, queremos gerar versões com resposta e outro sem, ou que resposta fique no final do documento, mas que seja diagramado junto as questões.
Neste caso, podemos usar o pacote \texttt{answers}\index{answers} (existem vários pacotes para tal propósito).
Note que o pacote \texttt{xsim} pode fazer mesma tarefa, mas o pacote \texttt{answers} pode ser combinados com ambientes diversos, como com o ambiente gerado pelo \verb+\newtheorem+.
Veja o Exemplo~\ref{ex:e:answers}.

%%\setcounter{lstorichapter}{\value{chapter}}
%\setcounter{lstorisection}{\value{section}}
%%\setcounter{chapter}{0}
%\setcounter{section}{0}
\let\orithesection\thesection
\renewcommand{\thesection}{\arabic{section}}
%\begin{singlespacing}
%\begin{LTXexample}[pos=b,caption={ex12-11.tex},label=ex:12-11,preset=\let\addcontentsline\emptyaddcontentsline]
\begin{ltxlisting}[caption={ex-e-answers.tex},label=ex:e:answers]
\documentclass[12pt,a4paper]{article}
\usepackage[T1]{fontenc}
\usepackage[utf8]{inputenc}
\usepackage[brazil]{babel}

\usepackage{answers} % resposta no arquivo separado
%\usepackage[nosolutionfiles]{answers} % resposta no lugar

\usepackage{amsthm}

% "sol" é ambiente a ser usado. "Solution" não deve ser modificada. A resposta sera gravada no arquivo referenciado internamente por "ans"
\Newassociation{sol}{Solution}{ans}
% \renewcommand{\solutionextension}{ans}

\newtheorem{ex}{Exercicío}[section] % ambiente de exercicios

\begin{document}
% inicia a gravacao da resposta no arquivo "ans" cuja nome externo e \jobname-ans (\jobname e nome do arquivo atual)
\Opensolutionfile{ans}[\jobname-ans]

\section{Problemas}

\begin{ex}
Primeiro exercício.
\begin{sol}
Solução do primeiro exercício.
\end{sol}
\end{ex}

\begin{ex}
Segundo exercício.
\begin{sol}
Solução do segundo exercício.
\end{sol}
\end{ex}
\Closesolutionfile{ans} % finaliza a gravação das respostas

\section{Soluções}
\Readsolutionfile{ans} % colocar solucoes do "ans" aqui.
\end{document}
\end{ltxlisting}
%\end{LTXexample}
%\end{singlespacing}
%%\setcounter{chapter}{\value{lstorichapter}}
%\setcounter{section}{\value{lstorisection}}
\let\thesection\orithesection
%\let\addcontentsline\oriaddcontentsline

%shortlst não, por causa da licenca.

%\usepackage{tasks}.
%\usepackage{exsheets} usar o sucessor xsim
%\usepackage{answers} % put answer in the separated file
%%\usepackage[nosolutionfiles]{answers} % put answer in the place

\clearpage\thispagestyle{empty}\cleardoublepage
\chapter{Para Projetos}

Aqui será tratado de pacotes úteis ao desenvolvimento de documentos grandes como no caso de livros.

\section{Pacote \texttt{standalone}}
Quando cria muitas ilustrações e diagramas, as vezes deixamos estas ilustrações como arquivo separado para compilar e ajustar mais rapidamente.
Por exemplo, cada ilustração em \texttt{tikz} pode estar em um único arquivo para que o documento principal inclua eles.
Neste caso, não podemos usar \verb+\input+, nem \verb+\include+ pois a ilustração tem a sua própria classe de documento.
Neste caso, podemos usar a classe \texttt{standalone}\index{standalone}.
Cada ilustração de \texttt{tikz} usará a classe \texttt{standalone} em vez do \texttt{article} ou similar.
Assim, podemos compilar e ajustar a ilustração quando bem entender.
No documento principal, use o pacote \texttt{standalone}\index{standalone} e demais pacotes que arquivos inclusos podem precisar.
Para incluir a ilustração feito pela classe \texttt{standalone}, basta usar \verb+\input+\index{input@\verb+\input+} ou similar.

A figura diagramado com a classe \texttt{standalone} é como do Exemplo~\ref{ex:f:standalone:fig}.

%\begin{singlespacing}
%\begin{LTXexample}[pos=b,caption={ex12-12-fig.tex},label=ex:12-12:fig]
\begin{ltxlisting}[caption={ex-f-standalone-fig.tex},label=ex:f:standalone:fig]
\documentclass[12pt,a4paper]{standalone}
\usepackage[utf8]{inputenc}
\usepackage{tikz}
\usetikzlibrary{patterns}
\begin{document}
\begin{tikzpicture}
\draw[pattern=north east lines] (-1,-1) rectangle(1,1);
\draw[fill=white] (0,0) circle(1);
\end{tikzpicture}
\end{document}
\end{ltxlisting}
%\end{LTXexample}
%\end{singlespacing}

O documento que usa a figura, usa o pacote \texttt{standalone} e usa o \verb+\input+ para incluir o arquivo de figura, como no Exemplo~\ref{ex:f:standalone}.

\begin{singlespacing}
\begin{lstlisting}[caption={ex-f-standalone.tex},label=ex:f:standalone]
\documentclass[12pt,a4paper]{article}

\usepackage{standalone} % para incluir arquivos diagramado com standalone

\usepackage[T1]{fontenc}
\usepackage[utf8]{inputenc}
\usepackage[brazil]{babel}

\usepackage{amsmath}
\usepackage{amssymb}

% pacotes usados pelo arquivo a ser incluido
\usepackage{tikz}
\usetikzlibrary{patterns}

\begin{document}
Para incluir o arquivo diagramado com classe \texttt{standalone}, basta usar o \verb+\input+.

\begin{figure}[hbp!]
\center
\input{ex-f-standalone-fig}
\caption{Inclusão de figuras}
\end{figure}

\end{document}
\end{lstlisting}
\end{singlespacing}
%
%\begin{framed}
\begin{ltxlistingout}
Para incluir o arquivo diagramado com classe \texttt{standalone}, basta usar o \verb+\input+.
\begin{center}
\begin{tikzpicture}
\draw[pattern=north east lines] (-1,-1) rectangle(1,1);
\draw[fill=white] (0,0) circle(1);
\end{tikzpicture}

{Figura 1: Inclusão de figuras}
\end{center}
\end{ltxlistingout}
%\end{framed}

\section{Dividindo o documento em vários arquivos}
Para dividir o documento em vários arquivos, poderá usar o \verb+\include+, mas ele não é eficiente, pois os pedaços de documentos não podem ser compilados separadamente.
O pacote \texttt{subfiles}\index{subfiles} resolve este problema.
O uso de \texttt{subfiles} é similar ao do \texttt{standalone}, mas em vez de ignorar o \texttt{preamble} dos arquivos inclusos, os arquivos a serem incluídos usam o \texttt{preamble} do documento principal. 
Assim, como o \texttt{standalone}, o documento principal deve conter todos os pacotes e definições que suas partes vão precisar, mas não há necessidade de colocar tais pacotes e definições no arquivo de partes.
A estrutura do documento principal com \texttt{subfiles} é algo como do Exemplo~\ref{ex:f:subfiles:principal}.
\begin{singlespacing}
\begin{lstlisting}[caption={ex-f-subfiles-principal.tex},label=ex:f:subfiles:principal]
\documentclass[12pt,a4paper,oneside]{book}
\usepackage{subfiles} % para incluir partes do documento

\usepackage[T1]{fontenc}
\usepackage[utf8]{inputenc}
\usepackage[brazil]{babel}
\usepackage{amssymb,amsmath}
\usepackage{tikz}
\usetikzlibrary{patterns}
...
\begin{document}
...
\subfiles{ex-f-subfiles-prefacio}
\subfiles{ex-f-subfiles-capitulo1}
\subfiles{ex-f-subfiles-capitulo2}
\subfiles{ex-f-subfiles-capitulo3}
...
\end{document}
\end{lstlisting}
\end{singlespacing}
Exceto pelo uso do pacote \texttt{subfiles} e o comando \verb+\subfiles+\index{subfiles@\verb+\subfiles+} em vez do \verb+\include+, tem nada de especial.
Cada pedaço do documento tem a forma como do Exemplo~\ref{ex:f:subfiles:capitulo1}.
\begin{singlespacing}
\begin{lstlisting}[caption={ex-f-subfiles-capitulo1.tex},label=ex:f:subfiles:capitulo1]
\documentclass[ex-f-subfiles-principal]{subfiles}
\begin{document}
...
\end{document}
\end{lstlisting}
\end{singlespacing}

Note que a classe de documento é \texttt{subfiles}\index{subfiles} e sua opção é o nome do arquivo principal.
Como o \texttt{preamble} será obtido do documento principal, não há \texttt{preambles} e já começa a escrever o documento no ambiente \texttt{document} normalmente.

Com isso, podemos compilar partes do documento, assim como todo o documento.

%standalone e subfiles

% \clearpage
\clearpage\thispagestyle{empty}\cleardoublepage
%\phantomsection
%\addcontentsline{toc}{chapter}{Alguns Comentários Finais}
\chapter{Alguns Aplicativos Auxiliares para Usuário de \LaTeX}
Para melhor aproveitamento do \LaTeX, costumam usar diversos aplicativos auxiliares para melhorar o seu uso. Aqui, listaremos alguns desses aplicativos populares gratuitos.

\section{Editor para \LaTeX\index{editor para \LaTeX}}

\subsection*{TeXMaker/TeXStudio\index{editor para \LaTeX!TeXMaker/TeXStudio}}

site: \url{https://www.xm1math.net/texmaker/}

site: \url{https://www.texstudio.org/}

TeXMaker é um editor de código fonte do \LaTeX{} multiplataforma leve e eficiente. 
Ele apresenta painel de inserção de símbolos matemáticos e visualizador de PDF integrado. Apesar de não vir com menu para chamar \texttt{biber} usado no Bib\LaTeX, pode ajustar a chamada de \BibTeX{} ou acrescentar no menu personalizado (menu ``usuário'').

TeXStudio, por sua vez, é um ``fork'' do TeXMaker, incorpolando mais recursos. Isto torma um pouco mais pesado, mas apresenta a capacidade de compilar um trecho do código no tempo real, mostrando na janela de edição.
Para isso, basta selecionar um trecho, e no menu aberto pelo botão direito, escolher ``Preview selection/parentheses''.


\subsection*{LyX\index{editor para \LaTeX!\LyX}}

site: \url{https://www.lyx.org/}

site: \url{https://sourceforge.net/projects/lyxwininstaller/}

O \LyX{} é um editor com aparência similar a aplicativos Office (MS Office/Libre Office), mas que compila o documento em \LaTeX. Ele também importa/exporta documentos \LaTeX.

O instalador de \LyX{} para MS Windows pode encarregar de instalar o MikTeX básico, caso tenha conexão com a internet, mas também pode deixar o MikTeX pré-instalado. Ele também instala automaticamente os pacotes adicionais.

No menu de ajuda, encontrará o tutorial e guia de usuário.
O \LyX{} foi desenvolvido para usuários de \LaTeX. Como o usuário de \LaTeX{} ganha desempenho significativo em relação ao usuário comum, é altamente recomendado que estude o \LaTeX.
Pra ativar o verificador ortográfico no tempo real, entre em ``tools->preferences'', e em
``[language settings]->[spell checker]'', selecione ``enchant'' ou similar para ``spell check engine'' e cheque o ``[ ]spell check continuouslly''.


\section{Editor gráfico\index{editor gráfico}}
Editores gráficos ajudam a elaboração de ilustrações e similares.

\subsection*{InkScape\index{editor gráfico!InkScape}}

site: \url{http://www.inkscape.org/}

inkscape é um editor gráfico vetorial para criar ilustrações no estilo do Corel Draw (comercial) ou Adobe Illustrator (comercial). 
Ele é bem mais leve e fácil de ser usado do que os seus concorrentes.
Pode ser usado para criar uma ilustração nova, 
ou retocar o arquivo \texttt{pdf} já existente.

A versão atual permite instalar plugin para inserir fórmulas \LaTeX{} diretamente na figura. 
Note que nem todo efeito pode ser salva no \texttt{eps} e também no \texttt{pdf}.
Por isso, é necessário checar o \texttt{eps} ou \texttt{pdf} quando criando.

O manual pode ser acessado pelo menu de ajuda, sem precisar ficar procurando na internet.

Quando salva como \texttt{PDF}, selecione ``usar tamanho do objeto exportado'' na opção ``tamanho da saída de página''.
Também ajuste o ``resolução para renderização (dpi)'' para mínimo de 300 em ''[v] Efeeitos de filtro de renderização''.
Se quer deixar o tamanho do papel já ajustado para tamanho da figura, selecione toda figura (<ctrl>A) e entre em "file->document propries".
No "custom size", clique em [fit page to selection].

Para inserir fórmulas diretamente na figura, instale o \texttt{textext}.
%Veja o site \url{http://pav.iki.fi/software/textext/index.html} para detalhes.
%No caso de MS Windows, o instalador deste plugin abaixará seus dependentes, mas se quer instalar offline (sem a conexão da rede), "necessary library files from pyGTK" mencionado pelo site, deve estar na mesma pasta do instalador de textext.
%Para que \texttt{textext} funcione no MS Windows, é necessário que \texttt{ghost script} e \texttt{pstoedit} estejam funcionando corretamente, além do \texttt{MikTeX} (ou outro TeX com pdflatex).
%A fórmula é inserido ou editada pelo "effects->tex text".
%Não confundir com "effects->render->latex formula" na qual não permite reedição (apenas insere).
%No caso de inserir fórmulas, há diferença em inserir direto na figura ou pelo pacote \texttt{psfrag} na figura \texttt{eps}. 
%A fórmula inserida diretamente na figura terá tamanho diferente do texto, conforme ajusta o tamanho da figura.
%No caso de usar o \texttt{psfrag}, a fórmula é inserida durante a compilação do documento, sempre mantendo tamanho de letras coerente com o texto, independente do ajuste de tamanho da figura. Para usar \texttt{psfrag}, não esqueça de tirar o check do "convert texts to path" na hora de salvar como \texttt{eps}.

Na versão atual, existe a opção de separar o texto para ser processado diretamente no \LaTeX. 
Para isso, basta escolher ``Omitir texto no PDF e criar um arquivo LaTeX'' na opção de ``Orietnação do teto''.
Assim, ele criará um arquivo tex para cada pdf. No documento do \LaTeX, insere o arquivo tex com o comando \verb+\input+. Com esta opção, o comando \LaTeX{} pode ser usado diretamente na figura que será processado quando gera o documento no \LaTeX.

\subsection*{Dia\index{editor gráfico!Dia}}

site: \url{http://projects.gnome.org/dia/}

Editor de diagramas, similar ao comercial Microsoft Visio.
Diagramas costumam usar peças prontas, assim como curvas que se quebram no meio, o que é recomendável usar um editor próprio.
Se for o caso de criar ilustrações, use o inkscape.

\subsection*{GIMP\index{editor gráfico!GIMP}}

site: \url{http://www.gimp.org/}

É o editor de imagem como o Adobe Photoshop (comercial) ou Corel Photo Paint (comercial). O "GIMP Help" é uma espécie de manual que precisa ser abaixado/instalado a parte devido ao seu tamanho.
Ainda não há versão traduzida para português deste manual.

\subsection*{LaTeX Draw\index{editor gráfico!LaTeX Draw}}

site: \url{http://latexdraw.sourceforge.net/}

Editor escrito em java, especializado para pacote \texttt{pstricks} do LaTeX.
Ele permite abrir muitas figuras com \texttt{pstrick}, editado manualmente ou gerado pelo outro programa. 
Como vários programas tais como \texttt{gnuplot}, \texttt{geogebra}, \texttt{inkscape}, etc podem salvar no formato \LaTeX{} com \texttt{pstrick}, poderá usar este editor para retoques.
Note que o ajuste de tamanho da figura deve ser feito no editor e não no documento, para ter espessura da linha inalterada, mas se entender um pouco do \texttt{pstrick}, 
poderá editar o código da figura exportada para deixar escalável dentro do documento (sem alterar a espessura).
Como \texttt{pstrick} é um pacote desenvolvido para \texttt{dvi/ps}, precisará usar o pacote adicional \texttt{pstool} (não é pstools), \texttt{pst-pdf} ou similar, quanto pretende compilar diretamente para \texttt{PDF} através do \texttt{pdflatex}.
Uma maneira clássica de gerar \texttt{PDF} sem tais pacotes é usar o LaTeX=>dvips=>ps2pdf.

\subsection*{FlowFramTk\index{editor gráfico!FlowFramTk}}

site: \url{http://www.dickimaw-books.com/apps/flowframtk/}

O \texttt{FlowFramTk}, anteriormente conhecido como \texttt{jpgf Draw} é escrito em java e exporta figura no formato \LaTeX{} usando o pacote \texttt{pgf} (usado pelo pacote beamer).
O pacote \texttt{pgf} tem uma grande vantagem de poder ser compilado tanto para \texttt{dvi} como para \texttt{pdf} (por isso que o beamer normalmente compilado pelo pdflatex pode ser compilado com latex quando não usa o recurso específico do PDF). \texttt{FlowFramTk} armazena a figura no formato próprio e permite exportar como \texttt{pgf}, mas parece que não tem recurso de importação.
Como não sei ajustar a escala da figura criado pelo pacote \texttt{pgf}, não sei se permite alterar a escala dentro do documento sem comprometer a espessura das linhas.

Um dos recursos adicionais interessantes do \texttt{FlowFramTk} é a capacidade de gerar modelo de poster e material publicitário usando o pacote consolidado \texttt{flowfram}. O pacote \texttt{flowfram} permite criar caixas de texto com rótulo e diagramar dentro dele de forma simples (parece ser um dos mais simples desta categoria).
No entanto, criar layout complexo é um pouco trabalhoso.
O \texttt{FlowFramTk} permite desenhar o layout para usar com \texttt{flowfram}, o que simplifica o trabalho.

\subsection*{TikZit\index{editor gráfico!TikZit}}

site: \url{https://tikzit.github.io/}

Editor gráfico que exposta para formato \texttt{tikz} (parte do pacote \texttt{pgf}).
Ele permite criar facilmente os diagramas em \texttt{tikz} através da interface gráfica.


\section{Gráfico científico\index{gráfico científico}}

\subsection*{Asymptote\index{gráfico científico!Asymptote}}

site: \url{https://sourceforge.net/projects/asymptote/}

Asymptote é uma linguagem/interpretador gráfico 2D/3D inspitado em MetaPost que permite produzir gráficos em Post Script, PDF e SVG.

\subsection*{Geogebra\index{gráfico científico!Geogebra}}

site: \url{http://www.geogebra.org/}

Ele é um aplicativo de geometria dinâmica, mas permite incorporar gráficos de funções. Além de exportar para ser colocado na página web, permite exportar para \texttt{eps/pdf/pstrick/tikz}. 
Como o texto da figura em \texttt{eps} ficará desenhado, não é possível usar o \texttt{psfrag}.
Para retocar texto/fórmulas, exporte no formato \LaTeX{} como \texttt{pstrick} ou \texttt{tikz}. 

\subsection*{gnuplot\index{gráfico científico!gnuplot}}

site: \url{http://www.gnuplot.info/}

Este aplicativo permite gerar gráfico de boa qualidade a partir do dado armazenado no arquivo texto, mas pode gerar gráfico a partir das expressões também. 
Poderá escolher diversos formatos de exportação, dependendo do objetivo.
Como ele é um interpretador, precisará aprender alguns comandos básicos para gerar gráfico de boa qualidade.

\subsection*{Maxima\index{gráfico científico!Maxima}}

site: \url{http://maxima.sourceforge.net/}

Algebra computacional como Maple (comercial) e Mathematica (comercial), implementando a linguagem MacSyma.
Costuma ser usado com a interface gráfica WxMaxima que permite efetuar diversas tarefas sem saber a linguagem, inclusive gerar gráficos 2D e 3D das funções.
%Ele ainda não suporta traçar as curvas de níveis ou superfícies de nível.
A interface gráfica mais recomendada é o WxMaxima que o instalador do Maxima instala por padrão no caso de MS Windows.
%Para o tutorial em português, veja a seção de maxima.

\subsection*{GNU R\index{gráfico científico!GNU R}}

site: \url{http://www.r-project.org/}

GNU R é um aplicativo popular especial para estatística e implementa a linguagem R, bem parecido com a linguagem S.
Ele produz ótimos gráficos, mas requer conhecimento de um pouco da linguagem R.
Uma das interfaces gráficas recomendados é o R Studio, mas para quem quer usar com o menu como no MiniTab (comercial), poderá instalar o Deducer (não testado).

site do R Studio: \url{http://www.rstudio.org/}

site do Deducer: \url{http://www.deducer.org/}

\subsection*{GNU Octave e outros clones do MatLab\index{gráfico científico!GNU Octave}}

gnu octave: \url{http://www.gnu.org/software/octave/}

scilab: \url{http://www.scilab.org/}

freemat: \url{http://freemat.sourceforge.net/}

Para quem entende um pouco da linguagem MatLab, o octave, scilab e freemat podem ser usados para gerar gráficos matemáticos no formato vetorial.
O GNU Octave é mais popular.
O emergente FreeMat é simples de instalar e usar, mas ainda perde dos concorrentes em termos de recursos.

\subsection*{SciDAVis e outras imitações do Microcal Origin\index{gráfico científico!SciDavis}}

SciDAVis: \url{http://scidavis.sourceforge.net/}

RLPlot: \url{http://rlplot.sourceforge.net/}

SciDAVis é programa multiplataforma para análise de daos e criação de gráficos como o famoso Microcal Origin.
O gráfico é de boa qualidade e permite eportar em diversos formatos, incluindo o formato vetorial \texttt{eps}.
O RLPlot é mais simples, mas tem recurso de adicionar texto e linhas poligonais no gráfico gerado.

%Existe também o QtiPlot (gratuito para linux) que é uma espécie de clone do Microcal Origin que permite até importar o arquivo do Microcal Origin 7.5, mas a versão para MS Windows é comercial (ou a versão demo limitada).
%O FitYK (gratuito para linux) é desenvolvimento especialmente para ajuste de curvas, mas versão para MS Windows é comercial.
%

O aplicativo gratuito e multiplataforma FitYK (\url{http://fityk.nieto.pl/}) é desenvolvimento especialmente para ajuste de curvas que é interessante, apesar de não ser clone de Microcal Origin.


\subsection*{LibreCAD\index{gráfico científico!LibreCAD}}

site: \url{http://librecad.org/}

LibreCAD é un CAD 2D de código aberto e é multiplataforma.

\subsection*{Outros}
Nem sempre os aplicativos matemáticos tem a habilidade de exportar gráficos vetoriais, como no caso de MathMod (sucessor do k3dsurf).
Isto ocorre devido a dificuldade ou impossibilidade de criar gráfico vetorial de alguns tipos específicos de imagens matematicos obtidos. 
Neste caso, exporte a figura com resolução de 600DPI quando tem os traços bem definidos (desenho técnico) ou 300DPI no caso de traços estar menos definidos (como imagem no estilo de fotos), o que é recomendado para impressão.
Se for ampliar dentro do documento, a escala da ampliação deve ser considerada, exportando para ter o míniml de DPI após ampliação.

\begin{itemize}
\item MathMod\index{gráfico científico!MathMod} (sucessor do k3DSurf para linux/win): Plotador de superfícies tanto a paramétrica como implícita, usando interface gráfica amigável. 

Site: \url{http://sourceforge.net/projects/mathmod/}

\item Fractint\index{gráfico científico!FractInt} (multi plataforma): Um dos mais poderosos softers para plotar fractais. Por ser baseado em DOS, a interface gráfica não é amigável na versão estável. A versão de teste contém a implementação para Windows.

Site: \url{http://www.fractint.org/}

\item Graphvis\index{gráfico científico!Graphvis} (multiplataforma). Especial para produzir grafos, usando a linguagen dot.

Site: \url{https://www.graphviz.org/}
    
\end{itemize}
    
\section{Alguns convertores\index{convertor}}    
    
\subsection*{Image Magick\index{convertor!Image Magick}}

site: \url{http://www.imagemagick.org/}

Se instalou o \LyX, já deve ter instalado.
Este convertor é para usuário mais avançado, útil para conversão em lotes.
Ele converte praticamente de qualquer formato.
O comando é \texttt{convert}, mas usuário de MS Windows deve tomar cuidado 
pois a ferramenta do MS Windows para converter sistema de arquivo \texttt{FAT} para \texttt{NTFS} também se chama \texttt{convert}.
Assim, deverá assegurar de que está chamando \texttt{convert} do image magick, como especificar o caminho.

Exemplo:

\texttt{convert img.jpg img.pdf}

converterá no formato \texttt{pdf} e

\texttt{convert img.jpg eps2:img.eps}

Criará \texttt{eps} compactado.

% (se não colocar o \texttt{eps2}, gerará \texttt{eps} grande por usar \texttt{PS} de nível 1.

\subsection*{sam2p\index{convertor!sam2p}}

site: \url{https://github.com/pts/sam2p/releases}

Converte imagem bitmap para eps ou pdf.
Atualmente o inkscape pode ser usado para converter para EPS/PDF,
mas pode ser útil para usuários avançados.

sam2p figura.jpg figura.pdf

converterá para pdf.

%Usando a opção correta no comando, poderá gerar arquivos pequenos.
%Note que ele é único convertor conhecido até o momento que permite
%criar eps com compressão usando o postscript de nivel 1.
%Descompacte os arquivos abaixados e copie para local desejado.
%No caso de MS Windows, é recomendável que coloque a pasta de sam2p no caminho 
%(variavel PATH do sistema) para facilitar a execussão. 
%Outra opção é colocar na pasta do MikTeX (c:\textbackslash arquivos de programas\textbackslash miktex 2.7\textbackslash miktex\textbackslash bin) que já deve estar no PATH.
%%A versão com GUI não era usável em 2008, mas pode checar o site dele de vez em quando para verificar se saiu a versão usável.
%
%Exemplo
%
%sam2p -c:pdf figura.jpg figura.pdf
%
%converte para pdf e
%
%sam2p -c:jpeg figura.jpg figura.eps
%
%converte jpeg para eps, mantendo compressão jpeg.

%\subsection*{jpeg2ps}
%
%site: \url{http://gnuwin32.sourceforge.net/packages/jpeg2ps.htm}
%
%Convertor clássico de jpeg para eps de nível 2 ou 3 (com compressão).
%Eu aconselho usar o Imagemagick ou sam2p.

\subsection*{TeX4ht\index{convertor!TeX4ht}}
\texttt{TeX4ht} é um convertor de \TeX/\LaTeX para HTML/XHTML, similar a Hevea, TTH, e LaTeX2HTML,  que roda em diversas plataformas.
Note que todos 4 programas são de multiplataformas e cada um tem vantagens e desvantagens. 
Hevea e TTH, não apresenta suporte decente as fórmulas matmáticas. 
O Tex4ht implementado como pacote para TeX e o LaTeX2HTML que é escrito em perl, conseguem converter eficientemente os documentos com equações.
Ambas podem optar em usar a imagem (GIF, PNG ou JPEG) ou MathML para representar equações.
Além disso, TeX4ht e LaTeX2HTML são de código aberto. 
Por estas e outras razões, eles são os mais usados.
O HTML gerado por Tex4ht é mais parecido com o documento original e não consegue dividir documentos em pedaços (o que é interessante para HTML grande, para acelerar o acesso na internet).
O LaTeX2HTML produz documentos de acordo com a filosofia do HTML e consequentemente, não produz documento ``fiel'' ao original. 
Ele particiona documentos grandes em pedaços menores e cria recurso de navegação, mas isto pode ser desativado, se desejar.
O HTML gerado por LaTeX2HTML acomoda bem em maioria dos navegadores, o que torna a escolha preferida dos profissionais.
No entanto, Tex4ht já vem com muitas distribuições \LaTeX, incluindo o MikTeX para MS Windows.

Além disso, ele permite converter para formato do Libre Office e TeXMaker e LyX usam ele para gerr arquivo de Libre Office.


%No caso de Windows, o comando htlatex.bat converte usando imagem na equação e htmlatex.bat usa MathML para fórmula. O oolatex.bat converte para formato de Libre Office.

\textbf{Observação:} Acentuação dentro do ambiente tabular (\a{'}, \a{~}, etc) não será reconhecido. Neste caso, use a acentuação direta.

%\begin{enumerate}[{Observação}~1]
%\item É possível usar a imagem no formato JPEG ou PNG em vez de GIF. Por exemplo, o comando
%
%\verb+htlatex foo "html,png"+
%
%gera o codigo HTML a partir do arquivo foo.tex, com imagem PNG. 
%O segundo parâmetro que é colocado entre aspas, é passado como opção do pacote \texttt{tex4ht.sty} que é lido automaticamente. 
%Outros arquivos de lotes trabalham de forma análoga. 
%Para saber sobre opções no tex4ht.sty, veja o manual .
%
%\item Durante a conversão, mesmo especificando o modo de MathML, algumas fórumulas podem ser convertidos na imagem, pois nem toda fórmula do LaTeX apresenta correspondênte no MathML.

%\item A interface gráfica encontra-se em \url{http://www.mayer.dial.pipex.com/tex.htm}. 
%Ele pedia o  htrun.exe para todar o TeX4ht do MikTex, que não existia, mas não sei se isso já foi resolvido.

%\item Acentuação dentro do ambiente tabular (\a{'}, \a{~}, etc) não será reconhecido. Neste caso, use a acentuação direta.
%\end{enumerate}

%\subsection*{RTF2LaTeX2e}
%
%Site: h\url{ttp://sourceforge.net/projects/rtf2latex2e/}
%
%%(cópia no CTAN: https://www.ctan.org/pkg/rtf2latex2e).
%RTF2LaTeX2e é um aplicativo gratuito multiplataforma que converte RTF (Rich Text Format) exportável por MS Word para LaTeX2e e suporta equações do MS Equations. 
%Este é um dos melhores convertores livres para documento de MS Word para LaTeX quando tem equações. Para exportar do Libre Office, basta instalar o plugin writer2latex caso ainda não tenha.
%Abaixe o arquivo e descompacte numa pasta apropriada tal como
%
%\verb+c:\arquivos de programas\rtf2latex2e+
%
%Para converter documento do MS Word, salve no formato RTF usando "arguivo->salvar como". 
%Suponhe que o arquivo foi salvo como "test.rtf" na pasta ``\verb+c:\temp+''
%Agora, abra o prompt de comando (DOS prompt) e mude para pasta onde se encontra o arquivo RTF.
%
%\verb+c:\desktop>cd\temp+
%
%Execute o comando.
%
%\verb+C:\temp>"arquivos de programas\rtf2latex2e\rtf2latex2e.exe" test.rtf+
%
%e o arquivo test.tex será criado.
%Note que maioria das imagens incluidas serão manuseadas corretamente, mas algumas figuras especiais tal commo do Microsoft Draw pode ser ignorados.
%A configuração padrão é ignorar as cores, mas isto pode ser modificado, editando o arquivo ''r2l-pref'' em ``c:\textbackslash arquivos de programas\textbackslash rtf2latex\textbackslash pref''. O ``ignoreColor=true'' e o ``ignorecolor=false'' desativa/ativa o uso de cor de texto durante a conversão.
%
%\subsection*{LaTeX2RTF}
%
%site: \url{http://sourceforge.net/projects/latex2rtf/}
%
%%(cópia do CTAN: https://www.ctan.org/pkg/latex2rtf)
%
%LaTeX2RTF é um aplicatibvo gratuíto multiplataforma que converte LaTeX para RTF (Rich Text Format) importável pelo MS Word e suporta equações usando ``EQ field'' (não o Equation Object) que é reconhecido pelo MS Equations. 
%Para Libre Office, use o ooolatex incluso em TeX4ht que faz parte do MikTeX ou W32TeX. 
%Lembre-se que, apesar de muitos elementos tais como equações, tabelas, figuras etc serem convertidos para RTF, nem tudo do LaTeX apresenta equivalência no MS Word. 
%Para instalar, abaixe o arquivo e descompacte para a pasta desejada. 
%Vamos supor que ele foi colocado no
%
%\verb+c:\arquivos de programas\l2r+
%
%Note que o nome do executável é latex2tr.exe, mas e conveniente executar pelo arquivo de lote.
%Edite o l2r.bat e mude a linha
%
%\verb+C:\l2r\latex2rt -P C:\l2r\cfg %1 %2 %3 %4 %5 %6 %7 %8 %9+
%
%para
%
%\verb+"C:\arquivos de programas\l2r\latex2rt.exe"  -P "C:\arquivos de programas\l2r\cfg" %1 %2 %3 %4 %5 %6 %7 %8 %9+
%
%Agora edite o l2rsem.bat e mude a linha
%
%\verb+C:\l2r\latex2rt -S -P C:\l2r\cfg %1 %2 %3 %4 %5 %6 %7 %8 %9+
%
%para
%
%\verb+"C:\arquivos de programas\l2r\latex2rt.exe"  -S -P "C:\arquivos de programas\l2r\cfg" %1 %2 %3 %4 %5 %6 %7 %8 %9+
%
%O arquivo l2r.bat servirá para converter no RTF estilo americano (campo de dados separado pela vírgula) e l2rsem.bat (que tem -S a mais no parámetro) servirá para converter no RTF estilo alemão (campo separado pelo ponto e virgula). No caso de portugues, deverá usar o l2rsem.bat para que não ocorra problemas com equações.
%Para que conversão de imagens funcione bem, ou que equações sejam incluidas como BMP, deverá instalar o GhostView/GhostScript e ImageMagick e ajustar seus o caminhos em l2rprep.bat.
%
%Para usar, basta compilar o arquivo TeX usando LaTeX, BibTeX, etc normalmente para que arquivos auxiliares sejam criados e referencias cruzadas resolvidos. Após isto, execute o arquivo de lote desejado (seguido de nome do arquivo tex sem extensão) para efetuar conversão.o
%Abra o prompt de comando de se caso precisar converter figuras ou gerar eauações em BMP, execute o l2rprep.bat primeiro.
%
%\subsection*{WP2LaTeX}
%
%Site: \url{http://sourceforge.net/projects/wp2latex/}
%
%Word Perfect para LaTeX que suporta tabelas e equações.

\subsection*{Writer2LaTeX\index{convertor!Writer2LaTeX}}

site: \url{http://writer2latex.sourceforge.net/}

Editor de texto do Open Office/Libre Office para LaTeX/XHTML+MathML.
Ele pode ser usado como plugin ou pelo comando de linha.
Dependendo da instalação do Libre Office, já estará instalado.

\subsection*{Excel2LaTeX\index{convertor!Excel2LaTeX}}

site: \url{https://www.ctan.org/pkg/excel2latex}

Excel2LaTeX é um macro para MS Excel, para converter planilha do Excel para tabela do LaTeX. Maioria das formatações tais como espacificação das fontes (negrito, itálico, etc), bordas, celula mescrada, etc são mantidas. Para instalar, copie o arquivo excel2latex.xla para pasta do MS Office e dê um double click sobre ele. O MS Excel abre o arquivo e perguntará se vai executar o "macro". Responda "sim" e a instalação está completa.
No MS Excel, selecione a região desejada da tabela e crique em ``convert table to LaTeX'' (ou ferramentas->convert table to LaTeX). 
O resultado pode ser copiado para ``clipboard'' ou salvo no arquivo. 
%Para versão em japonês, veja o http://plaza19.mbn.or.jp/~Butcher_Bird/Mac/Excel.html.

\subsection*{Calc2LaTeX\index{convertor!Calc2LaTeX}}

site: \url{http://calc2latex.sourceforge.net/}

Macro para Open Calc (do Open Office) para gerar tabelas de LaTeX a partir da planilha.
Não há atualização recente.

\section{Outras ferramentas\index{ferramentas}}

\subsection*{jabref\index{ferramentas!jabref}}

site: \url{http://jabref.sourceforge.net/}

Se usar o \BibTeX, é recomendável que use este editor para editar arquivo ".bib".
Muitos usuários de \LaTeX{} deixam de usar o \BibTeX{} devido a chatice de editar o arquivo do \BibTeX{} que usa a sintaxe diferente do \LaTeX, mas com o \texttt{jabref}, as coisas mudam.

Ele também suporta o Bib\LaTeX (para isso, deverá usar o modo Bib\LaTeX).
Também é aconselhável usar a codificação \texttt{utf-8} quando usa o modo Bib\LaTeX{} para evitar problemas com caracteres acentuados.


\subsection*{jpdftweak\index{ferramentas!jpdftweak}}

site: \url{http://jpdftweak.sourceforge.net/}

Esta ferramenta é essencial para ajustes finais do documento PDF. Permite dividir ou juntar PDF, trocar ordem das páginas, alterar o tamanho de pápel, ajustar o tamanho  e posição do corpo do texto, colocar várias páginas em uma, etc. O domínio do jpedtweak é importante para quem precisa manipular PDF pronta, como preparar para edição.
Para elaborar poster, poderá diagramar no papel de tamanho 1/4 (1/2 de escala em cada dimensão) com letra 12pt e ao finalizar, poderá usar o jpefTweak para ampliar no tamanho normal. Com o fator de ampliação 2x, a letra ficará com 24pt, apropriado para posters. Com isso, poderá criar posters sem precisar de pacote ou truque especial.
Para quem quer automatizar com script, o Multvalent pode ser interessante. Note que Multivalent é um visualizador para HTML, DVI, PDF, etc e o processamento de PDF será feito pelo comando de linha em vez de GUI, o que fez perder o espaço para jPDFTweak que apareceu mais tarde.
Para recorte rápido do arquivo PDF, o PDFSAM (http://www.pdfsam.org/) é mais prático que o jpdfTWeak.

\subsection*{pdfarranger (linux e windows)\index{ferramentas!pdfarranger}}

site: \url{https://github.com/jeromerobert/pdfarranger}

É um fork do pdfshuffler (\url{ttp://sourceforge.net/projects/pdfshuffler/}) que era somente para linux. Usado para manuseio dos arquivos em PDF. Usando interface amigável, poderá concatenar arquivos, eliminar páginas, reordenar páginas, rotacionar, etc. Não é potente como o jPDFTWeak, mas para manuseio simples, ele é prático. Usuários de Mac, podrá usar o PDFSAM (\url{http://www.pdfsam.org/}).

\subsection*{xournal++ e jaurnal\index{ferramentas!xournal++/jaurnal}}

xournal++: \url{https://github.com/xournalpp/xournalpp}

jaurnal: \url{http://www.dklevine.com/general/software/tc1000/jarnal.htm}

xournal++ é uma reimplementação do xournal (\url{http://xournal.sourceforge.net/}) que é um caderno eletrônico. mas pode abrir um arquivo PDF, escrever sobre ele e exportar como um novo aruivo PDF. Este recurso é interessante para corrigir trabalhos. Se tiver uma mesa digitalizadora, poderá efetuar correção tão rápida como corrigir sobre texto impresso. O jaurnal é similar ao xournal, mas é implementado em java. A xournal++ é mais recomendado.

\subsection*{Scribus\index{ferramentas!scribus}}

site: \url{http://www.scribus.net/}

O Scribus é um aplicativo para elaborar revistas, jornais, posters, etc como o Microsoft Publisher (comercial). 
O Scribus permite inserir ``caixa de LaTeX'', o que facilita o desenvolvimento de material científico. Apesar do usuário de LaTeX costumam querer diagramar o poster de apresentação no LaTeX, se precisar produzir poster, folhetos, etc com frequência, é aconselhável aprender a usar o Scribus.

\subsection*{Alguns serviços web\index{ferramentas!serviços Web}}

\url{http://detexify.kirelabs.org/classify.html} reconhecimento de símbolos LaTeX a partir do traço manual

\url{https://webdemo.myscript.com/views/math/index.html} Reconhecimento de fórmula escrito a mão e cria código LaTeX e MathML.

\url{http://www.imagemagick.org/MagickStudio/scripts/MagickStudio.cgi} Serviço de conversão do formato de imagens.

\url{https://pt.overleaf.com/}
Serviço de web com ambiente LaTeX. Para uso pessoal é gratuito.

Note que o central de repositório do (La)\TeX{} é o CTAN. Portanto, se estiver procurando algo, pode dar olhada primeiro no CTAN (\url{https://www.ctan.org/}). 

% \clearpage
\clearpage\thispagestyle{empty}\cleardoublepage
\phantomsection
\addcontentsline{toc}{chapter}{Alguns Comentários Finais}
\chapter*{Alguns Comentários Finais}

Neste texto está direcionado ao usuário de \LaTeX, mas não para os desenvolvedores.
Assim, não foi incluído detalhes sobre caixas e similares.
Para quem pretende criar comandos e ambientes complexos ou desenvolver pacotes, o conhecimento sobre caixas é importante. 
Para este assunto, recomendo a leitura de outro material tais como \cite{book:oeticker:2018} e \cite{book:wikibook:latex}.

Neste texto, foram apresentados pacotes e classes mais utilizados, mas vários pacotes e classes populares foram omitidos.

Por exemplo, quem trabalha com fórmulas complexas, é recomendável que considere o uso do pacote \texttt{mathtools}\index{mathtools} que resolve algumas deficiências do pacote \texttt{amsmath}.
Para criar ficha catalográfica, é interessante colocar o número total de páginas usando o pacote \texttt{lastpage}\index{lastpage}.
Para inserir uma ou mais páginas diretamente no documento em vez de inserir como figuras, usa-se o pacote \texttt{pdfpages}.
Quem trabalha com tabelas, pode precisar do pacote \texttt{multirow}\index{multirow} para mesclar linhas, além de vários pacotes específicos para tabelas omitidos neste documento.
%Para elaborar livros: koma script, memoir, tufte-latex, bookest, willowtreebook, octavo, ElegantBook (templates), ycbook, nostarch, simplivre
Ainda existem pacotes específicos para xadrez, diagrama química, música, etc, além dos pacotes destinados para cada tipo de ajustes.

%A lista dos pacotes interessantes são longas.

Dentre as classes omitidas aqui, as classes da família \texttt{koma script}\index{koma script} e a classe \texttt{memoir}\index{memoir} são um dos mais importantes.
Estas classes implementam várias funcionalidades adicionais em relação as classes básicas.

Como existem pacotes específicos para cada assunto, é impossível conhecer todos eles. 
Mas, pelo menos podemos tentar conhecer os pacotes e classes populares existentes para área de atuação de cada um de nós.

%A quantidade de pacotes no \LaTeX{} é muito extensa.
%Assim, se precisar efetuar alguns ajustes refinados ou precisar de comandos ou ambientes complexos, vale a pena pesquisar se tem pacotes específicos para o que está pretendendo.

%Além dos pacotes, também existem várias classes de documentos adicionais.

% \clearpage
\clearpage\thispagestyle{empty}\cleardoublepage
\backmatter
% Referências Bibliográficas
\phantomsection
\bibliography{latex-via-exemplos}
% \bibliographystyle{amsalpha}
\bibliographystyle{alphaurl}

\clearpage\thispagestyle{empty}\cleardoublepage
\phantomsection
\begin{singlespacing}
\printindex
\end{singlespacing}

%\chapter{Todo}
%Já feito
%\begin{itemize}
%\item Estruturação de texto antes da estruturação de documentos. Nele, incluir o minipage, já usando o \verb+\linewidth+ como o \verb+0.5\linewidth+. OK.
%
%\item A seleção de fontes (família, peso, forma e tamanho) ficar depois de ver estrutura de documentos, estruturação de textos e fórmulas matemáticas usando AMS. Talvez, até mesmo depois das figuras. OK (ver se fica antes ou depois das figuras). OK
%Neste caso, algumas seleções de fontes matemáticos (\verb+\mathrm+, \verb+\mathbb+ e \verb+\mathcal+) e o \verb+\emph+ para textos (talvés \verb+\textbf+ também) devem ser tratados antes de entrar no capítulo de configuração de fontes. OK.
%
%\item Medidas e espaçamentos também ficar para depois, como do item anterior. Algumas medidas como o \verb+\linewidth+ poderá ser apresentado junto ao \texttt{minipage} na  estruturação de textos. OK.
%
%
%\item short name for caption. Deve ser junto aos elementos flutuantes. OK
%\item indentfirst? OK
%\item multicols? OK

%\item comandos frágeis.
%\item Onde tratar de babel
%\item Onde tratar de contadores?
%\item hifenização?
%\item onde tratar de \verb+\makeatletter+/\verb+\makeatother+? 
%
%\item versão ``*''? Comentar que tem o pacote ifthen tambem?
%
%\item pacote \texttt{color}? Onde colocar?
%\item alguns pacotes de fontes do base e required?
%\item integrais múltiplas OK
%\end{itemize}

%Todo:
%
%\begin{itemize}
%\item Onde tratar de poster? Deixar junto com fancybox e lpsum.
%
%\end{itemize}
%
%Da parte adicional. Decidir qual serão vistos e como agrupar.

%\begin{itemize}
%\item lmodern e mais alguns pacotes de fontes tais como  newtxtext/newtxmath, etc.

% \item dsfont,
%\item hyperref, geometry, fancyhdr, fancybox, setspace
%\item titlesec, float, wrapfig (ou floatflt), xcolor, subfigure
%\item caption e hyphenat
%\item tocbibind
%\item beamer, pgf/tikz,

%\item natbib, biblatex

%\item xelatex/lualatex, iffont.
% \item O editor LyX
% \item eso-pic, framed


% \item Exemplo do pacote caption e listings e showexpl
%\item arquivos de estilo.

%\item \verb+\section{\texorpdfstring{$E=mc^2$}{E = mc ** 2}}+ para bookmark do PDF (hyperref)? Uso do comando \verb+\url{}+ no arquivo de BibTeX.

%\item diferenca de amssymb/amsmath com unicode-math:
%mathbb, mathcal, \verb+\int\limits+
%item \verb+\newfontfamily+ do fontspec

% \end{itemize}


% Referencias: https://pt.wikipedia.org/wiki/Ajuda:Guia_de_edi%C3%A7%C3%A3o/F%C3%B3rmulas_TeX
% Tem breve descricao de como editar formulas em LaTeX de forma adequada (elemento matematico por elemento matematico, apesar de pequenos inadequacoes: underbrace/overbrace, overset em vez do substack no indice da somatoria, omissao do ambiente cases, mbox em vez de mathrm)
%
% https://en.wikibooks.org/wiki/LaTeX/
% Livro eletronico em ingles.
% \printindex

\end{document}
