\documentclass[10pt,landscape,a4paper]{article}
\usepackage{multicol}
\usepackage{calc}
\usepackage{ifthen}
\usepackage[landscape]{geometry}
\usepackage[utf8]{inputenc}
\usepackage[T1]{fontenc}
\usepackage[brazil]{babel}
\usepackage{url}

% To make this come out properly in landscape mode, do one of the following
% 1.
%  pdflatex latexsheet.tex
%
% 2.
%  latex latexsheet.tex
%  dvips -P pdf  -t landscape latexsheet.dvi
%  ps2pdf latexsheet.ps


% If you're reading this, be prepared for confusion.  Making this was
% a learning experience for me, and it shows.  Much of the placement
% was hacked in; if you make it better, let me know...


% 2008-04
% Changed page margin code to use the geometry package. Also added code for
% conditional page margins, depending on paper size. Thanks to Uwe Ziegenhagen
% for the suggestions.

% 2006-08
% Made changes based on suggestions from Gene Cooperman. <gene at ccs.neu.edu>


% To Do:
% \listoffigures \listoftables
% \setcounter{secnumdepth}{0}


% This sets page margins to .5 inch if using letter paper, and to 1cm
% if using A4 paper. (This probably isn't strictly necessary.)
% If using another size paper, use default 1cm margins.
\ifthenelse{\lengthtest { \paperwidth = 11in}}
	{ \geometry{top=.5in,left=.5in,right=.5in,bottom=.5in} }
	{\ifthenelse{ \lengthtest{ \paperwidth = 297mm}}
		{\geometry{top=1cm,left=1cm,right=1cm,bottom=1cm} }
		{\geometry{top=1cm,left=1cm,right=1cm,bottom=1cm} }
	}

% Turn off header and footer
\pagestyle{empty}
 

% Redefine section commands to use less space
\makeatletter
\renewcommand{\section}{\@startsection{section}{1}{0mm}%
                                {-1ex plus -.5ex minus -.2ex}%
                                {0.5ex plus .2ex}%x
                                {\normalfont\large\bfseries}}
\renewcommand{\subsection}{\@startsection{subsection}{2}{0mm}%
                                {-1explus -.5ex minus -.2ex}%
                                {0.5ex plus .2ex}%
                                {\normalfont\normalsize\bfseries}}
\renewcommand{\subsubsection}{\@startsection{subsubsection}{3}{0mm}%
                                {-1ex plus -.5ex minus -.2ex}%
                                {1ex plus .2ex}%
                                {\normalfont\small\bfseries}}
\makeatother

% Define BibTeX command
\def\BibTeX{{\rm B\kern-.05em{\sc i\kern-.025em b}\kern-.08em
    T\kern-.1667em\lower.7ex\hbox{E}\kern-.125emX}}

% Don't print section numbers
\setcounter{secnumdepth}{0}


\setlength{\parindent}{0pt}
\setlength{\parskip}{0pt plus 0.5ex}


% -----------------------------------------------------------------------

\begin{document}

\raggedright
\footnotesize
\begin{multicols}{3}


% multicol parameters
% These lengths are set only within the two main columns
%\setlength{\columnseprule}{0.25pt}
\setlength{\premulticols}{1pt}
\setlength{\postmulticols}{1pt}
\setlength{\multicolsep}{1pt}
\setlength{\columnsep}{2pt}

\begin{center}
 %    \Large{\textbf{\LaTeXe\ Cheat Sheet}} \\
    \Large{\textbf{Guia Rápido \LaTeXe\ }} \\
\end{center}

\section{Classes de documentos}
\begin{tabular}{@{}ll@{}}
\verb!book!    & Padrão é frente-e-verso. \\
\verb!report!  & Sem divisões \verb!\part! . \\
\verb!article! & Sem divisões \verb!\part! ou \verb!\chapter! . \\
\verb!letter!  & Carta (?). \\
\verb!slides!  & Fonte sem-serifa grande.
\end{tabular}

Usado no começo de um documento:
\verb!\documentclass{!\textit{classe}\verb!}!.  
Use \verb!\begin{document}! para iniciar os conteúdos e \verb!\end{document}! para 
finalizar o documento.


\subsection{Opções comuns \texttt{documentclass}}
\newlength{\MyLen}
\settowidth{\MyLen}{\texttt{letterpaper}/\texttt{a4paper} \ }
\begin{tabular}{@{}p{\the\MyLen}%
                @{}p{\linewidth-\the\MyLen}@{}}
\texttt{10pt}/\texttt{11pt}/\texttt{12pt} & Tamanho da fonte. \\
\texttt{letterpaper}/\texttt{a4paper} & Tamanho do papel. \\
\texttt{twocolumn} & Usa duas colunas. \\
\texttt{twoside}   & Ajusta margens para frente-e-verso (dois-lados). \\
\texttt{landscape} & Orientação de paisagem.  Deve usar
                     \texttt{dvips -t landscape}. \\
\texttt{draft}     & Linhas de espaço-duplo (?).
\end{tabular}

Uso:
\verb!\documentclass[!\textit{opt,opt}\verb!]{!\textit{class}\verb!}!.


\subsection{Pacotes}
\settowidth{\MyLen}{\texttt{multicol} }
\begin{tabular}{@{}p{\the\MyLen}%
                @{}p{\linewidth-\the\MyLen}@{}}
%\begin{tabular}{@{}ll@{}}
\texttt{fullpage}  &  Usa margens de 1 polegada. \\
\texttt{anysize}   &  Ajusta margens: \verb!\marginsize{!\textit{l}%
                        \verb!}{!\textit{r}\verb!}{!\textit{t}%
                        \verb!}{!\textit{b}\verb!}!.            \\
\texttt{geometry}  &  Ajusta margens: \verb!\geometry{!\textit{verbose%
                        ,a4paper,tmargin =4cm,bmargin=2.5cm,lmargin=2cm%
                        ,rmargin=2cm, headheight=2cm,headsep=1cm,%
                         ~footskip=1cm}\verb!}!.\\
\texttt{multicol}  &  Usa $n$ colunas: 
                        \verb!\begin{multicols}{!$n$\verb!}!.   \\
\texttt{latexsym}  &  Usa fonte \LaTeX\ symbol . \\
\texttt{graphicx}  &  Mostra imagens:
                        \verb!\includegraphics[width=!%
                        \textit{x}\verb!]{!%
                        \textit{file}\verb!}!. \\
\texttt{url}       & Insere URL: \verb!\url{!%
                        \textit{http://\ldots}%
                        \verb!}!.
\end{tabular}

Use antes de \verb!\begin{document}!. 
Uso: \verb!\usepackage{!\textit{pacote}\verb!}!


\subsection{Título}
\settowidth{\MyLen}{\texttt{.autor.texto.} }
\begin{tabular}{@{}p{\the\MyLen}%
                @{}p{\linewidth-\the\MyLen}@{}}
\verb!\author{!\textit{texto}\verb!}! & Autor do documento. \\
\verb!\title{!\textit{texto}\verb!}!  & Título do documento. \\
\verb!\date{!\textit{texto}\verb!}!   & Data. \\
\end{tabular}

Estes comandos vão antes de  \verb!\begin{document}!.  A declaração 
\verb!\maketitle! vai no topo do documento, em geral.

\subsection{Miscelânea}
\settowidth{\MyLen}{\texttt{.pagestyle.empty.} }
\begin{tabular}{@{}p{\the\MyLen}%
                @{}p{\linewidth-\the\MyLen}@{}}
\verb!\pagestyle{empty}!     &   Cabeçalho, rodapé vazios %
                                 e sem número de página. \\

\end{tabular}



\section{Estrutura do documento}
\begin{multicols}{2}
\verb!\part{!\textit{titulo}\verb!}!  \\
\verb!\chapter{!\textit{titulo}\verb!}!  \\
\verb!\section{!\textit{titulo}\verb!}!  \\
\verb!\subsection{!\textit{titulo}\verb!}!  \\
\verb!\subsubsection{!\textit{titulo}\verb!}!  \\
\verb!\paragraph{!\textit{titulo}\verb!}!  \\
\verb!\subparagraph{!\textit{titulo}\verb!}!
\end{multicols}
{\raggedright
Comandos de seção podem ser seguidos com um  \texttt{*}, como
\verb!\section*{!\textit{title}\verb!}!, para suprimir números.
 \verb!\setcounter{secnumdepth}{!$x$\verb!}! suprime números 
numbers de profundidade $>x$, onde \verb!chapter! tem profundidade 0.
}

\subsection{Ambientes de texto}
\settowidth{\MyLen}{\texttt{.begin.quotation.}}
\begin{tabular}{@{}p{\the\MyLen}%
                @{}p{\linewidth-\the\MyLen}@{}}
\verb!\begin{comment}!    &  Bloco de comentário (não imprimível). \\
\verb!\begin{quote}!      &  Bloco quotado indentado. \\
\verb!\begin{quotation}!  &  Como \texttt{quote} com parágrafos indentados. \\
\verb!\begin{verse}!      &  Bloco de quotação para verso.
\end{tabular}

\subsection{Listas}
\settowidth{\MyLen}{\texttt{.begin.description.}}
\begin{tabular}{@{}p{\the\MyLen}%
                @{}p{\linewidth-\the\MyLen}@{}}
\verb!\begin{enumerate}!        &  Lista numbereda. \\
\verb!\begin{itemize}!          &  Lista pontuada. \\
\verb!\begin{description}!      &  Lista de Descrição. \\
\verb!\item! \textit{texto}      &  Adiciona um item. \\
\verb!\item[!\textit{x}\verb!]! \textit{texto}
                                &  Use \textit{x} ao invés de pontos ou números normais.
                             Necessário para descrições. \\
\end{tabular}




\subsection{Referências}
\settowidth{\MyLen}{\texttt{.pageref.marcador..}}
\begin{tabular}{@{}p{\the\MyLen}%
                @{}p{\linewidth-\the\MyLen}@{}}
\verb!\label{!\textit{marcador}\verb!}!   &  Cria um marcador para referência cruzada, 
                          na  forma \verb!\label{sec:item}!. \\
\verb!\ref{!\textit{marcador}\verb!}!   &  Dá o número de seção/corpo do marcador. \\
\verb!\pageref{!\textit{marcador}\verb!}! &  Dá o númeor de página do marcador. \\
\verb!\footnote{!\textit{texto}\verb!}!  &  imprime nota-de-rodapé no fundo da página. \\
\end{tabular}


\subsection{Materiais flutuantes}
\settowidth{\MyLen}{\texttt{.begin.equation..place.}}
\begin{tabular}{@{}p{\the\MyLen}%
                @{}p{\linewidth-\the\MyLen}@{}}
\verb!\begin{table}[!\textit{lugar}\verb!]!     &  Adiciona uma tabela numerada. \\
\verb!\begin{figure}[!\textit{lugar}\verb!]!    &  Adiciona uma figura numerada. \\
\verb!\begin{equation}[!\textit{lugar}\verb!]!  &  Adiciona uma equação numerada. \\
\verb!\caption{!\textit{texto}\verb!}!           &  Título para o material flutuante. \\
\end{tabular}

O \textit{lugar} é uma lista de  colocação válida do material.  \texttt{t}=topo,
\texttt{h}=aqui, \texttt{b}=fundo, \texttt{p}=página separada, \texttt{!}=aqui definitivamente. Títulos e marcadores devem estar dentro do ambiente.

%---------------------------------------------------------------------------

\section{Propriedades do texto}

\subsection{Cara da fonte}
\newcommand{\FontCmd}[3]{\PBS\verb!\#1{!\textit{text}\verb!}!  \> %
                         \verb!{\#2 !\textit{text}\verb!}! \> %
                         \#1{#3}}
\begin{tabular}{@{}l@{}l@{}l@{}}
\textit{Comando} & \textit{Declaração} & \textit{Efeito} \\
\verb!\textrm{!\textit{texto}\verb!}!                    & %
        \verb!{\rmfamily !\textit{texto}\verb!}!               & %
        \textrm{Família Romana} \\
\verb!\textsf{!\textit{texto}\verb!}!                    & %
        \verb!{\sffamily !\textit{texto}\verb!}!               & %
        \textsf{Família Sem serifa} \\
\verb!\texttt{!\textit{texto}\verb!}!                    & %
        \verb!{\ttfamily !\textit{texto}\verb!}!               & %
        \texttt{Fam.~Monoespaçada} \\
\verb!\textmd{!\textit{texto}\verb!}!                    & %
        \verb!{\mdseries !\textit{texto}\verb!}!               & %
        \textmd{Série Média} \\
\verb!\textbf{!\textit{texto}\verb!}!                    & %
        \verb!{\bfseries !\textit{texto}\verb!}!               & %
        \textbf{Série Negrito} \\
\verb!\textup{!\textit{texto}\verb!}!                    & %
        \verb!{\upshape !\textit{texto}\verb!}!               & %
        \textup{Forma reta} \\
\verb!\textit{!\textit{texto}\verb!}!                    & %
        \verb!{\itshape !\textit{texto}\verb!}!               & %
        \textit{Forma Itálica} \\
\verb!\textsl{!\textit{texto}\verb!}!                    & %
        \verb!{\slshape !\textit{texto}\verb!}!               & %
        \textsl{Forma Inclinada} \\
\verb!\textsc{!\textit{texto}\verb!}!                    & %
        \verb!{\scshape !\textit{texto}\verb!}!               & %
        \textsc{Forma Small Caps} \\
\verb!\emph{!\textit{texto}\verb!}!                      & %
        \verb!{\em !\textit{text}\verb!}!               & %
        \emph{Emfatizado} \\
\verb!\textnormal{!\textit{texto}\verb!}!                & %
        \verb!{\normalfont !\textit{texto}\verb!}!       & %
        \textnormal{Fonte padrão} \\
\verb!\underline{!\textit{texto}\verb!}!                 & %
                                                        & %
        \underline{Sublinhado}
\end{tabular}

O comando na forma (\verb.\textit{tt}.) lida melhor com 
              espaços do que a forma  (\verb.\itshape{tt}.) .

\subsection{Tamanho de fonte}
\setlength{\columnsep}{14pt} % Need to move columns apart a little
\begin{multicols}{2}
\begin{tabbing}
\verb!\footnotesize!          \= \kill
\verb!\tiny!                  \>  \tiny{menor} \\
\verb!\scriptsize!            \>  \scriptsize{muito reduzido} \\
\verb!\footnotesize!          \>  \footnotesize{mais reduzido} \\
\verb!\small!                 \>  \small{pequeno} \\
\verb!\normalsize!            \>  \normalsize{normal} \\
\verb!\large!                 \>  \large{maior} \\
\verb!\Large!                 \=  \Large{grande} \\  % Tab hack for new column
\verb!\LARGE!                 \>  \LARGE{grandão} \\
\verb!\huge!                  \>  \huge{enorme} \\
\verb!\Huge!                  \>  \Huge{monstro}
\end{tabbing}
\end{multicols}
\setlength{\columnsep}{1pt} % Set column separation back

Estas são as declarações e deveriam ser usadas na forma
\verb!{\small! \ldots\verb!}!, ou sem chaves para afetar o documento inteiro.% a partir do ponto de sua inclusão.


\subsection{Texto Verbatim}

\settowidth{\MyLen}{\texttt{.begin.verbatim..} }
\begin{tabular}{@{}p{\the\MyLen}%
                @{}p{\linewidth-\the\MyLen}@{}}
\verb@\begin{verbatim}@ & ambiente Verbatim. \\
\verb@\begin{verbatim*}@ & Espaces são mostrados com  \verb*@ @. \\
\verb@\verb!text!@ & Texto entre os caracteres delimitadores (neste caso %
                      `\texttt{!}') é verbatim.
\end{tabular}


\subsection{Justificação}
\begin{tabular}{@{}ll@{}}
\textit{Ambiente}  &  \textit{Declaração}  \\
\verb!\begin{center}!      & \verb!\centering!  \\
\verb!\begin{flushleft}!  & \verb!\raggedright! \\
\verb!\begin{flushright}! & \verb!\raggedleft!  \\
\end{tabular}

\subsection{Miscelânea}
\verb!\linespread{!$x$\verb!}! muda o espaço entre linhas pelo multiplo $x$.





\section{Símbolos no modo texto}

\subsection{Símbolos}
\begin{tabular}{@{}l@{\hspace{1em}}l@{\hspace{2em}}l@{\hspace{1em}}l@{\hspace{2em}}l@{\hspace{1em}}l@{\hspace{2em}}l@{\hspace{1em}}l@{}}
\&              &  \verb!\&! &
\_              &  \verb!\_! &
\ldots          &  \verb!\ldots! &
\textbullet     &  \verb!\textbullet! \\
\$              &  \verb!\$! &
\^{}            &  \verb!\^{}! &
\textbar        &  \verb!\textbar! &
\textbackslash  &  \verb!\textbackslash! \\
\%              &  \verb!\%! &
\~{}            &  \verb!\~{}! &
\#              &  \verb!\#! &
\S              &  \verb!\S! \\
\end{tabular}

\subsection{Marcas diacríticas ou acentos}
\begin{tabular}{@{}l@{\ }l|l@{\ }l|l@{\ }l|l@{\ }l|l@{\ }l@{}}
\`o   & \verb!\`o! &
\'o   & \verb!\'o! &
\^o   & \verb!\^o! &
\~o   & \verb!\~o! &
\=o   & \verb!\=o! \\
\.o   & \verb!\.o! &
\"o   & \verb!\"o! &
\c o  & \verb!\c o! &
\v o  & \verb!\v o! &
\H o  & \verb!\H o! \\
\c c  & \verb!\c c! &
\d o  & \verb!\d o! &
\b o  & \verb!\b o! &
\t oo & \verb!\t oo! &
\oe   & \verb!\oe! \\
\OE   & \verb!\OE! &
\ae   & \verb!\ae! &
\AE   & \verb!\AE! &
\aa   & \verb!\aa! &
\AA   & \verb!\AA! \\
\o    & \verb!\o! &
\O    & \verb!\O! &
\l    & \verb!\l! &
\L    & \verb!\L! &
\i    & \verb!\i! \\
\j    & \verb!\j! &
!`    & \verb!~`! &
?`    & \verb!?`! &
\end{tabular}


\subsection{Delimitadores}
\begin{tabular}{@{}l@{\ }ll@{\ }ll@{\ }ll@{\ }ll@{\ }ll@{\ }l@{}}
`       & \verb!`!  &
``      & \verb!``! &
\{      & \verb!\{! &
\lbrack & \verb![! &
(       & \verb!(! &
\textless  &  \verb!\textless! \\
'       & \verb!'!  &
''      & \verb!''! &
\}      & \verb!\}! &
\rbrack & \verb!]! &
)       & \verb!)! &
\textgreater  &  \verb!\textgreater! \\
\end{tabular}

\subsection{Traços}
\begin{tabular}{@{}llll@{}}
\textit{Nome} & \textit{Fonte} & \textit{Exemplo} & \textit{Uso} \\
hyphen  & \verb!-!   & raio-X          & Em palavras (hífen). \\
en-dash & \verb!--!  & 1--5           & Intervalos de números. \\
em-dash & \verb!---! & Sim---ou não?    & Pontuação (apóstrofe).
\end{tabular}


\subsection{Quebras de linha e página}
\settowidth{\MyLen}{\texttt{.pagebreak} }
\begin{tabular}{@{}p{\the\MyLen}%
                @{}p{\linewidth-\the\MyLen}@{}}
\verb!\\!          &  Começa nova linha sem novo parágrafo.  \\
\verb!\\*!         &  Proíbe quebra de página após quebra de linha. \\
\verb!\kill!       &  Não imprime a linha corrente. \\
\verb!\pagebreak!  &  Começa uma nova página. \\
\verb!\noindent!   &  Não indenta a linha corrente.
\end{tabular}


\subsection{Miscelânea}
\settowidth{\MyLen}{\texttt{.rule.w..h.} }
\begin{tabular}{@{}p{\the\MyLen}%
                @{}p{\linewidth-\the\MyLen}@{}}
\verb!\today!  &  \today. \\
\verb!$\sim$!  &  Imprime $\sim$ ao invés de \verb!\~{}!, que faz  \~{}. \\
\verb!~!       &  Espaço protegido, não permite quebra de linha (\verb!P.A.~Cabral!). \\
\verb!\@.!     &  Indica que o \verb!.! finaliza a sentença quando sequido de uma 
                        letra maiúscula. \\
\verb!\hspace{!$l$\verb!}! & Espaço horizontal de comprimento  $l$
                                (Ex: $l=\mathtt{20pt}$). \\
\verb!\vspace{!$l$\verb!}! & Espaço vertical de coprimento  $l$. \\
\verb!\rule{!$w$\verb!}{!$h$\verb!}! & Linha de largura $w$ and altura $h$. \\
\end{tabular}



\section{Ambientes de tabulação}

\subsection{Ambiente \texttt{tabbing}}
\begin{tabular}{@{}l@{\hspace{1.5ex}}l@{\hspace{10ex}}l@{\hspace{1.5ex}}l@{}}
\verb!\=!  &   Marca a tabulação. & \verb!\>!  &   Vai para a  tabulação.
\end{tabular}

Paradas de tabulação podem ser feitas sobre linhas ``invisíveis'' com  \verb!\kill!
no final da linha.  Normalmente \verb!\\! é usado para separar linhas.


\subsection{Ambiente \texttt{tabular}}
\verb!\begin{array}[!\textit{pos}\verb!]{!\textit{cols}\verb!}!   \\
\verb!\begin{tabular}[!\textit{pos}\verb!]{!\textit{cols}\verb!}! \\
\verb!\begin{tabular*}{!\textit{largura}\verb!}[!\textit{pos}\verb!]{!\textit{cols}\verb!}!


\subsubsection{Especificação de coluna no \texttt{tabular}}
\settowidth{\MyLen}{\texttt{p}\{\textit{width}\} \ }
\begin{tabular}{@{}p{\the\MyLen}@{}p{\linewidth-\the\MyLen}@{}}
\texttt{l}    &   Coluna justificada à esquerda.  \\
\texttt{c}    &   Coluna centralizada.  \\
\texttt{r}    &   Coluna justificada à direita. \\
\verb!p{!\textit{width}\verb!}!  &  O mesmo que %
                              \verb!\parbox[t]{!\textit{largura}\verb!}!. \\ 
\verb!@{!\textit{decl}\verb!}!   &  Insere \textit{decl} ao invés de
                                    espaço entre colunas space. \\
\verb!|!      &   Insere uma linha vertical entre colunas. 
\end{tabular}


\subsubsection{Elementos do \texttt{tabular}}
\settowidth{\MyLen}{\texttt{.cline.x-y..}}
\begin{tabular}{@{}p{\the\MyLen}@{}p{\linewidth-\the\MyLen}@{}}
\verb!\hline!           &  Linha horizontal entre linhas.  \\
\verb!\cline{!$x$\verb!-!$y$\verb!}!  &
                        Linha horizontal da coluna $x$ até  $y$. \\
\verb!\multicolumn{!\textit{n}\verb!}{!\textit{cols}\verb!}{!\textit{text}\verb!}! \\
        &  Uma célula que engloba  \textit{n} colunas, com especificação de \textit{cols} colunas.
\end{tabular}


\section{Modo matemático}
Para usar o modo matemático, envolva o texto com  \texttt{\$} ou use
\verb!\begin{equation}!.

\begin{tabular}{@{}l@{\hspace{1em}}l@{\hspace{2em}}l@{\hspace{1em}}l@{}}
Sobrescrito$^{x}$       &
\verb!^{x}!             &  
Subscrito$_{x}$         &
\verb!_{x}!             \\  
$\frac{x}{y}$           &
\verb!\frac{x}{y}!      &  
$\sum_{k=1}^n$          &
\verb!\sum_{k=1}^n!     \\  
$\sqrt[n]{x}$           &
\verb!\sqrt[n]{x}!      &  
$\prod_{k=1}^n$         &
\verb!\prod_{k=1}^n!    \\ 
\end{tabular}

\subsection{Símbolos no modo matemático}

% The ordering of these symbols is slightly odd.  This is because I had to put all the
% long pieces of text in the same column (the right) for it all to fit properly.
% Otherwise, it wouldn't be possible to fit four columns of symbols here.

\begin{tabular}{@{}l@{\hspace{1ex}}l@{\hspace{1em}}l@{\hspace{1ex}}l@{\hspace{1em}}l@{\hspace{1ex}} l@{\hspace{1em}}l@{\hspace{1ex}}l@{}}
$\leq$          &  \verb!\leq!  &
$\geq$          &  \verb!\geq!  &
$\neq$          &  \verb!\neq!  &
$\approx$       &  \verb!\approx!  \\
$\times$        &  \verb!\times!  &
$\div$          &  \verb!\div!  &
$\pm$           & \verb!\pm!  &
$\cdot$         &  \verb!\cdot!  \\
$^{\circ}$      & \verb!^{\circ}! &
$\circ$         &  \verb!\circ!  &
$\prime$        & \verb!\prime!  &
$\cdots$        &  \verb!\cdots!  \\
$\infty$        & \verb!\infty!  &
$\neg$          & \verb!\neg!  &
$\wedge$        & \verb!\wedge!  &
$\vee$          & \verb!\vee!  \\
$\supset$       & \verb!\supset!  &
$\forall$       & \verb!\forall!  &
$\in$           & \verb!\in!  &
$\rightarrow$   &  \verb!\rightarrow! \\
$\subset$       & \verb!\subset!  &
$\exists$       & \verb!\exists!  &
$\notin$        & \verb!\notin!  &
$\Rightarrow$   &  \verb!\Rightarrow! \\
$\cup$          & \verb!\cup!  &
$\cap$          & \verb!\cap!  &
$\mid$          & \verb!\mid!  &
$\Leftrightarrow$   &  \verb!\Leftrightarrow! \\
$\dot a$        & \verb!\dot a!  &
$\hat a$        & \verb!\hat a!  &
$\bar a$        & \verb!\bar a!  &
$\tilde a$      & \verb!\tilde a!  \\

$\alpha$        &  \verb!\alpha!  &
$\beta$         &  \verb!\beta!  &
$\gamma$        &  \verb!\gamma!  &
$\delta$        &  \verb!\delta!  \\
$\epsilon$      &  \verb!\epsilon!  &
$\zeta$         &  \verb!\zeta!  &
$\eta$          &  \verb!\eta!  &
$\varepsilon$   &  \verb!\varepsilon!  \\
$\theta$        &  \verb!\theta!  &
$\iota$         &  \verb!\iota!  &
$\kappa$        &  \verb!\kappa!  &
$\vartheta$     &  \verb!\vartheta!  \\
$\lambda$       &  \verb!\lambda!  &
$\mu$           &  \verb!\mu!  &
$\nu$           &  \verb!\nu!  &
$\xi$           &  \verb!\xi!  \\
$\pi$           &  \verb!\pi!  &
$\rho$          &  \verb!\rho!  &
$\sigma$        &  \verb!\sigma!  &
$\tau$          &  \verb!\tau!  \\
$\upsilon$      &  \verb!\upsilon!  &
$\phi$          &  \verb!\phi!  &
$\chi$          &  \verb!\chi!  &
$\psi$          &  \verb!\psi!  \\
$\omega$        &  \verb!\omega!  &
$\Gamma$        &  \verb!\Gamma!  &
$\Delta$        &  \verb!\Delta!  &
$\Theta$        &  \verb!\Theta!  \\
$\Lambda$       &  \verb!\Lambda!  &
$\Xi$           &  \verb!\Xi!  &
$\Pi$           &  \verb!\Pi!  &
$\Sigma$        &  \verb!\Sigma!  \\
$\Upsilon$      &  \verb!\Upsilon!  &
$\Phi$          &  \verb!\Phi!  &
$\Psi$          &  \verb!\Psi!  &
$\Omega$        &  \verb!\Omega!  
\end{tabular}
\footnotesize

%\subsection{Special symbols}
%\begin{tabular}{@{}ll@{}}
%$^{\circ}$  &  \verb!^{\circ}! Ex: $22^{\circ}\mathrm{C}$: \verb!$22^{\circ}\mathrm{C}$!.
%\end{tabular}

\section{Referências bibliográficas e citações}
Quando usar \BibTeX, você precisa rodar  \texttt{latex}, \texttt{bibtex},
e \texttt{latex} duas vezes mais para resolver as dependências.

\subsection{Tipos de citação}
\settowidth{\MyLen}{\texttt{.shortciteN.key..}}
\begin{tabular}{@{}p{\the\MyLen}@{}p{\linewidth-\the\MyLen}@{}}
\verb!\cite{!\textit{key}\verb!}!       &
        Lista completa de autores e ano. (Watson and Crick 1953) \\
\verb!\citeA{!\textit{key}\verb!}!      &
        Lista completa de autores. (Watson and Crick) \\
\verb!\citeN{!\textit{key}\verb!}!      &
        Lista completa de autores e ano. Watson and Crick (1953) \\
\verb!\shortcite{!\textit{key}\verb!}!  &
        Lista abreviada de autores e ano. ? \\
\verb!\shortciteA{!\textit{key}\verb!}! &
        Lista abreviada de autores. ? \\
\verb!\shortciteN{!\textit{key}\verb!}! &
        Lista abreviada de autores e ano. ? \\
\verb!\citeyear{!\textit{key}\verb!}!   &
        Cita ano somente. (1953) \\
\end{tabular}

Todas as anteriores tem uma variante  \texttt{NP} sem parênteses;
Ex. \verb!\citeNP!.


\subsection{Tipos de entrada \BibTeX\ }
\settowidth{\MyLen}{\texttt{.mastersthesis.}}
\begin{tabular}{@{}p{\the\MyLen}@{}p{\linewidth-\the\MyLen}@{}}
\verb!@article!         &  Artigo de jornal ou revista. \\
\verb!@book!            &  Livro com publicador. \\
\verb!@booklet!         &  Livro sem publicador. \\
\verb!@conference!      &  Artigo em um encontro. \\
\verb!@inbook!          &  Uma parte de um livro e/ou intervalo de páginas. \\
\verb!@incollection!    &  Uma parte de um Livro com seu próprio título. \\
%\verb!@manual!          &  Technical documentation. \\
%\verb!@mastersthesis!   &  Master's thesis. \\
\verb!@misc!            &  Se nada se adequa. \\
\verb!@phdthesis!       &  Tese de doutorado. \\
\verb!@proceedings!     &  Anais de uma conferência. \\
\verb!@techreport!      &  Relatótio técnico, usualmente numerado em série. \\
\verb!@unpublished!     &  Não publicado. \\
\end{tabular}

\subsection{Campos \BibTeX\ }
\settowidth{\MyLen}{\texttt{organization.}}
\begin{tabular}{@{}p{\the\MyLen}@{}p{\linewidth-\the\MyLen}@{}}
\verb!address!         &  Endereço do publicador.  Não necess. em geral.  \\
\verb!author!           &  Nomes de autores, de formato .... \\
\verb!booktitle!        &  Título do livro quando parte deste é citado. \\
\verb!chapter!          &  Capítulo ou número de seção. \\
\verb!edition!          &  Edição do livro. \\
\verb!editor!           &  Nomes dos editores. \\
\verb!institution!      &  Instituição sede do relatório técnico. \\
\verb!journal!          &  Nome do jornal. \\
\verb!key!              &  Usado para referência cruzada, sem autor. \\
\verb!month!            &  Mês publicado. Use abreviação de 3 letras. \\
\verb!note!             &  Informação adicional. \\
\verb!number!           &  Númerodo jornal ou revista. \\
\verb!organization!     &  Organização que sedia a conferência. \\
\verb!pages!            &  Intervalo de páginas (\verb!2,6,9--12!). \\
\verb!publisher!        &  Nome do publicador. \\
\verb!school!           &  Nome da escola (para tese). \\
\verb!series!           &  Nome da série de livros. \\
\verb!title!            &  Título do trabalho. \\
\verb!type!             &  Tipo de relat. técnico,  ex. ``Nota de Pesquisa''. \\
\verb!volume!           &  Volume de um jornal ou livro. \\
\verb!year!             &  Ano de publicação. \\
\end{tabular}
Nem todos os campos precisam sem preenchidos.  Veja exemplo abaixo.

\subsection{Arquivos de estilo \BibTeX\ comuns}
\begin{tabular}{@{}l@{\hspace{1em}}l@{\hspace{3em}}l@{\hspace{1em}}l@{}}
\verb!abbrv!    &  Padrão & \verb!abstract! &  \texttt{alpha} com resumo (abstract) \\
\verb!alpha!    &  Padrão & \verb!apa!      &  APA \\
\verb!plain!    &  Padrão & \verb!unsrt!    &  Não ordenado \\
\end{tabular}

O documento \LaTeX\ deve ter as  duas linhas seguintesantes do 
\verb!\end{document}!, onde \verb!bibfile.bib! é o nome do arquivo 
\BibTeX\ .
\begin{verbatim}
\bibliographystyle{plain}
\bibliography{bibfile}
\end{verbatim}

\subsection{Exemplo \BibTeX\ }
O banco de dados  \BibTeX\ vai em um arquivo chamado
\textit{file}\texttt{.bib}, que é processado com  \verb!bibtex file!. 
\begin{verbatim}
@String{N = {Na\-ture}}
@Article{WC:1953,
  author  = {James Watson and Francis Crick},
  title   = {A structure for Deoxyribose Nucleic Acid},
  journal = N,
  volume  = {171},
  pages   = {737},
  year    = 1953
}
\end{verbatim}


\section{Documento amostra \LaTeX\ }
\begin{verbatim}
\documentclass[11pt]{article}
\usepackage{fullpage}
\title{Template}
\author{Name}
\begin{document}
\maketitle

\section{seção}
\subsection*{subseção sem número}
texto \textbf{texto negrito} texto. 
Com matemática: $2+2=5$
\subsection{subseção}
texto \emph{texto enfatizado} texto. \cite{WC:1953}
descobrio a estrutura do ADN.

Uma  tabela:
\begin{table}[!th]
\begin{tabular}{|l|c|r|}
\hline
primeira  &  linha   &  dados \\
segunda &  linhas  &  dados \\
\hline
\end{tabular}
\caption{Este é o título}
\label{ex:table}
\end{table}

A tabela é numerada \ref{ex:table}.
\end{document}
\end{verbatim}



\rule{0.3\linewidth}{0.25pt}
\scriptsize

Copyright \copyright\ 2006 Winston Chang  \\
Tradução e adaptação \textregistered\ 2009 Silvio C. G. Granja 



% Should change this to be date of file, not current date.
%\verb!$Revision: 1.13 $, $Date: 2008/05/29 06:11:56 $.!
\verb!$Revision: 1.13 $, $Data: 2009/04/05 15:26:47 $.!

\url{http://www.stdout.org/~winston/latex/}\\
\url{http://tug.ctan.org/cgi-bin/ctanPackageInformation.py?id=latexcheat}

\end{multicols}
\end{document}
