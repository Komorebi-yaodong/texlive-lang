  %% diplomarbeit.tex
  %% Copyright 2015 Simon M. Laube
  %
  % This work may be distributed and/or modified under the
  % conditions of the LaTeX Project Public License, either version 1.3
  % of this license or (at your option) any later version.
  % The latest version of this license is in
  %   http://www.latex-project.org/lppl.txt
  % and version 1.3 or later is part of all distributions of LaTeX
  % version 2005/12/01 or later.
  %
  % This work has the LPPL maintenance status `author maintained'.
  % 
  % The Current Maintainer of this work is S. M. Laube
  %
  % This work consists of the files listed in ./Help/files.txt

%%=====================================================%%
%% Neues Diplomarbeitstemplate der ET	   			   %%
%% Abteilung ab 2013/2014				   			   %%
%% Erstellt von Simon Michael Laube		   			   %%
%% Betreut von  Prof. Mag. Dipl.-Ing. Dr. Daniel Asch  %%
%%			    Prof. Dipl.-Ing. Dr. Wilhelm Haager	   %%
%%=====================================================%%
%% Dokumentklasse KOMA-Script Report
\documentclass[paper=a4,12pt]{scrreprt}
% Encoding UTF8
\usepackage[utf8]{inputenc}
% 8 Bit Aufloesung der Buchstaben
\usepackage[T1]{fontenc}
% Seitenraender
\usepackage[scale=0.72]{geometry}
% Spracheinstellungen
\usepackage[english, naustrian]{babel} % your native language must be the last one!!
% erweiterte Farbenpalette
\usepackage[dvipsnames]{xcolor}
% Abbildungen
\usepackage{graphicx}
% Tabellen (erweitert)
\usepackage{tabularx}
% TikZ + Circuit-TikZ (fuer Schaltungen)									
\usepackage[europeanresistors,							
			europeaninductors]{circuitikz}
% Nuetzliche TikZ Libraries
\usetikzlibrary{arrows,automata,positioning}
% Mathematikpakete!
\usepackage{amsmath,amssymb}							
%\usepackage{mathtools}	
% PDF Einbindung (zB Datenblaetter)
\usepackage{pdfpages}
% Source Code Einbindung, Setup siehe:
% http://en.wikibooks.org/wiki/LaTeX/Source_Code_Listings									
\usepackage{listings,scrhack} %scrhack vermeidet Umschaltung auf KOMA Floats..			
			
\usepackage{eurosym}
\usepackage{lscape}
% Diplomarbeitsspezifisches Package etdipa
\usepackage{etdipa}

%% Abkuerzungsverzeichnis
\usepackage[]{acronym}

%% Todos
\usepackage[]{todonotes}

%% Ganttdiagramme
\usepackage{pgfgantt}

%% Subfigures
\usepackage[lofdepth]{subfig}



%%==== Definitionen fuer die Diplomarbeit ============%%
\dokumenttyp{DIPLOMARBEIT}
\title{Musterdokument}
\author{Simon Michael Laube \and NoName1 \and NoName2}
\date{\today}
\place{St. P\"olten}
\schuljahr{2014/15}
\professor{Prof. NoName \and Dr. Engineer}
\dipacolor{ETred}
%%====================================================%%


% Hyperlinks im Dokument
\usepackage[colorlinks=true,
			linkcolor=black,
			citecolor=green,
			bookmarks=true,
			urlcolor=black,
			bookmarksopen=true]{hyperref}

\begin{document}

\frontmatter

%%================ Titelseite ==========================%%
\maketitle
% Verantwortliche/Verfasser
\responsible{Simon Michael Laube}
%%======================================================%%


%%================ Eidesstattliche Erklaerung ==========%%
\begin{Eid}%Unterschrift der Diplomanden hinzufuegen!
\unterschrift{Simon Michael Laube}
\unterschrift{Schüler 2}
\unterschrift{Schüler 3}
\end{Eid}\newpage
%%======================================================%%

%%================ Diplomandenvorstellung ==============%%
%% start of file diplomanden.tex

%% Diplomandenvorstellung:
\begin{Diplomandenvorstellung}
%% Schueler1 
\diplomand{Max~Mustermann}
		  {12.12.2012 in St.P\"olten}
		  {Langestra\ss e 13}
		  {3100 St.Pölten}
		  {\schule{2010--2015}{HTBLuVA St.Pölten, Abteilung für Elektrotechnik}
		  \schule{2006--2010}{Gymnasium XY}}
		  {max.muster@xy.at}
		  {Images/bild}
\newpage	  
%% Schueler2
\diplomand{Manuela~Musterfrau}
		  {12.12.2012 in St.P\"olten}
		  {Langestra\ss e 14}
		  {3100 St.Pölten}
		  {\schule{2010--2015}{HTBLuVA St.Pölten, Abteilung für Elektrotechnik}
		   \firma{2006--2010}{Elektrofirma XY}
		   \schule{2002--2006}{Gymnasium XY}}
		  {manuela.muster@xy.at}
		  {Images/bild}
\end{Diplomandenvorstellung}

%% end of file diplomanden.tex

%%======================================================%%



%%================ Danksagungen ========================%%
%% start of file danksagungen.tex

%% Danksagungen:
\begin{Danksagung}
Wir bedanken uns bei \dots
\end{Danksagung}
\newpage

%% end of file danksagungen.tex
%%======================================================%%

%%================ Abstract /Zusammenfass. =============%%
%% start of file abstract.tex

\selectlanguage{english}
\begin{abstract}

\end{abstract}
\selectlanguage{naustrian}

%% end of file abstract.tex
%% start of file zusammenfassung.tex

\selectlanguage{naustrian}
\begin{abstract}

\end{abstract}

%% end of file zusammenfassung.tex
%\selectlanguage{english} % necessary for English speaking users
% delete this line if your native language is German 
%%======================================================%%

%%================ Inhaltsverzeichnis ==================%%
\tableofcontents
%%======================================================%%


%Ab hier Hauptteil
\mainmatter

\appendix

%%================ Abkuerzungsverzeichnis ==============%%
%% start of file abkuerzungen.tex

% Abkuerzungsverzeichnis
\addchap{
	\iflanguage{english}{Acronyms}{Abkürzungsverzeichnis}}
\begin{acronym}[ACRONYM]
\acro{tikz}[TikZ]{\TikZ{} ist kein Zeichenprogramm}
\acro{spi}[SPI]{Serial Peripheral Interface}
\end{acronym}\newpage

%% end of file abkuerzungen.tex
%%======================================================%%


%%================ Abbildungsverzeichnis ===============%%
\setcounter{lofdepth}{2}
\dipalistoffigures
%%======================================================%%

%%================ Tabellenverzeichnis  ================%%
\setcounter{lotdepth}{2}
\dipalistoftables
%%======================================================%%

%%================ Literaturverzeichnis ================%%
\newpage
%% start of file literatur.tex

%% Literaturverzeichnis:
\begin{Literatur}
% The TeXbook by D. E. Knuth
\bibitem[1]{TeXbook}{\textbf{Donald~E.~Knuth:} \emph{The \TeX{}book}. 1986, {\scshape Addison--Wesley} Verlag,\\
ISBN-13: 978-0-201-13447-6} 
%
\end{Literatur}

%% end of file literatur.tex
 
	 
%%======================================================%%
\end{document}
