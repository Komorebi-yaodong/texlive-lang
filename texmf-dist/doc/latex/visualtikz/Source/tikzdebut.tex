\documentclass[a4paper,10pt]{article}

 \usepackage{fontspec}
\usepackage[french,english]{babel}

%\TFRGB{\selectlanguage{french}}{\selectlanguage{english}}
\usepackage{tikzpeople}

\usepackage{amsmath,amsfonts,amssymb}

\usepackage{pdfpages}  


%\usepackage{pst-all}

\usepackage{graphicx} 
\usepackage{hyperref}

\usepackage{animate}
\usepackage{makeidx}
%\usepackage{wrapfig}
%\usepackage{tikz-dependency}
\usepackage{pgfplots} %<<<<<<<<<<<<<<<<<<<<<<<<<<<<< 
\usepackage{tikz}
\usepackage{tkz-tab}

 
%\usepgflibrary{shapes.callouts}
\usepackage{tikz-qtree}
\usepackage{tkz-tab}
\usepackage{csquotes}

  
\usetikzlibrary{angles}
\usetikzlibrary{arrows}

\usetikzlibrary{shadings}
\usetikzlibrary{calc}
\usetikzlibrary{backgrounds}
\usetikzlibrary{decorations.pathmorphing}

\usetikzlibrary{decorations.markings}
\usetikzlibrary{decorations.footprints}
\usetikzlibrary{decorations.shapes}
\usetikzlibrary{decorations.text}
\usetikzlibrary{decorations.fractals}
\usepgflibrary{shapes.geometric}
\usetikzlibrary{intersections}
\usetikzlibrary{scopes}
\usetikzlibrary{shapes.symbols}
\usetikzlibrary{shapes.arrows}
\usetikzlibrary{shapes.callouts}
\usetikzlibrary{shapes.misc}
\usepgflibrary{shapes.multipart}
\usetikzlibrary{plotmarks}
\usetikzlibrary{trees}
\usetikzlibrary{fadings}
\usetikzlibrary{arrows.meta}
\usetikzlibrary{bending}
\usetikzlibrary{fit}
%\usetikzlibrary{circuits}
\usetikzlibrary{circuits.ee.IEC}
\usetikzlibrary{circuits.logic.IEC}
\usetikzlibrary[circuits.logic.US]
\usetikzlibrary{circuits.logic.CDH}
%\usetikzlibrary{decorations}
\usetikzlibrary{shapes.gates.logic.IEC}
\usetikzlibrary{matrix}
\usetikzlibrary{chains}
%\usetikzlibrary{circuit.plc.sfc}
\usepackage{tikzsymbols}
\usetikzlibrary{datavisualization}
\usetikzlibrary{datavisualization.formats.functions}
%
\usepackage{tikzducks}

\usepackage{tikzrput}
\usepackage{pgfornament}

%\usetikzlibrary{babel}
\usetikzlibrary{math}
\usetikzlibrary{optics}
\usetikzlibrary{through}
\usetikzlibrary{turtle}
\usetikzlibrary{quotes}


\pgfplotsset{compat=1.8}
\usetikzlibrary{positioning}

\usepackage{geometry}
\geometry{a4paper,top={3cm}}

\usepackage{ifpdf}
\usepackage{ifluatex}
\usetikzlibrary{spy}




%====================================================================

\makeindex

\newcommand{\AC}[1]{\{#1\}}

\newcommand{\BS}[1]{$\backslash$#1}

\newcommand{\BSB}[1]{\textbf{\color{blue} {$\backslash$#1}}}


\newcommand{\BSR}[1]{\textbf{\color{red}  $\backslash$#1}}

%\newcommand{\RDDX}[2]{{\color{red}#1} \index{\textbf{3 Paramètres et options}!#2=#1}}


\newcommand{\RRR}[1]{\tikz[baseline=-1mm,inner sep=2pt]  \draw node[draw,fill=red!20] {{\footnotesize  PGFmanual section :  #1}} ; }

\newcommand{\RRP}[1]{\tikz[baseline=-1mm]  \draw node[draw,fill=red!20] {{\footnotesize  pgfplots section :  #1}} ; }

%\newcommand{\RRR}[1]{\tikz[baseline=-1mm]  \draw node[draw,fill=red!20] {{\footnotesize  PGFmanual section :  #1}} ;\index{\textbf{5 PGFmanual }!#1} }

\newcommand{\DFR}{ \tikzpicture[scale=.25]
\draw [fill=blue](0,0) rectangle (3,1.5);
\draw [fill=white](1,0) rectangle (2,1.5);
\draw [fill=red](2,0) rectangle (3,1.5);\endtikzpicture }

\newcommand{\DGE}{ \tikzpicture[scale=.25]
\draw [fill=yellow](0,0) rectangle (3,.5);
\draw [fill=red]((0,.5) rectangle (3,1);
\draw [fill=black](0,1) rectangle (3,1.5); \endtikzpicture }

\newcommand{\DGB}{ \tikzpicture[scale=.25]
\draw [fill=blue](0,0) rectangle (3,1.5);
\draw [white,line width=.1cm](0,0) -- (3,1.5);
\draw [white,line width=.1cm](0,1.5) -- (3,0);
\draw [white,line width=.1cm](1.5,0) -- (1.5,1.5);
\draw [white,line width=.1cm](0,0.75) -- (3,0.75);
\draw [red,line width=.05cm](0,0) -- (3,1.5);
\draw [red,line width=.05cm](0,1.5) -- (3,0);
\draw [red,line width=.05cm](1.5,0) -- (1.5,1.5);
\draw [red,line width=.05cm](0,0.75) -- (3,0.75);
\endtikzpicture }



\TFRGB{
\newcommand{\ESS}[1]{\textbf{\textbackslash begin\AC{#1}}\index{\textbf{1 Environnements}!#1}}

\newcommand{\BSS}[1]{\textbf{\textbackslash{#1}}\index{\textbf{2 Commandes}!#1 @\textbackslash{}#1}}


\newcommand{\DDD}[1]{{\color{red}  #1}\index{\textbf{3 Paramètres et options}!#1}}
\newcommand{\RDD}[1]{{\color{red}  #1}\index{\textbf{3 Paramètres et options}!#1}}
%4
\newcommand{\BDD}[1]{{\color{blue}  #1}\index{\textbf{4 Valeurs Tikz}!#1}}

\newcommand{\RDDX}[2]{{\color{red}#1} \index{\textbf{4 Valeurs Tikz}!#1 (#2)}}
%5
\newcommand{\FDD}[1]{{\color{red}  #1}\index{\textbf{5 Extrémit\'es}!#1}}
}
{
\newcommand{\ESS}[1]{\textbf{\textbackslash{#1}}\index{\textbf{1 Environments}!#1 @\textbackslash{}#1}}
%2
\newcommand{\BSS}[1]{\textbf{\textbackslash{#1}}\index{\textbf{2 Commands}!#1 @\textbackslash{}#1}}
%3
\newcommand{\RDD}[1]{{\color{red}  #1}\index{\textbf{3 Parameters and options}!#1}}

\newcommand{\DDD}[1]{{\color{red}  #1}\index{\textbf{3 Parameters and options}!#1}}

\newcommand{\RDDX}[2]{{\color{red}#1} \index{\textbf{4 Values Tikz}!#1 (#2)}}
%
\newcommand{\BDD}[1]{{\color{blue}  #1}\index{\textbf{4 Values Tikz}!#1}}

\newcommand{\FDD}[1]{{\color{red}  #1}\index{\textbf{5 Extremities}!#1}}
}

\newcommand{\rouge}[1] {{\color{red}  #1}}
\newcommand{\blll}[1] {{\color{blue}  #1}}
