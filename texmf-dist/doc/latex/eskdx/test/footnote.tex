\documentclass[russian,utf8,footnoteasterisk]{eskdtext}
%\usepackage[T2A]{fontenc}
%\usepackage{pscyr}
\usepackage[unicode]{hyperref}

\begin{document}
\section{Сноски}
\point Если необходимо пояснить отдельные данные, приведенные в
документе, то эти данные следует обозначать надстрочными знаками
сноски.

Сноски в тексте располагают с абзацного отступа в конце страницы, на
которой они обозначены, и отделяют от текста короткой тонкой
горизонтальной линией с левой стороны, а к данным, расположенным в
таблице, в конце таблицы над линией, обозначающей окончание таблицы.

\point Знак сноски ставят непосредственно после того слова, числа,
символа, предложения, к которому дается пояснение, и перед текстом
пояснения.

\point Знак сноски выполняют арабскими цифрами со скобкой и помещают
на уровне верхнего обреза шрифта.

Пример "--- ,,... печатающее устройство\footnote{текст сноски}...''.
\newpage
Eot один пример "--- ,,... печатающее устройство\footnote{текст сноски}...''.

\end{document}
