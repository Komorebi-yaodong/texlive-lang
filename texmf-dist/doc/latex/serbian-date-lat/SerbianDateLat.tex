
\documentclass[a4paper,12pt]{article}
\usepackage[T1]{fontenc}
\usepackage[latin1]{inputenc}
\usepackage{lmodern}
\usepackage{verbatim}
\usepackage{soul}
\usepackage{xspace}

\title{The introduction of date definitions for Serbian language in latin script \\
%       \textit{\&} 
\textsf{serbian-date-lat} package}  
\author{Zoran T. Filipovi\'{c}\protect\footnote{email: zoran.filipovic@yahoo.com} \\ 
        Jurija Gagarina 263/6 \\ 
        11070 Novi Beograd, Srbija}
        
\usepackage[english,serbian]{babel}
\usepackage{serbian-date-lat}
\usepackage{quotmark}

\begin{document}
\frenchspacing
\maketitle

\selectlanguage{english}
\begin{abstract}
The \verb|babel| documentacions in \textbf{section 53}, and in the \verb|serbian.dtx| 
defines date for serbian language in latin scripts, but in the way which not 
recognize modern gramaticall rules of serbian language. Major mistake in the name of month
which now call \verb|\juni| and \verb|\juli| which is propriate translatation is 
\verb|\jun| and \verb|\jul| and after a year is dot. In this document I try to introduction propriate way of date definitions for Serbian language in latin script. 
\end{abstract}

\section{Confusion about numbers}

In serbian language exist several way to type a date. The most common way is:

\begin{enumerate}
\item{Danas je 13. jun 2011. godine. (Today is 13. july 2011. year.)}
\item{Danas je 01.06.2011. godine. (Today is 01.06.2011. year.)}
\item{Danas je 13.VI.2011. godine. (Today is 13.VI.2011. year.)}
\end{enumerate}

The first way is a best way. He represents all grammatical rule, which exist in 
serbian language. Date is contains a nambers of a day, name of the month writing 
in lowercase, and year. After a day is dot. Day and month is separate by 
\textit{space}, and month and year is separate by \textit{space}. After a year is dot. 
 
The second way is a law way, which is exist in lawer processing, and for a court.
Date is contains a nambers of a day, month and year separating whith a dots, whith no 
\textit{extra space}, because is, in this world, always exist attempt to overwriting or 
retyping a date, and that is a confusion for a legal paper. 

\begin{description}
\item[Note:] In a legal papers we maybe wrote a date in samething like that: \\
1. 6. 2011. (whith a \textit{space}) and that is a correct date. If a wrote in 
this way, error is posible, if we retyping by 11. 6. 2011. or 21. 6. 2011.
Because of that, in legal papers, always is typing a full number with numbers 
\textit{zero}, and no \textit{extra space}. \textbf{Example:} 01.06.2011. If we have 
typing a date in this way, we have no confusing. 
\end{description}

The last way exist is same papers in local purpose or, if you back in past, etc. In a 
modern worlds this style is no more purpose. Date is contains a nambers of a day, 
month and year separating whith a dots, whith no \textit{extra space}. Day and year is a 
roman letters, month is arabic letters. 

\bigskip
\hrule

\begin{description}
%\selectlanguage{serbian} 
\item[Example:] Od mnogih pohvala, evo ukratko samo nekoliko, koje ubedljivo 
govore o velikom uspehu Perunovog pevanja i guslanja. \tqt{Svome milom Petru
Perunovi\'{c}u, prvom srpskom guslaru koga sam ikad \v{c}uo i koji me je svojom
divnom pesmom zadivio. Njujork, 11. oktobar 1916. dr. Mihailo Pupin.} --- \tqt{Svojom
pesmom svuda \'{c}ete ste\'{c}i prijatelje, naro\v{c}ito me{\dj}u Srbima i 
Hrvatima \dots{} Ve\v{c}no pesmom i guslama \v{s}irite slogu i jedinsvo. Time 
\'{c}ete najvi\v{s}e koristiti svome narodu. Njujork, 21. jun 1919. Nikola 
Tesla.}\footnote{Book: Golgota i Vaskrs Srbije 1916---1918., article: Zna\v{c}aj 
guslara Petra Perunovi\'{c}a---Peruna, author: Slavko Peji\'{c}, page 385, 
paragraph no.5, publisher: Beogradsko Izdava\v{c}ko---Grafi\v{c}ki Zavod (BIGZ), 
Beograd 1971.} 
\end{description}

\hrule
\bigskip

\section{The way out}

Major changes in translation date for serbian language is in \verb|\juni,\juli| and 
after a years is a dot. The correct name of the month is \verb|\jun,\jul| and today 
is\selectlanguage{serbian} \today \selectlanguage{english} but not tomorow.
To produce a date in serbian language in latin script just put 
in preambula samething like that:

\begin{verbatim}
\usepackage[T1]{fontenc}
\usepackage[latin1,cp1250]{inputenc}
\end{verbatim}

That is correct setup for most users in Serbia, because is almost \textit{all}
on Microsoft operating system. After that you put, in preambula, few more 
line like this:

\begin{verbatim}
\usepackage[serbian]{babel}
\def\dateserbian{%
  \def\today{\number\day .~\ifcase\month\or
    januar\or februar\or mart\or april\or maj\or
    jun\or jul\or avgust\or septembar\or oktobar\or 
    novembar\or decembar\fi \space \number\year.\space}}
\end{verbatim}

So, when you type \verb|\today| this a produce a date in serbian langunage, in 
latin script, in a best way which know a serbian grammatical rule 
(\textbf{Section 1.}, first most common way).  

\section{The package}

Also, I am produce a package with \LaTeX\ style extensions. So, if you prefer packages
you are must just \textit{after} a line \verb|\selectlanguage{serbian}| put the line 
\verb|\usepackage{serbian-date-lat}| and push you PC machine to working.

\section{Working}

This way \textit{\&} package, which I represents in section 2 and section 3, working 
in \textit{4all} \LaTeX\ documentclass \textit{\&} \verb|memoir| documentclass.

\section{Conclusion}

I suppose, in the future, the correct setup of date definintions for serbian 
language in latin scripts, which I represent in this paper, is in the integrate 
in a babel package, and replace date definitions which we know now.  

\section{Acknowledgements}

The greatest appreciation for the patient and thorough explanation of writing 
the date in the Serbian language, I owe Zoranu Duku\'{c}u i Jelici {\DJ}or{\dj}evi\'{c}. 
Jelica {\DJ}or{\dj}evi\'{c} is slavistic. She has worked in the publishing book company
which called \textsc{Prosveta}. Actively writes poetry books that are printed in the Serbian book market.

\selectlanguage{serbian}
\begin{flushleft}
\textsc{Beograd}, \today
\end{flushleft} 
\end{document}

\end{document}