% test file for the Ibycus-Babel interface, version 2 
% Peter Heslin, Walter Schmidt, Nov.2004

\documentclass[12pt]{article}
\usepackage[ibycus,english]{babel}
\usepackage[textwidth=7cm]{geometry}
\begin{document}

\def\mytext{%
  (Hrodo'tou Qouri'ou i(stori'hs a)po'decis h('de, w(s mh'te ta`
  geno'mena e)c a)nqrw'pwn tw=| xro'nw| e)ci'thla ge'nhtai, mh'te
  e)'rga mega'la te kai` qwmasta', ta` me`n ('Ellhsi, ta` de`
  barba'roisi a)podexqe'nta, a)kle'a ge'nhtai, ta' te a)'lla kai` di'
  h(`n ai)ti'hn e)pole'mhsan a)llh'loisi.
  Perse'wn me'n nun oi( lo'gioi Foi'nikas ai)ti'ous fasi` gene'sqai
  th=s diaforh=s; tou'tous ga'r, a)po` th=s )Eruqrh=s kaleome'nhs
  qala'sshs a)pikome'nous e)pi` th'nde th`n qa'lassan kai`
  oi)kh'santas tou=ton to`n xw=ron to`n kai` nu=n oi)ke'ousi, au)ti'ka
  nautili'h|si makrh=|si e)piqe'sqai, a)pagine'ontas de` forti'a
  Ai)gu'ptia' te kai` )Assu'ria th=| te a)'llh| [xw'rh|]
  e)sapikne'esqai kai` dh` kai` e)s )'Argos; to` de` )'Argos tou=ton
  to`n xro'non proei=xe a('pasi tw=n e)n th=| nu=n (Ella'di
  kaleome'nh| xw'rh|. )Apikome'nous de` tou`s Foi'nikas e)s dh` to`
  )'Argos tou=to diati'qesqai to`n fo'rton. Pe'mpth| de` h)` e('kth|
  h(me'rh| a)p' h(=s a)pi'konto, e)cempolhme'nwn sfi sxedo`n pa'ntwn,
  e)lqei=n e)pi` th`n qa'lassan gunai=kas a)'llas te polla`s kai` dh`
  kai` tou= basile'os qugate'ra; to` de' oi( ou)'noma ei)=nai, kata`
  tw)uto` to` kai` ('Ellhnes le'gousi, )Iou=n th`n )Ina'xou. Tau'tas
  sta'sas kata` pru'mnhn th=s neo`s w)ne'esqai tw=n forti'wn tw=n sfi
  h)=n qumo`s ma'lista, kai` tou`s Foi'nikas diakeleusame'nous
  o(rmh=sai e)p' au)ta's. Ta`s me`n dh` ple'onas tw=n gunaikw=n
  a)pofugei=n, th`n de` )Iou=n su`n a)'llh|si a(rpasqh=nai;
  e)sbalome'nous de` e)s th`n ne'a oi)'xesqai a)pople'ontas e)p'
  Ai)gu'ptou. Ou('tw me`n )Iou=n e)s Ai)'gupton a)pike'sqai le'gousi
  Pe'rsai, ou)k w(s ('Ellhnes, kai` tw=n a)dikhma'twn prw=ton tou=to
  a)'rcai; meta` de` tau=ta (Ellh'nwn tina's (ou) ga`r e)'xousi
  tou)'noma a)phgh'sasqai) fasi` th=s Foini'khs e)s Tu'ron
  prossxo'ntas a(rpa'sai tou= basile'os th`n qugate'ra Eu)rw'phn;
}

\begin{otherlanguage}{ibycus}
  \mytext
  \showhyphens{\mytext}
\end{otherlanguage}

\end{document}

Dimitrios Filippou's improved hyphenation patterns discover far more
hyphenation points than the default Latex patterns, and are more
accurate, especially for compound words.  Here is an example of the
default patterns at work on the start of Herodotus:
 
(H-ro-do'-tou Qou-ri'-ou i(sto-ri'hs a)po'-de-cis h('de, w(s mh'-te
 ta` ge-no'-me-na e)c a)n-qrw'-pwn tw=| xro'-nw| e)ci'-th-la
 ge'-nh-tai, mh'-te e)'r-ga me-ga'-la te kai` qw-ma-sta', ta` me`n
 ('El-lh-si, ta` de` bar- ba'-roi-si a)po-de-xqe'n-ta, a)kle'a
 ge'-nh-tai, ta' te a)'l-la kai` di' h(`n a i)-ti'hn e)po-le'-mh-san
 a)l-lh'-loi-si. Per-se'wn me'n nun oi( lo'-gioi Foi'-n i-kas
 ai)-ti'-ous fa-si` ge-ne'-sqai th=s dia-fo-rh=s; tou'-tous ga'r,
 a)po` th =s )E-ru-qrh=s ka-leo-me'-nhs qa-la's-shs a)pi-ko-me'-nous
 e)pi` th'n-de th`n q a'-las-san kai` oi)-kh'-san-tas tou=-ton to`n
 xw=-ron to`n kai` nu=n oi)-ke'ou- si, au)-ti'-ka nau-ti-li'h|-si
 ma-krh=|-si e)pi-qe'-sqai, a)pa-gi-ne'on-tas de` for-ti'-a
 Ai)-gu'-ptia' te kai` )As-su'-ria th=| te a)'l-lh| [xw'-rh|] e)sa-pi
 k-ne'e-sqai kai` dh` kai` e)s )'Ar-gos; to` de` )'Ar-gos tou=-ton
 to`n xro'-non proei=-xe a('pa-si tw=n e)n th=| nu=n (El-la'-di
 ka-leo-me'-nh| xw'-rh|. )A-pi -ko-me'-nous de` tou`s Foi'-ni-kas e)s
 dh` to` )'Ar-gos tou=-to dia-ti'-qe-sqai to`n fo'r-ton. Pe'm-pth| de`
 h)` e('kth| h(me'-rh| a)p' h(=s a)pi'-kon-to, e)c em-po-lh-me'-nwn
 sfi sxe-do`n pa'n-twn, e)l-qei=n e)pi` th`n qa'-las-san gu-nai =-kas
 a)'l-las te pol-la`s kai` dh` kai` tou= ba-si-le'os qu-ga-te'-ra; to`
 de' oi( ou)'-no-ma ei)=-nai, ka-ta` tw)u-to` to` kai` ('El-lh-nes
 le'-gou-si, )Iou =n th`n )I-na'-xou.

And here is what Filippou's finds in the same passage:  (You should
obtain the same result when compiling the present file, except for the
hyphenation points before the last letter of a word, which the
Ibycus-Babel interface suppresses by default.)

(H-ro-do'-tou Qou-ri'-ou i(-sto-ri'-hs a)-po'-de-cis h('-de, w(s
 mh'-te ta` ge-no'-me-na e)c a)n-qrw'-pwn tw=| xro'-nw| e)c-i'-th-la
 ge'-nh -tai, mh'-te e)'r-ga me-ga'-la te kai` qw-ma-sta', ta` me`n
 ('El-lh-si, ta` de` bar-ba'-roi-si a)-po-de-xqe'n-ta, a)-kle'-a
 ge'-nh-tai, ta' te a)'l-la kai` di ' h(`n ai)-ti'-hn e)-po-le'-mh-san
 a)l-lh'-loi-si. Per-se'-wn me'n nun oi( lo'- gi-oi Foi'-ni-kas
 ai)-ti'-ous fa-si` ge-ne'-sqai th=s di-a-fo-rh=s; tou'-tous g a'r,
 a)-po` th=s )E-ru-qrh=s ka-le-o-me'-nhs qa-la's-shs a)-pi-ko-me'-nous
 e)-p i` th'n-de th`n qa'-las-san kai` oi)-kh'-san-tas tou=-ton to`n
 xw=-ron to`n kai ` nu=n oi)-ke'-ou-si, au)-ti'-ka nau-ti-li'-h|-si
 ma-krh=|-si e)-pi-qe'-sqai, a )-pa-gi-ne'-on-tas de` for-ti'-a
 Ai)-gu'-pti-a' te kai` )As-su'-ri-a th=| te a) 'l-lh| [xw'-rh|]
 e)s-a-pi-kne'-e-sqai kai` dh` kai` e)s )'Ar-gos; to` de` )'Ar- gos
 tou=-ton to`n xro'-non pro-ei=-xe a('-pa-si tw=n e)n th=| nu=n
 (El-la'-di k a-le-o-me'-nh| xw'-rh|. )A-pi-ko-me'-nous de` tou`s
 Foi'-ni-kas e)s dh` to` )'A r-gos tou=-to di-a-ti'-qe-sqai to`n
 fo'r-ton. Pe'm-pth| de` h)` e('-kth| h(-me' -rh| a)p' h(=s
 a)-pi'-kon-to, e)c-em-po-lh-me'-nwn sfi sxe-do`n pa'n-twn, e)l-q ei=n
 e)-pi` th`n qa'-las-san gu-nai=-kas a)'l-las te pol-la`s kai` dh`
 kai` tou = ba-si-le'-os qu-ga-te'-ra; to` de' oi( ou)'-no-ma
 ei)=-nai, ka-ta` tw)-u-to` to` kai` ('El-lh-nes le'-gou-si, )I-ou=n
 th`n )I-na'-xou.

Notice that the hyphenation points right after the first letter of
words beginning with a vowel+consonant+vowel are legal, according to
the rules for hyphenation of Greek, ancient and modern.  See the
account by Yannis Haralambous "From Unicode to Typography, a Case
Study:  the Greek Script",
<http://omega.enstb.org/yannis/pdf/boston99.pdf>, pp~18f.

% finis
