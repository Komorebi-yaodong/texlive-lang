%%%%%%%%%%%%%%%%%%%%%%%%%%%%%%%%%%%%%%%%%%%%%%%%%%%%%%%%%%%%%%%%%
% Contents: Who contributed to this Document
% $Id: overview.tex,v 1.2 2003/03/19 20:57:46 oetiker Exp $
%%%%%%%%%%%%%%%%%%%%%%%%%%%%%%%%%%%%%%%%%%%%%%%%%%%%%%%%%%%%%%%%%

% Because this introduction is the reader's first impression, I have
% edited very heavily to try to clarify and economize the language.
% I hope you do not mind! I always try to ask "is this word needed?"
% in my own writing but I don't want to impose my style on you... 
% but here I think it may be more important than the rest of the book.
% --baron

\chapter{Predgovor}

\LaTeX{} \cite{manual} je program za urejanje besedil, ki je zelo primeren za 
izdelavo znanstvenih in matematičnih besedil z visoko tiskarsko kvaliteto.
Program je uporaben tudi za izdelavo vseh vrst drugih dokumentov, od 
preprostih pisem do celotnih knjig.
\LaTeX{} uporablja \TeX{} \cite{texbook} za stavljenje besedila.

Ta kratka navodila opisujejo \LaTeXe{} in bi morala zadoščati za večino
primerov uporabe \LaTeX{}a. Kompleten opis \LaTeX{}a lahko najdete v~\cite{manual,companion}, podrobnejša 
slovenska navodila pa v \cite{Batagelj}.

\bigskip
\noindent Navodila so razdeljena na 6 poglavij:
\begin{description}
\item[Poglavje 1] opisuje osnovno strukturo \LaTeXe{}
  dokumentov. Tu spoznamo tudi nekaj zgodovine \LaTeX{}a.
  Na koncu poglavja boste imeli grobo predstavo o tem, kako 
  deluje \LaTeX{}.
\item[Poglavje 2] obravnava podrobnosti sestavljanja dokumentov. 
  Razloži večino ključnih ukazov v samem \LaTeX{}u in v okolju, v katerem ga
  poganjamo.
  Po tem poglavju boste lahko napisali svoj prvi tekst v \LaTeX{}u. 
\item[Poglavje 3] razloži, kako se v \LaTeX{} vstavljajo matematične 
  formule. Veliko zgledov pomaga razumeti, 
  kako se uporablja enega izmed \LaTeX{}ovih glavnih 
  adutov. Na koncu poglavja se nahajajo tabele vseh matematičnih 
  simbolov, ki so na voljo v \LaTeX{}u. 
\item[Poglavje 4] obravnava izdelavo indeksov, urejanje seznama literature
  in vključevanje EPS slik. Predstavi tudi generiranje PDF dokumentov preko
  pdf\LaTeX{}a in nekaj drugih priročnih razširitvenih paketov.
\item[Poglavje 5] pokaže, kako lahko uporabimo \LaTeX{} za kreiranje slik.
Namesto tega, da sliko narišemo s kakšnim grafičnim programom, jo shranimo
na datoteko in nato vključimo v \LaTeX{}, jo lahko opišemo kar v \LaTeX{}u,
ki jo nariše za nas.
\item[Poglavje 6] vsebuje nekaj potencialno nevarnih informacij o tem, kako lahko spremenimo
  standardno obliko dokumenta, narejenega z \LaTeX{}om. Tu boste izvedeli, 
  kako lahko stvari nastavimo tako, da se čudovita oblika \LaTeX{}a 
  spremeni v zanikrno ali bleščečo, odvisno od vaših sposobnosti.
\end{description}
\bigskip
\noindent 
Zelo pomembno je, da poglavja preberete v navedenem vrstnem redu.
Knjiga konec koncev sploh ni tako dolga. Poskrbite za to, da pazljivo 
preberete zglede,
saj se pomemben del podatkov skriva v različnih primerih, 
ki so razporejeni po celi knjigi.

\bigskip
\noindent 
\LaTeX{} je na voljo za večino računalnikov, od PC in Mac do velikih UNIX 
in VMS sistemov. V teh navodilih se ne bomo ukvarjali s podrobnostmi
namestitve in poganjanja \LaTeX{}a na različnih 
platformah, temveč s tem, kako je potrebno napisati datoteko, 
ki jo želimo obdelati z \LaTeX{}om.

\bigskip
\noindent Če potrebujete kakršno koli stvar povezano z \LaTeX{}om, 
poglejte na enega izmed obsežnih \TeX{}ovih spletnih arhivov 
(Comprehensive \TeX{} Archive Network)
oz.~krajše (\texttt{CTAN}). Domača stran je 
\texttt{http://www.ctan.org}. Vse pakete lahko dobite tudi z 
ftp arhiva \texttt{ftp://www.ctan.org} oziroma z ene izmed njegovih 
številnih zrcalnih 
kopij po vsem svetu. 

V knjigi boste našli še druge reference na CTAN, še posebno spletne naslove 
programov in dokumentov, ki jih lahko po potrebi prenesete na svoj računalnik. Namesto
celotnih naslovov so le ti skrajšani v \texttt{CTAN:}, ki mu sledi 
ustrezna lokacija v CTAN drevesu.

Če nameravate poganjati \LaTeX{} na vašem lastnem računalniku, si poglejte, kaj je na voljo 
na \texttt{CTAN:/tex-archive/systems}.

\vspace{\stretch{1}}
\noindent Če imate kakršnokoli idejo o tem, kaj bi se dalo v tem dokumentu dodati, odstraniti oziroma
popraviti, sporočite to za angleško različico na naslov 
\begin{verse}
\contrib{Tobias Oetiker}{oetiker@ee.ethz.ch}%
\noindent{Department of Electrical Engineering,\\
Swiss Federal Institute of Technology}
\end{verse}
oziroma na 
\begin{verse}
\contrib{Bor Plestenjak}{bor.plestenjak@fmf.uni-lj.si}%
\noindent{Oddelek za matematiko, Fakulteta za matematiko in fiziko,\\
Jadranska 19, 1000 Ljubljana}
\end{verse}
za slovensko različico. Še posebno so dobrodošle povratne informacije
začetnikov v \LaTeX{}u o tem, kateri deli so dobro razumljivi in kaj bi se
dalo še bolje razložiti.

\vspace{3ex}
\noindent Zadnja različica angleškega dokumenta je na voljo na\\
\CTAN|info/lshort|


\endinput



%

% Local Variables:
% TeX-master: "lshort2e"
% mode: latex
% mode: flyspell
% End:
