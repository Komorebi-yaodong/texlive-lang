%%%%%%%%%%%%%%%%%%%%%%%%%%%%%%%%%%%%%%%%%%%%%%%%%%%%%%%%%%%%%%%%%
%%%%%%%%%%%%%%%%%%%%%%%%%%%%%%%%%%%%%%%%%%%%%%%%%%%%%%%%%%%%%%%%%
\setcounter{chapter}{4}
\newcommand{\graphicscompanion}{\emph{The \LaTeX{} Graphics Companion}~\cite{graphicscompanion}} 
\newcommand{\hobby}{\emph{A User's Manual for MetaPost}~\cite{metapost}}
\newcommand{\hoenig}{\emph{\TeX{} Unbound}~\cite{unbound}}
\newcommand{\graphicsinlatex}{\emph{Graphics in \LaTeXe{}}~\cite{ursoswald}}

\chapter{Priprava matematičnih slik}
\label{chap:graphics}

\begin{intro}
Večina uporabnikov z \LaTeX{}om ureja besedila. Čeprav \LaTeX{}ov pristop
k urejanju besedil poudarja strukturo pred vsebino, \LaTeX{} vseeno 
ponuja nekaj zelo omejenih možnosti za prikaz grafičnih 
vsebin, ki jih opišemo z ukazi v vhodni datoteki. Obstaja veliko
razširitev \LaTeX{}a, ki premostijo te omejitve in omogočajo še
lažjo pripravo slik. Nekaj jih bomo predstavili v tem poglavju.
\end{intro}

\section{Kratki pregled}

V okolju \ei{picture} lahko slike programiramo direktno v \LaTeX{}u. Podroben
opis lahko poiščete v  \manual. Na eni strani smo v tem okolju 
zelo omejeni z nekaterimi pogoji, tako so npr.~nakloni daljic in 
polmeri krogov omejeni na majhen izbor možnih vrednosti.  
Na drugi strani pa imamo v okolju \ei{picture} v \LaTeXe\ na voljo ukaz
\ci{qbezier} za B\'ezierove krivulje, kjer ">\texttt{q}"< pomeni 
"<kvadratna"< (quadratic)"<.  Številne pogosto uporabljene
krivulje kot so krogi, elipse, verižnice in ostale, se da dovolj 
dobro aproksimirati s kvadratnimi B\'ezierovimi krivuljami, res pa je, da
za to včasih potrebujemo malo računanja. Za pripravo
slik iz ukazov \ci{qbezier} lahko uporabimo tudi programski jezik kot je 
npr.~Java in v tem primeu postane okolje \ei{picture} zelo zmogljivo.

Čeprav je zapisovanje slike direktno v \LaTeX{} dokaj omejeno in dostikrat
tudi zelo utrudljivo, še vedno obstajajo razlogi, da se slik 
lotimo na ta način. Tako dobljeni dokumenti zavzamejo manj spomina kot če
vključujemo zunanje slike, pa še nobenih dodatnih datotek s slikami ne
potrebujemo.

Paketi kot so \pai{epic}, \pai{eepic} 
(ta dva sta med drugim opisana v \companion) in 
\pai{pstricks} odpravljajo omejitve originalnega okolja
\ei{picture} in močno povečajo grafične sposobnosti \LaTeX{}a.


Med tem ko prva dva prej omenjena paketa le razširita okolje
\ei{picture}, ima paket \pai{pstricks} svoje okolje \ei{pspicture}. 
Moč paketa \pai{pstricks} izvira iz dejstva, da se v paketu v veliki 
meri uporabljajo možnosti, ki jih ponuja \PSi{}.
Poleg tega obstajajo še številni paketi za posebne potrebe. En izmed
teh je 
\texorpdfstring{\Xy}{Xy}-pic, ki ga bomo predstavili na koncu tega poglavja.
Širok izbor teh paketov je podrobno predstavljen v \graphicscompanion{} 
(ne zamenjujte z \companion).


Najmočnejše grafično orodje, povezano z \LaTeX{}om, je najbrž
\texttt{MetaPost}, dvojček 
Donald E. Knuthovega orodja \texttt{METAFONT}. 
\texttt{MetaPost} vsebuje zelo zmogljiv in matematično prefinjen 
programski jezik tako kot \texttt{METAFONT}. Razlika je, da \texttt{METAFONT}
generira bitne slike, \texttt{MetaPost} pa generira slike v formatu 
Encapsulated \PSi{}, ki jih lahko vključimo v \LaTeX. Za uvod v 
\texttt{MetaPost} poglejte \hobby{} ali navodila na \cite{ursoswald}.

Zelo temeljito razpravo o strategijah za vključevanje slik (in
pisav) v \LaTeX{} in \TeX{} lahko najdete v \hoenig.

\section{Okolje \texttt{picture}}
\secby{Urs Oswald}{osurs@bluewin.ch}

\subsection{Osnovni ukazi}

Okolje \ei{picture}\footnote{Če verjamete ali ne, okolje picture
deluje v povsem navadnem \LaTeXe{} in zanj ni potrebno 
naložiti nobenega paketa.} vključimo z enim izmed
naslednjih dveh ukazom
\begin{lscommand}
\ci{begin}\verb|{picture}(|$x,y$\verb|)|\ldots\ci{end}\verb|{picture}|
\end{lscommand}
\noindent ali
\begin{lscommand}
\ci{begin}\verb|{picture}(|$x,y$\verb|)(|$x_0,y_0$\verb|)|\ldots\ci{end}\verb|{picture}|
\end{lscommand}
Števila $x,\,y,\,x_0,\,y_0$ se nanašajo na količino \ci{unitlength}, ki jo lahko
spremenimo kadarkoli
(a ne znotraj okolja \ei{picture}) z ukazom v stilu
\begin{lscommand}
\ci{setlength}\verb|{|\ci{unitlength}\verb|}{1.2cm}|
\end{lscommand}
Privzeta vrednost za \ci{unitlength} je \texttt{1pt}. Prvi par, $(x,y)$, pove, koliko pravokotnega prostora znotraj dokumenta je potrebno rezervirati za sliko,
neobvezni drugi par, $(x_0,y_0)$, pa nastavi te koordinate v spodnji levi kot rezerviranega pravokotnika. 

Večina ukazov za risanje ima eno izmed naslednjih dve oblik
\begin{lscommand}
\ci{put}\verb|(|$x,y$\verb|){|\emph{object}\verb|}|
\end{lscommand}
\noindent ali
\begin{lscommand}
\ci{multiput}\verb|(|$x,y$\verb|)(|$\Delta x,\Delta y$\verb|){|$n$\verb|}{|\emph{object}\verb|}|\end{lscommand}
Izjema so B\'ezierove krivulje, ki jih narišemo z ukazom
\begin{lscommand}
\ci{qbezier}\verb|(|$x_1,y_1$\verb|)(|$x_2,y_2$\verb|)(|$x_3,y_3$\verb|)|
\end{lscommand}
%\newpage

\subsection{Daljice}
\begin{example}
\setlength{\unitlength}{5cm}
\begin{picture}(1,1)
  \put(0,0){\line(0,1){1}}
  \put(0,0){\line(1,0){1}}  
  \put(0,0){\line(1,1){1}}  
  \put(0,0){\line(1,2){.5}}
  \put(0,0){\line(1,3){.3333}}
  \put(0,0){\line(1,4){.25}}  
  \put(0,0){\line(1,5){.2}}
  \put(0,0){\line(1,6){.1667}}
  \put(0,0){\line(2,1){1}}
  \put(0,0){\line(2,3){.6667}}
  \put(0,0){\line(2,5){.4}}
  \put(0,0){\line(3,1){1}}  
  \put(0,0){\line(3,2){1}}
  \put(0,0){\line(3,4){.75}}
  \put(0,0){\line(3,5){.6}}
  \put(0,0){\line(4,1){1}}
  \put(0,0){\line(4,3){1}}  
  \put(0,0){\line(4,5){.8}}
  \put(0,0){\line(5,1){1}}
  \put(0,0){\line(5,2){1}}
  \put(0,0){\line(5,3){1}}
  \put(0,0){\line(5,4){1}}
  \put(0,0){\line(5,6){.8333}}
  \put(0,0){\line(6,1){1}}
  \put(0,0){\line(6,5){1}}
\end{picture}
\end{example}
Daljico narišemo z ukazom
\begin{lscommand}
\ci{put}\verb|(|$x,y$\verb|){|\ci{line}\verb|(|$x_1,y_1$\verb|){|$length$\verb|}}|
\end{lscommand}
Ukaz \ci{line} ima dva argumenta:
\begin{enumerate}
  \item smerni vektor,
  \item dolžino.
\end{enumerate}
Vrednosti komponent v smernem vektorju so omejene 
na cela števila
\[
  -6,\,-5,\,\ldots,\,5,\,6,
\]
komponenti pa morata biti tuji števili (razen 1 nimata nobenega skupnega delitelja). Na sliki je predstavljenih vseh 25 možnih naklonov v prvem kvadrantu. Dolžina je podana relativno glede na \ci{unitlength}.
V primeru navpične daljice kot dolžino podamo navpično koordinato, v vseh drugih pa vodoravno koordinato konca daljice.

\subsection{Puščice}

\begin{example}
\setlength{\unitlength}{0.75mm}
\begin{picture}(60,40)
  \put(30,20){\vector(1,0){30}}
  \put(30,20){\vector(4,1){20}}
  \put(30,20){\vector(3,1){25}}
  \put(30,20){\vector(2,1){30}}
  \put(30,20){\vector(1,2){10}}
  \thicklines
  \put(30,20){\vector(-4,1){30}}
  \put(30,20){\vector(-1,4){5}}
  \thinlines
  \put(30,20){\vector(-1,-1){5}}
  \put(30,20){\vector(-1,-4){5}}
\end{picture}
\end{example}
Puščico narišemo z ukazom
\begin{lscommand}
\ci{put}\verb|(|$x,y$\verb|){|\ci{vector}\verb|(|$x_1,y_1$\verb|){|$length$\verb|}}|
\end{lscommand}
Pri puščicah so komponente smernega vektorja
še bolj omejene kot pri daljicah, saj jih lahko izbiramo le
med celimi števili
\[
  -4,\,-3,\,\ldots,\,3,\,4.
\]
Tako kot pri daljicah morata komponenti biti tuji števili.
Bodite pozorni na vpliv ukaza 
\ci{thicklines} na dve puščici, ki kažeta v zgornji levi kot.

\subsection{Krožnice}

\begin{example}
\setlength{\unitlength}{1mm}
\begin{picture}(60, 40)
  \put(20,30){\circle{1}}
  \put(20,30){\circle{2}}
  \put(20,30){\circle{4}}
  \put(20,30){\circle{8}}
  \put(20,30){\circle{16}}
  \put(20,30){\circle{32}}
  
  \put(40,30){\circle{1}}
  \put(40,30){\circle{2}}
  \put(40,30){\circle{3}}
  \put(40,30){\circle{4}}
  \put(40,30){\circle{5}}
  \put(40,30){\circle{6}}
  \put(40,30){\circle{7}}
  \put(40,30){\circle{8}}
  \put(40,30){\circle{9}}
  \put(40,30){\circle{10}}
  \put(40,30){\circle{11}}
  \put(40,30){\circle{12}}
  \put(40,30){\circle{13}}
  \put(40,30){\circle{14}}
  
  \put(15,10){\circle*{1}}
  \put(20,10){\circle*{2}}
  \put(25,10){\circle*{3}}
  \put(30,10){\circle*{4}}
  \put(35,10){\circle*{5}}
\end{picture}
\end{example}
Ukaz
\begin{lscommand}
  \ci{put}\verb|(|$x,y$\verb|){|\ci{circle}\verb|{|\emph{diameter}\verb|}}|
\end{lscommand}
\noindent nariše krožnico s središčem v točki $(x,y)$ in premerom \emph{diameter}. V okolju \ei{picture} lahko uporabljamo le premere do
približno 14\,mm, s tem da pod to vrednostjo niso dopustni vse premeri.
Ukaz \ci{circle*} nariše zapolnjeni krog.

Tako kot pri daljicah se moramo, če želimo premer, ki ni podprt, obrniti
na dodatne pakete, kot sta \pai{eepic} in \pai{pstricks}. 
Temeljit opis teh dveh paketov lahko najdete v \graphicscompanion.

Obstaja pa tudi možnost znotraj okolja 
\ei{picture}. Če lahko opravite potrebne izračune (oziroma
jih prepustite ustreznemu programu), potem lahko poljubne daljice
in krožnice sestavite iz kvadratnih B\'ezierovih krivulj. 
Poglejte \graphicsinlatex\ za primere in kode pomožnih programov v Javi.

\subsection{Besedila in formule}

\begin{example}
\setlength{\unitlength}{0.8cm}
\begin{picture}(6,5)
  \thicklines
  \put(1,0.5){\line(2,1){3}}
  \put(4,2){\line(-2,1){2}}
  \put(2,3){\line(-2,-5){1}}
  \put(0.7,0.3){$A$}
  \put(4.05,1.9){$B$}
  \put(1.7,2.95){$C$}
  \put(3.1,2.5){$a$}
  \put(1.3,1.7){$b$}
  \put(2.5,1.05){$c$}
  \put(0.3,4){$F=
    \sqrt{s(s-a)(s-b)(s-c)}$}  
  \put(3.5,0.4){$\displaystyle
    s:=\frac{a+b+c}{2}$}
\end{picture}
\end{example}
Kot prikazuje ta zgled, lahko besedilo in formule vstavimo
v okolje \ei{picture} s ukazom \ci{put} na povsem 
običajni način.

\subsection{\ci{multiput} in \ci{linethickness}}

\begin{example}
\setlength{\unitlength}{2mm}
\begin{picture}(30,20)
  \linethickness{0.075mm}
  \multiput(0,0)(1,0){26}%
    {\line(0,1){20}}
  \multiput(0,0)(0,1){21}%
    {\line(1,0){25}}
  \linethickness{0.15mm}    
  \multiput(0,0)(5,0){6}%
    {\line(0,1){20}}
  \multiput(0,0)(0,5){5}%
    {\line(1,0){25}}
  \linethickness{0.3mm}    
  \multiput(5,0)(10,0){2}%
    {\line(0,1){20}}
  \multiput(0,5)(0,10){2}%
    {\line(1,0){25}}
\end{picture}
\end{example}
Ukaz
\begin{lscommand}
  \ci{multiput}\verb|(|$x,y$\verb|)(|$\Delta x,\Delta y$\verb|){|$n$\verb|}{|\emph{object}\verb|}|
\end{lscommand}
\noindent ima 4 argumente: začetno točko, vektor premika od enega
objekta do naslednjega, število objektov in objekt, ki naj se nariše.
Ukaz \ci{linethickness} vpliva na vodoravne in navpične daljice,
ne pa na poševne daljice in krožnice. Vpliva pa, presenetljivo, 
na kvadratne B\'ezierove krivulje!

\subsection{Ovali}

\begin{example}
\setlength{\unitlength}{0.75cm}
\begin{picture}(6,4)
  \linethickness{0.075mm}
  \multiput(0,0)(1,0){7}%
    {\line(0,1){4}}
  \multiput(0,0)(0,1){5}%
    {\line(1,0){6}}
  \thicklines
  \put(2,3){\oval(3,1.8)} 
  \thinlines
  \put(3,2){\oval(3,1.8)} 
  \thicklines
  \put(2,1){\oval(3,1.8)[tl]} 
  \put(4,1){\oval(3,1.8)[b]} 
  \put(4,3){\oval(3,1.8)[r]} 
  \put(3,1.5){\oval(1.8,0.4)}     
\end{picture}
\end{example}
Ukaz
\begin{lscommand}
  \ci{put}\verb|(|$x,y$\verb|){|\ci{oval}\verb|(|$w,h$\verb|)}|
\end{lscommand}
\noindent ali v obliki
\begin{lscommand} \ci{put}\verb|(|$x,y$\verb|){|\ci{oval}\verb|(|$w,h$\verb|)[|\emph{polozaj}\verb|]}|
\end{lscommand}
\noindent vrne oval s središčem v $(x,y)$, s širino $w$ in z višino $h$. 
Neobvezni argument
\emph{polozaj} ima lahko vrednosti \texttt{b}, \texttt{t}, \texttt{l}, \texttt{r} ki po vrsti pomenijo zgoraj, spodaj, levo, desno, lahko pa jih
tudi kombiniramo, kot je prikazano v zgledu.

\subsection{Debelina črt}

Na debelino črt lahko vplivamo z dvema razredoma ukazov. Na eni strani je 
\ci{linethickness}\verb|{|\emph{length}\verb|}|
na drugi strani pa sta ukaza \ci{thinlines} in \ci{thicklines}.
Medtem kot \ci{linethickness}\verb|{|\emph{length}\verb|}|
vpliva le na vodoravne in navpične črte (in na kvadratne B\'ezierove krivulje),
ukaza \ci{thinlines} in \ci{thicklines}
vplivata na poševne črte, krožnice in ovale. 


\subsection{Večkratna uporaba vnaprej pripravljenih škatel s slikami}

\begin{example}
\setlength{\unitlength}{0.5mm}
\begin{picture}(120,168)
\newsavebox{\foldera}
\savebox{\foldera}
  (40,32)[bl]{% definicija 
  \multiput(0,0)(0,28){2}
    {\line(1,0){40}}
  \multiput(0,0)(40,0){2}
    {\line(0,1){28}}
  \put(1,28){\oval(2,2)[tl]}
  \put(1,29){\line(1,0){5}}
  \put(9,29){\oval(6,6)[tl]}
  \put(9,32){\line(1,0){8}}
  \put(17,29){\oval(6,6)[tr]}
  \put(20,29){\line(1,0){19}}
  \put(39,28){\oval(2,2)[tr]}  
}
\newsavebox{\folderb}
\savebox{\folderb}
  (40,32)[l]{%         definicija
  \put(0,14){\line(1,0){8}}
  \put(8,0){\usebox{\foldera}}
}
\put(34,26){\line(0,1){102}} 
\put(14,128){\usebox{\foldera}}
\multiput(34,86)(0,-37){3}
  {\usebox{\folderb}} 
\end{picture}
\end{example}
Škatlo s sliko lahko \emph{napovemo} z ukazom
\begin{lscommand}
  \ci{newsavebox}\verb|{|\emph{ime}\verb|}|
\end{lscommand}
\noindent potem jo \emph{definiramo} z   
\begin{lscommand}
  \ci{savebox}\verb|{|\emph{ime}\verb|}(|\emph{width,height}\verb|)[|\emph{position}\verb|]{|\emph{content}\verb|}|
\end{lscommand}
\noindent na koncu pa jo lahko \emph{narišemo} kolikor krat želimo z
\begin{lscommand}
  \ci{put}\verb|(|$x,y$\verb|)|\ci{usebox}\verb|{|\emph{ime}\verb|}|
\end{lscommand}

Dodatno neobvezni parameter \emph{pozicija} definira mesto, na katerem
je referenčna točka shranjene škatle s sliko, oziroma mesto, kjer je 
škatla ">zasidrana"<. V zgornjem primeru je to mesto nastavljeno na \texttt{bl} kar
postavlja mesto priveza v gornji levi kot škatle. Ostale možno
vrednosti za ta parameter so še \texttt{t} (zgoraj) in \texttt{r} (desno).

Argument \emph{ime} se nanaša na \LaTeX{}ov pomnilnik in ima naravo
ukaza (zato v zgornjem primeru imena vsebujejo tudi poševnice). 
Škatle s slikami lahko vlagamo v druge škatle, v zgornjem primeru je
tako npr.~\ci{foldera} uporabljen v definiciji \ci{folderb}.

V definiciji smo morali uporabiti ukaz \ci{oval}, saj ukaz \ci{line} 
ne podpira daljic, kjer je dolžina segmenta manjša od 3\,mm.

\subsection{Kvadratne B\'ezierove krivulje}

\begin{example}
\setlength{\unitlength}{0.8cm}
\begin{picture}(6,4)
  \linethickness{0.075mm}
  \multiput(0,0)(1,0){7}
    {\line(0,1){4}}
  \multiput(0,0)(0,1){5}
    {\line(1,0){6}}
  \thicklines
  \put(0.5,0.5){\line(1,5){0.5}}    
  \put(1,3){\line(4,1){2}} 
  \qbezier(0.5,0.5)(1,3)(3,3.5)
  \thinlines   
  \put(2.5,2){\line(2,-1){3}}
  \put(5.5,0.5){\line(-1,5){0.5}}
  \linethickness{1mm}
  \qbezier(2.5,2)(5.5,0.5)(5,3)
  \thinlines
  \qbezier(4,2)(4,3)(3,3)
  \qbezier(3,3)(2,3)(2,2)
  \qbezier(2,2)(2,1)(3,1)
  \qbezier(3,1)(4,1)(4,2)
\end{picture}
\end{example}
Kot prikazuje ta primer, ni dovolj dobro, če krožnico razdelimo le
na 4 kvadratne B\'ezierove krivulje. Potrebujemo jih najmanj 8.
Na sliki vidimo tudi, kakšne je vpliv ukazov
\ci{linethickness} na vodoravne in navpične premice in kako
\ci{thinlines} in \ci{thicklines} vplivata na poševne daljice. Vidimo
lahko tudi, da oba razreda ukazov delujeta na kvadratne B\'ezierove krivulje,
kjer vsak nov ukaz povozi prejšnjega.

Naj $P_1=(x_1,\,y_1),\,P_2=(x_2,\,y_2)$ označujeta končni točki, in naj
bosta $m_1$ in $m_2$ naklona kvadratne B\'ezierove krivulje v teh točkah. 
Vmesna kontrolna točka
$S=(x,\,y)$ je potem podana z enačbami
\begin{equation} \label{zwischenpunkt}
  \left\{
    \begin{array}{rcl}
      x & = & \displaystyle \frac{m_2 x_2-m_1x_1-(y_2-y_1)}{m_2-m_1}, \\
      y & = & y_i+m_i(x-x_i)\qquad (i=1,\,2).
    \end{array}
  \right.
\end{equation}
\noindent Poglejte \graphicsinlatex\ za program v Javi, ki generira vrstice
s potrebnimi ukazi \ci{qbezier}.

\subsection{Verižnica}

\begin{example}
\setlength{\unitlength}{1cm}
\begin{picture}(4.3,3.6)(-2.5,-0.25)
\put(-2,0){\vector(1,0){4.4}}
\put(2.45,-.05){$x$}
\put(0,0){\vector(0,1){3.2}}
\put(0,3.35){\makebox(0,0){$y$}}
\qbezier(0.0,0.0)(1.2384,0.0)
  (2.0,2.7622) 
\qbezier(0.0,0.0)(-1.2384,0.0)
  (-2.0,2.7622)
\linethickness{.075mm}
\multiput(-2,0)(1,0){5}
  {\line(0,1){3}}
\multiput(-2,0)(0,1){4}
  {\line(1,0){4}}
\linethickness{.2mm}
\put( .3,.12763){\line(1,0){.4}}
\put(.5,-.07237){\line(0,1){.4}}
\put(-.7,.12763){\line(1,0){.4}}
\put(-.5,-.07237){\line(0,1){.4}}
\put(.8,.54308){\line(1,0){.4}}
\put(1,.34308){\line(0,1){.4}}
\put(-1.2,.54308){\line(1,0){.4}}
\put(-1,.34308){\line(0,1){.4}}
\put(1.3,1.35241){\line(1,0){.4}}
\put(1.5,1.15241){\line(0,1){.4}}
\put(-1.7,1.35241){\line(1,0){.4}}
\put(-1.5,1.15241){\line(0,1){.4}}
\put(-2.5,-0.25){\circle*{0.2}}
\end{picture}
\end{example}

Na tej sliki je vsaka simetrična polovica grafa verižnice $y=\cosh x -1$ 
aproksimirana s kvadratno B\'ezierovo krivuljo. 
Desna polovica krivulje se konča v točki \((2,\,2.7622)\), kjer ima
odvod vrednost \(m=3.6269\). S pomočjo enačb (\ref{zwischenpunkt}) lahko
izračunamo vmesne kontrolne točke. Za njiju se izkaže, da sta 
 $(1.2384,\,0)$ in $(-1.2384,\,0)$. 
Križi označujejo točke \emph{prave} verižnice. Vidimo, da razlika
skoraj ni opazna in je povsod pod enim procentom.

Ta primer kaže, kako lahko uporabljamo neobvezni argument
v ukazu \\
\verb|\begin{picture}|.
Slika je definirana v smiselnih matematičnih koordinatah, 
medtem ko ukaz
\begin{lscommand} 
  \ci{begin}\verb|{picture}(4.3,3.6)(-2.5,-0.25)|
\end{lscommand}
\noindent za njen spodnji levi kot (označen z črnim pobarvanim diskom)
privzame koordinate $(-2.5,-0.25)$. 

\subsection{Hitrost v posebni teoriji relativnosti}

\begin{example}
\setlength{\unitlength}{0.8cm}
\begin{picture}(6,4)(-3,-2)
  \put(-2.5,0){\vector(1,0){5}}
  \put(2.7,-0.1){$\chi$}
  \put(0,-1.5){\vector(0,1){3}}
  \multiput(-2.5,1)(0.4,0){13}
    {\line(1,0){0.2}}
  \multiput(-2.5,-1)(0.4,0){13}
    {\line(1,0){0.2}}
  \put(0.2,1.4)
    {$\beta=v/c=\tanh\chi$}
  \qbezier(0,0)(0.8853,0.8853)
    (2,0.9640)
  \qbezier(0,0)(-0.8853,-0.8853)
    (-2,-0.9640)
  \put(-3,-2){\circle*{0.2}}
\end{picture}
\end{example}
Kontrolne točke dveh B\'ezierovih krivulj so bile izračunane s pomočjo
formul (\ref{zwischenpunkt}).
Pozitivna veja je določena s $P_1=(0,\,0),\,m_1=1$ in $P_2=(2,\,\tanh 2),\,m_2=1/\cosh^2 2$.
Slika je definirana z matematično najustreznejšimi koordinatami, potem pa 
levemu spodnjemu kotu določimo koordinate $(-3,-2)$ (črn krog).


\section{\texorpdfstring{\Xy}{Xy}-pic}
\secby{Alberto Manuel Brand\~ao Sim\~oes}{albie@alfarrabio.di.uminho.pt}
\pai{xy} je poseben paket za risanje diagramov. Za njegovo
uporabo moramo v preambulo dokumenta dodati:
\begin{lscommand}
\verb|\usepackage[|\emph{opcije}\verb|]{xy}|
\end{lscommand}
Pri tem v \emph{opcije} navedemo seznam funkcij iz 
 \Xy-pic ki jih želimo naložiti. Te opcije pridejo prav kadar je potrebno
iskati napake v paketu, sicer pa je najbolj priporočljivo 
na mestu \emph{opcije} podati \verb!all!, s čimer povzročimo,
da \LaTeX{} naloži vse ukaze iz paketa \Xy-pic.

\Xy-pic diagrami se rišejo na podlago, ki je urejena v 
stilu matrike, pri čemer se vsak element
diagrama zapiše na mesto elementa matrike:
\begin{example}
\begin{displaymath}
\xymatrix{A & B \\
          C & D }
\end{displaymath}
\end{example}
Ukaz \ci{xymatrix} moramo uporabiti v matematičnem načinu. V tem primeru
smo podali dve vrstici in dva stolpca. Iz te matrike dobimo diagram, ko
dodamo usmerjene povezave z ukazom \ci{ar}.
\begin{example}
\begin{displaymath}
\xymatrix{ A \ar[r] & B \ar[d] \\
           D \ar[u] & C \ar[l] }
\end{displaymath}
\end{example}
Ukaz za puščico je potrebno postaviti na mesto tistega elementa
v matriki, od koder puščica izvira. Argumenti ukaza \texttt{ar} so
smeri, kamor naj kaže puščica (\texttt{u} (gor),
\texttt{d} (dol), \texttt{r} (desno) in \texttt{l} (levo).

\begin{example}
\begin{displaymath}
\xymatrix{
  A \ar[d] \ar[dr] \ar[r] & B \\
  D                       & C }
\end{displaymath}
\end{example}
Diagonalne puščice dobimo tako, da uporabimo več kot eno črko za opis smeri. 
Opis smeri lahko tudi sestavimo iz več črk in tako dobimo daljše diagonale.
\begin{example}
\begin{displaymath}
\xymatrix{
  A \ar[d] \ar[dr] \ar[drr] & & \\
  B                      & C & D }
\end{displaymath}
\end{example}

Še bolj zanimive diagrame dobimo, če puščice opremimo z napisi. To dosežemo
z uporabo običajnih operatorjev za indekse in potence..
\begin{example}
\begin{displaymath}
\xymatrix{
  A \ar[r]^f \ar[d]_g &
             B \ar[d]^{g'} \\
  D \ar[r]_{f'}       & C }
\end{displaymath}
\end{example}

Kot vidite, operatorje za napise na puščicah uporabljamo kot v matematičnem načinu.Edina razlika je, da sedaj potenca (zgornji indeks) pomeni 
">na vrhu puščice,"< indeks pa pomeni ">pod puščico."< Obstaja še tretji
operator, pokončna črta: \verb+|+
Ta operator razporedi tekst \emph{na} puščico.
\begin{example}
\begin{displaymath}
\xymatrix{
  A \ar[r]|f \ar[d]|g &
             B \ar[d]|{g'} \\
  D \ar[r]|{f'}       & C }
\end{displaymath}
\end{example}

Če potrebujemo puščico z luknjo, uporabimo \verb!\ar[...]|\hole!.

V nekaterih primerih je potrebno ločiti 
med različnimi tipi puščic. To lahko naredimo tako,
 da jih opremimo z napisi ali pa spremenimo njihov videz:

\begin{example}
\shorthandoff{"}
\begin{displaymath}
\xymatrix{
\bullet\ar@{->}[rr] && \bullet\\
\bullet\ar@{.<}[rr] && \bullet\\
\bullet\ar@{~)}[rr] && \bullet\\
\bullet\ar@{=(}[rr] && \bullet\\
\bullet\ar@{~/}[rr] && \bullet\\
\bullet\ar@{^{(}->}[rr] &&
                       \bullet\\
\bullet\ar@2{->}[rr] && \bullet\\
\bullet\ar@3{->}[rr] && \bullet\\
\bullet\ar@{=+}[rr]  && \bullet
}
\end{displaymath}
\shorthandon{"}
\end{example}

Bodite pozorni na razlike med naslednjima dvema diagramoma:

\begin{example}
\begin{displaymath}
\xymatrix{
 \bullet \ar[r] 
         \ar@{.>}[r] & 
 \bullet
}
\end{displaymath}
\end{example}

\begin{example}
\begin{displaymath}
\xymatrix{
 \bullet \ar@/^/[r] 
         \ar@/_/@{.>}[r] &
 \bullet
}
\end{displaymath}
\end{example}

Znamenje med poševnicama pove, kako 
naj se nariše krivulja.
\Xy-pic ponuja mnogo možnosti, ki vplivajo
na risanje krivulj. Za več informacij poglejte v dokumentacijo, ki je
priložena paketu.


% \begin{example}
% \begin{lscommand}
% \ci{dum}
% \end{lscommand}
% \end{example}

