% Time-stamp: <2019-09-24 10:45:10 administrateur>
% Création : 2019-08-24T14:40:52+0200

Il s'agit maintenant de fournir quelques commandes de document permettant de
créer un exemple du crible d'Eratosthène comme ceci:
\begin{center}\label{lecrible100}
  \eratosthene{100}
  \EcrireCribleEratosthene{100}
\end{center}

Ce sera l'occasion d'évoquer un des \og objets de haut niveau\fg
fourni par \Expliii. Non, je ne parle pas de programmation objet!

Une fois encore, je place le code dans un ``bête''
fichier~\texttt{.tex} d'où les bien connus désormais
\Macro{ExplSyntaxOn} et \Macro{ExplSyntaxOff} en début et fin de
fichier.

\section{Buts et algorithmes}
\label{sec:butsalgo}

L'objet dont il est question est appelé \English{intarray}. Il est
global et, une fois créé, ne peut être détruit\footnote{C'est du moins
  ce que je comprends à la lecture \TO que j'ai voulu rapide\TF du
  source. En tout cas \Expliii ne fournit pas d'outil pour ce
  faire.}. Dans certains langages on parle de \emph{vecteurs} mais il
s'agit ici d'un vecteur d'\glspl{entier} (relatifs) et d'entiers
seulement.  Sa taille doit être donnée à la création et ne peut pas
être changée.

Avec la commande \Macro{eratosthene}, on crée deux \emph{vecteurs
  d'entiers}. Le premier, \Macro{__ERA_dv_ia}, contient en \(n\)\ieme
place le plus petit diviseur premier de~\(n\). Le second,
\Macro{__ERA_qt_ia}, contient en \(n\)\ieme place le quotient de~\(n\)
et de ce diviseur premier dont je viens de parler.

La commande  \Macro{eratosthene} ne prend qu'un argument, un entier
naturel, qui est la taille des deux vecteurs sus-définis.

Une fois les vecteurs créés, on pourra les utiliser pour écrire le
crible d'Eratosthène comme vu précédemment ou, pour donner un autre
exemple, obtenir la décomposition d'un entier comme produit de ses
diviseurs premiers.

\section{Création des vecteurs}
\label{sec:creationvecteurs}

\Fragment[commande \texttt{eratosthene},
initialisation]{5}{10}{./codes/erato-def}

En \Ligne{6} j'ouvre un groupe avec \Macro{group_begin+} qui sera
fermé juste avant l'accolade fermante marquant la fin de la définition
de \Macro{eratosthene} en \Ligne{45}.

Dans ce groupe je vais créer un \emph{entier} avec
\Macro{int_new+N}. Il est créé globalement mais je le détruis en
\Ligne{44} avec \Macro{cs_undefine+N}. Il ne survivra donc pas à la
macro.

Je crée également les deux vecteurs en lignes~\(9\) et~\(10\) à l'aide
de \Macro{intarray_new+Nn} qui prend en 2\ieme argument la taille du
vecteur qui est l'argument passé à \Macro{eratosthene}. Initialement
ces vecteurs ne contiennent que des zéros.

Les noms des vecteurs devraient se terminer par \texttt{_intarray}
mais je me suis permis un raccourci en \texttt{_ia}. Leurs noms
commencent avec le double souligné puisque ce sont des
\glspl{variable} \emph{privées}. \texttt{ERA} est le sigle de ce
pseudo-module.

En \Ligne{8} je donne à la \gls{variable}
\Macro{l_ERA_diviseur_max_int} la partie entière de la racine carrée
de la taille des vecteurs \TO valeur donnée par~|#1|\TF en
\emph{transtypant} le résultat de l'opération |floor(| |sqrt (| |#1|
|))| qui est un \gls{flottant} en un \gls{entier}. 

\subsection{Une commande auxiliaire}

Je crée maintenant, à l'intérieur de la définition de
\Macro{eratosthene} la commande \Macro{ERA_marquer_de_a_par+nnn} que,
de fait, je n'utiliserai qu'une fois, en \Ligne{33}, et dont, par
conséquent, j'aurais pu faire l'économie, au risque d'embrouiller
considérablement le code suivant. 

\Fragment[commande \texttt{eratosthene},
auxiliaire]{12}{20}{./codes/erato-def}

Je crée cette commande localement grâce à \Macro{cs_set+Nn}, elle ne
survivra donc pas au groupe. Je commence, dans la définition de
\Macro{ERA_marquer_de_a_par+nnn}, par donner comme valeur à la
\gls{variable} \emph{locale} \Macro{l_tmpa_int} le quotient entier des
1\ier et 3\ieme arguments de la commande.

Puis, lignes~\(15\) à~\(19\), j'utilise \Macro{int_step_inline+nnnn}
pour accomplir le travail. C'est une boucle \POUR. L'index \TO
implicite et anonyme\TF parcourt les entiers en partant du premier
argument de cette dernière macro jusqu'au troisième argument avec un
pas donné par le deuxième. Le 4\ieme argument est le code \emph{en
  ligne} dans lequel |#1| est l'indice de la boucle. Enfin, |#1|
serait l'indice de la boucle si on utilisait directement la macro
\Macro{int_step_inline+nnnn} au niveau du document. Ici comme je
l'utilise dans une définition incluse elle-même dans une définition,
je dois doubler les dièses deux fois d'où le |####1| que l'on voit
dans le code.

En \Ligne{19}, j'incrémente \Macro{l_tmpa_int}. De la \Ligne{16} à la
\Ligne{18}, j'ai un test \SIALORS avec \Macro{int_compare+nNnT}:
les actions des lignes~\(17\) et~\(18\) ne seront accomplies que s'il
est vrai que la |####1|-ième composante du vecteur \Macro{__ERA_dv_ia}
est nulle \CAD que ce nombre n'a pas encore été traité. Dans ce cas,
je place la valeur du pas~|##3| \TO c'est le plus petit diviseur
premier du nombre considéré\TF à la bonne place dans le vecteur des
diviseurs et je place le quotient à la place idoine dans le vecteur
des quotients. L'incrémentation faite en-dessous me dispense de
calculer ce quotient à l'aide d'une division entière.

Comme le nom l'indique la macro \Macro{intarray_gset+Nnn} réalise une
affectation \textbf{g}lobale dans le vecteur donné en 1\ier argument. C'est
bien ce que je veux. Le 2\ieme argument donne l'indice, le 3\ieme la
valeur à y placer.

\subsection{Traitement des entiers pairs}
\label{sec:lespairs}

Je retourne maintenant au niveau de la définition de la commande
\Macro{eratosthene} et je commence par traiter les nombres pairs.

\Fragment[commande \texttt{eratosthene},
nombres pairs]{23}{28}{./codes/erato-def}

À ce moment de l'action, les deux  vecteurs ne contiennent que des
zéros, cela nous dispense du test tel qu'il apparaît dans la macro
auxiliaire vue ci-dessus. De plus, le premier quotient (\(2/2\)) est
clairement~\(1\) d'où la \Ligne{23}.

On retrouve ensuite \Macro{int_step_inline+nnnn} pour traiter les
nombres pairs en partant de~\(2\), avec un pas de~\(2\), et en
finissant à la taille des vecteurs~|#1|.

\subsection{Traitement des entiers impairs}
\label{sec:lesimpairs}

Le traitement des entiers impairs se fera en deux étapes, la deuxième
étant d'une simplicité déconcertante.

\Fragment[commande \texttt{eratosthene},
impairs]{31}{36}{./codes/erato-def}

Nous avons déjà vu tout ce que j'utilise ici. On notera, lignes~\(35\)
et~\(36\), la boucle permettant, en avançant de~\(2\) en~\(2\), de
trouver le prochain nombre impair non traité avec
\Macro{int_do_until+nNnn} analogue, pour réaliser une boucle sur les
entiers, de \Macro{bool_do_until+Nn} présentée page~\pageref{booldountilNn}. 

La dernière boucle traite les nombres impairs qui n'ont pas encore été
vus. On pourrait améliorer la chose en commençant avec le premier
impair non-vu, une fois encore je laisse ça en exercice!  

\Fragment[commande \texttt{eratosthene}]{39}{42}{./codes/erato-def}

\section{Deux exemples d'utilisation}
\label{sec:utilisationcrible}

Les deux vecteurs ayant été créés, on peut les utiliser. Je donne deux
exemples: l'écriture du \og crible\fg, l'écriture d'un nombre comme
produit de ses facteurs premiers \TO version améliorable\TF.

\subsection{Écriture du crible}
\label{sec:ecriturecrible}

Je présente maintenant la commande permettant de créer le tableau
présenté en page~\pageref{lecrible100}.

\Fragment[écriture du crible, commande de
document]{69}{70}{./codes/erato-def}

L'unique argument de \Macro{EcrireCribleEratosthene} doit être un
entier positif plus petit que la taille des deux vecteurs créés
auparavant sinon on aura une erreur d'accès aux vecteurs dû à un
indice hors bornes.

Bien entendu, cette commande se contente de refiler le boulot à la
commande interne \Macro{ERA_presenter_nieme+n} par l'intermédiaire de
la boucle \Macro{int_step_inline+nn}.

Soulevons le capot.

\Fragment[écriture du crible, commande interne]{50}{67}{./codes/erato-def}

Pour la création du tableau, je copie servilement le code que fourni
Christian \textsc{Tellechea} dans~\autocite{tellechea}. J'utilise
quand même la macro \Macro{framebox} fournie par \LaTeXe{} car
personne n'est obligé de réinventer la roue chaque matin!

\subsection{Décomposition d'un nombre en facteurs premiers}
\label{sec:facteurspremiers}

Regardons à présent la commande permettant d'obtenir:
\EcrireDiviseurs{78} \qquad \EcrireDiviseurs*{75} \qquad
\EcrireDiviseurs*{64}
\par\noindent
avec le code
\par\noindent
|\EcrireDiviseurs{78}| |\qquad| |\EcrireDiviseurs*{75}| |\qquad| |\EcrireDiviseurs*{64}|.

Voici le code: 
\Fragment[facteurs premiers]{73}{96}{./codes/erato-def}

Comme on l'aura remarqué, la commande \Macro{EcrireDiviseurs} admet
une variante étoilée. On l'obtient facilement avec
\Macro{NewDocumentCommand} en donnant comme descriptif de 1\ier
argument le~|s| que l'on voit en \Ligne{73}.

Avec \Macro{IfBooleanF} de la \Ligne{82}, on ne place un saut de
paragraphe \TO avec \Macro{par}\TF que s'il n'y a \textsb{pas}
d'étoile.

Pour réagir à la présence ou l'absence d'un lexème \TO une étoile~|*|
avec le descripteur d'argument~|s|, n'importe quel lexème avec~|t| \TF
on dispose également de \Macro{IfBooleanT} et de \Macro{IfBooleanTF}.

On retrouve les mêmes techniques de création d'\glspl{entier} dans un
groupe et leur destruction en sortie du corps de la définition de la
commande. J'évite les entiers de brouillon fournis par \Expliii{} car
je préfère, pour ne pas me perdre, des noms plus explicites. De toute
façon, j'ai besoin d'au moins trois entiers locaux. On devrait pouvoir
écrire la décomposition de \(75\) sous la forme \(75=3\times5^{2}\) à
l'aide d'un \gls{entier} supplémentaire. Là encore, l'exercice etc.


%%% Local Variables:
%%% mode: latex
%%% TeX-master: "dun19expl3"
%%% End:
