%%
%% This is file `ResourcesHelp.tex'
%% 
%% Copyright 2017 American Mathematical Society.
%% 
%% This file is part of the collection comprising the AMS Author Handbooks.
%% For details and license information, see the file README-AH.txt.
%%
%% The Current Maintainer of this work is the American Mathematical
%% Society.
%% 
%% ========================================================================
%% 

\chapter{Resources and getting help}\label{ch:resandhelp}

\enlargethispage{.5pc}

\phantomsection
\section{Getting help: AMS resources}\label{sec:amsresources}
\label{ch:amsresources}

Many questions raised by authors are answered in the AMS Author FAQ
\cite{FAQ}.  Please check there before asking for assistance.

\hypertarget{adr-tech-support}{}%
If you encounter difficulties in preparing or submitting your
manuscript in electronic form after it has been accepted for
publication by the appropriate editorial board, you can ask for help
from AMS Technical Support:

\medskip
\begingroup
\obeylines
Publications Technical Group
Phone: 800-321-4267, ext.\ 4080 \quad or \quad 401-455-4080
Email: \href{mailto:tech-support@ams.org}{\texttt{tech-support@ams.org}}
\endgroup
\medskip


All written correspondence should be sent to the appropriate AMS
department at:

\medskip
\begingroup
\obeylines
American Mathematical Society
201 Charles Street
Providence, RI \ 02904-2294 \ USA
\endgroup
\medskip

\noindent or by FAX to 401-331-3842.


%% instructions for journals or Memoirs
\ifmonograph
\else \ifproceedings
\else
\medskip
See submission instructions on the web starting from

\begingroup
\obeylines
\href{http://www.ams.org/authors/journal-list}{\texttt{www.ams.org/authors/journal-list}}
\endgroup
\fi\fi
%% end instructions for journals or Memoirs

%% instructions for books: monographs or proceedings/collections
\ifjournal
\else \ifmemoirs
\else
\medskip
\hypertarget{adr-acquisitions}{}Questions concerning what you need
to prepare your manuscript should be directed to:

\beginexample{\rm}
Acquisitions Department
Phone: 800-321-4267, ext.\ 4051\quad or\quad 401-455-4051
Email: \href{mailto:acquisitions@ams.org}{\texttt{acquisitions@ams.org}}
\endexample
\fi\fi
%% end instructions for books: monographs or proceedings/collections

\medskip
Problems in accessing the web server should be reported to:

\beginexample{\rm}
Email: \href{mailto:webmaster@ams.org}{\texttt{webmaster@ams.org}}
\endexample


\phantomsection
\section{\texorpdfstring{\protect\tex/}{TeX} resources}
\label{ch:texresources}

\latex/ and \tex/ are available on the web free of charge.
There are also several commercial \tex/ implementations.
AMS web pages devoted to \tex/ information can be accessed at
\href{http://www.ams.org/tex}{\texttt{www.ams.org/tex}}\,.
The first of these pages has links to other
pages that identify the various sources for the \tex/ program.

\latex/ is the most popular of the free front ends designed for use
with \tex/, the basic typesetting program.  Whereas plain \tex/
defines basic macros, \latex/ defines stylistic packages, setting up
styles for a monograph, journal article, and article in a proceedings
collection, which you can then alter to your own specifications.

\amslatex/ is a collection of \latex/ extensions that make various
kinds of mathematical constructions easier to produce and take more
care with certain finer details in order to yield publication-quality
results.  It consists of two parts: \pkg{amsmath} (the part concerned
with the mathematics) and \pkg{amscls}.
The latter is a collection of companion design
setup packages (variously referred to as `document class' or `class' files)
 that enable authors writing a monograph or article to
get largely the same visual appearance in their preliminary drafts as
in a final publication with the AMS\@.  Both parts of \amslatex/ are
included in the canonical \latex/ distribution as part of \TeX\,Live.

Updates for \pkg{amsmath} are best obtained from
\href{http://www.ctan.org/search.html}{CTAN};
updates for \pkg{amscls} can be obtained either from CTAN or from the
AMS web server at
\href{http://www.ams.org/tex}{\texttt{www.ams.org/tex}}\,.
Other AMS packages and collections are the AMSFonts and \pkg{amsrefs}.
These too are included in \TeX\,Live as well as available from both
the AMS web server and CTAN\@.  All distributions include a copy of
the relevant User's Guide and other related documentation in PDF form,
which can either be printed or viewed electronically.  (This
Author Handbook is the User's Guide to the \pkg{amscls} collection.)

The book \textit{More Math into \latex/} \cite{Gr} is written from the point
of view of a mathematician using \amslatex/, and contains many examples.
The \textit{Guide to \latex/}, fourth edition \cite{KD}, is a good
general introduction to \latex/.  The original and authoritative manual
 for \latex/ is the \textit{\latex/ User's Guide \& Reference Manual}
\cite{La}.  George Gr\"atzer has also written a series of articles for
\emph{Notices of the AMS} \cite{Gr1,Gr2,Gr3,Gr4,Gr5,Gr6,Gr7} that keeps
the interested user up-to-date with the latest developments in \latex/.

Another source of information on \tex/ and \latex/ is the \tex/ Users
Group (TUG)\@.  They can be contacted at:

\beginexample{\rm}
  \href{http://tug.org}{\tex/ Users Group}
  P.\,O.\,Box 2311
  Portland, OR 97208-2311
  (503) 223-9994, FAX: (206) 203-3960
  \href{mailto:office@tug.org}{\texttt{office@tug.org}}
\endexample

\noindent
TUG also distributes the \tex/ Live collection, which includes ready-to-run
implementations of \tex/ for Windows, Mac, and Unix platforms, as well as \latex/
%, \amstex/, 
and an extensive selection of packages, all freeware.

\section{Online assistance}

One of the best places to ask for assistance is the group known by the
acronym CTT,
\href{https://groups.google.com/forum/#!forum/comp.text.tex}{\texttt{groups.google.com/forum/comp.text.tex}}\,.
Most of the people who use CTT are more than willing to answer questions and give advice.

Another online source of assistance is
\href{http://tex.stackexchange.com}{\texttt{tex.stackexchange.com}}\,.
This is organized differently from most discussion groups. After signing
up, you pose and answer questions. In the process, you gain points
which in turn allow you to do more in the group. Be sure to read
\href{http://tex.stackexchange.com/about}{\texttt{tex.stackexchange.com/about}}
to get you started.

The AMS is not equipped to handle questions about specific platforms.
Links to sites providing such support, as well as addresses for discussion
lists and links for on-line forums, are given on this AMS web page:\\
\href{http://www.ams.org/tex/additional-sources}{\texttt{www.ams.org/tex/additional-sources}}.

\endinput
