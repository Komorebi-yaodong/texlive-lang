%%
%% This is file `Mem-Checklist.tex'
%% 
%% Copyright 2017 American Mathematical Society.
%% 
%% This file is part of the collection comprising the AMS Author Handbooks.
%% For details and license information, see the file README-AH.txt.
%%
%% The Current Maintainer of this work is the American Mathematical
%% Society.
%% 
%% ========================================================================
%% 

\enlargethispage{1\baselineskip}

\section{A checklist for using the \Memos\ class}\label{sec:scheck}

\begin{itemize}

\item \Memos{}
are published both in print and online.  In order to expedite
processing, the following restrictions are placed on
\Memos\ articles:

\begin{itemize}
\item Only ``public'' macro packages (packages available from
  \href{http://www.ctan.org/search.html}{CTAN}, the Comprehensive
  \TeX\ Archive Network) may be used.
\item Other macro definitions must be embedded in the preamble of
  the main file.  Electronic files must
  be able to be processed
  independently with all macros (not entire macro files) included.
\item Macros should be defined with \ttcs{newcommand}, not with \ttcs{def}.
\end{itemize}

\item \the\AddHyperref
 
 \item Academic or other \textbf{affiliations} should be provided.
 Addresses are not included in the print version but are displayed
 in the on-line Abstract page.  Use \ttcs{address}, \ttcs{curraddr},
 and \ttcs{email} as provided in the template.
 
\item \Memos\ must contain an \textbf{abstract}.
 The main purpose of the abstract is to enable readers to take in
 the nature and results of the article quickly.  \textit{Zentralblatt}
 publishes authors' abstracts instead of reviews, so the abstract may also
 appear there.  The abstract should contain no text references to the
 bibliography unless the bibliographic reference is fully supplied. For
 example, [3] is meaningless to the reader once the abstract is separated
 from the \Memo. The abstract may comprise multiple paragraphs
 and include displayed material if appropriate. The length of the abstract
 depends primarily on the length of the \Memo\ itself and on the
 difficulty of summarizing the material, but an upper limit of about
 300 words is suggested.

\item \the\GrantsThanks
 
\item Unmarked, unnumbered \textbf{footnotes} on the abstract page of
 a \Memo{}
 should include primary classification numbers according to the
 2010 Mathematics Subject Classification scheme (\url{www.ams.org/msc})
 (required); grant information (optional); and key words and phrases
 describing the subject matter of the \Memo\ (optional).
 Formatting is automatic when using the AMS style files.

\item Use a \textbf{driver file} and
 put the source code for each chapter in a separate file, using
 \verb+\include+ (not \verb+\input+) to pull them together into a single document.

\item \the\ChapterRight
 
\item \the\ChapterTitleUC

\item \the\RunHeadMM

\item The title and copyright pages are for information only, so that
 a printed copy can be associated with the correct author(s).  The final
 copy will be prepared at the AMS using the information you have provided.

\item \the\BiblioBooks
 
\item \the\ConsentToPublish

\end{itemize}

\endinput
