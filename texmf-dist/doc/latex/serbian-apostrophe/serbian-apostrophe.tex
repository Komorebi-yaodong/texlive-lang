%%
%% serbian-apostrophe.tex
%% Copyright 2011 Zoran Filipovi\'{c}
%
% This work may be distributed and/or modified under the
% conditions of the LaTeX Project Public License, either version 1.3
% of this license or (at your option) any later version.
% The latest version of this license is in
% http://www.latex-project.org/lppl.txt
% and version 1.3 or later is part of all distributions of LaTeX
% version 2005/12/01 or later.
%
% This work has the LPPL maintenance status `maintained'.
%
% The Current Maintainer of this work is Zoran Filipovi\'{c}
%
%%
\documentclass[a4paper,12pt]{memoir}
\usepackage[T1]{fontenc}
\usepackage[cp1250]{inputenc}
\usepackage{textcomp}

\usepackage{serbian-apostrophe}

\usepackage{xspace}
\usepackage{pifont}
\usepackage{tipa}

\title{The \textsf{serbian-apostrophe} package}
\author{Zoran T. Filipovi\'{c} \\ Jurija Gagarina 263/6 \\ 11070 New Belgrade, Serbia}

\begin{document}
\frenchspacing
\maketitle

\begin{abstract}
This package\footnote{this package is under LPPL licence, version 1.3c} contains command 
for serbian words in latin scripts and output this command is same serbian words with 
apostrophes, such as \verb|\krenuti| $\rightarrow$ \krenuti or 
\verb|\ovako| $\rightarrow$ \ovako or \verb|\mogao| $\rightarrow$ \mogao.
\end{abstract}

\section*{Introduction}

This package is activate by typing \verb|\usepacake{serbian-apostrophe}| in preamble 
for your document, and provide lot of command for words in serbian language in latin 
scripts and output is same words with apostrophe. The list of words with apostrophes 
you may see in document \verb|apostrophe-list.pdf| and same words in
\verb|serbian-apostrophe.sty| package. 

\section*{Example}

For example we use translating in serbian language book of \textit{Milton's poems 
J. M. Dont and Sons LTD, London---New York, 1963.} Now, if you type 

\begin{verbatim}
\PlainPoemTitle
\PoemTitle{\v{S}esto Pevanje}
\settowidth{\versewidth}{Pod pokrovom joj tmastim po�inu�e oba}
\begin{verse}[\versewidth]
\linenumberfrequency{5}
\setverselinenums{405}{5}
Sad no� smer svoj za�e, preko Neba  \\
Strela je tamu i po�in \darujuci dragi, \\
U ti�inu ratnoj buci mrskoj; \\
Pod pokrovom joj tmastim po�inu�e oba, \\
Pobednik i gonjeni. Na bojnome polju \\ 
Mihajlo i Angeli mu �to nadja�ahu \\
Utabori�e se, \postavivsi uokrug stra�e, \\
Heruvse \vijuci vatre; na drugom kraju \\
Satan, s pobunjenim svojim nesta, \\
Daleko se u tami \smestivsi, i bez predaha, \\
Mo�nike svoje u savet sazva no�ni, \\ 
I u sred njih neustra�en im stade: \\
\end{verse}
\end{verbatim}

\noindent
this produce:

\PlainPoemTitle
\PoemTitle{\v{S}esto Pevanje}
\settowidth{\versewidth}{Pod pokrovom joj tmastim po�inu�e oba}
\begin{verse}[\versewidth]
\linenumberfrequency{5}
\setverselinenums{405}{5}
Sad no� smer svoj za�e, preko Neba  \\
Strela je tamu i po�in \darujuci dragi, \\
U ti�inu ratnoj buci mrskoj; \\
Pod pokrovom joj tmastim po�inu�e oba, \\
Pobednik i gonjeni. Na bojnome polju \\ 
Mihajlo i Angeli mu �to nadja�ahu \\
Utabori�e se, \postavivsi uokrug stra�e, \\
Heruvse \vijuci vatre; na drugom kraju \\
Satan, s pobunjenim svojim nesta, \\
Daleko se u tami \smestivsi, i bez predaha, \\
Mo�nike svoje u savet sazva no�ni, \\ 
I u sred njih neustra�en im stade: \\
\end{verse}


\begin{comment}
\begin{verbatim}
\PlainPoemTitle
\PoemTitle{Prvo Pevanje}
\settowidth{\versewidth}{Nepoku\v{s}ano u Prozi \ili Rimi}
\begin{verse}[\versewidth]
\linenumberfrequency{5}
\setverselinenums{15}{5}
Preko Heliona, \zheleci da �ini \\
Nepoku�ano u Prozi \ili Rimi. \\
A prvo Ti, O Du�e, �to si \shtovao \\
\end{verse}
\end{verbatim}

\noindent
this produce:

\PlainPoemTitle
\PoemTitle{Prvo Pevanje}
\settowidth{\versewidth}{Nepoku\v{s}ano u Prozi \ili Rimi}
\begin{verse}[\versewidth]
\linenumberfrequency{5}
\setverselinenums{15}{5}
Preko Heliona, \zheleci da �ini \\
Nepoku�ano u Prozi \ili Rimi. \\
A prvo Ti, O Du�e, �to si \shtovao \\
\end{verse}

\noindent or if you tupe:

\begin{verbatim}
\PlainPoemTitle
\PoemTitle{�etvrto Pevanje}
\settowidth{\versewidth}{A ne pokornost, \hvaleci se da mogu \podjarmiti}
\begin{verse}[\versewidth]
\linenumberfrequency{5}
\setverselinenums{85}{85}
A ne pokornost, \hvaleci se da mogu \podjarmiti \\
Svemo�nog. Aj, meni, kako malo oni znaju \\
Kako te�ko ja snosim tu ispraznu hvalu, \
Kakva me mu�enja kidaju iznutra \\
Dok me obo�avaju na prestolju Pakla, \\
Gde dijademu i skiptar uzdignut nosim, \\ 
\end{verse}
\end{verbatim}

\noindent
this produce:

\PlainPoemTitle
\PoemTitle{�etvrto Pevanje}
\settowidth{\versewidth}{A ne pokornost, \hvaleci se da mogu \podjarmiti}
\begin{verse}[\versewidth]
\linenumberfrequency{5}
\setverselinenums{85}{85}
A ne pokornost, \hvaleci se da mogu \podjarmiti \\
Svemo�nog. Aj, meni, kako malo oni znaju \\
Kako te�ko ja snosim tu ispraznu hvalu, \\
Kakva me mu�enja kidaju iznutra \\
Dok me obo�avaju na prestolju Pakla, \\
Gde dijademu i skiptar uzdignut nosim, \\ 
\end{verse}
\end{comment}

\begin{center}
\ding{167} \qquad \ding{167}  \qquad \ding{167}
\end{center}

\centerline{So, happy \TeX ing in serbian language.}

\end{document}