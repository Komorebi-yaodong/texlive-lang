\documentclass[12pt]{article}
\usepackage[czech]{babel} % Zapne deleni slov.
\usepackage{times} % Misto CS-fontu budou pouzity PostScriptove fonty
\usepackage[utf8]{inputenc} % nastavení kódování
\usepackage[IL2]{fontenc}
\parskip=\baselineskip % Mezi odstavci bude prázdný řádek
\emergencystretch=4em  % abych dovolil vetsi mezery a nebyly Overfull

\begin{document}
Tady je text napsaný PostScriptovým fontem včetně zpracování akcentů v~českém
jazyce a správného dělení slov.

Matematika i při sazbě PostScriptovým fontem zůstává ve fontu CM.
Proto je rozdíl mezi číslovkou \uv{2} napsanou přímo v textu a číslovkou
\uv{$2$} napsanou na vstupu mezi dolary.

Kromě stylu {\tt times.sty} je možno použít styly pro přepínání do dalších
PostScriptových fontů: {\tt avant.sty}, {\tt bookman.sty}, 
{\tt helvet.sty}, {\tt newcent.sty}, {\tt palatino.sty}
(vyzkoušejte si).

V plainu lze též použít stejným způsobem PostScriptové fonty, ovšem
místo volání výše uvedených stylů je nutno volat soubory 
{\tt ctimes.tex}, {\tt cavantga.tex}, {\tt cbookman.tex}, 
{\tt chelvet.tex}, {\tt cncent.tex}, {\tt cpalatin.tex}.  Například kdekoli
v textu píšeme \verb|\input ctimes| a od této chvíle máme text sázený 
fontem Times-Roman.

\end{document}

