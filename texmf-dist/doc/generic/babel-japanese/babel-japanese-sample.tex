%%
%% This is file `babel-japanese-sample.tex',
%% generated with the docstrip utility.
%%
%% The original source files were:
%%
%% babel-japanese.dtx  (with options: `sample')
%% 
\ifx\epTeXinputencoding\undefined\else
  \epTeXinputencoding utf8
\fi
\ifx\directlua\undefined
 \ifx\kanjiskip\undefined
  \ifx\XeTeXversion\undefined
    % pdfLaTeX: not beautiful
    \documentclass{book}
    \usepackage[whole]{bxcjkjatype}
  \else
    % XeLaTeX: not beautiful
    \documentclass{book}
    \usepackage{zxjatype}
    \setCJKmainfont{ipaexm.ttf}
  \fi
 \else
  \ifx\ucs\undefined
    % pLaTeX: OK
    \documentclass{jbook}
    \usepackage{minijs}
  \else
    % upLaTeX: OK
    \documentclass{ujbook}
  \fi
 \fi
\else
    % LuaLaTeX: OK
    \documentclass{ltjbook}
\fi
\usepackage[german,english,japanese]{babel}
\makeatletter
\def\tbcaption{\def\@captype{table}\caption{キャプションの例}}
\def\fgcaption{\def\@captype{figure}\caption{キャプションの例}}
\makeatother
\def\yes{--- はい。}
\def\no{--- いいえ。}
\def\TEXT{Textverarbeitung mit einem Rechner kann in vielf\"altiger Weise
erfolgen. Eigenschaften und Leistungsf\"ahigkeit sind hierbei weniger
vom jeweiligen Rechnertype, sondern vielmehr vom verwendeten
\textit{Textverarbeitungsprogramm} bestimmt.}
\def\se{\selectlanguage{english}}
\def\sj{\selectlanguage{japanese}}
\def\sg{\selectlanguage{german}}
\setlength{\hoffset}{-13mm}
\setlength{\textwidth}{16cm}
\begin{document}

\chapter{babel}
\section{japaneseパッケージ}
japaneseパッケージは日本語による見出し語と日付を出力するためのマクロを
定義しています。

\fgcaption

\begin{itemize}
\se
\item ここで英語(\texttt{english})に変更します。
(languageの値は\the\language)

\TEXT

ここは英語? \iflanguage{english}{\yes}{\no}\par
ここはドイツ語? \iflanguage{german}{\yes}{\no}\par
ここは日本語? \iflanguage{japanese}{\yes}{\no}

※ \verb:\adddialect\l@japanese0: と設定しているため,
日本語?も「はい」となります。

\sg
\item ここでドイツ語(\texttt{german})に変更します。
(languageの値は\the\language)

\TEXT

ここは英語? \iflanguage{english}{\yes}{\no}\par
ここはドイツ語? \iflanguage{german}{\yes}{\no}\par
ここは日本語? \iflanguage{japanese}{\yes}{\no}

※ ハイフネーションがドイツ語―旧正書法―に切り替わっている点に注目。
なお,新正書法(\texttt{ngerman})では
\texttt{Leis-tungs-f\"a-hig-keit}のように分綴します。

\sj
\item ここで日本語(\texttt{japanese})に変更します。
(languageの値は\the\language)
\tbcaption
\item \verb:\和暦: は日付の表示をデフォルトの西暦「\today 」から
和暦「\和暦\today 」に変更します。
\end{itemize}
\end{document}
\endinput
%%
%% End of file `babel-japanese-sample.tex'.
