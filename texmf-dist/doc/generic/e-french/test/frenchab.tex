% !TeX encoding = latin9
% This is frenchab.tex
%
%         French Torture Test with AmsLaTeX (main style: book)
%                                    Copyright Bernard Gaulle as in french.doc
%             
%                                                last mods 2006/04/25
%
\advance\hoffset by -10pt% marge droite (chez moi) pas totalement imprimee
%
\documentclass[twocolumn]{book}
 \usepackage[latin9]{inputenc}% pour le codage d'entr�e, ici latin9
 \usepackage[T1]{fontenc}% pour utiliser les fontes ec (et cm en math)
\usepackage{amsmath,mylist,mya4,graphicx,french}
\usepackage{eurosym} %for the � 
\usepackage{epstopdf}% so graphicx accepts eps images
\setlength{\textwidth}{410pt}%% 
\let\Stop\undefined
% Localisation code for numbered \typeout: "\kbAissue% localise it."
% For debugging one can remove "msg" access, just uncomment
%\let\kbAissue\relax% this line.
\makeatletter
   \ifx\kbAissue\undefined%
% Firstly we add the material to use the "msg" package for localization.
      \def\kb@issue#1#2{\kb@issue@[#1]#2\void}% The local \issuemsg macro.
                                      % which will call the real one;
                                      % #1 is the macro message required.
                                      % #2 is the message header + msg number
                                      %    such as "^^J -234-", just message 
                                      %    number (234) is kept. 
      \let\kbAissue\kb@issue%
      \def\kb@issue@[#1]#2-#3-#4\void{\issuemsg[#1]#3(french)}%
   \ifx\issuemsg\undefined\let\kbAissue\relax\let\typeouA\typeout\fi%
   \fi%
\makeatother
%
\ifx\endfrench\switchtolanguage
\kbAissue% localise it.
                 \typeout{-37a- ANOMALIE : french actif.}\Stop\fi
\def\LPLAIN{lplain}
\ifx\fmtname\LPLAIN\else\def\LPLAIN{LaTeX2e}\fi
\ifx\fmtname\LPLAIN\else\relax
                 \typeout{**********************************************}
\kbAissue% localise it.
     \typeout{-36- ANOMALIE : format LaTeX (\fmtname) non standard.}
     \typeout{CORRIGER pour avoir \fmtname=\LPLAIN\space ou ancien nom.}
\Stop% Vous devez avoir  dans ltvers.dtx : \def\fmtname{LaTeX2e} !
\fi
\let\iflatex\iftrue
\def\SmsG{\typeout%
    {\string<\string<--------------------------------------------------------}}
\def\FmsG{\typeout%
    {--------------------------------------------------------\string>\string>}}
{\SmsG\obeyspaces
\typeout{(          Test de torture du style \frenchname}
\typeout{(Ce test genere deliberement quelques messages d'erreur. )}
\typeout{(Pour vous permettre de distinguer les messages d'erreur )}
\typeout{(normaux de ceux qui sont anormaux, j'ai encadre, comme  )}
\typeout{(ici, ces messages ; ne vous en preoccupez pas.  --bg    )}
\FmsG}%
\frenchtest
\ifx\endfrench\switchtolanguage\else
\kbAissue% localise it.
                 \typeout{-37b- ANOMALIE : french inactif.}\Stop\fi
\end{document}
%%
%%      checksum        = "21525 63 238 2681"
%%
