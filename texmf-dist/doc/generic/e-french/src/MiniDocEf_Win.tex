% !TeX encoding = utf-8
\documentclass[11pt, a4paper]{article}

\usepackage[utf8]{inputenc}
\usepackage[T1]{fontenc}
\usepackage{times}
\usepackage{hyperref}
\usepackage{french}
\author{Raymond Juillerat}
\begin{document}
\section*{eFrench Windows : mini-doc d'installation }
revue pour eFrench sous LPPL le 13 août 2015, par RayJ\\[3ex]

  Cette distribution de eFrench a été testée avec Mik\TeX{} mais elle doit 
fonctionner avec tous les autres moteurs \TeX. Cette distribution
est LPPL. 

\subsection*{Consignes d'installation}

Placer le répertoire \backslash french  (et si nécessaire aussi \backslash msg, voir plus bas) à un endroit où 
   tous les fichiers seront accessibles par \backslash input.
\subsection*{Pour Mik\TeX 2.8, par exemple sous} 
   c:\backslash LocalTexMf\backslash tex\\
ou

   c:\backslash LocalTexMf\backslash tex\backslash plain\\
   et annoncez à Mik\TeX \textit{(Maintenance - Setting - Roots - Add ... )}
   ce répertoire  c:\backslash LocalTexMf en tant que répertoire supplémentaire de sources    
Cette configuration a aussi été testée avec succès sous Mik\TeX 2.9.

Les fichiers de \backslash makeindex vont dans le répertoire de même nom de  Mik\TeX {}
ou dans le répertoire de travail selon besoins.
Voir aussi les indications de \emph{FilesInTDS.txt} utilisables en tant qu'administrateur.

\paragraph*{Remarque}
Si pour une raison ou une autre (Mik\TeX version 2.9 ou plus ancien, autre système), french.sty annonce que msg.sty  ou msg-msg.tex manque, 
des solutions sont indiquées dans Probl\_Msg.pdf du dossier install.
\subsection*{PS}
\begin{enumerate}
\item Les messages sont codés en ANSI (latin9) 
\item    Le fichier language.dat doit définir french en tant
	 que langue, ce qui est le cas généralement
\end{enumerate}

\subsection*{Documentation }

Les manuels sont fournis sous forme de fichier en format pdf : 
efrench.pdf et frenchle.pdf, ce dernier étant aussi valable
pour efrench, les explications y sont plus détaillées 
mais concernent moins de possibilités. Et frenchle 
fait partie maintenant d'e-french.

\centering{Bonne utilisation de eFrench !}

  -- \href{mailto:raymond@juil.ch}{rayj }
 pour l'équipe d'eFrench, pour toute question passer par son initiateur,
{Laurent Bloch}
\href{https://www.laurentbloch.net/MySpip3/e-french-desormais-dans-les-depots?lang=fr}{e-french désormais dans les dépôts TeXLive et MiKTeX} .


\end{document}
