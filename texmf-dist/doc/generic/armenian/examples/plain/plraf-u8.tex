%%%%%%%%%%%%%%%%%%%%%%%%%%%%%%%%%%%%%%%%%%%%%%%%%%%%%%%%%%%%%%%%%%%%%%%%%%%%%%
%%
%% This is the `plraf-u8.tex' file (sample plain TeX text in UTF-8).
%%
%% This file is a part of the ArmTeX project [2014/04/09 v3.0-beta3]
%%
%% ArmTeX is a system for writing in Armenian with plain TeX and/or LaTeX(2e).
%%
%% Copyright 1997 - 2013:
%%   Serguei Dachian (Serguei.Dachian_AT_math.univ-bpclermont.fr),
%%   Arnak Dalalyan  (arnak.dalalyan_AT_ensae.fr),
%%   Vardan Akopian  (vakopian_AT_yahoo.com).
%%
%% ArmTeX may be distributed and/or modified under the conditions of the LaTeX
%% Project Public License, either version 1.3 of this license or (at your
%% option) any later version.
%%
%% The latest version of this license is in
%%   http://www.latex-project.org/lppl.txt
%% and version 1.3 or later is part of all distributions of LaTeX version
%% 2005/12/01 or later.
%%
%% ArmTeX has the LPPL maintenance status `author-maintained'.
%%
%% For more details, installation instructions and the complete list of files
%% see the provided `readme.txt' file.
%%
%%%%%%%%%%%%%%%%%%%%%%%%%%%%%%%%%%%%%%%%%%%%%%%%%%%%%%%%%%%%%%%%%%%%%%%%%%%%%%

\magnification=\magstep1

\input arm
\input armkb-u8

\font\artmLARGE=artmr10 at 16pt
\font\artmlarge=artmr10 at 12pt
\font\artmbflarge=artmb10 at 12pt
\font\artmsmall=artmr10 at 9pt
\hyphenchar\artmsmall=-1
\artm

\parskip=5pt

\vglue 20pt

\centerline{\artmLARGE ԱՐԱՐԱՏՅԱՆ ԴԱՇՏԻ ԱՌԱՎՈՏԸ}

\smallskip

\centerline{\artmlarge (հատված {\artmbflarge Րաֆֆու} «ՍԱՄՎԵԼ» վեպից)}

\vskip 60pt

{\leftskip=150pt\artmsmall
%
«Իսկ յայսմ ժամանակի (Մերուժանն Արծրունի եւ Վահանն Մամիկոնեան) աւերեցին
զքաղաքսն, եւ գերեցին զբնակեալսն անդ… եւ զայլ գերութիւնս՝ գաւառաց գաւառաց,
կողմանց կող\-մանց, փորի փորի, զաշխարհի աշխարհի, ածին ժողովեցին ի քաղաքն
Նախճուան, զի անդ էր զօրաժողով իւրեանց զօրացն»։
%
\par}

\bigskip

\hbox to \hsize{\hskip 150pt\hfill\artmsmall Փաւստոս։\hfill}

\vskip 40pt

Առավոտ էր, Արարատյան դաշտի լուսապայծառ առավոտներից մեկը։

Արևի առաջին ճառագայթների ներքո՝ Մասիսի սպիտակափառ գագաթը փայլում էր վարդագույն
շողքերով, որ աչք էին շլացնում։ Արագածի պսակաձև գագաթը չէր երևում։ Նա դեռ պատած
էր ձյունի պես ճերմակ մշուշով, որպես մի ամոթխած հարսիկ, որ սքողում է յուր դեմքը
անթափանցիկ շղարշով։ Կանաչազարդ դաշտավայրը, ցողված վաղորդյան մարգարիտներով,
վառվում էր ծիածանի ամենանուրբ գույներով։ Փչում էր մեղմ հովիկը, ծաղիկները
ժպտում էին, դալար խոտաբույսերը ծփում ու ծածանվում էին, և դաշտի խաղաղ
տարածությունը օրորվում էր սքանչելի ալեկոծությամբ։

Գեղեցի՜կ էր այդ առավոտը։

Թռչունները ուրախ-ուրախ ճախրում էին մի թուփից դեպի մյուսը։ Գույնզգույն
թի\-թեռ\-ները, գույնզգույն ծաղիկների նման, ցանված էին օդի մեջ։ Սպիտակ արագիլը,
կարմիր ոտները հորիզոնական դիրքում ուղիղ մեկնած, լայն թևքերով թափահարում էր,
շտապելով դեպի Արաքսի մորուտները։ Ձեռնասուն եղջերուները, վայրենի վիթն ու
այծյամը, դուրս էին եկել Խոսրովի արքայական անտառներից, և ազատ, համարձակ
վազվզում էին շրջակա մարգերի վրա։

Չէր երևում միայն մարդը։

Ամեն առավոտ, արծաթյա փողերի հնչյունը, որսորդական բարակների մռնչյունը,
սի\-գա\-պանծ նժույգների խրխինջը, խռովում էին սորամուտ անասունների վաղորդյան
հանգիստը։ Ամեհի վարազը սարսափելով նետվում էր մթին շամբուտների մեջ, իսկ թավամազ
արջը ապաստանի տեղ էր որոնում։ Իսկ այս առավոտ չկային նրանք,~—~չկային
նախարարական իշխանազն պատանիները, որոնց որսորդական ուրախ զվարճությունները մի
առանձին կեն\-դա\-նու\-թյուն էին բաշխում երեաշատ դաշտավայրին։

Ամեն առավոտ թռչունը կարդում էր յուր նախարշալույսյան մեղեդին, և նրա հետ լսելի
էր լինում ժրաջան մշակի երգը։ Փայլում էր մանգաղը, եռում էր գործը, և ոսկեղեն
հունձքը՝ յուր լիառատ բեղմնավորությամբ՝ պարգևատրում էր վաստակաբեկ շինականի
աշխատանքը։ Իսկ այս առավոտ չկար հնձվորը, չկար և հերկավարը։ Հասունացած արտը
մնացել էր կիսաքաղ, և անվաստակելի արորը անգործ ընկած էր դեռ չվերջացրած ակոսների
մոտ։

Ամեն առավոտ սուրբ տաճարի կոչնակի առաջին հնչման հետ՝ զարթնում էր հովիվը։
Ոչխարների անուշ բառանչը, արջառների ուրախ ձայնարկությունը կենդանացնում էին
խո\-տա\-վետ հովիտները խիստ ախորժելի աղմուկով։ Իսկ այս առավոտ չէին երևում ոչ
հովիվը և ոչ նրա հոտերը։ Ցիրուցան գառնուկները թափառում էին սար ու ձոր, և
հայրակորույս որբիկների նման, կարծես, որոնում էին հովվին։

Ամեն առավոտ, երբ ծագում էր տվընջյան լուսատուն, նրա առաջին ճառագայթները
ողջունում էին շինական աղջիկների աշխատանքը։ Կարմիր, դեղին, կապույտ
հագուստներով, որպես կարմիր, դեղին, կապույտ ծաղիկներ, սփռված էին լինում նրանք
այգիներում, բան\-ջա\-րա\-նոց\-նե\-րում և ագարակներում։ Երգում էին և գործում
էին։ Եվ նրանց ուրախությանը ձայնակից էր լինում երգասեր սոխակը։ Իսկ այս առավոտ
չէին երևում անխոնջ մշա\-կու\-թյուն\-նե\-րի այդ գեղեցիկ զարդերը։ Այգիները
մնացել էին անխնամ, ագարակները կորցրել էին իրանց սիրելի բանվորներին։

Արևը բարձրացավ, և որքան բարձրանում էր նա, այնքան Արարատյան ընդարձակ
դաշ\-տա\-վայրը, որպես մի հսկայական բուրվառ, խնկարկում էր յուր վաղորդյան
անուշհոտությունը։ Ամբողջ հովիտը ծխում էր, գոլորշիանում էր։ Ցողազարդ
բուսականությունը ետ էր տալիս երկնքին յուր ընդունած մարգարտյա կաթիլները։

\bye
