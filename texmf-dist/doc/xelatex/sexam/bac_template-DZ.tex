% !TEX TS-program = XeLaTeX
\documentclass[12pt]{exam}
\usepackage{bacex}

%*****************************************
%--------------------info---------------------
\newcommand{\alg}{\bf الجمهورية الجزائرية الديمقراطية الشعبية}
\newcommand{\gov}{\bf وزارة التربية الوطنية \hfill   الديوان الوطني للامتحانات و المسابقات}
\newcommand{\bac}{\bf امتحـان بكالوريا التعليم الثانوي 
  \hfill 
   دورة :  
جوان 2017
}
\newcommand{ \duree }{\bf المدة :}
\newcommand{ \niveau }{\bf 
الشعبة : 
تقني رياضي }
\newcommand{ \exam }{\bf اختبار في مادة : }
%--------------------------------------------------------------------------

\begin{document} 
%---------------------------------------------------------------
\begin{center}
\alg  
\\
\gov 
\\
\bac
\end{center}
{\niveau}

\rule{\textwidth}{1.4pt}\\[-2pt]
\centerline{ \exam 
الرياضيات
  \hfill  
\duree 
04 سا و 30 د }\\[-8pt]
\rule{\textwidth}{1.4pt}

%----------------------------------------------
\choi   %           اختيار الموضوعين
\\
%---------------------------------------
\one              % الموضوع  الأول
%----------------------------------------------
\vspace{0.6cm}
%--------------------------------------------------
\begin{questions}

\question [04]    %التمرين الأول
الفضاء منسوب إلى المعلم المتعامد المتجانس 
$\left(O;\overrightarrow{i};\overrightarrow{j};\overrightarrow{k}\right)$ 
.
\\

نعتبر النقط 
$A(0;-1;2)$
،
$B(3;2;5)$
،
$C(3;-1;-1)$
و
$D(-3;5;-1)$
\\

ليكن 
$(P)$
و
$(Q)$
المستويين اللذان معادلتاهما على الترتيب :
$x+y+z-1=0$
و
$x-z+2=0$
.\\
\begin{parts}
	\part
	بين أن المثلث 
	$ABC$
	قائم.
،
ثم عين معادلة ديكارتية للمستوي 
	$(ABC)$
	.	
	\part
\begin{subparts}
\subpart
بين أن المستويين
$(P)$
و
$(Q)$
متعامدان ثم جد تمثيلا وسيطيا للمستقيم 
 $(\Delta)$
 ،
 تقاطع المستويين
 $(P)$
و
$(Q)$
. 
\subpart
عين تقاطع المستويات 
$(P)$
 ،
$(Q)$
و
$(ABC)$
.
\end{subparts}
	\part
	تحقق أن 
	$A$
	هي المسقط العمودي للنقطة
	$D$
	على المستوي 
	$(ABC)$
	ثم احسب حجم رباعي الوجوه 
	$DABC$
	.
	\part
	بين أّن
	$\dfrac{\pi}{4}$
	قيس بالراديان للزاوية 
	$\widehat{BDC}$
	،
	ثم استنتج المسافة بين النقطة 
$A$
و المستوي 	
$(BDC)$
.	
\end{parts}


\question[04]       % بداية التمرين الثاني عليه 7 نقاط مثلا


\begin{parts}
\part
عين 
،
حسب قيم العدد الطبيعي 
$n$
،
باقي القسمة  الإقليدية  للعدد 
$3^n$
على
$5$
.
\part
استنتج باقي القسمة الإقليدية للعدد 
$1437^{2017}$
على
$5$
.
\part
برهن أن: من أجل كل عدد طبيعي
$n$
،
العدد
$(48^{4n+3}-2\times{9}^{2n+1}+1)$
مضاعف للعدد
$5$
.
\part
عين الأعداد الطبيعية 
$n$
حتى يكون العدد 
$(3^{4n}+27^n-4)$
قابلا للقسمة على
$5$
.
\end{parts}	


\question[05]     % التمرين الثالث 5 نقاط

{(I}
حل في مجموعة الأعداد المركبة 
$\mathbb{C}$
المعادلة ذات المجهول المركب 
$z$
الآتية:
$(z-4)(z^2-2z+4)=0$\\

{(II}
المستوي المركب منسوب إلى المعلم المتعامد المتجانس
$\left(O;\overrightarrow{u};\overrightarrow{v}\right)$ 
.
\\

نعتبر النقط
$A$
،
$B$
و
$C$
التي لاحقاتها
$z_A=4$
،
$z_B=1+i\sqrt{3}$
و
$z_C=1-i\sqrt{3}$
.
\begin{parts}
\part
اكتب العدد المركب 
$\dfrac{z_C-z_A}{z_B-z_A}$
على الشكل الأسي ثم استنتج طبيعة المثلث
$ABC$
.
\part
\begin{subparts}
\subpart
عين لاحقة النقطة
$D$
صورة
$B$
بالدوران
$r$
الذي مركزه المبدأ
$O$
و زاويته
$\dfrac{2\pi}{3}$
.
\subpart
عين طبيعة الرباعي 
$ABCD$
.
\end{subparts}
\part
من أجل كل عدد طبيعي 
$n$
،
نضع:
$z_n=(z_B)^n+(z_C)^n$
.
\begin{subparts}
\subpart
بين أن:
من أجل كل عدد طبيعي 
$n$
،
$z_n=2^{n+1}\times{\cos \left(\dfrac{n\pi}{3}\right)}$
.
\subpart
نضع من أجل كل عدد طبيعي 
$n$
:
$t_n=z_{6n}$
.
\\[5pt]

- عبر عن 
$t_n$
بدلالة
$n$
ثم احسب
$P_n$
بدلالة
$n$
جيث
$P_n=t_0\times{t_1}\times{t_2}\times{\cdots}\times{t_n}$
.
\end{subparts}
\end{parts}

 \question[07]   \hfill\\
 
(I
لتكن الدالة 
$g$
المعرفة على المجال
$\left]0;+\infty\right[$
كما يلي:
$g(x)=-\dfrac{1}{2}+\dfrac{2-\ln x}{x^2}$
.
\begin{parts}
\part
احسب
$\mathop {\lim }\limits_{x\mathop  \to \limits^
 \succ  0} g(x)$
و
$\mathop {\lim }\limits_{x \to  + \infty } 
g(x)
$
\part
ادرس اتجاه تغير الدالة
$g$
ثم شكل جدول تغيراتها
.
\part
بين أن المعادلة
$g(x)=0$
تقبل حلا وحيدا 
$\alpha$
حيث
$1,71<\alpha<1,72$
ثم استنتج اشارة 
$g(x)$
حسب قيم
$x$
. 
\end{parts} \hfill\\

 (II
نعتبر الدالة 
$f$
المعرفة على 
$\left]0;+\infty\right[$
كما يلي :
$f(x)=-\dfrac{1}{2}x+2+\dfrac{-1+\ln x}{x}$
.
\\
$(C_f)$
التمثيل البياني للدالة 
$f$
في المستوي المنسوب إلى المعلم المتعامد المتجانس 
$\left(O;\overrightarrow{i};\overrightarrow{j}\right)$ 
حيث 
$||\vec{i}||=1cm$
.
\begin{parts}
\part
\begin{subparts}
\subpart
احسب 
$\mathop {\lim }\limits_{x\mathop  \to \limits^
 \succ  0} f(x)$
و
$\mathop {\lim }\limits_{x \to  + \infty } 
f(x)$ .
\subpart
ادرس اتجاه تغير الدالة
$f$
ثم شكل جدول تغيراتها
.
\end{subparts}
\part
\begin{subparts}
\subpart
بين أن المستقيم 
$(\Delta)$
ذا المعادلة 
$y=-\dfrac{1}{2}x+2$
مقارب مائل للمنحنى 
$(C_f)$
.
\subpart
ادرس وضعية المنحنى 
$(C_f)$
بالنسبة إلى المستقيم 
$(\Delta)$
.
\end{subparts}
\part
" نقبل أن 
$f(\alpha)\simeq{0,87}$
و
$f(\gamma)=f(\beta)=0$
حيث 
$0,76<\beta<0,78$
و
$4,19<\gamma<4,22$
".
\\
- أنشئ في المعلم السابق المستقيم 
$(\Delta)$
و المنحنى
$(C_f)$
.
\part
ليكن 
$\lambda$
عدد حقيقي حيث 
$1<\lambda\leqslant{e}$
،
نرمز ب
$\mathscr{A}(\lambda)$
إلى مساحة الحيز المستوي المحدد بالمنحنى
 $(C_f)$
 \\
 و المستقيم 
 $(\Delta)$
 و المستقيمين اللذين معادلتاهما :
 $x=1$
 و
 $x=\lambda$
 .
 \begin{subparts}
 \subpart
 احسب 
$\mathscr{A}(\lambda)$
بدلالة
$\lambda$
. 
  \subpart
  عين قيمة 
  $\lambda$
  حيث 
 $\mathscr{A}(\lambda)=\dfrac{1}{2}cm^2$
 . 
 \end{subparts}
\end{parts}

\end{questions}     % نهاية التمارين
%=========================================================
                   %  fin
%========================================================


%========================================================
\newpage
%      الموضوع الثاني
\two
 
 \begin{questions}
\question[04]   
الفضاء منسوب إلى المعلم المتعامد المتجانس
$\left(O;\overrightarrow{i};\overrightarrow{j};\overrightarrow{k}\right)$ 
. 
نعتبر النقط 
$A(1;1;-1)$
،
$B(1;7;-3)$
و
$I(O;1;-2)$
\\ 
و الشعاع
$\vec{v}(2;0;2)$
،
$(\Delta_1)$
المستقيم الذي يشمل النقطة 
$A$
و
$\vec{v}$
شعاع توجيه له و 
$(\Delta_2)$
المستقيم المعرف
\\
بالتمثيل الوسيطي :
$\begin{cases}
x=-1+2t \\
y=2-t \;;\;(t \in \mathbb {R}) \\
z=3-4t
\end{cases}$ 
\begin{parts}
\part
بين ان 
$A$
تنتمي الى المستقيم 
$(\Delta_2)$
و أن
$(\Delta_1)$
و
$(\Delta_2)$
غير متطابقان
.
\part
ليكن 
$(P)$
المستوي المعين بالمستقيمين 
$(\Delta_1)$
و
$(\Delta_2)$
.
\\
-بين أن الجملة:
$\begin{cases}
x=1+2\alpha+2\beta \\
y=1-\alpha \;;\;(\alpha\in \mathbb{R},\beta \in \mathbb{R}) \\
z=-1-4\alpha+2\beta
\end{cases}$
تمثيل وسيطي للمستوي
$(P)$
.
\part
أثبت أن 
$I$
هي المسقط العمودي للنقطة 
$B$
على المستوي
$(P)$
.
\part
لتكن
$(S)$
مجموعة النقط
$M(x;y;z)$
من الفضاء حيث 
$x^2+y^2+z^2-2x-14y+6z+21=0$
.
\begin{subparts}
\subpart
بين أن 
$(S)$
سطح كرة يطلب تحديد مركزها و نصف قطرها
.
\subpart
تحقق أن المستوي 
$(P)$
يمس
$(S)$
في نقطة يطلب تعيينها
.
\end{subparts}
\end{parts}

\question[04]  

نعتبر المتتالية 
$(u_n)$
المعرفة ب:
$u_1=\dfrac{1}{a}$
و من أجل كل عدد طبيعي 
$n$
غير معدوم ،
 $u_{n+1}=\dfrac{n+1}{an}u_n$
 \\
 حيث 
 $a$
 عدد حقيقي أكبر من أو يساوي 
 $2$
 .
 \begin{parts}
\part
 \begin{subparts}
 \subpart
 بين أن : من أجل كل عدد طبيعي 
 $n$
 غير معدوم :
 $u_n>0$
 .
 \subpart
 بين ان المتتالية 
 $(u_n)$
 متناقصة تماما ثم استنتج أنها متقاربة
 .
 \end{subparts}
  \part
  نعتبر المتتالية 
  $(v_n)$
  المعرفة كما يلي :من أجل كل عدد طبيعي 
  $n$
  غير معدوم ،
  $v_n=\dfrac{1}{an}u_n$
  .
  \begin{subparts}
  \subpart
بين أن المتتالية 
$(v_n)$
هندسية أساسها 
$ \dfrac{1}{a} $
و عين حدها الأول
$v_1$
بدلالة
$a$
.
   \subpart
   جد بدلالة 
   $n$
   و
$a$
عبارة الحد العام 
  $v_n$
  ثم استنتج عبارة 
  $u_n$
  و احسب 
  $\mathop {\lim }\limits_{n \to  + \infty } 
u_n $
  . 
  \end{subparts}
   \part
   احسب بدلالة 
    $n$
   و
$a$
المجموع 
$S_n$
حيث
$S_n=u_1+\dfrac{1}{2}u_2+\cdots+\dfrac{1}{n}u_n$
\\
ثم عين قيمة 
$a$
حيث
$\lim\limits_{n \to +\infty}S_n=\dfrac{1}{2016}$
.
 \end{parts}
 %*****
 \newpage
 %*****************
\question[04]  

(I
حل في مجموعة الأعداد المركبة
$\mathbb{C}$
المعادلة ذات المجهول
$z$
الآتية:
$(z+1-\sqrt{3})(z^2+2z+4)=0$
.\\

(II
المستوي المركب منسوب إلى المعلم المتعامد المتجانس
$\left(O;\overrightarrow{u};\overrightarrow{v}\right)$ 
. 
\\
نعتبر النقط 
$A$
،
$B$
و
$C$
التي لاحقاتها 
$z_A=-1+\sqrt{3}$
،
$z_B=-1-i\sqrt{3}$
و
$z_c=\overline{z_B}$
.
\begin{parts}
\part
بين أن 
$z_B-z_A=i(z_C-z_A)$
ثم استنتج طبيعة المثلث
$ABC$
و احسب مساحته
.
\part
\begin{subparts}
\subpart
أكتب على الشكل الجبري العدد المركب 
$L$
حيث 
$L=\dfrac{z_C-z_A}{z_C}$
.
\subpart
بين أن:
$L=\dfrac{\sqrt{6}}{2}\left(\cos \dfrac{\pi}{12}+i \sin \dfrac{\pi}{12}\right)$
ثم استنتج القيمة المضبوطة ل
$\tan \dfrac{\pi}{12}$
.
\end{subparts}
\part
نعتبر التحويل النقطي 
$S$
الذي يحول النقطة 
$M$
ذات الاحقة
$z$
الى النقطة 
$M'$
ذات الاحقة
$z'$
و المعرف
\\
بـ:
$z'=(z-z_B)L+z_B$
\\

- بين أن 
$S$
تشابه مباشر يطلب تحديد عناصره المميزة
.
\part
لتكن النقط
$A'$
،
$B'$
و
$C'$
صور النقط 
$A$
،
$B$
و
$C$
على الترتيب بالتحويل 
$S\circ{S}$
.
\\

-احسب مساحة المثلث
$A'B'C'$
.
\end{parts}


\question[07]  

(I
لتكن الدالة 
$g$
المعرفة على
$\mathbb{R}$
كما يلي:
$g(x)=1-2xe^{-x}$
.
\\
-ادرس اتجاه تغير الدالة 
$g$
ثم استنتج اشارة
$g(x)$
.\\

(II
نعتبر الدالة 
$f$
المعرفة على
$\mathbb{R}$
كما يلي:
$f(x)=(x+1)(1+2e^{-x})$
.
\\
$(C_f)$
التمثيل البياني للدالة 
$f$
في المستوي المنسوب إلى المعلم المتعامد المتجانس
$\left(O;\overrightarrow{i};\overrightarrow{j}\right)$ 
حيث
$||\vec{i}||=1cm$
.
\begin{parts}
\part
\begin{subparts}
\subpart
احسب 
$\lim\limits_{x \to -\infty} f(x)$
و
$\lim\limits_{x \to +\infty}f(x)$
.
\subpart
ادرس اتجاه تغير الدالة 
$f$
ثم شكل جدول تغيراتها
.
\end{subparts}
\part
\begin{subparts}
\subpart
بين أن:
$\lim\limits_{x \to +\infty}\left[f(x)-1\right]=1$
ثم استنتج معادلة لــ
$(\Delta)$
،
المستقيم المقارب المائل للمنحنى
$(C_f)$
.
\subpart
ادرس وضعية المنحنى 
$(C_f)$
بالنسبة الى المستقيم 
$(\Delta)$
.
\end{subparts}
\part
اثبت أن المنحنى 
$(C_f)$
يقبل مماسا وحيدا
$(T)$
يوازي
$(\Delta)$
يطلب تعيين معادلة له
.
\part
باستعمال المنحنى
$(C_f)$
،
عين قيم الوسيط الحقيقيي
$m$ 
حتى يكون للمعادلة 
$f(x)=x+m$
حلين مختلفين
.
\part
ليكن 
$\alpha$
عددا حقيقيا موجبا
،
نرمز ب
$\mathscr{A}(\alpha)$
الى مساحة الحيز المستوي المحدد بالمنحنى
$(C_f)$
\\
و بالمستقيمات التي معادلاتها على الترتيب :
$y=x+1$
،
$x=-1$
و 
$x=\alpha$
.
\\

-احسب 
$\mathscr{A}(\alpha)$
بدلالة
$\alpha$
ثم 
$\lim\limits_{\alpha \to +\infty}\mathscr{A}(\alpha)$
.
\end{parts}


\end{questions}


\end{document}