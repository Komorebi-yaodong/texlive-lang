\documentclass[autodetect-engine,dvi=dvipdfmx,ja=standard,
  a4paper]{bxjsarticle}
\usepackage{setspace}
\usepackage{pxrubrica}
\newcommand*{\vb}{\symbol{`\|}}
\newcommand*{\Opt}[1]{\texttt{#1}}
\newcommand*{\PKN}[1]{\textsf{#1}}
\newsavebox{\myexample}
\rubysetup{f}
\begin{document}

\title{\PKN{pxrubrica}パッケージサンプル}
\author{某ZR}
\date{コンパイル日付: \today}
\maketitle

\section{サンプル}

\subsection{基本的な用法}

\begin{itemize}
\item モノルビ(\Opt{m}オプション):各漢字に一つのルビブロック\\
例:\quad
\verb+\ruby[m]{鷹}{たか}+ → \ruby[m]{鷹}{たか}\quad
\verb+\ruby[m]{鶯}{うぐいす}+ → \ruby[m]{鶯}{うぐいす}
\item グループルビ(\Opt{m}オプション):漢字列全体に一つのルビブロック\\
例:\quad
\verb+\ruby[g]{雲雀}{ひばり}+ → \ruby[g]{雲雀}{ひばり}\quad
\verb+\ruby[g]{不如帰}{ほととぎす}+ → \ruby[g]{不如帰}{ほととぎす}
\item 熟語ルビ(\Opt{j}オプション):各漢字にルビを対応させるが熟語として読む\\
例:\quad
\verb+\ruby[j]{孔雀}{く|じゃく}+ → \ruby[j]{孔雀}{く|じゃく}\quad
\verb+\ruby[j]{七面鳥}{しち|めん|ちょう}+ → \ruby[j]{七面鳥}{しち|めん|ちょう}
\item ルビ文字列中の \verb+|+ は各漢字の読みの境界を示す。
(孔=く、雀=じゃく)。
グループルビでは不要である。
\item 組版結果の比較:
\begin{quote}\begin{tabular}{ll@{\ }l@{\ }c*3{@{\quad}c}}
モノルビ & (\verb+\ruby[m]{小鳩}{こ|ばと}+) & →
  & \ruby[m]{小鳩}{こ|ばと}
  & \ruby[m]{鶺鴒}{せき|れい}
  & \ruby[m]{雷鳥}{らい|ちょう}
  & \ruby[m]{燕}{つばめ}
\\
グループルビ & (\verb+\ruby[g]{小鳩}{こばと}+) & →
  & \ruby[g]{小鳩}{こばと}
  & \ruby[g]{鶺鴒}{せきれい}
  & \ruby[g]{雷鳥}{らいちょう}
  & \ruby[g]{燕}{つばめ}
\\
熟語ルビ & (\verb+\ruby[j]{小鳩}{こ|ばと}+) & →
  & \ruby[j]{小鳩}{こ|ばと}
  & \ruby[j]{鶺鴒}{せき|れい}
  & \ruby[j]{雷鳥}{らい|ちょう}
  & \ruby[j]{燕}{つばめ}
\end{tabular}\end{quote}
熟語の各漢字とルビが対応する場合は、熟語ルビ(\Opt{j})を使い、
そうでない(熟字訓の)場合はグループルビ(\Opt{g})を使うのが通例である。
特に熟語の各漢字ごとの読みを明示したい場合は
モノルビ(\Opt{m})を使うとよい。
なお、漢字一文字に対するルビの場合は、
\Opt{m}、\Opt{g}、\Opt{j}の何れも同じ結果になる。
\item オプションの既定値を \verb+\rubysetup+ 命令で設定できる。
例えば、\verb+\rubysetup{g}\ruby{軍鶏}{しゃも}+
は \verb+\ruby[g]{軍鶏}{しゃも}+ と等価になる。
“既定値の既定値”は\Opt{|cjPeF|}である。
\end{itemize}

\subsection{進入・突出}

\begin{itemize}
\item ルビの進入の制御:
\begin{center}\begin{tabular}{ll@{\ }l@{\ }c*2{@{\quad}c}}
進入無し & \verb+この\ruby[|-|]{鵲}{かささぎ}の+ & →
 & この\ruby[|-|]{鵲}{かささぎ}の
 & この\ruby[|-|]{鸛}{こうのとり}の
 & この\ruby[|-|]{鵜}{う}の
\\
進入量小 & \verb+この\ruby[(-)]{鵲}{かささぎ}の+ & →
 & この\ruby[(-)]{鵲}{かささぎ}の
 & この\ruby[(-)]{鸛}{こうのとり}の
 & この\ruby[(-)]{鵜}{う}の
\\
進入量大 & \verb+この\ruby[<->]{鵲}{かささぎ}の+ & →
 & この\ruby[<->]{鵲}{かささぎ}の
 & この\ruby[<->]{鸛}{こうのとり}の
 & この\ruby[<->]{鵜}{う}の
\end{tabular}\end{center}
\item もし「ルビは仮名にはかけてよいが漢字はダメ」という場合は、
“\verb+この\ruby[<-|]{鵲}{かささぎ}等+”
と書くと「この\ruby[<-|]{鵲}{かささぎ}等」の出力が得られる。
\item 基本モード(\Opt{m}/\Opt{g}/\Opt{j})と進入を同時に指定したい場合は、
オプション文字列を \Opt{|g|} や \Opt{|m>} のようにする。
ここで、“\Opt{-}”は「基本モードは既定値を用いる」ことを意味する。
\item 突出の制御:
オプション \Opt{||} で突出が抑止される。
\begin{quote}
\begin{lrbox}{\myexample}
\small$\leftarrow$ \verb+\ruby[||->]{雀}{すずめ}+
\end{lrbox}
\fbox{\parbox{.42\linewidth}{%
\ruby[||->]{雀}{すずめ}の…
\quad \usebox{\myexample}%
\rule{0pt}{12pt}\\
インコの
}}\quad vs.\quad
\begin{lrbox}{\myexample}
\small$\leftarrow$ \verb+\ruby[|->]{雀}{すずめ}+
\end{lrbox}
\fbox{\parbox{.42\linewidth}{%
\ruby[|->]{雀}{すずめ}の…
\quad \usebox{\myexample}%
\rule{0pt}{12pt}\\
インコの
}}
\end{quote}
\end{itemize}

\subsection{発展的な用法}

\begin{itemize}
\item \verb+\aruby+:欧文に対してルビを付ける:
\par\noindent 例:\quad
\verb+\aruby{Get out}{ゲラウッ}!+ →\
  \aruby{Get out}{ゲラウッ}!
\item \verb+\rubyfontsetup+:ルビ出力のためのフォントを指定する。
例えば、ゴシック体の漢字列に対して明朝体のルビを振りたい場合は、
次のようにする:
\par\noindent
{\small
\verb+\rubyfontsetup{\mcfamily}この{\gtfamily \ruby[j]{明朝体}{みん|ちょう|たい}}+}
→\
{\rubyfontsetup{\mcfamily}この{\gtfamily \ruby[j]{明朝体}{みん|ちょう|たい}}}
\end{itemize}

\end{document}
