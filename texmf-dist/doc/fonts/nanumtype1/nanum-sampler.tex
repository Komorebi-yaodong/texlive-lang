\documentclass[a4paper,12pt]{article}
\usepackage[cjk,hangul,usecjkt1font]{kotex}
\parindent0pt
\title{Nanum Font Sampler}\author{}\date{}
\begin{document}
\maketitle

NanumMyeongjo:

나눔글꼴은 네이버가 공개한 유니코드 기반의 한글 지원 글꼴이다.
그 가운데 나눔명조는 폰트릭스에서 제작했다.
기본 형식과 굵은 형식(Bold) 총 두 종으로 구성되어 있다.
\textbf{나눔글꼴 2.0 이상은 아주 굵은 형식(EB, Extra Bold)도 포함한다.
이 패키지는 아주 굵은 형식을 가지고서 굵은 글꼴을 식자한다.}

\bigskip
\sffamily

NanumGothic:

나눔글꼴은 네이버가 공개한 유니코드 기반의 한글 지원 글꼴이다.
그 가운데 나눔고딕은 산돌커뮤니케이션에서 제작했다.
기본 형식과 굵은 형식(Bold) 총 두 종으로 구성되어 있다.
\textbf{나눔글꼴 2.0 이상은 아주 굵은 형식(EB, Extra Bold)도 포함한다.
이 패키지는 굵은 형식을 가지고서 굵은 글꼴을 식자한다.}

\end{document}
