%!TEX TS-program = xetex
%&encoding=UTF-8 Unicode
%\\XeTeX_input_normalization=1
%\XeTeXinterwordspaceshaping=2

% Arabic text from http://www.unicode.org/standard/translations/arabic.html
%
\TeXXeTstate=1
\nopagenumbers \frenchspacing
\font\title="Alkalami Light:script=arab" at 28pt
\font\heading="Alkalami:script=arab" at 18pt
\font\body="Alkalami:script=arab" at 12pt \body
\font\romfont="Alkalami" at 12pt \def\rom#1{{\beginL\romfont #1\endL}}
\parindent=0.5in \baselineskip=22pt \lineskiplimit=-1000pt

\def\s#1{\bigskip \rightline{\beginR\heading #1\endR}\nobreak\medskip}

\centerline{\beginR\title ما هي الشفرة الموحدة يونِكود؟\endR}
\everypar={\tolerance=1600\setbox0=\lastbox \beginR \box0 }
\bigskip
تتعامل الحواسيب بالأسام مع الأرقام فقط، و تخزن الحروف و المحارف الأخرى بتخصيص رقم لكل واحد منها. قبل اختراع يونيكود كان هناك المئات من أنظمة الترميز المختلفة لتخصيص هذه الأرقام، و لم يكن هناك ترميز واحد يمكنه أن يحتوي كل المحارف المطلوبة: فمثلا الاتحاد الأوروبي وحده يحتاج إلى العديد من الترميزات المختلفة لتغطية جميع لغاته. و حتى لو نظرنا إلى لغة واحدة كالإنجليزية، فلم يكن هناك ترميز واحد قادر على استيعاب جميع الأحرف و علامات الترقيم و الرموز التقنية و العلمية شائعة الاستعمال.
‏

حتى أنظمة الترميز هذه تتعارض مع بعضها البعض. بعبارة أخرى، يمكن أن يستخدم ترميزان مختلفان نفس الرقم لتمثيل محرفين مختلفين، أو رقمين مختلفين لتمثيل نفس المحرف. أي حاسوب (خاصة الخواديم) عليه أن يدعم العديد من الترميزات المختلفة؛ و بعد كل هذا، فعندما تمرر البيانات عبر ترميزات أو منصات مختلفة فإنها تبقى عرضة للتلف.

\bye
