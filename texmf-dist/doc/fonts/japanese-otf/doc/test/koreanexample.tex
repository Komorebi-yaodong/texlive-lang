\documentclass[papersize]{jsarticle}
\usepackage[utf8]{inputenc}
\usepackage[expert, multi]{otf}
\input{otf-hangul.dfu}
\DeclareUnicodeCharacter{5F3A}{\UTF{5F3A}}% 強
\DeclareUnicodeCharacter{654E}{\UTF{654E}}% 教
\DeclareUnicodeCharacter{5FB5}{\UTF{5FB5}}% 徴 jisl34k_uo.dfu: 1-84-36 cjk
\usepackage{palatino}
\renewcommand{\ttdefault}{lmtt}
\usepackage{url}
\pagestyle{empty}

\begin{document}

% 横書き (horizontal writing)
\fbox{\vbox{\hsize=21zw
{\TeX}はスタンフォード大学のクヌース教授によって開発された組版システムであり、組版の美しさと強力なマクロ機能を特徴としている。\par
\bigskip
{\noautoxspacing
{\TeX}은 스탠포드 大學의 크누스 敎授에 의해 開發된 組版 시스템으로, 組版의 美와 强力한 매크로 機能이 特徵이다.\par
}}}

% 縦書き (vertical writing)
\fbox{\vbox{\hsize=21zw \tate\adjustbaseline
{\TeX}はスタンフォード大学のクヌース教授によって開発された組版システムであり、組版の美しさと強力なマクロ機能を特徴としている。\par
\bigskip
{\noautoxspacing
{\TeX}은 스탠포드 大學의 크누스 敎授에 의해 開發된 組版 시스템으로、組版의 美와 强力한 매크로 機能이 特徵이다。
\par}}}

\begin{thebibliography}{9}
 \bibitem{OkumuraCho2004}
         Cho, J.-H. and Okumura, H.:
         Typesetting CJK languages with Omega, \TeX, XML, and Digital Typography,
         Lecture Notes in Computer Science \textbf{3130}, Springer, 2004,
         pp.~139--148. \newline\url{http://project.ktug.or.kr/omega-cjk/cjk-otp/}.
 \bibitem{TsuchimuraKuroki2008}
         Tsuchimura, N. and Kuroki, Y.:
         Development of Japanese \TeX\ environment,
         \textit{The Asian Journal of \TeX}, \textbf{2}~(2008), pp.~53--62.
         \newblock\url{http://ajt.ktug.kr/assets/2008/5/1/0201tsuchimura_kuroki.pdf}.
\end{thebibliography}

\end{document}
