%
% wn-comp.tex
%
%% Cyrillic font container with T2 encoding beta-support
%
% This file is future part of lxfonts package
% Version 3.5 // Patchlevel=0
% (c) O.Lapko
%
% This package is freeware product under conditions similar to
% those of D. E. Knuth specified for the Computer Modern family of fonts.
% In particular, only the authors are entitled to modify this file
% (and all this package as well) and to save it under the same name.
%
% Content:
%
% Driver TeX file for Katalogs of WN and WNCYR fonts (to compare),
%  5 main font shapes, font encoding tables and text test
%
%%%%%%%%%%%%%%%%%%%%%%%%%%%%%%%%%%%%%%%%%%%%%%%%%%%%%%%%%%%%%%%%%%%%%%%%%%%%%%%
%
\vsize10in
\voffset-.5in\eject
\let\noinit!\input testfox
\input testLHtxt
\let\xtables\xtable\def\medskip{\par\kern-8pt}
\def\FontPage#1{\medbreak\vbox\bgroup\def\fontname{#1}\startfont\xtable\egroup\vfill}
\FontPage{wnr10}
\FontPage{wncyr10}
\eject
\FontPage{wnti10}
\FontPage{wncyi10}
\eject
\FontPage{wnbx10}
\FontPage{wncyb10}
\eject
\FontPage{wnss10}
\FontPage{wncyss10}
\eject
\FontPage{wncsc10}
\FontPage{wncysc10}
\eject
\FontPage{wntt10}
\FontPage{cmtt10}
\eject
\FontPage{wntt10}
\FontPage{wncyr10}
\eject
\def\FontPage#1{\medbreak\vbox\bgroup\def\fontname{#1}\strut\startfont\wntext\egroup\vfill}
\FontPage{wnr10}
\FontPage{wncyr10}
\eject
\FontPage{wnti10}
\FontPage{wncyi10}
\eject
\FontPage{wnbx10}
\FontPage{wncyb10}
\eject
\FontPage{wnss10}
\FontPage{wncyss10}
\eject
\FontPage{wncsc10}
\FontPage{wncysc10}
\eject
\FontPage{wntt10}
\eject

\end
