%
% ************ tb26haralambous.tex
\input tugboat.sty
 
%\input mssymb
\font\cyr=mcyr10
%\input cyracc.def
 
\input rgreekmacros
\newcount\m \newcount\n \newcount\p \newdimen\dim
\chardef\other=12
\def\oct#1{\hbox{\tenrm\'{}\kern-.2em\tenit#1\/\kern.05em}} % octal constant
\def\hex#1{\hbox{\tenrm\H{}\tentt#1}} % hexadecimal constant
\def\setdigs#1"#2{\gdef\h{#2}% \h=hex prefix; \0\1=corresponding octal
 \m=\n \divide\m by 64 \xdef\0{\the\m}%
 \multiply\m by-64 \advance\m by\n \divide\m by 8 \xdef\1{\the\m}}
\def\testrow{\setbox0=\hbox{\penalty 1\def\\{\char"\h}%
 \\0\\1\\2\\3\\4\\5\\6\\7\\8\\9\\A\\B\\C\\D\\E\\F%
 \global\p=\lastpenalty}} % \p=1 if none of the characters exist
\def\oddline{\cr
  \noalign{\nointerlineskip}
  \multispan{19}\hrulefill&
  \setbox0=\hbox{\lower 2.3pt\hbox{\hex{\h x}}}\smash{\box0}\cr
  \noalign{\nointerlineskip}}
\newif\ifskipping
\def\evenline{\loop\skippingfalse
 \ifnum\n<128 \m=\n \divide\m 16 \chardef\next=\m
 \expandafter\setdigs\meaning\next \testrow
 \ifnum\p=1 \skippingtrue \fi\fi
 \ifskipping \global\advance\n 16 \repeat
 \ifnum\n=128 \let\next=\endchart\else\let\next=\morechart\fi
 \next}
\def\morechart{\cr\noalign{\hrule}
 \chartline \oddline \m=\1 \advance\m 1 \xdef\1{\the\m}
 \chartline \evenline}
\def\chartline{&\oct{\0\1x}&&\:&&\:&&\:&&\:&&\:&&\:&&\:&&\:&&}
\def\chartstrut{\lower4.5pt\vbox to14pt{}}
\def\table{$$\global\n=0
  \halign to\hsize\bgroup
    \chartstrut##\tabskip0pt plus10pt&
    &\hfil##\hfil&\vrule##\cr
    \lower6.5pt\null
    &&&\oct0&&\oct1&&\oct2&&\oct3&&\oct4&&\oct5&&\oct6&&\oct7&\evenline}
\def\endchart{\cr\noalign{\hrule}
  \raise11.5pt\null&&&\hex 8&&\hex 9&&\hex A&&\hex B&
  &\hex C&&\hex D&&\hex E&&\hex F&\cr\egroup$$\par}
\def\:{\setbox0=\hbox{\char\n}%
  \ifdim\ht0>7.5pt\reposition
  \else\ifdim\dp0>2.5pt\reposition\fi\fi
  \box0\global\advance\n 1 }
\def\reposition{\setbox0=\vbox{\kern2pt\box0}\dim=\dp0
  \advance\dim 2pt \dp0=\dim}
\def\centerlargechars{
  \def\reposition{\setbox0=\hbox{$\vcenter{\kern2pt\box0\kern2pt}$}}}
\greekdelims
\font\ninergr=rgrrg10 scaled 900
 
\TBremark{Pick up fonts for Imagen.  Make fonts for APS}
 
\def\blash{\math\backslash\math}
\let\bs=\blash
\def\begineight{\smallskip\bgroup\eightpoint}
\def\endeight{\par\egroup\smallskip}
%\font\smallrg=rgrrg8         % commented out by rw
\def\QED.{\math\square\mah}
\def\O#1{\hbox{\rm\char'23\kern-.2em\it#1\/\kern.05em}} % octal constant
\def\refersto#1{\math{}^#1\math}
\def\referred#1{\par\hang\noindent#1.}  % overwritten below
\newdimen\thehangindent
\setbox0=\hbox{0. }
\thehangindent=\wd0
\def\referred#1{\par\hangindent\thehangindent\noindent
  \hbox to \thehangindent{\hss#1.\ }\ignorespaces}
 
\def\infoot{\baselineskip9.4pt\eightrm}
\def\@{@}
%\def\.#1{{\def\\{\backslash}\tentex#1}} % commented out by rfw
\font\qq=cmssqi8
\def\mottoline#1{{\rightline{\qq #1}}}
\def\htb{\hskip2pt\relax}
\def\frak{\fam\euffam}
%\def\cyr{\cyracc\tencyr}
 
% ***********************************************************************
 
\title * Typesetting Modern Greek with\\ 128 Character Codes \\ version 1.1*
 
\author * Yannis Haralambous *
\address * U.F.R. de Math\'ematiques\\
          Universit\'e de Lille--Flandres--Artois\\
          59655 Villeneuve d'Ascq Cedex, France *
\netaddress[\network{Bitnet}] * yannis@frcitl81 *
 
\author * Klaus Thull *
\address * Freie Universit\"at Berlin *
\netaddress[\network{Usenet}] * thull@fubinf.uucp *
\TBremark{Check Thull's network address; gateway didn't like "unido".}
\TBremark{this one generally works--kt}
 
\article
 
{\baselineskip11.3pt
\mottoline{``Ved dette v\ae ret,}
\mottoline{n\aa r det regner}
\mottoline{s\aa\ sn\o r det''}
\smallskip
\mottoline{--- Fr\o ydis Frosk}
}
 
\vskip8pt\noindent
In european scripts where diacritical marks are common,
there are (at least) two reasons to avoid \TeX's accent mechanism
in favor of many accented characters.
 
One is the possible misplacement of accents by |dvitype|'s rounding
algorithm;\TBremark{dvitype or a driver?}
\TBremark{really dvitype--kt}
the second is lack or invalidity of hyphenation.  For
example, large portions of german text may be unhyphenatable, and,
given the german inclination to long words, may not be in shape to be
typeset at all.  Thus, in Europe, the obvious thing to do is: let
$\MF$\ put the accents onto the letters, then access these characters
via \TeX's ligature mechanism.
 
Accordingly, the greek fonts created by Silvio Levy\refersto1 have
256 characters each, and are a fine tool to typeset greek texts,
ancient as well as modern, except those containing the most recent
unique accent ``$\monotoniko '{}$'' (see below).  But alas, there is
the commercial world, whose device drivers just cannot do 256-code
fonts (even |.pxl|-fonts were seen on the ``Big-Tech'' sales
exhibition in West Berlin last winter).  The free drivers are in
better shape generally, but often the commercial ones cannot be
disposed of in a hurry.  So we decided to reduce these fonts to 128
characters.  We kept only the ones strictly necessary for writing
modern greek without misusing the |\accent| primitive.  At the same
time, we constructed some new fonts, which we describe below.
 
\head * The Reduced Greek Fonts *
 
In modern post-war greek, the use of the grave accent ``$\grave{}$''
($bare'ia$) progressively faded, so that only two accents and the
breathings were left (this was the kind of greek the first author
learned at school). So the first reduction we did on Levy's fonts was
to omit all grave accents. Secondly, we made $s$ a free character
again, so that in the transliteration one has to type |s| for $s$,
and |c| for $c$.  Thirdly, we omitted the iota subscribed letters.
All these, however, can still be accessed by macros if so wished.
 
Let's recall now the procedure, most of which is due to Levy:  to
typeset greek, you get into ``Greek mode'' by typing |\begingreek|.
Similarly, you get out by typing |\endgreek|, but if you have to do
this often, it is better to type |\greekdelims| at the beginning of
your file. In that case |$| is used to enter and leave greekmode, and
|\math| takes the former meaning of |$| (do not forget to type
``{\tt\bs\SP}'').  The transliteration code is the following:
\display
\tabskip1.1pt
\hbox{\valign
{\hbox to 
8pt{\null\hfil\strut$#$\hfil}&\hbox to 8pt{\null\hfil\strut\tt#\hfil}\cr
a&a\cr b&b\cr g&g\cr d&d\cr e&e\cr z&z\cr h&h\cr j&j\cr
i&i\cr k&k\cr l&l\cr m&m\cr n&n\cr x&x\cr o&o\cr p&p\cr
r&r\cr s&s\cr c&c\cr t&t\cr u&u\cr f&f\cr q&q\cr y&y\cr w&w\cr}}
\enddisplay
To get an acute ($>oxe'ia$), or circumflex ($perispwm'enh$) accent you
type |'|~(single quote), or |~|~(tilde) resp.\ in front of the vowel.
To get a rough ($dase'ia$) or smooth ($yil'h$) breathing you type |<|
or |>|, resp., in front of the vowel or the accent, if there is one.
A diaeresis ($dialutik'a$) is represented by |"|~(double quote), and
for a diaeresis with acute accent, just type |"'|~(double quote,
single quote).  To get a vowel with a iota subscript
($<upogegramm'enh$) you have to use the macro |\I{...}|.
 
If you need one of the omitted accents or combination of accents,
you can get it by a macro:
for example
\list[\unitemized]
$\grave{e}$ by |\grave{e}|,
$\breve{a}$ by |\breve{a}|,
$\macron{h}$ by |\macron{h}|,
$\roughgrave{i}$ by |\roughgrave{i}|,
$\smoothgrave{o}$ by |\smoothgrave{o}|,
$\diaeresisgrave{w}$ by |\diaeresisgrave{w}|,
$\diaeresiscircumflex{u}$ by |\diaeresiscircumflex{u}|,
$\rhorough$ by |\rhorough|, and
$\rhosmooth$ by |\rhosmooth|.
\endlist
 
Finally, you have access to the following punctuation marks:
\display
\hbox{\tabskip1.5pt
\valign
{\hbox to 15pt{\hfil\strut$#$\hfil}&\hbox to 15pt{\hfil\strut\tt#\hfil}\cr
.&.\cr ,&,\cr ;&;\cr :&:\cr !&!\cr ?&?\cr ''&''\cr ((&((\cr
))&))\cr
}}
\enddisplay
as well as the ten digits, parentheses, hyphen, en-dash, slash,
percent sign, asterisk, plus and equal signs.
 
\head * Some New Fonts *
 
To write mathematics in greek one also needs slanted letters (for
statements of theorems, according to |amsppt| style) and small
capital letters (for titles and references). We have constructed
these fonts, in the same reduced way, so that, together with the
reduced Levy fonts, we obtain a complete family of greek fonts,
namely regular, boldface, slanted, and small caps.  We have called
these fonts |rgrrg|, |rgrbf|, |rgrsl|, |rgrsc|.  Inside of greek mode
you just write |\bf|, |\sl|, |\smc| as usual and |\rg|  (instead of
|\rm|) to get the regular gree font.
 
Here is an example of an alleged mathematical text, complete with
translation:
$
\proclaim \smc {1.1.4.} Jewrhma.
Gi'a k'aje jetik'o >ak'eraio
\math n\math,
<up'arqei m'ia tri'ada m'h mhdenik~wn >akera'iwn \math(x,y,z)\math, t'etoia
<~wste\math\math x^n+y^n=z^n.\math\math
 
\noindent{\smc Apodeixh tou 1.1.4.}  Gi'a \math n=2\math, ft'anei n'a p'aroume
\math x=3,y=4,z=5\math. Gi'a \math n\gt 2\math,
>af'hnoume t'hn >ap'odeixh st'on >anagn'wsth s'an >'askhsh.
{\bf <'o.>'e.d.}
$
 
\proclaim \smc {1.1.4.} Theorem.
For each positive integer \math n\math,
there exists a triple of non-zero integers \math (x,y,z)\math\ such that
\math\math x^n+y^n=z^n.\math\math
 
\noindent{\smc Proof of 1.1.4.}  For \math n=2\math, we find
\math x=3,y=4,z=5\math. For \math n\gt 2\math, the proof
left to the reader as an exercise.  {\bf q.e.d.}
 
\head * Fonts for one-accent greek *
 
Some years after the re-establishment of democracy in Greece in 1974,
a new system of accentuation has been introduced, omitting completely
breathings and subscript iota, and simplifying the two remaining
accents into one ``universal accent,'' $\monotoniko ' \polutoniko
(tonik'o shme~io).$ This system is currently taught at school and it
seems that any official document (including written examinations in
some high schools) written in the old fashion multi-accent system is
considered invalid (!).
 
So we thought that perhaps people would like to write in the old
system and have a font to print the same text in fully official
one-accent greek. We created fonts analogous to the reduced regular,
boldface, and slanted which we have denoted by the prefix ``|m|''
(for $monotonik'o$): |mrgrrg|, |mrgrbf|, |mrgrsl|.  Note that the
small capitals font |rgrsc| doesn't have any accents at all, and so
may be used in any accent system.
 
These new fonts are designed to work with the same input as the old
accent system. The printed text will follow the current
grammar\refersto2 (at least concerning the accent), with one
exception:  monosyllables (like articles, prepositions and other
auxiliary words) don't take any accent at all. To solve this problem
we are working at a Pascal word processor program, based on Fred
M.~Liang's packed trie device, which will, once given the list of the
accented monosyllables, recognize them, and replace them by
non-accented words. According to the dictionary of 
H.~Mihiotis\refersto3, there are  284  such words, to which we ust
add many new and foreign words.
 
Of course, you can write your text in one-accent greek right away
(unfortunately there is no ``magic'' macro to transform it back into
multi-accent greek~\dots). With these new fonts you will get a nice
symmetric ``universal'' accent instead of an acute or a circumflex.
 
To write in one-accent greek you get into ``Greek one-accent mode'',
by typing |\beginmgreek|. If you are in greek multi-accent mode
already, you must use the macro |\monotoniko|. There is also the
converse macro |\polutoniko|. So if you want to obtain
\display\vbox{$
\centerline{\monotoniko O Hr'akleitoc 'elege \polutoniko
((t\grave{a} p'anta \rhorough e~i))}
\centerline{\monotoniko kai e'iqe d'ikio! \dots}
$}\enddisplay
you type
\verbatim
\beginmgreek O Hr'akleitoc 'elege
\polutoniko
   ((t\grave{a} p'anta \rhorough e~i))
\monotoniko
   kai e'iqe d'ikio!...\endgreek
\endverbatim
 
\head * The Greek Numeral Symbols *
 
The so-called Ionian\refersto4 system of numeration (\math{\sim}\math
fifth century {\smc bc}) consisted of the following numerals:
\display
\hbox{\valign
{\hbox to 15pt{\hfil\strut$#$\hfil}&\hbox to 15pt{\hfil\strut\rm#\hfil}\cr
A&1\cr B&2\cr G&3\cr D&4\cr E&5\cr {\Digamma}&6\cr Z&7\cr H&8\cr J&9\cr}}
\enddisplay
\display
\hbox{\valign
{\hbox to 15pt{\hfil\strut$#$\hfil}&\hbox to 15pt{\hfil\strut\rm#\hfil}\cr
I&{10}\cr K&{20}\cr L&{30}\cr M&{40}\cr N&{50}\cr X&{60}\cr
O&{70}\cr P&{80}\cr {\Qoppa}&{90}\cr}}
\enddisplay
\display
\hbox{\valign
{\hbox to 20pt{\hfil\strut$#$\hfil}&\hbox to 20pt{\hfil\strut\rm#\hfil}\cr
R&{100}\cr S&{200}\cr T&{300}\cr U&{400}\cr F&{500}\cr Q&{600}\cr Y&{700}\cr
W&{800}\cr {\Sanpi}&{900}\cr}}
\enddisplay
 
The letters $\Digamma, \Qoppa, \Sanpi$ are called digamma, qoppa,
sanpi. They belong to an older alphabet. Later on, as lowercase
letters were introduced and as the need for higher numbers grew, the
numerals became:
\display
\hbox{\tabskip1pt
\valign
{\hbox to 15pt{\hfil\strut$#\overstroke$\hfil}&
     \hbox to 15pt{\hfil\strut\rm#\hfil}\cr
a&1\cr b&2\cr g&3\cr d&4\cr e&5\cr {\digamma}&6\cr z&7\cr h&8\cr
j&9\cr}}
\enddisplay
\display
\hbox{\tabskip1pt
\valign
{\hbox to 15pt{\hfil\strut$#\overstroke$\hfil}&
   \hbox to 15pt{\hfil\strut\rm#\hfil}\cr
i&{10}\cr k&{20}\cr l&{30}\cr m&{40}\cr n&{50}\cr x&{60}\cr
o&{70}\cr p&{80}\cr {\qoppa}&{90}\cr}}
\enddisplay
\display
\hbox{\tabskip1pt
\valign
{\hbox to 20pt{\hfil\strut$#\overstroke$\hfil}&
  \hbox to 20pt{\hfil\strut\rm#\hfil}\cr
r&{100}\cr s&{200}\cr t&{300}\cr u&{400}\cr f&{500}\cr q&{600}\cr y&{700}\cr
w&{800}\cr {\sanpi}&{900}\cr}}
\enddisplay
\display
\hbox{\tabskip2.5pt\valign
{\hbox to 30pt{\hfil\strut$\understroke#$\hfil}&
  \hbox to 30pt{\hfil\strut\rm#000\hfil}\cr
a&1\cr b&2\cr g&3\cr d&4\cr e&5\cr {\digamma}&6\cr z&7\cr}}
\enddisplay
\display
\hbox{\valign
{\hbox to 30pt{\hfil\strut$#$\hfil}&
  \hbox to 30pt{\hfil\strut\rm#000\hfil}\cr
h&8\cr j&9\cr M&{10}\cr}} 
\enddisplay
So, for example, the date {\sl February 16th, 1989} would be written
$i\digamma\overstroke\ Febrouar'iou \understroke a\sanpi pj$ and the
following equality holds:
$
\display\vbox{
\centerline{sz\overstroke\math{} + {}\math ypj\overstroke\ =
 \sanpi\qoppa\digamma\hskip1pt\overstroke.}
}
\enddisplay
$
 
Notice that there is no zero. Zero is, and has always been, the
cardinal of the empty set which in Ancient Greece was not considered
an entity in its own right.
 
To express numbers greater than 10,000 there were many ways. One of
them was to use 10,000 as a base:  thus, for example, 67,536,753
(\math {}= 6753 \mathbin\cdot 10,000 + 6753\math) was written
$M\understroke\digamma yng\math\cdot\math \understroke\digamma yng$.
 
\medskip
\noindent{\smc Exercise}:
  {\sl If $-gwno$ means ``-gon'', which of the following
polygons can by constructed by rule and compasses?}\par
\smallskip
$\centerline{iz\overstroke -gwno, l\digamma\overstroke -gwno, \understroke
dtxj-gwno,}
\centerline{\understroke akd-gwno, Me\overstroke\math\cdot\math\understroke
eflz-gwno, \sanpi\qoppa\digamma\overstroke-gwno.}$
\medskip
 
Let's return now to \TeX: you can obtain these symbols by the following
macros:
|\digamma| for
$\digamma$,
|\vardigamma| for
$\vardigamma$,
|\Digamma| for
$\Digamma$,
|\qoppa| for
$\qoppa$,
|\Qoppa| for
$\Qoppa$,
|\sanpi| for
$\sanpi$, and
|\Sanpi| for
$\Sanpi$.
To get the tick marks which distinguish units and thousands, you can
use |\overstroke| after the numeral, or |\understroke| in front of
the numeral.
 
\head * Symbols for cypriotic greek *
 
The official language of Cyprus is greek. It is also the language
used in the mass-media and at school. But the language actually
spoken is a dialect, derived from byzantine greek (and as it seems,
far more faithful to ancient greek than the one spoken
in Greece). Some literature has been written in the dialect, and
since there are phonemes not available in the greek alphabet,
cypriotic writers use several conventions of new symbols to express
them.
 
In the convention we followed,\refersto5 the symbols $\ssh$, $\SSH$
stand for the sound ``sh'' (like ``shower'' in english, or ``{\cyr
shashka}'' in russian), $\dz$, $\DZ$ stand for ``j'' (like ``jazz''
in english or ``{\cyr dzhungli}'' in russian), $\psh$, $\PSH$ stand
for a $y$ followed by a $\ssh$ (like ``{\cyr pshenitza}'' in russian)
and $\ksh$, $\KSH$ stand for a $k$ followed by a $\ssh$ (like in
hindi ``k\d setriya'').
 
You can get these symbols by the macros |\ssh|, |\SSH|, |\dz|, |\DZ|,
|\psh|, |\PSH|, |\ksh|, |\KSH|.  Here is an example of a small text
using these symbols:
\smallskip
$
\ninergr
\frenchspacing
\parindent0pt
\advance\baselineskip1.0pt
\tolerance=5000
\obeylines
Dki'alexec to'uc xer'orotsouc, dentr'on mou, n'a riz'wseic,
n'a f'aeic t'hn zwo'ullan sou, pott'e s>'on j'a sterk'wseic.
T\dz i >'en kane~i p>'on''xerokagi'ac t\dz i >'en >'e\ssh ei st'axhn %
q~wma
n'a \ssh\ssh iepasto~un o<i r'izec sou, plast~hkan t\dz i >'all''>a\-k'o\-ma.
O<i spalajki'ec t\dz a'i o<i bati'ec p~asin n'a s'e tul'ixoun,
>epki'asan se po'u t'on laim'on, sf'iggoun se n'a s'e pn'i\-xoun.
T\dz i >'an pp'esei mi'a sta\ksh i'a ner'on, >enn'a t'hn p'innoun >'alloi
t\dz i >enn'a skent\dz 'ereic t\dz eiaqama'i, <'enan xerodroump'a\-lin.
>'Ejja xort'wseic n'a \ssh\ssh iaste~ic pott'e to~u <'hliou >amm'a\-ti,
giat''>enn'a s'e qaski'azousin o<i spalajki'ec t\dz i o<i b'atoi.
T\dz i >eso'unh, kakor'izikon, >enn'a doule'ukeic gr'onouc
t\dz a'i n'a gure'ukeic ''pospa\ssh i'an po'u to'uc kako'uc geit'o\-nouc.
T\dz i o<i k'opoi e>ic t''>an'ajjeman. M'agkoumou m'en prok'a\-meic\dots
t\dz i >eso'unh t'o qa"'irin mou, ftwq'on mou, >'enna k'ameic.
$
\rightline{\sevenrm From Ilias Georgiou's ``Geloklaman"}
 
\smallskip
 
\head * On hyphenation *
 
There is still no greek hyphenation list, so one has to use
hyphenation from other languages. We have compared on an ordinary
text, the standard english (Liang), the german (Schwarz) and
portuguese (Rezende)\refersto6 hyphenation patterns.  The results
were surprising: on 267 possible hyphens, these three patterns missed
199 (!!), 141 and 149 resp, found 46, 115, 111 correct ones and 22,
11, 7 bad ones (the portuguese mistakes were less embarrassing than
the german ones). So, for a temporary substitute, we would choose
either the german or the portuguese patterns.
 
And since you will be forced to make corrections by hand, here are the
complete actual rules of greek hyphenation:
 
{\sl Let \math c_1,c_2,\dots,c_n\math\ be consonants \math (n\geq2)\math\
and \math v_1,v_2,v_3,v_4\math\ vowels. Then we have}
\list[\parindent0pt\leftskip0pt\sl][\def\tagform#1{{\smc #1\enspace}}]
 
\item[\tag{Rule 1.}] The combination \math v_1c_1v_2\math\ is
separated as \math v_1-c_1v_2\math\ {\rm(ex. $pa-ra-ka-l~w$)}
 
\item[\tag{Rule 2.}] The combination \math v_1c_1\dots c_nv_2\math\
is separated as \math v_1-c_1\dots c_nv_2\math\ {\bf if }there
is a greek word starting with \math c_1c_2\math\ $(l'a-sph,
ko-fte-r'oc),$ {\bf else }\math v_1c_1-c_2\dots c_nv_2\math\
$(j'ar-roc, >eq-jr'oc)$
 
\item[\tag{Rule 3.}]  The combinations of vowels \math v_1v_2,
v_1v_2v_3\math\ or \math v_1v_2v_3v_4\math\ are not to be separated
if they are pronounced as one phoneme $(>ah-d'o-ni$ but
$>a-'ht-th-toc$, $pi'o$ but $p'u-o).$
\endlist
 
S.~Levy made in his fonts separate characters of all possible accented
letters, to prevent problems of hyphenation (\TeX\ doesn't yet
hyphenate words with accents). The only exceptions he made, were the
two combinations $ \roughgrave{}, \smoothgrave{}$ which occur only on
monosyllables.
 
In our case, to be able to reduce the fonts, we were forced to make
accents also of $\char\rq134$, $\char\rq100$, $"{}$, $"'{}$, $\I{}$
(and macros of $\grave{}$, $\smoothgrave{}$, $\roughgrave{}$,
$\breve{}$, $\macron{}$, $\diaeresisgrave{}$,
$\diaeresiscircumflex{}$, $\rhorough$, $\rhosmooth$ as already
mentioned). This of course adds problems to hyphenation.
Nevertheless, $\char\rq134$, $\char\rq100$ occur on one- and
2-syllable words only, $"{}$, $"'{}$ occur rarely and $\I{}$ depends
on the kind of language one is writing (to find it, you have to go
back to older versions of $kajare'uousa$ as for example in the
following lovely text:
 
\medskip
 
{\tolerance=5000
$ \advance\baselineskip1pt >En\I{~w} <esp'eran tin\grave{a}
>exantl'hsac t\grave{a} murol'ogi'a tou >ekoim~ato <o Froum'entioc
>ep\grave{i} t~hc >'ammou t~hc paral'iac, katab\grave{a}c >ex
o>uran~wn <o >ap'ostoloc >eke~inoc t~wn Sa\-x'o\-nwn >'hnoixe
di\grave{a} maqa'irac t\grave{a} st'hjh to~u koimwm'enou, e>is'hgage
to\grave{u}c <iero\grave{u}c dakt'ulouc tou e>ic \def\hhh{\grave{h}}
t\hhh n >op\hhh n ka\grave{i} >exag\grave{w}n t\hhh n kard'ian
>eb'ujisen a>ut'hn e>ic l'akkon pl'hrh <'udatoc, <'oper <hg'iasen
prohgoum'enwc.  <H fl'egousa >eke'inh kard'ia >'efrixen e>ic
t\grave{o} <'udwr <wc smar\grave{i}c
\def\ooo{\grave{o}}%
\def\aaa{\grave{a}}%
\def\uuu{\grave{u}}%
\def\iii{\grave{i}}%
>ent\ooo c to~u thgan'iou, >afo~u d\grave{e} >ekr'uwsen,
>'ejese p'alin a>ut\hhh n <o <'agioc e>ic t\ooo n t'opon thc ka\iii\
kle'isac t\hhh n plhg\hhh n >ep'estreyen e>ic t\ooo n >idik'on tou.
 
>'Etuq'e pote, >anagn~wst'a mou, n\aaa\ >apokoimhj\I{~h}c m\grave{e}
>anup'oforon b~hqa, koim'wmenoc n\aaa\ <idr'ws\I{h}c ka\iii\ >exu\-pn'h\-sac
n\aaa\ e<urej\I{~h}c >iatreum'enoc? >Agno~wn <'oti e>~isai kal\aaa\
>ano'igeic mhqanik~wc t\ooo\ st'oma, <'ina plh\-r'w\-s\I{h}c e>ic t\ooo n
>epikat'araton b~hqa t\ooo n sun'hjh f'oron. >All\aaa\ p'oshn a>isj'anesai
qar'an, m\hhh\ e<ur'iskwn e>ic t\ooo n l'a\-rugga t\ooo\ >oqlhr\ooo n
jhr'ion! O<'utw <'ama >'hnoixe ka\iii\ <o Frou\-m'entioc to\uuu c
>ofjalmo'uc, <htoim'asjh n\aaa\ prosf'e\-r\I{h} e>ic t\hhh n >aq'a\-riston
>Iw'annan t\hhh n sun'hjh dakr'uwn spo\-nd'hn, >all\aaa\ par\aaa\ p~asan
prosdok'ian o<i >ofjalmo'i tou e<u\-r'e\-jhsan xhro\iii\ ka\iii\ n\aaa\
progeumat'is\I{h} m~allon \smoothgrave{h} n\aaa\ kla'us\I{h}
\I{>h}sj'a\-ne\-to >'orexin met\aaa\ polu'hmeron nhste'i\-an <o kal\ooo c
Be\-ne\-de\-kt~i\-noc.
$
}
 
\rightline{\sevenrm From Emmanouil~Ro\"\i dis' ``The Popess Johanna (1866)''}
\medskip
 
Note that in one-accent greek, all accented letters are represented
by separate characters in the code table, so that no hyphenation
problem arises.
 
$
\gdef\tttt{
Kaj'otan mprost'a thc ka'i t'hn k'uttaze. Noi'w\-jo\-ntac mi'a >ap'eranth
 e>uqar'isthsh n'a t'hn bl'epei >'etsi mprost'a tou ka'i <'ena >ap'eranto
 >ani\-ka\-no\-po'i\-h\-to po'u d'en mpo\-ro~u\-se n'a t'hn
  tra\-b'h\-xei st'hn >agka\-li'a tou ka'i n'a t'hn fi\-l'h\-sei
   >eke~i st'on laim'o
  po'u t'hn e>~iqe fi\-l'h\-sei t'hn m'ia ka'i mo\-na\-di\-k'h
   for'a ka'i e>~iqe
  noi'w\-sei m'esa tou t'hn pi'o >'omor\-fh stig\-m'h to~u ka\-lo\-kai\-rio~u
   ki <'oti
  t'o ka\-lo\-ka'i\-ri a>u\-t'o, ft'a\-no\-ntac st'o >apo\-ko\-r'u\-fw\-m'a
   tou, e>~iqe ki'o\-lac pe\-r'a\-sei. >All'a d'en e>~iqe >ak'oma
   pe\-r'a\-sei tele'iwc, giat'i >eke'inh bris\-k'o\-tan t'wra mpro\-st'a
   tou. Ka'i d'en mpo\-ro~u\-se n'a k'anei t'i\-po\-ta.
}
$
 
 
\head * Samples, Tables, and Remarks *
 
\head * The font {\tt rgrrg10} *
$\tttt$
 
\head * The font {\tt rgrbf10} *
$\bf\tttt$
 
\head * The font {\tt rgrsl10} *
$\sl\tttt$
 
\figure[\top]
\raggedcenter
Layout for fonts |rgrrg|, |rgrbf|, |rgrsl|
$\table$
\endfigure
 
$\monotoniko
\gdef\ttttt
{
Kaj'otan mprost'a thc kai thn k'uttaze. Noi'w\-jo\-ntac mia >ap'eranth
 e>uqar'isthsh na thn bl'epei >'etsi mprost'a tou kai <'ena >ap'eranto
 >ani\-ka\-no\-po'i\-h\-to pou den mpo\-ro~u\-se na thn
  tra\-b'h\-xei sthn >agka\-li'a tou kai na thn fi\-l'h\-sei
   >eke~i ston laim'o
  pou thn e>~iqe fi\-l'h\-sei thn m'ia kai mo\-na\-di\-k'h
   for'a kai e>~iqe
  noi'w\-sei m'esa tou thn pio >'omor\-fh stig\-m'h tou ka\-lo\-kai\-rio~u
   ki <'oti
  to ka\-lo\-ka'i\-ri a>u\-t'o, ft'a\-no\-ntac sto >apo\-ko\-r'u\-fw\-m'a
   tou, e>~iqe ki'o\-lac pe\-r'a\-sei. >All'a den e>~iqe >ak'oma
   pe\-r'a\-sei tele'iwc, giat'i >eke'inh bris\-k'o\-tan t'wra mpro\-st'a
   tou. Kai den mpo\-ro~u\-se na k'anei t'i\-po\-ta.
}
$
 
 
\figure[\bot]
\raggedcenter
Layout for fonts |mrgrrg|, |mrgrbf|, |mrgrsl|
$\monotoniko\table$
\endfigure
 
\head * The font {\tt mrgrrg10} *
$\monotoniko\ttttt$
 
\head * The font {\tt mrgrbf10} *
$\monotoniko\bf\ttttt$
 
\head * The font {\tt mrgrsl10} *
$\monotoniko\sl\ttttt$
 
 
\figure[\top]
\raggedcenter
Layout for font |rgrsc|
$\smc\table$
\endfigure
 
 
We conclude with the following remark: people writing french, czech,
turkish or other languages with many diacritical marks complain that
there is no space left in Computer Modern to incorporate
already-accented letters. The solution (in the case of French) that
D\'esarm\'enien\refersto7  proposed, was to replace greek uppercase
letters by the french \'e, \`e, \^ e, \^ o, \^\i, \^ a, \^ u, \`a,
\`u.  But then the question is: where to put the greek uppercase
letters, which are necessary for mathematical formulas. We answer: if
you have the greek |rgr|  family of fonts, you already have all kinds
of greek uppercase letters. Just take them from there!  Of course,
math families must be restructured in that case since math family~7
cannot be used for those letters anymore.  As Gariepy\refersto8
pointed out already, the inconvenience with this solution is that for
every language with accents you will need another |cm| family of
fonts. That's why we still believe that the best once and for all
solution would be to be able to work with fonts of 256 characters.
 
 
\head * Improvements and changes -- version 1.1  (as of March 4, 1990) *
 
(by the first author)\medskip
 
\noindent An hyphenation list has been added. 
Greek grammar has changed very often in
the last 20 years, mainly because greek is a very manifold language (some
people find it {\it chaotic}, I find it {\it beautiful\/}). I followed the
hyphenation rules of \refersto2\ as presented in section ``On hyphenation''
of this paper. These rules
are very simple and hold as well for the multi-accent as for the one-accent
system. I found exactly 1168 patterns; many of them came simply from the 
following fact: since |'|,|~|,|>|,|<| are of category 11 during hyphenation, 
|<'o-so| could occur. This leaves $<'o$ alone, which is rather ugly. 
 
Italics fonts (|rgrti| and |mrgrti|) have been added. 
Since in the |cm| family,
|cmti| has almost the same lowercase letters as |cmmi| and the same uppercase
ones as |cmsl|, I tried to do the same. Some changes were necessary: 1) the
handwritten modern greek alpha looks more like {\it a} and 2) tau's stem has a
hook. There is still some kerning to do. I call this version {\it
``experimental''}. Maybe in some later (1.2?) version there will be an entirely
new italics font.
 
In the |rgrsc| font (as you can see in the table) the symbols $\rhorough,
\rhosmooth, \Digamma$ and $\vardigamma$ exist now in all styles: regular,
slanted, italic and boldface. \TeX\ will know which one to choose, depending on
the font you are using (inside |greekmode|). If you add some new style (for
example typewriter or sans-serif) 1) please let me know, 2) add these four
letters in |rgrsc|, 3) add the information in |greekmacros.tex|.
 
There is |\smallDigamma| for the small capital digamma $\smallDigamma$. Some
people still persist in calling it {\it ``ef''}. Also there are two new
symbols: 
 
1) following an advice of my father, I added a variation $\varqoppa$
of lowercase qoppa. This symbol is used in our days for the numeral 90 (the
current year is $\understroke a\sanpi\varqoppa$). The macro for $\varqoppa$ is
|\varqoppa|. 
 
2) a funny ``upside-down'' iota with circumflex $\inviota$ (the macro is
|\inviota|). This symbol was used in the last century for the sound of $i$ in
$gi'a$ (cf. ``$pi'o$'' in hyphenation {\smc rule 3}). Here is an example 
of such a text, taken from a 1907 edition\refersto9:
 
\let\w=\grave
$
{\ninergr
\frenchspacing
\parindent0pt
\advance\baselineskip1.0pt
\tolerance=5000
\obeylines
Kair\w oc f'ernei t\w a l'aqana, kair\w oc t\w a parapo'ul\inviota a, 
M\w e t\w on kair\w o ka\w i t\w o dendr\w i k'anei karp\w o ka\w i f'ulla. 
>'A"inte, >or'e, m\inviota\w a kopan\inviota\w a s>'an n>~atan qulop'hta, 
N\w a ''p\inviota\I{~h}c t\w hn Gk'olfw j'arreuec? 
D\w en e>~in'' mall\inviota\w a t\w a g'ene\inviota a.
J'el''<h >ag'ap'' >apomon\w h ki''>ahto~u grhgorws'unh.
J'elei lago~u pat'hmata, >allo\inviota ~wc <uge\inviota\w a s''>af'inei.
X'ereic t'i magar'oplash po~u e>~inai \I{<h} guna~ikec?
$
\rightline{\sevenrm From Spyridon Peresiadis's ``I Golfo"}
}
 
\smallskip
 
Finally I inserted the logos $\grMF$ and $\MF$ in the |rgrsc| font (the macros
are |\grMF| and the usual |\MF|). There is still no PASCAL word processor
program to detect monosyllabes, volunteers are welcome. This upgrading has been
done on a Macintosh SE/30 using Andrew Trevorrow's Oz\TeX\ and Victor
Ostromoukhov's Mac$\MF$.
 
 
\setbox0=\hbox{22. }
\thehangindent=\wd0
\def\referred#1{\par\hangindent\thehangindent\noindent
  \hbox to \thehangindent{\hss#1.\ }\ignorespaces}
\raggedright
 
\head * References *
 
\referred1 {\smc S.Levi}: Using Greek Fonts with \TeX, {\sl TUGboat},
{\bf 9} (1988) 20--24
%\referred2 $\monotoniko {\smc M.Triantafullidh:} Grammatik'h
\referred2 $\monotoniko {\smc M.Triantafullidh:}
{\rg Neoellhnik'h Grammatik'h, }
{\sl Organism'oc 'Ek\-do\-shc Sqolik'wn Bibl'iwn},
Aj'hna 1982$
\referred3 {$\smc Q.Mhqiwth:$} $Ne'wtaton Lexik'on t~hc Neoellhnik~hc Gl'wsshc,
{\sl >Ek\-d'o\-seic Kastal'ia}, >Aj~hnai 1972$
\referred4 {\smc C.B.Boyer}: A History of Mathematics, {J.\ Wiley \&
Sons,} New~York 1968
\referred5 ${\smc H.Gewrgiou:} Gel'oklam'an, {\sl Seir'a Kupriak~hc
La"ik~hc Po'ihshc <Upourge'iou paide'iac\/} {\bf 4}, Leukws'ia 1980$
\referred6 {\smc Unix} \TeX\ distribution tape, Seattle 1988
\referred7 {\smc J.D\'esarm\'enien:} How to run \TeX\ in a French
environment: hyphenation, fonts, typography, {\sl TUGboat}, {\bf 5} (1984)
91--102
\referred8 {\smc A.Gariepy:} French in \TeX, {\sl TUGboat}, {\bf 9}
(1988) 65--69
\referred9 {$\smc S.Peresiadhc:$} $<H Gk'olfw. Dr'ama e>idulliak\w on e>ic
pr'axeic p'ente, {\sl >Ekdotik\w oc O>~ikoc Gewrg'iou F'exh,} >Aj~hnai 1907$
 
 
\makesignature
\endarticle
 
%  version 19.6.1989
 
\hbox{Yannis Haralambous}
\hbox{Universit\'e de Lille--Flandres--Artois}
\smallskip
\hbox{Klaus Thull}
\hbox{Freie Universit\"at Berlin}
-------
 
