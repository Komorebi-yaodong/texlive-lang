\useoptex % We are using OpTeX, no LaTeX

\fontfam[pagella]
\report

\emergencystretch=2em
\hbadness=2100
\def\thed#1{\ifnum#1<10 0\fi\the#1}

\let\isAleBsaved=\_isAleB
\def\preprocessindex{%  sort \_iilist with the rule:  .word is before :word is before word
   \def\_isAleB ##1##2{%
      \edef\tmpb{\csstring ##1&\relax\csstring ##2&\relax}%
      \_ea\testAleB \tmpb
   }
   \ifx\_iilist\empty \else \_dosorting\_iilist \fi
   \let\_isAleB=\isAleBsaved
}
\def\testAleB #1#2#3\relax #4#5#6\relax{%
   \ifnum `#2<`#5 \_AleBtrue \else \_AleBfalse \fi
}

\catcode`<=13
\def<#1>{{\def\_dsp { }\iindex{:#1}\iis {:#1} {$\langle\hbox{\it#1}\rangle$}}%
   $\def\,{\hskip1.5pt plus.3pt minus1.2pt}\,\langle\hbox{\it#1}\rangle\,$}
\def\l#1>{$\def\,{\hskip1.5pt plus.3pt minus1.2pt}\,\langle\hbox{\it#1}\rangle\,$}
\everyintt={\catcode`<=13 \Blue}
\onlyrgb
\addto\_titfont\Blue
\def\Green{\setrgbcolor{0 .8 0}}

\def\myunderline#1{\vtop{\hbox{#1}\kern-\prevdepth \kern2pt \hrule}}
\adef/#1.{\myunderline{#1}} 
\adef|#1;{{\Red#1}}
\toksapp\everyintt{\catcode`\/=13 \catcode`\|=13}
\catcode`\/=12 \catcode`\|=12
\ifx\_partokenset\undefined
   \def\comment#1\par{{\adef\%{\%}\Green\%#1}\par}
\else
   \def\comment#1\_par{{\adef\%{\%}\Green\%#1}\_par}
\fi
\catcode`\%=14
\toksapp\everytt{\adef\%{\comment}}

\def\*{*}
\def\c#1{$_{#1}$}

\activettchar`
\enquotes

% to index macros:
\def\i #1 {\makedest{#1}\ii .#1 \iis .#1 {\ilink[cs:#1]{\code{\\#1}}}}
\def\x`{\bgroup\_setverb\xx}
\bgroup \lccode\string`\.=\string`\` \lowercase{\egroup \def\xx #1#2.{\i #2 \egroup `#1#2.}}
\def\y`{\bgroup\_setverb\yy}
\def\yy #1#2={\i #2 \egroup `#1#2=}
\def\z`{`\let<=\l}
\let\_cslinkcolor=\Blue

% tex-nutshell.pdf includes destinations to the explanation of the primitive
% control sequences and plain TeX macros in the form: "cs:sequence". For example,
% you can try:
%
%    http://petr.olsak.net/ftp/olsak/optex/tex-nutshell.pdf#cs:hbox
%
% All such sequences are listed in the tex-nutshell.eref file. You can read
% this eref file into your document and create external links to these
% destinations.

\newwrite \eref
\immediate\openout\eref=\jobname.eref
\def\makedest#1{}
\def\makedestactive#1{%
   \ifcsname cs:#1\endcsname \else
      \immediate\write\eref{\string\Xeref{#1}}%
      \dest[cs:#1]%
      \sxdef{cs:#1}{}%
   \fi
}
\def\noda{\def\makedest##1{}}
\def\doda{\let\makedest\makedestactive}

% Hyperlinks
\hyperlinks\Red\Green
\fnotelinks\Magenta\Magenta



\tit \TeX/ in a Nutshell

\author Petr Olšák

The pure \TeX/ features are described here, no features provided by
macro extensions. Only the last section gives a summary of plain \TeX/
macros.

The main goal of this document is its brevity. So features are described
only roughly and sometimes inaccurately here. If you need to know more then you can read
free available books, for example 
\ulink[https://eijkhout.net/texbytopic/texbytopic.html]{\TeX/ by topic} 
or
\ulink[http://petr.olsak.net/tbn.html]{\TeX/book naruby}.
Try to type `texdoc texbytopic` in your system.

\ii OpTeX The \ulink[http://petr.olsak.net/optex]{\OpTeX/} manual 
supposes that the user already knows the basic principles of \TeX/ itself. 
If you are converting from \LaTeX/ to \OpTeX/ for example\fnote
{Congratulations on your decision:-)}
then you may welcome a summary
document that presents these basic principles because \LaTeX/ manuals
typically don't distinguish between \TeX/ features and features specially
implemented by \LaTeX/ macros. 

I would like to express my special thanks to Barbara Beeton who read my text
very carefully and suggested hundreds of language corrections and improvements
and also discovered many of my real mistakes. Thanks to her, my text is
better. But if there are any other mistakes then they are only mine and I'll be
pleased if you send me a bug report in such case.  

\notoc\nonum\sec Table of contents

\maketoc \outlines0


\sec[termi] Terminology

The main principle of \TeX/ is that its input files can be a mix
of the material which could be printed and \ii control~sequence {\em control sequences}
which give a setting for built-in algorithms of \TeX/ or give a special message
to \TeX/ what to do with the inputted material.

Each control sequence (typically a word prefixed by a backslash) has its 
\ii meaning~of~control~sequence {\em meaning}. 
There are four types of meanings of control sequences:

\begitems
* the control sequence can be a \ii register {\em register}; this means it represents a variable which
  is able to keep a value. There are \ii primitive/register {\em primitive registers}. Their values influence
  behavior of built-in algorithms (e.g., \i hsize `\hsize`, \i parindent `\parindent`,
  \i hyphenpenalty `\hyphenpenalty`). On the other hand \ii declared/register {\em declared registers} 
  are used by macros (e.g., \i medskipamount `\medskipamount` used in plain \TeX/ 
  or {\doda\i ttindent `\ttindent`}
  used by \ii OpTeX \OpTeX/).
* the control sequence can be a \ii primitive/command {\em primitive command}, which runs a built-in
  algorithm (e.g., \i def `\def` declares a macro, \i halign `\halign` runs the algorithm for
  tables, \i hbox `\hbox` creates a box in typesetting output).
* the control sequence can be a \ii character/constant {\em character constant} 
  (declared by \i chardef `\chardef`
  or \i mathchardef `\mathchardef` primitive command) or a font selector (declared
  by \i font `\font` primitive command).
* the control sequence can be a \ii macro {\em macro}. When it is read, it is
  replaced by its \ii replacement/text {\em replacement text} in the input queue. If there are more
  macros in the replacement text, all macros are replaced. This is called the 
  \ii expansion/process {\em expansion process} which ends when only printable text,
  primitive commands (listed in section~\ref[main]), 
  registers (section~\ref[reg]), character constants, or font selectors remain.
\enditems

Example. When \TeX/ reads:
\begtt
\def\TeX{T\kern-.1667em\lower.5ex\hbox{E}\kern-.125emX}
\endtt
in a macro file, then the `\def` primitive command saves the information that
{\doda\i TeX `\TeX`} is a control sequence with meaning \"macro", the replacement text is
declared here, and it is a mix of a material to be typeset: `T`, `E` and `X` 
and primitive commands 
\i kern `\kern`, \i lower `\lower`, \i hbox `\hbox` 
with their parameters in given syntax. Each
primitive command has a declared syntax; for example, `\kern` must be followed
by a dimension specification in the format \"decimal number followed by a unit".
More about this primitive syntax is in sections~\ref[reg], \ref[expand] and~\ref[main].

When a control sequence `\TeX` with meaning \"macro" occurs in the
input stream, then it is \ii expansion {\em expanded} to its replacement text, i.e.\ the sequence of
typesetting material and primitive commands. The `\TeX` macro
expands to `T\kern-.1667em\lower.5ex\hbox{E}\kern-.125emX` and 
the logo \TeX/ is printed as a result of this processing.

None of the control sequences have their definitive meaning. A control sequence
could change its meaning by re-defining it as a new macro (using `\def`), 
redeclaring it as an arbitrary object in \TeX/ (using \i let `\let`), etc.
When you re-define a primitive control sequence then the access to its value
or built-in algorithm is lost. This is a reason why \ii OpTeX \OpTeX/ macros duplicate
all primitive sequences (\i hbox `\hbox` and `\_hbox`) with the same meaning and use
only \"private" control sequences (prefixed by `_`). So, a user can re-define 
`\hbox` without the loss of the primitive command `\_hbox`.

\sec Formats, engines

\TeX/ is able to start without any macros preloaded in the so-called \ii ini-TeX/state {\em ini-\TeX/
state} (the `-ini` option on the command line must be used). It already knows
only primitive registers and primitive commands at this state.\fnote
{Roughly speaking, if you know all these primitive objects (about 300 in
classical \TeX/, 700 in Lua\TeX/) and the syntax of all these
primitive commands and all the built-in algorithms, then you know all about \TeX.
But starting to produce ordinary documents from this primitive level without
macro support is nearly impossible.}
When ini-\TeX/ reads macro files then new control sequences are
declared as macros, declared registers, character constants or font
selectors. The primitive command \i dump `\dump` saves the binary image of the \TeX/ memory
(with newly declared control sequences) to the \ii format,format/file {\em format file} (`.fmt`
extension). 

The original intention of existing format files was to prepare a
collection of macro declarations and register settings, to load default fonts, and
to dump this information to a file for later use. Such a collection typically
declares macros for the markup of documents and for typesetting design. This is
the reason why we call these files {\em format files}: they give a format
of documents on the output side and declare markup rules for document source files.

When \TeX/ is started without the `-ini` option, it tries to load a prepared
format file into its memory and to continue with reading more macros or a real
document (or both). The starting point is at the place where \i dump `\dump` was
processed during the ini-\TeX/ state. If the format file is not specified
explicitly (by `-fmt` option on the command line) then \TeX/ tries to read the
format file with the same name which is used for running \TeX. For example
`tex document` runs \TeX, it loads the format `tex.fmt` and reads the
`document.tex`. Or `latex document` runs \TeX, it loads the format `latex.fmt`
and reads the `document.tex`. 

The `tex.fmt` is the format file dumped when \ii plain~TeX/macros {\em plain \TeX/ macros}\fnote
{Plain \TeX/ macros were made by \iindex {Knuth,/Donald}Donald Knuth, the author of \TeX. It is a
set of basic macros and settings which is used (more or less) as a subset of
all other macro packages.} 
were read, and `latex.fmt` is the format file dumped when \ii LaTeX/macros {\em \LaTeX/ macros} were read.
This is typically done when a \TeX/ distribution is installed without any user
intervention. So, the user can run `tex document` or `latex document` without
worry that these typical format files exist.

From this point of view, \LaTeX/ is nothing more than a format of \TeX/,
i.e.~a collection of macro declarations and register settings.

A typical \TeX/ distribution has four common \ii TeX/engines {\em \TeX/ engines}, i.e.~programs.
They implement classical \TeX/ algorithms with various extensions: 

\begitems
* \TeX/ -- only classical \TeX/ algorithms by Donald Knuth, 
* \ii pdfTeX pdf\TeX/ -- an extension supporting PDF output directly and
             micro-typographical features,
* \ii XeTeX \XeTeX/ -- an extension supporting Unicode and PDF output, 
* \ii luaTeX Lua\TeX/ -- an extension supporting Lua programming, Unicode,
            micro-typographical features and PDF output.
\enditems

Each of them is able to run in ini-\TeX/ state or with a format file. For
example the command `luatex -ini macros.ini` starts Lua\TeX/ at ini-\TeX/ state,
reads the `macros.ini` file and the final `\dump` command is supposed here to create
a format `macros.fmt`. Then a user can use the command `luatex -fmt macros document` to
load `macros.fmt` and process the `document.tex`.
Or the command `luatex document` processes Lua\TeX/ with `document.tex` and 
with `luatex.fmt` which is a little extension of plain \TeX/ macros. Another example:
`lualatex document` runs Lua\TeX/ with `lualatex.fmt`. It is a format with
\LaTeX/ macros for Lua\TeX/ engine. Final example: 
`optex document` runs Lua\TeX/ with `optex.fmt` which is
a format with \ii OpTeX \ulink[http://petr.olsak.net/optex]{\OpTeX/ macros}.

\sec Searching data

If \TeX/ needs to read something from the file system (for example the
primitive command \i input `\input<file name>` or 
\i font `\font<font selector>=<file name>` is used) 
then the rule
\"first wins" is applied. \TeX/ looks at the current directory first or
somewhere in the \TeX/ installation second. The behavior in the second step
depends on the used \TeX/ distribution. For example 
\ii TeXlive \ulink[https://www.tug.org/texlive]{\TeX/live} programs are
linked with a \ii kpathsea {\em kpathsea} library and they do the following: Search for the given
file in the current directory, then in the \ii texmf~tree \code{~/texmf} tree (data are saved by
the user here), then in the `texmf-local` tree (data are saved by the system administrator
here; they are not removed when the \TeX/ distribution is upgraded), 
then in `texmf-var` tree (data are saved automatically by programs from the
\TeX/ distribution here), and then in the `texmf-dist` tree (data from the \TeX live
distribution). Each directory tree can be divided into sub-trees: first level
`tex`, `fonts`, `doc`, etc.; the second level is divided by \TeX/ engines or font types, etc.;
more levels are typically organized to keep clarity.
New files in the current directory or in the \code{~/texmf} tree are found without 
doing anything more, but new files in other places have to be registered by the `texhash`
program (\TeX/ distributions do this automatically during their installation).


\sec Processing the input

The lines from input files are first transformed by the \ii tokenizer {\em tokenizer}.
It reads input lines and generates a sequence of tokens. These are the
main goals of the tokenizer:

\begitems
* It converts each control sequence to a single token characterized by its name.
* Other input material is tokenized as \"one token per character".
* A continuous sequence of multiple spaces is transformed into one space token.
* The end of the line is transformed into a space token, so that paragraph text
  can continue on the next input line and one space token is added between the last word
  on the previous line and the first word on the next line.
* The comment character `%` is ignored and all the text after it to the end of line
  is ignored too. No space is generated at the end of this line.
* Spaces from the begining of each line are ignored. Thus, you can
  use arbitrary indentation in your source file without changing the result.
* Each empty line (or line with only spaces) is transformed to the token
  \i par `\par`. This token has primitive meaning: \"finalize the current paragraph".
  This implies the general rule in \TeX/ source files: paragraphs are terminated 
  by empty lines. 
\enditems

The behavior of the tokenizer is not definitive. The tokenizer works with a table of
category codes. Any change of category codes of characters
(done by the primitive command \i catcode {\let\,=\relax`\catcode`\code{`}`\<character>=<code>`}) influences tokenizer processing. For
example, the verbatim environment is declared using setting all characters to
normal meaning.

By default, there are the following characters with special meaning. The tokenizer
converts them or sets them as special tokens used in syntactic rules in
\TeX/ later. The corresponding category codes are mentioned here as an index of the
character.

\begitems 
* `\`\c0\quad -- starts completion of a control sequence by the tokenizer.
* `{`\c1 and `}`\c2\quad -- open and close group or have special syntactic meaning.
  The main syntactic rule is: each subsequence of tokens treated by macros
  or primitive commands must have these pairs of tokens balanced. There is no
  exception. The tokenizer treats them as special tokens with meaning
  \"opening character\c1" and \"closing character\c2".
* `%`\c{14}\quad -- comment character, removed by the tokenizer, along with
  everything that follows it on the line.
* `$`\c3, `&`\c4, `#`\c6, `^`\c7, `_`\c8, `~`\c{13}\quad -- 
  tokenizer treats them as a special tokens with meaning:
  \ii math/mode/selector \"math-mode selector\c3", 
  \ii table/separator \"table separator\c4", \ii parameter/prefix \"parameter prefix for
  macros\c6", \ii superscript/prefix \"superscript prefix in math\c7", 
  \ii subscript/prefix \"subscript prefix in math\c8",
  \ii active/character \"active character\c{13}" 
  (the active character `~` is defined as no-breakable space in all typical formats).
* Letters and other characters are tokenized as \"letter character\c{11}" or
  \"other character\c{12}".
\enditems

If you need to print these special characters you can use 
\ii -percent,-at,-dollar,-hash
`\%`, `\&`, `\$`, `\#` or `\_`. These five control sequences are declared as
\"print this character" in all typical \TeX/ formats.
Another possibility is to use a verbatim environment (it depends on the used format).
Last alternative: you can use \i csstring {\let\,=\relax`\csstring\<character>`} 
in Lua\TeX/, because it has
the primitive command `\csstring` which converts `\<character>`
to `<character>`\c{12}.

The \ii active/character \"active character\c{13}"
can be declared by \i catcode `\catcode`\code{`}`\<character>=13`. 
Such a `<character>` behaves like a control sequence. For example, you can
define it by
\i def `\def<character>{...}` and use this `<character>` as a macro. 
If the term `<control sequence>` is used in syntactical rules in this document
then it means a real control sequence or an active character.

Each control sequence is built by the tokenizer starting from `\`\c0. Its name is
a continuous sequence of letters\c{11} finalized by the first non-letter. 
Note that \OpTeX/ sets
`_` as letter\c{11}, thus control sequence names can include this character.
\LaTeX/ sets the `@` as letter\c{11} when reading styles and macro files.
You can look to such files and you will see many such characters inside
private control sequence names declared by \LaTeX/ macros.

If the first character after `\`\c0 is non-letter (i.e.\ `<something>`\c{\ne11}), then the control sequence is
finalized with only this character in its name. So called 
\ii one~character/control~sequence {\em one-character control sequence} is created. Other control sequences 
are \ii multiletter/control~sequence {\em multiletter control sequences}.

Spaces {\char9251}\c{10} after multi-letter control sequences are ignored, so the space can be used
as a terminating character of the control sequence. Other characters used
immediately after a control sequence are not ignored. So `\TeX !` and
`\TeX!` gives the same result: the control sequence `\TeX` followed
immediately by `!`\c{12}.

The tokenizer's output (a sequence of tokens) goes to the \ii expand/processor {\em expand processor}
and its output goes to the \ii main/processor {\em main processor} of \TeX. The expand processor
performs expansions of macros or a primitive command which is working at the expand
processor level. See a summary of such commands in section~\ref[expand]. 

The main processor performs assignment of registers, declares macros by the \i def `\def`
primitive command, and runs all primitive commands at the main processor level.
Moreover, it creates the typesetting output as described in the next
section.

The very important difference between \TeX/ and other programs is that there are no
strings, only sequences of tokens. We can return to the example
`\def\TeX{...}` above in section~\ref[termi]. 
The token \i def `\def` is a control sequence with meaning \"declare a macro".
It gets the following token \i TeX `\TeX` and
declares it as a macro with replacement text, which is the sequence of tokens:

\medskip
\bgroup \def\b#1{\kern1pt\lower2ex\llap{\c{#1}\kern-1.5pt}}
   \def`#1`#2,{\frame{\strut\tt\Blue\string#1}\b{#2}}\catcode`\{=12 \catcode`\}=12 \indent
`T`11, `\kern`, `-`12, `.`12, `1`12, `6`12, `6`12, `7`12, `e`11, `m`11,
`\lower`, `.`12, `5`12, `e`11, `x`11, `\hbox`, `{`1, `E`11, `}`2, 
`\kern`, `-`12, `.`12, `1`12, `2`12, `5`12, `e`11, `m`11, `X`11,
\egroup
\medskip

If you are thinking like \TeX/ then you must forget the term \"string"
because all texts in \TeX/ are preprocessed by the tokenizer when input lines are
read and only sequences of tokens are manipulated inside \TeX/.

The tokenizer converts two `^`\c7`^`\c7 characters followed by an ASCII uppercase
letter to the Ctrl-letter ASCII code. For example `^^M` is Ctrl-M (carriage
return). It converts two `^`\c7`^`\c7 followed by two hexadecimal digits 
(`0123456789abcdef`) to a one-byte code, for example, `^^0d` is Ctrl-M too 
because it has code 13. Moreover, the tokenizer of \XeTeX/ or Lua\TeX/ converts
`^`\c7`^`\c7`^`\c7`^`\c7 followed by four hexadecimal digits or 
`^`\c7`^`\c7`^`\c7`^`\c7`^`\c7`^`\c7 followed by six hexadecimal digits to one
character with a given Unicode.


\sec Vertical and horizontal modes

When the main processor creates the typesetting output, it alternates
between vertical and horizontal mode. It starts in \ii vertical/mode,@ {\em vertical mode}: all
materials are put vertically below in this mode. For example
\i hbox `\hbox{a}\hbox{b}\hbox{c}` creates a above b above c in vertical mode.

If something is incompatible with the vertical mode principle --- a special
command working only in horizontal mode or a character itself ---
then the main processor switches to \ii horizontal/mode,@ {\em horizontal mode}: it opens 
an unlimited horizontal data row for
typesetting material and puts material next to each other. For example
\i hbox `\hbox{a}\hbox{b}\hbox{c}` creates abc in horizontal mode.

When an empty line is scanned, the tokenizer creates a \i par `\par` token here and
if the main processor is in horizontal mode, the `\par` command finalizes the
paragraph. More exactly it returns to vertical mode, 
it breaks the horizontal data row filled in previous horizontal mode 
to parts with the \i hsize `\hsize` width. These parts are completed as 
\ii box {\em boxes} and they are put one below another in vertical mode. So, a
paragraph of \i hsize `\hsize` width is created.

Repeatedly: if there is something incompatible with the current vertical mode
(typically a character), then the horizontal mode is opened and all characters
(and spaces between them) are put to the horizontal data row. When an empty line
is scanned, then the `\par` command is started and the horizontal 
data row is broken into lines of \i hsize `\hsize` width and the next paragraph 
is completed.

In vertical mode, the material is accumulated in a vertical data column
called the \ii main/vertical/list {\em main vertical list}.
If the height of this material is greater than \i vsize `\vsize` then its part with
maximum `\vsize` height is completed as a \ii page/box {\em page box} and shipped to the
\ii output/routine {\em output routine}. A programmer or designer can declare a design of pages
using macros in the output routine: header, footer, pagination, 
the position of the main page box, etc. The output routine completes the main page 
box with other material declared in the output routine and the result is
shipped out as one page of the document. The main processor continues in
vertical mode with the rest of the unused material in the main vertical list.
Then it can switch to horizontal mode if a character occurs, etc...

The plain \TeX/ macro \i bye `\bye` (or primitive command \i end `\end`\fnote
{\LaTeX/ format re-defines this primitive control sequence `\end` to another
meaning which follows the logic of \LaTeX/'s markup rules.}) 
starts the last
`\par` command, finalizes the last paragraph (if any), completes the last page
box, sends it to the output routine, finalizes the last page in it, and \TeX/
is terminated.

There are \ii internal/vertical/mode {\em internal vertical mode} and 
\ii internal/horizontal/mode {\em internal horizontal mode}.
They are activated when the main processor is typesetting material 
inside \i vbox `\vbox{...}` or \i hbox `\hbox{...}` primitive commands. More about boxes is in
sections~\ref[boxes] and~\ref[main]. 

Understanding of switching between modes is very important for \TeX/ users.
There are primitive commands which are context dependent on the current mode.
For example, the \i par `\par` primitive command (generated by an empty line) does nothing
in vertical mode but it finalizes paragraph in horizontal mode and it causes
an error in math mode. Or the \i kern `\kern` primitive command creates a vertical space in
vertical mode or horizontal space in horizontal mode.

The following primitive commands used in vertical mode start
horizontal mode: the first character of a paragraph (most common
situation) or \i indent `\indent`, \i noindent `\noindent`, \i hskip `\hskip` 
(and its alternatives),
\i vrule `\vrule` and the plain \TeX/ macro 
\i leavevmode `\leavevmode`\fnote {The list is not exhaustive, but most important commands
are mentioned.}.
When horizontal mode is opened, an indentation of \i parindent `\parindent` width 
is included. The exception is only if horizontal mode is started by
\i noindent `\noindent`; then the paragraph has no indentation.

The following primitive commands used in horizontal mode finalize the
paragraph and return to vertical mode: \i par `\par`, \i vskip `\vskip` (and its
alternatives), \i hrule `\hrule`, \i end `\end` and the plain \TeX/ macro \i bye `\bye`.

\sec Groups in \TeX/

Each assignment to registers, declaration macros or font selecting is local
in groups. When the current group ends then the assignments made inside the
group are forgotten and the values in effect before this group was opened are restored.
The groups can be delimited by `{`\c1 and `}`\c2 pair or by \i begingroup `\begingroup` and 
\i endgroup `\endgroup` primitive commands or by 
\i bgroup `\bgroup` and \i egroup `\egroup` control
sequences declared by plain \TeX. 
For example, plain \TeX/ declares the macros {\doda\i rm `\rm` (selects roman font),
\i bf `\bf` (selects bold font) and \i it `\it`} (selects italics) and it initializes by
\i rm `\rm` font. A user can write:

\begtt
The roman font is here {\it here is italics} and the roman font continues.
\endtt
%
Not only fonts but all registers are set locally inside a group. The macro
designer can declare a special environment with font selection and with
more special typographical parameters in groups.

The following example is a test of understanding vertical and horizontal
modes switching. 

\begtt
{\hsize=5cm This is the first paragraph which should be formatted 
            to 5\,cm width.}

But it is not true...
\endtt
%
Why does the example above not create the paragraph with a 5\,cm width?
The empty line \i par (`\par` command) is placed {\em after} the group is finished, so the
\i hsize `\hsize` parameter has its previous value at the time when the paragraph is
completed, not the value 5\,cm. The value of the \i hsize `\hsize` register\fnote
{and about twenty other registers which declare the paragraph design}
is used when the paragraph is
completed, not at the beginning of the paragraph. This is the reason why
macro programmers explicitly put a \i par `\par` command into macros before the local
environment is finished by the end of the group. Our example should look like this:

\begtt
{\hsize=5cm This is the first ... to 5\,cm width.\par}
\endtt


\sec[boxes] Box, kern, penalty, glue

You can look at one character, say the `y`. It is represented by three
dimensions: \ii height height (above baseline), \ii depth depth (below baseline) 
and \ii width width.
Suppose that there are more characters printed in horizontal mode and
completed as a line of a paragraph. This line has its height equal to
the maximum height of characters inside it, it has the depth equal to maximum
depth of all characters inside it and it has its width. Such a sequence of
characters encapsulated as one typesetting element
with its height, depth and width is
called a \ii box {\em box}. Boxes are placed next to each other (from left to
right\fnote{There is an exception for special languages.})
in horizontal mode or one below another in vertical mode. 

The boxes can include individual characters or spaces or boxes. The boxes can
include more boxes. Paragraph lines are boxes. The page box includes paragraph
lines (boxes). The finalized page with a header, page box, pagination,
etc., is a box and it is shipped out to the PDF page. Understanding boxes is
necessary for macro programmers and designers.

You can create an individual box by the primitive command
\i hbox `\hbox{<horizontal material>}`
or \i vbox `\vbox{<vertical material>}`. The `<horizontal material>` is completed in
internal horizontal mode and `<vertical material>` in internal vertical
mode. Both cases open a group, create the material in a specified
mode and close the group, where all settings are local.

The `<horizontal material>` can include individual characters, boxes,
horizontal \ii glue {\em glues} or \ii kern {\em kerns}. \"Glue" is a special term for
stretchable or shrinkable and possibly breakable spaces and \"kern" is a 
term used for fixed nonbreakable spaces.

The `<vertical material>` can include boxes, vertical glues or kerns. No
individual characters. If you put an individual character in
vertical mode (for example in a \i vbox `\vbox`)
then horizontal mode is opened. At the end of a \i vbox `\vbox`\fnote 
{before the \i vbox `\vbox` group is closed}
or when the \i par `\par` command is invoked, the opened paragraph is finished (with
current \i hsize `\hsize` width) and the resulting lines are vertically placed
inside the \i vbox `\vbox`. 

The completed boxes are unbreakable and they are treated as a single object in
the surrounding printed material. 

The line boxes of a paragraph have the fixed width \i hsize `\hsize`, so there must
be something stretchable or shrinkable in order to get the desired fixed width
of lines. Typically the spaces between words have this feature.\fnote {When
the microtypographical 
feature \i pdfadjustspacing `\pdfadjustspacing` is activated, then not
only spaces are stretchable and shrinkable but individual characters are
slightly deformed (by an invisible amount) too.} These spaces have declared
their \ii default/size/of~space {\em default size}, their 
\ii stretchability {\em stretchability} and their 
\ii shrinkability {\em shrinkability} in the font metric data of the currently used font.

You can place such glue explicitly by the primitive command `\hskip`:

\begtt \catcode`<=13
\hskip <default size> plus<stretchability> minus<shrinkability>
for example:
\hskip 10pt plus5pt minus2.5pt
\endtt
%
This example places the glue with 10\,pt default size, stretchable to
15\,pt\fnote
{It can be stretchable ad absurdum (more than 15\,pt) but with very
considerable \ii badness {\em badness} calculated by \TeX/ whenever glues are stretched 
or shrunk.}
and shrinkable to 7.5\,pt as its minimal size.
All glues in one line are stretched or shrunk equally but with weights given
from their stretchability/shrinkability values.

You can do experiments of this feature if you say \i hbox `\hbox to<size>{...}`.
Then the \i hbox `\hbox` is created with a given width. Probably, the glues inside
this \i hbox `\hbox` must be stretched or shrunk. You can see in the log
that the total \ii badness {\em badness} is calculated, it represents the amount of a 
\"force" used for all glue included in such an \i hbox `\hbox`.

An infinitely stretchable (to an arbitrary positive value) or shrinkable 
(to an arbitrary negative value) glue
can exist. This glue is stretched/shrunk and other glues with
finite amounts of stretching or shrinking keep their default size in such
case.
You can put infinitely stretchable/\penalty0shrinkable 
\ii glue glue using the reserved unit \ii fil `fil`
in an \i hskip `\hskip` command, for example
the command \i hskip `\hskip 0pt plus 1fil` means zero default size but infinitely stretchable.
There is a shortcut for such glue: \i hfil `\hfil`. When you type
\z`\hbox to\hsize{\hfil <text>\hfil}` then the \z`<text>` is centered.
But if the \z`<text>` is wider than \i hsize `\hsize` then \TeX/ reports an 
\ii overfull/box `overfull \hbox`. If you want to center a wide \z`<text>` too, you can use 
\i hss `\hss` instead of \i hfil `\hfil`. The \i hss `\hss` primitive command is equal to
`\hskip 0pt plus1fil minus1fil`. The \z`<text>` printed by
\z`\hbox to\hsize{\hss<text>\hss}` is now centered in its arbitrary size.

A glue created with \ii fill `fill` stretchability or shrinkability (double ell)
is infinitely more stretchable or shrinkable than glues with only a \ii fil `fil` unit.
So, glues with \ii fill `fill` are stretched or shrunk and glues with only `fil` in
the same box keep their default size. For example, a macro declares centering
a \z`<text>` by 
\i hbox \z`\hbox to\hsize{\hss <text>\hss}` and a user can create the \z`<text>` in the form
\i hfill \z`\hfill <real text>`. Then \z`<real text>` is printed flushed right because
\i hfill `\hfill` is a shortcut to \i hskip `\hskip0pt plus1fill` and has greater priority than
glues with only a \ii fil `fil` unit. 

Common usage is \i hbox \z`\hbox to0pt{<text>\hss}` or \z`\hbox to0pt{\hss<text>}`.
The box with zero width is created and the text overlaps the adjacent text to the
right (first example) or to the left (second example). \ii plain~TeX Plain \TeX/ declares
macros for these cases: \i rlap \z`\rlap{<text>}` or \i llap \z`\llap{<text>}`.

The last line of each paragraph is finalized by a glue of type \i hfil `\hfil` by
default. When you write \i hfill `\hfill <object>` in vertical mode (`<object>` is
something like a table, image or whatever else in the box) then `<object>` is
flushed right, because the paragraph is started by the `\hfill` space but
finalized only by \i hfil `\hfil` space. If you type 
\i noindent `\noindent\hfil <object>` then
the <object> is centered. And putting only `<object>` places it to the left
side because the common left side is the default placement rule in vertical
mode.

The same principles that apply to horizontal glues are also applicable to vertical
modes where glues are created by \i vskip `\vskip` commands instead of `\hskip`
commands. You can
write \i vbox `\vbox to<size>{...}` and do experiments.

When the
paragraph breaking algorithm decides about the suitable breakpoints for creating
lines with the desired width \i hsize `\hsize`, then each glue is a potentially breakable
point. Each glue can be preceded by a \ii penalty {\em penalty} value (created by the
\i penalty `\penalty` primitive) in the typical range
$-10000$ to $10000$. The paragraph breaking algorithm gets a penalty if it decides
to break line at the glue preceded by the given penalty value. If no penalty
is declared for a given glue, then it is the same as a penalty equal to zero.\fnote
{More precisely: the paragraph breaking algorithm or page breaking algorithm
can break horizontal list to lines (or vertical list to pages)
at penalties (then it gets the given penalty) or at glues (then the penalty
is zero). The second case is possible only if no penalty nor glue precedes. 
The item where the list is broken (penalty or glue), is discarded and all 
immediatelly followed glues, penalties and kerns are discarded too. They are called 
\ii discardable/item {\em discardable items}}.
The penalty value $10000$ or more
means \"impossible to break". A negative penalty means a bonus for the paragraph
breaking algorithm. The penalty $-10000$ or less means \"you must break here".

The paragraph breaking algorithm tries to find an optimum of breakpoint
positions concerning to all penalties, to all badnesses of all created lines and to
many more values not mentioned here in this brief document.
The analogous optimal breakpoint is found in vertical material when \TeX/
breaks it into pages.

The concept \"box, penalty, glue" with the optimum-fit breaking
algorithms makes \TeX/ unique among many other typesetting software.


\sec Syntactic rules

A primitive command can get its parameters written after it. These
parameters must suit syntactic rules given for
each primitive command. Some parameters are optional. For example 
\i hskip `\hskip<dimen> /plus<stretchability>. /minus<shrinkability>.` means that
the parameter `<dimen>` must follow (it must suit syntactic rules for
dimensions, see section \ref[reg]) then the optional parameter prefixed by keyword 
\ii plus `plus` can follow
and then the optional parameter prefixed by \ii minus `minus` can follow.
We denote the optional parameters by underline in this document.
{\emergencystretch=2em\par}

\ii keyword {\em Keywords} (typically prefixes to some parameters) may have optional spaces
around them.

The explicit expressions of numbers (i.e.\ `75`, `"4B`, \code{`K}; see
section~\ref[reg]) should be terminated by one optional
space which is not printed. This space can serve as a termination character
which says that \"whole number is presented here; no more digits are expected". 

If the syntactic rule mentions the pair `{`, `}` then these characters are
not definitive: other characters may be tokenized with this special meaning but it
is not common. The text between this pair must be \ii balanced/text {\em balanced} with respect to this
pair. For example the syntactic rule \i message \z`\message{<text>}` supposes that 
\z`<text>` must not be `ab{cd`, but `ab{c{}}d` is allowed for instance.

By default, all parameters read by primitive commands are got from the input
stream, tokenized and fully expanded by the expand processor. But
sometimes, when \TeX/ reads parameters for a primitive command, the expand
processor is deactivated. We denote these parameters by red color. For
example, \i let `\let|<control sequence>=<token>;` means that these parameters
processed by the `\let` command are not expanded.

Whenever a syntactic rule mentions the \ii equal/sign `=` character (see the previous example
with the \i let `\let` command), then this is the equal sign tokenized as a normal character
and it is optional. The syntactic rule allows to omit it. Optional spaces
are allowed around this equal sign.

The concept of the optional parameters of primitive commands 
(terminated if something different from
the keyword follows) may bring trouble if a macro programmer
forgets to terminate an incomplete parameter text by the \x`\relax` command
(`\relax` does nothing but it can terminate a list of optional parameters of
the previous command). 
Suppose, for example, that `\mycoolspace` is defined 
by `\def\mycoolspace{\penalty42\hskip2mm}`. If a user writes
`first\mycoolspace plus second` then \TeX/ reports the error
`missing` `number,` `treated` `as` `zero` in the position of `s` character
and appends: \code{<to be read again> s}. A user who is unfamiliar with
\TeX/ primitive commands and their parameters is totally lost. The correct
definition looks like: `\def\mycoolspace{\penalty42\hskip2mm\relax}`.

\sec[def] Principles of macros

Macros can be declared by the \i def `\def` primitive command (or `\edef`, `\gdef`,
`\xdef` commands; see below). The syntax is
`\def|<control sequence>/<parameters>.{<replacement text>};`.

The `<parameters>` are a sequence of formal parameters of the declared macro
written in the form `#1`, `#2`, etc.
They must be numbered from one and incremented by one.
The maximum number of declared parameters is nine. 
These parameters can be used in the `<replacement text>`. This specifies the
place where the real parameter is positioned when the macro is expanded. For
example:

\begtt
\def\test #1{here is "#1".}
\test A        % expands to: here is "A".
\def\swap #1#2{#2#1}
\swap AB       % expands to: BA
\test {param}  % expands to: here is "param".
\swap A{param} % expands to: paramA
\endtt
%
Note that there are two possibilities of how to write real macro
parameters when a macro is in use. The parameter is one token by default but
if there is `{<something>}` then the parameter is `<something>`. The braces
here are delimiters for the real parameter (no \TeX/ group is opened/closed
here).

The example above shows a declaration of \ii unseparated/parameter,@ {\em unseparated parameters}. The
parameters were declared by `#1` or `#1#2` with no text appended to such
a declaration. But there is another possibility. 
Each formal parameter can have a text appended in its declaration, so
the general syntax of the declaration of formal parameters is
\z`#1/<text1>.#2/<text2>.` etc. If such \z`<text>` is appended then we say that
the parameter is \ii separated/parameter,@ {\em separated} or 
\ii delimited/parameter,@ {\em delimited} by text.
The same delimiter must be used when the macro is in use. For example

\begtt
\def\Test #1#2..#3 {first "#1", second "#2", third "#3".}
\Test ABC..DEF G % expands to: first "A", second "BC", third "DEF".
                 % the letter G follows after expansion.
\endtt
%
In the example above the `#1` parameter is unseparated (one token is read as
a real parameter if the syntax \z`{<parameter>}` is not used). The `#2` parameter
is delimited by two dots and the `#3` parameter is delimited by space.

There may be a \z`<text0>` immediately before `#1` in the parameter
declaration. This means that the declared macro must be used with the same \z`<text0>`
immediately appended. If not, \TeX/ reports the error.
The general rule for declaration of a macro with three parameters should be:
\i def
\z`\def|<control sequence>/<text0>.#1/<text1>.#2/<text2>.#3/<text3>.{<replacement text>};`.

The rule \"everything must be balanced" is applied to separated parameters
too. It means that `\Test AB{C..DEF G}.. H` from the example above reads
`B{C..DEF G}` to the `#2` parameter and the `#3` parameter is empty 
because the space (the delimiter of `#3` parameter) immediately follows two dots.

The separated parameter can bring a potential problem if the user forgets the
delimiter or the delimiter is specified incorrectly. Then \TeX/ reports an
error. This error is reported when the first \i par `\par` is scanned as part of the
parameter (probably generated from an empty line). If you really want to scan
as part of the parameter more paragraphs including `\par` between them, then you can
use the \i long `\long` prefix before `\def`. For example `\long\def\scan#1\stop{...}`
reads the parameter of the `\scan` macro up to the `\stop` control sequence,
and this parameter can include more paragraphs.
If the delimiter is missing when a \i long `\long` defined macro is processed, then
\TeX/ reports an error at the end of the file.

When a real parameter of a macro is scanned then the expand processor is deactivated.
When the `<replacement text>` is processed then the expand processor works
normally. This means that if parameters are used in the `<replacement text>`, then
they are expanded here.

If a macro declaration is used inside the `<replacement text>` of another macro
then the number of `#` must be doubled for inner declaration. Example:

\begtt
\def\defmacro#1#2{%
   \def#1##1 ##2 {##1 says: #1 ##2.}%
}
\defmacro \hello   {hello} % expands to \def\hello#1 #2 {#1 says: hello #2.}
\defmacro \goodbye {good bye}
\hello   Jane Eric         % expands to: Jane says: hello Eric.
\goodbye Eric John         % expands to: Eric says: good bye John.
\endtt

The exact implementation of the feature above: when \TeX/ reads macro body
(during `\def`, `\edef`, `\gdef`, `\xdef`) then each double `#`\c6  is converted 
to single `#`\c6 and each (unconverted yet) single `#`\c6 followed by a digit is
converted to an internal mark of future parameter. This mark is replaced
by real prameter when the defined macro is used. This rule of conversion of
macro body has one exception: `\edef{...\the\toks...}` keeps the toks
content unexpanded and without conversion of hashes. And there exists 
a reverse conversion from internal marks to~`#`\c{12}<number> and 
from `#`\c6 to `#`\c{12}`#`\c{12} 
when \TeX/ writes macro body by \i meaning `\meaning` primitive.

Note the `%` characters used in the `\defmacro` definition in the exmample above. They mask
the end of lines. If you don't use them, then the space tokens are included here (generated
by the tokenizer at the end of each line). The `<replacement text>` of `\defmacro` will be
`<space>\def#1...{...}<space>` in such a case. Each usage of `\defmacro` 
generates two unwanted spaces. It is not a problem if `\defmacro` is used in
the vertical mode because spaces are ignored in this mode. But 
if `\defmacro` is used in horizontal mode then
these spaces are printed.\fnote
{More precisely, they are transformed into horizontal glues used between words.}

The macro declaration behaves as another assignment, so the information about
such a declaration is lost if it is used in a group and the group is left.
But you can use a \i global `\global` prefix before \i def `\def` or the primitive
\i gdef `\gdef`. Then the assignment is global regardless of groups.

When `\def` or `\gdef` is processed then `<replacement text>` is read with the
deactivated expand processor. We have alternatives \i edef `\edef` (expanded def)
and \i xdef `\xdef` (global expanded def) which read their `<replacement text>`
expanded by the expand processor. The summary of `\def` syntax is:

\begtt \catcode`\|=13 \catcode`\/=13 \catcode`\<=13 \Blue 
\def|<control sequence>/<parameters>.{<replacement text>};  % local assignment
\gdef|<control sequence>/<parameters>.{<replacement text>}; % global assignment
\edef|<control sequence>/<parameters>.{;<replacement text>} % local assignment
\xdef|<control sequence>/<parameters>.{;<replacement text>} % global assignment
\endtt   

If you set \i tracingmacros `\tracingmacros=2`, you can see in the log file how the macros are expanded.

\sec Math modes

The `$`\c3`<math text>$`\c3 specifies a math formula inside a line of the
paragraph. It processes the `<math text>` in a group and in 
\ii internal/math/mode,math/mode/internal {\em internal math mode}. 
The `$`\c3`$`\c3`<math text>$`\c3`$`\c3 generates a separate line with math 
formula(s). It processes the `<math text>` in a group and in 
\ii display/math/mode,math/mode/display {\em display math mode}. 

The fonts in math mode are selected in a very specific manner which is independent 
of the current text font. Six different math objects are automatically detected
in math mode: \x`\mathord` (normal material), \x`\mathop` (big operators),
\x`\mathbin` (binary operators), \x`\mathrel` (relations), \x`\mathopen` (open
brackets), \x`\mathclose` (close brackets), \x`\mathpunct` (punctuation). They
can be processed in four styles \x`\displaystyle` (default in the display mode),
\x`\textstyle` (default in the internal math mode), \x`\scriptstyle` (used for
indexes or exponents, smaller text) and \x`\scriptscriptstyle` 
(used in indexes of indexes, even smaller text).

The math typesetting algorithms were implemented in \TeX/ by its author with great care.
All typographical traditions of math typesetting were taken into account.
There are three chapters about math typesetting in his \TeX/book. Moreover,
there is the detailed appendix G containing the exact specification of generating math
formulae. This topic is unfortunately out of the scope of this short
text.

There is a good a piece of news: all formats (including \LaTeX/) take the default \TeX/
syntax for `<math text>`. So, \LaTeX/ manuals or \LaTeX/ documents
serve a good source if you want to get to know the rules of math typesetting
by \TeX. There is only one significant difference. Fractions are constructed at
the primitive level by the \x`\over` primitive:
`{<numerator>\over<denominator>}` but \LaTeX/ uses a macro {\doda\x`\frac`} in the
syntax `\frac{<numerator>}{<denominator>}`. Plain \TeX/ users (including the
author of \TeX/) prefer the syntax which follows the
principle \"how a human reads the formula". On the other hand, the 
\x`\frac` syntax is derived from machine languages. You can define the
\x`\frac` macro by  `\def\frac#1#2{{#1\over#2}}` if you want.


\sec[reg] Registers

\ii register There are four types of registers used in \TeX:

\begitems \def\!{\kern-1pt}
* \ii counter/type/register {\em Counters}; their values are integer numbers. Counters are declared by 
  \i newcount `\newcount|<register>;`\fnote
  {The declarators \x`\newcount`, \x`\newdimen`, \x`\newskip` and \x`\newtoks`
   are plain \TeX/ macros used in all known \TeX/ formats. They provide
   `<address>` allocation and use the  
   `\count<address>`\!, `\dimen<address>`\!, `\skip<address>` and `\toks<address>` 
   \TeX/ registers. The \x`\countdef`, \x`\dimendef`, \x`\skipdef` and \x`\toksdef` 
   primitive commands are used internally.}
   or they are primitive registers (\x`\linepenalty` for example).
   \TeX/ interprets primitive commands which represent an integer from
   an internal table as counter type register too 
   (examples: \x`\catcode`\code{`A}, \x`\lccode`\code{`A}).
* \ii dimen/type/register {\em Dimen type}; their values are dimensions. They are declared by
  \i newdimen `\newdimen|<register>;` or they are primitive registers (\x`\hsize`, for example).
   \TeX/ interprets primitive commands which represent a dimension
   value as dimen type register too (example: \i wd `\wd0`).
* \ii glue/type/register {\em Glue type}; their values are triples like in general 
  \x`\hskip` parameters.
  They can be declared by \i newskip `\newskip|<register>;` or they are primitive
  registers (\x`\abovedisplayskip` for example).\fnote
  {Very similar {muglue type} for math glues exists too 
   but it is not described in this text.}
* \ii token/type/register {\em Token lists}; their values are sequences of tokens. They are
  declared by \i newtoks `\newtoks|<register>;` or they are primitive registers
  (\x`\everypar` for example). 
\enditems

The following example shows how registers are declared, how a value is
saved to the register, and how to print the value of the register. 

\tmpnum=42 \tmpdim=-13cm \skip0=10mm plus 12mm minus1fil \toks0={abCd ef}
\begtt
\newcount  \mynumber
\newdimen  \mydimen
\newskip   \myskip
\newtoks   \mytoks
\mynumber = 42
\mydimen = -13cm
\myskip = 10mm plus 12mm minus1fil
\mytoks = {abCd ef}
To print these values use the primitive command "the":
\the\mynumber, \the\mydimen, \the\myskip, \the\mytoks.
\bye
\endtt
%
This example prints: To print these values use the primitive command "the":
\the\tmpnum, \the\tmpdim, \the\skip0, \the\toks0. Note that the human
readable dimensions are converted to typographical points~(pt).

The general syntactic rule for storing values to registers is 
`<register>=<value>` where the equal sign is optional and it can be surrounded by
optional spaces. Syntactic rules for each type of `<value>` depending on
type of the register (i.e.\ `<number>`, `<dimen>`, `<skip>` and `<toks>`) follows.

\begitems \let\_aboveliskip=\relax 
* The `<number>` could be 
\begitems \style N
* a register of counter type;
* a character constant declared by \x`\chardef` or \x`\mathchardef` primitive command. 
* an integer decimal number (with optional `+` or `-` prefixed)
* {\let\,=\relax `"<hexa number>`} where `<hexa number>` can include digits
  `0123456789ABCDEF`\,;
* {\let\,=\relax`'<octal number>`} where `<octal number>` can include digits 
  `01234567`\,;
* {\let\,=\relax \code{`}`<character>`} (the prefix is the reverse single quote \code{`}). 
  It returns the code of the `<character>`. Examples:
  \code{`}{`|A;`} or one-character control sequence \code{`}{`|\A;`}).
  Both examples represent the number 65. The Unicode of the character 
  is taken here if Lua\TeX/ or \XeTeX/ is used;
* \i numexpr `\numexpr<num. expression>`.\fnote
  {This is a feature of the $\varepsilon$\TeX/ extension. It is implemented in pdf\TeX, \XeTeX/ and Lua\TeX.}
  The `<num. expression>` uses operators `+`, `-`, `*` and \code{/} and
  brackets `(`, `)` in normal sense. The operands are `<number>`s. It
  is terminated by something incompatible with
  the syntactic rule of `<num. expression>` or by `\relax`.
  The `\relax` (if it is used as a separator) is removed.
  If the result is non-integer, then it is rounded (not truncated).
\enditems
  The rules 3)--6) can be terminated by one optional space.
* The `<dimen>` could be
\begitems \style N
* a register of dimen type or counter type;
* a decimal number with an optional decimal point (and optional `+` or `-`
  prefixed) followed by `<dimen unit>`. The `<dimen unit>` is \ii pt `pt` (point)\fnote 
  {1\,pt = 1/72.27\,in $\doteq$ 0.35\,mm\,;\ 1\,pc = 12\,pt\,;\
   1\,bp = 1/72\,in\,;\ 1\,dd $\doteq$ 1.07\,pt\,;\ 1\,cc = 12\,dd\,;\
   1\,sp = $2^{-16}$\,pt = \TeX/ accuracy.}
  or \ii mm `mm` or \ii cm `cm` or \ii in `in` or 
  \ii bp `bp` (big point) or \ii dd `dd` (Didot point) or \ii pc `pc` (pica) or 
  \ii cc `cc` (cicero) or \ii sp `sp` (scaled point) or \ii em `em` (quad of current font) or
  \ii ex `ex` (ex~height of current font) or a register of dimen type;  
* \i dimexpr `\dimexpr<dimen expression>`.
  The `<dimen expression>` uses operators `+`, `-`, `*` and \code{/} and
  brackets `(`, `)` in their normal sense. The operands of `+` and `-` are
  `<dimen>`s, the operators of `*` or \code{/} are the pair `<dimen>` and
  `<number>` (in this order). The `<dimen expression>`
  is terminated by something incompatible with the
  syntactic rule of `<dimen expression>` or by `\relax`. The `\relax` (if it
  is used as a separator) is removed.
\enditems
  The rule 2) can be terminated by one optional space.
* The `<skip>` could be:
\begitems \style -
* a register of glue type or dimen type or counter type;
* `<dimen>/plus<generalized dimen>. /minus<generalized dimen>.`. The 
  `<generalized dimen>` is the same as `<dimen>`, but normal `<dimen unit>` 
   or pseudo-unit `fil` or `fill` or `filll` can be used.
\enditems
* The `<toks>` could be
\begitems \style -
* `/<expandafters>.`\z`{|<text>};`. The `<expandafters>` is typically a sequence of
  `\expandafter` primitive commands (zero or more). The \z`<text>` is
  scanned without expansion but the exception can be given by 
  `<expandafters>`.
\enditems
\enditems

\removelastskip
The main processor reads input tokens (from the output of activated or
deactivated expand processor) in two contexts: 
\ii do/something/context,context/do/something {\em do something} or {\em
read parameters}. By default it is in the context {\em do something}. When a
primitive which allows parameters is read, the main processor reads the
parameters in the context \ii read/parameters/context,context/read/parameters {\em read parameters}.

Whenever the main processor reads a register in the context
{\em do something} it assumes that an assignment of a value to the register
is declared here. The following text (equal sign and `<value>`) is read in the context
{\em read parameters}. If the following text isn't
compliant to the appropriate syntactic rule, \TeX/ reports an error.

Examples of register manipulations:

\begtt
\newcount\mynumber \newdimen\mydimen \newdimen\myskip
\hsize = .7\hsize  % see the rule for <dimen>, unit could be a register 
\hoffset = \dimexpr 10mm - (\parindent + 1in) \relax % usage of \dimexpr
\myskip = 10pt plus15pt minus 3pt
\mydimen = \myskip    % the information "plus15pt minus 3pt" is lost
\mynumber = \mydimen  % \mynumber = 10*2^16  because \mydimen = 10*2^16 sp
\endtt
%
Each dimension is saved internally as an integer multiple of the `sp` unit in
\TeX. When we need a conversion `<dimen>` $\to$ `<number>`, then simply the
internal unit `sp` is omitted.

The summary of most commonly used primitive registers including their default
value given by plain \TeX/ follows.

\let\makedest=\makedestactive

\begitems  \rightskip=0pt plus1fil
* \y`\hsize=6.5in`,
  \y`\vsize=8.9in`
  are paragraph width and page height.

* \y`\hoffset=0pt`,
  \y`\voffset=0pt` give left margin and top margin of the page. They are
  calculated from the \ii page/origin {\em page origin} which is defined by coordinates
  \y`\pdfvorigin=1in` and \y`\pdfhorigin=1in` measured from left upper corner of
  the page.

* \y`\parindent=20pt` is the indentation of the first line of each paragraph.

* \y`\parfillskip=0pt plus 1fil` is horizontal glue added to the last line of the
  paragraph.

* \y`\leftskip=0pt`, \y`\rightskip=0pt`. Glues added to each line in the
  paragraph from the left and the right side. If the stretchability is declared here,
  then the paragraph is ragged left/right.

* \y`\parskip=0pt plus 1pt` is the vertical space between paragraphs.

* \y`\baselineskip=12pt`,
  \y`\lineskiplimit=0pt`,
  \y`\lineskip=1pt`.
  \ii baselineskiprule The {\em `\baselineskip` rule} says: 
  Two consecutive lines in the vertical list have the baseline distance given 
  by \x`\baselineskip` by default. The appropriate real glue is inserted
  between the lines.
  But if this real glue (between boxes) is less than \x`\lineskiplimit` 
  then \x`\lineskip` is inserted between the boxes instead.

* \y`\topskip=10pt` is the distance between the top of the page box and the baseline of
  the first line.

* \y`\linepenalty=10`,
  \y`\hyphenpenalty=50`,
  \y`\exhyphenpenalty=50`,
  \y`\binoppenalty=700`,
  \y`\relpenalty=500`,
  \y`\clubpenalty=150`,
  \y`\widowpenalty=150`,
  \y`\displaywidowpenalty=50`,
  \y`\brokenpenalty=100`,
  \y`\predisplaypenalty=10000`,
  \y`\postdisplaypenalty=0`,
  \y`\interlinepenalty=0`,
  \y`\floatingpenalty=0`,
  \y`\outputpenalty=0`.
  These penalties apply to various places in the vertical or horizontal
  list. Most important are \x`\clubpenalty` (inserted below the first line of a paragraph)
  and \x`\widowpenalty` (inserted before the last line of a paragraph). Typographical rules
  often demand us to set these registers to 10000 (no page break is allowed here).

* \y`\looseness=0` allows us to create of a \"suboptimal" paragraph. The
  paragraph building
  algorithm tries to build the paragraph with \x`\looseness` lines more than
  the optimal solution. If the {\noda\x`\tolerance`} does not have a sufficiently large value
  then this setting is simply ignored. It is reset to zero after each
  paragraph is completed.

* \y`\spaceskip=0pt`,
  \y`\xspaceskip=0pt`. If non-negative they are used as glues between words.
  Default values are read from the font metric data of the current font.

* \y`\pretolerance=100`,
  \y`\tolerance=200`, \y`\emergencystretch=0pt`
  \y`\doublehyphendemerits=10000`,
  \y`\finalhyphendemerits=5000`,
  \y`\adjdemerits=10000`,
  \y`\hfuzz=0.1pt`,
  \y`\vfuzz=0.1pt`
are parameters for the paragraph building algorithm (not described here in
detail).

* \y`\hbadness=1000`,
  \y`\vbadness=1000`. \TeX/ reports a warning about \iid badness on the terminal
  and to the log file if it is greater than these values. The warning has the form
  \ii underfull/box `underfull` `\hbox` or `underfull \vbox`. The value `100`
  means that the `plus` limit for glues is reached.

* \y`\tracingonline=0`,
  \y`\tracingmacros=0`,
  \y`\tracingstats=0`,
  \y`\tracingparagraphs=0`,
  \y`\tracingpages=0`,
  \y`\tracingoutput=0`,
  \y`\tracinglostchars=1`,
  \y`\tracingcommands=0`,
  \y`\tracingrestores=0`,
  \y`\tracingscantokens=0`, 
  \y`\tracingifs=0`,
  \y`\tracinggroups=0`, 
  \y`\tracingassigns=0`. 
  If these registers have positive values then \TeX/ reports details about
  the processing of built-in
  algorithms to the log file. If \i tracingonline `\tracingonline>0` then the same
  output is shown on the terminal.

* \y`\showboxbreadth=5`,
  \y`\showboxdepth=3`,
  \y`\errorcontextlines=5`.
  The amount of information shown when boxes are traced to the log file or an error is
  reported.

* \y`\language=0`. 
  \TeX/ is able to load more hyphenation patterns for more
  languages. This register points to the index of currently used
  hyphenation patterns. Zero means English.

* \y`\lefthyphenmin=2`, \y`\righthyphenmin=3`. Maximum letters left or right in
  hyphenated words.

* \y`\defaulthyphenchar=`\code{`}`\-`. This character is used when words are
  hyphenated.

* \y`\globaldefs=0`. If it is positive then all settings are global.

* \y`\hangafter=1`,
  \y`\hangindent=0pt`.
  If \x`\hangindent` is positive, then after \x`\hangafter` lines all following
  lines are indented. Negative/positive values of \x`\hangindent` or \x`\hangafter` applies
  indentation from left or right and from the top or bottom of the paragraph.
  The \x`\hangindent` is set to 0 after each paragraph.

* \y`\mag=1000`. Magnification factor of all used dimensions.
  The value 1000 means 1:1.

* \y`\escapechar=`\code{`}`\\` 
  use this character in the `\string` primitive.

* \y`\newlinechar=-1`. If positive, this character is interpreted as the end of
  the line when printing to the log or by the {\noda\x`\write`} primitive command.

* \y`\endlinechar=`\code{`}`^^M`. 
  This character is appended to the end of each input
  line. The tokenizer converts it (the Ctrl-M character) to the space token.

* \y`\time=now`,
  \y`\day=now`,
  \y`\month=now`,
  \y`\year=now`. The values about current time/date are set here when \TeX/
  starts to process the document. The \x`\time`
  counts minutes after midnight.

* \y`\prevdepth=*` includes the depth of the last box in vertical mode.

* \y`\prevgraph=*` includes the number of lines of the paragraph when `\par`
  finishes.

* \y`\overfullrule=5pt`. A rectangle to this width is appended after each
  \ii overfull/box overfull `\hbox`.

* \y`\mathsurround=0pt` is the space inserted around a formula in internal math mode.

* \y`\abovedisplayskip=12pt plus3pt minus9pt`,
  \y`\abovedisplayshortskip=0pt plus3pt`,
  \y`\belowdisplayskip=12pt plus3pt minus9pt`,
  \y`\belowdisplayshortskip=7pt plus3pt minus 4pt`.
  These spaces are inserted above and below a formula generated in math display
  mode.

* \y`\tabskip=0pt` is used by the `\halign` primitive command for creating tables.

* \y`\output={\plainoutput}`, \y`\everypar={}`, \y`\everymath={}`
  \y`\everydisplay={}`, 
  \y`\everyhbox={}`
  \y`\everyvbox={}`
  \y`\everycr={}`,
  \*\y`\everyeof={}`,
  \y`\everyjob={}`.
  These token lists are processed when an algorithm of \TeX/ reaches a corresponding
  situations respectively: opens output routine, paragraph, internal math mode, display
  math mode, {\noda\x`\vbox`, \x`\hbox`}, is at the end of a line in a table,
  at the end of an input file, or starts the job.
\enditems

\sec[expand] Expandable primitive commands

These commands are processed like macros, i.e.\ they expand to
another sequence of tokens. 

Notes about notation are in this and the following sections. 
If the documented command is from the $\varepsilon$\TeX{} extension 
(i.e.\ implemented in pdf\TeX, \XeTeX/
and Lua\TeX) then one * is prefixed. If it is from the pdf\TeX/ extension
(implemented in \XeTeX/ and Lua\TeX/ too) then two ** are prefixed. 
If it is~a~Lua\TeX/ only command then three *** are prefixed.

\begitems
* \i string `\string|<control sequence>;` expands to \"the \x`\escapechar`"
  followed by the name of the control sequence. \i escapechar \"The \x`\escapechar`" means a
  character with code equal to \x`\escapechar` or nothing if its value is
  out of range of character codes.
  All characters of the output are \"other characters\c{12}", only spaces (if any exist)
  are kept as space tokens {\char9251}\c{10}. 

* \*\*\*\i csstring `\csstring|<control sequence>;` works like`\string` but without
  `\escapechar`.

* \*\i detokenize `\detokenize/<expandafters>.`\z`{|<text>};` re-tokenizes all tokens in the text.
  Control sequences used in \z`<text>` are re-tokenized like the `\string` primitive, spaces
  are tokens {\char9251}\c{10}, and all other tokens are set as \"other
  characters\c{12}".

* \i the `\the<register>` expands to the value of the register. Examples appear in the
  previous section. The output is tokenized like of `\detokenize`.
  The exception is `\the<tokens register>`: the output is the value
  of the `<tokens register>` without re-tokenizing and the expand processor 
  does not expand this output in {\noda`\edef`, `\write`, `\message`}, etc., arguments.

* \i scantoken `\scantokens/<expandafters>.`\z`{|<text>};` re-tokenizes \z`<text>` using the actual
  tokenizer setting. The behavior is the same as when writing \z`<text>` to a 
  virtual file and reading this file immediately. 

* \*\*\*\i scantextokens `\scantextokens/<expandafters>.`\z`{|<text>};` is the same as
  `\scantokens` but removes problems with end-of-virtual-file.

* \i meaning `\meaning|<token>;` expands to the meaning of the `<token>`. The text is
  tokenized like the `\detokenize` output. 

* \i csname \i endcsname \z`\csname<text>\endcsname` creates a control sequence with name \z`<text>`.
  If it is not already defined, then it gets the `\relax` meaning.
  For example `\csname TeX\endcsname` is the same as `\TeX`.
  The \z`<text>` must be expandable to characters only. Non-expandable
  control sequences (a primitive command at the main processor level, a register,
  a character constant, a font selector) are disallowed here. \TeX/ reports the
  error `missing \endcsname` when this rule isn't compliant.

  Example: `\csname foo:\the\mynumber\endcsname` expands to control sequence
  `\foo:42` if the `\mynumber` is a register with the value 42. 
  Another example: a macro programmer should implement a key/value dictionary 
  using this primitive: 
\begtt
\def\keyval #1 #2 {\expandafter\def\csname dict:#1\endcsname{#2}}
\def\value #1 {\csname dict:#1\endcsname} 
\keyval Peter 21 % key=Peter, value=21, saved to the dictionary
                 % it does \def\dict:Peter{21}
\value Peter     % expands to \dict:Peter which expands to 21. 
\endtt

* \i expandafter \z`\expandafter|<token 1><token 2>;` does the transformation
  \z`<token 1><expanded token2>`. The token processor will expand \z`<token 1>`
  after such a transformation. The \z`<expanded token2>` is only the first level of
  expansion. For example, a macro is transformed to its `<replacement text>`
  but without expansion of `<replacement text>` at this time.
  Or the `\csname...\endcsname` pair creates a control sequence but does not
  expand it at this time.

  If \z`<token 2>` is not expandable then \x`\expandafter` silently does
  nothing.

  The example above (the `\keyval` macro) shows the usage of \x`\expandafter`.
  We need not define `\csname` by `\def`; we want to define 
  a `\dict:key`. The \x`\expandafter` helps here.

  The \z`<token 2>` can be another \x`\expandafter`. We can see
  \x`\expandafter` chains in many macro files. For example
  `\expandafter\A\expandafter\B\expandafter\C\D` is processed as
  \z`\A \B \C <expanded>\D`.  

  The `/<expandafters>.`\z`{|<text>};` syntax rule enables us to prepare \z`|<text>;` by
  `\expandafter`(s). For example \i detokenize `\detokenize{\macro}` expands to
  `\`\c{12}`m`\c{12}`a`\c{12}`c`\c{12}`r`\c{12}`o`\c{12}. But if you need to detokenize
  the `<replacement text>` of the `\macro` then use
  `\detokenize\expandafter{\macro}`. Not only `\expandafter`s should be
  here. The expand processor does full expansion here until an opening brace 
  `{`\c{1} is found.

* \i if \i else \i fi The general rule for all `\if*` commands is 
  `<if condition><true text>/\else<false text>.\fi`. 
  The `<if condition>` is evaluated and `<true text>` or `<false text>`
  is skipped or processed depending on the result of `<if condition>`.
  When the expand processor is skipping the text due to an `\if*` command, it
  expands nothing in the skipped text. But it is noticing all control
  sequences with meaning `\if*`, `\else` and `\fi` during skipping in order 
  to skip correctly all nested `\if*.../\else....\fi` constructions.

  The following `<if condition>`s are possible:
  \_printitem={$\circ$\enspace}

* \i if \z`\if<token 1><token 2>` is true if
  \begitems \removelastskip \style a 
  * both tokens are characters with the same Unicode (or ASCII code in classical \TeX) or
  * both tokens are control sequences 
   (with arbitrary meaning but not \"the character") or
  * one token is a character, second is a control sequence equal to the character (by `\let`) or
  * both tokens are control sequences, their meaning (set by `\let`) is the same character code.
  \enditems
  \removelastskip
  \noindent Example: you can say `\let\test=a` then `\if\test a` returns true.

* \i ifx \z`\ifx|<token 1><token 2>;` is true if the meanings of \z`<token 1>`
   and \z`<token 2>` are the same. 

* \i ifnum `\ifnum<number 1><relation><number 2>`. The `<relation>` could be
  \code{<} or \code{=} or \code{>}. It returns true if the comparison of the two
  numbers is true.

* \i ifodd `\ifodd<number>` returns true if the `<number>` is odd.

* \i ifdim `\ifdim<dimen><relation><dimen>` The `<relation>` could be
  \code{<} or \code{=} or \code{>}. It returns true if the comparison of the two
  dimensions is true.

* \x`\iftrue` returns constantly true, \x`\iffalse` returns constantly false.

* \x`\ifhmode`, \x`\ifvmode`, \x`\ifmmode` -- 
   true if the current mode is horizontal, vertical, math.

* \x`\ifinner` returns true if the current mode is internal vertical, internal
  horizontal or internal math mode.

* \i ifhbox `\ifhbox<box number>`, 
  \i ifvbox `\ifvbox<box number>`, \i ifvoid `\ifvoid<box number>`
  returns true if the specified `<box number>` represents {\noda\x`\hbox`, \x`\vbox`}, void box
  respectively.

* \i ifcat \z`\ifcat<token 1><token 2>` is true if the category codes of \z`<token 1>`
  and \z`<token 2>` are equal.

* \i ifeof `\ifeof<file number>` is true if the file attached to the `<file number>`
  by the {\noda\x`\openin`} primitive does not exist, or the end of file was reached by
  the {\noda\x`\read`} primitive.
  \_printitem{$\bullet$\enspace}

* \*\i unless `\unless<if condition>` negates the result of `<if condition>` before
   skipping or processing the following text.

* \i ifcase `\ifcase<number><case 0>\or<case 1>\or<case 2>` `...` `\or<case n>/\else<else text>.\fi`. 
  This processes the branch given by `<number>`. It processes `<else text>`
  (or nothing if no `<else text>` is declared) when a branch with a given
  `<number>` does not exist.

* \*\i pdfstrcmp `\pdfstrcmp{<stringA>}{<stringB>}` returns $-1$ if
  <stringA>$\string<$<stringB>, 0 if they are equal or 1 in other cases.
  It is not implemented in \LuaTeX.

* \i noexpand `\noexpand|<token>;`. 
   The expand processor does not expand the `<token>` if it is expanding the
   text in {\noda\x`\edef`, \x`\write`, \x`\message`} or similar lists.

* \*\i unexpanded `\unexpanded/<expandafters>.`\z`{|<text>};` returns \z`<text>` and applies 
  `\noexpand` to all tokens in the \z`<text>`. 

* \*\*\i expanded `\expanded{<tokens>}` expands `<tokens>` and
  reads these expanded `<tokens>` again.

* \*\i numexpr `\numexpr<num. expression>`, \i dimexpr \*`\dimexpr<dimen expression>`.
  Documented in the `<dimen>` and `<number>` syntax rules in section~\ref[reg].

* \i number `\number<number>`, \i romannumeral `\romannumeral<number>` prints <number> in decimal
  digits or as a roman numeral (with lowercase letters).

* \x`\topmark` (last from previous page), 
  \x`\firstmark` (first on current page), 
  \x`\botmark` (last on current page). They expand to the corresponding
  {\noda\x`\mark`} included in the current or previous page-box.
  Usable for implementing running headers in the output routine.

* \i fontname `\fontname<font selector>` expands to the file name \*\*\*(or font name) of
  the font given by its `<font selector>`. The `\fontname\font` expands to
  the file name of the current font. 

* \x`\jobname` expands to the name of the main file of this document (without
  extension `.tex`).

* \i input `\input<file name><space>` (classical \TeX) or \ `\input"<file name>"` \
  or \ `\input{<file name>}` \ opens the given
  `<file name>` and starts to read input from it. If the `<file name>` 
  doesn't exist then \TeX/ tries again to open \,{\def\,{\kern-1pt}`<file name>.tex`}. 
  If that doesn't exist, \TeX/ reports an error.
  The alternative syntax  with `"..."`  or `{...}` allows having spaces in the file
  names.

* \i endinput `\endinput`. The current line is the last line of the file being input. The file
  is closed and reading continues from the place where `\input` of this file 
  was started. `\endinput` done in the main file causes future reading from the 
  terminal and a headache for the user.

* \*\*\*\i directlua \z`\directlua {<text>}` runs a Lua script given in \z`<text>`.

\enditems


\sec[main] Primitive commands at the main processor level

{\bf Commands used for declaration of control sequences}
\par\nobreak\medskip\nobreak
\begitems
* \x`\def`, \x`\edef`, \x`\gdef`, \x`\xdef` were documented in section~\ref[def].

* \x`\long` is a prefix; it can be used before `\def`, `\edef`, `\gdef`, `\xdef`.
  The declared macro accepts the control sequence `\par` in its parameters.

* \*\x`\protected` is a prefix; it can be used before `\def`, `\edef`, `\gdef`, `\xdef`.
  The declared macro is not expanded by the expand processor in {\noda\x`\write`,
  \x`\message`, \x`\edef`}, etc., parameters.

* \x`\outer` is a prefix; it can be used before `\def`, `\edef`, `\gdef`, `\xdef`.
  The declared macro must be used only when the main processor is in the context 
  {\em do something} or \TeX/ reports an error. 

* \x`\global` is a prefix; it can be used before any assignment (commands
  from this subsection and `<register>=<value>` settings). The
  assignment is global regardless of the current group.

* \i chardef `\chardef|<control sequence>;=<number>`, 
  \i mathchardef `\mathchardef|<control sequence>;=<number>`
  \ declares a constant <number>. When the main processor is in the context
  {\em do something} and it gets a \x`\chardef`-ed control sequence, it prints
  the character with Unicode (ASCII code) `<number>` to the typesetting output. 
  If it gets a \x`\mathchardef`-ed control sequence, it prints a math object (it works
  only in math mode, not documented here).

* \i countdef `\countdef|<control sequence>;=<number>` declares `<control sequence>` as
  an equivalent to the `\count<number>` which is a register of counter type. 
  The `<number>` here means an address in the
  array of registers of counter type. The `\count0` is reserved for the page
  number. Macro programmers rarely use direct addresses (1 to 9), more
  common is using the allocation macro `\newcount|<control sequence>;`.
 
* \x`\dimendef`, \x`\skipdef`, \x`\muskipdef`, \x`\toksdef` followed by
  `|<control sequence>;=<number>` 
  declare analogically 
  equivalents to \i dimen `\dimen<number>`, \i skip `\skip<number>`, 
  \i muskip `\muskip<number>` and \i toks `\toks<number>`. 
  Usage of allocation macros {\noda\x`\newdimen`, \x`\newskip`, \x`\newmuskip`,
  \x`\newtoks`} are preferred.

* \i font `\font|<font selector>;=<file name><space>/<size specification>.` declares
  `<font selector>` of a font implemented in the `<file name>.tfm`. The 
  `<size specification>` can be `at<dimen>` or `scaled<factor>`.
  The `<factor>` equal to {\tt 1000} means 1:1.
  New syntax (supported by Unicode engines) is

\begtt \catcode`\|=13 \catcode`\/=13 \catcode`\<=13 \Blue
\font|<font selector>;="<font name>/:<font features>." /<size specification>.  % or 
\font|<font selector>;="[<font file>]/:<font features>." /<size specification>.  
\endtt
%
  The `<font file>` is a file name without an `.otf` or `.ttf` extension.
  The `<font features>` are font features prefixed by `+` or `-` and
  separated by a semicolon. The {\let\,=\relax `otfinfo -f <file name>.otf`} command 
  (on command line) can list them. 
  Lua\TeX/ supports alternative syntax: `{...}` instead of `"..."`.
  Example: `\font\test={[texgyretermes-regular]:+onum;-liga} at12pt`. 

* \i let `\let|<control sequence>=<token>;` sets to the `<control sequence>`
  the same meaning as `<token>` has. The `<token>` can be whatever, a
  character or a control sequence.

* \i futurelet \z`\futurelet|<control sequence><token 1><token 2>;` works in two steps.
  In the first step it does \z`\let|<control sequence>=<token 2>;` and in the
  second step \z`<token 1><token 2>` is processed with activated token
  processor. Typically \z`<token 1>` is a macro that needs to know the next token.
\enditems

\goodbreak
\noindent {\bf Commands for box manipulation}

\begitems
* \i hbox `\hbox{<cmds>}` or `\hbox to<dimen>{<cmds>}` or `\hbox spread<dimen>{<cmds>}`
  creates a box. The material inside this box is a `<horizontal list>`
  generated by `<cmds>` in horizontal mode in a group. The width of the box is the natural width
  of the `<horizontal list>` or `<dimen>` given by the \ii to `to<dimen>` parameter or
  it is spread by the `<dimen>` given by the \ii spread `spread<dimen>`
  parameter. The height of the box is the maximum of heights of all elements in
  the `<horizontal list>`. The depth of the box is the maximum of depths of all
  such elements. These elements are set on the common baseline (exceptions
  can be given by {\noda\x`\lower` or \x`\raise`} commands).

* \i vbox `\vbox{<cmds>}` or `\vbox to<dimen>{<cmds>}` 
  or `\vbox spread<dimen>{<cmds>}`
  creates a box. The material inside this box is a `<vertical list>`
  generated by `<cmds>` in vertical mode in a group. The height of the
  box is the natural height of the `<vertical list>` (eventually modified by values
  from \ii to `to` or \ii spread `spread` parameters) without the depth of the last
  element. The depth of the last element is set as the depth of the box. 
  The width of the box is the maximum of widths of elemens in the `<vertical list>`.
  All elements are placed at the
  common left margin of the box (exceptions can be given by {\noda\x`\moveleft` or
  \x`\moveright`} commands).

* \i vtop `\vtop{<cmds>}` (with optional `to` or `spread`
  parameters) is the same as `\vbox`,
  but the baseline of the resulting box goes through the baseline of the first element in
  the `<vertical list>` (note that `\vbox` has its baseline equal to
  the baseline of the last element inside).

* \i vcenter `\vcenter{<cmds>}` (with optional `to` or `spread`
  parameters) is equal to
  `\vbox`, but its \ii math/axis {\em math axis}\fnote
  {The math axis is a horizontal line which goes through centers of + and $-$
  symbols. Its distance from the baseline is declared in the math font metrics.}
  is exactly in the middle of the box. So its baseline is appropriately shifted.
  The `\vcenter` can be used only in math modes but given `<cmds>` are
  processed in vertical mode.

* \i lower `\lower<dimen><box>`, \i raise `\raise<dimen><box>` move the `<box>` up or down by
  the `<dimen>` in horizontal mode. 
  \i moveleft `\moveleft<dimen><box>`, 
  \i moveright `\moveright<dimen><box>` move the `<box>` by
  the `<dimen>` in vertical mode.

* \i setbox `\setbox<box number>=<box>`. \TeX/ has a set of 
  \ii box/register {\em box registers} addressed by `<box number>` and
  accessed via \i box `\box<box number>` or alternatives described below.
  The `\setbox` command saves the given `<box>` to the register addressed by
  `<box number>`.

  Macro programmers use only 0 to 9 \z`<box numbers>` directly. Other
  addresses to box registers should be allocated by 
  the {\noda\i newbox `\newbox|<control sequence>;`} macro. The `|<control sequence>;`
  is equivalent to a `<box number>`, not to the box register itself.

  The `\setbox` command does an assignment, so the \x`\global` prefix is needed
  if you want to use the saved box outside the current group.

* \i box `\box<box number>` returns the box from `<box number>` box
  register. Example: you can do `\setbox0=\hbox{abc}`. This `\hbox` isn't
  printed but saved to the register 0. At a different place you use
  `\box0`, which prints `\hbox{abc}`, or you can do
  `\setbox0=\hbox{cde\box0}` which saves the `\hbox{cde\hbox{abc}}`
  to the register~0.

* \i copy `\copy<box number>` returns the box from
  `<box number>` box register and keeps the same box in this box register.
  Note that the \i box `\box<box number>` returns the
  box and empties the register `<box number>` immediately. If you don't want
  to empty the register, use `\copy`. 

* \i wd `\wd<box number>`, `\ht<box number>`, `\dp<box number>`. You can
  measure or use the width, height and depth of a box saved in a register addressed
  by `<box number>`. Examples `\mydimen=\ht0`, `\hbox to\wd0{...}`. 
  You can re-set the dimensions of a box saved in a register addressed by 
  `<box number>`. For example \i setbox `\setbox0=\hbox{abc}` `\wd0=0pt` `\box0`
  gives the same result as \i hbox `\hbox to0pt{abc}` but without the warning about
  \ii overfull/box overfull `\hbox`.

* \i unhbox `\unhbox<box number>`, \kern-.4pt\i unvbox `\unvbox<box number>`, 
  \kern-.4pt\i unhcopy `\unhcopy<box number>`, \kern-.4pt\i unvcopy `\unvcopy<box number>`
  do the same work as `\box` or `\copy` but they don't return the whole
  box but only its contents, i.e.~the horizontal or vertical material.
  Example: try to do \i setbox `\setbox0=\hbox{abc}` and later 
  `\setbox0=\hbox{cde\unhbox0}` saves
  the \i hbox `\hbox{cdeabc}` to the box register~0. 

  The \x`\unhbox` and \x`\unhcopy` commands return the \x`\hbox` contents and 
  \x`\unvbox`, \x`\unvcopy` commands return the \x`\vbox` contents. If incompatible
  contents are saved, then \TeX/ reports an error. You can test the type of
  saved contents by \x`\ifhbox` or \x`\ifvbox`.

* \i vsplit `\vsplit<box number> to<dimen>` does a column break. 
  The `<vertical material>` saved in the box `<box number>` is broken
  into a first part of `<dimen>` height and the rest remains in the
  box `<box number>`. The broken part is saved as a `\vbox` which is the
  result of this operation. For example, you can say `\newbox\column`
  `\setbox\column=\vbox{...}` and later 
  `\setbox0=\vsplit\column to5cm`. The `\box0` is a `\vbox` containing the first 5cm of
  saved material.

* \x`\lastbox` returns the last box in the current vertical or horizontal
  material and removes it. 
\enditems

\noindent {\bf Commands for rules (lines in the typesetting output) and patterns}

\begitems
* \x`\hrule` creates a horizontal line in the current vertical
  list. If it is used in horizontal mode, it finishes the paragraph by
  {\noda\x`\par`} first.
  `\hrule /width<dimen>. /height<dimen>. /depth<dimen>.` creates (in
  general, with given parameters) a full
  rectangle (something like a box, but it isn't treated as the box) with given
  dimensions. Default values are: \"width"~=width of outer \x`\vbox`,
  \"height"~=0.4\,pt, \"depth"~=0\,pt.
  {\emergencystretch=2em\par}

* \x`\vrule` creates a vertical line in the current horizontal
  list. If it is used in vertical mode, it opens the horizontal mode
  first.
  `\vrule /width<dimen>. /height<dimen>. /depth<dimen>.` creates (in
  general, with given parameters) a full rectangle with given
  dimensions. Default values are: \"width"~=0.4\,pt,
  \"height"~=height of outer `\hbox`, \"depth"~=depth of outer \x`\hbox`.
  {\emergencystretch=2em\par}

  The optional parameters of \x`\hrule` and \x`\vrule` can be specified in
  arbitrary order and they can be specified more than once. In such a case,
  the rule \"last wins" is applied.

* \i leaders `\leaders<rule><glue>` creates a glue 
  (maybe shrinkable or stretchable) filled by a full rectangle.
  The `<rule>` is \x`\vrule` or \x`\hrule` (maybe with its optional parameters).
  If the `<glue>` is specified by an {\noda\x`\hskip`} command
  (maybe with its optional parameters)
  or by its alternatives {\noda\x`\hss`, \x`\hfil`, \x`\hfill`}, then the resulting glue
  is horizontal (can be used only in horizontal mode) and its dimensions
  are: width derived from `<glue>`, height plus depth derived from `<rule>`.
  If the `<glue>` is specified by a {\noda\x`\vskip`} command
  (maybe with its optional parameters)
  or by its alternatives {\noda\x`\vss`, \x`\vfil`, \x`\vfill`}, then the resulting glue
  is vertical (can be used only in vertical mode) and its dimensions
  are: height derived from `<glue>`, width derived from `<rule>`, depth is zero.

* \i leaders `\leaders<box><glue>` creates a vertical or horizontal glue
  filled by a pattern of repeated `<box>`. The positions of boxes are
  calculated from the boundaries of the outer box. It is used for the dots patterns 
  in the table of contents.
  \i cleaders `\cleaders<box><glue>` does the same, but the pattern of boxes is
  centered in the space derived by the <glue>. Spaces between boxes are not~inserted. 
  \i xleaders `\xleaders<box><glue>` does the same, but the spaces between
  boxes are inserted equally.
\enditems

\noindent {\bf More commands for creating something in typesetting output}

\begitems
* \x`\par` closes horizontal mode and finalizes a paragraph.
* \x`\indent`, \x`\noindent`. They leave vertical mode and open a paragraph
  with/without paragraph indentation. If horizontal mode is current then 
  `\indent` inserts an empty box of `\parindent` width; `\noindent` does
  nothing.
* \x`\hskip`, \x`\vskip`. They insert a horizontal/vertical glue. Documented in
  section~\ref[boxes].
* \x`\hfil`, \x`\hfill`, \x`\hss`, \x`\vfil`, \x`\vfill`, \x`\vss` are alternatives
  of \x`\hskip`, \x`\vskip`, see section~\ref[boxes].
* \x`\hfilneg`, \x`\vfilneg` are shortcuts for `\hskip 0pt plus-1fil`
  and `\vskip 0pt plus-1fil`.
* \i kern `\kern<dimen>` puts unbreakable horizontal/vertical space 
  depending on the current mode.
* \i penalty `\penalty<number>` puts the penalty `<number>` on the current
  horizontal/vertical list.
* \i char `\char<number>` prints the character with code 
  \kern-2pt`<number>`\kern-2pt. The
  \"character itself" does the same.
* \i accent `\accent<number><character>` places an accent with code
  `<number>` above the `<character>`.
* \ii -space `\`{\tt\char9251} is the \ii control/space control space. In
  horizontal mode, it inserts the space glue (like normal space but without
  modification by the \x`\spacefactor`). In vertical mode, it opens horizontal
  mode and puts the space. Note that normal space does nothing in vertical
  mode.
* \i discretionary `\discretionary{<pre break>}{<post break>}{<no break>}` 
  works in horizontal mode. It prints `<no break>`
  in normal cases but if there is a line break then `<pre break>`
  is used before and `<post break>` after the breaking point.
  German Zucker/Zuk-ker (sugar) can be implemented by 
  `Zu\discretionary{k-}{k}{ck}er`.
* \ii -hyphen `\-` is equal to \i hyphenchar 
  `\discretionary{\char\hyphenchar<font>}{}{}`.
  The `\hyphenchar<font>` is used as a hyphenation character. It is set
  to \x`\defaulthyphenchar` value when the font is loaded, but it can be changed.
* \ii -italiccorr \code{\\/} does an
  \ii italic/correction italic correction. 
  It puts a little space if the last character is slanted.
* \x`\unpenalty`, \x`\unskip` removes the last penalty/last glue
  from the current horizontal/vertical list.
* \i vadjust `\vadjust{<cmds>}`. This works in horizontal mode. The `<cmds>`
  must create a `<vertical list>` and `\vadjust` saves a pointer to this list
  into the current horizontal list. When `\par` creates lines of the paragraph
  and distributes them to a vertical list, each line with the pointer from
  `\vadjust` has the corresponding `<vertical list>` immediately appended 
  after this line.
* `\insert<number>{<cmds>}`. The `<cmds>` create a `<vertical list>` and
  `\insert` saves a pointer to such a `<vertical list>` into the current list.
  The output routine can work with such `<vertical list>`\kern-2pts. The footnotes or
  \ii floating~object {\em floating objects} (tables, figures) 
  are implemented by the `\insert` primitive.  
* \i halign `\halign{<declaration>\cr`\z`<row 1>\cr<row 2>\cr...\cr<row n>\cr}`
  creates a table of boxes in vertical mode. The `<declaration>` declares
  one or more column patterns separated by `&`\c4. The rows use the same
  character to separate the items of the table in each row. The `\halign`
  works in two passes. First it saves all items to boxes and the second pass
  performs `\hbox to`~$\Blue w$ for each saved item, where $\Blue w$ 
  is the maximum width of items in each actual column. 

  Detailed documentation of `\halign` is out of scope of this manual.
  Only one example follows: the macro `\putabove` puts `#1` above `#2`
  centered. The width of the resulting box is equal to the maximum of widths of these 
  two parameters. The `<declaration>` `\hfil##\hfil` means that the items
  will be centered:\nl
  `\def\putabove#1#2{\vbox{\halign{\hfil##\hfil\cr#1\cr#2\cr}}}`.

* \x`\valign` does the same as `\halign` but rows $\leftrightarrow$ columns.
  It is not commonly used.

* \x`\cr`, \x`\crcr`, \x`\span`, \x`\omit`, \i noalign `\noalign{<cmds>}`
  are primitives used by `\halign` and `\valign`. 

\enditems

\noindent {\bf Commands for register calculations}

\begitems
* \i advance `\advance<register>/by.<value>` does (formally)
  `<register>=<register>+<value>`. The `<register>` is counter type or dimen
  type. The `<value>` is `<number>` or `<dimen>` (depending on the type of
  `<register>`).
* \i multiply `\multiply<register>/by.<number>` does
  `<register>=<register>*<number>`.
* \i divide `\divide<register>/by.<number>` does
  `<register>=<register>`\code{/}`<number>`. If the `<register>` is 
  number type then the result is truncated.
* See \*`\numexpr` and \*`\dimexpr`, expandable primitives documented in
  sections~\ref[reg] and~\ref[expand].
\enditems

\noindent {\bf Internal codes}

\begitems
* \i catcode `\catcode<number>` is category code of the character with
  `<number>` code. Used by tokenizer.
* \i lccode `\lccode<number>` is the lowercase alternative to the
  `\char<number>`. If it is zero then a lowercase alternative doesn't exist
  (for example for punctuation). Used by the `\lowercase` primitive and when
  breaking points are calculated from hyphenation patterns.
* \i uccode `\uccode<number>` is the uppercase alternative to the
  `\char<number>`. If it is zero, then the uppercase alternative doesn't exist.
  Used by the `\uppercase` primitive.
* \i lowercase \i uppercase `\lowercase/<expandafters>.`\z`{|<text>};`, \
  `\uppercase/<expandafters>.`\z`{|<text>};` transform \z`|<text>;` to
  lowercase/uppercase using the current `\lccode` or `\uccode` values.
  Returns transformed \z`|<text>;` where catcodes of tokens and 
  tokens of type `<control sequence>` are unchanged.
* \i sfcode `\sfcode<number>` is the spacefactor code of the `\char<number>`.
  The `\spacefactor` register keeps (roughly speking) 
  the `\sfcode` of the last printed character. The glue between words
  is modified (roughly speaking) by this `\spacefactor`. The value
  1000 means factor 1:1 (no modification is done). It is used for enlarging spaces
  after periods and other punctuation in English texts.\fnote{
  This feature is not compliant with other typographical traditions, so the
  `\frenchspacing` macro which sets all `\sfcodes` to 1000 is used very often.}
  {\emergencystretch=2em\par}
\enditems

\noindent {\bf Commands for reading or writing text files}

\begitems
* Note that the main input stream is controlled by `\input` and `\endinput`
  expandable primitive commands documented in section~\ref[expand].

* \i openin `\openin<file number>`\,`=`\,`<file name><space>` \
  (or `\openin`\,`<file number>`\,`=`\,`{<file name>}`) \ opens the
  file `<file name>` for reading and creates a file descriptor
  connected to the `<file number>`.\fnote
  {\noda Note that `<file number>` is an address to the file descriptor. Macro
  programmers don't use these addresses directly but by the
  \i newread `\newread|<control sequence>;` and 
  \i newwrite `\newwrite|<control sequence>;` allocation macros.} 
  If the file doesn't exist nothing happens but a
  macro programmer can test this case by `\ifeof<file number>`.

* \i read `\read<file number>to|<control sequence>;` does
  `\def|<control sequence>{<replacement text>};`\nl where
  the `<replacement text>` is the tokenized next line from the file declared by
  `\openin` as `<file number>`.

* \i openout `\openout<file number>=<file name><space>`
  (or `\openout<file number>="<file name>"`) \  opens the `<file name>` for
  writing and creates a file descriptor connected to `<file number>`.
  If the file already exists, then its contents are removed.
  {\emergencystretch=2em\par}

* \i write `\write<file number>`\z`|{<text>};` writes a line of \z`<text>`
  to the file declared by `\openout` as `<file number>`. But this isn't done
  immediately. \TeX/ does not know the value of the current page
  when the `\write` command is processed because
  the paragraph building and page building algorithms are processed
  asynchronously. But a macro programmer typically needs to save current 
  page to the file in order to read it again and to create 
  a Table of contents or an Index.

  `\write<file number>`\z`|{<text>};` saves \z`|<text>;` into memory and puts a
  pointer to this memory into the typesetting output. When the page is shipped
  out (by output routine), then all such pointers from this page are
  processed: the \z`<text>` is expanded at this time and its expansion
  is saved to the file. If (for example) the \z`<text>` includes 
  `\the\pageno` then it is expanded to the correct page number of this page. 

* \i closein `\closein<file number>`, \i closeout `\closeout<file number>`
  closes the open file. It is done automatically when \TeX/ terminates its
  job.

* \x`\immediate` is a prefix. It can be used before `\openout`, `\write` and
  `\closeout` in order to do the desired action immediately (without waiting for
  the output routine). 
\enditems

\noindent {\bf Others primitive commands}
\par\nobreak\medskip\nobreak
\begitems
* \x`\relax` does nothing. Used for terminating incomplete optional
  parameters, for example.
* \x`\begingroup` opens group, \x`\endgroup` closes group. 
  The `{`\c1 and `}`\c2 do the same but moreover, they are syntactic
  constructors for primitive commands and math lists (in math mode).
  These two types of groups (declared by mentioned commands or
  by mentioned characters) cannot be mixed, i.e.\ 
  `\begingroup...}` gives an error. Plain \TeX/ declares
  {\noda\x`\bgroup` and \x`\egroup`} control sequences as equivalents to
  `{`\c1 and `}`\c2. They can be used instead  of `{`\c1 and~`}`\c2 when we
  need to open/close a group, to create a math list, or when a box is constructed.
  For example, \z`\hbox\bgroup<text>\egroup` is syntactically correct.
* \i aftergroup `\aftergroup|<token>;` saves the `|<token>;` and puts it
  back in the input queue immediately after the current group is closed. 
  Then the expand processor expands it (if it is expandable). More
  `\aftergroup`s in one group create a queue of `|<token>;`s used after
  the group is closed.
* \i afterassignment `\afterassignment|<token>;` saves the `|<token>;`
  and puts it back immediately after a following assignment (`<register>=<value>`,
  `\def`, etc.)\ is done.
* \x`\lastskip`, \x`\lastpenalty` return the value of the last element in the
  current horizontal or vertical list if it is a glue/penalty. It returns
  zero if the element found is not the last.
* \x`\ignorespaces` ignores spaces in horizontal mode until the
  next primitive command occurs.
* \i mark \z`\mark{<text>}` saves \z`<text>` to memory and puts a pointer to it in the
  typesetting output. The \z`<text>` is used as expansion output of
  \x`\firstmark`, \x`\topmark` and \x`\botmark` expansion primitives in
  the output routine.
* \i parshape `\parshape<number>`% 
  {\def<#1>{\,$\langle\it#1\rangle$\,}%
  `<I1><W1><I2><W2>...<In><Wn>`
  enables to set arbitrary shape of the paragraph. The `<number>` 
  declares the amount of data: the `<number>` pairs of `<dimen>`s follow.
  The $i$-th line of the paragraph is shifted by `<Ii>` to the right and
  its width is `<Wi>`.} The `\parshape` data are re-set after
  each paragraph to zero values (normal paragraph).
* \i special \z`\special{<text>}` puts the message \z`<text>` into the typesetting output. It
  behaves as a zero-dimension pointer to \z`<text>` and it can be read by
  printer drivers. It is recommended to not use this old technology
  when PDF output is created directly.
* \i shipout `\shipout<box>` outputs the `<box>` as one page. Used in the 
  \ii output/routine output routine.
* \x`\end` completes the last page and terminates the job.
* \x`\dump` dumps the memory image to a file named `\jobname.fmt` and terminates the job.
* \i patterns `\patterns{<data>}` reads hyphenation patterns for the current 
  \x`\language`.
* \i hyphenation `\hyphenation{<data>}` reads hyphenation exceptions for
  current \x`\language`.
* \i message \z`\message{<text>}` prints \z`<text>` on the terminal and to the 
  log file.
* \i errmessage \z`\errmessage{<text>}` behaves like \z`\message{<text>}` but
  \TeX/ treats it as an error.
* Job processing modes can be set by \x`\scrollmode` (don't pause at 
  errors), \x`\nonstopmode` (don't pause at errors or missing files),
  \x`\batchmode` (\x`\nonstopmode` plus no output to the terminal). Default is
  \x`\errorstopmode` (stop at errors).
* \x`\inputlineno` includes the number of the current line from current file
  being input.
* \i show `\show|<control sequence>;`, \ \i showbox `\showbox<box number>`, 
  \ \x`\showlists`, \ and \ 
  \i showthe `\showthe|<register>;` \ are tracing commands. \TeX/ prints desired result
  on the terminal and to the log file and pauses.
\enditems

\noindent {\bf Commands specific for PDF output} 
(available in pdf\TeX, \XeTeX/ and Lua\TeX)

\begitems
* \i pdfliteral \z`\pdfliteral{<text>}` puts the \z`<text>` interpreted 
  in a low level PDF language to the typesetting output. All PDF constructs
  defined in the PDF specification are allowed. The dimensions of
  the `\pdfliteral` object in the output are considered zero. So, if 
  \z`<text>` moves the current typesetting point then the notion about its
  position from the \TeX/ point of view differs from the real position. 
  A good practice is to close \z`<text>` to `q...Q` PDF commands.
  The command `\pdfliteral` is typically 
  used for generating graphics and for linear transformation.
* \i pdfcolorstack `\pdfcolorstack<number><op>`\z`{<text>}` (where `<op>` is `push`
  or `pop` or `set`) behaves like \z`\pdfliteral{<text>}` and it is used for
  color switchers. For example when \z`<text>` is `1 0 0 rg` then the red color is
  selected. \TeX/ sets the color stack at the top of each page to the
  color stack opened at the bottom of the previous page.
* \i pdfximage 
  `\pdfximage` `/height<dimen>.` `/depth<dimen>.` `/width<dimen>.` `/page<number>.{<file name>}`
  loads the image from `<file name>` to the PDF output and returns the
  number of such a data object in the \x`\pdflastximage` register. Allowed
  formats are PDF, JPG, PNG. The image is not drawn at this moment. A macro
  programmer can save `\mypic=\pdflastximage` and draw the image by
  \i pdfrefximage `\pdfrefximage\mypic` (maybe repeatedly). Data of 
  the image are loaded to the PDF output only once. 
  The `\pdfximage` allows more parameters;
  see pdf\TeX/ documentation.
* \i pdfsetmatrix {\def<#1>{$\,\langle{\it#1}\rangle\,$}`\pdfsetmatrix {<a><b><c><d>}`
  multiplies the current transformation matrix (used for linear
  transformations) by `\matrix{<a>&<c>\cr <b>&<d>}`.}
* \i pdfdest `\pdfdest name{<label>}<type>\relax` declares a destination of
  a hyperlink. The `<label>` must match with the `<label>` used in 
  `\pdfoutline` or `\pdfstartlink`. The `<type>` declares the behavior of the
  pdf viewer when the hyperlink is used. For example, `xyz` means without changes
  of the current zoom (if not specified). Other types should be `fit`, `fith`,
  `fitv`, `fitb`. 
* \i pdfstartlink
  `\pdfstartlink` `/height<dimen>.` `/depth<dimen>.` `/<attributes>.` `/goto name{<label>}.`
  declares the beginning of a hyperlink. A text (will be sensitive on mouse click)
  immediately follows and it is terminated by \x`\pdfendlink`. The height
  and depth of the sensitive area and the `<label>` used in `\pdfdest` are declared
  here. More parameters are allowed; see the pdf\TeX/ documentation. 
* \i pdfoutline \z`\pdfoutline /goto name{<label>}. /count<number>. {<text>}`
  creates one item with \z`<text>` in PDF outlines. `<label>` must be used
  somewhere by `\pdfdest name{<label>}`. The `<number>` is the number of
  direct descentants in the outlines tree.
* \i pdfinfo `\pdfinfo {<key>`\z`(<text>)}` saves to PDF the information which
  can be listed by the command `pdfinfo <file>.pdf` on the command line for
  example. More `<key>`\z`(<text>)` should be here. The `<key>` can be 
  \code{/Author}, \code{/Title}, \code{/Subject},  
  \code{/Keywords}, \code{/Creator}, \code{/Producer},
  \code{/CreationDate}, \code{/ModDate}. The last two keywords need
  a special format of the \z`<text>` value. All \z`<text>` values (including
  \z`<text>` used in the `\pdfoutline`) must be ASCII encoded or they can use
  a very special PDFunicode encoding.
* \x`\pdfcatalog` enables us to set of a default behavior of the PDF viewer when it starts.
* \x`\pdfsavepos` saves an internal invisible point to the typesetting output. These
  points are processed when the page is shipped out: the numeric registers
  \x`\pdflastxpos` and \x`\pdflastypos` get values for the absolute position of
  this invisible point (measured from the left upper corner of the page in `sp` units).
  The macro programmer can follow `\pdfsavepos` by the `\write` command and save
  these absolute positions to a text file which can be read in the next run
  of \TeX/ in order to get these absolute positions by macros. 
\enditems

\noindent {\bf Microtypographical extensions}
(available in pdf\TeX/, Lua\TeX/ and not all of them in \XeTeX)

\begitems
* \i pdffontexpand 
  `\pdffontexpand <font selector> <stretching> <shrinking> <step>`
  declares a possibility to deform the characters from the font given by
  `<font selector>`. This deformation is used when stretching or shrinking
  paragraph lines or doing `\hbox to{...}` in general. I.e.\ not only glues are
  stretchable and shrinkable. The numeric parameters are given in 1/1000
  of the font size. `<stretching>` and `<shrinking>` are the maximum allowed
  values. The stretching or shrinking are not applied continuously but by
  the given `<step>`.
  To activate this feature you must set the \x`\pdfadjustspacing`
  numeric register to a positive value. 
* \i efcode `\efcode <font selector><char. code>=<number>`
  sets the degree of willigness of given character to be deformed when
  `\pdffontexpand` is used. Default value for all
  characters is 1000 and `<number>`/1000 gives the proportion coefficient for
  stretching or shrinking of the character with respect to the \"normal" deformation
  of characters with default value 1000.
* \i rpcode \i lpcode 
  `\rpcode <font selector><char. code>=<number>`,
  `\lpcode <font selector><char. code>=<number>` allows the declaration of
  hanging punctuation. Such punctuation is slightly moved to the right
  margin (if `\rpcode` is declared and the character is at the right margin) 
  or to the left margin (for `\lpcode` by analogy).
  The `<number>` gives the amount of such movement in 1/1000 of the font size.
  To activate this feature you must set \x`\pdfprotrudechars` to
  a positive value (2 or more means a better algorithm).
* \i letterspacefont
  `\letterspacefont |<control sequence>; <font selector> <number>`
  declares a new font selector `|<control sequence>;` as a font given by
  the `<font selector>`. Additional space declared by `<number>` is added
  between each two characters when the font is used. The `<number>` is 1/1000 of
  the font size. Unicode fonts support an analogous
  `letterspace=<number>` font feature. 
* The following commands have the same syntax as `\rpcode`:
  \x`\knbscode` (added space after the character),
  \x`\stbscode` (added stretchability of the glue after the character),
  \x`\shbscode` (added shrinkability after the character),
  \x`\knbccode` (added kern before the character),
  \x`\knaccode` (added kern after the character).
  To activate this feature you must to set
  \x`\pdfadjustinterwordglue` to a positive value.
  This feature is supported by pdf\TeX/ only.
\enditems

\noindent {\bf Commands used in math mode}

\begitems
* \x`\displaystyle`, \x`\textstyle`, \x`\scriptstyle`, \x`\scriptscriptstyle`
  switch to the specified style.
* \x`\mathord`, \x`\mathop`, \x`\mathbin`, \x`\mathrel`, \x`\mathopen`, \x`\mathclose`,
  \x`\mathpunct` followed by `{<math list>}` create a math object of the given type.
* \i over `{<numerator>\over<denominator>}` creates a fraction.
  The primitive commands \x`\atop` (without fraction rule), 
  \i above `\above<dimen>` (fraction rule with given thickness) should be used in the same
  manner. The commands \x`\atopwithdelims`, \x`\overwithdelims`,
  \x`\abovewithdelims` allow us to specify brackets around the generalized
  fraction.
* \i left \i right `\left<delimiter><formula>\right<delimiter>` creates
  a math `<formula>` and gives `<delimiter>`s around it with an appropriate size
  (compatible with the size of the formula). The `<delimiter>`s are typically
  brackets.
* \*\i middle `\middle<delimiter>` can be used inside the <formula>
  surronded by `\left`, `\right`. The given <delimiter> gets the same size
  as delimiters declared by appropriate `\left`, `\right`.
* Exponents and scripts are typically at the right side of the preceding
  math object. But if this object is a \"big operator" (summation, integral)
  then exponents and scripts are printed above and below this operator.
  The commands \x`\limits`, \x`\nolimits`, \x`\displaylimits` 
  used before exponents and scripts constructors 
  (`^`\c7 and `_`\c8) declare an exception from this rule.
* \i eqno `$$<formula>\eqno<mark>$$` puts the `<mark>` to the right
  margin as `\llap{$<mark>$}`. Analogously, 
  \i leqno `$$<formula>\leqno<mark>$$` puts it to the left margin.
\enditems

\sec[plain] Summary of plain \TeX/ macros

\noindent{\bf Allocators}

\begitems
* \x`\newcount`, \x`\newdimen`, \x`\newskip`, \x`\newmuskip`, \x`\newtoks`
  followed by a `|<control sequence>;` allocate a new register of the given type
  and set it as the `|<control sequence>;`. 
  \x`\newbox`, \x`\newread`, \x`\newwrite` 
  followed by a `|<control sequence>;` allocate a new address to given data
  (to a box register or to a file descriptor) and set it as the
  `|<control sequence>;`.
  All these allocation macros are declared as `\outer` in plain \TeX/,
  unfortunately. This brings problems when you need to use them in
  skipped text or in macros (in `<replacement text>` for example).
  Use `\csname newdimen\endcsname \yoursequence` in such cases. 
* \i newif `\newif|<control sequence>;` sets the `|<control sequence>;`
  as a boolean variable. It must begin with `if`; for
  example `\newif\ifsomething`. Then you can set values by
  `\somethingtrue` or `\somethingfalse` and you can use this variable by
  `\ifsoemthing` which behaves like other `\if`\code{*} primitive commands.
\enditems

\noindent{\bf Vertical skips}

\begitems
* \x`\bigskip` does \x`\vskip` by one line, \x`\medskip` does `\vskip` by
  one half of a line and \x`\smallskip` does the vertical skip by one quarter of a line.
  The registers \x`\bigskipamount`, \x`\medskipamount` and \x`\smallskipamount`
  are allocated for this purpose. 
* \x`\nointerlineskip` ignores the \x`\baselineskip` rule 
  %(see section~\ref[reg]) 
  for the following box in
  the current vertical list. This box is appended immediately after
  the previous box. 
  {\noda\x`\offinterlineskip`} ignores the \x`\baselineskip` rule for all following
  boxes until the current group is closed. 
* All vertical glues at the top of the page inserted by \x`\vskip` are
  ignored. Macro \x`\vglue` behaves like the `\vskip` primitive command but its glue
  is not ignored at the top of the page.

* Sometimes we must switch off the \x`\baselineskip` rule (by the \x`\offinterlineskip`
  macro for example). This is common in tables. But we need to keep the
  baseline distances equal. Then the \x`\strut` can be inserted on each
  line. It is an invisible box with zero width and with
  height+depth=`\baselineskip`.

* \x`\normalbaselines` sets the registers for vertical placement
  \x`\baselineskip`, \x`\lineskip` and \x`\lineskiplimit` to default values given
  by the format. The user can set other values for a while and then he/she can restore
  `\normalbaselines`. 
\enditems

\noindent{\bf Penalties}
\par\nobreak\medskip\nobreak
\begitems
* \x`\break` puts penalty $-10000$, so a line/page break is forced here.
  \x`\nobreak` puts penalty 10000, so a line/page break is disabled here.
  It should be specified before a glue, which is \"protected" by this penalty.
  \x`\allowbreak` puts penalty 0; it allows breaking similar to a normal space.
* \x`\goodbreak` puts penalty $-500$ in vertical mode, this is a \"recommended"
  point for a page break.
* \x`\filbreak` breaks the page only if it is \"almost full" or if a big object
  (that doesn't fit the current page) follows. The bottom of such a page is
  filled by a vertical glue, i.e.\ the default typographical rule about equal
  positions of all bottoms of common pages is broken here.
* \x`\eject` puts penalty $-10000$ in the vertical list, i.e.\ it breaks the page.
\enditems

\noindent{\bf Miscellaneous macros}\par\nobreak\medskip\nobreak

\begitems
* \i magstep `\magstep<number>` expands to a magnification factor $1.2^x$ where $x$ is the
  given `<number>`. This follows old typographical traditions that all sizes
  (of fonts) are distinguished by factors 1, 1.2, 1.44, etc.
  For example, `\magstep2` expands to 1440, because $1.2^2=1.44$ 
  and 1000 is factor 1:1 in \TeX/. The \x`\magstephalf` macro expands to
  1095 which corresponds to $1.2^{(1/2)}$.
* \x`\nonfrenchspacing` sets special space factor codes (bigger spaces after
  periods, commas, semicolons, etc.). This follows English typographical
  traditions. \x`\frenchspacing` sets all space factors as 1:1 (usable for non
  English texts). 
* \x`\endgraf` is equivalent to \x`\par`; \x`\bgroup` and \x`\egroup` are
  equivalents to `{`\c1 and `}`\c2.
* \x`\space` expands to space, \x`\empty` is an empty macro and \x`\null` is an empty
  \i hbox `\hbox{}`.
* \x`\quad` is horizontal space 1\,em (size of the font), \x`\qquad` is
  double `\quad`,
  \x`\enspace` is kern 0.5\,em, \x`\thinspace` is kern 1/6\,em, and 
  \x`\negthinspace` makes kern $-$1/6\,em.
* \i loop \i repeat `\loop` \z`<body 1><if condition><body 2>\repeat`
  repeats \z`<body 1>` and \z`<body 2>` in a loop until the `<if condition>` returns
  false. Then \z`<body 2>` is not processed and the loop is finished.
* \x`\leavevmode` opens a paragraph like `\indent` but it does nothing if
  the horizontal mode is already in effect.

* \i line \z`\line{<text>}` creates a box of line width (which is \x`\hsize`). 
  \x`\leftline`, \x`\rightline`, \x`\centerline` do the same as 
  `\line` but \z`<text>` is shifted left / right / is centered.

* \i rlap \z`\rlap{<text>}` makes a box of zero size, the \z`<text>` is stuck
  out to the right. \i llap \z`\llap{<text>}` does the same and the \z`<text>` is pushed
  left. 

* \x`\ialign` is equal to `\halign` but the values of the registers used by
  `\halign` are set to default.

* \x`\hang` starts the paragraph where all lines (except for the first) are
  indented by `\parindent`.

* \i textindent `\texindent{<mark>}` starts a paragraph with `\llap{<mark>}`. 

* \i item `\item{<mark>}` starts the paragraph with `\hang` and with
  `\llap{<mark>}`. Usable for item lists.
  \i itemitem `\itemitem{<mark>}` can be used for the second level of items.

* \x`\narrower` sets wider margins for paragraphs (`\parindent` is appended
  to both sides); i.e.\ the paragraphs are narrower.

* \x`\raggedright` sets the paragraph shape with the ragged right margin.
  \x`\raggedbottom` sets the page-setting shape with the ragged bottoms.
\enditems

\noindent{\bf Floating objects}

\begitems
* \i footnote `\footnote{<mark>}`\z`{<text>}` creates a footnote with given
  `<mark>` and \z`<text>`.

* \i midinsert \i topinsert \i endinsert 
  `\topinsert<object>\endinsert` creates the `<object>` as a 
  \ii floating~object {\em floating object}. It is printed at the top
  of the current page or on the next page.
  `\midinsert<object>\endinsert` does the same as `\topinsert` but it tries
  if the `<object>` fits on the current page. If it is true then it is printed to
  its current position; no floating object is created.
\enditems


\noindent{\bf Controlling of input, output}

\begitems
* \x`\obeyspaces` sets the space as normal, i.e.\ it deactivates special
  treatment of spaces by the tokenizer: more spaces will be more spaces and
  spaces at the beginning of the line are not ignored.
* \x`\obeylines` sets the end of each line as `\par`. Each line in the input is
  one paragraph in the output.

* \x`\bye` finalizes the last page (or last pages if more floating objects
  must be printed) and terminates the \TeX/ job. The \x`\end` primitive command
  does the same but without worrying about floating objects.
\enditems

\noindent{\bf Macros used in math modes}

\begitems
* Spaces in math mode are 
  \ii -comma `\,` (thin space), \ii -greater `\>` (medium space)
  \ii -column `\;` (thick space, but still small), \ii -exclam `\!` (negative thin space).

* \i choose `{<above>\choose<below>}` creates a combination number with
  brackets around it.

* \i sqrt `\sqrt{<math list>}` creates the square root symbol
  with the `<math list>` under it.

* \i root \z`\root<n>\of{<math list>}` creates a general root symbol
  with the order of the root `<n>`.

* \i cases {\def<#1>{\,$\langle#1\rangle$\,}%
  \z`\cases{<case 1>&<condition 1>\cr...\cr<case n>&<condition n>}`
  creates a list of variants (preceded by a brace $\{$) in math mode.

* \i matrix \z`\matrix{<a>&<b>...&<e>\cr...\cr<u>&<v>...&<z>}`
  creates a matrix of given values in math mode (without brackets around it).
  `\pmatrix{<data>}` does the same but with ().

* \i displaylines `$$\displaylines{<formula 1>\cr...\cr<formula n>}$$`
  prints multiple (centered) formulae in display mode.

* \i eqalign 
  `$$\eqalign{<form.1 left>&<form.1 right>\cr...\cr<form.n left>&<form.n right>}$$`\hfil\break
  prints multiple formulae aligned by `&` character in display mode.

* \x`\eqalignno` behaves like `\eqalign` but a second `&` followed by a `<mark>` 
  can be in some lines. These lines place the `<mark>` in the right margin.
  \x`\leqalignno` does the same as `\eqalignno` but `<mark>` is put to the
  left margin.  
}
\enditems

\raggedbottom

\vfil\break

\nonum\sec[index] Index

\iis LaTeX/macros {\LaTeX{} macros}
\iis plain~TeX/macros {plain \TeX{} macros}
\iis OpTeX {\OpTeX}
\iis pdfTeX {pdf\TeX}
\iis luaTeX {Lua\TeX}
\iis XeTeX {\XeTeX}
\iis TeX/engines {\TeX{} engines}
\iis -percent {{\code{\\\%}}}
\iis -at {{\code{\\\&}}}
\iis -dollar {{\code{\\\$}}}
\iis -hash {{\code{\\\#}}}
\iis -space {{\code{\\}{\tt\char9251}}}
\iis -italiccorr {{\code{\\/}}}
\iis -hyphen {{\code{\\-}}}
\iis TeXlive {\TeX{}live}
\iis baselineskiprule {\code{\\baselineskip} rule}
\iis -comma {{\code{\\,}}}
\iis -greater {{\code{\\>}}}
\iis -column {{\code{\\;}}}
\iis -exclam {{\code{\\!}}}
\iis plain~TeX {plain \TeX}

{\let\Blue=\relax \typosize[9/11] \preprocessindex \begmulti 3 \makeindex \endmulti }

\normalbottom

\vfill

\noindent
Petr Olšák {\tt petr@olsak.net}\nl
Czech Technical University in Prague\nl
Version of the text: 0.8 (\the\year-\thed\month-\thed\day)

\break

\end
