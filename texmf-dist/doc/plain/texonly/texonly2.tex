%texonly
%(c)Matthias Borck-Elsner
%begin readme
\newwrite\readme
\immediate\openout\readme=README.md
\immediate\write\readme{
Title: TeXonly
Part: 2
Author: Matthias Borck-Elsner
Date: 18.04.2019
Version: 2
License:  LPPL Ver­sion 1.3c 2008-05-04
Description: A file written with TeX , not using any packgages or sty-files, to be compiled with TeX or pdfTeX only, not with Latex et al. The text ist written in german, maybe people join in and create similar files in their lanquages.
}
% Plain TeX interface to graphicx package.
% David Carlisle
% Plain TeX interface to color package.
% David Carlisle
%\input miniltx.tex
%\input graphicx.tex
% Plain TeX interface to graphicx package.
%David Carlisle

\def\thecommentsfile[#1]{\section[#1]\immediate\closeout\comments \input\jobname.cmt\vskip\baselineskip}%
\newcount\commentsno%
\commentsno=0%
\newwrite\comments%
\immediate\openout\comments=\jobname.cmt%
\def\comment[#1,#2]{\advance\commentsno by 1 \immediate\write\comments{\ifnum\the\commentsno<10\hskip0.5em \fi{\the\commentsno\ {\bf #1} #2\hfil} \vskip0.2\baselineskip}}%
%beginfonts%
\def\myfont{pcrr }%
\def\boldfont{pcrbo  }%
\def\itfont{pcrro }%
\def\slfont{pcrro }% 
\font\titlefont=\myfont scaled 3800%
\font\partfont=\myfont scaled 1600%
\font\sectionfont=\myfont scaled 1400%
\font\subsectionfont=\myfont scaled 1200%
\font\subsubsectionfont=\myfont scaled 1000%
\font\paragraphfont=\myfont scaled 900%
\font\normalfont\myfont scaled 900%
\font\it=\itfont scaled 900%
\font\bf=\boldfont scaled 860%
\font\captionfont=\myfont scaled 600%
\font\smallfont=\myfont scaled 600%
\font\tinyfont=\myfont scaled 600%
\font\notesfont=\myfont scaled 600%
\font\tablenotesfont=\myfont scaled 600%
\comment[fonts, Jeder Font wird einzeln angegeben und skaliert]
%endfonts%
%register
\def\theregistersfile[#1]{\section[#1]\immediate\closeout\registers \input\jobname.regs}%
\newcount\registersno%
\registersno=0%
\newwrite\registers%
\immediate\openout\registers=\jobname.regs%
\def\register[#1,#2]{\advance\registersno by 1 \immediate\write\registers{\ifnum\the\registersno<10\ \fi{\notesfont \the\registersno\ } {\normalfont {#1} #2}\hfill \vskip0.2\baselineskip}}%
%end register
%begin boxes
%end boxes
%begin dimen
\newdimen\paperwidth \comment[paperwidth, Einstellen der Papierbreite]
\paperwidth=210mm
\newdimen\paperheight
\paperheight=297mm\comment[paperheight,Einstellen der Papierhoehe]
%%end dimen
\parindent=0pt\comment[parindent,Einstellen des Erstzeileneinzuges]%
\raggedright \comment[raggedright,Flattersatz rechts]%
\def\mbe{\hfill mbe \hfil} \comment[texonlylogo,Mein Logo]%
\def\texonlylogo{\smallfont \hfill\hbox{\raise16pt\hbox{\mbe}{%
$\sqrt{\TeX only {\raise5pt\hbox{(c)}}}$}\hfil}\vskip\baselineskip}%
\def\creator{Matthias Borck-Elsner}\comment[creator,Name des Autoren]%
\def\date{\the\day.\the\month.\the\year} \comment[date,Datum]%
%2019-03-29%
\hyphenation{%
Norm-ier-ungs%
aus-schliess-lich%
Mo-eg-lich-kei-ten%
auf-ge-ruf-en%
dar-zu-stel-len%
wer-den%
ge-waehl-ten%
Schri-ft-stue-ck%
Ein-stell-ung-en%
}\comment[hyphenation,Trennungsvorgaben]%
%\raggedright%
\parindent=0pt%
%counters%
\newcount\parts\comment[parts,Zaehler fuer Teil]%
\parts=0%
\newcount\sections\comment[sections,Zaehler fuer Abschnitt]%
\sections=0%
\newcount\subsections\comment[subsections,Zaehler fuer Unterabschnitt]%
\subsections=0%
\newcount\subsubsections \comment[subsubsections,Zaehler fuer Unterunterabschnitt]%
\subsubsections=0%
\newcount\paragraphs \comment[paragraphs,Zaehler fuer Paragraph]%
\paragraphs=0%
\newcount\notes \comment[notes,Zaehler fuer Anmerkungen]%
\notes=0%
%endcounters%
%begin notes%
\newwrite\notesfile \comment[newwrite,Erzeugen einer neuen Auslagerungsdatei]%
\immediate\openout\notesfile=\jobname.nt \comment[openout,Oeffnen einer Auslagerungsdatei]%
\def\note[#1]{\advance\notes by 1\raise5pt\hbox{\kern-5pt\notesfont\the\notes}%
\def\currentsections{
\ifnum\the\parts>0 \the\parts\fi
\ifnum\the\sections>0 .\the\sections\fi
\ifnum\the\subsections>0 .\the\subsections\fi
\ifnum\the\subsubsections>0 .\the\subsubsections \fi
\ifnum\the\paragraphs>0 (\the\paragraphs)\fi
}
\immediate\write\notesfile{\notesfont\the\notes\ #1\hfill \currentsections \vskip0.1\baselineskip}}\comment[note, Definieren einer Anmerkung]%
\def\thenotesfile[#1]{\section[#1]\immediate\closeout\notesfile\input\jobname.nt} \comment[thenotesfile,Schliessen und Einfuegen der Auslagerungsdatei]%
% end notes
%begin contents%
\newwrite\contents%
\immediate\openout\contents=\jobname.toc%
\def\thecontentsfile[#1]{\section[#1]\vskip\baselineskip\immediate\closeout\contents \input\jobname.toc}%
\def\pagebreak{\vfill \break}%
%end contents%
%beginsections%
\def\title[#1]{%
\immediate\write\contents{\titlefont #1 \vskip2\baselineskip}%
\vfill \titlefont \hfil #1 \raise10pt\hbox{\tinyfont(c)\hfill} 
\vskip\baselineskip\hfil \sectionfont \creator \hfill
\vskip\baselineskip\hfil \sectionfont \date \hfill \vfill\normalfont}\comment[title,Definition des Titelformats]%
%
\def\part[#1]{\sections=0 \advance\parts by 1%
\immediate\write\contents{\partfont \vskip2\baselineskip #1 \hskip10pt\the\parts\ \hfill\normalfont{\the\pageno}\normalfont}%
\pagebreak\partfont #1 \hskip12pt\the\parts\vskip\baselineskip \normalfont} \comment[part,Definition des Teilformats]%
%
\def\section[#1]{\subsections=0 \advance\sections by 1 %
\immediate\write\contents{\sectionfont \vskip2\baselineskip\the\parts.\the\sections\hskip10pt #1 \hfill {\normalfont\the\pageno}\vskip\baselineskip \normalfont}%
\vskip\baselineskip\sectionfont \the\parts.\the\sections\hskip12pt  #1\vskip\baselineskip \normalfont}\comment[section,Definition des Abschnittsformats]%
%
\def\subsection[#1]{\subsubsections=0 \advance\subsections by 1%
\immediate\write\contents{\subsectionfont \the\parts.\the\sections.\the\subsections\hskip12pt #1 \hfill\normalfont{\the\pageno}\vskip\baselineskip \normalfont}%
\vskip\baselineskip\subsectionfont \the\parts.\the\sections.\the\subsections\hskip12pt #1\vskip\baselineskip \normalfont} \comment[subsection,Definition des Unterabschnittformats]
%
\def\subsubsection[#1]{\paragraphs=0 \advance\subsubsections by 1%
\immediate\write\contents{\subsubsectionfont  \the\parts.\the\sections.\the\subsections.\the\subsubsections\hskip12pt #1 \hfill\normalfont{\the\pageno}\vskip\baselineskip \normalfont}%
\vskip\baselineskip\subsubsectionfont \the\parts.\the\sections.\the\subsections.\the\subsubsections\hskip12pt  #1\vskip\baselineskip \normalfont}\comment[subsubsection,Defintion des Unterabschnittformats]%
%
\def\paragraph[#1]{\advance\paragraphs by 1%
\immediate\write\contents{\paragraphfont (\the\paragraphs)\hskip12pt #1 \hfill\normalfont{\the\pageno}\vskip\baselineskip\normalfont}%
\vskip\baselineskip\paragraphfont (\the\paragraphs)\hskip12pt  #1 \vskip\baselineskip\normalfont}\comment[paragraph,Definition des Paragraphformats]%
%endsections
%tables
\newdimen\cellwidth \comment[cellwidth,Einstellen der Zellenweite fuer Tabellen]%
\newdimen\cellheight \comment[cellheight,Einstellen der Zellenhoehe fuer Tabellen]%
\newcount\jails\comment[jails,Zaehler fuer Tabellen]%
\newcount\cellblocks \comment[cellblocks,Zaehler fuer Untertabellen]%
\newcount\cellfloors \comment[cellfloors,Zaehler fuer Tabellenzeilen]%
\newcount\cells \comment[cells, Zaehler fuer Zellen]%
\newdimen\cellwall \comment[cellwall,Wert fuer Abstand zwischen Zellen und Spalten]%
\def\cellno{\the\jails.\the\cellblocks.\the\cellfloors.\the\cells} \comment[cellno,Anzeige der Zellennummer]%
\def\cellempty{\cellno} \comment[cellempty,Anzeige der Zellennummer falls die Zelle leer ist]%
%\def\cellchains{}\comment[cellchains,Definition zur Aufnahme der cellchain-Kommandos]%
%beg cellchain
\def\cellchain[#1,#2,#3,#4]#5{%
\ifnum\the\jails=#1\ifnum\the\cellblocks=#2%
\ifnum\the\cellfloors=#3\ifnum\the\cells=#4 %
\def\cellempty{#5\vskip\baselineskip}\fi\fi\fi\fi}\comment[cellchain,Definition des Inhalts einer Zelle mit Adresse derselben, abzuarbeiten in cellchains]%
%end cellchain
\newwrite\tablesfile
\immediate\openout\tablesfile=\jobname.tab
\def\thetablesfile[#1]{\immediate\closeout\tablesfile \section[#1] \input\jobname.tab}
\def\jailname#1{\immediate\write\tablesfile{ #1 \the\jails \hfill \currentsections \vskip0.5\baselineskip} #1 \the\jails \vskip0.5\baselineskip}\comment[jailname,Definition Tabellenname]%
%
\def\jail#1{\vskip\baselineskip\advance\jails by 1\cellblocks=0\cellfloors=0\cells=0%
\cellchains #1\vskip2\baselineskip}\comment[jail,Definition einer Tabelle]%
%
\def\cellblockname#1{\immediate\write\tablesfile{#1 \the\jails.\the\cellblocks  \hfill \currentsections \vskip0.5\baselineskip} #1 \the\jails.\the\cellblocks \vskip0.5\baselineskip}\comment[cellblockname,Definition Teiltabellenname]
\def\cellblock#1{\advance\cellblocks by 1\cellfloors=0%
\cellchains#1}\comment[cellblock,Definition einer Untertabelle]%
%
\def\cellfloor#1{\advance\cellfloors by 1%
\hbox{\cellchains#1}}\comment[cellfloor,Definition einer Tabellenzeile]%
%
\def\cell{%
\advance\cells by 1\vtop{\hsize=\cellwidth%
\cellchains\cellempty}\hskip\cellwall\hfill}\comment[cell,Definition einer Tabellenzelle]%
%
\texonlylogo \vskip\baselineskip%
\cellwall=12pt
\cellwidth=4cm
\def\cellchains{%
\cellchain[1,1,1,1]{$\sqrt2=1,4142135624$}
\cellchain[1,1,1,2]{$\sqrt3=1,7320508076$}
\cellchain[1,1,1,3]{$\sqrt4=2$}
\cellchain[1,1,2,1]{$\sqrt6=2,4494897428$}
\cellchain[1,1,2,2]{$\sqrt7=2,6457513111$}
\cellchain[1,1,2,3]{$\sqrt8=2,8284271247$}
}%
%%%%%%%%%%%%end of definitions
\def\text{Lorem ipsum dolor sit amet, {\bf consectetur adipisici elit}, {\it sed eiusmod tempor incidunt} ut labore et dolore magna aliqua. Ut enim ad minim veniam, quis nostrud exercitation ullamco laboris nisi ut aliquid ex ea commodi consequat. Quis aute iure reprehenderit in voluptate velit esse cillum dolore eu fugiat nulla pariatur. Excepteur sint obcaecat cupiditat non proident, sunt in culpa qui officia deserunt mollit anim id est laborum. }
\title[TeXonly]
\part[Teil]%
\section[Abschnitt]
\text\note[Test]
\subsection[Unterabschnitt]
\text
\text\note[Anmerkung]
\paragraph[Paragraph]
\text
\paragraph[Paragraph]
\text
\section[Abschnitt]
\text
\subsection[Unterabschnitt]
\subsubsection[Unterunterabschnitt]
\text
\jail{\jailname{Tabelle}
\cellblock{\cellblockname{Teiltabelle}
\cellfloor{
\cell\cell\cell}
\cellfloor{
\cell\cell\cell}
}}
\jail{\jailname{Tabelle}
\cellblock{\cellblockname{Teiltabelle}
\cellfloor{
\cell\cell\cell}
\cellfloor{
\cell\cell\cell}
}}
\text
\paragraph[Paragraph]

\text
\text\text
\paragraph[Paragraph]

\text
\part[Anhaenge]
\thecontentsfile[Inhaltsverzeichnis]%
\thetablesfile[Tabellenverzeichnis]%
\thenotesfile[Anmerkungen]%
\theregistersfile[Sachregister]%
\thecommentsfile[Befehlsreferenz]%



\bye
