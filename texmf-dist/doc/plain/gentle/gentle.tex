 
 
%% Thanks for your interest in A Gentle Introduction to TeX. 
%% Any comments on this manual would be appreciated. These may 
%% be typesetting, English, or TeX criticisms. If you decide to 
%% translate this document into another language, I'd appreciate 
%% receiving a copy. 
%% 
%% This file is a complete TeX input file.  Just run it 
%% through TeX and print out the resulting "DVI" file. If you are 
%% familiar with TeX, the macros at the top of the file have a few 
%% switches which you may want to set.  If you have problems or 
%% can't run TeX at all, write to me and I'll send you a hard copy. 
%% You can get a bound copy from the TeX Users Group at the address 
%% given below. 
%% 
%% You should feel free to photocopy and/or distribute this manual. 
%% My only request is that it remain in one piece and not be chopped 
%% up.  The only machine dependent section (#1.2) may need to be 
%% rewritten for your local site, of course. 
%% 
%%%                Michael Doob 
%%                 Department of Mathematics 
%%                 The University of Manitoba 
%%                 Winnipeg, Manitoba  R3T 2N2 
%%                 Canada 
%%                 mdoob@uofmcc                    (bitnet) 
%%                 mdoob@ccu.umanitoba.ca          (internet) 
%% 
%% Here is a character listing to check to be sure that no 
%% unwanted translations took place within the bowels of the net. 
%% Upper case letters: ABCDEFGHIJKLMNOPQRSTUVWXYZ 
%% Lower case letters: abcdefghijklmnopqrstuvwxyz 
%% round parentheses, square brackets, curly braces: ()  []  {} 
%% Exclaim, at, sharp, dollar, percent: ! @ # $ % 
%% Caret, ampersand, star, underscore, hyphen: ^ & * _ - 
%% vertical bar, backslash, tilde, backprime, plus: | \ ~ ` + 
%% plus, equal, prime, quote, colon: + = ' " : 
%% less than, greater than, slash, question, comma: < > / ? , 
%% period, semicolon: . ; 
%% 
%% 
%% 
%% Now here come the macros used in this manual. If you are 
%% already familiar with TeX, you may want to fiddle with them. 
%% In particular, the hooks are left to generate a new control 
%% word index and table of contents if you change section 1.2. 
%% 
%% 
 
 
 
%%%%%%%% Here are the fonts other than the sixteen defined in %%%%%%%% 
%%%%%%%%%%%%%        plain.tex that are used        %%%%%%%%%%%%%%%%%% 
 
       %%%% first, choose between amr or cmr fonts %%% 
\newif \ifamrfonts 
\amrfontsfalse  % use this line if you use the cmr fonts 
%\amrfontstrue  % use this line if you use the old amr fonts 
 
\ifamrfonts \font\brm=amr10 scaled \magstep1 
      \else \font\brm=cmr10 scaled \magstep1 \fi 
\ifamrfonts \font\halfrm=amr10 scaled \magstephalf 
      \else \font\halfrm=cmr10 scaled \magstephalf \fi 
\ifamrfonts \font\bbrm=amr10 scaled \magstep2 
      \else \font\bbrm=cmr10 scaled \magstep2 \fi 
\ifamrfonts \font\bbbrm=amr10 scaled \magstep3 
      \else \font\bbbrm=cmr10 scaled \magstep3 \fi 
\ifamrfonts \font\bbbbrm=amr10 scaled \magstep4 
      \else\font \bbbbrm=cmr10 scaled \magstep4 \fi 
\ifamrfonts \font\bbbbbrm=amr10 scaled \magstep5 
      \else \font\bbbbbrm=cmr10 scaled \magstep5 \fi 
\ifamrfonts \font\sf = amssmc10 
      \else \font\sf = cmss10 \fi 
\ifamrfonts \font\chapfont=ambx10 scaled \magstep2 
      \else \font\chapfont=cmbx10 scaled \magstep2 \fi 
\ifamrfonts \font\secfont=ambx10 scaled \magstep1 
      \else \font\secfont=cmbx10 scaled \magstep1 \fi 
\ifamrfonts \font\sc= amcsc10 
      \else \font\sc= cmcsc10 \fi 
%%%%%%%%%%%%%%%%%%%%%%%%%%%%%%%%%%%%%%%%%%%%%%%%%%%%%%%%%%%%%%%%%%%%% 
 
 
%%%%\vbadness=10000 %%%%%% I don't want to hear about underfull vboxes 
\raggedbottom      %%%%% delete this line for aligned page bottoms 
\hsize=5.5in 
\vsize=7in 
\voffset=.75in 
\parskip=\baselineskip 
\widowpenalty=1000 \clubpenalty=1000  %%% I hate widows and orphans! %%% 
 
%%%%%%%%% choosing between Canadian and American spellings %%%%%%%%% 
\newif \ifcanspell 
\canspelltrue      %%%% Canadian spelling 
%\canspellfalse    %%% use this line for American spelling 
 
\def\aesthetic{\ifcanspell \ae{}sthetic\else esthetic\fi} 
\def\analogue{\ifcanspell analogue\else analog\fi} 
     \let\analog=\analogue 
\def\cancelled{\ifcanspell cancelled\else canceled\fi} 
     \let\canceled=\cancelled 
\def\centimetre{\ifcanspell centimetre\else centimeter\fi} 
     \let\centimeter=\centimetre 
\def\centre{\ifcanspell centre\else center\fi} 
     \let\center=\centre 
\def\centred{\ifcanspell centred\else centered\fi} 
     \let\centered=\centred 
\def\our{\ifcanspell our\else or\fi} 
\def\postcode{\ifcanspell postalcode\else zipcode\fi} 
\def\province{\ifcanspell province\else state\fi} 
\def\theatre{\ifcanspell theatre\else theater\fi} 
     \let\theater=\theatre 
%%%%%%%%%%%%%%%%%%%%%%%%%%%%%%%%%%%%%%%%%%%%%%%%%%%%%%%%%%%%%%%%%%%%% 
 
%%%%%%%%%% First check to see if using plain format %%%%%%%%%%%%%%%%% 

\def\shouldbefmt{plain}
\ifx\shouldbefmt\fmtname
\else
\immediate\write16{}
\immediate\write16{***** WARNING WARNING WARNING *****}
\immediate\write16{You probably need to use the command}
\immediate\write16{tex \string<filename\string>}
\immediate\write16{to make your dvi file (not latex).}
\immediate\write16{***** WARNING WARNING WARNING *****}
\immediate\write16{}
\fi

%%%%%%%%%%%%%%%%%%%%%%%%%%%%%%%%%%%%%%%%%%%%%%%%%%%%%%%%%%%%%%%%%%%%% 
 
%%%%%%%%%%%%%%%%%%%%%%% headline routines %%%%%%%%%%%%%%%%%%%%%%%%%%%% 
\def\gentleheadline{% 
    \vbox {\hrule% 
        \line {\strut \vrule \quad \tenrm A \TeX{} intro 
              \ifcanspell (Canadian \else (U.S. \fi spelling) 
              \hfil 
              \ifnum \secnum > 0 Section \the\secnum: \fi \sectiontitle \quad 
              \vrule}% 
        \hrule}% 
} 
 
\newif \iftitlepage    \titlepagetrue 
\headline= 
  {\iftitlepage \hfil \global\titlepagefalse \else \gentleheadline \fi} 
%%%%%%%%%%%%%%%%%%%%%%%%%%%%%%%%%%%%%%%%%%%%%%%%%%%%%%%%%%%%%%%%%%%%% 
 
%%%%%%%%%%%%%%%%%%%% define contents and index files %%%%%%%%%%%%%%%%%%%%%% 
%%% Normally the contents and index are hard coded into the input file. %%% 
%% To generate new ones, use \writingcontentstrue and \writingindextrue. %% 
 
\newif \ifwritingcontents 
\newif \ifwritingindex 
\newwrite\contents         \newwrite\index 
 
\writingcontentsfalse      \writingindexfalse 
%%% \writingcontentstrue  %%% use this line to make the contents 
%%% \writingindextrue     %%% use this line to make the index 
 
\ifwritingcontents \openout\contents=contents.tex     \fi 
\ifwritingindex \openout\index=index.tex 
          \def\toindex#1{\immediate\write\index{#1 \the\pageno}} 
          \else \def\toindex#1{} \fi 
%%%%%%%%%%%%%%%%%%%%%%%%%%%%%%%%%%%%%%%%%%%%%%%%%%%%%%%%%%%%%%%%%%%%%%%%%%% 
 
%%%%%%%%%%%%%%%%%%%% determine whether answers are printed %%%%%%%%%%%%%%%% 
\newif \ifwritinganswers 
\writinganswersfalse        %%% use this line to suppress answer section 
\writinganswerstrue         %%% use this line to include answer section 
%%%%%%%%%%%%%%%%%%%%%%%%%%%%%%%%%%%%%%%%%%%%%%%%%%%%%%%%%%%%%%%%%%%%%%%%%%% 
 
%%%%%%%%%%%%%%%%      footnote macro with counter      %%%%%%%%%%%% 
\newcount\footnotenum \footnotenum=0 
\def\fnote#1{\advance \footnotenum by 1% 
\footnote{$^{\the\footnotenum}$}{#1}} 
%%%%%%%%%%%%%%%%%%%%%%%%%%%%%%%%%%%%%%%%%%%%%%%%%%%%%%%%%%%%%%%%%%% 
 
 
 
%%%%%%%%%%%%%%%% exercise, section, and subsection macros %%%%%%%%%%%% 
\newcount\exno                   %%%%%% counter for exercises %%%%%%%% 
\newcount\secnum \secnum=0       %%%% counter for section numbers %%%% 
\newcount\subsecnum 
 
\def\section#1{ 
            \vfill\eject %%%%% new section starts on a new page 
            %%%\ifodd\pageno \else\ \vfill\eject \fi %start on an odd page 
            \advance\secnum by 1 \subsecnum=0 \exno=0 
            \ifnum \secnum = 1 \pageno=1 \fi 
            \ifnum \secnum > 0 
                 \leftline{\chapfont Section \the\secnum} 
                 \vskip 3pt \fi 
            \leftline{\chapfont #1} 
            \def\sectiontitle{#1} 
            \vskip\baselineskip 
            \hrule 
            \vskip 1cm 
            \ifwritingcontents \write\contents{\string\line\string{#1 
                   \string\dotfill{} 
                   \ifnum \pageno < 0 \romannumeral-\pageno 
                   \else \the\pageno \fi 
                   \string}}\fi 
            \titlepagetrue} 
 
\def\subsection#1{\advance\subsecnum by 1 
               \vskip 30pt 
               \leftline{\secfont \the\secnum .\the\subsecnum\ #1} 
               \nobreak 
               \ifwritingcontents 
                     \write\contents{\string\line\string{\string\qquad{}#1 
                     \string\dotfill{} \the\pageno\string}}\fi 
                 } 
 
\def\exercise{\global\advance \exno by 1 
         \vskip\baselineskip 
         \noindent $\triangleright$ Exercise \the\secnum.\the\exno\quad 
         } 
%%%%%%%%%%%%%%%%%%%%%%%%%%%%%%%%%%%%%%%%%%%%%%%%%%%%%%%%%%%%%%%%%%%%%%%%%%% 
 
 
%% definitions of control sequences for characters in the typewriter font %%% 
%%%% for short phrases this is easier than using a literal construction %%%% 
\def\\{\char92{}}          %%%%% backslash %%%%% 
\def\lb{\char'173{}}       %%%%% left brace %%%%% 
\def\rb{\char'175{}}       %%%%% right brace %%%%% 
\def\sp{\char32{}}         %%%%% special space symbol %%%%% 
 
\def\beginliteral{ 
\vskip\baselineskip 
\begingroup 
\tt 
\obeylines 
%{\obeyspaces\global\let =\ } 
\catcode`\@=0 
\parskip=0pt\parindent=0pt 
\catcode`\$=12\catcode`\&=12\catcode`\^=12\catcode`\#=12 
\catcode`\_=12\catcode`\~=12 
\def\par{\leavevmode\endgraf} 
\catcode`\{=12\catcode`\}=12\catcode`\%=12\catcode`\\=12 
} 
 
\def\endliteral{\endgroup} 
%%%%%%%%%%%%%%%%%%%%%%%%%%%%%%%%%%%%%%%%%%%%%%%%%%%%%%%%%%%%%%%%%%%%% 
 
 
%%%%%%%%%%%%%% inhibit hyphenation of typewriter text %%%%%%%%%%%%%% 
\hyphenchar\tentt=-1 
 
 
%%%%%%%%%%%% grouping to make input listing in typewriter type %%%%%%% 
\def\beginuser{\vskip\parskip 
               \everypar={\nobreak} 
               \begingroup 
               \tt \obeylines \parskip=0pt \parindent=0pt} 
 
\def\enduser{\endgroup} 
 
 
%%%%%%%%%%% macro to construct tables (easily) %%%%%%%%%%%%%%%% 
%%% parameters: title goes between brackets, rest of the 
%%%             paragraph is the table 
\def\maketable[#1]#2\par{ 
\setbox1=\vbox{#2} 
$$\vbox{ 
        \hbox to \wd1{\bf \hss #1 \hss} 
        \vskip 12pt 
        \box1 
       } 
$$ 
} 
%%%%%%%%%%%  end of macro to construct tables  %%%%%%%%%%%%%%%% 
 
 
 
%%%%%%%%%%% macro to put a box around the text %%%%%%%%%%%%%%%% 
\def\makebox#1#2#3% vsize, hsize, inserted text 
{\hbox{\vrule 
       \vbox to  #1{\hrule \vss 
                   \hbox to #2{\hss#3\hss}\vss 
                   \hrule}\vrule}} 
 
\def\displaytext#1{$$\hbox{#1}$$} 
 
\def\LaTeX{{\rm L\kern-.36em\raise.3ex\hbox{\sc a}\kern-.15em 
   T\kern-.1667em\lower.7ex\hbox{E}\kern-.125emX}} 
 
\def\AMSTeX{{$\cal A$}\kern-.1667em\lower.5ex\hbox 
 {$\cal M$}\kern-.125em{$\cal S$}-\TeX} 
 
 
%%%%%%%%%%%% macro to put TeX references in right margin %%%%%%%% 
\newdimen\theight 
\def \TeXref#1{% 
             \vadjust{\setbox0=\hbox{\sevenrm\quad\quad\TeX book: #1}% 
             \theight=\ht0 
             \advance\theight by \dp0    \advance\theight by \lineskip 
             \kern -\theight \vbox to \theight{\rightline{\rlap{\box0}}% 
             \vss}% 
             }}% 
%%%%%%%%%%%%%%%%%%%%%%%%%%%%%%%%%%%%%%%%%%%%%%%%%%%%%%%%%%%%%%%%%%%%%%%%% 
 
\def\version{1.1} 
 
%%%%%%%%%%%% macro to write the date out the date and time %%%%%%%%%%% 
%%%%%%%%%%%%          TeXbook   p. 406 for date            %%%%%%%%%%% 
\def\today{\ifcase\month\or January\or February\or March\or April\or 
May\or June\or July\or August\or September\or October\or November\or 
December\fi \space\number\day, \number\year} 
 
\newcount\hour \newcount\minute 
\hour=\time  \divide \hour by 60 
\minute=\time 
\loop  \ifnum \minute > 59 \advance \minute by -60 \repeat 
\def\writetime{\ifnum \hour<13 \number\hour:%  % supresses leading 0's 
                      \ifnum \minute<10 0\fi%  % so add it it 
                      \number\minute 
                      \ifnum \hour < 12 \ A.M. \else \ P.M. \fi 
      \else \advance \hour by -12 \number\hour:%  % supresses leading 0's 
                      \ifnum \minute<10 0\fi%     % add it in 
                      \number\minute \ P.M. \fi} 
 
\def\datestamp{\vfill 
    \rightline{\sevenrm Gentle Intro \version{} run through \TeX{} 
               on \today{} at \writetime}} 
%%%%%%%%%%%%%%%%%%%%%%%%%%%%%%%%%%%%%%%%%%%%%%%%%%%%%%%%%%%%%%%%%%%%% 
 
\titlepagetrue 
{\nopagenumbers 
\def\sectiontitle{Introduction} 
\topinsert \vskip 6 cm \endinsert 
\centerline{\chapfont A Gentle Introduction to \TeX{}} 
\vskip 15 pt 
\centerline{\secfont A Manual for Self-study} 
\vskip 3cm 
\leftline{Michael Doob} 
\leftline{Department of Mathematics} 
\leftline{The University of Manitoba} 
\leftline{Winnipeg, Manitoba, Canada R3T 2N2} 
\vskip\baselineskip 
\leftline{MDOOB@UOFMCC.BITNET} 
\leftline{MDOOB@CCU.UMANITOBA.CA} 
 
\vfill\eject 
\titlepagetrue. 
\vfill\eject 
} 
\secnum=-2 \pageno=-1 
 
\section{Introduction} 
 
First the bad news: \TeX{} is a large and complicated program that 
goes to extraordinary lengths to produce attractive typeset material. 
This very complication can cause unexpected things to happen at times. 
Now the good news: straightforward text is very easy to typeset using 
\TeX\null.  So it's possible to start with easier text and work up to 
more complicated situations. 
 
The purpose of this manual is to start from the very beginning and 
to move towards these more complicated situations.  No previous knowledge 
of \TeX{} is assumed. By proceeding a section at a time, greater 
varieties of text can be produced. 
 
Here are a few suggestions: there are some exercises in each 
section.  Be sure and do them! The only way to learn \TeX{} is 
by using it.  Better yet, experiment on your own; try to make 
some variations on the exercises.  There is no way that you can 
damage the \TeX{} program with your experiments.  You can find a 
complete answer to most exercises by looking at the \TeX{} source 
file that was used to produce this document.  You'll notice that 
there are references in the right margin to {\bf The \TeX 
book}\fnote{Addison-Wesley, Reading,Massachusetts, 1984, ISBN 
0-201-13488-9}\null.  When you feel that you want more 
information on a topic, that's where to look. 
 
Incidentally, there are a few fibs that appear in this manual; 
they are used to hide complications (I look at this as something 
like poetic license).  As you become more experienced at using 
\TeX{}, you'll be able to find them. 
 
\TeX{} is a public domain program that is available for no 
license fee.  It was developed by Donald Knuth at Stanford 
University as a major project. In the profit-oriented market 
place, the program would certainly cost many thousands of 
dollars.  The \TeX{} Users Group (TUG) is a nonprofit 
organization which distributes copies of \TeX{}, this manual, 
updates software, and gives information about new developments in 
both hardware and software in its publications TUGboat and \TeX 
niques.  Joining this users group is inexpensive; please consider 
doing so.  The address is: 
\vskip\baselineskip 
\centerline{\TeX{} Users Group} 
\centerline{P.O. Box 869}  
\centerline{Santa Barbara, CA 93102}
\centerline{U.S.A.}
 
This manual would not have come into being without the aid of others. 
In particular the proofreading and suggestions of the following 
people have been invaluable: 
\begingroup 
\frenchspacing 
Waleed A.~Al-Salam (University of Alberta), 
Debbie L.~Alspaugh (University of California), 
Nelson H.~F.~Beebe (University of Utah), 
Barbara Beeton (American Mathematical Society), 
Anne Br\"uggemann-Klein (University of Freiburg), 
Bart Childs (Texas A.~\&~M\null. University), 
Mary Coventry (University of Washington), 
Dimitrios Diamantaras (Temple University), 
Roberto Dominimanni (Naval Underwater Systems Center), 
Lincoln Durst (Providence, RI), 
Victor Eijkhout (University of Nijmegen), 
Moshe Feder (St. Lawrence University), 
Josep~M.~Font (Universidad Barcelona), 
Jonas de Miranda Gomes 
        (Instituto de Matematica Pura e Aplicada - Brazil), 
Rob Gross (Boston College), 
Klaus Hahn (University of Marburg), 
Anita Hoover (University of Delaware), 
J\"urgen Koslowski (Macalester College), 
Kees van der Laan (Rijksuniversiteit Groningen), 
John Lee (Northrop Corporation), 
Silvio Levy (Princeton University), 
Robert Messer (Albion College), 
Emily H.~Moore (Grinnell College), 
Young Park (University of Maryland), 
Craig Platt (University of Manitoba), 
Kauko Saarinen (University of Jyv\"askyl\"a), 
Jim Wright (Iowa State University), 
and 
Dominik Wujastyk (Wellcome Institute for the History of Medicine). 
 \endgroup 
 
In addition several people have sent me parts or all of their local 
manuals.  I have received a few others on the rebound. 
In particular 
\begingroup 
\frenchspacing 
Elizabeth Barnhart (TV Guide), 
Stephan v.~Bechtolsheim (Purdue University), 
Nelson H.~F.~Beebe (University of Utah) 
             and Leslie Lamport (Digital Equipment Corporation), 
Marie McPartland-Conn and Laurie Mann (Stratus Computer), 
Robert Messer (Albion College), 
Noel Peterson (Library of Congress), 
Craig Platt (University of Manitoba), 
Alan Spragens (Stanford Linear Accelerator Center, now of Apple Computers), 
Christina Thiele (Carleton University), 
and Daniel M.~Zirin (California Institute of Technology) 
\endgroup 
have written various types of lecture notes that have been most helpful. 
 
\vfill \eject 
\section{Contents} 
 
\line{Introduction \dotfill{} i} 
\line{Contents \dotfill{} iii} 
\line{1.~Getting Started \dotfill{} 1} 
\line{\qquad{}1.1 What \TeX{} is and what \TeX{} isn't \dotfill{} 1} 
\line{\qquad{}1.2 From \TeX{} file to readable output, 
                                        the big set up \dotfill{} 2} 
\line{\qquad{}1.3 Let's do it! \dotfill{} 4} 
\line{\qquad{}1.4 \TeX{} has everything under control \dotfill{} 7} 
\line{\qquad{}1.5 What \TeX{} won't do \dotfill{} 8} 
\line{2.~All characters great and small \dotfill{} 9} 
\line{\qquad{}2.1 Some characters are more special than others \dotfill{} 9} 
\line{\qquad{}2.2 Typesetting with an accent \dotfill{} 10} 
\line{\qquad{}2.3 Dots, dashes, quotes, 
       $\textfont0=\tenbf \mathinner {\ldotp \ldotp \ldotp }$ \dotfill{} 13} 
\line{\qquad{}2.4 Different fonts \dotfill{} 15} 
\line{3.~The shape of things to come \dotfill{} 19} 
\line{\qquad{}3.1 Units, units, units \dotfill{} 19} 
\line{\qquad{}3.2 Page shape \dotfill{} 20} 
\line{\qquad{}3.3 Paragraph shape \dotfill{} 22} 
\line{\qquad{}3.4 Line shape \dotfill{} 26} 
\line{\qquad{}3.5 Footnotes \dotfill{} 27} 
\line{\qquad{}3.6 Headlines and footlines \dotfill{} 28} 
\line{\qquad{}3.7 Overfull and underfull boxes \dotfill{} 29} 
\line{4.~$\Bigl\{$Groups, $\bigl\{$Groups, $\{$and More% 
 \setbox0=\hbox{$\biggl\{$} \vrule height \ht0 depth \dp0 width 0pt 
    Groups$\}\bigr\}\Bigr\}$ \dotfill{} 31} 
\line{5.~No math anxiety here! \dotfill{} 33} 
\line{\qquad{}5.1 Lots of new symbols \dotfill{} 33} 
\line{\qquad{}5.2 Fractions \dotfill{} 37} 
\line{\qquad{}5.3 Subscripts and superscripts \dotfill{} 38} 
\line{\qquad{}5.4 Roots, square and otherwise \dotfill{} 39} 
\line{\qquad{}5.5 Lines, above and below \dotfill{} 39} 
\line{\qquad{}5.6 Delimiters large and small \dotfill{} 40} 
\line{\qquad{}5.7 Those special functions \dotfill{} 41} 
\line{\qquad{}5.8 Hear ye, hear ye! \dotfill{} 42} 
\line{\qquad{}5.9 Matrices \dotfill{} 43} 
\line{\qquad{}5.10 Displayed equations \dotfill{} 45} 
\line{6.~All in a row \dotfill{} 48} 
\line{\qquad{}6.1 Picking up the tab \dotfill{} 48} 
\line{\qquad{}6.2 Horizontal alignment with more sophisticated patterns 
      \dotfill{} 51} 
\line{7.~Rolling your own \dotfill{} 55} 
\line{\qquad{}7.1 The long and short of it \dotfill{} 55} 
\line{\qquad{}7.2 Filling in with parameters \dotfill{} 57} 
\line{\qquad{}7.3 By any other name \dotfill{} 60} 
\line{8.~To err is human \dotfill{} 62} 
\line{\qquad{}8.1 The forgotten bye \dotfill{} 62} 
\line{\qquad{}8.2 The misspelled or unknown control sequence \dotfill{} 62} 
\line{\qquad{}8.3 The misnamed font \dotfill{} 64} 
\line{\qquad{}8.4 Mismatched mathematics \dotfill{} 64} 
\line{\qquad{}8.5 Mismatched braces \dotfill{} 65} 
\line{9.~Digging a little deeper \dotfill{} 68} 
\line{\qquad{}9.1 Big files, little files \dotfill{} 68} 
\line{\qquad{}9.2 Larger macro packages \dotfill{} 69} 
\line{\qquad{}9.3 Horizontal and vertical lines \dotfill{} 70} 
\line{\qquad{}9.4 Boxes within boxes \dotfill{} 72} 
\line{10.~Control word list \dotfill{} 77} 
\ifwritinganswers 
    \line{11.~I get by with a little help\dotfill{} 80} 
\fi 
 
 
 
\section{Getting Started} 
 
\subsection{What \TeX{} is and what \TeX{} isn't} 
 
First of all, let's see what steps are necessary to produce a 
document using \TeX\null.  The first step is to type the file 
that \TeX{} reads. This is usually called the \TeX{} file or the 
input file, and it can be created using a simple text editor (in 
fact, if you're using a fancy word processor, you have to be sure 
that your file is saved in ASCII or nondocument mode without any 
special control characters). The \TeX{} program then reads your 
input file and produces what is called a DVI file (DVI stands for 
DeVice Independent). This file is not readable, at least not by 
humans. The DVI file is then read by another program (called a 
device driver) that produces the output that is readable by 
humans\TeXref{23}. Why the extra file? The same DVI file can be 
read by different device drivers to produce output on a dot 
matrix printer, a laser printer, a screen viewer, or a 
phototypesetter. Once you have produced a DVI file that gives 
the right output on, say, a screen viewer, you can be assured 
that you will get exactly the same output on a laser printer 
without running the \TeX{} program again. 
 
The process may be thought of as proceeding in the following way: 
$$ 
{\hbox{edit text\quad} \atop \longrightarrow} 
\lower .6cm \makebox{1.5cm}{1.8cm}% 
{\vbox{\hbox{\TeX{}} 
       \hbox{input} 
       \hbox{file} 
       }% 
}% 
{\hbox{\quad \TeX{} program\quad} \atop \longrightarrow} 
\lower .6cm \makebox{1.5cm}{1.8cm}% 
{\vbox{\hbox{DVI} 
       \hbox{file}} 
}% 
{\hbox{\quad device driver\quad} \atop \longrightarrow} 
\lower .6cm \makebox{1.5cm}{1.8cm}% 
{\vbox{\hbox{readable} 
       \hbox{output}} 
}% 
$$ 
 
This means that we don't see our output in its final form when it 
is being typed at the terminal. But in this case a little 
patience is amply rewarded, for a large number of symbols not 
available in most word processing programs become available. In 
addition, the typesetting is done with more precision, and the 
input files are easily sent between different computers by 
electronic mail or on a magnetic medium. 
 
Our focus will be on the first step, that is, creating the \TeX{} 
input file and then running the \TeX{} program to produce 
appropriate results.  There are two ways of running the \TeX{} 
program; it can be run in batch mode or interactively.  In 
batch mode you submit your \TeX{} input file to your computer; 
it then runs the \TeX{} program without further intervention and 
gives you the result when it is finished.  In interactive mode 
the program can stop and get further input from the user, that 
is, the user can interact with the program.  Using \TeX{} 
interactively allows some errors to be corrected by the user, 
while the \TeX{} program makes the corrections in batch mode as 
best it can. Interactive is the preferred mode, of course. All 
personal computer and many mainframe implementations are 
interactive.  On some mainframes, however, the only practical 
method of running \TeX{} is in batch mode. 
 
 
\subsection{From \TeX{} file to readable output, the big set up} 
\vskip\baselineskip 
 
{\parskip = 0pt \noindent 
[Note from MD: 
{\sl This is the only system dependent section in the manual and 
may be replaced by a local guide. No reference is made to it 
outside of the section itself. The following local information 
should be included: 
\item{$\bullet$} What initial steps, if any, should be taken by the reader 
to permit the running of \TeX{} and your local device driver(s). 
\item{$\bullet$} How to run \TeX\null. 
\item{$\bullet$} How to read the log file. 
\item{$\bullet$} How to preview and/or print the dvi file. 
 
The following sample is applicable here at the University of 
Manitoba.  We use a locally written editor (MANTES) on an 
Amdahl running MVS; I'm fairly certain that it's the worst possible case.}] 
} 
 
 
In this section we'll see how to run \TeX{} at the University of 
Manitoba.  It is assumed that the reader is familiar with MANTES 
and can create text files using it. 
 
First, there are several things that must be done {\bf one time 
only}. To start you must do the following (you type in the 
material that looks like typewriter type): 
\item{(1)} allocate the files that \TeX{} will use by typing the 
following lines (while in MANTES): 
\itemitem{}  C: {\tt alloc da=source.tex format=vb,256,6144} 
\itemitem{}  C: {\tt alloc da=dvi format=fb,1024,6144} 
 
\item{(2)} Create a file called RUNTEX in your MANTES aggregate 
containing the following JCL: 
 
\beginuser 
//        JOB  ,'RUN TEX' 
//        EXEC TEXC 
//SRC     DD DSN=<userid>.SOURCE.TEX,DISP=SHR 
//DVI     DD DSN=<userid>.DVI,DISP=OLD 
\enduser 
 
 
The name {\tt <userid>} is replaced by your own user id, of 
course. The use of upper case and the spaces must be followed 
exactly. 
 
\item{(3)} Create a file called PRINTTEX in your MANTES aggregate 
containing the following JCL: 
 
\beginuser 
//        JOB  ,'PRINT TEX' 
//        EXEC TEXP 
//DVIFILE DD DSN=<userid>.DVI,DISP=SHR 
\enduser 
 
 
Once you have completed these three steps, you are ready to run a 
\TeX{} job. The files you have created will allow you to produce 
about ten pages of ordinary text. 
 
Here are the steps you use {\bf each time you run a job}. 
 
\item{(1)} create a MANTES file containing your \TeX{} input. 
\item{(2)} save and submit your file using the commands 
\itemitem{} C: {\tt save f/l to da=source.tex noseq} 
\itemitem{} C: {\tt submit runtex} 
\item{(3)} when you get a message saying that your job is finished, 
enter the command 
\itemitem{} C: {\tt out <jobname>; list ttyout} 
 
In this command, {\tt <jobname>} is replaced by your user id with 
a dollar sign appended.  This file listing will tell you of any 
errors that might have occurred.  It is an abbreviated version of 
what is called the ``log file''; we will use the term ``log 
file'' to refer to the ttyout file produced by \TeX\null. 
 
If you want, you can check on the status of your job while it is 
executing by using the command 
\itemitem{} C: {\tt q <jobname>} 
 
When you are finished looking at the log file, the command {\tt 
end scratch} will throw away the log file while the command {\tt 
end release} will send the log file to the printer, and it can 
then be picked up with your \TeX{} output. 
 
\item{(4)} when your output from RUNTEX program is error free, \TeX{} 
will have created a legal DVI file.  To print it, use the 
command 
\itemitem{} C: {\tt submit printtex nohold} 
 
As in (3), you can check on the status of your job while it is 
executing. 
\item{(5)} Pick up your output at the I/O window, sixth floor 
Engineering building.  It usually takes about twenty minutes for the 
output to be ready.  Ask for it by ``{\tt <jobname>}''. 
 
The files created are large enough for running \TeX{} jobs of about 
10 pages.  A job of that size will take about one second of CPU time 
to run through \TeX\null.  It will take about 15 seconds of CPU time to 
print 10 pages on the Xerox 8600 using the current device driver. 
 
You can print your own copy of this manual using the command 
{\tt \%docu tex}. You can also find lots of other useful 
online information about \TeX{} by using {\tt \%texinfo}. 
 
\subsection{Let's do it!} 
 
So, from our viewpoint, the name of the game is to create the \TeX{} 
file that produces the right readable output. What does 
a \TeX{} file look like? It consists of characters from an 
ordinary terminal, that is, upper and lower case letters, 
numbers, and the usual punctuation and accent characters (these 
are the usual ASCII characters). Text, for the most part, is just 
typed normally. Special instructions usually start with one of a 
few special symbols such as {\tt \#} or {\tt \&} (these will be 
described in more detail later). Here is an example of a \TeX{} 
input file: 
 
\beginuser 
Here is my first \\TeX\\ sentence. 
\\bye 
\enduser 
\toindex{TeX{} } \toindex{ } 
 
 
First note that the characters in this example look like 
typewriter type. We use these characters with all examples that 
are meant to be typed from the computer terminal. Second, note 
that the backslash appears three times in the text. We'll soon 
see that this is one of the special symbols mentioned 
previously, and it is used very frequently when making \TeX{} 
documents. Make a file containing this example. Use the \TeX{} 
program to make a DVI file and a device driver to see what you 
have produced. If all goes well, you'll have a single page with 
the following single sentence: 
 
Here is my first \TeX\ sentence. 
 
  There will also be a page number at the bottom of the page. If 
you've gotten this far, congratulations! Once you can produce one 
\TeX{} document, it's just a matter of time before you can do 
more complicated ones. Now let's compare what we typed in with 
what we got out. The straightforward words were just typed in 
normally, and \TeX{} set them in ordinary type. But the word ``\TeX'', 
which can't be typed in on a terminal because the letters 
aren't on the same line, is entered by using a word starting with 
a backslash, and \TeX{} made the proper interpretation. Most 
symbols that are not ordinary letters, numbers, or punctuation 
are typeset by entering a word starting with a backslash. If we 
look a little closer, we'll note that the word ``first'' is also 
changed. 
\TeXref{4} 
The first two letters have been joined together and there isn't 
a separate dot over the letter ``i''. This is standard 
typesetting practice: certain letter combinations are joined up 
to form what are called {\sl ligatures}. There is actually a good 
\aesthetic{} reason for this. Compare the first two letters of 
``first'' and ``{f}irst'' to see the difference. We note that 
{\tt \\bye} appears in the input file with no corresponding word 
in the final copy. 
\toindex{bye} 
This a typesetting instruction that tells \TeX{} that the input 
is finished. We'll learn about lots of different typesetting 
instructions as we go along. 
 
Let's look at the log file that was created when we ran \TeX 
\null. It may vary slightly at your site, but should look 
something like this: 
\beginuser 
   1. 
   2.  This is TeX, MVS Version 2.9 (no format preloaded) 
   3.   ** File  PLAIN.FMT <-- DD=FMTLIB MEM=PLAIN 
   4. 
   5.   ** File  SRC.TEX <-- DD=SRC 
   6.  (src.tex [1] 
   7.  Output written on DVI (1 page, 256 bytes). 
   8.  Transcript written on TEXLOG. 
\enduser 
 
 
This is the file that will contain any error messages. On line 6, 
{\tt (src.tex\ } indicated that \TeX{} has started reading that 
file. The appearance of {\tt [1]} indicates that page 1 has been 
processed. If there were errors on page 1, they would be listed 
at that point. 
 
\vskip .5cm 
\exercise Add a second sentence to your original \TeX{} file to get: 
\beginuser 
Here is my first \\TeX\\ sentence. 
I was the one who typeset it! 
\\bye 
\enduser 
 
Use \TeX{} and look at your output. Is the second sentence on a 
new line? 
 
 
\exercise Now add this line to the beginning of your file: 
\beginuser 
\\nopagenumbers 
\enduser 
\noindent 
Guess what will happen when you run the new file through 
\TeX\null.  Now try it and see what happens. 
\toindex{nopagenumbers} 
 
\exercise Add three or four more sentences to your file. 
Use letters, numbers, periods, commas, question marks, 
and exclamation points, but don't use any other symbols. 
 
 
\exercise Leave a blank line and add some more sentences. 
You can now get new paragraphs. 
\medskip 
 
We have now seen a major principle concerning the preparation of 
\TeX{} input files. The placement of the text on your computer 
terminal does not necessarily correspond to the placement of the 
text on your output. You can not, for example, add space between 
words in your output by adding spaces in your input file. Several 
consecutive spaces and one space will produce exactly the same 
output. As would be expected, a word at the end of one line will 
be separated from the first word of the following line. In fact, 
sometimes when working on a file that will be heavily edited, it 
is convenient to start each sentence on a separate line. Spaces 
at the beginning of a line, however, are always ignored. 
 
\exercise Add the following sentence as a new paragraph, and then 
typeset it: 
\beginuser 
Congratulations! You received a grade of 100\% on your latest 
examination. 
\enduser 
\noindent 
The {\tt\%} sign is used for comments in your input file. 
Everything on a line following this symbol is ignored. Notice 
that even the space that normally separates the last word on one 
line from the first word on the next line is lost. Now put a 
backslash in front of the {\tt\%} sign to correct the 
sentence. \toindex{\$} \toindex{\%} 
 
\exercise Add the following sentence as a new paragraph: 
\beginuser 
You owe me \$10.00 and it's about time you sent it to me! 
\enduser 
\noindent 
This will produce an error in your log file (if your 
implementation of \TeX{} is interactive, that is, sends you 
messages and waits for answers, just hit the carriage return or 
enter key when you get the error message). You will get output, 
but not what you might expect. Look at the log file and see 
where the error messages are listed. Don't worry about the 
actual messages. We'll have a lot more to say about errors 
(including this one) later. Now fix the error by putting a 
backslash in front of the {\tt \$} sign, and typeset the result 
(there are a small number of characters like the per cent and 
dollar signs that \TeX{} uses for its own purposes. A table of 
these characters will be provided shortly). 
 
\subsection{\TeX{} has everything under control} 
 
We have seen that the backslash has a special role. Any word 
starting with a backslash will be given a special interpretation 
when \TeX{} reads it from your input file. Such a word is called 
a {\sl control sequence}. There are, in fact, two types of 
control sequences: a {\sl control word\/} is a backslash followed 
by letters only (for example, {\tt \\TeX}) and a {\sl control 
symbol\/} is a backslash followed by a single nonletter (for 
example, {\tt \\\$})\null. Since a space is a nonletter, a 
backslash followed by a space is a legitimate control symbol. 
\TeXref{7--8} 
When we want to emphasize that a space is present, we will use a 
special symbol {\tt\sp} to indicate the space, as in the control 
symbol {\tt\\\sp}.  This convention is used in {\bf The \TeX 
book} as well as in this manual. 
 
When \TeX{} is reading your input file and comes to a backslash 
followed by a letter, it knows that a control word is being 
read. So it continues reading the name of the control word until 
a nonletter is read. So if your file contains 
 
\displaytext{\tt I like \\TeX!} 
 
\noindent 
the control word {\tt \\TeX} is terminated by the exclamation 
point. But this presents a problem if you want to have a space 
after a control word. If you have, for example, the sentence 
 
\displaytext{\tt I like \\TeX and use it all the time.} 
 
\noindent 
in your input file, the control word {\tt\\TeX} is terminated by 
the space (which is, of course, a nonletter). But then you won't 
have a space between the words ``\TeX{}'' and ``and''; inserting 
more spaces won't help, since \TeX{} doesn't distinguish between 
one space and several consecutive spaces. But if you put the 
control symbol {\tt\\\sp} after a control word, you will both 
terminate the control word and insert a space. It's really easy 
to forget to put in something like {\tt\\\sp} after a control 
word. I promise you that you will do it at least once while 
you're learning to use \TeX\null. 
 
\exercise Make an input file that will produce the following 
paragraph: 
 
I like \TeX! Once you get the hang of it, \TeX{} is really easy 
to use. You just have to master the \TeX nical aspects. 
 
Most control words are named so as to give a hint of their uses. 
You can use {\tt \\par} to make a new paragraph, for example, 
instead of putting in a blank line. 
\toindex{par} 
 
\subsection{What \TeX{} won't do} 
 
\TeX{} is very good at setting type, but there are things that \TeX{} 
can't do well.  One is making figures.  Space can be left 
to insert figures, but there are no graphic procedures built into 
the language (at present).  Some implementations allow graphic 
instructions to be inserted using the {\tt \\special} control 
word but these are exceptional and definitely site dependent. 
 
\TeX{} sets type in horizontal straight lines and not in 
straight lines at other angles. In particular, it is generally 
not possible to make insertions in ``landscape mode'', that is, 
with the text rotated so that the baseline is parallel with 
the long edge of the paper, or to include text that has a curve for 
a baseline. Perspective type (gradually increasing or 
decreasing in height) is not handled well by \TeX\null. 
 
We have seen that there is an ``edit, \TeX{}, driver'' cycle that 
is necessary for each different copy of output.  This is true 
even when the output device is a terminal.  In particular, it's 
not possible to type the input file and see the results on the 
screen immediately without going through the full cycle.  Some 
implementations have both text and graphics displays with 
reasonably quick turnaround (a few seconds for a single page); as 
hardware becomes less expensive and processors become faster, we 
may see improvement. 
 
 
 
 
 
 
\section{All characters great and small} 
 
\subsection{Some characters are more special than others} 
 
We saw in the last section that most text is entered at the 
terminal as sentences of ordinary words just as when typing with 
a typewriter. But we also saw that, in particular, the backslash 
could be used for at least two different purposes. It can be 
used for symbols (or combinations of symbols) that don't appear 
on the keyboard such as typing {\tt \\TeX} to get \TeX\null. It 
can also be used to give \TeX{} special instructions such as 
typing {\tt \\bye} to indicate the end of the input file. In 
general, a word starting with a backslash will be interpreted by 
\TeX{} as one requiring special attention. There are several 
hundred words that \TeX{} knows, and you can define more 
yourself, and so the backslash is very important. We'll spend a 
lot of time as we proceed learning some of these words; 
fortunately we'll only need to use a small number of them most of 
the time. 
 
There are ten characters which, like the backslash, are used 
by \TeX{} for special purposes, and we now want to give the 
complete list. 
\TeXref{37--38}What if we want to use a sentence with one of these 
special characters in it? With this in mind we'll ask the following 
questions: 
\item{(1)} What are the different special characters? 
\item{(2)} How do we use a special character if we really want to 
typeset it in our text? 
 
Here is a table of each special character, its purpose, and the 
method of typesetting the special character itself if you need 
it: 
 
 
\maketable [Characters requiring special input] 
\halign{ 
   \strut \hfil#\hfil & \quad # \hfil & \quad\tt # \hfil\cr 
   \bf Character & \bf Purpose & \bf Input for literal output \cr 
   \noalign{\hrule} \noalign{\smallskip} 
   $\backslash$   & Special symbols and instructions & \$\\backslash\$   \cr 
   $\{$ & Open group                       & \$\\\lb\$    \cr 
   $\}$ & Close group                      & \$\\\rb\$    \cr 
   \%   & Comments                         & \\\%         \cr 
   \&   & Tabs and table alignments        & \\\&         \cr 
   \~{} & Unbreakable space                & \\\~{}\lb\rb \cr 
   \$   & Starting or ending math text     & \\\$         \cr 
   \^{} & Math superscripts                & \\\^{}\lb\rb \cr 
   \_{} & Math subscripts                  & \\\_{}\lb\rb \cr 
   \#   & Defining replacement symbols     & \\\#         \cr 
       } 
 
\toindex{\lb} 
\toindex{\rb} 
\toindex{\%} 
\toindex{\&} 
\toindex{\~{}} 
\toindex{\$} 
\toindex{\^{}} 
\toindex{\_{}} 
\toindex{\#} 
 
\subsection{Typesetting with an accent} 
 
Now we're going to start using some of \TeX{}'s goodies! So far 
we've just been using \TeX{} to make our output look attractive, 
but now we'll start to do things that are difficult or impossible 
on the typewriter. In particular, we're going to look at accents 
now. How do you produce an accent when the symbol doesn't appear on 
the keyboard? Just as with the symbol \TeX{}, it is necessary to 
enter a word starting with a backslash. For the word ``premi\`ere'', 
as a first example, you need to type in {\tt premi\\`ere} 
(you may have to hunt around to find the ``back prime'' 
or ``grave'' symbol {\tt `} on your keyboard, but it's there 
somewhere\fnote{If you have a very old or obscure 
keyboard and the back prime is {\it really\/} not there, you can 
use {\tt \\lq\lb\rb} instead.  Similarly {\tt \\rq\lb\rb} can be 
used for the symbol {\tt '}. You can think of the symbols as 
being abbreviations for ``left quote'' and ``right quote.'' In 
addition, {\tt \\lq\\lq\lb\rb} and {\tt \\rq\\rq\lb\rb} give the 
usual quotation marks. But this won't work as a method to produce 
an accent over the following letter, so you're really better off 
with a proper keyboard.}). In general, to put an accent 
on a letter, the appropriate control sequence precedes the letter. 
 
Here are some examples: 
$$\vbox{ 
 
\halign{ 
\strut \hfil\tt # & \quad # \hfil\cr 
\bf \TeX{} input & \bf \TeX{} output \cr 
\noalign{\hrule} \noalign{\smallskip} 
\\`a la mode          & \`a la mode \cr 
r\\'esum\\'e          & r\'esum\'e  \cr 
soup\\c{\sp}con   & soup\c con  \cr 
No\\"el               & No\"el      \cr 
na\\"\\i{\sp}ve    & na\"\i ve   \cr 
} 
}$$ 
 
 
We see several principles illustrated by these examples. Most 
accents are produced by using a control symbol with a similar 
shape. A few of them are produced by control words containing a 
single letter. Some care must be used in this case, for a space 
must be used to terminate the control word. If you have {\tt 
soup\\ccon} in your file, for example, \TeX{} will look for the 
control word {\tt\\ccon}\fnote{We'll see that there is another 
method when we look at the grouping concept in Section~4.}. 
\TeXref{52--53} 
 
Notice that there is a control word {\tt\\i} also. This produces 
the letter ``i'' without the dot over it; this allows an accent 
to be put over the lower part of the letter. There is an 
analogous control word {\tt\\j} that produces a dotless ``j'' 
for accenting purposes. 
 
\maketable [Accents that may be immediately followed by a letter] 
\halign{ 
   \strut \hfil # \hfil & \quad \hfil \tt # \hfil & \hfil \quad # \hfil\cr 
   \bf Name & \bf \TeX{} input & \bf \TeX{} output \cr 
   \noalign{\hrule} \noalign{\smallskip} 
   grave           & \\`o & \`o \cr 
   acute           & \\'o & \'o \cr 
   circumflex      & \\\^{}o & \^o \cr 
   umlaut/dieresis/tr\'emat & \\"{}o    & \"o \cr 
   tilde           & \\\~{}o & \~o \cr 
   macron          & \\={}o    & \=o \cr 
   dot             & \\\char'056{}o    & \.o \cr 
       } 
 
\toindex{`} 
\toindex{'} 
\toindex{\^{}} 
\toindex{"} 
\toindex{i} 
\toindex{j} 
\toindex{=} 
\toindex{.} 
 
\maketable [Accents requiring an intervening space] 
\halign{ 
   \strut \hfil # \hfil & \quad \hfil \tt # \hfil & \hfil \quad # \hfil\cr 
   \bf Name & \bf \TeX{} input & \bf \TeX{} output \cr 
   \noalign{\hrule} \noalign{\smallskip} 
   cedilla         & \\c o   & \c{o} \cr 
   underdot        & \\d o   & \d{o} \cr 
   underbar        & \\b o   & \b{o} \cr 
   h\'a\v{c}ek     & \\v o   & \v{o} \cr 
   breve           & \\u o   & \u{o} \cr 
   tie             & \\t oo  & \t oo \cr 
   Hungarian umlaut& \\H o   & \H{o} \cr 
      } 
 
\toindex{c} 
\toindex{d} 
\toindex{b} 
\toindex{v} 
\toindex{u} 
\toindex{t} 
\toindex{H} 
 
\TeX{} also allows some letters from languages other than English to 
be typeset. 
 
\maketable [A\kern-1pt vailable foreign language symbols] 
\halign{ 
   \strut \hfil # \hfil & \quad \hfil \tt # \hfil & \hfil \quad # \hfil\cr 
   \bf Example& \bf \TeX{} input & \bf \TeX{} output \cr 
   \noalign{\hrule} \noalign{\smallskip} 
   \AE gean, \ae sthetics     & \\AE, \\ae    & \AE, \ae \cr 
   \OE uvres, hors d'\oe uvre & \\OE \\oe     & \OE, \oe \cr 
   \AA ngstrom                & \\AA, \\aa    & \AA, \aa \cr 
   \O re, K\o benhavn         & \\O, \\o      & \O, \o   \cr 
   \L odz, \l\'odka           & \\L, \\l      & \L, \l   \cr 
   Nu\ss                      & \\ss          & \ss      \cr 
   ?`Si?                      & {?}{`}        & ?`       \cr 
   !`Si!                      & {!}{`}        & !`       \cr 
   Se\~nor                    & \\\~{}        & \~{}       \cr 
                             & \lb\\it\\\$\rb & {\it\$}  \cr 
      } 
 
\toindex{AE} 
\toindex{ae} 
\toindex{OE} 
\toindex{oe} 
\toindex{AA} 
\toindex{aa} 
\toindex{O} 
\toindex{o} 
\toindex{L} 
\toindex{l} 
\toindex{ss} 
 
Typeset the sentence in each of the following exercises: 
 
\exercise Does \AE schylus understand \OE dipus? 
 
\exercise The smallest internal unit of \TeX{} is about 53.63\AA. 
 
\exercise They took some honey and plenty of money wrapped up in a 
{\it \$}5 note. 
 
\exercise \'El\`eves, refusez vos le\c cons! Jetez vos cha\^\i nes! 
 
\exercise Za\v sto tako polako pijete \v caj? 
 
\exercise Mein Tee ist hei\ss. 
 
\exercise Peut-\^etre qu'il pr\'ef\`ere le caf\'e glac\'e. 
 
\exercise ?`Por qu\'e no bebes vino blanco? !`Porque est\'a avinagrado! 
 
\exercise M\'\i\'\j n idee\"en worden niet be\"\i nvloed. 
 
\exercise Can you take a ferry from \"Oland to \AA land? 
 
\exercise T\"urk\c ce konu\c san ye\u genler nasillar? 
 
 
\def\bdots{$\textfont0=\tenbf \ldots$} %%% boldface version of \ldots 
 
\subsection{Dots, dashes, quotes, \bdots} 
 
Typing has always been a compromise: the small number of keys 
(compared to the number of typeset symbols available) has forced 
some changes on the typist.  When preparing material using \TeX{}, 
there is no need to be so restricted. In this section we'll look 
at some differences between typing and using \TeX\null. 
 
There are four types of dashes that are used.  The hyphen is used 
for combining words into one unit as with mother-in-law. 
\TeXref{3--5} 
The en-dash is used to indicate a sequence of page numbers, years 
or such things. The em-dash is a grammatical symbol.  The minus 
sign is used for negative numbers. Here they are with their uses: 
 
\maketable [Different types of dashes] 
\halign{ 
   \strut \hfil # \hfil & \quad \hfil \tt # \hfil & \hfil \quad # \hfil 
                        & \hfil # \hfil\cr 
   \bf Name& \bf \TeX{} input & \bf \TeX{} output & \bf Example\cr 
   \noalign{\hrule} \noalign{\smallskip} 
   hyphen     & -         & -   & The space is 3-dimensional. \cr 
   en-dash    & {-}{-}    & --  & Read pages 3--4. \cr 
   em-dash    & {-}{-}{-} & --- & I saw them---there were 3 men alive. \cr 
   minus sign & \$-\$     & $-$ & The temperature dropped to $-$3 degrees. \cr 
      } 
 
\exercise I entered the room and---horrors---I saw both my 
father-in-law and my mother-in-law. 
 
\exercise The winter of 1484--1485 was one of discontent. 
\bigskip 
 
 
Another difference between typing and using \TeX{} is the use of 
quotation marks.  Opening and closing quotation marks are the 
same on a typewriter.  They are produced in \TeX{} by using the 
apostrophe or prime key {\tt '} and backprime key {\tt `}\null. 
An open quote is produced by {\tt ``} and the 
close quote by {\tt ''}\null.\TeXref{3} Similarly the 
opening single quote is produced by {\tt `} and the closing 
single quote by {\tt '}\null. Notice that there is no need 
to use the usual typing quotation mark (it normally gives a close 
quote, but don't count on it). 
 
\exercise His ``thoughtfulness'' was impressive. 
 
\exercise Frank wondered, ``Is this a girl that can't say `No!'?'' 
\bigskip 
 
Sometimes ellipses (three dots) are used to indicate a passage 
of time or missing material.  Typing three periods will give 
three dots very close together. The dots with proper spacing will 
be produced by having {\tt \\dots} in your input file. 
\TeXref{173} 
\toindex{dots} 
 
\exercise He thought, ``\dots and this goes on forever, perhaps to the 
last recorded syllable.'' 
 
Another problem with dots is that the period after an 
abbreviation normally has less space after it than does the 
period at the end of a sentence. There are two ways to solve this 
problem: the dot can be followed by either {\tt\\\sp} or {\tt\~{}} 
to change the spacing. 
\TeXref{91--92} 
The second alternative will give an unbreakable space; when such 
a space is between two words, these words must typeset on the 
same line. The input {\tt Prof.\~{}Knuth} would cause those two 
words to be typeset on one line.  This is desirable for names 
like Vancouver, B.~C. and Mr.~Jones so that ``Mr.'' and ``Jones'' 
do not end up on separate lines. Notice that no backslash is used 
with the unbreakable space. 
 
\exercise Have you seen Ms.~Jones? 
 
\exercise Prof.~Smith and Dr.~Gold flew from 
\ifcanspell 
Halifax N.~S. to Montr\'eal, P.~Q. via Moncton, N.~B. 
\else 
L.~A. to Washington, D.~C., via Fargo, N.~D. 
\fi 
 
 
\subsection{Different fonts} 
 
The most obvious difference between typewritten text and \TeX{} 
output is undoubtedly the different fonts or types of symbols 
used. When \TeX{} starts up it has sixteen fonts available.  Some 
of these are used for technical purposes, but even so there are 
several different typefaces available as can be seen in the 
following table. A complete list of the sixteen fonts is given in 
Appendix F of {\bf The \TeX{}book}. 
\TeXref{427--432} 
Most of them are used automatically; a mathematical subscript, 
for example, is put in smaller type by \TeX{} with no special 
user intervention. 
 
To change from the usual (roman) typeface to italic, the control 
word {\tt \\it} is used.  To change back to the usual roman 
typeface, use {\tt \\rm}. For example, you could have the 
following sentence: {\tt I started with roman type, \\it switched 
to italic type, \\rm and returned to roman type}. This would give 
the following: I started with roman type, \it switched to italic 
type,\fnote{Notice that the comma and this footnote are in 
italic type, and this looks a little funny. We'll see that there 
is another method for changing fonts when we talk about grouping 
in Section 4.} 
\rm and returned to roman type. 
 
Typefaces other than italic are also available. Those given below 
are the ones most commonly used.  They are available immediately 
when \TeX{} starts. A little later we'll see how to use other 
typefaces that aren't available when \TeX{} starts. 
 
\global\advance\footnotenum by 1 
\maketable [Font Samples] 
\halign{ 
   \strut \hfil # \hfil & \quad \hfil \tt # \hfil & \hfil \quad # \hfil\cr 
   \bf Font name & \bf \TeX{} switch sequence & \bf Sample of typeface \cr 
   \noalign{\hrule} \noalign{\smallskip} 
   Roman       & \\rm  & \rm This is roman type. \cr 
   Boldface    & \\bf  & \bf This is boldface type. \cr 
   Italic      & \\it  & \it This is italic type. \cr 
   Slanted     & \\sl  & \sl This is slanted type. \cr 
   Typewriter  & \\tt  & \tt This is typewriter type. \cr 
   Math symbol${}^\the\footnotenum$ & \\cal & $\cal SOME\hbox{\quad}SCRIPT 
                                     \hbox{\quad}LETTERS$\rm. \cr 
      } 
 
\footnote{}{\llap{$^\the\footnotenum$}% 
This example is cheating since you need to know a little about 
mathematics input to use it, but I wanted to include it anyway. 
You'll be able to use these letters after you study the use of 
mathematical symbols.} 
\nobreak 
\toindex{it} 
\toindex{rm} 
\toindex{bf} 
\toindex{sl} 
\toindex{tt} 
 
The slanted and italic fonts seem quite similar at first.  It is 
easy to see the difference between the letter ``a'' in each 
sample. When changing from a slanted or italic font to a roman 
font, the last letter in the first font will lean towards the 
first letter of the roman font; this looks cramped, and to 
compensate there is a little extra space that can be added called 
the {\sl italic correction}. This is done using the control 
symbol {\tt \\/}. 
\toindex{/} 
In the following sentence, there is no italic correction after 
the first sequence of italic letters but there is after the 
second sequence; see the difference: {\it If} the italic 
correction is not used the letters are too close together, but 
{\it if\/} the correction is used, the spacing is better.  There 
is no need for the italic correction when the italic characters 
are followed by a comma or a period, but your text will 
definitely look better if you use it before quotation marks or 
parentheses. 
 
 
It is possible to use fonts other than those initially defined in 
\TeX{} when desired (assuming that they are available on 
your computer system, of course).  Different sizes can be used 
with the aid of the control word {\tt \\magstep}. To define your 
new font you will have to know its name on your computer system. 
For example, the roman typeface is called ``cmr10'' on most 
systems.  If you use {\tt \\font\\bigrm = cmr10 scaled \\magstep 
1}, you can then use {\tt\\bigrm} in exactly the same way as you 
use {\tt\\it} or {\tt\\rm}. 
\TeXref{13--17} 
\toindex{magstep} 
\toindex{font} 
\toindex{scaled} 
 
Switching via {\tt\\bigrm} will give roman type that is larger 
than the usual by 20\%\null.  {\tt \\font\\bigbigrm = cmr10 
scaled \\magstep 2} will define a font that is about 44\% larger 
than the usual roman typeface.  The sizes {\tt \\magstep~0} to 
{\tt \\magstep~5} are available.  On most computers {\tt \\magstephalf} 
is also available; this represents an enlargement of 
about 9.5\%\null. Here are some samples at various sizes: 
 
\maketable [Different Font Magnifications] 
\halign{ 
   \strut \tt # \hfil & \quad # \hfil \cr 
   \bf Magnification & \bf Sample \cr 
   \noalign{\hrule} \noalign{\smallskip} 
   \\magstep 0     & \rm  Sample text at magstep 0. \cr 
   \\magstephalf   & \halfrm  Sample text at magstephalf. \cr 
   \\magstep 1     & \brm  Sample text at magstep 1. \cr 
   \\magstep 2     & \bbrm  Sample text at magstep 2. \cr 
   \\magstep 3     & \bbbrm  Sample text at magstep 3. \cr 
   \\magstep 4     & \bbbbrm  Sample text at magstep 4. \cr 
   \\magstep 5     & \bbbbbrm  Sample text at magstep 5. \cr 
      } 
\bigskip 
 
It's also possible to use completely new typefaces.  These 
are dependent on system availability, of course.  Many systems 
have a file called CMSS10 which is a font that contains sans 
serif letters.  Using {\tt \\font\\sf = cmss10} will allow the 
control word {\tt\\sf} to be used in the same manner as {\tt\\bf}. 
Having made this definition, the input 
\displaytext{\tt \\sf Here is a sample of our new Sans Serif font.} 
will produce 
\displaytext{\sf Here is a sample of our new Sans Serif font.} 
 
\exercise What problem might have arisen if we had used {\tt \\ss} 
instead of {\tt \\sf} to turn on the Sans Serif font? Hint: if 
the answer doesn't occur to you at first, think about the German 
alphabet and it will finally come to you. 
 
\exercise Typeset a paragraph of magnified sans serif text. 
 
The extra fonts available may vary from site to site.  The ones 
in the following table are available at most places. 
 
\maketable [External Fonts Names] 
\halign{ 
\strut \tt # \hfil & \quad \tt # \hfil & \quad \tt # \hfil 
                   & \quad \tt # \hfil & \quad \tt # \hfil 
                   & \quad \tt # \hfil  \cr 
\noalign{\hrule} \noalign{\smallskip} 
CMBSY10   &CMBXSL10  &CMBXTI10  &CMBX10    &CMBX12    &CMBX5      \cr 
CMBX6     &CMBX7     &CMBX8     &CMBX9     &CMB10     &CMCSC10    \cr 
CMDUNH10  &CMEX10    &CMFF10    &CMFIB8    &CMFI10    &CMITT10    \cr 
CMMIB10   &CMMI10    &CMMI12    &CMMI5     &CMMI6     &CMMI7      \cr 
CMMI8     &CMMI9     &CMR10     &CMR12     &CMR17     &CMR5       \cr 
CMR6      &CMR7      &CMR8      &CMR9      &CMSLTT10  &CMSL10     \cr 
CMSL12    &CMSL8     &CMSL9     &CMSSBX10  &CMSSDC10  &CMSSI10    \cr 
CMSSI12   &CMSSI17   &CMSSI8    &CMSSI9    &CMSSQI8   &CMSSQ8     \cr 
CMSS10    &CMSS12    &CMSS17    &CMSS8     &CMSS9     &CMSY10     \cr 
CMSY5     &CMSY6     &CMSY7     &CMSY8     &CMSY9     &CMTCSC10   \cr 
CMTEX10   &CMTEX8    &CMTEX9    &CMTI10    &CMTI12    &CMTI7      \cr 
CMTI8     &CMTI9     &CMTT10    &CMTT12    &CMTT8     &CMTT9      \cr 
CMU10     &CMVTT10                                                \cr 
} 
\bigskip 
 
Here's a little background about these names.  The first two 
letters {\tt CM} stands for Computer Modern, the name given to 
this font family by the designer. The number at the end is the 
point size: 10 point type is the normal size, 7 point is the 
normal size for subscripts, and 5 point is the normal size for 
subscripts of subscripts, 12 point is 20\% larger than 10 point, 
etc. If the letters {\tt CM} are followed by {\tt B}, it is a 
boldface type.  Similarly {\tt R} indicates roman, {\tt I} is for 
italics, {\tt CSC} is for small caps, {\tt SL} is for slanted, 
{\tt SS} is for sans serif, {\tt SY} is for symbols, 
and {\tt TT} is for typewriter type. 
 
\exercise Find the fonts available on your system, and print out 
all the letters and numbers in several of them. 
 
\exercise The font {\tt CMR12} is 20\% larger than {\tt CMR10}. 
Also {\tt \\magstep~1} magnifies the type by 20\%.  Take some 
text and print it using {\tt CMR12} then using {\tt CMR10} scaled 
by {\tt \\magstep~1}. The results are quite different! 
 
 
 
 
\section{The shape of things to come} 
 
In this section we want to see how to make text have different 
shapes or sizes.  One may use \TeX{} with the default sizes, of 
course, just as we have been doing so far.  We'll now become more 
creative with our output.  In discussing the size of various 
parts of a page of text, there are several units of measure we 
can use. 
 
\subsection{Units, units, units} 
 
\TeX{} can measure length using many types of units.  The most 
common are the inch, the \centimeter, the point, the pica. The 
abbreviations for these units are {\tt in}, {\tt cm}, {\tt pt}, 
and {\tt pc} respectively. The point is defined by the equation 1 
inch = 72.27 points, and the pica by 1 pica = 12 points. Thus a 
point is rather small and is about the size of a decimal point. 
Here is a sample to give an idea of the comparative sizes: 
\def\pip{\vrule height 4 true pt } 
\TeXref{57} 
 
$$ \hbox{1 inch: } \hbox to 1 true in{\pip\hrulefill\pip}$$ 
$$ \hbox{1 \centimeter: }\hbox to 1 true cm{\pip\hrulefill\pip}$$ 
$$ \hbox{20 points: }\hbox to 20 true pt{\pip\hrulefill\pip}$$ 
$$ \hbox{1 pica: } \hbox to 1 true pc{\pip\hrulefill\pip}$$ 
 
Thus points are used to make fine changes; a pica is about the 
distance between the baselines of two consecutive lines of 
(unmagnified) normal text.  \TeX{} is quite exact about 
dimensions; internally its smallest unit is less than one 
four-millionth of an inch.  Hence it is the resolution of the 
output device that will determine the accuracy of the output. 
 
There are two other units that are sometimes useful that have 
different sizes in different fonts. The {\tt ex} is about the 
height of a small ``x'' and the {\tt em} is a little smaller than 
the width of a capital ``M''\null. 
\TeXref{60} 
 
The shape of the output is generally determined by control 
words. There are many such words; these allow very fine control 
of the resulting text.  But most of the time only a small number 
of them are necessary. 
 
\subsection{Page shape} 
 
The text on a page consists of three basic parts.  There is the 
headline above the main text; this often contains a chapter 
title, section title, or a page number, and may differ on odd and 
even pages. Below this is the main text, which includes any 
footnotes. And finally there is the footline that might contain a 
page number. 
 
In the examples we have seen so far, the headline has been blank. 
The footline has contained either a \centred{} number or, if {\tt 
\\nopagenumbers} had been used, has also been blank.  We'll have 
more to say about headlines and footlines later in this section. 
For the moment let's concentrate on the main text. 
 
To finish one page and start a new one, {\tt \\vfill \\eject} can 
be used.  The control word {\tt \\eject} forces the completion of 
the current page, while {\tt \\vfill} causes any left over 
vertical space to be gathered at the bottom of the page (you might 
try leaving out the {\tt \\vfill} as an experiment to see how the 
vertical \analog{} of justified horizontal lines of text works). 
\toindex{vfill} 
\toindex{eject} 
 
 
{\hsize=4in 
The current horizontal width of the text on the page is described 
by the control word {\tt \\hsize}.  It can be changed, say to 4 
inches, by {\tt \\hsize = 4 in\ } at any desired time using 
methods to be described in the next few sections. The value of 
{\tt \\hsize} in effect when the paragraph is completed 
determines the width of the paragraph. As can be seen in this 
paragraph, the text width can be changed (in this case to 4 
inches) for a single paragraph.  Because it represents the 
current value of the horizontal width, expressions such as {\tt 
 \\hsize = 0.75\\hsize} can make a new value that is relative 
to the old one. \par} 
\toindex{hsize} 
 
The vertical \analog{} of {\tt \\hsize} is {\tt \\vsize}, which 
is the current height of the main text.  It can be reset, just 
like {\tt \\hsize}.  Thus, {\tt \\vsize = 8 in\ } can be used to 
change the vertical height of the main text. Note that {\tt \\vsize} 
is the size of the text exclusive of any headline or 
footline material. 
\toindex{vsize} 
 
Text can also be shifted across the page.  The upper left corner 
of the text is one inch down and one inch in from the upper left 
corner of the page.  The control words {\tt \\hoffset} and {\tt \\voffset} 
are used to shift the text horizontally and 
vertically.  Thus {\tt \\hoffset = .75 in} and {\tt \\voffset = 
-.5 in} would shift the text an additional .75 inches to the 
right and .5 inches up. Most of the time you'll need to set {\tt 
\\hoffset}, {\tt \\voffset}, and {\tt \\vsize} at the beginning 
of your document only. 
\TeXref{251} 
\toindex{hoffset} 
\toindex{voffset} 
 
\global\advance\footnotenum by 1 
\maketable [Control words for page sizes] 
\halign{ 
   \strut \hfil # & \quad \hfil \tt # \hfil & \hfil \quad # \hfil\cr 
   \bf Name & \bf \TeX{} Control Word & \bf \TeX{} default value (inches)\cr 
   \noalign{\hrule} \noalign{\smallskip} 
   horizontal width  & \\hsize   & 6.5 \cr 
   vertical width    & \\vsize   & 8.9 \cr 
   horizontal offset${}^\the\footnotenum$  & \\hoffset & 0 \cr 
   vertical offset${}^\the\footnotenum$   & \\voffset & 0 \cr 
      } 
 
\footnote{}{\llap{$^\the\footnotenum$}The text is initially 1 inch 
down and 1 inch in from upper left corner of the page.} 
\nobreak 
 
\exercise Enter a paragraph of text that is a few lines long. 
Take several copies of your paragraph and put {\tt \\hsize = 5 
in\ } before one and {\tt \\hsize = 10 cm\ } in front of a second 
one.  Try a few other values for {\tt \\hsize}. 
 
\exercise Put {\tt \\hoffset = .5 in\ } and {\tt \\voffset = 1 
in\ } before the first paragraph of your previous exercise. 
 
\exercise Take your previous exercise and put {\tt \\vsize = 2 
in\ } before the first paragraph. 
\bigskip 
 
In the previous section we saw that it is possible to use fonts 
of larger size by using the {\tt \\magstep} control word.  It's 
also possible to magnify the entire document at once. If {\tt 
\\magnification = \\magstep 1} appears at the top of your document, 
then all of your output is magnified by 20\%\null.  The other 
values of {\tt \\magstep} can be used, too. {\bf It should be 
emphasized that {\tt \\magnification} can only be used before 
even a single character is typeset.}  This magnification creates 
a problem in units; if the document is to be magnified by 20\% 
and {\tt \\hsize = 5 in} appears in the \TeX{} input file, should 
the final output have an {\tt \\hsize} of 5 inches or should it 
be magnified by 20\% to 6 inches? Unless told otherwise, all the 
dimensions will be magnified so that, in this case, {\tt \\hsize} 
would be 6 inches in the final output.  This means that all the 
dimensions will increase uniformly when {\tt \\magnification} is 
used.  But for some special situations it might not be desirable 
for this to happen; for example, you might want to leave a space 
of exactly 3 inches to insert a figure.  In this case any unit 
can be modified by {\tt true} so that {\tt \\hsize = 5 true in} 
will always set the line width to 5 inches, regardless of 
magnification. \TeXref{59--60} \toindex{magnification} 
 
\exercise Put {\tt \\magnification = \\magstep 1} as the first 
line in one of your input files and see how the output changes. 
 
\subsection{Paragraph shape} 
 
When \TeX{} is reading the text to be typeset from your input 
file, it reads in a paragraph at a time and then typesets it. 
This means that there is a lot of control over paragraph shapes, 
but also that some care must be taken.  We've already seen that 
{\tt \\hsize} can be used to control the width of a single 
paragraph.  But suppose the following appeared as part of your 
input file: 
 
\beginuser 
\\hsize = 5 in 
Four score and seven years $\ldots$ 
$\vdots$ 
$\ldots$ from this earth. 
\\hsize = 6.5 in 
 
\enduser 
 
What is the width of the paragraph?  The {\tt \\hsize} was 
changed at the start of the paragraph and again at the end. 
Since the paragraph was not completed (by putting in a blank line 
or {\tt \\par}) until after the second change, the paragraph 
would be typeset with a width of 6.5 inches. However, if a blank 
line were inserted before {\tt \\hsize = 6.5 in}, then the 
paragraph would be typeset with a width of 5 inches.  So, in 
general, when a paragraph is set, the values of the parameters 
that are in effect when the paragraph is completed are the ones 
that are used. 
 
Here is a table with some paragraph parameters: 
 
\maketable [Some paragraph shape parameters] 
\halign{ 
   \strut \hfil # & \quad \hfil \tt # \hfil & \hfil \quad # \hfil\cr 
   \bf Function & \bf \TeX{} Control Word & \bf \TeX{} default\cr 
   \noalign{\hrule} \noalign{\smallskip} 
   width & \\hsize & 6.5 inches \cr 
   indentation on first line & \\parindent & 20 points\cr 
   distance between lines   & \\baselineskip & 12 points\cr 
   distance between paragraphs & \\parskip & 0 points \cr 
      } 
 
\toindex{parindent} 
\toindex{parskip} 
\toindex{baselineskip} 
 
The control word {\tt \\noindent} may be used at the beginning of 
a paragraph to avoid the automatic indentation on the first line. 
This control word will only affect the paragraph being set when 
it is invoked (of course {\tt \\parindent = 0 pt} will cause no 
indentation for all paragraphs). 
\toindex{noindent} 
 
{\narrower 
A more flexible way to control the width of a paragraph is to use 
{\tt \\rightskip} and {\tt \\leftskip}.  Setting {\tt \\leftskip 
= 20 pt} causes the left margin of the paragraph to be moved in 
an extra twenty points. Negative values may be assigned to {\tt 
\\leftskip} to move the left margin out. The control word {\tt 
\\rightskip} does the same to the right side of the paragraph. 
The single control word {\tt \\narrower} is the equivalent of 
setting both {\tt \\leftskip} and {\tt \\rightskip} equal to {\tt 
\\parindent}.  This is quite useful for long quotations, and 
this paragraph is an example of its use. As with {\tt \\hsize}, 
the value of {\tt \\leftskip} and {\tt \\rightskip} in effect 
when the paragraph is completed is the one which will apply to 
the whole paragraph. 
\TeXref{100} 
 
} 
\toindex{leftskip} 
\toindex{rightskip} 
\toindex{narrower} 
 
\exercise Make two paragraphs with the following specifications: 
the left margin of both paragraphs is indented by 1.5 inches, 
the right margin of the first paragraph is indented by 0.75 
inches, and the right margin of the second paragraph is indented 
by 1.75 inches. 
\bigskip 
 
\def\hangparagraph{ Lines can be made with different lengths 
within one paragraph by using {\tt \\hangindent} and {\tt 
\\hangafter}.  The amount of the indentation is determined by value 
of {\tt \\hangindent}. If {\tt \\hangindent} is positive, the 
indentation is made from the left, and if it is negative it is 
made from the right. The lines on which the indentation occurs is 
controlled by {\tt \\hangafter}.  If {\tt \\hangafter} is 
positive then it determines the number of lines at full width 
before the indentation starts.  Thus if {\tt \\hangindent = 1.75 
in} and {\tt \\hangafter = 6}, then the first six lines will be 
at full width and the rest will be indented by 1.75 inches from 
the left. On the other hand if {\tt \\hangindent = \hbox{-1.75} 
in} and {\tt \\hangafter = -6}, then the first six lines will be 
indented by 1.75 inches from the right and the rest will be at 
full width. \TeX{} resets to the default values {\tt \\hangindent 
= 0 pt} and {\tt \\hangafter = 1} after each paragraph. These 
control words are useful for paragraphs with ``hanging indents'' 
and for flowing a paragraph around space reserved for a figure. 
\TeXref{355}The control word {\tt \\hang} at the beginning of the 
paragraph will cause the first line to be of full width ({\tt 
\\hangafter=1}) and the rest of the paragraph to be indented by 
the current value of {\tt \\parindent}. But you do have to use 
{\tt \\noindent} if you want the first line to extend all the way 
to the left margin. 
\TeXref{102} 
\par} 
 
\hangafter=6 \hangindent=1.75in 
\hangparagraph 
\toindex{hangindent} 
\toindex{hangafter} 
\toindex{hang} 
 
Here is the previous paragraph repeated with {\tt \\hangafter = -6} 
and {\tt \\hangindent = -1.75 in}. 
 
\hangafter=-6 \hangindent=-1.75in \hangparagraph 
 
 
The control word {\tt \\parshape} can be used to make paragraphs with 
a greater variety of shapes. 
\TeXref{101} 
\toindex{parshape} 
 
Another useful control word for setting paragraphs is {\tt \\item}. 
It can be used to make various types of itemized lists, 
and is invoked using the pattern {\tt \\item\lb$\ldots$\rb}\null. 
This causes the next  paragraph to be formed with every line 
indented by {\tt \\parindent} and, in addition, the first line 
labeled on the left by whatever is between the braces.  It is 
usually used with {\tt \\parskip = 0 pt}, since that control word 
determines the vertical space between the different items. The control 
word {\tt \\itemitem} is the same as {\tt \\item} except that the 
indentation is twice as far, that is, twice the value of 
{\tt \\parindent}. 
\TeXref{102}Here is an example: 
\toindex{item} 
\toindex{itemitem} 
 
\beginliteral 
\parskip = 0pt \parindent = 30 pt 
\noindent 
Answer all the following questions: 
\item{(1)} What is question 1? 
\item{(2)} What is question 2? 
\item{(3)} What is question 3? 
\itemitem{(3a)} What is question 3a? 
\itemitem{(3b)} What is question 3b? 
@endliteral 
 
 
\noindent 
will produce 
\vfill\eject 
 
{\parskip = 0pt \parindent = 30 pt 
\noindent 
Answer all the following questions: 
\item{(1)} What is question 1? 
\item{(2)} What is question 2? 
\item{(3)} What is question 3? 
\itemitem{(3a)} What is question 3a? 
\itemitem{(3b)} What is question 3b? 
 
} 
 
\exercise Make a paragraph several lines long and use it with 
the {\tt \\item} control word to see the ``hanging indent.'' Now 
take the same paragraph and use it with different values of {\tt 
\\parindent} and {\tt \\hsize}. 
\bigskip 
 
Now let's see how to put space between paragraphs.  The control 
word {\tt \\parskip} is used to determine how much space is 
normally left between paragraphs. So if you put {\tt \\parskip = 
12 pt} at the beginning of your \TeX{} source file, there will be 
12 points between paragraphs unless other instructions are 
given. The control word {\tt \\vskip} can be used to insert extra 
vertical space between paragraphs.  If {\tt \\vskip 1 in} or {\tt 
\\vskip 20 pt} appears between two paragraphs, then the extra 
space is inserted. 
 
There are a couple of peculiarities of {\tt \\vskip} that seem 
quite strange at first.  If you have {\tt \\vskip 3 in} and the 
skip starts two inches from the bottom of the page, the rest of 
the page is skipped, but the extra one inch is {\it not\/} 
skipped at the top of the next page.  In other words, {\tt \\vskip} 
{\bf will not insert space across page boundaries}.  In 
fact, {\tt \\vskip~1~in} will have no effect at all if it happens 
to appear at the top of a page! For many types of vertical 
spacing this is quite appropriate.  The space before a section 
heading, for example, should not continue across page boundaries. 
 
A similar phenomenon occurs at the beginning of your document. 
If, for example, you want a title page with the title about half 
way down the page, you can not insert the space at the top of the 
page using {\tt \\vskip}. 
 
But what if you really want some blank space at the top of a 
page? You could start the page with {\tt\\\sp} but this in effect 
typesets a one line paragraph containing a blank. So while 
nothing is typeset, the extra space due to the values of {\tt 
\\baselineskip} and {\tt \\parskip} will add extra space. An easier 
method is to use {\tt \\vglue} instead of {\tt \\vskip} to get 
the desired result.  Thus {\tt \\vglue 1 in\ }will leave one inch 
of blank space at the top of the page.\TeXref{352} 
\toindex{vglue} 
 
 
We can note in passing that there is another more general method 
to put material at the top of a page using the control words {\tt 
\\topinsert} and {\tt \\endinsert}. If {\tt \\topinsert $\ldots$ 
\\endinsert} is used within a page, the material between 
{\tt \\topinsert} and {\tt \\endinsert} will appear at the top 
of the page, if possible. For example: 
 
\vbox{ 
   \beginuser 
   \\topinsert 
   \\vskip 1 in 
   \\centerline\lb Figure 1\rb 
   \\endinsert 
   \enduser 
} 
\noindent 
is useful for inserting figures of a prescribed size. 
\TeXref{115} 
\toindex{topinsert} 
\toindex{endinsert} 
 
There are also some special control words for making small 
vertical skips. They are {\tt \\smallskip}, {\tt \\medskip}, and 
{\tt \\bigskip}. Here is the size of each one: 
 
\def\hrl{\hrule width .5 in} 
 
\centerline{{\tt \\smallskip}: \vbox{\hrl \smallskip \hrl} \quad 
            {\tt \\medskip}: \vbox{\hrl \medskip \hrl} \quad 
            {\tt \\bigskip}: \vbox{\hrl \bigskip \hrl} 
            } 
\toindex{smallskip} 
\toindex{medskip} 
\toindex{bigskip} 
 
\subsection{Line shape} 
 
For most material \TeX{} does a good job of breaking up a 
paragraph into lines.  But sometimes it's necessary to add 
further instructions. It's possible to force a new line by 
inserting {\tt \\hfill \\break} in your input file. It's also 
possible to  put some text on a line by itself using the control 
word {\tt \\line\lb $\ldots$\rb}; then the material between the 
braces will be restricted to one line (although the result will 
be spread out over the whole line and may be horrible). The 
control words {\tt \\leftline\lb $\ldots$\rb}, {\tt \\rightline\lb 
$\ldots$\rb}, and {\tt \\centerline\lb $\ldots$\rb} will, 
respectively, set the material between the braces on a single 
line on the left margin, on the right margin, or \centred.  Thus 
\toindex{hfill} 
\toindex{break} 
\toindex{centerline} 
\toindex{leftline} 
\toindex{rightline} 
\toindex{line} 
 
 
\beginliteral 
\leftline{I'm over on the left.} 
\centerline{I'm in the @centre.} 
\rightline{I'm on the right.} 
\line{I just seem to be spread out all over the place.} 
@endliteral 
 
\noindent 
produces the four lines: 
\vskip\baselineskip 
 
 
{ 
\hbadness = 10000 
\leftline{I'm over on the left.} 
\centerline{I'm in the \center.} 
\rightline{I'm on the right.} 
\line{I just seem to be spread out all over the place.} 
 
Other types of spacing can be obtained by using the control word {\tt 
\\hfil}.  This causes all the extra space in the line to be 
accumulated at the position where {\tt \\hfil} appears. 
Thus if we alter our last example to 
{\tt \\line\lb I just seem to be spread out \\hfil all over the place.\rb} 
we will get \hfil\break 
\medskip 
\line{I just seem to be spread out \hfil all over the place.} 
\medskip 
If we have more than one {\tt \\hfil}, they will divide any extra 
space among themselves equally.  Hence 
{\tt \\line\lb left text \\hfil \centre{} text \\hfil right text.\rb} 
will produce 
\medskip 
\line{left text \hfil \centre{} text \hfil right text.} 
\toindex{hfil} 
} 
 
\exercise Typeset the following line: 
 
\line{left end \hfil left tackle \hfil left guard \hfil \centre{} \hfil 
right guard \hfil right tackle \hfil right end} 
 
\exercise Typeset the following line with twice as much space between 
``left'' and ``right-\centre{}'' as between ``right-\centre{}'' and 
``right'': 
 
\line{left \hfil \hfil right-\centre{} \hfil right} 
 
It's possible to move horizontally using {\tt \\hskip} in a 
manner analogous with {\tt \\vskip}. 
\toindex{hskip} 
 
\exercise What happens with the following input:\hfil\break 
{\tt \\line\lb\\hskip 1 in ONE \\hfil TWO \\hfil THREE\rb} 
\bigskip 
 
The right justification can be canceled by using the 
control word {\tt \\raggedright}. 
\toindex{raggedright} 
 
\subsection{Footnotes} 
 
The general pattern to make footnotes using \TeX{} is 
{\tt \\footnote\lb$\ldots$\rb\lb$\ldots$\rb}\null. 
 
The footnote mark goes in between the first set of braces.  Some 
available marks are {\tt \\dag (\dag)}, {\tt \\ddag (\ddag)}, {\tt \\S 
(\S)}, and {\tt \\P (\P)}\null.  The text of the footnote 
goes between the second set of braces. The use of numbers as 
marks is a little less straightforward. The footnote% 
\footnote{${}^{21}$}{This is the footnote at the bottom of the page.} 
at the bottom of the page was produced by using {\tt \\footnote\lb\$\lb 
\rb\^{}\lb{21}\rb\$\rb\lb This is the footnote at the 
bottom of the page.\rb}  after the word ``footnote'' in the 
text. This construction is somewhat complicated; we'll see why 
it's this way when we know a little more about typesetting 
mathematics. But for the moment we can look at it as a way of 
getting the job done. You may want to use {\tt \\rm} within the 
footnote to ensure that the right font appears. Usually you will 
not want a space between the text and the control word {\tt 
footnote}.\TeXref{117} 
\toindex{footnote} 
\toindex{ddag} 
\toindex{S} 
\toindex{P} 
 
\exercise Make up a page with a relatively long footnote spanning 
several lines. 
 
\exercise Make up a page with two different footnotes on it. 
 
\subsection{Headlines and Footlines} 
 
\headline={\hfil \tenrm Page \the\pageno} %example for this section 
 
The lines for title and page numbers that go above and below the 
main text are produced by using {\tt \\headline=\lb$\dots$\rb} 
and {\tt \\footline=\lb$\dots$\rb}\null. 
\TeXref{252--253} 
 
The principle is the same as using the control word {\tt \\line\lb 
$\dots$\rb} within the usual text on the page.  A helpful 
control word is {\tt \\pageno} which represents the current page 
number. Thus {\tt \\headline=\lb\\hfil \\tenrm Page \\the\\pageno\rb} 
would cause the page number to appear in the upper 
right corner preceded by the word ``Page'' (now look at the upper 
right corner of this page). It is safer to explicitly name the 
font that you want to use (in this case {\tt \\tenrm} to use the 
10 point roman font), since you are never guaranteed that a 
particular font will be in use when the headline or footline is 
set. The control word {\tt \\the} takes the internal value of the 
next word if it is an appropriate control word and prints it as 
text. You can also use the control word {\tt \\folio} instead of 
{\tt \\the \\pageno}. The difference is that {\tt \\folio} will 
give roman numerals when {\tt \\pageno} is negative. 
\toindex{headline} 
\toindex{footline} 
\toindex{pageno} 
\toindex{the} 
\toindex{folio} 
 
You can also assign values to {\tt \\pageno} if you want your 
document to use a different sequence of page numbers from the 
usual.  Roman numerals can be produced by using negative numbers; 
{\tt \\pageno=-1} at the beginning of a document will cause page 
numbers to be in roman numerals. 
\TeXref{252} 
 
Different headlines can be produced for even and odd pages in 
the following manner: 
 
{\tt 
\\headline=\lb\\ifodd \\pageno \lb$\dots$\rb \\else \lb$\dots$\rb 
\\fi\rb} 
 
\noindent 
where the material between the first set of braces is for the 
right-hand pages and the material between the second set of 
braces is for the left-hand pages. 
 
\exercise Change the footline so that the page number is \centred{} 
with an en-dash on either side. 
 
\subsection{Overfull and underfull boxes} 
 
One of the most frustrating experiences for the new \TeX{} user 
is the occurrence of overfull and underfull boxes.  They are duly 
noted in the log file and are also written to the screen when \TeX{} 
is being run interactively.  Overfull boxes are also marked 
on the output by a slug (a large filled-in black rectangle that 
looks like this: \vrule width \overfullrule) in the right 
margin. Various obscene terms are applied to this slug.  It 
appears even though there is nothing wrong with the \TeX{} input. 
So why is the slug there and what can be done about it? 
 
A good way to visualize the way \TeX{} organizes a page is to 
think of the printed material as being put into boxes. There are 
two types of boxes: {\sl hboxes\/} and {\sl vboxes}.  Most of the 
time these correspond to the organization of horizontal text 
into lines and vertical paragraphs into pages.  In particular, it 
is the spacing of the words in a hbox corresponding to a line of 
text that causes the slug to appear. 
 
Recall that \TeX{} reads in the complete paragraph before 
deciding how to break it up into lines.  This is better than 
working a line at a time, since a slight improvement in one line 
might cause catastrophic changes farther down in the paragraph. 
When words are gathered together to form a line, space is added 
between the words to justify the right margin. Very large spaces 
between words is obviously undesirable; the badness of the line 
is a measure of how badly the words are spaced.  An underfull 
hbox means that there is too much space between words.  More 
specifically, the badness of any line is a number between 0 
(perfect) and 10000 (horrible).  There is a parameter called {\tt 
\\hbadness} whose default value is 1000\null.  Any line whose 
badness is greater than {\tt \\hbadness} is reported as an 
underfull hbox.  If the value of {\tt \\hbadness} is increased, 
then fewer underfull hboxes will be reported. In fact {\tt 
\\hbadness = 10000} will suppress the reporting of underfull 
hboxes altogether. Similarly, if the words must be squeezed into 
a line so that the spaces are very small or even so the words 
extend a little into the right margin, then an overfull hbox 
results. \toindex{hbadness} 
 
%%%%%%%%%%%%%% restore original headline %%%%%%%%%%%%%%%%%%%%%%%% 
\headline={\iftitlepage \hfil \global\titlepagefalse 
                                      \else \gentleheadline \fi} 
 
In a similar way, sometimes \TeX{} allows a line to be slightly 
longer than {\tt \\hsize} in order to achieve a more balanced 
appearance. The control word {\tt \\tolerance} determines when 
this happens.  If the badness of a line is greater than {\tt 
\\tolerance}, \TeX{} will make the line longer by adding a new word 
at the end of the line, even though it might exceed the value of 
{\tt \\hsize}. A line which is only slightly longer is set 
without being reported. The control word {\tt \\hfuzz} determines 
how much excess is allowed; the default is {\tt \\hfuzz 
=~0.1~pt}.  A line that is more than slightly longer than {\tt \\hsize} 
is obviously a serious problem; \TeX{} puts a slug in 
the margin to make its point forcefully. It's possible to avoid 
overfull hboxes altogether by increasing the value of {\tt \\tolerance}. 
With {\tt \\tolerance = 10000} there will never be 
overfull boxes or slugs. The default is {\tt \\tolerance = 200}\null. 
\TeXref{29} 
\toindex{hfuzz} 
\toindex{tolerance} 
 
The width of the slug is determined by the control word {\tt 
\\overfullrule}.  Including {\tt \\overfullrule = 0 pt} in your 
file will delete any further  printing of the slugs.  The 
overfull boxes will still be there, of course, and they will 
probably be harder to spot. 
\toindex{overfullrule} 
 
So we see why overfull boxes and underfull boxes are reported; we 
can also change the reporting by changing the values of {\tt 
\\badness}, {\tt \\hfuzz}, and {\tt \\tolerance}.  In addition, a 
small value of {\tt \\hsize} obviously makes it more difficult to 
set lines and causes more overfull and underfull hboxes to be 
reported. These are warnings that you ignore at your own peril! 
 
Adding new possibilities of hyphenation will sometimes eliminate 
an overfull box. \TeX{} has automatic hyphenation and usually 
finds good places to hyphenate a word; however, it's possible to 
add hyphenations to let lines break at new places.  For example, 
the automatic hyphenation will never put a hyphen in the word 
``database''.  If you type in {\tt data\\-base}, it will allow a 
hyphen to be inserted after the second letter ``a'' in the word. 
More generally, if you put {\tt \\hyphenation\lb data-base\rb} in 
the beginning of your input file, then all occurrences of the 
word ``database'' will allow hyphenation after the letter ``a''. 
\TeXref{28} The log file will show the possible hyphenations on 
the line containing the overfull or underfull box. Sometimes the 
best solution to an overfull or underfull hbox is a little 
judicious editing of the original document. 
\toindex{hyphenation} 
 
Our discussion has involved the setting of type into lines, that is, 
the horizontal page structure.  There are several vertical \analog{}s. 
Overfull and underfull hboxes indicate how well words are gathered 
into lines.  Similarly, overfull and underfull vboxes are reported 
when paragraphs are gathered to form pages.  A large table that 
can't be broken in the middle, for example, can produce an underfull 
vbox that is reported in the log file when the page being typeset is 
completed.  The control word {\tt \\vbadness} works for the vertical 
placement of text in the same way as {\tt \\hbadness} works for 
horizontal text. 
\toindex{vbadness} 
 
\exercise Take a few paragraphs and print them using various (small) 
values of {\tt \\hsize} to see what kind of overfull boxes result. 
Repeat with various values of {\tt \\hbadness}, {\tt \\hfuzz}, and 
{\tt \\tolerance}. 
 
 
 
 
 
 
\section{$\Bigl\{$Groups, $\bigl\{$Groups, 
        $\{$and More Groups$\}\bigr\}\Bigr\}$} 
 
The concept of gathering text into groups allows \TeX{} input 
files to be greatly simplified.  A new group is started by the 
character {\tt\lb} and terminated by the character {\tt \rb}\null. 
Changes made within a group will lose their effect when the 
group terminates.  So, for example if {\tt \lb \\bf three 
boldface words\rb} appears in your text, the opening brace starts 
the group, the {\tt \\bf} control word changes to a boldface 
font, and the closing brace finishes up the group.  Upon 
completion of the group the font being used is the one in effect 
before the group started.  This is the (easier) way of having a 
few words in a different font.  It's also possible to have groups 
nested within groups. 
 
 
As another example, size changes can be made in the text that are 
only temporary.  For example 
\beginuser 
\lb 
\\hsize = 4 in 
\\parindent = 0 pt 
\\leftskip = 1 in 
will produce a paragraph that is four 
$\vdots$ 
(this is an easy mistake to make). 
\\par 
\rb 
\enduser 
 
{\hsize = 4 in 
\parindent = 0 pt 
\leftskip = 1 in 
will produce a paragraph that is four inches wide with the text 
offset into the paragraph by one inch regardless of the settings 
in effect before the start of the group.  This paragraph is set 
with those values. After the end of the group, the old settings 
are in effect again. Note that it is necessary to include 
{\tt \\par} or to use a blank line before the closing brace to end 
the paragraph, since otherwise the group will end and \TeX{} will 
go back to the old parameters before the paragraph is actually 
typeset (this is an easy mistake to make). \par} 
 
When line-spacing control words (like {\tt \\centerline}) act on text 
following it in braces, that text is implicitly in a group. Thus 
{\tt \\centerline\lb\\bf A bold title\rb} will produce a \centred{} 
boldface line, and the text following that line will be in 
whatever font was in effect before the {\tt \\centerline} was 
invoked. 
 
The empty group {\tt\lb\rb} is useful.  One use allows accents to be 
typeset with no accompanying letter.  For example, {\tt \\\~{}\lb\rb} 
will print a tilde with no letter under it.  It can also be used to 
stop \TeX{} from eating up consecutive spaces. Hence {\tt I use 
\\TeX\lb\rb{} all the time}  will leave a space after ``\TeX'' in the 
output. This is an alternative to using {\tt \\\sp} as we did in 
Section~1. 
\TeXref{19--21} 
 
Grouping can also be used to avoid spaces in the middle of a word 
when including accents. Either {\tt soup\\c\sp con} or {\tt 
soup\\c\lb c\rb on} will produce the word soup\c{c}on. 
 
\exercise Change the dimensions of one paragraph on a page using the 
grouping idea. 
 
\exercise Mathematicians sometime use the word ``i{f}f'' as an 
abbreviation for ``if and only if''\null.  In this case it looks 
better if the first and second letter ``f'' are {\sl not\/} 
joined as a ligature.  How do you do this (there are several 
solutions)? 
 
It's really easy to forget to match braces properly.  The effect 
can be dramatic; if you get output that suddenly changes to an 
italic font for the rest of the document, a mismatched brace is 
probably the cause.  If you have an extra {\tt \lb} \TeX{} will 
give a message in the log file: {\tt (\\end occurred inside a 
group at level 1)}. An extra {\tt\rb} will result in the message 
\hbox{\tt! Too many \rb's.} 
 
Here's a little hint to help you keep track of the braces in more 
complicated groups: put the opening brace on a line by itself and 
do the same for the closing brace.  If there are braces nested 
within the original ones, put them on separate lines also, but 
indent them a few spaces.  The text within the nested braces can 
also be indented since \TeX{} ignores all spaces at the beginning 
of a line. The matching braces will then stand out when you look 
at your \TeX{} source file.  In fact, if your editor is smart 
enough, you can create the two lines with the braces first and 
then insert the appropriate material within them with automatic 
indenting. 
 
\exercise In section 2 we changed fonts the following method: 
{\tt I started with roman type, \\it switched to italic type, 
\\rm and returned to roman type}. Get the same result using the 
idea of grouping. 
 
 
\section{No math anxiety here!} 
 
\TeX{} is at its best when typesetting mathematics.  The 
conventions for doing this are many and complex, and the ability 
of \TeX{} to take them into account makes the production of high 
quality, attractive mathematical output possible.  If you plan to 
produce papers with mathematical symbols in them, this section 
will give you all the basics necessary for creating beautiful 
output in almost all circumstances; \TeX{} may be used without 
any mathematics, of course, and if this is your goal, then the 
following two subsections are probably sufficient for your needs. 
 
\subsection{Lots of new symbols} 
 
Mathematical text is inserted into normal text in two possible 
ways: it can be {\sl in-line\/}, that is, as part of the usual 
lines of text.  It can also be {\sl displayed}, that is, \centred{} 
in a gap between the usual text.  The results in the spacing and 
placement of symbols can be quite different in each case.  The 
in-line equation $\sum_{k=1}^{\infty} {1\over k^2} = {\pi^2\over6}$ 
is easily seen to be different from the same equation when 
displayed: 
$$\sum_{k=1}^{\infty} {1\over k^2} = {\pi^2\over6}.$$ 
 
Since the spacing and the fonts used for mathematics are quite 
different from those of ordinary text, \TeX{} needs to be told 
when mathematics rather than normal text is being typeset.  This 
is done using the {\tt\$} symbol.  More specifically, mathematics 
is typeset in-line by surrounding the symbols to be entered by 
single dollar signs: {\tt \$$\ldots$\$}, and is typeset displayed 
by surrounding the symbols to be entered by double dollar signs: 
{\tt \$\$$\ldots$\$\$}\null.  Hence {\tt \$x = y+1\$} will give 
$x=y+1$ in-line while {\tt \$\$x = y+1.\$\$} will display 
$$x=y+1.$$ 
 
The spacing for both in-line and displayed mathematics is 
completely controlled by \TeX\null.  Adding spaces to your input 
has no effect at all.  What if you need a space or some text in 
the middle of some mathematics?  You can insert text by inserting 
it into an hbox (don't worry about the definition of an hbox for 
now): {\tt \\hbox\lb$\ldots$\rb}\null.  This is particularly 
useful for displayed mathematics.  Hence ``$x=y+1 \hbox{ whenever 
} y=x-1$'' can be typeset using {\tt \$x=y+1 \\hbox\lb\ whenever 
\rb y=x-1\$}. Note the spaces on either side of the word within 
the braces. Usually it isn't necessary to insert spaces into 
mathematical text, but here are the control sequences that do the 
job. 
\TeXref{167} 
 
\maketable [Adding space to mathematical text] 
\halign{ 
\strut \hfil # & \quad \hfil\tt# \hfil \quad 
   & \hbox to 2cm{\hrulefill\vrule height 8pt#\vrule height 8pt\hrulefill} \cr 
   Name & \rm Control Sequence & \hfil{}$\gets$Size$\to$\cr 
   \noalign{\hrule} \noalign{\smallskip} 
   Double quad         & \\qquad &\qquad \cr 
   Quad                & \\quad  &\quad \cr 
   Space               & \\\sp\   &\ \cr 
   Thick space         & \\;     &$\;$\cr 
   Medium space        & \\>     &$\>$\cr 
   Thin space          & \\,     &$\,$\cr 
   Negative thin space & \ \\!\    &$\!$\cr 
       } 
 
\toindex{quad} 
\toindex{qquad} 
\toindex{\sp} 
\toindex{;} 
\toindex{>} 
\toindex{,} 
\toindex{!} 
 
If you look closely at the negative thin space, you'll notice 
that, unlike the other entries, the two arms overlap.  This is 
because the negative space is one in the opposite direction, that 
is, while the other control sequences increase the amount of 
space between two symbols being typeset, the negative thin space 
decreases the space between them, even if it causes them to 
overlap. 
 
\exercise Typeset the following: $C(n,r) = n!/(r!\,(n-r)!)$\null. Note the 
spacing in the denominator. 
\bigskip 
 
You shouldn't have any blank lines between the dollar signs 
delimiting the mathematical text. \TeX{} assumes that all the 
mathematical text being typeset is in one paragraph, and a blank 
line starts a new paragraph; consequently, this will generate an 
error message. This turns out to be useful, for one of the 
easiest errors to make is to forget to put in the trailing dollar 
sign(s) after the mathematical input (I promise that you'll do it 
at least once while learning \TeX{}); if \TeX{} allowed more than 
one paragraph to be between the dollars signs, then one omitted 
trailing dollar sign might cause the rest of the document to be 
typeset as mathematics. 
 
Most mathematical text is entered in exactly the same way for 
in-line typesetting as for displayed typesetting (except for the 
surrounding dollar signs, of course). The exceptions, such as 
aligning multiline displays and placing equation numbers at the 
left or right margin will be discussed in the last part of this 
section. 
 
Many new symbols can appear when typesetting mathematics.  Most 
of the ones that actually appear on the keyboard can be used 
directly. The symbols {\tt + - / * = ' | < > (} and {\tt )} are 
all entered directly. Here they are as mathematics: $+ - / * =\> 
'\> | <\> >\> ( \> )$. 
 
\exercise Typeset the equation $a+b=c-d=xy=w/z$ as in-line and 
displayed mathematical text. 
 
\exercise Typeset the equation $(fg)' = f'g + fg'$ as in-line and 
displayed mathematical text. 
\bigskip 
 
Many other symbols, as you would expect, are predefined control 
words.  All Greek letters are available.  Here is a table of them: 
\TeXref{434} 
 
\maketable [Greek letters] 
\halign{ 
\strut \hfil$#$ & \quad \tt# \hfil \qquad &\hfil$#$ & \quad \tt# \hfil \qquad 
      &\hfil$#$ & \quad \tt# \hfil \qquad &\hfil$#$ & \quad \tt# \hfil \cr 
\noalign{\hrule} \noalign{\smallskip} 
\alpha & \\alpha &\beta & \\beta &\gamma & \\gamma &\delta & \\delta \cr 
\epsilon & \\epsilon & \varepsilon & \\varepsilon & \zeta & \\zeta & 
\eta & \\eta \cr 
\theta & \\theta & \vartheta & \\vartheta & \iota & \\iota & \kappa & 
\\kappa \cr 
\lambda & \\lambda & \mu & \\mu & \nu & \\nu & \xi & \\xi \cr 
o & o & \pi & \\pi & \rho & \\rho & \varrho & \\varrho \cr 
\sigma & \\sigma & \varsigma & \\varsigma & \tau & \\tau & \upsilon & 
\\upsilon \cr 
\phi & \\phi & \varphi & \\varphi & \chi & \\chi & \psi & \\psi \cr 
\omega & \\omega & \Gamma & \\Gamma & \Delta & \\Delta & \Theta & 
\\Theta \cr 
\Lambda & \\Lambda & \Xi & \\Xi & \Pi & \\Pi & \Sigma & \\Sigma \cr 
\Upsilon & \\Upsilon & \Phi & \\Phi & \Psi & \\Psi & \Omega & \\Omega \cr 
       } 
 
\toindex{alpha} 
\toindex{beta} 
\toindex{gamma} 
\toindex{delta} 
\toindex{epsilon} 
\toindex{varepsilon} 
\toindex{zeta} 
\toindex{eta} 
\toindex{theta} 
\toindex{vartheta} 
\toindex{iota} 
\toindex{kappa} 
\toindex{lambda} 
\toindex{mu} 
\toindex{nu} 
\toindex{xi} 
\toindex{pi} 
\toindex{rho} 
\toindex{varrho} 
\toindex{sigma} 
\toindex{varsigma} 
\toindex{tau} 
\toindex{upsilon} 
\toindex{phi} 
\toindex{varphi} 
\toindex{chi} 
\toindex{psi} 
\toindex{omega} 
\toindex{Gamma} 
\toindex{Delta} 
\toindex{Theta} 
\toindex{Lambda} 
\toindex{Xi} 
\toindex{Pi} 
\toindex{Sigma} 
\toindex{Upsilon} 
\toindex{Phi} 
\toindex{Psi} 
\toindex{Omega} 
 
 
\exercise Typeset $\alpha\beta=\gamma+\delta$ as in-line and displayed 
mathematical text. 
 
\exercise Typeset $\Gamma(n) = (n-1)!$ as in-line and displayed 
mathematical text. 
 
Sometimes accents are put above or below symbols.  The control 
words used for accents in mathematics are different from those 
used for normal text.   The normal text control words may not be 
used for mathematics and vice-versa. 
\TeXref{135--136} 
 
\maketable [Mathematical accents] 
\halign{ 
   \strut \hfil$#$ & \quad \tt# \hfil \qquad &\hfil$#$ & 
      \quad \tt# \hfil \qquad &\hfil$#$ & \quad \tt# \hfil \qquad  \cr 
   \noalign{\hrule} \noalign{\smallskip} 
   \hat o   & \\hat o   & \check o & \\check o  & \tilde o & \\tilde o \cr 
   \acute o & \\acute o & \grave o & \\grave o  & \dot o & \\dot o \cr 
   \ddot o  & \\ddot o  &\breve o  & \\breve o  & \bar o & \\bar o \cr 
   \vec o   & \\vec o   & \widehat {abc} & \\widehat \lb abc\rb 
                    & \widetilde {abc} & \\widetilde \lb abc\rb\cr 
       } 
 
\toindex{hat} 
\toindex{check} 
\toindex{tilde} 
\toindex{acute} 
\toindex{grave} 
\toindex{dot} 
\toindex{ddot} 
\toindex{breve} 
\toindex{bar} 
\toindex{vec} 
\toindex{widehat} 
\toindex{widetilde} 
 
 
 
Binary operators combine two mathematical objects to get another 
object.  Ordinary addition and multiplication, for example, 
combine two numbers to get another number, and so they are binary 
operators. When a binary operator such as $+$ or $\times$ is 
typeset, a little extra space is put around it. Here is a list of 
some of the available binary operators: 
\TeXref{436} 
 
\maketable [Binary operators] 
\halign{ 
\strut \hfil$#$ & \quad \tt# \hfil \qquad &\hfil$#$ & \quad \tt# \hfil \qquad 
      &\hfil$#$ & \quad \tt# \hfil \qquad &\hfil$#$ & \quad \tt# \hfil \cr 
\noalign{\hrule} \noalign{\smallskip} 
\cdot & \\cdot &\times & \\times &\ast & \\ast &\star & \\star \cr 
\circ & \\circ & \bullet & \\bullet & \div & \\div & \diamond & \\diamond \cr 
\cap & \\cap & \cup & \\cup & \vee & \\vee & \wedge & \\wedge \cr 
\oplus & \\oplus &\ominus & \\ominus & \otimes &\\otimes &\odot &\\odot \cr 
       } 
 
\toindex{cdot} 
\toindex{times} 
\toindex{ast} 
\toindex{star} 
\toindex{circ} 
\toindex{bullet} 
\toindex{div} 
\toindex{diamond} 
\toindex{cap} 
\toindex{cup} 
\toindex{vee} 
\toindex{wedge} 
\toindex{oplus} 
\toindex{ominus} 
\toindex{otimes} 
\toindex{odot} 
 
Ellipses are commonly used with binary operators.  The control 
word {\tt \\cdots} will raise the dots so that they are level 
with the binary operator. Thus {\tt \$a + \\cdots + z\$} will 
produce $a + \cdots + z$.  The control word {\tt \\ldots} will 
put the dots on the baseline, and so {\tt \$1\\ldots n\$} 
produces $1\ldots n$. 
\toindex{cdots} 
\toindex{ldots} 
 
\exercise Typeset: $x\wedge (y\vee z) = (x\wedge y) \vee (x\wedge z)$. 
 
\exercise Typeset: $2+4+6+\cdots +2n = n(n+1)$. 
\bigskip 
 
A relation indicates a property of two mathematical objects.  We 
already know how to show two objects equal, or how to show one 
number less than or greater than another number (since these are 
symbols on most terminal keyboards).  To negate a relation, the 
control word {\tt \\not} is put in front of the relation.  Here 
are some relations: 
\TeXref{436} 
\toindex{not} 
 
\maketable [Relations ] 
\halign{ 
\strut \hfil$#$ & \quad \tt# \hfil \qquad &\hfil$#$ & \quad \tt# \hfil \qquad 
      &\hfil$#$ & \quad \tt# \hfil \qquad &\hfil$#$ & \quad \tt# \hfil \cr 
\noalign{\hrule} \noalign{\smallskip} 
\leq  & \\leq  &\not\leq & \\not \\leq 
       & \geq & \\geq & \not\geq & \\not \\geq \cr 
\equiv & \\equiv & \not\equiv & \\not \\equiv 
       & \sim & \\sim & \not\sim & \\not \\sim \cr 
\simeq & \\simeq & \not\simeq & \\not \\simeq 
       & \approx & \\approx & \not\approx & \\not \\approx \cr 
\subset & \\subset & \subseteq & \\subseteq 
        & \supset & \\supset & \supseteq & \\supseteq \cr 
\in & \\in & \ni & \\ni & \parallel & \\parallel & \perp & \\perp \cr 
       } 
 
\toindex{leq} 
\toindex{geq} 
\toindex{equiv} 
\toindex{sim} 
\toindex{simeq} 
\toindex{approx} 
\toindex{subset} 
\toindex{subseteq} 
\toindex{supset} 
\toindex{supseteq} 
\toindex{in} 
\toindex{ni} 
\toindex{parallel} 
\toindex{perp} 
 
 
\exercise 
Typeset: $\vec x\cdot \vec y  = 0$ if and only if $\vec x \perp \vec y$. 
 
\exercise 
Typeset: $\vec x\cdot \vec y \not= 0$ if and only if $\vec x \not\perp \vec y$. 
 
Here are some other available mathematical symbols:\TeXref{435--438} 
 
\maketable [Miscellaneous symbols ] 
\halign{ 
\strut \hfil$#$ & \quad \tt# \hfil \qquad &\hfil$#$ & \quad \tt# \hfil \qquad 
      &\hfil$#$ & \quad \tt# \hfil \qquad &\hfil$#$ & \quad \tt# \hfil \cr 
\noalign{\hrule} \noalign{\smallskip} 
\aleph & \\aleph & \ell & \\ell & \Re & \\Re & \Im & \\Im \cr 
\partial & \\partial & \infty & \\infty & \| & \\| & \angle & \\angle \cr 
\nabla & \\nabla & \backslash &\\backslash & \forall & \\forall 
              & \exists & \\exists \cr 
\neg & \\neg & \flat & \\flat & \sharp & \\sharp & \natural & \\natural \cr 
          } 
 
\toindex{aleph} 
\toindex{ell} 
\toindex{Re} 
\toindex{Im} 
\toindex{partial} 
\toindex{infty} 
\toindex{|} 
\toindex{angle} 
\toindex{nabla} 
\toindex{backslash} 
\toindex{forall} 
\toindex{exists} 
\toindex{neg} 
\toindex{flat} 
\toindex{sharp} 
\toindex{natural} 
 
\exercise Typeset: $(\forall x\in \Re)(\exists y\in\Re)$ $y>x$. 
 
\subsection{Fractions} 
 
There are two methods of typesetting a fraction: it can be 
typeset either in the form $1/2$ or in the form ${1\over2}$\null. 
The first case is just entered with no special control sequences, 
that is, {\tt \$1/2\$}\null.  The second case uses the control 
word {\tt \\over} and the following pattern: {\tt \lb 
<numerator> \\over <denominator>\rb}\null. Hence {\tt \$\$\lb a+b 
\\over c+d\rb.\$\$} gives \TeXref{139--140} 
\toindex{over} $${a+b\over c+d}.$$ 
 
\exercise Typeset the following: ${a+b\over c}\quad {a\over b+c} 
\quad {1\over a+b+c} \not= {1\over a}+{1\over b}+{1\over c}$. 
 
\exercise Typeset: What are the points where 
${\partial \over \partial x} f(x,y) = {\partial \over \partial y} 
f(x,y) = 0$? 
 
\subsection{Subscripts and superscripts} 
 
Subscripts and superscripts are particularly easy to enter using 
\TeX\null. The characters {\tt \_{}} and {\tt \^{}} are used to 
indicate that the next character is a subscript or a superscript. 
Thus {\tt \$x\^{}2\$} gives $x^2$ and {\tt \$x\_{}2\$} gives 
$x_2$\null.  To get several characters as a subscript or 
superscript, they are grouped together within braces. Hence we 
can use {\tt \$x\^{}\lb 21\rb\$} to get $x^{21}$ and {\tt \$x\_{}\lb 
21\rb\$} to get $x_{21}$\null. Notice that the 
superscripts and subscripts are automatically typeset in a 
smaller type size. The situation is only slightly more 
complicated for a second layer of subscripts or superscripts. 
You can {\sl not\/} use {\tt \$x\_{}2\_{}3\$} since this could 
have two possible interpretations, namely, {\tt \$x\_{}\lb 2\_{}3\rb\$} 
or {\tt \$\lb x\_{}2\rb\_{}3\$}; this gives two 
different results: $x_{2_3}$ and ${x_2}_3$, the first of which is 
the usual mathematical subscript notation.  Thus you must put in 
the complete braces to describe multiple layers of subscripts and 
superscripts.  They may be done to any level. 
\TeXref{128--130} 
 
To use both subscripts and superscripts on one symbol, you use 
both the {\tt \_{}} and {\tt \^{}} in either order.  Thus either 
{\tt \$x\_{}2\^{}1\$} or {\tt\$x\^{}1\_{}2\$} will give $x_2^1$. 
 
\exercise 
Typeset each of the following: $e^x \quad e^{-x} \quad 
e^{i\pi}+1=0 \quad x_0 \quad x_0^2 \quad {x_0}^2 \quad 2^{x^x}$. 
 
\exercise Typeset: 
$\nabla^2 f(x,y) = {\partial^2 f \over\partial x^2} 
+ {\partial^2 f \over\partial y^2}$. 
\bigskip 
 
A similar method is used for summations and integrals.  The input 
of {\tt \$\\sum\_{}\lb k=1\rb\^{}n k\^{}2\$} will give $\sum_{k=1}^n 
k^2$, and {\tt \$\\int\_{}0\^{}x f(t) dt\$} will 
give $\int_0^x f(t) dt$. 
\TeXref{144--145} 
\toindex{sum} 
\toindex{int} 
 
Another use of this type of input is for expressions 
involving limits.  You can use 
{\tt \$\\lim\_\lb n\\to \\infty\rb (\lb n+1 \\over n\rb)\^{}n = e\$} 
to get $\lim_{n\to \infty} ({n+1\over n})^n = e$. 
\toindex{lim} 
 
\exercise Typeset the following expression: $\lim_{x\to 0} 
(1+x)^{1\over x}=e$. 
 
\exercise Typeset: The cardinality of $(-\infty, \infty)$ is $\aleph_1$. 
 
\exercise Typeset: $\lim_{x\to {0^+}} x^x = 1$. 
\bigskip 
 
Here's a hint to make integrals look a little nicer: look at the 
difference between $\int_0^x f(t) dt$ and $\int_0^x f(t)\, dt$\null. 
In the second case there is a little extra space after 
$f(t)$, and it looks nicer; {\tt \\,} was used to add the 
additional space. 
\toindex{,} 
 
\exercise Typeset the following integral: $\int_0^1 3x^2\,dx = 1$. 
 
\subsection{Roots, square and otherwise} 
 
To typeset a square root it is only necessary to use the construction 
{\tt \\sqrt\lb$\ldots$\rb}\null. Hence {\tt \$\\sqrt\lb x\^{}2+y\^{}2\rb\$} 
will give $\sqrt{x^2+y^2}$\null.  Notice that \TeX{} takes care of the 
placement of symbols and the height and length of the radical. 
To make cube or other roots, the control words {\tt \\root} and {\tt 
\\of} are used.  You get $\root n \of {1+x^n}$ from the 
input\TeXref{130--131} {\tt \$\\root n \\of \lb1+x\^{}n\rb\$}. 
\toindex{root} 
\toindex{sqroot} 
 
A possible alternative is to use the control word {\tt \\surd}; the 
input {\tt \$\\surd 2\$} will produce $\surd 2$. 
\toindex{surd} 
 
\exercise Typeset the following: $\sqrt2 \quad \sqrt {x+y\over x-y} 
\quad \root 3 \of {10}$ \quad $e^{\sqrt x}$. 
 
\exercise Typeset: $\|x\| = \sqrt{x\cdot x}$. 
 
\exercise Typeset: 
$\phi(t) = {1 \over \sqrt{2\pi}} \int_0^t e^{-x^2/2}\,dx$. 
 
 
\subsection{Lines, above and below} 
 
Use the constructions {\tt \\overline\lb$\ldots$\rb} and 
{\tt \\underline\lb$\ldots$\rb} to put lines above or below mathematical 
expressions. Hence {\tt \$\\overline\lb x+y\rb=\\overline x + \\overline 
y\$} gives $\overline{x+y}=\overline x + \overline y$\null. 
But notice that the lines over the letters are at 
different heights, and so some care is necessary. The use of {\tt 
\\overline\lb\\strut x\rb} will raise the height of the line 
over\TeXref{130--131} $x$\null. 
\toindex{overline} 
\toindex{underline} 
 
To underline non-mathematical text, use {\tt \\underbar\lb$\dots$\rb}. 
\toindex{underbar} 
 
\exercise Typeset the following: $\underline x \quad \overline y 
\quad \underline{\overline{x+y}}$. 
 
 
\subsection{Delimiters large and small} 
 
The most commonly used mathematical delimiters are brackets, 
braces, and parentheses.  As we have seen, they may be produced 
by using {\tt [ ] \\\lb\ \\\rb\ ( )} to get 
$[\>]\>\{\>\}\>(\>)\>$. Sometimes larger delimiters increase the 
clarity of mathematical expressions, as in 
$$\bigl(a\times(b+c)\bigr) \bigl((a\times b)+c\bigr).$$ To make 
larger left delimiters the control words {\tt \\bigl}, {\tt 
\\Bigl}, {\tt \\biggl}, and {\tt \\Biggl} are used in front of 
the delimiter; similarly, {\tt \\bigr}, {\tt \\Bigr}, {\tt 
\\biggr}, and {\tt \\Biggr} are used\TeXref{145--147} for the 
right delimiters. Hence {\tt \$\\Bigl[\$} and {\tt \$\\Bigr]\$} 
will produce $\Bigl[$ and $\Bigr]$. 
 
\toindex{bigl} 
\toindex{Bigl} 
\toindex{biggl} 
\toindex{Biggl} 
\toindex{bigr} 
\toindex{Bigr} 
\toindex{biggr} 
\toindex{Biggr} 
 
Here is a table to compare the size of some of the delimiters. 
 
%% \everycr can add 4 points between lines in the following table %% 
\everycr={\noalign{\vskip 4 pt}} 
\maketable [Delimiters of various sizes] 
\halign{ 
\strut \hfil$#$ & \quad \tt# \hfil \quad\qquad 
      &\hfil$#$ & \quad \tt# \hfil \quad\qquad 
      &\hfil$#$ & \quad \tt# \hfil \quad\qquad 
      &\hfil$#$ & \quad \tt# \hfil \cr 
\noalign{\hrule} \noalign{\smallskip} 
\{ & \\\lb & \} & \\\rb & ( & ( & ) & )\cr 
\bigl\{ & \\bigl\\\lb & \bigr\} & \\bigr\\\rb & \bigl( & \\bigl( & \bigr) & 
\\bigr)\cr 
\Bigl\{ & \\Bigl\\\lb & \Bigr\} & \\Bigr\\\rb & \Bigl( & \\Bigl( & \Bigr) & 
\\Bigr)\cr 
\biggl\{ & \\biggl\\\lb & \biggr\} & \\biggr\\\rb & \biggl( 
         & \\biggl( & \biggr) & \\biggr) \cr 
\Biggl\{ & \\Biggl\\\lb & \Biggr\} & \\Biggr\\\rb & \Biggl( 
         & \\Biggl( & \Biggr) & \\Biggr)\cr 
       } 
 
\everycr={} 
 
If you want, you can let \TeX{} choose the size of delimiter by 
using the control words {\tt \\left} and {\tt \\right} before 
your delimiters. 
\TeXref{148}Thus {\tt \\left[$\ldots$\\right]}  will cause the material 
to be enclosed by brackets that are appropriately big.  {\bf Note 
well:} each use of a {\tt \\left} delimiter must have a matching 
{\tt \\right} delimiter (although the delimiters themselves may be 
different). Hence {\tt \$\$\\left|\lb a+b \\over 
c+d\rb\\right|.\$\$} gives $$\left|{a+b \over c+d}\right|.$$ 
 
\maketable [Mathematical delimiters] 
\halign{ 
\strut \hfill$#$ & \quad \tt # \qquad\qquad & 
       \hfill$#$ & \quad \tt # \qquad\qquad & 
       \hfill$#$ & \quad \tt #  \cr 
\noalign{\hrule} \noalign{\smallskip} 
(       & (        & )          & )           & [        & [         \cr 
]       & ]        &\{          & \\\lb       & \}       & \\\rb     \cr 
\lfloor & \\lfloor &\rfloor     &\\rfloor     & \lceil   & \\lceil   \cr 
\rceil  & \\rceil  &\langle     & \\langle    & \rangle  & \\rangle  \cr 
/       & /        & \backslash & \\backslash &|         & |         \cr 
\|      & \\|      &\uparrow    & \\uparrow   & \Uparrow & \\Uparrow \cr 
\downarrow & \\downarrow & \Downarrow & \\Downarrow 
                        & \updownarrow & \\updownarrow \cr 
\Updownarrow & \\Updownarrow \cr 
       } 
 
\toindex{lfloor} 
\toindex{rfloor} 
\toindex{lceil} 
\toindex{rceil} 
\toindex{langle} 
\toindex{rangle} 
\toindex{|} 
\toindex{uparrow} 
\toindex{Uparrow} 
\toindex{downarrow} 
\toindex{Downarrow} 
\toindex{updownarrow} 
\toindex{Updownarrow} 
 
\exercise Typeset $\bigl \lceil \lfloor x \rfloor \bigr \rceil 
              \leq \bigl \lfloor \lceil x \rceil \bigr \rfloor$. 
 
\subsection{Those special functions} 
 
There are several types of functions that appear frequently in 
mathematical text.  In an equation like ``$\sin^2x + \cos^2x = 
1$'' the trigonometric functions ``sin'' and ``cos'' are in roman 
rather than italic type. This is the usual mathematical 
convention to indicate that it is a function being described and 
not the product of three variables.  The control words {\tt 
\\sin} and {\tt \\cos} will use the right typeface automatically. 
Here is a table of these and some other special functions: 
\TeXref{162} 
 
\maketable [Special mathematical functions] 
\halign{ 
   \strut \tt {\\}#\hfil && \quad \tt {\\}#\hfil \cr 
   \noalign{\hrule} \noalign{\smallskip} 
   sin    & cos    & tan  & cot  & sec  & csc & arcsin & arccos \cr 
   arctan & sinh   & cosh & tanh & coth & lim & sup    & inf    \cr 
   limsup & liminf & log  & ln   & lg   & exp & det    & deg    \cr 
   dim    & hom    & ker  & max  & min  & arg & gcd    & Pr     \cr 
       } 
 
\toindex{sin} 
\toindex{cos} 
\toindex{tan} 
\toindex{cot} 
\toindex{sec} 
\toindex{csc} 
\toindex{arcsin} 
\toindex{arccos} 
\toindex{arctan} 
\toindex{sinh} 
\toindex{cosh} 
\toindex{tanh} 
\toindex{coth} 
\toindex{lim} 
\toindex{sup} 
\toindex{inf} 
\toindex{limsup} 
\toindex{liminf} 
\toindex{log} 
\toindex{ln} 
\toindex{lg} 
\toindex{exp} 
\toindex{det} 
\toindex{deg} 
\toindex{dim} 
\toindex{hom} 
\toindex{ker} 
\toindex{max} 
\toindex{min} 
\toindex{arg} 
\toindex{gcd} 
\toindex{Pr} 
 
\exercise Typeset: $\sin(2\theta) = 2\sin\theta\cos\theta \quad 
\cos(2\theta) = 2\cos^2\theta - 1  $. 
 
\exercise Typeset: $$\int \csc^2x\, dx = -\cot x+ C 
\qquad \lim_{\alpha\to 0} {\sin\alpha \over \alpha} = 1 
\qquad \lim_{\alpha\to \infty} {\sin\alpha \over \alpha} = 0.$$ 
 
\exercise Typeset: $$\tan(2\theta) = {2\tan\theta \over 
1-\tan^2\theta}.$$ 
 
 
\subsection{Hear ye, hear ye!} 
 
There is a particular macro that is used in almost every 
mathematical paper, and is different enough to require a special 
explanation. This is the {\tt \\proclaim} macro.  It is used when 
stating theorems, corollaries, propositions, and the like.  The 
paragraph following {\tt \\proclaim} is broken into two parts: 
the first part goes up to and including the first period that is 
followed by a space, and the second part is the rest of the 
paragraph. \TeXref{202--203}The idea is that the first part 
should be something like ``Theorem 1.'' or ``Corollary B.'' The 
second part is the statement of the theorem or corollary.  Here 
is an example: 
\toindex{proclaim} 
 
\beginliteral 
\proclaim Theorem 1 (H.~G.~Wells). In the country of the blind, 
the one-eyed man is king. 
@endliteral 
 
\noindent gives 
 
\proclaim Theorem 1 (H.~G.~Wells). In the country of the blind, 
the one-eyed man is king. 
 
The statement of the theorem may contain mathematical expressions, of 
course. 
 
\vfill\eject 
\exercise Typeset: 
\nobreak 
\proclaim Theorem (Euclid). There exist an infinite number of 
primes. 
 
\exercise Typeset: 
\proclaim Proposition 1. 
$\root n \of {\prod_{i=1}^n X_i} \leq {1 \over n} \sum_{i=1}^n X_i$ 
with equality if and only if $X_1=\cdots=X_n$. 
 
\subsection{Matrices} 
 
Matrices are typeset using combinations of the alignment 
character {\tt \&} and the control word {\tt \\cr} to indicate 
the end of the line.  Start with {\tt \$\$\\pmatrix\lb$\dots$\rb\$\$}. 
Into the space between the braces go the rows of the matrix, each 
one ended by {\tt \\cr}.  The entries are separated by the 
{\tt \&}\null.  For example the input 
\TeXref{176--178} 
 
\beginliteral 
$$\pmatrix{ 
a & b & c & d \cr 
b & a & c+d & c-d \cr 
0 & 0 & a+b & a-b \cr 
0 & 0 & ab  & cd \cr 
}.$$ 
@endliteral 
 
\noindent 
gives as printed output 
$$\pmatrix{ 
a & b & c & d \cr 
b & a & c+d & c-d \cr 
0 & 0 & a+b & a-b \cr 
0 & 0 & ab  & cd \cr 
}.$$ 
\toindex{pmatrix} 
 
The matrix entries in our examples have all been \centred{} 
within their columns with a little space on each side.  They can 
be made flush right or flush left by inserting {\tt \\hfill} 
before or after the entry. Notice the differences between the 
following example and the previous one. 
 
\beginliteral 
$$\pmatrix{ 
a & b & c \hfill  & \hfill d  \cr 
b & a & c+d      & c-d      \cr 
0 & 0 & a+b      & a-b      \cr 
0 & 0 & ab \hfill & \hfill cd \cr 
}.$$ 
@endliteral 
 
\noindent 
gives as printed output 
 
$$\pmatrix{ 
a & b & c \hfill & \hfill d   \cr 
b & a & c+d     & c-d       \cr 
0 & 0 & a+b     & a-b       \cr 
0 & 0 & ab \hfill & \hfill cd \cr 
}.$$ 
 
 
\vbox{ 
\exercise Typeset 
$$ I_4 = \pmatrix{ 1 &0 &0 &0 \cr 
                   0 &1 &0 &0 \cr 
                   0 &0 &1 &0 \cr 
                   0 &0 &0 &1 \cr}$$ 
} 
 
It's possible to have matrices that use other delimiters. Using 
{\tt \\matrix} instead of {\tt \\pmatrix} will leave off the 
parentheses, so the delimiters must be explicitly included using 
{\tt \\left} and {\tt \\right}.  Here is how we can change the 
matrix of our first example. 
\toindex{matrix} 
\toindex{left} 
\toindex{right} 
 
\beginliteral 
$$ \left | 
\matrix{ 
a & b & c & d \cr 
b & a & c+d & c-d \cr 
0 & 0 & a+b & a-b \cr 
0 & 0 & ab  & cd \cr 
} 
\right | $$ 
@endliteral 
 
\noindent 
gives as printed output 
$$ \left | 
\matrix{ 
a & b & c & d \cr 
b & a & c+d & c-d \cr 
0 & 0 & a+b & a-b \cr 
0 & 0 & ab  & cd \cr 
} 
\right | $$ 
 
It's even possible to use {\tt \\left.} and {\tt \\right.} to 
indicate that the opening or closing delimiter is deleted (note 
the use of the period). 
 
\exercise Use a matrix construction to typeset 
 
$$ |x| = \left\{ \matrix{ x & x \ge 0 \cr 
                         -x & x \le 0 \cr} \right.$$ 
 
This exercise and more general constructions of this type 
may also be typeset using the {\tt \\cases} macro. 
\TeXref{175} 
 
Sometimes ellipses are used within matrices.  The control words 
{\tt \\cdots}, {\tt \\vdots}, and {\tt \\ddots} can be used to 
insert horizontal, vertical, and diagonal dots. 
 
Thus we can use 
\beginliteral 
$$ \left [ 
\matrix{ 
aa     & \cdots & az     \cr 
\vdots & \ddots & \vdots \cr 
za     & \cdots & zz     \cr 
} 
\right ] $$ 
@endliteral 
 
\noindent 
to get as printed output 
$$ \left [ 
\matrix{ 
aa     & \cdots & az     \cr 
\vdots & \ddots & \vdots \cr 
za     & \cdots & zz     \cr 
} 
\right ] $$ 
 
 
Matrices may also be typeset in-line, but they are pretty ugly 
unless they have a small number of rows. 
 
 
\subsection{Displayed equations} 
 
All of the mathematics covered so far has identical input whether 
it is to be typeset in-line or displayed.  At this point we'll 
look at some situations that apply to displayed equations only. 
 
The first is that of aligning multiline displays.  This is done with 
the alignment character {\tt \&} and the control words {\tt \\cr} and 
{\tt \\eqalign}. Starting with {\tt \$\$\\eqalign\lb$\dots$\rb\$\$}, 
the equations to be aligned are entered with each one terminated by 
{\tt \\cr}.  In each equation there should be one alignment symbol 
{\tt \&} to indicate where the alignment should take place.  This is 
usually done at the equal signs, although it is not necessary to do 
so.  For example 
\TeXref{190--192} 
\toindex{eqalign} 
 
\beginliteral 
$$\eqalign{ 
a+b &= c+d \cr 
x &= w + y + z \cr 
m + n + o + p &= q \cr 
}$$ 
@endliteral 
 
 
\noindent 
yields 
$$\eqalign{ 
a+b &= c+d \cr 
x &= w + y + z \cr 
m + n + o + p &= q \cr 
}$$ 
 
Displayed equations can be numbered at either the right or left 
margin.  When the control word {\tt \\eqno} appears in a displayed 
equation, everything after the control word is put at the right 
margin.  Hence {\tt \$\$ x+y=z. \\eqno (1)\$\$} yields 
$$ x+y=z. \eqno (1)$$ 
To number an equation at the left margin, use {\tt \\leqno} 
in place of {\tt \\eqno}. 
\toindex{eqno} 
\toindex{leqno} 
 
It's possible to number aligned equations by using the control 
word {\tt \\eqalignno}.  The alignment character {\tt \&} is 
used to separate the equation from the equation number. 
\beginliteral 
$$\eqalignno{ 
a+b &= c+d & (1) \cr 
x &= w + y + z \cr 
m + n + o + p &= q & * \cr 
}$$ 
@endliteral 
 
\noindent 
yields 
$$\eqalignno{ 
a+b &= c+d & (1) \cr 
x &= w + y + z \cr 
m + n + o + p &= q & * \cr 
}$$ 
 
Use {\tt \\leqalignno} to put the equation numbers on the left. 
\TeXref{192--193} 
\toindex{eqalignno} 
\toindex{leqalignno} 
 
Finally, suppose some text needs to appear in the middle of a 
displayed equation.  This can be done by putting it in an hbox. 
We will describe hboxes in more detail in the next section. 
For now we want to use them to temporarily resume using ordinary 
roman type and to also allow the insertion of space between 
words (remember that all spaces are ignored when typesetting 
mathematics). Hence 
{\tt \$\$X=Y \\hbox\lb{} if and only if \rb x=y.\$\$} will give 
$$X=Y \hbox{ if and only if } x=y.$$ 
Note carefully the spaces in the hbox. 
 
\exercise Do some of the challenge problems on pages 180--181 of The 
\TeX book. 
 
 
 
 
\section{All in a row} 
 
It's not uncommon to want to put a table in the middle of some 
text. Fortunately \TeX{} makes it easy to do this.  In fact there 
are two separate methods of aligning text.  The first is by using 
the tabbing environment.  This is similar to setting the tab 
stops on a typewriter. Each line is handled individually, 
according to set tab columns, but with greater flexibility than 
that provided by a typewriter. The second is the horizontal 
alignment environment which typesets the whole table at once 
using a prescribed pattern. 
 
\subsection{Picking up the tab} 
 
To align material using the tabbing environment, you must first 
set the tab positions using the {\tt \\settabs} control word. 
Having done this, a line to use these tabs starts with the 
control symbol {\tt \\+ } and ends with {\tt \\cr } (remember 
that the actual spacing on lines in the input file in 
unimportant). \toindex{settabs} 
 
The easiest way to use the {\tt \\settabs} control word is to put 
the text into equal columns. 
\TeXref{231} 
Using {\tt \\settabs 4 \\columns} will set the tabs that will 
produce four equal columns.  The tabbing is then done by using 
the alignment character {\tt \&} to move to the next tab stop. 
So, for example, \toindex{columns} 
 
\beginliteral 
\settabs 4 \columns 
\+ British Columbia & Alberta & Saskatchewan & Manitoba \cr 
\+ Ontario & Quebec & New Brunswick & Nova Scotia \cr 
\+ & Prince Edward Island & Newfoundland \cr 
@endliteral 
 
 
\noindent 
will produce the table 
\vskip\baselineskip 
 
\settabs 4 \columns 
\+ British Columbia & Alberta & Saskatchewan & Manitoba \cr 
\+ Ontario & Quebec & New Brunswick & Nova Scotia \cr 
\+ & Prince Edward Island & Newfoundland \cr 
 
Notice that it is possible to skip over some tab positions, and 
it is not necessary to use all of the tabs in a given line. To 
make the same table using five columns, it is only necessary to 
use {\tt \\settabs 5 \\columns} to reset the tab stops; then the 
same three lines from the last example will produce: 
\vskip\baselineskip 
 
\settabs 5 \columns 
\+ British Columbia & Alberta & Saskatchewan & Manitoba \cr 
\+ Ontario & Quebec & New Brunswick & Nova Scotia \cr 
\+ & Prince Edward Island & Newfoundland \cr 
 
In this example, the columns are smaller, of course.  In fact 
there are two overlapping entries in the last row.  This is 
because \TeX{} will tab to the next tab position even if (unlike 
a typewriter) it means going backward on the page. 
 
There is an interesting relationship between grouping and 
tabbing. The {\tt \\settabs} values are only applicable to the 
group in which it is defined, as would be expected.  Thus it is 
possible to temporarily change the tab settings by grouping 
within braces.  In addition, each table entry is in a group of 
its own. Hence we may make a single entry boldface, for example, 
by using {\tt \\bf}  without braces. In addition, for any column 
but the last one it is possible to \centre{} the entry or to 
align it either on the left or on the right, or to fill a column 
with a line or dots. Each entry has an implicit {\tt \\hfil} at 
the end so that it will be at the left of the column by default. 
Adding {\tt \\hfil} at the beginning of the entry will then cause 
it to be \centred{}, just as with the {\tt \\line} control word. 
Adding {\tt \\hfill} to the beginning will cause the entries to 
be pushed to the right ({\tt \\hfill} acts just like {\tt \\hfil} 
in that it absorbs excess space; when both {\tt \\hfil} and {\tt 
\\hfill} appear, the {\tt \\hfill} takes precedence). 
\toindex{hfill} 
 
\beginliteral 
\settabs 4 \columns 
\+ \hfil British Columbia & \hfill Alberta \qquad & \bf Saskatchewan 
@hskip 2.5in                                  & Manitoba \cr 
\+ \hfil Ontario & \hfill Quebec \qquad & \bf New Brunswick 
@hskip 2.5in                                 & Nova Scotia \cr 
\+ \hfil ---  & \hfill * \qquad & \bf Newfoundland 
@hskip 2.5in          & Prince Edward Island \cr 
\+ \dotfill && \hrulefill & \cr 
@endliteral 
 
\noindent 
will produce a table with the first column \centred{}, the second 
column flush right with a {\tt \\qquad} of padding, and the third 
column boldface. The control words {\tt \\dotfill} and {\tt \\hrulefill} 
give alternative column entries. 
\toindex{dotfill} 
\toindex{hrulefill} 
\vskip\baselineskip 
 
\settabs 4 \columns 
\+ \hfil British Columbia & \hfill Alberta \qquad & \bf Saskatchewan 
                                        & Manitoba \cr 
\+ \hfil Ontario & \hfill Quebec \qquad & \bf New Brunswick 
                                        & Nova Scotia \cr 
\+ \hfil ---  & \hfill * \qquad & \bf Newfoundland & 
                      Prince Edward Island \cr 
\+ \dotfill && \hrulefill & \cr 
 
\exercise Take the table of Canadian provinces above and \centre{} each 
entry within its column. 
 
The tab positions can be set with much more flexibility than just 
in equal columns.  The general pattern is to use a sample line of 
the form {\tt \\settabs \\+ $\ldots$ \& $\ldots$ \& $\ldots$ \\cr}. 
The spacing between the alignment characters {\tt \&} 
determines the position of the tabs.  For example, {\tt \\settabs 
\\+ \\hskip 1 in \& \\hskip 2 in \& \\hskip 1.5 in \& \\cr} would 
set the first tab one inch from the left margin, the next another 
two inches further in, and the third 1.5 inches more.  It's also 
possible to use text to determine the distance between tabs. So, 
for example, another possible sample line is 
{\tt \\settabs \\+ \\quad Province \\quad \& \\quad Population \\quad 
\& \\quad Area \\quad \& \\cr}. 
The tab column would then be just wide enough to accept the headings 
with a quad of space on each side.  Here's a more complete example: 
 
\beginuser \obeyspaces 
\\settabs \\+ \\quad Year \\quad \& \\quad Price \\quad 
                                            \& \\quad Dividend \& \\cr 
\\+ \\hfill Year \\quad \& \\quad Price  \\quad \& \\quad Dividend \\cr 
\\+ \\hfill 1971 \\quad \& \\quad 41--54 \\quad \& \\qquad \\\$2.60 \\cr 
\\+ \\hfill 2    \\quad \& \\quad 41--54 \\quad \& \\qquad \\\$2.70 \\cr 
\\+ \\hfill 3    \\quad \& \\quad 46--55 \\quad \& \\qquad \\\$2.87 \\cr 
\\+ \\hfill 4    \\quad \& \\quad 40--53 \\quad \& \\qquad \\\$3.24 \\cr 
\\+ \\hfill 5    \\quad \& \\quad 45--52 \\quad \& \\qquad \\\$3.40 \\cr 
\enduser 
 
\noindent 
gives \TeXref{247} 
\vskip\baselineskip 
 
\settabs \+ \quad Year \quad & \quad Price \quad & \quad 
Dividend \quad & \cr 
\+ \hfill Year \quad & \quad Price  \quad & \quad Dividend \cr 
\+ \hfill 1971 \quad & \quad 41--54 \quad & \qquad \$2.60   \cr 
\+ \hfill 2    \quad & \quad 41--54 \quad & \qquad \$2.70 \cr 
\+ \hfill 3    \quad & \quad 46--55 \quad & \qquad \$2.87 \cr 
\+ \hfill 4    \quad & \quad 40--53 \quad & \qquad \$3.24 \cr 
\+ \hfill 5    \quad & \quad 45--52 \quad & \qquad \$3.40 \cr 
 
\exercise Take the table given above and move it closer to the 
\centre{} of the page. 
 
\exercise One way to \centre{} a block of text, possibly several 
lines long, is to use: {\tt \$\$\\vbox\lb$\ldots$\rb\$\$}. Use 
this to \centre{} the table given above.  Does the 
{\tt \\settabs} line need to be included in the {\tt \\vbox}? 
 
\exercise Improve your last result by putting a line under the 
column heads.  The control word {\tt \\hrule} will insert a 
horizontal line if introduced between two rows of a table.  Now 
repeat with the control word {\tt \\strut} after the {\tt \\+ } 
of the line containing the column heads. (A {\tt \\strut} 
effectively makes the spacing between lines a little greater. The 
size can be altered from the default.)\TeXref{82} Note the extra 
space that results. 
\toindex{strut} 
 
\exercise Make the following table with decimal alignment, that 
is, with the decimal points above each other (think of the dollar 
figure as being right aligned and the cents figure as being left 
aligned against the decimal point): 
\medskip 
\settabs \+ \hskip 2 in & \hskip .75in & \hskip 1cm& \cr 
\+ &Plums &\hfill\$1&.22 \cr 
\+ &Coffee &\hfill1&.78 \cr 
\+ &Granola &\hfill1&.98 \cr 
\+ &Mushrooms & &.63 \cr 
\+ &{Kiwi fruit} & &.39 \cr 
\+ &{Orange juice} &\hfill1&.09 \cr 
\+ &Tuna &\hfill1&.29 \cr 
\+ &Zucchini & &.64 \cr 
\+ &Grapes &\hfill1&.69 \cr 
\+ &{Smoked beef} & &.75 \cr 
\+ &Broccoli &\hfill\underbar{\ \ 1}&\underbar{.09} \cr 
\+ &Total &\hfill \$12&.55 \cr 
 
 
\exercise Devise a method to make a rough table of contents by using 
{\tt \\settabs} and having entries looking something like:\hfil\break 
\leftline{\tt Getting Started \\dotfill \& \\hfill 1} 
\leftline{\tt All Characters Great and Small \\dotfill \& \\hfill 9.} 
 
\subsection{Horizontal alignment with more sophisticated patterns} 
 
The {\tt \\settabs} environment is not difficult to use, and once 
the pattern is set, it can be used repeatedly in different 
portions of the text that follows.  It does have some drawbacks, 
however. For one, the column size must be set before the entries 
are known. Also, even though in one case we wanted the third 
column to be boldface, it had to be specified in each line. 
These problems can be handled more easily by using the {\tt 
\\halign} environment. 
\TeXref{235--238} 
\toindex{halign} 
 
 
The general pattern in the {\tt \\halign} is as follows: 
\beginuser 
\\halign\lb{} <template line> \\cr 
<first display line> \\cr 
<second display line> \\cr 
$\vdots$ 
<last display line> \\cr 
\rb 
\enduser 
 
Both the template line and the display lines are divided into 
sections by the alignment symbol {\tt \&}\null. In the template 
line each section uses control words in the same manner as does 
{\tt \\line\lb\rb}\null.  The control word {\tt \\hfil}, for 
example, can be used to display flush left, flush right, or \centred. 
Fonts can be changed using {\tt \\bf}, {\tt \\it}, etc. 
Text may also be entered in the template line.  In addition the 
special symbol {\tt \#} must appear once in each section.  Each 
display line is then set by substituting each section of the 
display line into its corresponding section of the template line 
at the occurrence of the {\tt \#}\null. 
 
Consider the following example: 
 
\beginuser 
\\halign\lb\\hskip 2 in \$\#\$\& \\hfil \\quad \# \\hfil \& \\qquad \$\#\$ 
       \hskip 3in \& \\hfil \\quad \# \\hfil \\cr 
\\alpha   \& alpha   \& \\beta  \& beta  \\cr 
\\gamma   \& gamma   \& \\delta \& delta \\cr 
\\epsilon \& epsilon \& \\zeta  \& zeta  \\cr 
\rb 
\enduser 
 
\noindent The template line indicates that the first section of 
the typeset text will always be set two inches in from the left 
and also be set as mathematics.  The second section will be \centred{} 
after adding a quad of space on the left.  The third 
and fourth sections are handled similarly.  Here is the result: 
\vskip\baselineskip 
 
\halign{\hskip 2in $#$& \hfil\quad # \hfil & \qquad $#$ 
                       & \hfil\quad # \hfil\cr 
\alpha   & alpha   & \beta  & beta \cr 
\gamma   & gamma   & \delta & delta \cr 
\epsilon & epsilon & \zeta  & zeta \cr 
} 
 
In this case the first display line is formed by substituting {\tt 
\\alpha} for the first {\tt \#} in the template line, {\tt alpha} for 
the second {\tt \#}, {\tt \\beta} for the third and {\tt beta} for 
the fourth.  The whole line is then saved for setting.  This 
continues until all the lines are accumulated, and then they are set 
with each column being as wide as necessary to accept all of its 
entries (an implication of this accumulation process is that a table 
with too many entries could cause \TeX{} to run out of memory; it's 
better not to set tables that are more than a page or so long). 
 
Hence the template line establishes the pattern for the table entries 
and the display lines insert the individual entries. 
 
Sometimes horizontal and vertical lines are used to delimit 
entries in a table.  To put in horizontal lines, we use {\tt \\hrule}, 
just as we did in the {\tt \\settabs} environment. 
However, we don't want the rule to be aligned according to the 
template, so we use the control word {\tt \\noalign}.  Hence 
horizontal lines are inserted by putting {\tt \\noalign\lb\\hrule\rb}; 
vertical lines are inserted by putting {\tt \\vrule} 
in either the template or the display line. But still all is not 
completely straightforward.  Suppose we take our last example and 
change the template to get vertical lines and also insert 
horizontal lines. 
\toindex{noalign} 
 
\beginuser 
\\halign\lb\\hskip 2in\\vrule\\quad \$\#\$\\quad \& \\vrule \\hfil\\quad % 
\# \\hfil 
\hskip 2 in \& \\quad \\vrule \\quad \$\#\$\\quad 
\hskip 2 in \& \\vrule\\hfil \\quad \# \\quad \\hfil \\vrule \\cr 
\\noalign\lb\\hrule\rb 
\\alpha   \& alpha   \& \\beta  \& beta \\cr 
\\noalign\lb\\hrule\rb 
\\gamma   \& gamma   \& \\delta \& delta \\cr 
\\noalign\lb\\hrule\rb 
\\epsilon \& epsilon \& \\zeta  \& zeta \\cr 
\\noalign\lb\\hrule\rb 
\rb 
\enduser 
 
\noindent 
doesn't give exactly what we want. 
 
\vskip\baselineskip 
 
\halign{\hskip 2in\vrule\quad $#$\quad & \vrule \hfil\quad # \hfil 
& \quad \vrule \quad $#$\quad & \vrule \hfil\quad # \quad \hfil 
\vrule \cr 
\noalign{\hrule} 
\alpha   & alpha   & \beta  & beta \cr 
\noalign{\hrule} 
\gamma   & gamma   & \delta & delta \cr 
\noalign{\hrule} 
\epsilon & epsilon & \zeta  & zeta \cr 
\noalign{\hrule} 
} 
 
There are several deficiencies: the most obvious is the extended 
horizontal lines, but also the text looks somewhat squashed into 
the boxes.  In addition, the text has a little extra space on the 
right rather than being perfectly \centred. As in the {\tt \\settabs} 
environment, lines can be made taller by including the control 
word {\tt \\strut} in the template.\TeXref{82} A further problem 
can occur when the page is set since \TeX{} may spread lines 
apart slightly to improve the appearance of the page.  This would 
leave a gap between the vertical lines, so we use the control 
word {\tt \\offinterlineskip} within the {\tt \\halign} to avoid 
this.  Finally we can get rid of the lines sticking out on the 
left by deleting the {\tt \\hskip 2 in} from the template line. 
To move the table to the same position we use {\tt \\moveright}. 
Finally, we can see how to \centre{} the text by noting that the 
extra space occurs in the template line after the {\tt \#} where 
the text is inserted. 
Hence we can improve our result by using 
\toindex{offinterlineskip} 
\toindex{moveright} 
 
\beginuser 
\\moveright 2 in 
\\vbox\lb\\offinterlineskip 
 
\\halign\lb\\strut \\vrule \\quad \$\#\$\\quad \&\\vrule \\hfil \\quad % 
\#\\quad \\hfil 
\&\\vrule \\quad \$\#\$\\quad \&\\vrule \\hfil \\quad \#\\quad \\hfil % 
\\vrule \\cr 
\\noalign\lb\\hrule\rb 
\\alpha   \& alpha   \& \\beta  \& beta \\cr 
\\noalign\lb\\hrule\rb 
\\gamma   \& gamma   \& \\delta \& delta \\cr 
\\noalign\lb\\hrule\rb 
\\epsilon \& epsilon \& \\zeta  \& zeta \\cr 
\\noalign\lb\\hrule\rb 
\rb\rb 
\enduser 
 
\noindent 
to get 
\vskip\baselineskip 
 
\moveright 2 in 
\vbox{\offinterlineskip 
\halign{\strut \vrule \quad $#$\quad &\vrule \hfil \quad #\quad \hfil 
&\vrule \quad $#$\quad &\vrule \hfil \quad #\quad \hfil 
\vrule \cr 
\noalign{\hrule} 
\alpha   & alpha   & \beta  & beta \cr 
\noalign{\hrule} 
\gamma   & gamma   & \delta & delta \cr 
\noalign{\hrule} 
\epsilon & epsilon & \zeta  & zeta \cr 
\noalign{\hrule} 
} 
} 
 
 
 
In general, if we want to construct a table with boxed entries 
that is \centred{} on the page, we can do so by putting the {\tt 
\\vbox} within a {\tt \\centerline\lb\rb}\null. But here is a 
trick that will produce a nicer result.  If the {\tt \\vbox} is 
put in between double dollar signs, it will be typeset as 
displayed mathematics.  Of course, there is no actual 
mathematics being displayed, but \TeX{} will put in a little 
extra space above and below the table as is appropriate for a 
display.  Hence a \centred{} table with this nice spacing may be 
formed using the following four steps: 
(1) put a {\tt \\vbox} between double dollar signs, 
(2) put an {\tt \\offinterlineskip} and an {\tt \\halign} within 
the {\tt \\vbox}, 
(3) in the {\tt \\halign} put a template line with a {\tt \\strut} 
in the beginning,  and a {\tt \\vrule} surrounding each entry, 
(4) each row of the table should be preceded and followed by 
{\tt \\noalign\lb\\hrule\rb}. 
 
Here is the pattern to be followed: 
 
\beginuser 
\$\$\\vbox\lb 
\\offinterlineskip 
\\halign\lb 
\\strut \\vrule \# \& \\vrule \# \& \dots \& \\vrule \# \\vrule \\cr 
\\noalign\lb\\hrule\rb 
<first column entry> \& <second column entry> \& \dots % 
\& <last column entry> \\cr 
\\noalign\lb\\hrule\rb 
\dots 
\\noalign\lb\\hrule\rb 
<first column entry> \& <second column entry> \& \dots % 
\& <last column entry> \\cr 
\\noalign\lb\\hrule\rb 
\rb 
\rb\$\$ 
\enduser 
 
 
 
\section{Rolling your own} 
 
In this section we'll create new control words.  The making of 
these new definitions, also called macros, is one of the most 
powerful techniques available in \TeX\null.  For the first 
application of this facility, we'll see how a new definition can 
save a lot of typing by substituting short strings for long ones. 
 
\subsection{The long and short of it} 
 
The control word {\tt \\def} is used to define new control words. 
The simplest form for doing this is {\tt \\def\\newname\lb$\ldots$\rb}. 
Then whenever {\tt \\newname} appears in your 
input file, it will be replaced by whatever is between the braces 
in the definition.  Of course {\tt \\newname} must satisfy the 
convention for naming control sequences, that is, it must be a 
control word (all letters) or a control symbol (exactly one 
nonletter). So, for example, suppose you write a document that 
contains the phrase ``University of Manitoba'' many times.  Then 
{\tt \\def\\um\lb University of Manitoba\rb} defines a new 
control sequence {\tt \\um} which can then be used at any time. 
The sentence {\tt I take courses at the \\um.}\ then makes 
sense. If the control word exists, your new definition will 
replace it (this includes the control words defined by \TeX{}, so 
a little care must be taken in the choice of name). Any 
definition is, however,  local to the group in which it is 
defined.  For example, \toindex{def} 
 
\beginuser 
\\def\\um\lb University of Manitoba\rb 
I took my first course at the \\um. 
\lb 
\\def\\um\lb Universit\\'e de Montr\\'eal\rb 
Then I took my next course at the \\um. 
\rb 
Finally I took my last course at the \\um. 
\enduser 
 
\noindent gives \medskip 
 
\def\um{University of Manitoba} 
I took my first course at the \um. 
{ 
\def\um{Universit\'e de Montr\'eal} 
Then I took my next course at the \um. 
} 
Finally I took my last course at the \um. 
 
Remember that all spaces after a control word are absorbed; this 
includes the control words that you define. In the previous 
example, any space after {\tt \\um} would be ignored. However, 
the space after the first period and the space after the first 
opening brace are different; if you look closely at the end of 
the first sentence typeset using the example, you'll see some 
extra space. This can be eliminated by putting a {\tt \%} after 
the opening brace to make the rest of the line a comment. The 
same holds for the line with the last closing brace. Careful 
control of spaces often calls for the ``commenting out'' of the 
end of lines in this manner. 
 
Once a new control sequence has been defined, it may be used in 
new definitions. This is one way of making simple form letters. 
First let's define a simple letter. 
 
\beginuser 
\\def\\letter\lb 
\\par \\noindent 
Dear \\name, 
\ 
This is a little note to let you know that your name is \\name. 
\ 
\\hskip 2 in Sincerely yours, 
\\vskip 2\\baselineskip 
\\hskip 2 in The NameNoter 
\\smallskip \\hrule 
\rb 
\enduser 
 
\def\letter{ 
\par \noindent 
Dear \name, 
 
This is a little note to let you know that your name is \name. 
 
\hskip 2 in Sincerely yours, 
\vskip 2\baselineskip\nobreak 
\hskip 2 in The NameNoter 
\smallskip \hrule 
} 
 
Now this letter uses the control word {\tt \\name}, which is 
undefined at this point.  When {\tt \\letter} is used, the 
current value of {\tt \\name} will appear in the body of the 
letter. Hence 
 
\beginuser 
\\def\\name\lb Michael Bishop\rb 
\\letter 
\\def\\name\lb Michelle L\\'ev\\\^{}eque\rb 
\\letter 
\enduser 
 
\noindent will produce two copies of the letter, each with the 
correct name, followed by a horizontal rule: 
 
\def\name{Michael Bishop} 
\letter 
\goodbreak 
\def\name{Michelle L\'ev\^eque} 
\letter 
 
We could have put anything between the braces in 
{\tt \\def\\name\lb$\ldots$\rb}; it could be several paragraphs 
long and use other control sequences (although in this context it 
would be a little strange). Of course it is possible to use {\tt 
\\vfill \\eject} as part of the definition of {\tt \\letter} to 
eject the page when the letter is completed. 
 
\exercise Make a form letter that uses the control words {\tt \\name}, 
{\tt \\address}, {\tt \\city}, {\tt \\\province}, and {\tt 
\\\postcode}. 
 
\exercise An unnumbered list of items is often made 
using {\tt \\item\lb\$\\bullet\$\rb}.  Define a macro {\tt \\bitem} 
that does this, and use it for several paragraphs. Now change 
each bullet to a dash (note that a simple change in the macro 
propagates all the necessary changes in all of the paragraphs). 
 
\exercise Suppose that you are going to have to format several 
paragraphs in a paper using {\tt \\hangindent = 30 pt}, {\tt \\hangafter 
= 4}, and {\tt \\filbreak} (don't worry about what 
these control sequences actually do; the only important thing 
for now is that once they are set, they remain in effect for only 
one paragraph).  Define a single control sequence {\tt \\setpar} 
which can then be put in front of each paragraph that needs to be 
so formatted. 
 
 
\subsection{Filling in with parameters} 
 
It's possible to use macros in much greater generality by 
allowing parameters to be passed.  The idea is somewhat similar 
to the template line in the {\tt \\halign} environment.  First, 
let's look at the case where there is one parameter.  In this 
case a control sequence is defined by {\tt \\def\\newword\#1\lb 
$\ldots$\rb}\null.  The symbol {\tt \#1} may appear between the 
braces (several times) in the definition of {\tt \\newword}.  The 
material between the braces acts like a template. When {\tt 
\\newword\lb$\ldots$\rb} appears in the text, it will use the 
definition of {\tt \\newword} with the material between the 
braces inserted into the template at every occurrence of {\tt 
\#1} in the original definition. {\bf The spacing in the original 
definition is crucial here; there must be no spaces before the 
opening brace}. 
 
As an example, we could use the form letter of the last section in 
the following way: 
 
 
\beginuser 
\\def\\letter\#1\lb 
\\par \\noindent 
Dear \#1, 
\vskip\baselineskip 
This is a little note to let you know that your name is \#1. 
\vskip\baselineskip 
\\hskip 2 in Sincerely yours, 
\\vskip 2\\baselineskip 
\\hskip 2 in The NameNoter 
\\smallskip \\hrule 
\rb 
\enduser 
 
\def\letter#1{ 
\par \noindent 
Dear #1, 
 
This is a little note to let you know that your name is #1. 
 
\hskip 2 in Sincerely yours, 
\vskip 2\baselineskip 
\hskip 2 in The NameNoter 
\smallskip \hrule 
} 
 
Now we can use 
 
\beginuser 
\\letter\lb Michael Bishop\rb 
\\letter\lb Michelle L\\'ev\\\^{}eque\rb 
\enduser 
 
\noindent to get \medskip 
\letter{Michael Bishop} 
\goodbreak 
\letter{Michelle L\'ev\^eque} 
 
\def\displaytext#1{$$\vbox{\hsize=12cm #1}$$} 
\bigskip 
\displaytext{ 
Now let's define 
{\tt\\def\\displaytext\#1\lb \$\$\\vbox\lb\\hsize = 12 cm \#1\rb\$\$\rb} 
as a new macro to display text. 
Then {\tt \\displaytext\lb$\ldots$\rb} will cause the material 
between the braces to be put in a paragraph with width 12 \centimetre 
s and then \centred{} with some space added above and 
below as is appropriate for a display.  This paragraph was set 
using this {\tt \\displaytext} macro. 
} 
 
The parameter of a macro can be no more than one paragraph long. 
If a new paragraph is encountered as part of a parameter, an 
error will be generated.  This is a safety feature, for 
otherwise the accidental omission of a closing brace would cause 
\TeX{} to eat up the rest of the file as the parameter. 
 
\exercise Define a macro {\tt \\yourgrade} so that {\tt 
\\yourgrade\lb89\rb} will cause the following sentence to be 
typeset: The grade you received is 89\%\null. It should be able 
to work with any other percentage, of course. 
\medbreak 
 
It's not really any harder to use more than one parameter.  The form 
used to define a new control word with two parameters is 
{\tt \\def\\newword\#1\#2\lb$\ldots$\rb}. The definition between the 
braces may have {\tt \#1} and {\tt \#2} occurring in it, perhaps 
several times.  When {\tt \\newword\lb$\ldots$\rb\lb$\ldots$\rb} 
appears in the text, the material between the first set of braces 
replaces {\tt \#1} in the definition and the material between the 
second set of braces replaces {\tt \#2} in the definition.  Here is 
an example followed by its result: 
 
\beginuser 
\\def\\talks\#1\#2\lb \#1 talks to \#2.\rb 
\\talks\lb John\rb\lb Jane\rb 
\\talks\lb Jane\rb\lb John\rb 
\\talks\lb John\rb\lb me\rb 
\\talks\lb She\rb\lb Jane\rb 
\enduser 
 
\def\talks#1#2{#1 talks to #2.} 
\talks{John}{Jane} 
\talks{Jane}{John} 
\talks{John}{me} 
\talks{She}{Jane} 
 
\exercise In a manner similar to the previous exercise, define a 
macro {\tt \\yourgrade} so that {\tt \\yourgrade\lb89\rb\lb85\rb} 
causes the following sentence to be typeset: You received a grade 
of 89\% on your first exam and a grade of 85\% on your second 
exam. 
 
\exercise Write a macro {\tt \\frac} so that {\tt \\frac\lb 
a\rb\lb b\rb} will typeset the fraction ${a\over b}$. 
 
\bigskip 
It's important not to put any spaces before the first brace in 
the definition. If you do, \TeX{} will interpret the definition 
differently from the way described here. For more than two 
parameters, the method of definition is similar. To define a 
control word with three parameters, start with 
{\tt \\def\\newword\#1\#2\#3\lb$\ldots$\rb}. Then {\tt \#1}, 
{\tt \#2} and {\tt \#3} may occur between the braces. When 
{\tt \\newword\lb$\ldots$\rb\lb$\ldots$\rb\lb$\ldots$\rb} 
appears in the text, the material between each set of braces 
replaces its corresponding symbol in the definition of the 
control word. The parameters may go up to {\tt \#9}. 
 
\subsection{By any other name} 
 
Sometimes it's convenient to be able to give a control word an 
alternative name.  For example, if you prefer a different spelling, 
you might want to call the control word {\tt \\centerline} by the 
name of {\tt \\centreline}.  This can be done by using the {\tt 
\\let} control word.  The use of {\tt \\let \\centreline = 
\\centerline} now makes a new (as well as the old) control word 
available.  This can also be used with mathematical names as with 
{\tt \\let \\tensor = \\otimes}.  It is then possible to use 
\TeXref{206--207} 
 
\toindex{let} 
\toindex{centreline} 
\toindex{tensor} 
 
\let\tensor=\otimes 
\beginuser 
\$\$ (A \\tensor B) (C \\tensor D) = AC \\tensor BD. \$\$ 
\enduser 
 
\noindent to get 
 
$$ (A \tensor B) (C \tensor D) = AC \tensor BD. $$ 
 
 
 
\exercise Define control sequences {\tt \\ll}, {\tt \\cl}, and 
{\tt \\rl} that are equivalent to {\tt \\leftline}, {\tt \\centerline}, 
and {\tt \\rightline}. 
 
The {\tt \\let} control word allows users to name their own 
control sequences.  This allows a personalized set of control 
sequences that may be used in place of the ones provided by \TeX{} 
when desired. 
 
\section{To err is human} 
 
In some ways \TeX{} is not completely divine. \TeX{} will respond 
to invalid input by giving an error message to the screen if you 
are using it interactively and also to the log file.  Because 
\TeX{} is very complicated, the actual point where the error is 
detected may be deep within the program, so a full report of the 
error may be rather long and involved.    Not only that, \TeX{} 
will try to recover from errors, and will report what was done in 
that process.  For this reason the reading of error messages may 
be a little difficult for the uninitiated.  The key is to know 
what is important from your perspective and what can be safely 
ignored. So let's look at some typical errors and the messages 
that they generate. 
 
\subsection{The forgotten bye} 
 
The first mistake that we'll look at is one that everyone makes 
at some time, namely, the omission of {\tt \\bye} at the end of 
the file. If you're using \TeX{} interactively, an asterisk \hfil 
\break 
\leftline{\tt *} 
will be printed on the screen and nothing will happen since, 
having not been told to finish up,  \TeX{} is waiting for input 
(from your keyboard).  Whatever you type in will be appended to 
whatever has been input from your files.  The usual response is 
to type {\tt \\bye<CR>}\fnote{{\tt <CR>} is the key 
used to end a line of input.  It might be called the carriage 
return, enter, or simply the return key on your terminal. 
Sometimes it is indicated by a large left arrow.} since that will 
finish things up. 
 
\subsection{The misspelled or unknown control sequence} 
 
Using a misspelled or other control sequence unknown to \TeX{} is 
a common error.  If \TeX{} is being run as a batch job, an error 
message is printed and the job goes on ignoring the control 
sequence.  When using \TeX{} interactively, it is possible to 
repair errors (of course this does not change the original input 
file, so that must be done when the \TeX{} job is completed). 
Suppose we have a \TeX{} input file consisting of the following 
two lines: 
\beginuser 
\\line\lb The left side \\hfli the right side\rb 
\\bye 
\enduser 
 
The control word should be {\tt \\hfil}, of course. Here is the 
message that would be sent to your terminal: 
 
\beginuser 
\obeyspaces 
! Undefined  control sequence. 
l.1 \\line\lb The left side \\hfli 
\                               the right side\rb 
? 
\enduser 
 
The first line starts with {\tt !} and gives the error message. 
Next comes the line number on which the error occurred and the 
part of the line that was read successfully.  The next line gives 
the continuation of the line after the error. At this point the 
question mark means that \TeX{} is waiting for a response.  There 
are several legal ones: 
\nobreak 
\maketable [Responses to \TeX{} error messages] 
\halign{ 
\strut \hfil # \hfil & \quad \hfil \tt # \hfil & # \hfil\cr 
   \bf Desired response & \bf Input to \TeX{} & \bf \hfil Result\cr 
   \noalign{\hrule} \noalign{\smallskip} 
   Help   & h<CR>& Reason for stopping listed on terminal.\cr 
   Insert & i<CR>& Next line inserted into \TeX{} input file.\cr 
   Exit   & x<CR>& Exit from \TeX\null. Completed pages to DVI file.\cr 
   Scroll & s<CR>& List message and continue after minor errors.\cr 
   Run    & r<CR>& List message and continue after any errors.\cr 
   Quiet  & q<CR>& All terminal listings suppressed.\cr 
   Carry on &<CR> & \TeX{} continues as best it can.\cr 
      } 
 
In our last example a reasonable response might be to enter {\tt 
h<CR>} to get a help message, then {\tt i<CR>} to insert more 
text, (at which point \TeX{} responds with {\tt insert> } and 
finally {\tt \\hfil} as the correct control word.  Here is the 
result: 
 
 
\beginuser 
? h <CR> 
The control sequence at the end of the top line 
of your error message was never \\def'ed. If you have 
misspelled it (e.g., `\\hobx'), type `I' and the correct 
spelling (e.g., `I\\hbox'). Otherwise just continue, 
and I'll forget about whatever was undefined. 
 
? i <CR> 
insert>\\hfil 
[1] 
\enduser 
 
The final {\tt [1]} means that the first (and only) page has 
been completed and sent to the DVI file.  The original input file 
still needs to be fixed, of course. 
 
\subsection{The misnamed font} 
 
A misspelled font name is an error similar to the misspelled 
control sequence.  The error message is different and a little 
confusing at first. Suppose for example the following appears in 
your input file: 
 
\leftline{\tt \\font\\sf = cmss01} 
 
It should be {\tt cmss10}, that is, the numbers have been 
transposed.  Here are the error and help messages: 
 
\beginuser \obeyspaces 
! Font \\sf=cmss01 not loadable: Metric (TFM) file not found. 
<to be read again> 
\                   \\par 
\\bye ->\\par 
\            \\vfill \\supereject \\end 
l.2 \\bye 
 
? h <CR> 
I wasn't able to read the size data for this font, 
so I will ignore the font specification. 
[Wizards can fix TFM files using TFtoPL/PLtoTF.] 
You might try inserting a different font spec; 
e.g., type `I\\font<same font id>=<substitute font name>'. 
\enduser 
 
The TFM (\TeX{} font metric) file is an auxiliary file that is 
used by \TeX\null. So this strange message is just telling you 
that the font you defined doesn't exist on your computer system. 
 
\subsection{Mismatched mathematics} 
 
Another common error is to use {\tt \$} or {\tt \$\$} to start a 
mathematical expression and then to forget the second {\tt \$} 
or {\tt \$\$} when finished.  The text that follows is then 
treated as mathematics, and to make matters worse, if more 
mathematics is started by a new {\tt \$} or {\tt \$\$}, it will 
then be treated as ordinary text.  Needless to say, error 
messages galore may be generated. \TeX{} will attempt to correct 
the problem by inserting a new {\tt \$} or {\tt \$\$}; in any 
case, the problem is corrected by the end of the paragraph since 
a new paragraph will automatically start as ordinary text. 
 
Consider the following correct input and its output: 
\beginuser 
Since \$f(x) > 0\$, \$a<b\$,  and \$f(x)\$ is continuous, we know that 
\$\\int\_{}a\^{}b f(x)\\,dx >0\$. 
\enduser 
 
Since $f(x) > 0$, $a<b$,  and $f(x)$ is continuous, we know that 
$\int_a^b f(x)\,dx >0$. 
 
If we now leave out the second dollar sign in {\tt \$f(x)\$} we 
then get the following error and help messages: 
 
\beginuser \obeyspaces 
! Missing \$ inserted. 
<inserted text> 
\               \$ 
<to be read again> 
\                  \\intop 
\\int ->\\intop 
\             \\nolimits 
l.2 \$\\int 
\          \_{}a\^{}b f(x)\\,dx >0\$. 
? h <CR> 
I've inserted a begin-math/end-math symbol since I think 
you left one out. Proceed, with fingers crossed. 
 
? 
\enduser 
 
The line starting with {\tt !} tells us what has been done. The 
line starting with {\tt l.2} shows us where we were in the input 
file when the error occurred.  As in our other examples, the part 
of the line successfully read, that is, through {\tt \\int}, 
appears on one line, and the continuation appears on the next 
line. The remaining material may seem somewhat obscure.  These 
intermediate messages show what was happening further in the guts 
of the \TeX{} program when the error occurred.  The newer user 
may ignore them. 
 
Here is what you get as output after \TeX{} tries to recover from 
the error. 
 
Since $f(x) > 0$, $a<b$,  and $f(x) is continuous, we know that 
\int_a^b f(x)\,dx >0$. 
 
There is a stretch of text that is italic with no spacing.  This 
is typical for normal text being processed as mathematics; if you 
see this in your output, you have almost certainly left out a 
{\tt \$} or {\tt \$\$}. 
 
\subsection{Mismatched braces} 
 
It's easy to forget or mismatch the closing braces when making 
groups.  The result may be a relatively benign error, or it may 
be catastrophic. Suppose, for example, you have {\tt \lb\\bf A 
bold title } in your text with the closing right brace omitted. 
The result will be the same as if no opening brace were there; 
that is, the rest of the paper will be boldface if no other font 
changes are made.  You will get the following message at the end 
of the file: 
 
{\tt (\\end occurred inside a group at level 1)} 
 
If you had made the same mistake twice, then there would be two 
more opening braces than closing braces, and you would get the 
message: 
 
{\tt (\\end occurred inside a group at level 2)} 
 
\TeX{} doesn't know that the closing brace is missing until it 
reaches the end of the input file.  Hence the message doesn't 
tell you where you went wrong.  If the location of the missing 
brace isn't obvious, it's always possible to insert {\tt \\bye} 
halfway through your document.  Running \TeX{} again will cause 
only the first half to be processed, and if the error message 
persists, you will know that the error is in the first half of 
the document.  By moving the {\tt \\bye} to different places, the 
error can be localized.  Also, looking at the output often 
reveals what has gone wrong. 
 
Missing opening braces are much easier to spot.  Here is a two 
line input file and the resulting error and help messages: 
 
\beginuser 
\\bf Here is the start\rb, but there is the finish. 
\\bye 
\enduser 
 
\beginuser \obeyspaces 
! Too many \rb's. 
l.1 \\bf Here is the start\rb 
\                          , but there is the finish. 
 
? h <CR> 
You've closed more groups than you opened. 
Such booboos are generally harmless, so keep going. 
\enduser 
 
It's quite possible, of course, that the line that is supposed to 
have the missing left brace will not be on the line where \TeX{} 
catches the error. 
 
A mismatched brace in the definition of a new control sequence 
can cause a major error.  Since such a definition may include 
several paragraphs, it may not be caught by the end of a 
paragraph, but, rather will just keep piling more and more text 
into the unfinished definition.  It's even possible for \TeX{} to 
run out of memory as it keeps eating up more text! This is called 
a ``runaway definition''. 
\TeXref{206} 
Here is a two line input file with a runaway definition: 
 
\vbox{ 
\beginuser 
\\def\\newword\lb the def 
\\newword 
\\bye 
\enduser 
} 
 
Here are the resulting error and help messages: 
 
\beginuser \obeyspaces 
Runaway definition? 
->the def 
! Forbidden control sequence found while scanning definition of \\newword. 
<inserted text> 
\                \rb 
<to be read again> 
\                   \\bye 
l.3 \\bye 
 
? h <CR> 
I suspect you have forgotten a `\rb', causing me 
to read past where you wanted me to stop. 
I'll try to recover; but if the error is serious, 
you'd better type `E' or `X' now and fix your file. 
 
? <CR> 
No pages of output. 
\enduser 
 
This is obviously a serious error.  If it occurs at the beginning 
of a file (as in the previous example), there will be no output 
at all! 
 
If a closing brace is left out while using a macro with 
parameters, the runaway definition will be terminated at the end 
of the paragraph. So if {\tt \\def\\newword\#1\lb$\ldots$\rb} has 
been defined and you use {\tt \\newword\lb$\dots$ } without the 
closing brace, then at most one paragraph will be ruined. 
\TeXref{205} 
 
In short, when an error occurs, make a note of the line number to 
see how much of the input file has been read, and also the line 
starting with an exclamation point to get a short description of 
the error. If the error is still not clear, ask \TeX{} for more 
information by typing {\tt h<CR>}. For small errors, \TeX{} can 
carry on quite a way if you just keep hitting the {\tt <CR>}. 
 
 
 
 
\section{Digging a little deeper} 
 
In this section we look at a few topics that allow \TeX{} to be 
used with greater flexibility or efficiency.  As the documents 
being produced get longer, different techniques can help make 
their creation easier. 
 
\subsection{Big files, little files} 
 
\TeX{} can read and write files as it runs.  This makes it 
possible to use files that are smaller and more convenient to 
handle by creating a master file that reads the smaller files in 
the proper order. This document, for example, consists of ten 
sections plus an introduction.  In addition, there are macros 
that are used for all sections. The macros can be put in a file 
called, say, {\tt macros.tex}, the introduction can be put in {\tt 
intro.tex}, and each section put in its own file.  The control 
word {\tt \\input} is then used to read in a file.  In general, 
{\tt \\input filename} will cause the file called {\tt 
filename.tex} to be read in and processed immediately, just as if 
the text of {\tt filename.tex} had been part of the file that 
read it in.  This file may input other files. In fact it's often 
convenient to make a single file that reads in smaller pieces, 
perhaps as follows: 
\toindex{input} 
 
\beginuser 
\\input macros 
\\input intro 
\\input sec1 
\\input sec2 
\\input sec3 
\\input sec4 
\\input sec5 
\\input sec6 
\\input sec7 
\\input sec8 
\\input sec9 
\\input sec10 
\enduser 
 
While the text is still being heavily edited, it's possible to 
process only some of the files by putting a {\tt \%} at the 
beginning of each line that contains a file to be skipped (this 
is sometimes called ``commenting out'' the unwanted files). 
 
The {\tt \\input} control word also allows the use of 
predesigned macros.  The macros for a memorandum, for example, 
might be put in a file called {\tt memo.tex}.  These macros might 
set up the right {\tt \\hsize}, {\tt \\vsize} and other 
parameters, and might stamp the time and date.  Once this has 
been set up, all memoranda may be started with {\tt \\input memo} 
to make them come out with a common format. 
 
Be sure that you don't have the control word {\tt \\bye} in your 
input file or the \TeX{} program will stop at that point. 
 
\exercise Make a \TeX{} input file that reads in a second file. 
Try reading in the second file twice using the {\tt \\input} 
control word twice. 
 
\subsection{Larger macro packages} 
 
Designing macros that can be used with many types of documents 
is obviously useful.  Most universities, for example, have 
specific and often complicated format requirements for theses.  A 
collection of macros, that is, a macro package that meets all 
these specifications could be somewhat time consuming to design 
and could be quite long.  It is possible to use the {\tt \\input} 
command to use such a macro package, just as it is possible to 
use it with your own macros.  But \TeX{} has a better facility 
for larger packages. 
 
A macro package can be put in a special form that can be quickly 
read by \TeX\null.  This is called a {\sl format file}, and 
the exact form is of technical interest only.  The important 
thing is it allows \TeX{} to be run with many new control 
sequences predefined. Certain commands called {\sl primitives\/} 
are part of the definition of \TeX\null. 
 
What we have described in this manual is sometimes called {\sl 
plain \TeX\/}, and consists of the \TeX{} primitives plus a set 
of macros in a format file (that is usually included in \TeX{} 
automatically) called {\tt plain.fmt}. For the curious, any 
control word can be viewed using {\tt \\show}.  The command {\tt 
\\show\\centerline} will display 
\beginliteral 
> \centerline=macro: 
#1->\line {\hss #1\hss }. 
@endliteral 
\noindent 
on the screen and in the log file. You can use {\tt \\show} with 
your own macros, too.  If you end up using several macro 
packages, you can use the {\tt \\show} command to see if a 
particular macro is defined. 
 
Many computer \centre{}s have the \LaTeX{} macro package.  This 
package allows the user to create an index, a table of contents, 
and a bibliography automatically.  It also has the ability to 
insert some elementary graphic figures such as circles, ovals, 
lines, and arrows. \LaTeX{} also uses special predefined files 
called {\sl style files\/} to set up 
specific page parameters. Many different style files are 
available; some journals will accept papers on a magnetic medium 
for direct processing if they are prepared using \LaTeX{} and a 
designated style file. It is not difficult to shift from \TeX{} 
to \LaTeX\null. A user's guide by the author of the macro 
package, Leslie Lamport, is available: 
{\bf \LaTeX{}: A document preparation system}% 
\fnote{Addison-Wesley, Reading, Massachusetts, 1986, 
ISBN 0-201-15790-X.}. 
 
The American Mathematical Society uses the \AMSTeX{} macro 
package for its journals.  It is readily available from that 
Society, and papers may be submitted to their journals on a 
magnetic medium using \AMSTeX\null.  A manual by Michael Spivak, 
{\bf The Joy of \TeX{}}\fnote{American Mathematical 
Society, 1986, ISBN 0-8218-2999-8}, is available from the 
American Mathematical Society. 
 
Other macro packages exist, and undoubtedly more will be 
developed. They are usually of modest cost and can be very 
effective in some circumstances.  The \TeX{} Users Group 
announces the existence of new macro packages in its 
publications. 
 
\subsection{Horizontal and vertical lines} 
 
Making horizontal and vertical lines is easy using \TeX\null. 
When typing in text, {\tt \\hrule } will cause the current 
paragraph to end, will draw a horizontal line whose width is the 
current value of {\tt \\hsize}, and then will continue on with a 
new paragraph.  It's possible to specify the width of the hrule 
as, for example, with {\tt \\hrule width 5 cm}; also you can use 
{\tt \\vskip} or {\tt \\bigskip} to put some space above or below 
the hrule.  Here is an example: 
 
\beginuser 
\\parindent = 0 pt \\parskip = 12 pt 
Here is the text before the hrule. 
\\bigskip 
\\hrule width 3 in 
And here is some text after the hrule. 
\enduser 
 
\noindent that produces 
\vfill\eject 
 
{ 
\parindent = 0 pt 
Here is the text before the hrule. 
\bigskip 
\hrule width 3 in 
And here is some text after the hrule. 
} 
 
In fact this hrule not only has width of three inches, but also 
by default has a height (the amount by which the hrule extends 
above the baseline on which the type is being set) of 0.4 points 
and a depth (the amount by which the hrule extends below the 
baseline on which the type is being set) of 0 points.  Each of 
these parameters can be individually set.  Thus if we change the 
last example to say 
 
{\tt \\hrule width 3 in height 2 pt depth 3 pt } we get 
 
{ 
\parindent = 0 pt 
Here is the text before the hrule. 
\bigskip 
\hrule width 3 in height 2 pt depth 3 pt 
And here is some text after the hrule. 
} 
 
The three parameters {\tt width}, {\tt height}, and {\tt depth} may 
be given in any order. 
\toindex{hrule} 
\TeXref{221--222} 
 
A vrule may be defined analogously to an hrule by specifying the 
{\tt width}, {\tt height}, and {\tt depth} if desired. 
\TeXref{221--222} 
But, unlike the hrule, the vrule will not automatically start a 
new paragraph when it appears.  By default the vrule will be 0.4 
points wide, and will be as high as the line on which it is being 
set. Hence 
 
\toindex{vrule} 
 
\beginuser 
Here is some text before the vrule 
\\vrule\\ 
and this follows the vrule. 
\enduser 
 
\noindent will give 
 
Here is some text before the vrule 
\vrule\ 
and this follows the vrule. 
 
\exercise Make three horizontal lines that are 15 points apart, 
3 inches in length, and one inch in from the left margin. 
 
Although we usually think of hrules and vrules as horizontal and 
vertical lines, they need not necessarily be used that way.  For 
example: 
 
\beginuser 
\\noindent 
Name: \\vrule height 0 pt depth 0.4 pt width 3 in 
\enduser 
 
\noindent will give 
 
\noindent 
Name: \vrule height 0 pt depth 0.4 pt width 3 in 
 
\exercise Make the following grid (each box is 1 \centimetre{} square): 
\medskip 
\settabs \+ \hskip 1cm&\hskip 1 cm&\hskip 1 cm& \cr 
\moveright 2 in 
\vbox{ 
\hrule width 3 cm 
\+  \vrule height 1 cm & \vrule height 1 cm & \vrule height 1 cm 
  & \vrule height 1 cm \cr 
\hrule width 3 cm 
\+  \vrule height 1 cm & \vrule height 1 cm & \vrule height 1 cm 
  & \vrule height 1 cm \cr 
\hrule width 3 cm 
\+  \vrule height 1 cm & \vrule height 1 cm & \vrule height 1 cm 
  & \vrule height 1 cm \cr 
\hrule width 3 cm 
} 
 
 
\subsection{Boxes within boxes} 
 
We have already seen (in our discussion of line shapes) that 
vboxes and hboxes are objects that may be overfull or underfull. 
In this section we will look at these boxes in a bit more detail. 
They may be stacked or lined up to allow a variety of positions 
for text on the page. 
 
An hbox is formed by using {\tt \\hbox\lb $\ldots$\rb}\null. 
Once the material between the braces has been put into an hbox, 
it is set and can not be further split (this means that material 
that must go on one line can be put into an hbox, and it will 
then remain as one unit).  It's possible to specify the size of 
an hbox.\TeXref{64--66} Thus {\tt \\hbox to 5 cm\lb contents of 
the box\rb } will produce an hbox exactly five \centimetre{}s 
wide containing the typeset text ``contents of the box''. 
It's easy to get an underfull or overfull box in this way. An 
underfull box can be avoided by using {\tt \\hfil} to absorb the 
extra space. When no dimension is given, an hbox is formed that 
is just wide enough to hold the enclosed text. 
\toindex{hbox} 
 
Similarly, vboxes are formed using {\tt \\vbox\lb $\ldots$\rb}. 
What makes these boxes interesting is that when a vbox contains 
hboxes, these hboxes are stacked one above the other and set as a 
unit. Similarly, an hbox can contain vboxes, which will be set in 
a row. Suppose we take three hboxes and put them in a vbox: 
\toindex{vbox} 
 
\beginuser \obeyspaces 
\\vbox\lb 
\      \\hbox\lb{}Contents of box 1\rb 
\      \\hbox\lb{}Contents of box 2\rb 
\      \\hbox\lb{}Contents of box 3\rb 
\      \rb 
\enduser 
 
\noindent gives 
\vskip \baselineskip 
\vbox{ 
      \hbox{Contents of box 1} 
      \hbox{Contents of box 2} 
      \hbox{Contents of box 3} 
      } 
 
Now suppose we take another vbox: 
 
\beginuser \obeyspaces 
\\vbox\lb 
\      \\hbox\lb{}Contents of box 4\rb 
\      \\hbox\lb{}Contents of box 5\rb 
\      \rb 
\enduser 
 
These two vboxes can be put into an hbox; this will cause them to 
be placed side by side. In other words 
 
\beginuser \obeyspaces 
\\hbox\lb 
\     \\vbox\lb 
\           \\hbox\lb{}Contents of box 1\rb 
\           \\hbox\lb{}Contents of box 2\rb 
\           \\hbox\lb{}Contents of box 3\rb 
\           \rb 
\     \\vbox\lb 
\           \\hbox\lb{}Contents of box 4\rb 
\           \\hbox\lb{}Contents of box 5\rb 
\           \rb 
\     \rb 
\enduser 
 
\noindent gives 
\vskip \baselineskip 
\hbox{ 
     \vbox{ 
           \hbox{Contents of box 1} 
           \hbox{Contents of box 2} 
           \hbox{Contents of box 3} 
           } 
     \vbox{ 
           \hbox{Contents of box 4} 
           \hbox{Contents of box 5} 
          } 
     } 
 
 
Notice that the two vboxes are aligned so that the bottoms are 
level; also there is a little space at the beginning of each line 
and also between the vboxes.  Actually, the reason these spaces 
appear is rather subtle. Unless a line ends in a control word, 
there is always a space between the last entry in one line and 
the first one in the next line. For this reason the space between 
the vboxes comes from the end of the line containing the closing 
brace of the first vbox.  Similarly, the space at the beginning 
of the line is caused by the space after the opening brace of the 
hbox. These spaces can be avoided by ``commenting out'' the end of 
the line, that is, by putting a {\tt \%} immediately after the 
closing brace of the first vbox or the opening brace of the hbox. 
If you try to put some vboxes together and accidentally get extra 
space by forgetting to comment out the end of the line, you're in 
good company.  Some very able and experienced \TeX{} users have 
done the same thing! 
 
Extra space, say one \centimetre,  can be added by putting an 
{\tt \\hskip 1 cm } between the vboxes.  They can be aligned so 
that the tops are level by using {\tt \\vtop } instead of 
{\tt \\vbox}.  Making these two changes results in: 
 
\toindex{vtop} 
\vskip \baselineskip 
\hbox{ 
     \vtop{ 
           \hbox{Contents of box 1} 
           \hbox{Contents of box 2} 
           \hbox{Contents of box 3} 
           } 
     \hskip 1 cm 
     \vtop{ 
           \hbox{Contents of box 4} 
           \hbox{Contents of box 5} 
          } 
     } 
 
 
We can combine vboxes, hboxes, vrules, and hrules to get boxed 
text.  How might we construct such a box? One way is to take the 
material to be boxed and put it in an hbox preceded and followed 
by a vrule.  Then put this in a vbox with hrules above and below 
it. This gives us: 
 
\beginuser \obeyspaces 
\\vbox\lb 
\      \\hrule 
\      \\hbox\lb\\vrule\lb\rb The text to be boxed \\vrule\rb 
\      \\hrule 
\     \rb 
\enduser 
 
\noindent which results in 
\vskip \baselineskip 
\vbox{ 
      \hrule 
      \hbox{\vrule{} The text to be boxed \vrule} 
      \hrule 
     } 
 
This produces boxed material, but there is no margin around it 
and so it looks very cramped (of course \TeX{} is just giving us 
what we asked for).  We can improve the spacing by putting a {\tt 
\\strut} at the beginning of the hbox to make it a little taller 
and deeper.  This gives us: 
\vskip \baselineskip 
\vbox{ 
      \hrule 
      \hbox{\strut \vrule{} The text to be boxed \vrule} 
      \hrule 
     } 
 
\def\boxtext#1{% 
\vbox{% 
      \hrule 
      \hbox{\strut \vrule{} #1 \vrule}% 
      \hrule 
     }% 
} 
 
\exercise Why is it that we were forced to add extra space above 
and below the text but not before and after it? 
 
 
\exercise Use the method of boxing material to put text \centred{} 
in a box which extends from the left to the right margin. 
 
\exercise By stacking nine little boxes, make the following magic square: 
\vskip\baselineskip 
 
\moveright 2 in \vbox{\offinterlineskip 
\hbox{\boxtext 6\boxtext 1\boxtext 8} 
\hbox{\boxtext 7\boxtext 5\boxtext 3} 
\hbox{\boxtext 2\boxtext 9\boxtext 4} 
} 
 
\exercise Notice that the magic square in the previous exercise 
has internal lines that are twice as thick as the outside ones. 
Also, there is a tiny space at the intersection of the internal 
lines. Fix up the magic square so this doesn't happen. 
 
\def\boxtext#1 {% 
\vbox{% 
      \hrule 
      \hbox{\strut \vrule #1\vrule}% 
      \hrule 
     } 
} 
 
\exercise Write a macro { \tt \\boxtext\#1\lb$\ldots$\rb } which 
will take the text between the braces and put a box around it.  Test 
your macro by making up a sentence with every other word boxed. 
I'm \boxtext not quite \boxtext sure why \boxtext someone 
would \boxtext do this \boxtext since the \boxtext result is \boxtext 
pretty strange. Note how the baseline and the bottom of 
the surrounding boxes align. 
 
\def\boxtext#1 {% 
\lower 3.5pt \hbox{% 
    \vbox{% 
         \hrule 
         \hbox{\strut \vrule #1\vrule}% 
         \hrule 
        } 
    }% 
} 
 
It's easy to move boxes up, down, left, or right on the page. A 
{\tt \\vbox} can be moved to the right one inch by using {\tt 
\\moveright 1 in \\vbox\lb\dots\rb}.  To move it to the left, 
use {\tt \\moveleft}.  Similarly, an {\tt \\hbox} can be moved up 
or down using {\tt \\raise} or {\tt \\lower}. 
\toindex{moveright} 
\toindex{moveleft} 
\toindex{raise} 
\toindex{lower} 
 
\exercise Rewrite the {\tt \\boxtext} macro from the previous 
exercise so that all of the text is aligned (hint: by default the 
depth of a strut is 3.5 points).  This would give a sentence like 
the following: I'm \boxtext not quite \boxtext sure why \boxtext 
someone would \boxtext do this \boxtext since the \boxtext result 
is \boxtext pretty strange. 
 
It's possible to fill a box with either an hrule or with dots. 
The idea is to use {\tt \\hrulefill } or {\tt \\dotfill } in the 
hbox. 
 
\beginuser 
\\hbox to 5 in\lb Getting Started\\hrulefill 1\rb 
\\hbox to 5 in\lb All Characters Great and Small\\hrulefill 9\rb 
\\hbox to 5 in\lb The Shape of Things to come\\hrulefill 17\rb 
\\hbox to 5 in\lb No Math Anxiety Here!\\hrulefill 30\rb 
\enduser 
 
\noindent gives 
\vskip \baselineskip 
 
\hbox to 5 in{Getting Started\hrulefill 1} 
\hbox to 5 in{All Characters Great and Small\hrulefill 9} 
\hbox to 5 in{The Shape of Things to come\hrulefill 17} 
\hbox to 5 in{No Math Anxiety Here!\hrulefill 30} 
 
If {\tt \\hrulefill} is replaced by {\tt \\dotfill} we get 
\vskip \baselineskip 
 
\hbox to 5 in{Getting Started\dotfill 1} 
\hbox to 5 in{All Characters Great and Small\dotfill 9} 
\hbox to 5 in{The Shape of Things to come\dotfill 17} 
\hbox to 5 in{No Math Anxiety Here!\dotfill 30} 
 
\exercise Make a boxed headline appear at the top of the page 
that is like the one used in this manual. 
 
 
 
\section{Control word list} 
 
Here is a list of the control words given in this manual. 
If you want more detail about these words than is given here, 
check the index of {\bf The \TeX book}. 
 
\vskip 2\baselineskip 
\centerline{Control symbols} 
\vskip\baselineskip 
 
\settabs \+ \hskip 1.5 in & \hskip 1.65in & \hskip 1.3in & \cr 
{\tt 
\+  \\\sp{} 4 & \\!\ 34  & \\" 11 & \\' 11    \cr 
\+  \\, 34    & \\.\ 11  & \\/ 16 & \\; 34    \cr 
\+  \\= 11    & \\> 34   & \\\# 10&  \\\$ 6  \cr 
\+  \\\% 6    & \\\& 10   & \\\char '173{} 10  & \\\char '175{} 10 \cr 
\+  \\\underbar{ } 10 & \\` 11 & \\{\accent "7E } 10 & \\{\accent 94 } 10 \cr 
\+  \\| 40 \cr 
} 
 
\vskip 2\baselineskip 
\centerline{Control words} 
\vskip\baselineskip 
 
{\tt 
\+ \\AA 12 & \\aa 12 & \\acute 36 & \\AE 12 \cr 
\+ \\ae 12 & \\aleph 37 & \\alpha 35 & \\angle 37 \cr 
\+ \\approx 37 & \\arccos 41 & \\arcsin 41 & \\arctan 41 \cr 
\+ \\arg 41 & \\ast 36 & \\b 12 & \\backslash 37 \cr 
\+ \\bar 36 & \\baselineskip 22 & \\beta 35 & \\bf 16 \cr 
\+ \\biggl 40 & \\Biggl 40 & \\biggr 40 & \\Biggr 40 \cr 
\+ \\bigl 40 & \\Bigl 40 & \\bigr 40 & \\Bigr 40 \cr 
\+ \\bigskip 26 & \\break 26 & \\breve 36 & \\bullet 36 \cr 
\+ \\bye 4 & \\c 12 & \\cap 36 & \\cdot 36 \cr 
\+ \\centerline 26 & \\centreline 60 & \\check 36 & \\chi 35 \cr 
\+ \\circ 35 & \\columns 48 & \\cos 41 & \\cosh 41 \cr 
\+ \\cot 41 & \\coth 41 & \\csc 41& \\cup 36 \cr 
\+ \\d 12 & \\ddag 27 & \\ddot 36 & \\def 55 \cr 
\+ \\deg 41 & \\delta 35 & \\Delta 35 & \\det 41 \cr 
\+ \\diamond 36 & \\dim 41 & \\div 36 & \\dot 36 \cr 
\+ \\dotfill 49 & \\dots 14 & \\downarrow 41 & \\Downarrow 41 \cr 
\+ \\eject 20 & \\ell 37 & \\endinsert 26 & \\epsilon 35 \cr 
\+ \\eqalign 45 & \\eqalignno 46 & \\eqno 46 & \\equiv 37 \cr 
\+ \\eta 35 & \\exists 37 & \\exp 41 & \\flat 37 \cr 
\+ \\folio 28 & \\font 16 & \\footline 28 & \\footnote 27 \cr 
\+ \\forall 37 & \\gamma 35 & \\Gamma 35 & \\gcd 41 \cr 
\+ \\geq 37 & \\grave 36 & \\H 12 & \\halign 51 \cr 
\+ \\hang 23 & \\hangafter 23 & \\hangindent 23 & \\hat 36 \cr 
\+ \\hbadness 29 & \\hbox 72 & \\headline 28 & \\hfil 27 \cr 
\+ \\hfill 26 & \\hfuzz 29 & \\hoffset 20 & \\hom 41 \cr 
\+ \\hrule 71 & \\hrulefill 49 & \\hsize 20 & \\hskip 27 \cr 
\+ \\hyphenation 30 & \\i 11 & \\Im 37 & \\in 37 \cr 
\+ \\inf 41 & \\infty 37 & \\input 68 & \\int 38 \cr 
\+ \\iota 35 & \\it 16 & \\item 24 & \\itemitem 24 \cr 
\+ \\j 11 & \\kappa 35 & \\ker 41 & \\L 12 \cr 
\+ \\l 12 &  \\lambda 35 & \\Lambda 35 & \\langle 41 \cr 
\+ \\lceil 41 &  \\left 44 & \\leftline 26 & \\leftskip 23 \cr 
\+ \\leq 37 &  \\leqalignno 46 & \\leqno 46 & \\let 61 \cr 
\+ \\lfloor 41 &  \\lg 41 & \\lim 38 & \\liminf 41 \cr 
\+ \\limsup 41 &  \\line 26 & \\ln 41 & \\log 41 \cr 
\+ \\lower 75 &  \\magnification 21 & \\magstep 16 & \\matrix 44 \cr 
\+ \\max 41 &  \\medskip 26 & \\min 41 & \\moveleft 75 \cr 
\+ \\moveright 75 &  \\mu 35 & \\nabla 37 & \\narrower 23 \cr 
\+ \\natural 37 &  \\neg 37 & \\ni 37 & \\noalign 52 \cr 
\+ \\noindent 22 &  \\nopagenumbers 5 & \\not 36 & \\nu 35 \cr 
\+ \\O 12 &  \\o 12 & \\odot 36 & \\OE 12 \cr 
\+ \\oe 12 &  \\offinterlineskip 53 & \\omega 35 & \\Omega 35 \cr 
\+ \\ominus 36 &  \\oplus 36 & \\otimes 36 & \\over 37 \cr 
\+ \\overfullrule 29 &  \\overline 39 & \\P 27 & \\pageno 28 \cr 
\+ \\par 7 &  \\parallel 37 & \\parindent 23 & \\parshape 24 \cr 
\+ \\parskip 22 &  \\partial 37 & \\perp 37 & \\phi 35 \cr 
\+ \\Phi 35 &  \\pi 35 & \\Pi 35 & \\pmatrix 43 \cr 
\+ \\Pr 41 &  \\proclaim 42 & \\psi 35 & \\Psi 35 \cr 
\+ \\qquad 34 &  \\quad 34 & \\raggedright 27 & \\raise 75 \cr 
\+ \\rangle 41 &  \\rceil 41 & \\Re 37 & \\rfloor 41 \cr 
\+ \\rho 35 &  \\right 44 & \\rightline 26 & \\rightskip 23 \cr 
\+ \\rm 16 &  \\root 39 & \\S 27 & scaled 16 \cr 
\+ \\sec 41 &  \\settabs 48 & \\sharp 37 & \\sigma 35 \cr 
\+ \\Sigma 35 &  \\sim 37 & \\simeq 37 & \\sin 41 \cr 
\+ \\sinh 41 &  \\sl 16 & \\smallskip 26 & \\sqroot 39 \cr 
\+ \\ss 12 &  \\star 36 & \\strut 50 & \\subset 37 \cr 
\+ \\subseteq 37 &  \\sum 38 & \\sup 41 & \\supset 37 \cr 
\+ \\supseteq 37 &  \\surd 39 & \\t 12 & \\tan 41 \cr 
\+ \\tanh 41 &  \\tau 35 & \\tensor 60 & \\TeX{} 5 \cr 
\+ \\tensor 60 &  \\the 28 & \\theta 35 & \\Theta 35 \cr 
\+ \\tilde 36 &  \\times 36 & \\tolerance 29  & \\topinsert 26 \cr 
\+ \\tt 16 &  \\u 12 & \\underbar 39 & \\underline 39 \cr 
\+ \\uparrow 41 &  \\Uparrow 41 & \\updownarrow 41 & \\Updownarrow 41 \cr 
\+ \\upsilon 35 &  \\Upsilon 35 & \\v 12 & \\varepsilon 35 \cr 
\+ \\varphi 35 &  \\varrho 35 & \\varsigma 35 & \\vartheta 35 \cr 
\+ \\vbadness 30 &  \\vbox 71 & \\vec 36 & \\vee 36 \cr 
\+ \\vfill 20 &  \\vglue 25 & \\voffset 20 & \\vrule 71 \cr 
\+ \\vsize 20 & \\vtop 74 &  \\wedge 36 & \\widehat 36 \cr 
\+ \\widetilde 36 & \\xi 35 &  \\Xi 35 & \\zeta 35 \cr 
} 
 
\ifwritinganswers 
   \let\next=\relax 
\else 
   \let\next=\endinput 
   \datestamp 
\fi 
 
\next 
 
\def\beginliteral{ 
\vskip\baselineskip 
\begingroup 
\obeylines 
\tt 
\catcode`\@=0\catcode`\~=12 
\catcode`\$=12\catcode`\&=12\catcode`\^=12\catcode`\#=12 
\catcode`\_=12\catcode`\=12 
\def\par{\leavevmode\endgraf} 
\catcode`\{=12\catcode`\}=12\catcode`\%=12\catcode`\\=12 
 
} 
\def\endliteral{\nobreak \vskip 6pt \endgroup} 
 
 
\section{I get by with a little help} 
Many of the exercises can be answered in several ways. If you 
like your way better than the way given below, by all means use 
it! 
\vskip 2\baselineskip 
 
\parskip=0pt \parindent=0pt 
\raggedright 
\hbadness=10000 
 
\hrule 
\beginliteral 
I like \TeX! 
Once you get the hang of it, \TeX\  is really easy to use. 
You just have to master the \TeX nical aspects. 
@endliteral 
I like \TeX! Once you get the hang of it, \TeX\  is really easy 
to use. You just have to master the \TeX nical aspects. 
\vskip \baselineskip \hrule 
 
 
\goodbreak \vskip 2pt \hrule 
\beginliteral 
Does \AE schylus understand \OE dipus? 
@endliteral 
Does \AE schylus understand \OE dipus? 
\vskip \baselineskip \hrule 
 
\goodbreak \vskip 2pt \hrule 
\beginliteral 
The smallest internal unit of \TeX{} is about 53.63\AA. 
@endliteral 
The smallest internal unit of \TeX{} is about 53.63\AA. 
\vskip \baselineskip \hrule 
 
 
\goodbreak \vskip 2pt \hrule 
\beginliteral 
 They took some honey and plenty of money wrapped up in a {\it \$}5 note. 
@endliteral 
 They took some honey and plenty of money wrapped up in a {\it \$}5 note. 
\vskip \baselineskip \hrule 
 
 
\goodbreak \vskip 2pt \hrule 
\beginliteral 
 \'El\`eves, refusez vos le\c cons! Jetez vos cha\^\i nes! 
@endliteral 
 \'El\`eves, refusez vos le\c cons! Jetez vos cha\^\i nes! 
\vskip \baselineskip \hrule 
 
 
\goodbreak \vskip 2pt \hrule 
\beginliteral 
 Za\v sto tako polako pijete \v caj? 
@endliteral 
 Za\v sto tako polako pijete \v caj? 
\vskip \baselineskip \hrule 
 
 
\goodbreak \vskip 2pt \hrule 
\beginliteral 
 Mein Tee ist hei\ss. 
@endliteral 
 Mein Tee ist hei\ss. 
\vskip \baselineskip \hrule 
 
 
\goodbreak \vskip 2pt \hrule 
\beginliteral 
 Peut-\^etre qu'il pr\'ef\`ere le caf\'e glac\'e. 
@endliteral 
 Peut-\^etre qu'il pr\'ef\`ere le caf\'e glac\'e. 
\vskip \baselineskip \hrule 
 
\goodbreak \vskip 2pt \hrule 
\def\`{\relax\lq} 
\beginliteral 
?@`Por qu\'e no bebes vino blanco? !@`Porque est\'a avinagrado! 
@endliteral 
?`Por qu\'e no bebes vino blanco? !`Porque est\'a avinagrado! 
\vskip \baselineskip \hrule 
 
\goodbreak \vskip 2pt \hrule 
\beginliteral 
 M\'\i\'\j n idee\"en worden niet be\"\i nvloed. 
@endliteral 
 M\'\i\'\j n idee\"en worden niet be\"\i nvloed. 
\vskip \baselineskip \hrule 
 
 
 
\goodbreak \vskip 2pt \hrule 
\beginliteral 
 Can you take a ferry from \"Oland to \AA land? 
@endliteral 
 Can you take a ferry from \"Oland to \AA land? 
\vskip \baselineskip \hrule 
 
 
\goodbreak \vskip 2pt \hrule 
\beginliteral 
 T\"urk\c ce konu\c san ye\u genler nasillar? 
@endliteral 
 T\"urk\c ce konu\c san ye\u genler nasillar? 
\vskip \baselineskip \hrule 
 
 
 
 
\goodbreak \vskip 2pt \hrule 
\beginliteral 
 I entered the room and---horrors---I saw both my father-in-law and my 
mother-in-law. 
@endliteral 
 I entered the room and---horrors---I saw both my father-in-law and my 
mother-in-law. 
\vskip \baselineskip \hrule 
 
 
\goodbreak \vskip 2pt \hrule 
\beginliteral 
The winter of 1484--1485 was one of discontent. 
@endliteral 
The winter of 1484--1485 was one of discontent. 
\vskip \baselineskip \hrule 
 
 
\goodbreak \vskip 2pt \hrule 
\beginliteral 
His ``thoughtfulness'' was impressive. 
@endliteral 
His ``thoughtfulness'' was impressive. 
\vskip \baselineskip \hrule 
 
 
\goodbreak \vskip 2pt \hrule 
\beginliteral 
 Frank wondered, ``Is this a girl that can't say `No!'?'' 
@endliteral 
 Frank wondered, ``Is this a girl that can't say `No!'?'' 
\vskip \baselineskip \hrule 
 
 
\goodbreak \vskip 2pt \hrule 
\beginliteral 
 He thought, ``\dots and this goes on forever, perhaps to the last recorded 
syllable.'' 
@endliteral 
 He thought, ``\dots and this goes on forever, perhaps to the last recorded 
syllable.'' 
\vskip \baselineskip \hrule 
 
 
\goodbreak \vskip 2pt \hrule 
\beginliteral 
 Have you seen Ms.~Jones? 
@endliteral 
 Have you seen Ms.~Jones? 
\vskip \baselineskip \hrule 
 
 
\goodbreak \vskip 2pt \hrule 
\beginliteral 
Prof.~Smith and Dr.~Gold flew from 
Halifax N.~S. to Montr\'eal, P.~Q. via Moncton, N.~B. 
@endliteral 
Prof.~Smith and Dr.~Gold flew from 
Halifax N.~S. to Montr\'eal, P.~Q. via Moncton, N.~B. 
\vskip \baselineskip \hrule 
 
 
\goodbreak \vskip 2pt \hrule 
\beginliteral 
\line{left end \hfil left tackle \hfil left guard \hfil @centre \hfil 
right guard \hfil right tackle \hfil right end} 
@endliteral 
\line{left end \hfil left tackle \hfil left guard \hfil \centre{} \hfil 
right guard \hfil right tackle \hfil right end} 
\vskip \baselineskip \hrule 
 
 
\goodbreak \vskip 2pt \hrule 
\beginliteral 
\line{left \hfil \hfil right-@centre \hfil right} 
@endliteral 
\line{left \hfil \hfil right-\centre{} \hfil right} 
\vskip \baselineskip \hrule 
 
 
\goodbreak \vskip 2pt \hrule 
\beginliteral 
\line{\hskip 1 in ONE \hfil TWO \hfil THREE} 
@endliteral 
\line{\hskip 1 in ONE \hfil TWO \hfil THREE} 
\vskip \baselineskip \hrule 
 
 
\goodbreak \vskip 2pt \hrule 
\beginliteral 
i{f}f if{}f if{f} 
@endliteral 
i{f}f if{}f if{f} 
\vskip \baselineskip \hrule 
 
 
\goodbreak \vskip 2pt \hrule 
\beginliteral 
I started with roman type {\it switched to italic type}, and 
returned to roman type. 
@endliteral 
I started with roman type {\it switched to italic type}, and 
returned to roman type. 
\vskip \baselineskip \hrule 
 
 
\goodbreak \vskip 2pt \hrule 
\beginliteral 
$C(n,r) = n!/(r!\,(n-r)!)$ 
@endliteral 
$C(n,r) = n!/(r!\,(n-r)!)$ 
\vskip \baselineskip \hrule 
 
 
\goodbreak \vskip 2pt \hrule 
\beginliteral 
$a+b=c-d=xy=w/z$ 
$$a+b=c-d=xy=w/z$$ 
@endliteral 
$a+b=c-d=xy=w/z$ 
$$a+b=c-d=xy=w/z$$ 
\vskip \baselineskip \hrule 
 
 
\goodbreak \vskip 2pt \hrule 
\beginliteral 
$(fg)' = f'g + fg'$ 
$$(fg)' = f'g + fg'$$ 
@endliteral 
$(fg)' = f'g + fg'$ 
$$(fg)' = f'g + fg'$$ 
\vskip \baselineskip \hrule 
 
 
\goodbreak \vskip 2pt \hrule 
\beginliteral 
$\alpha\beta=\gamma+\delta$ 
$$\alpha\beta=\gamma+\delta$$ 
@endliteral 
$\alpha\beta=\gamma+\delta$ 
$$\alpha\beta=\gamma+\delta$$ 
\vskip \baselineskip \hrule 
 
 
\goodbreak \vskip 2pt \hrule 
\beginliteral 
$\Gamma(n) = (n-1)!$ 
$$\Gamma(n) = (n-1)!$$ 
@endliteral 
$\Gamma(n) = (n-1)!$ 
$$\Gamma(n) = (n-1)!$$ 
\vskip \baselineskip \hrule 
 
\goodbreak \vskip 2pt \hrule 
\beginliteral 
$x\wedge (y\vee z) = (x\wedge y) \vee (x\wedge z)$ 
@endliteral 
$x\wedge (y\vee z) = (x\wedge y) \vee (x\wedge z)$ 
\vskip \baselineskip \hrule 
 
\goodbreak \vskip 2pt \hrule 
\beginliteral 
$2+4+6+\cdots +2n = n(n+1)$ 
@endliteral 
$2+4+6+\cdots +2n = n(n+1)$ 
\vskip \baselineskip \hrule 
 
 
\goodbreak \vskip 2pt \hrule 
\beginliteral 
$\vec x\cdot \vec y  = 0$ if and only if $\vec x \perp \vec y$. 
@endliteral 
$\vec x\cdot \vec y  = 0$ if and only if $\vec x \perp \vec y$. 
\vskip \baselineskip \hrule 
 
 
\goodbreak \vskip 2pt \hrule 
\beginliteral 
$\vec x\cdot \vec y \not= 0$ if and only if $\vec x \not\perp \vec y$. 
@endliteral 
$\vec x\cdot \vec y \not= 0$ if and only if $\vec x \not\perp \vec y$. 
\vskip \baselineskip \hrule 
 
 
\goodbreak \vskip 2pt \hrule 
\beginliteral 
$(\forall x\in \Re)(\exists y\in\Re)$ $y>x$. 
@endliteral 
$(\forall x\in \Re)(\exists y\in\Re)$ $y>x$. 
\vskip \baselineskip \hrule 
 
 
\goodbreak \vskip 2pt \hrule 
\beginliteral 
${a+b\over c}\quad {a\over b+c}\quad {1\over a+b+c} \not= {1\over a}+ 
{1\over b}+{1\over c}$. 
@endliteral 
${a+b\over c}\quad {a\over b+c}\quad {1\over a+b+c} \not= {1\over a}+ 
{1\over b}+{1\over c}$. 
\vskip \baselineskip \hrule 
 
 
\goodbreak \vskip 2pt \hrule 
\beginliteral 
What are the points where ${\partial \over \partial x} f(x,y) = {\partial \over 
\partial y} f(x,y) = 0$? 
@endliteral 
What are the points where ${\partial \over \partial x} f(x,y) = {\partial \over 
\partial y} f(x,y) = 0$? 
\vskip \baselineskip \hrule 
 
 
\goodbreak \vskip 2pt \hrule 
\beginliteral 
$e^x \quad e^{-x} \quad e^{i\pi}+1=0 \quad x_0 \quad x_0^2 
\quad {x_0}^2 \quad 2^{x^x}$. 
@endliteral 
$e^x \quad e^{-x} \quad e^{i\pi}+1=0 \quad x_0 \quad x_0^2 
\quad {x_0}^2 \quad 2^{x^x}$. 
\vskip \baselineskip \hrule 
 
 
\goodbreak \vskip 2pt \hrule 
\beginliteral 
$\nabla^2 f(x,y) = {\partial^2 f \over\partial x^2}+ {\partial^2 f \over 
\partial y^2}$. 
@endliteral 
$\nabla^2 f(x,y) = {\partial^2 f \over\partial x^2}+ {\partial^2 f \over 
\partial y^2}$. 
\vskip \baselineskip \hrule 
 
 
\goodbreak \vskip 2pt \hrule 
\beginliteral 
$\lim_{x\to 0} (1+x)^{1\over x}=e$. 
@endliteral 
$\lim_{x\to 0} (1+x)^{1\over x}=e$. 
\vskip \baselineskip \hrule 
 
 
\goodbreak \vskip 2pt \hrule 
\beginliteral 
The cardinality of $(-\infty, \infty)$ is $\aleph_1$. 
@endliteral 
The cardinality of $(-\infty, \infty)$ is $\aleph_1$. 
\vskip \baselineskip \hrule 
 
 
\goodbreak \vskip 2pt \hrule 
\beginliteral 
$\lim_{x\to {0^+}} x^x = 1$. 
@endliteral 
$\lim_{x\to {0^+}} x^x = 1$. 
\vskip \baselineskip \hrule 
 
 
\goodbreak \vskip 2pt \hrule 
\beginliteral 
$\int_0^1 3x^2\,dx = 1$. 
@endliteral 
$\int_0^1 3x^2\,dx = 1$. 
\vskip \baselineskip \hrule 
 
 
\goodbreak \vskip 2pt \hrule 
\beginliteral 
$\sqrt2 \quad \sqrt {x+y\over x-y} \quad \root 3 \of {10}$ \quad $e^{\sqrt x}$. 
@endliteral 
$\sqrt2 \quad \sqrt {x+y\over x-y} \quad \root 3 \of {10}$ \quad $e^{\sqrt x}$. 
\vskip \baselineskip \hrule 
 
 
\goodbreak \vskip 2pt \hrule 
\beginliteral 
$\|x\| = \sqrt{x\cdot x}$. 
@endliteral 
$\|x\| = \sqrt{x\cdot x}$. 
\vskip \baselineskip \hrule 
 
 
\goodbreak \vskip 2pt \hrule 
\beginliteral 
$\phi(t) = {1 \over \sqrt{2\pi}} \int_0^t e^{-x^2/2}\,dx$. 
@endliteral 
$\phi(t) = {1 \over \sqrt{2\pi}} \int_0^t e^{-x^2/2}\,dx$. 
\vskip \baselineskip \hrule 
 
 
\goodbreak \vskip 2pt \hrule 
\beginliteral 
$\underline x \quad \overline y \quad \underline{\overline{x+y}}$. 
@endliteral 
$\underline x \quad \overline y \quad \underline{\overline{x+y}}$. 
\vskip \baselineskip \hrule 
 
 
\goodbreak \vskip 2pt \hrule 
\beginliteral 
$\bigl \lceil \lfloor x \rfloor \bigr \rceil \leq \bigl \lfloor \lceil x \rceil 
\bigr \rfloor$. 
@endliteral 
$\bigl \lceil \lfloor x \rfloor \bigr \rceil \leq \bigl \lfloor \lceil x \rceil 
\bigr \rfloor$. 
\vskip \baselineskip \hrule 
 
 
\goodbreak \vskip 2pt \hrule 
\beginliteral 
$\sin(2\theta) = 2\sin\theta\cos\theta 
\quad \cos(2\theta) = 2\cos^2\theta - 1  $. 
@endliteral 
$\sin(2\theta) = 2\sin\theta\cos\theta 
\quad \cos(2\theta) = 2\cos^2\theta - 1  $. 
\vskip \baselineskip \hrule 
 
 
\goodbreak \vskip 2pt \hrule 
\beginliteral 
$$\int \csc^2x\, dx = -\cot x+ C 
\qquad \lim_{\alpha\to 0} {\sin\alpha \over \alpha} = 1 
\qquad \lim_{\alpha\to \infty} {\sin\alpha \over \alpha} = 0.$$ 
@endliteral 
$$\int \csc^2x\, dx = -\cot x+ C 
\qquad \lim_{\alpha\to 0} {\sin\alpha \over \alpha} = 1 
\qquad \lim_{\alpha\to \infty} {\sin\alpha \over \alpha} = 0.$$ 
\vskip \baselineskip \hrule 
 
 
\goodbreak \vskip 2pt \hrule 
\beginliteral 
$$\tan(2\theta) = {2\tan\theta \over 1-\tan^2\theta}.$$ 
@endliteral 
$$\tan(2\theta) = {2\tan\theta \over 1-\tan^2\theta}.$$ 
\vskip \baselineskip \hrule 
 
 
\goodbreak \vskip 2pt \hrule 
\beginliteral 
\proclaim Theorem (Euclid). There exist an infinite number of primes. 
@endliteral 
\proclaim Theorem (Euclid). There exist an infinite number of primes. 
 
\vskip \baselineskip \hrule 
 
 
\goodbreak \vskip 2pt \hrule 
\beginliteral 
\proclaim Proposition 1. 
$\root n \of {\prod_{i=1}^n X_i} \leq 
{1 \over n} \sum_{i=1}^n X_i$ with equality if and only if $X_1=\cdots=X_n$. 
@endliteral 
\proclaim Proposition 1. 
$\root n \of {\prod_{i=1}^n X_i} \leq 
{1 \over n} \sum_{i=1}^n X_i$ with equality if and only if $X_1=\cdots=X_n$. 
 
\vskip \baselineskip \hrule 
 
 
 
\goodbreak \vskip 2pt \hrule 
\beginliteral 
$$ I_4 = \pmatrix{ 
1 &0 &0 &0 \cr 
0 &1 &0 &0 \cr 
0 &0 &1 &0 \cr 
0 &0 &0 &1 \cr}$$ 
@endliteral 
$$ I_4 = \pmatrix{ 
1 &0 &0 &0 \cr 
0 &1 &0 &0 \cr 
0 &0 &1 &0 \cr 
0 &0 &0 &1 \cr}$$ 
\vskip \baselineskip \hrule 
 
 
\goodbreak \vskip 2pt \hrule 
\beginliteral 
$$ |x| = \left\{ \matrix{ 
x & x \ge 0 \cr 
-x & x \le 0 \cr} \right.$$ 
@endliteral 
$$ |x| = \left\{ \matrix{ 
x & x \ge 0 \cr 
-x & x \le 0 \cr} \right.$$ 
\vskip \baselineskip \hrule 
 
\goodbreak \vskip 2pt \hrule 
\beginliteral 
\settabs \+ \hskip 2 in & \hskip .75in & \hskip 1cm& \cr 
\+ &Plums &\hfill\$1&.22 \cr 
\+ &Coffee &\hfill1&.78 \cr 
\+ &Granola &\hfill1&.98 \cr 
\+ &Mushrooms & &.63 \cr 
\+ &{Kiwi fruit} & &.39 \cr 
\+ &{Orange juice} &\hfill1&.09 \cr 
\+ &Tuna &\hfill1&.29 \cr 
\+ &Zucchini & &.64 \cr 
\+ &Grapes &\hfill1&.69 \cr 
\+ &{Smoked beef} & &.75 \cr 
\+ &Broccoli &\hfill\underbar{\ \ 1}&\underbar{.09} \cr 
\+ &Total &\hfill \$12&.55 \cr 
@endliteral 
\settabs \+ \hskip 2 in & \hskip .75in & \hskip 1cm& \cr 
\+ &Plums &\hfill\$1&.22 \cr 
\+ &Coffee &\hfill1&.78 \cr 
\+ &Granola &\hfill1&.98 \cr 
\+ &Mushrooms & &.63 \cr 
\+ &{Kiwi fruit} & &.39 \cr 
\+ &{Orange juice} &\hfill1&.09 \cr 
\+ &Tuna &\hfill1&.29 \cr 
\+ &Zucchini & &.64 \cr 
\+ &Grapes &\hfill1&.69 \cr 
\+ &{Smoked beef} & &.75 \cr 
\+ &Broccoli &\hfill\underbar{\ \ 1}&\underbar{.09} \cr 
\+ &Total &\hfill \$12&.55 \cr 
\vskip \baselineskip \hrule 
 
 
\goodbreak \vskip 2pt \hrule 
\beginliteral 
\settabs \+ \hskip 4.5 in & \cr 
\+Getting Started \dotfill &1 \cr 
\+All Characters Great and Small \dotfill &9 \cr 
@endliteral 
\settabs \+ \hskip 4.5 in & \cr 
\+Getting Started \dotfill &1 \cr 
\+All Characters Great and Small \dotfill &9 \cr 
\vskip \baselineskip \hrule 
 
\goodbreak \vskip 2pt \hrule 
\beginliteral 
\settabs \+ \hskip 1cm&\hskip 1 cm&\hskip 1 cm& \cr 
\moveright 2 in 
\vbox{ 
\hrule width 3 cm 
\+  \vrule height 1 cm & \vrule height 1 cm & \vrule height 1 cm 
  & \vrule height 1 cm \cr 
\hrule width 3 cm 
\+  \vrule height 1 cm & \vrule height 1 cm & \vrule height 1 cm 
  & \vrule height 1 cm \cr 
\hrule width 3 cm 
\+  \vrule height 1 cm & \vrule height 1 cm & \vrule height 1 cm 
  & \vrule height 1 cm \cr 
\hrule width 3 cm 
} 
@endliteral 
\settabs \+ \hskip 1cm&\hskip 1 cm&\hskip 1 cm& \cr 
\moveright 2 in 
\vbox{ 
\hrule width 3 cm 
\+  \vrule height 1 cm & \vrule height 1 cm & \vrule height 1 cm 
  & \vrule height 1 cm \cr 
\hrule width 3 cm 
\+  \vrule height 1 cm & \vrule height 1 cm & \vrule height 1 cm 
  & \vrule height 1 cm \cr 
\hrule width 3 cm 
\+  \vrule height 1 cm & \vrule height 1 cm & \vrule height 1 cm 
  & \vrule height 1 cm \cr 
\hrule width 3 cm 
} 
\vskip \baselineskip \hrule 
 
 
\goodbreak \vskip 2pt \hrule 
\beginliteral 
\def\boxtext#1{% 
\vbox{% 
      \hrule 
      \hbox{\strut \vrule{} #1 \vrule}% 
      \hrule 
     }% 
} 
\moveright 2 in \vbox{\offinterlineskip 
\hbox{\boxtext{6}\boxtext{1}\boxtext {8}} 
\hbox{\boxtext{7}\boxtext{5}\boxtext{3}} 
\hbox{\boxtext{2}\boxtext{9}\boxtext{4}} 
} 
@endliteral 
\def\boxtext#1{% 
\vbox{% 
      \hrule 
      \hbox{\strut \vrule{} #1 \vrule}% 
      \hrule 
     }% 
} 
\moveright 2 in \vbox{\offinterlineskip 
\hbox{\boxtext 6\boxtext 1\boxtext 8} 
\hbox{\boxtext 7\boxtext 5\boxtext 3} 
\hbox{\boxtext 2\boxtext 9\boxtext 4} 
} 
\vskip \baselineskip \hrule 
 
 
\vfill 
\datestamp 
\eject

\parindent=20pt
This introduction is available for {\tt ftp} transfer on CTAN (Combined
\TeX{} Archive Network). Use an anonymous login to grab it from the
server nearest to you.


$$\vbox{
\halign{#\quad &#\cr
\bf Server & \bf file name \cr
ftp.tex.ac.uk (134.151.79.32)  &   
/tex-archive/documentation/gentle-introduction.tex\cr
ftp.uni-stuttgart.de (129.69.8.13) &
/tex-archive/documentation/gentle-introduction.tex\cr
ftp.shsu.edu (192.92.115.10) &
/tex-archive/documentation/gentle-introduction.tex\cr
}}
$$


\newdimen\squaredimen
\def\square#1{\setbox0=\hbox{#1}%
   \dp0=.4ex \squaredimen=\ht0 
   \ifdim \wd0 > \ht0 \squaredimen=\wd0\fi
   \advance \squaredimen by \dp0
   \ht0=\squaredimen \advance \squaredimen by \dp0
   \leavevmode
   \lower .4ex \vbox{%
      \hsize=\squaredimen 
      \hrule
      \hbox to \squaredimen{\vrule\hfil\box0\hfil\vrule}%
      \hrule height .4pt depth 0pt}}

Also, a somewhat expanded version of this introduction is available as a book:\
{\bf \TeX: Starting from \square1}. Springer-Verlag 1993
(ISBN 3-540-56441-1 or 0-387-56441-1).




\bye 
 
 
