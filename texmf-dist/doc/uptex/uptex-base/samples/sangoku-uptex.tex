伝 羅貫中「三國演義」

第一回~~宴桃園豪傑三結義 斬黃巾英雄首立功

話説天下大勢、分久必合、合久必分。周末七國分爭、幷入於秦。及秦
滅之後、楚・漢分爭、又幷入於漢。漢朝自高祖斬白蛇而起義、一統天下。
後來光武中興、傳至獻帝、遂分爲三國。推其致亂之由、殆始於桓・靈二
帝。桓帝禁錮善類、崇信宦官。及桓帝崩、靈帝卽位、大將軍竇武・太傅
陳蕃、共相輔佐。時有宦官曹節等弄權、竇武・陳蕃謀誅之、機事不密、
反爲所害、中涓自此愈橫。

建寧二年四月望日、帝御溫德殿。方陞座、殿角狂風驟起、只見一條大
青蛇、從梁上飛將下來、蟠於椅上。帝驚倒、左右急救入宮、百官俱奔避。
須臾、蛇不見了。忽然大雷大雨、加以冰雹、落到半夜方止、壞却房屋無
數。建寧四年二月、洛陽地震。又海水泛溢、沿海居民、盡被大浪捲入海
中。光和元年、雌雞化雄。六月朔、黑氣十餘丈、飛入溫德殿中。秋七月、
有虹見於玉堂、五原山岸、盡皆崩裂。種種不祥、非止一端。帝下詔問羣
臣以災異之由、議郞蔡邕上疏、以爲蜺墮雞化、乃婦寺干政之所致、言頗
切直。帝覽奏歎息、因起更衣。曹節在後竊視、悉宣告左右。遂以他事陷
邕於罪、放歸田里。後張讓・趙忠・封諝・段珪・曹節・侯覽・蹇碩・程
曠・夏惲・郭勝十人朋比爲奸、號爲「十常侍」。帝尊信張讓、呼爲「阿
父」、朝政日非、以致天下人心思亂、盜賊蜂起。

時鉅鹿郡有兄弟三人。一名張角、一名張寶、一名張梁。那張角本是個
不第秀才。因入山採藥、遇一老人、碧眼童顏、手執藜杖、喚角至一洞中、
以天書三巻授之曰、『此名《太平要術》。汝得之、當代天宣化、普救世
人。若萌異心、必獲惡報。』角拜問姓名。老人曰、『吾乃南華老仙也。』
言訖、化陣清風而去。

角得此書、曉夜攻習、能呼風喚雨、號爲「太平道人」。中平元年正月
内、疫氣流行、張角散施符水、爲人治病、自稱「大賢良師」。角有徒弟
五百餘人、雲游四方、皆能書符念咒。次後徒衆日多、角乃立三十六方、
大方萬餘人、小方六七千、各立渠帥、稱爲「將軍」。訛言、『蒼天已死、
黃天當立。』又云、『歳在甲子、天下大吉。』令人各以白土、書「甲子」
二字於家中大門上。青・幽・徐・冀・荊・揚・兗・豫八州之人、家家侍
奉大賢良師張角名字。角遣其黨馬元義、暗齎金帛、結交中涓封諝、以爲
内應。角與二弟商議曰、『至難得者、民心也。今民心已順、若不乘勢取
天下、誠爲可惜。』遂一面私造黃旗、約期舉事。一面使弟子唐州、馳書
報封諝。唐州乃逕赴省中告變。帝召大將軍何進調兵擒馬元義、斬之。次
收封諝等一干人下獄。張角聞知事露、星夜舉兵、自稱「天公將軍」、張
寶稱「地公將軍」、張梁稱「人公將軍」。申言於衆曰、『今漢運將終、
大聖人出。汝等皆宜順天從正、以樂太平。』四方百姓、裹黃巾從張角反
者、四五十萬。賊勢浩大、官軍望風而靡。何進奏帝火速降詔、令各處備
禦、討賊立功。一面遣中郞將盧植・皇甫嵩・朱雋、各引精兵、分三路討
之。

且説張角一軍、前犯幽州界分。幽州太守劉焉、乃江夏竟陵人氏、漢魯
恭王之後也。當時聞得賊兵將至、召校尉鄒靖計議。靖曰、『賊兵衆、我
兵寡、明公宜作速招軍應敵。』劉焉然其説、隨卽出榜招募義兵。榜文行
到涿縣、引出涿縣中一個英雄。

\end

このテキストは坂口丈幸さんのページ(http://rtk.art.coocan.jp/)で
フリーで公開されているUTF-8のテキストをJIS X 0213+JIS X 0212の範囲に
変換したものです。

